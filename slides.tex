\documentclass{beamer}

\usepackage{physics}

\usepackage{mathtools}  
\mathtoolsset{showonlyrefs} 

\mode<presentation>{}

\usefonttheme{serif}

%% preamble
\title{Quantum particle on a circle}
%\subtitle{The subtitle}
\author{Riccardo Antonelli}
\begin{document}

\begin{frame}
    \maketitle
\end{frame}

\begin{frame}{Divergent oscillations}
\begin{equation*}
    \mathcal{A} = \sum_{\phi(t)} e^{iF[\phi(t)]}
    \label{}
\end{equation*}

Quantum observables are oscillatory sums\footnote{politically correct language for ``divergent''}, need \textbf{regularization}.

\end{frame}

\begin{frame}{Wick rotation}
    Standard answer: tilt head $\pi/2$.

    \begin{equation}
        t = i\tau
        \label{}
    \end{equation}

    \begin{equation}
        F[\phi(i\tau)] = i \mathcal{F}[\varphi(\tau)]
        \label{}
    \end{equation}

    \begin{equation}
        \mathcal{A}(it) = \sum_{\varphi(\tau)} e^{-\mathcal{F}[\varphi(\tau)]}
        \label{}
    \end{equation}

    Now they converge. Then hopefully get back to real axis through analytic continuation.
\end{frame}

\begin{frame}{QM on circles}
    Place free quantum particle on a circle.

    \begin{align}
        x \sim x + 1\,, && \psi(x+1,t) = \psi(x,t)\\
        i \pdv{t} \psi(x,t) = - {\nabla \over 2} \psi(x,t) && \partial_x \psi(x+1,t) = \partial_x \psi(x,t)
        \label{}
    \end{align}
    
    Initial condition is a position (generalized) eigenstate

    \begin{equation}
        \psi(x,0) = \delta(x)
        \label{}
    \end{equation}

    $\psi(x,t)$ is the \textbf{propagator} or \textbf{fundamental solution}.

\end{frame}

\newcommand{\sumZ}{\sum_{n=-\infty}^\infty}

\begin{frame}{Unwrapping the circle}
    On $\mathbb{R}$, the propagator is innocuous enough:

    \begin{equation}
        G_\mathbb{R}(x,t) = {1 \over \sqrt{ 2\pi i t }} \exp(ix^2 \over 2t)
        \label{}
    \end{equation}

    But $\mathbb{S}^1 = \mathbb{R} / \mathbb{Z}$, therefore

    \begin{equation}
        G(x,t) = \sumZ G_\mathbb{R}(x+n,t) 
    \end{equation}
    \begin{equation}
        = \exp(ix^2 \over 2t) {1 \over \sqrt{2\pi i t}} \sumZ \exp( {in^2 \over 2t} + {ixn \over t} )
    \end{equation}

    Simply a method of images!
\end{frame}


\begin{frame}{A pleasant surprise}
    \begin{equation}
   G(x,t) = \exp(ix^2 \over 2t) {1 \over \sqrt{2\pi i t}} \sumZ \exp( {in^2 \over 2t} + {ixn \over t} ) 
   \end{equation}

   Change variables $\tau := 2\pi t$, $z := x$, and use Poisson resummation:

   \begin{equation}
       G(z,\tau) = \sumZ \exp(\pi i n^2 \tau + 2 \pi i n z)
       \label{}
   \end{equation}

   It's the Jacobi $\vartheta$ function!
\end{frame}

\begin{frame}{A less pleasant surprise}
    
\end{frame}



\end{document}
