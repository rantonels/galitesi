\documentclass{article}
\usepackage[margin=1.3in]{geometry}
\renewcommand{\baselinestretch}{1.3} 
\usepackage[parfill]{parskip}
\usepackage[utf8]{inputenc}

\usepackage{hyperref}
    
\usepackage{amsmath,amssymb,amsfonts,amsthm}

\usepackage{biblatex}
\bibliography{bibliography}

\usepackage{physics}
\usepackage{mathtools} % arows
\usepackage{color}

\usepackage{xfrac} % fractions like/this


\usepackage{pgf}


\newcommand{\ess}{\ensuremath{\mathbb{S}}}
\newcommand{\ar}{\ensuremath{\mathbb{R}}}

\newcommand{\T}{\ensuremath{\vartheta}}
%\renewcommand{\Im}{\operatorname{Im}}

\renewcommand{\deg}{{}^{\circ}}
\newcommand{\hil}{\ensuremath{\mathcal{H}}}
\newcommand{\cmnt}[1]{\textcolor{red}{\emph{#1}}}
\newcommand{\intR}{\int_{-\infty}^\infty}
\newcommand{\sumZ}{\sum_{n=-\infty}^{\infty}}
\newcommand{\locint}{L_{1,\operatorname{loc}}}

\usepackage{graphicx}
\graphicspath{{images/}}
\newcommand{\immagine}[4]{
    \begin{figure}[h]
    \centering{
    \def\svgwidth{\linewidth}
    {\input{images/#1}}
    \caption{#3}
    \label{fig:#4}
    }
    \end{figure}
}

\title{Quantum particle on a circle}
\author{Riccardo Antonelli}

\begin{document}

\maketitle

\begin{abstract}
    We study the probability amplitude for a quantum particle on a circle to return to its original position after a time $t$. This amplitude turns out to be a tempered distribution with an intricate structure of singularities on a dense set and transformation properties reminiscent of a modular form.
\end{abstract}

\tableofcontents

\section{Introduction}

Quantum mechanics includes mathematical tools routinely employed in the calculation of measurable quantities that on any remotely formal level appear nonsensical and ill-defined. The root of this widespread misbehaviour of general quantum theory would seem to trace back to the fact that quantum-mechanical observables are to be computed through expressions of the type:

\begin{equation}\label{illdefined}
    \mathcal{O} = \abs{ \sum_{c} e^{i\mathcal{F}[c]} }^2\,.
\end{equation}

The sum is over all possible configurations and can actually be a finite sum, a series, an integral, or even a functional integral, and for essentially all choices of the real functional $\mathcal{F}$ it has very poor convergence. Nevertheless, in physics expressions like \eqref{illdefined} are often produced and manipulated with no concerns for whether they make any sense \emph{temporarily}, and are then handled with an array of regularization procedures until they are carried to a sensible final result for a desired experimental prediction.

Perhaps the most general of such regularizations is Wick rotation, in which a transformation $t \rightarrow it$ in the complex time plane is performed. Generally, computed at purely imaginary time the functionals $\mathcal{F}$ acquire a positive imaginary part and the sum-over-paths becomes convergent. The real-time desired quantity is hopefully recovered through analytic continuation up to the real axis. This is usually understood as an occasion to cure the pathologies of quantum mechanics by \emph{defining} a quantum theory simply as the result of the Wick rotation procedure. This is a practical (and almost always successful) approach but is somewhat unsatisfactory or at least unsettling, as it delegates the description of real quantities to abstract dynamics in imaginary time and also requires strong assumption of analyticity of observables.

We would like here instead to attempt a sketch of alternative meaning that can be assigned to the naively divergent oscillatory sums of quantum mechanics. For this reason we consider the simplest possible quantum system in which such unruly summations occur, a free particle on $\mathbb{S}^1$. This is a sandbox even more basic than the more popular particle in a box\footnote{In fact, the box is a simple orbifold of the circle, as $[0,1] \cong \mathbb{S}^1/\{\theta \sim - \theta\}$. Thus the propagator in a box is easily obtained as a sum of two circle propagators through the method of images.}, and in this context we pose the simplest question:

\begin{quote} \emph{What is the probability that a particle left on a point on the circle will reach another given point after some given time?} \end{quote}

This probability (density) is the squared modulus of a complex amplitude, depending on the final position and elapsed time, called the propagator, which is simply the fundamental solution to the Schr\"odinger equation on the circle. The answer to this apparently absolutely innocuous question is already an incredibly bizzare object. Even after being rigorously defined through regularization, it still does not behave as a function of position and time in any naive sense. Normally, this is kept ``as is'' and read as a formal series, up until the point when one computes more regular observables starting from it. In this work, instead, we insist in providing an accurate description of this amplitude as it comes as an actual mathematical object (and it will turn out to be a tempered distribution of both time and space) and a study of its features.

The propagator will be found to be endowed with an intricate structure of infinite singularities and zeroes that are mapped to eachother by modular transformations, but will nonetheless be perfectly well-defined as a tempered distribution; in fact the sum-over-histories that produces it will be shown to be sensible if understood as a distributional limit. A greater attention will be provided on the time dependence of the amplitude.

Finally, after this anatomical study of the circle propagator, we will offer a possible physical understanding for the features found.

\section{Quantum mechanics on $\ess^1$}
\subsection{Schr\"odinger equation and propagators}

Arguably, a minimal description of a quantum mechanical system is in terms of a separable complex Hilbert space $\hil$, the state space, an algebra $\mathcal{A}$ of operators acting on $\hil$, and, if there is interest in studying time-evolution, a selected hermitian operator $H \in \mathcal{A}$, the Hamiltonian.  

The state, representing information available about the system, is encoded as a vector\footnote{This is actually only true for \emph{pure} (i.e., maximal) information. Imperfect/incomplete information, that is a mixed state, admits a representation as a state matrix but not as a state space vector. In addition, proportional vectors of $\hil$ map to the same pure state, so that the latter should better be made to correspond with rays of $\hil$. Both of these points are not relevant to our present discussion.} (ket) $\ket{\Psi(t)}$ in state space, where we have explicited the possibility that this knowledge is time-dependent. This possibility is realized, and information on the system at time $t=t_1$ determines the state at a later time $t=t_2$ by integration of the Schr\"odinger equation:

\begin{equation}
    i \hbar \dv{t}\ket{\Psi(t)} = H \ket{\Psi(t)}
\end{equation}


If, in addition, we exclude that the Hamiltonian depends explicitly on time, the Schr\"odinger equation can immediately be integrated through an operator exponential:

\begin{equation}
    \ket{\Psi(t_2)} = e^{-\frac{i}{\hbar} H (t_2-t_1)} \ket{\Psi(t_1)} 
\end{equation}


\cmnt{Introdurre path integral, particella su $M$ e carte con Schr\"odinger}

\subsection{Review: free particle on $\mathbb{R}$}

Consider the quantization of the simplest possible continuous classical system: the free particle in one dimension. This is the Hamiltonian system phase space $\ar^2$ (with the simple chart $(p,x)$) and the Hamiltonian $H(p,x) = \frac{p^2}{2}$ (having chosen units such that $m=1$). Configuration space is the real line. Consequently, the quantum description has as state a complex wavefunction with domain $\ar$, and evolution is according to the Hamiltonian

\begin{equation}
    H = - \frac{\partial_x^2}{2}\,.
\end{equation}

We are interested in calculation of the propagator $G(\Delta t, \Delta x)$. It is instructive to review the derivation of this Green function through both methods described before.

We begin with the path integral picture.

\cmnt{ripresentare il calcolo del propagatore con path-integral.}

the final expression for the propagator, which we call the Schr\"odinger kernel, is 

\begin{equation}
    G(t,x) = (2\pi i t)^{-1/2} \exp(\frac{i}{2} \frac{x^2}{t})\,.
\end{equation}

The same result can be obtained by diagonalization of $H$. As a simple derivative operator, it's obvious a basis of eigenfunctions of $H$ is given by the Fourier components $\phi_k(x) = e^{ikx}$. Each of them evolves according to

\begin{equation}
    \label{rfourierevolution}
    \phi_k(x; t) = \exp(\frac{it}{2}\partial_x^2) \phi_k(x;0) = e^{-ik^{2}t/2} e^{ikx}\,.
\end{equation}

Our initial wavefunction is the (non-normalizable) delta function: $\Psi(x;0) = \delta(x)$. Recalling the Fourier representation of the distribution $\delta$:

\begin{equation}
    \Psi(x;0) = \frac{1}{2\pi} \int_{-\infty}^\infty dk \; e^{ikx} 
\end{equation}

and inserting \eqref{rfourierevolution} we obtain the evolved state at a later time:

\begin{equation}
    \Psi(x;t) = \frac{1}{2\pi} \int_{-\infty}^{\infty} dk \; \exp( - i \frac{k^2}{2}t +  ikx ) = (2 \pi i t)^{-1/2} \exp(-\frac{x^2}{2it}) \propto G(t,x)\,,
\end{equation}

in agreement with the first derivation.

We note a paradoxical properties of the propagator. At any given $t$, $\abs{G(t,x)}^2 \equiv \frac{1}{2\pi t}$ identically for all $x$. The state at time $t>0$ would seem to assign uniform probability density to all positions on the real line, and is therefore non-normalizable. This behaviour becomes more intuitive in light of the Heisenberg uncertainty principle (HUP), which states that for any state the variance of the position and momentum observables satisfy the inequality

\begin{equation}
    \sigma_x \sigma_p \geq \frac{1}{2}
\end{equation}

The initial $\delta$-function state is a position eigenstate, and thus has $\sigma_x = 0$. It follows that $\sigma_p = \infty$, and indeed the Fourier transform is the constant function.

If any possible value of the momentum is equally likely, it makes sense that after any small time the particle could have moved to any point on the real line with equal probability. This statement of dubious rigorousness is as far as it is possible to go in the attempt to make physical sense of the propagator as an actual solution. In reality, physically realizable initial states are normalizable and will remain so as they evolve in time; nevertheless it will be possible to model such evolution through convolution with the Schr\"odinger kernel:

\begin{equation}
    \Phi(x;t) = \left(\Phi(0) * G(t) \right)(x) := \intR dx' \Phi(x',0) G(x-x',t)
\end{equation}

and since the kernel is a tempered distribution \cmnt{proof}, if $\Phi(x,0)$ is taken to be Schwartz so will be $\Phi(x,t)$. Thus the state will be normalizable at all times.

\cmnt{convolution with gaussian?}

\cmnt{heat equation}

\subsection{Free particle on the circle}

Let us consider now the case of interest for this work: a particle moving on the cirlce $\ess^1$. A change of units can fix the radius of the circle to any desired value; we take the circumference to be $1$. In a local chart, for example using an angle coordinate $x \in [0,1]$, the Schr\"odinger equation takes the form

\begin{equation}
    -i\partial_t \Psi(x;t) = - \frac{\partial_x^2}{2} \Psi(x;t)
\end{equation}

\begin{equation}    
    \label{boundaryconds}
    \Psi(0;t) = \Psi(1;t), \quad \partial_x \Psi(0;t) = \partial_x \Psi(1;t)
\end{equation}

Boundary conditions \eqref{boundaryconds} should be obvious intuitively; formally they can be derived easily by considering a second overlapping chart and imposing the Schr\"odinger equation applies there too.

Again the goal is to determine the propagator, that is to say the generalized solution corresponding to a $\delta$-function initial condition.

One way of integrating the equation is by exploiting the known non-compact solution through a method of images. The circle is interpreted as $\ess^1 = \ar / \mathbb{Z}$, that is the quotient of $\ar$ through the identification of points that differ by an integer. Thus, a free particle starting on a point $x_1$ in the circle at time $0$ and ending on another point $x_2$ after a certain time interval $t$ is equivalent to a free particle on $\mathbb{R}$ starting on $x_1 \in \mathbb{R}$ and ending up on \emph{any} representative $x_2 + n$, $n\in\mathbb{Z}$, of the class of $x_2$ under the equivalence relation. The index $n$ of the representative corresponds to the winding number, that is the number of turns around the circle performed by the particle before reaching $x_2$.

\cmnt{immagine}

In accordance with \eqref{amplitudessum} the amplitude for the particle on $\ess^1$ to move from $x_1$ to $x_2$ is equal to the sum over all images for the particle on $\ar$ to move from $x_1$ to that image over the same time interval. Therefore, setting again $x_1 = 0$ by translation invariance, the circle propagator is given by the sum over "unwrapped" propagators:

\begin{equation}
    A(x;t) := A(0,x;0,t) = \sumZ G(x + n; t)\,.
\end{equation}

Naively ignoring convergence issues, we would then obtain the sum

\begin{equation}
    A(x;t) = (2\pi i t)^{-1/2} \sumZ \exp(i \frac{(x+ n)^2}{2t})\,.
\end{equation}

Defining the variables\footnote{The reasoning behind the nature of the redefinitions will become apparent in the next section.} $\tau := - 2\pi t $ and $z := - x$, this becomes

\begin{equation}
    A(x;t) = \exp(\frac{ix^2}{2t}) \frac{1}{\sqrt{2\pi i t}} \sumZ \exp( \frac{in^2}{2t} + \frac{ixn}{t} ) 
\end{equation}


\begin{equation} \label{amplitudeztau}
    = e^{-i\pi z^2 \tau}\, \frac{1}{\sqrt{-i\tau}} \sumZ \exp(-\frac{\pi i n^2}{\tau} +  \frac{2\pi i n z}{\tau})   \,;
\end{equation}


in particular, the probability amplitude (density) for the particle to return to its original position ($x=0$) is

\begin{equation}
    A(0;t) = \frac{1}{\sqrt{-i\tau}} \sumZ e^{-{i \pi n^2}/{\tau}} =: \T(\tau)\,.
\end{equation}


The issue with the above is that the sum of oscillatory terms does not converge in any simple sense. (In fact, we will prove in \cmnt{proof!!} it almost never does converge pointwise). It does however converge in the sense of distributions in $t$ to a tempered distribution with highly singular behaviour. Our goal is to provide a complete description of the structure of this distribution. Since the $x$-dependence of the amplitude is not as fascinating as that on $\tau$, we will limit most of our considerations to the $x=0$ case, with no real loss of qualitative structure. We will recover the $x$ dependence after, in the discussion of fractional revivals and the physical interpretation of the amplitude.

Thus, the main focus of our attention is directed towards the $\T(\tau)$ "function".


\section{Deconstructing $\T(\tau)$}

\cmnt{breve introduzione}

\subsection{A resummation formula}\label{resummation}

We will first simplify $\T(t)$ considerably. We recall the Poisson resummation formula: if $f(x)$ and $\hat f(k)$ are Fourier transforms\footnote{Note we use the ``physicist's'' convention with $\hat f(k) = \int dx\, e^{-ikx} f(x)$, so the argument of $\hat f$ is the wavenumber and $\xi$ is the inverse wavelength.} of eachother, then

\begin{equation}
    \sumZ f(n) = \sum_{\xi=-\infty}^{\infty} \hat f(2\pi\xi)\,,
\end{equation}

provided any of the two series converges at least distributionally. Taking $f(x) = e^{-i\frac{\pi}{\tau} x^2 }$, and thus $\hat f(k) = \sqrt{-i\tau} e^{i \frac{\tau}{4\pi} k^2}$, the equality reads (renaming dummy $\xi$ to $n$):

\begin{equation}
    \sumZ e^{-i\frac{\pi}{\tau} n^2 } = \sqrt{-i\tau} \sumZ e^{i\tau\pi n^2} \,,
\end{equation}

so that the $\T(\tau)$ amplitude is rewritten as

\begin{equation}
    \T(\tau) = \sumZ e^{i\tau\pi n^2}
\end{equation}

\subsection{$\Im{\tau}>0$ and the modular group} \label{sec:modular}

It seems natural to try and promote $t$, or equivalently $\tau$, to complex variables. If $\tau$ is in the upper half plane the series is absolutely convergent pointwise, as

\begin{equation}
    | \exp(\pi i n^2 \tau) | \leq e^{-\pi n^2 \Im\tau}\,.
\end{equation}

In fact, the series converges to a function holomorphic on the upper half-plane, the Jacobi theta function:

\begin{equation}\label{jacobidef}
\vartheta(z;\tau) := \sumZ \exp( \pi i n^2 \tau + 2\pi i n z )\,.
\end{equation}

%% Creator: Matplotlib, PGF backend
%%
%% To include the figure in your LaTeX document, write
%%   \input{<filename>.pgf}
%%
%% Make sure the required packages are loaded in your preamble
%%   \usepackage{pgf}
%%
%% Figures using additional raster images can only be included by \input if
%% they are in the same directory as the main LaTeX file. For loading figures
%% from other directories you can use the `import` package
%%   \usepackage{import}
%% and then include the figures with
%%   \import{<path to file>}{<filename>.pgf}
%%
%% Matplotlib used the following preamble
%%   \usepackage{fontspec}
%%   \setmainfont{DejaVu Serif}
%%   \setsansfont{DejaVu Sans}
%%   \setmonofont{DejaVu Sans Mono}
%%
\begingroup%
\makeatletter%
\begin{pgfpicture}%
\pgfpathrectangle{\pgfpointorigin}{\pgfqpoint{5.000000in}{5.000000in}}%
\pgfusepath{use as bounding box, clip}%
\begin{pgfscope}%
\pgfsetbuttcap%
\pgfsetmiterjoin%
\definecolor{currentfill}{rgb}{1.000000,1.000000,1.000000}%
\pgfsetfillcolor{currentfill}%
\pgfsetlinewidth{0.000000pt}%
\definecolor{currentstroke}{rgb}{1.000000,1.000000,1.000000}%
\pgfsetstrokecolor{currentstroke}%
\pgfsetdash{}{0pt}%
\pgfpathmoveto{\pgfqpoint{0.000000in}{0.000000in}}%
\pgfpathlineto{\pgfqpoint{5.000000in}{0.000000in}}%
\pgfpathlineto{\pgfqpoint{5.000000in}{5.000000in}}%
\pgfpathlineto{\pgfqpoint{0.000000in}{5.000000in}}%
\pgfpathclose%
\pgfusepath{fill}%
\end{pgfscope}%
\begin{pgfscope}%
\pgfsetbuttcap%
\pgfsetmiterjoin%
\definecolor{currentfill}{rgb}{1.000000,1.000000,1.000000}%
\pgfsetfillcolor{currentfill}%
\pgfsetlinewidth{0.000000pt}%
\definecolor{currentstroke}{rgb}{0.000000,0.000000,0.000000}%
\pgfsetstrokecolor{currentstroke}%
\pgfsetstrokeopacity{0.000000}%
\pgfsetdash{}{0pt}%
\pgfpathmoveto{\pgfqpoint{0.625000in}{0.550000in}}%
\pgfpathlineto{\pgfqpoint{4.500000in}{0.550000in}}%
\pgfpathlineto{\pgfqpoint{4.500000in}{4.400000in}}%
\pgfpathlineto{\pgfqpoint{0.625000in}{4.400000in}}%
\pgfpathclose%
\pgfusepath{fill}%
\end{pgfscope}%
\begin{pgfscope}%
\pgfsetbuttcap%
\pgfsetroundjoin%
\definecolor{currentfill}{rgb}{0.000000,0.000000,0.000000}%
\pgfsetfillcolor{currentfill}%
\pgfsetlinewidth{0.803000pt}%
\definecolor{currentstroke}{rgb}{0.000000,0.000000,0.000000}%
\pgfsetstrokecolor{currentstroke}%
\pgfsetdash{}{0pt}%
\pgfsys@defobject{currentmarker}{\pgfqpoint{0.000000in}{-0.048611in}}{\pgfqpoint{0.000000in}{0.000000in}}{%
\pgfpathmoveto{\pgfqpoint{0.000000in}{0.000000in}}%
\pgfpathlineto{\pgfqpoint{0.000000in}{-0.048611in}}%
\pgfusepath{stroke,fill}%
}%
\begin{pgfscope}%
\pgfsys@transformshift{0.625000in}{0.550000in}%
\pgfsys@useobject{currentmarker}{}%
\end{pgfscope}%
\end{pgfscope}%
\begin{pgfscope}%
\pgftext[x=0.625000in,y=0.452778in,,top]{\sffamily\fontsize{10.000000}{12.000000}\selectfont 0.00}%
\end{pgfscope}%
\begin{pgfscope}%
\pgfsetbuttcap%
\pgfsetroundjoin%
\definecolor{currentfill}{rgb}{0.000000,0.000000,0.000000}%
\pgfsetfillcolor{currentfill}%
\pgfsetlinewidth{0.803000pt}%
\definecolor{currentstroke}{rgb}{0.000000,0.000000,0.000000}%
\pgfsetstrokecolor{currentstroke}%
\pgfsetdash{}{0pt}%
\pgfsys@defobject{currentmarker}{\pgfqpoint{0.000000in}{-0.048611in}}{\pgfqpoint{0.000000in}{0.000000in}}{%
\pgfpathmoveto{\pgfqpoint{0.000000in}{0.000000in}}%
\pgfpathlineto{\pgfqpoint{0.000000in}{-0.048611in}}%
\pgfusepath{stroke,fill}%
}%
\begin{pgfscope}%
\pgfsys@transformshift{1.110589in}{0.550000in}%
\pgfsys@useobject{currentmarker}{}%
\end{pgfscope}%
\end{pgfscope}%
\begin{pgfscope}%
\pgftext[x=1.110589in,y=0.452778in,,top]{\sffamily\fontsize{10.000000}{12.000000}\selectfont 0.25}%
\end{pgfscope}%
\begin{pgfscope}%
\pgfsetbuttcap%
\pgfsetroundjoin%
\definecolor{currentfill}{rgb}{0.000000,0.000000,0.000000}%
\pgfsetfillcolor{currentfill}%
\pgfsetlinewidth{0.803000pt}%
\definecolor{currentstroke}{rgb}{0.000000,0.000000,0.000000}%
\pgfsetstrokecolor{currentstroke}%
\pgfsetdash{}{0pt}%
\pgfsys@defobject{currentmarker}{\pgfqpoint{0.000000in}{-0.048611in}}{\pgfqpoint{0.000000in}{0.000000in}}{%
\pgfpathmoveto{\pgfqpoint{0.000000in}{0.000000in}}%
\pgfpathlineto{\pgfqpoint{0.000000in}{-0.048611in}}%
\pgfusepath{stroke,fill}%
}%
\begin{pgfscope}%
\pgfsys@transformshift{1.596178in}{0.550000in}%
\pgfsys@useobject{currentmarker}{}%
\end{pgfscope}%
\end{pgfscope}%
\begin{pgfscope}%
\pgftext[x=1.596178in,y=0.452778in,,top]{\sffamily\fontsize{10.000000}{12.000000}\selectfont 0.50}%
\end{pgfscope}%
\begin{pgfscope}%
\pgfsetbuttcap%
\pgfsetroundjoin%
\definecolor{currentfill}{rgb}{0.000000,0.000000,0.000000}%
\pgfsetfillcolor{currentfill}%
\pgfsetlinewidth{0.803000pt}%
\definecolor{currentstroke}{rgb}{0.000000,0.000000,0.000000}%
\pgfsetstrokecolor{currentstroke}%
\pgfsetdash{}{0pt}%
\pgfsys@defobject{currentmarker}{\pgfqpoint{0.000000in}{-0.048611in}}{\pgfqpoint{0.000000in}{0.000000in}}{%
\pgfpathmoveto{\pgfqpoint{0.000000in}{0.000000in}}%
\pgfpathlineto{\pgfqpoint{0.000000in}{-0.048611in}}%
\pgfusepath{stroke,fill}%
}%
\begin{pgfscope}%
\pgfsys@transformshift{2.081767in}{0.550000in}%
\pgfsys@useobject{currentmarker}{}%
\end{pgfscope}%
\end{pgfscope}%
\begin{pgfscope}%
\pgftext[x=2.081767in,y=0.452778in,,top]{\sffamily\fontsize{10.000000}{12.000000}\selectfont 0.75}%
\end{pgfscope}%
\begin{pgfscope}%
\pgfsetbuttcap%
\pgfsetroundjoin%
\definecolor{currentfill}{rgb}{0.000000,0.000000,0.000000}%
\pgfsetfillcolor{currentfill}%
\pgfsetlinewidth{0.803000pt}%
\definecolor{currentstroke}{rgb}{0.000000,0.000000,0.000000}%
\pgfsetstrokecolor{currentstroke}%
\pgfsetdash{}{0pt}%
\pgfsys@defobject{currentmarker}{\pgfqpoint{0.000000in}{-0.048611in}}{\pgfqpoint{0.000000in}{0.000000in}}{%
\pgfpathmoveto{\pgfqpoint{0.000000in}{0.000000in}}%
\pgfpathlineto{\pgfqpoint{0.000000in}{-0.048611in}}%
\pgfusepath{stroke,fill}%
}%
\begin{pgfscope}%
\pgfsys@transformshift{2.567356in}{0.550000in}%
\pgfsys@useobject{currentmarker}{}%
\end{pgfscope}%
\end{pgfscope}%
\begin{pgfscope}%
\pgftext[x=2.567356in,y=0.452778in,,top]{\sffamily\fontsize{10.000000}{12.000000}\selectfont 1.00}%
\end{pgfscope}%
\begin{pgfscope}%
\pgfsetbuttcap%
\pgfsetroundjoin%
\definecolor{currentfill}{rgb}{0.000000,0.000000,0.000000}%
\pgfsetfillcolor{currentfill}%
\pgfsetlinewidth{0.803000pt}%
\definecolor{currentstroke}{rgb}{0.000000,0.000000,0.000000}%
\pgfsetstrokecolor{currentstroke}%
\pgfsetdash{}{0pt}%
\pgfsys@defobject{currentmarker}{\pgfqpoint{0.000000in}{-0.048611in}}{\pgfqpoint{0.000000in}{0.000000in}}{%
\pgfpathmoveto{\pgfqpoint{0.000000in}{0.000000in}}%
\pgfpathlineto{\pgfqpoint{0.000000in}{-0.048611in}}%
\pgfusepath{stroke,fill}%
}%
\begin{pgfscope}%
\pgfsys@transformshift{3.052945in}{0.550000in}%
\pgfsys@useobject{currentmarker}{}%
\end{pgfscope}%
\end{pgfscope}%
\begin{pgfscope}%
\pgftext[x=3.052945in,y=0.452778in,,top]{\sffamily\fontsize{10.000000}{12.000000}\selectfont 1.25}%
\end{pgfscope}%
\begin{pgfscope}%
\pgfsetbuttcap%
\pgfsetroundjoin%
\definecolor{currentfill}{rgb}{0.000000,0.000000,0.000000}%
\pgfsetfillcolor{currentfill}%
\pgfsetlinewidth{0.803000pt}%
\definecolor{currentstroke}{rgb}{0.000000,0.000000,0.000000}%
\pgfsetstrokecolor{currentstroke}%
\pgfsetdash{}{0pt}%
\pgfsys@defobject{currentmarker}{\pgfqpoint{0.000000in}{-0.048611in}}{\pgfqpoint{0.000000in}{0.000000in}}{%
\pgfpathmoveto{\pgfqpoint{0.000000in}{0.000000in}}%
\pgfpathlineto{\pgfqpoint{0.000000in}{-0.048611in}}%
\pgfusepath{stroke,fill}%
}%
\begin{pgfscope}%
\pgfsys@transformshift{3.538534in}{0.550000in}%
\pgfsys@useobject{currentmarker}{}%
\end{pgfscope}%
\end{pgfscope}%
\begin{pgfscope}%
\pgftext[x=3.538534in,y=0.452778in,,top]{\sffamily\fontsize{10.000000}{12.000000}\selectfont 1.50}%
\end{pgfscope}%
\begin{pgfscope}%
\pgfsetbuttcap%
\pgfsetroundjoin%
\definecolor{currentfill}{rgb}{0.000000,0.000000,0.000000}%
\pgfsetfillcolor{currentfill}%
\pgfsetlinewidth{0.803000pt}%
\definecolor{currentstroke}{rgb}{0.000000,0.000000,0.000000}%
\pgfsetstrokecolor{currentstroke}%
\pgfsetdash{}{0pt}%
\pgfsys@defobject{currentmarker}{\pgfqpoint{0.000000in}{-0.048611in}}{\pgfqpoint{0.000000in}{0.000000in}}{%
\pgfpathmoveto{\pgfqpoint{0.000000in}{0.000000in}}%
\pgfpathlineto{\pgfqpoint{0.000000in}{-0.048611in}}%
\pgfusepath{stroke,fill}%
}%
\begin{pgfscope}%
\pgfsys@transformshift{4.024123in}{0.550000in}%
\pgfsys@useobject{currentmarker}{}%
\end{pgfscope}%
\end{pgfscope}%
\begin{pgfscope}%
\pgftext[x=4.024123in,y=0.452778in,,top]{\sffamily\fontsize{10.000000}{12.000000}\selectfont 1.75}%
\end{pgfscope}%
\begin{pgfscope}%
\pgfsetbuttcap%
\pgfsetroundjoin%
\definecolor{currentfill}{rgb}{0.000000,0.000000,0.000000}%
\pgfsetfillcolor{currentfill}%
\pgfsetlinewidth{0.803000pt}%
\definecolor{currentstroke}{rgb}{0.000000,0.000000,0.000000}%
\pgfsetstrokecolor{currentstroke}%
\pgfsetdash{}{0pt}%
\pgfsys@defobject{currentmarker}{\pgfqpoint{-0.048611in}{0.000000in}}{\pgfqpoint{0.000000in}{0.000000in}}{%
\pgfpathmoveto{\pgfqpoint{0.000000in}{0.000000in}}%
\pgfpathlineto{\pgfqpoint{-0.048611in}{0.000000in}}%
\pgfusepath{stroke,fill}%
}%
\begin{pgfscope}%
\pgfsys@transformshift{0.625000in}{0.550000in}%
\pgfsys@useobject{currentmarker}{}%
\end{pgfscope}%
\end{pgfscope}%
\begin{pgfscope}%
\pgftext[x=0.218533in,y=0.497238in,left,base]{\sffamily\fontsize{10.000000}{12.000000}\selectfont 0.00}%
\end{pgfscope}%
\begin{pgfscope}%
\pgfsetbuttcap%
\pgfsetroundjoin%
\definecolor{currentfill}{rgb}{0.000000,0.000000,0.000000}%
\pgfsetfillcolor{currentfill}%
\pgfsetlinewidth{0.803000pt}%
\definecolor{currentstroke}{rgb}{0.000000,0.000000,0.000000}%
\pgfsetstrokecolor{currentstroke}%
\pgfsetdash{}{0pt}%
\pgfsys@defobject{currentmarker}{\pgfqpoint{-0.048611in}{0.000000in}}{\pgfqpoint{0.000000in}{0.000000in}}{%
\pgfpathmoveto{\pgfqpoint{0.000000in}{0.000000in}}%
\pgfpathlineto{\pgfqpoint{-0.048611in}{0.000000in}}%
\pgfusepath{stroke,fill}%
}%
\begin{pgfscope}%
\pgfsys@transformshift{0.625000in}{1.032456in}%
\pgfsys@useobject{currentmarker}{}%
\end{pgfscope}%
\end{pgfscope}%
\begin{pgfscope}%
\pgftext[x=0.218533in,y=0.979695in,left,base]{\sffamily\fontsize{10.000000}{12.000000}\selectfont 0.25}%
\end{pgfscope}%
\begin{pgfscope}%
\pgfsetbuttcap%
\pgfsetroundjoin%
\definecolor{currentfill}{rgb}{0.000000,0.000000,0.000000}%
\pgfsetfillcolor{currentfill}%
\pgfsetlinewidth{0.803000pt}%
\definecolor{currentstroke}{rgb}{0.000000,0.000000,0.000000}%
\pgfsetstrokecolor{currentstroke}%
\pgfsetdash{}{0pt}%
\pgfsys@defobject{currentmarker}{\pgfqpoint{-0.048611in}{0.000000in}}{\pgfqpoint{0.000000in}{0.000000in}}{%
\pgfpathmoveto{\pgfqpoint{0.000000in}{0.000000in}}%
\pgfpathlineto{\pgfqpoint{-0.048611in}{0.000000in}}%
\pgfusepath{stroke,fill}%
}%
\begin{pgfscope}%
\pgfsys@transformshift{0.625000in}{1.514912in}%
\pgfsys@useobject{currentmarker}{}%
\end{pgfscope}%
\end{pgfscope}%
\begin{pgfscope}%
\pgftext[x=0.218533in,y=1.462151in,left,base]{\sffamily\fontsize{10.000000}{12.000000}\selectfont 0.50}%
\end{pgfscope}%
\begin{pgfscope}%
\pgfsetbuttcap%
\pgfsetroundjoin%
\definecolor{currentfill}{rgb}{0.000000,0.000000,0.000000}%
\pgfsetfillcolor{currentfill}%
\pgfsetlinewidth{0.803000pt}%
\definecolor{currentstroke}{rgb}{0.000000,0.000000,0.000000}%
\pgfsetstrokecolor{currentstroke}%
\pgfsetdash{}{0pt}%
\pgfsys@defobject{currentmarker}{\pgfqpoint{-0.048611in}{0.000000in}}{\pgfqpoint{0.000000in}{0.000000in}}{%
\pgfpathmoveto{\pgfqpoint{0.000000in}{0.000000in}}%
\pgfpathlineto{\pgfqpoint{-0.048611in}{0.000000in}}%
\pgfusepath{stroke,fill}%
}%
\begin{pgfscope}%
\pgfsys@transformshift{0.625000in}{1.997368in}%
\pgfsys@useobject{currentmarker}{}%
\end{pgfscope}%
\end{pgfscope}%
\begin{pgfscope}%
\pgftext[x=0.218533in,y=1.944607in,left,base]{\sffamily\fontsize{10.000000}{12.000000}\selectfont 0.75}%
\end{pgfscope}%
\begin{pgfscope}%
\pgfsetbuttcap%
\pgfsetroundjoin%
\definecolor{currentfill}{rgb}{0.000000,0.000000,0.000000}%
\pgfsetfillcolor{currentfill}%
\pgfsetlinewidth{0.803000pt}%
\definecolor{currentstroke}{rgb}{0.000000,0.000000,0.000000}%
\pgfsetstrokecolor{currentstroke}%
\pgfsetdash{}{0pt}%
\pgfsys@defobject{currentmarker}{\pgfqpoint{-0.048611in}{0.000000in}}{\pgfqpoint{0.000000in}{0.000000in}}{%
\pgfpathmoveto{\pgfqpoint{0.000000in}{0.000000in}}%
\pgfpathlineto{\pgfqpoint{-0.048611in}{0.000000in}}%
\pgfusepath{stroke,fill}%
}%
\begin{pgfscope}%
\pgfsys@transformshift{0.625000in}{2.479825in}%
\pgfsys@useobject{currentmarker}{}%
\end{pgfscope}%
\end{pgfscope}%
\begin{pgfscope}%
\pgftext[x=0.218533in,y=2.427063in,left,base]{\sffamily\fontsize{10.000000}{12.000000}\selectfont 1.00}%
\end{pgfscope}%
\begin{pgfscope}%
\pgfsetbuttcap%
\pgfsetroundjoin%
\definecolor{currentfill}{rgb}{0.000000,0.000000,0.000000}%
\pgfsetfillcolor{currentfill}%
\pgfsetlinewidth{0.803000pt}%
\definecolor{currentstroke}{rgb}{0.000000,0.000000,0.000000}%
\pgfsetstrokecolor{currentstroke}%
\pgfsetdash{}{0pt}%
\pgfsys@defobject{currentmarker}{\pgfqpoint{-0.048611in}{0.000000in}}{\pgfqpoint{0.000000in}{0.000000in}}{%
\pgfpathmoveto{\pgfqpoint{0.000000in}{0.000000in}}%
\pgfpathlineto{\pgfqpoint{-0.048611in}{0.000000in}}%
\pgfusepath{stroke,fill}%
}%
\begin{pgfscope}%
\pgfsys@transformshift{0.625000in}{2.962281in}%
\pgfsys@useobject{currentmarker}{}%
\end{pgfscope}%
\end{pgfscope}%
\begin{pgfscope}%
\pgftext[x=0.218533in,y=2.909519in,left,base]{\sffamily\fontsize{10.000000}{12.000000}\selectfont 1.25}%
\end{pgfscope}%
\begin{pgfscope}%
\pgfsetbuttcap%
\pgfsetroundjoin%
\definecolor{currentfill}{rgb}{0.000000,0.000000,0.000000}%
\pgfsetfillcolor{currentfill}%
\pgfsetlinewidth{0.803000pt}%
\definecolor{currentstroke}{rgb}{0.000000,0.000000,0.000000}%
\pgfsetstrokecolor{currentstroke}%
\pgfsetdash{}{0pt}%
\pgfsys@defobject{currentmarker}{\pgfqpoint{-0.048611in}{0.000000in}}{\pgfqpoint{0.000000in}{0.000000in}}{%
\pgfpathmoveto{\pgfqpoint{0.000000in}{0.000000in}}%
\pgfpathlineto{\pgfqpoint{-0.048611in}{0.000000in}}%
\pgfusepath{stroke,fill}%
}%
\begin{pgfscope}%
\pgfsys@transformshift{0.625000in}{3.444737in}%
\pgfsys@useobject{currentmarker}{}%
\end{pgfscope}%
\end{pgfscope}%
\begin{pgfscope}%
\pgftext[x=0.218533in,y=3.391975in,left,base]{\sffamily\fontsize{10.000000}{12.000000}\selectfont 1.50}%
\end{pgfscope}%
\begin{pgfscope}%
\pgfsetbuttcap%
\pgfsetroundjoin%
\definecolor{currentfill}{rgb}{0.000000,0.000000,0.000000}%
\pgfsetfillcolor{currentfill}%
\pgfsetlinewidth{0.803000pt}%
\definecolor{currentstroke}{rgb}{0.000000,0.000000,0.000000}%
\pgfsetstrokecolor{currentstroke}%
\pgfsetdash{}{0pt}%
\pgfsys@defobject{currentmarker}{\pgfqpoint{-0.048611in}{0.000000in}}{\pgfqpoint{0.000000in}{0.000000in}}{%
\pgfpathmoveto{\pgfqpoint{0.000000in}{0.000000in}}%
\pgfpathlineto{\pgfqpoint{-0.048611in}{0.000000in}}%
\pgfusepath{stroke,fill}%
}%
\begin{pgfscope}%
\pgfsys@transformshift{0.625000in}{3.927193in}%
\pgfsys@useobject{currentmarker}{}%
\end{pgfscope}%
\end{pgfscope}%
\begin{pgfscope}%
\pgftext[x=0.218533in,y=3.874431in,left,base]{\sffamily\fontsize{10.000000}{12.000000}\selectfont 1.75}%
\end{pgfscope}%
\begin{pgfscope}%
\pgfpathrectangle{\pgfqpoint{0.625000in}{0.550000in}}{\pgfqpoint{3.875000in}{3.850000in}} %
\pgfusepath{clip}%
\pgfsetbuttcap%
\pgfsetroundjoin%
\pgfsetlinewidth{0.250937pt}%
\definecolor{currentstroke}{rgb}{0.000000,0.000000,0.000000}%
\pgfsetstrokecolor{currentstroke}%
\pgfsetdash{}{0pt}%
\pgfpathmoveto{\pgfqpoint{0.625000in}{0.633077in}}%
\pgfpathlineto{\pgfqpoint{0.631068in}{0.627193in}}%
\pgfpathlineto{\pgfqpoint{0.625000in}{0.621326in}}%
\pgfusepath{stroke}%
\end{pgfscope}%
\begin{pgfscope}%
\pgfpathrectangle{\pgfqpoint{0.625000in}{0.550000in}}{\pgfqpoint{3.875000in}{3.850000in}} %
\pgfusepath{clip}%
\pgfsetbuttcap%
\pgfsetroundjoin%
\pgfsetlinewidth{0.250937pt}%
\definecolor{currentstroke}{rgb}{0.000000,0.000000,0.000000}%
\pgfsetstrokecolor{currentstroke}%
\pgfsetdash{}{0pt}%
\pgfpathmoveto{\pgfqpoint{0.625000in}{0.787038in}}%
\pgfpathlineto{\pgfqpoint{0.630905in}{0.781579in}}%
\pgfpathlineto{\pgfqpoint{0.625000in}{0.775877in}}%
\pgfusepath{stroke}%
\end{pgfscope}%
\begin{pgfscope}%
\pgfpathrectangle{\pgfqpoint{0.625000in}{0.550000in}}{\pgfqpoint{3.875000in}{3.850000in}} %
\pgfusepath{clip}%
\pgfsetbuttcap%
\pgfsetroundjoin%
\pgfsetlinewidth{0.250937pt}%
\definecolor{currentstroke}{rgb}{0.000000,0.000000,0.000000}%
\pgfsetstrokecolor{currentstroke}%
\pgfsetdash{}{0pt}%
\pgfpathmoveto{\pgfqpoint{0.625000in}{0.941723in}}%
\pgfpathlineto{\pgfqpoint{0.632289in}{0.935965in}}%
\pgfpathlineto{\pgfqpoint{0.625000in}{0.930155in}}%
\pgfusepath{stroke}%
\end{pgfscope}%
\begin{pgfscope}%
\pgfpathrectangle{\pgfqpoint{0.625000in}{0.550000in}}{\pgfqpoint{3.875000in}{3.850000in}} %
\pgfusepath{clip}%
\pgfsetbuttcap%
\pgfsetroundjoin%
\pgfsetlinewidth{0.250937pt}%
\definecolor{currentstroke}{rgb}{0.000000,0.000000,0.000000}%
\pgfsetstrokecolor{currentstroke}%
\pgfsetdash{}{0pt}%
\pgfpathmoveto{\pgfqpoint{0.625000in}{1.096036in}}%
\pgfpathlineto{\pgfqpoint{0.630928in}{1.090351in}}%
\pgfpathlineto{\pgfqpoint{0.625000in}{1.084687in}}%
\pgfusepath{stroke}%
\end{pgfscope}%
\begin{pgfscope}%
\pgfpathrectangle{\pgfqpoint{0.625000in}{0.550000in}}{\pgfqpoint{3.875000in}{3.850000in}} %
\pgfusepath{clip}%
\pgfsetbuttcap%
\pgfsetroundjoin%
\pgfsetlinewidth{0.250937pt}%
\definecolor{currentstroke}{rgb}{0.000000,0.000000,0.000000}%
\pgfsetstrokecolor{currentstroke}%
\pgfsetdash{}{0pt}%
\pgfpathmoveto{\pgfqpoint{0.625000in}{1.250564in}}%
\pgfpathlineto{\pgfqpoint{0.630904in}{1.244737in}}%
\pgfpathlineto{\pgfqpoint{0.625000in}{1.239070in}}%
\pgfusepath{stroke}%
\end{pgfscope}%
\begin{pgfscope}%
\pgfpathrectangle{\pgfqpoint{0.625000in}{0.550000in}}{\pgfqpoint{3.875000in}{3.850000in}} %
\pgfusepath{clip}%
\pgfsetbuttcap%
\pgfsetroundjoin%
\pgfsetlinewidth{0.250937pt}%
\definecolor{currentstroke}{rgb}{0.000000,0.000000,0.000000}%
\pgfsetstrokecolor{currentstroke}%
\pgfsetdash{}{0pt}%
\pgfpathmoveto{\pgfqpoint{0.625000in}{1.404646in}}%
\pgfpathlineto{\pgfqpoint{0.630806in}{1.399123in}}%
\pgfpathlineto{\pgfqpoint{0.625000in}{1.393597in}}%
\pgfusepath{stroke}%
\end{pgfscope}%
\begin{pgfscope}%
\pgfpathrectangle{\pgfqpoint{0.625000in}{0.550000in}}{\pgfqpoint{3.875000in}{3.850000in}} %
\pgfusepath{clip}%
\pgfsetbuttcap%
\pgfsetroundjoin%
\pgfsetlinewidth{0.250937pt}%
\definecolor{currentstroke}{rgb}{0.000000,0.000000,0.000000}%
\pgfsetstrokecolor{currentstroke}%
\pgfsetdash{}{0pt}%
\pgfpathmoveto{\pgfqpoint{0.625000in}{1.559053in}}%
\pgfpathlineto{\pgfqpoint{0.630820in}{1.553509in}}%
\pgfpathlineto{\pgfqpoint{0.625000in}{1.547784in}}%
\pgfusepath{stroke}%
\end{pgfscope}%
\begin{pgfscope}%
\pgfpathrectangle{\pgfqpoint{0.625000in}{0.550000in}}{\pgfqpoint{3.875000in}{3.850000in}} %
\pgfusepath{clip}%
\pgfsetbuttcap%
\pgfsetroundjoin%
\pgfsetlinewidth{0.250937pt}%
\definecolor{currentstroke}{rgb}{0.000000,0.000000,0.000000}%
\pgfsetstrokecolor{currentstroke}%
\pgfsetdash{}{0pt}%
\pgfpathmoveto{\pgfqpoint{0.625000in}{1.713408in}}%
\pgfpathlineto{\pgfqpoint{0.632104in}{1.707895in}}%
\pgfpathlineto{\pgfqpoint{0.625000in}{1.702346in}}%
\pgfusepath{stroke}%
\end{pgfscope}%
\begin{pgfscope}%
\pgfpathrectangle{\pgfqpoint{0.625000in}{0.550000in}}{\pgfqpoint{3.875000in}{3.850000in}} %
\pgfusepath{clip}%
\pgfsetbuttcap%
\pgfsetroundjoin%
\pgfsetlinewidth{0.250937pt}%
\definecolor{currentstroke}{rgb}{0.000000,0.000000,0.000000}%
\pgfsetstrokecolor{currentstroke}%
\pgfsetdash{}{0pt}%
\pgfpathmoveto{\pgfqpoint{0.625000in}{1.867799in}}%
\pgfpathlineto{\pgfqpoint{0.630727in}{1.862281in}}%
\pgfpathlineto{\pgfqpoint{0.625000in}{1.856876in}}%
\pgfusepath{stroke}%
\end{pgfscope}%
\begin{pgfscope}%
\pgfpathrectangle{\pgfqpoint{0.625000in}{0.550000in}}{\pgfqpoint{3.875000in}{3.850000in}} %
\pgfusepath{clip}%
\pgfsetbuttcap%
\pgfsetroundjoin%
\pgfsetlinewidth{0.250937pt}%
\definecolor{currentstroke}{rgb}{0.000000,0.000000,0.000000}%
\pgfsetstrokecolor{currentstroke}%
\pgfsetdash{}{0pt}%
\pgfpathmoveto{\pgfqpoint{0.625000in}{2.022334in}}%
\pgfpathlineto{\pgfqpoint{0.630673in}{2.016667in}}%
\pgfpathlineto{\pgfqpoint{0.625000in}{2.011045in}}%
\pgfusepath{stroke}%
\end{pgfscope}%
\begin{pgfscope}%
\pgfpathrectangle{\pgfqpoint{0.625000in}{0.550000in}}{\pgfqpoint{3.875000in}{3.850000in}} %
\pgfusepath{clip}%
\pgfsetbuttcap%
\pgfsetroundjoin%
\pgfsetlinewidth{0.250937pt}%
\definecolor{currentstroke}{rgb}{0.000000,0.000000,0.000000}%
\pgfsetstrokecolor{currentstroke}%
\pgfsetdash{}{0pt}%
\pgfpathmoveto{\pgfqpoint{0.625000in}{2.176453in}}%
\pgfpathlineto{\pgfqpoint{0.630711in}{2.171053in}}%
\pgfpathlineto{\pgfqpoint{0.625000in}{2.165498in}}%
\pgfusepath{stroke}%
\end{pgfscope}%
\begin{pgfscope}%
\pgfpathrectangle{\pgfqpoint{0.625000in}{0.550000in}}{\pgfqpoint{3.875000in}{3.850000in}} %
\pgfusepath{clip}%
\pgfsetbuttcap%
\pgfsetroundjoin%
\pgfsetlinewidth{0.250937pt}%
\definecolor{currentstroke}{rgb}{0.000000,0.000000,0.000000}%
\pgfsetstrokecolor{currentstroke}%
\pgfsetdash{}{0pt}%
\pgfpathmoveto{\pgfqpoint{0.625000in}{2.330783in}}%
\pgfpathlineto{\pgfqpoint{0.630706in}{2.325439in}}%
\pgfpathlineto{\pgfqpoint{0.625000in}{2.319720in}}%
\pgfusepath{stroke}%
\end{pgfscope}%
\begin{pgfscope}%
\pgfpathrectangle{\pgfqpoint{0.625000in}{0.550000in}}{\pgfqpoint{3.875000in}{3.850000in}} %
\pgfusepath{clip}%
\pgfsetbuttcap%
\pgfsetroundjoin%
\pgfsetlinewidth{0.250937pt}%
\definecolor{currentstroke}{rgb}{0.000000,0.000000,0.000000}%
\pgfsetstrokecolor{currentstroke}%
\pgfsetdash{}{0pt}%
\pgfpathmoveto{\pgfqpoint{0.625000in}{2.486118in}}%
\pgfpathlineto{\pgfqpoint{0.628147in}{2.489474in}}%
\pgfpathlineto{\pgfqpoint{0.628884in}{2.499123in}}%
\pgfpathlineto{\pgfqpoint{0.632894in}{2.508772in}}%
\pgfpathlineto{\pgfqpoint{0.634712in}{2.513252in}}%
\pgfpathlineto{\pgfqpoint{0.639776in}{2.518421in}}%
\pgfpathlineto{\pgfqpoint{0.644424in}{2.524706in}}%
\pgfpathlineto{\pgfqpoint{0.649704in}{2.528070in}}%
\pgfpathlineto{\pgfqpoint{0.654135in}{2.532139in}}%
\pgfpathlineto{\pgfqpoint{0.663847in}{2.536896in}}%
\pgfpathlineto{\pgfqpoint{0.667015in}{2.537719in}}%
\pgfpathlineto{\pgfqpoint{0.673559in}{2.540164in}}%
\pgfpathlineto{\pgfqpoint{0.683271in}{2.541740in}}%
\pgfpathlineto{\pgfqpoint{0.692982in}{2.541764in}}%
\pgfpathlineto{\pgfqpoint{0.702694in}{2.540279in}}%
\pgfpathlineto{\pgfqpoint{0.710903in}{2.537719in}}%
\pgfpathlineto{\pgfqpoint{0.712406in}{2.537180in}}%
\pgfpathlineto{\pgfqpoint{0.722118in}{2.532250in}}%
\pgfpathlineto{\pgfqpoint{0.728162in}{2.528070in}}%
\pgfpathlineto{\pgfqpoint{0.731830in}{2.524755in}}%
\pgfpathlineto{\pgfqpoint{0.737956in}{2.518421in}}%
\pgfpathlineto{\pgfqpoint{0.741541in}{2.513073in}}%
\pgfpathlineto{\pgfqpoint{0.744362in}{2.508772in}}%
\pgfpathlineto{\pgfqpoint{0.748532in}{2.499123in}}%
\pgfpathlineto{\pgfqpoint{0.750778in}{2.489474in}}%
\pgfpathlineto{\pgfqpoint{0.751253in}{2.483043in}}%
\pgfpathlineto{\pgfqpoint{0.751526in}{2.479825in}}%
\pgfpathlineto{\pgfqpoint{0.751253in}{2.476606in}}%
\pgfpathlineto{\pgfqpoint{0.750778in}{2.470175in}}%
\pgfpathlineto{\pgfqpoint{0.748532in}{2.460526in}}%
\pgfpathlineto{\pgfqpoint{0.744362in}{2.450877in}}%
\pgfpathlineto{\pgfqpoint{0.741541in}{2.446576in}}%
\pgfpathlineto{\pgfqpoint{0.737956in}{2.441228in}}%
\pgfpathlineto{\pgfqpoint{0.731830in}{2.434894in}}%
\pgfpathlineto{\pgfqpoint{0.728162in}{2.431579in}}%
\pgfpathlineto{\pgfqpoint{0.722118in}{2.427399in}}%
\pgfpathlineto{\pgfqpoint{0.712406in}{2.422469in}}%
\pgfpathlineto{\pgfqpoint{0.710903in}{2.421930in}}%
\pgfpathlineto{\pgfqpoint{0.702694in}{2.419370in}}%
\pgfpathlineto{\pgfqpoint{0.692982in}{2.417886in}}%
\pgfpathlineto{\pgfqpoint{0.683271in}{2.417910in}}%
\pgfpathlineto{\pgfqpoint{0.673559in}{2.419485in}}%
\pgfpathlineto{\pgfqpoint{0.667015in}{2.421930in}}%
\pgfpathlineto{\pgfqpoint{0.663847in}{2.422753in}}%
\pgfpathlineto{\pgfqpoint{0.654135in}{2.427510in}}%
\pgfpathlineto{\pgfqpoint{0.649704in}{2.431579in}}%
\pgfpathlineto{\pgfqpoint{0.644424in}{2.434943in}}%
\pgfpathlineto{\pgfqpoint{0.639776in}{2.441228in}}%
\pgfpathlineto{\pgfqpoint{0.634712in}{2.446397in}}%
\pgfpathlineto{\pgfqpoint{0.632894in}{2.450877in}}%
\pgfpathlineto{\pgfqpoint{0.628894in}{2.460526in}}%
\pgfpathlineto{\pgfqpoint{0.628147in}{2.470175in}}%
\pgfpathlineto{\pgfqpoint{0.625000in}{2.473531in}}%
\pgfusepath{stroke}%
\end{pgfscope}%
\begin{pgfscope}%
\pgfpathrectangle{\pgfqpoint{0.625000in}{0.550000in}}{\pgfqpoint{3.875000in}{3.850000in}} %
\pgfusepath{clip}%
\pgfsetbuttcap%
\pgfsetroundjoin%
\pgfsetlinewidth{0.250937pt}%
\definecolor{currentstroke}{rgb}{0.000000,0.000000,0.000000}%
\pgfsetstrokecolor{currentstroke}%
\pgfsetdash{}{0pt}%
\pgfpathmoveto{\pgfqpoint{0.625000in}{2.639922in}}%
\pgfpathlineto{\pgfqpoint{0.630698in}{2.634211in}}%
\pgfpathlineto{\pgfqpoint{0.625000in}{2.628874in}}%
\pgfusepath{stroke}%
\end{pgfscope}%
\begin{pgfscope}%
\pgfpathrectangle{\pgfqpoint{0.625000in}{0.550000in}}{\pgfqpoint{3.875000in}{3.850000in}} %
\pgfusepath{clip}%
\pgfsetbuttcap%
\pgfsetroundjoin%
\pgfsetlinewidth{0.250937pt}%
\definecolor{currentstroke}{rgb}{0.000000,0.000000,0.000000}%
\pgfsetstrokecolor{currentstroke}%
\pgfsetdash{}{0pt}%
\pgfpathmoveto{\pgfqpoint{0.625000in}{2.794126in}}%
\pgfpathlineto{\pgfqpoint{0.630686in}{2.788596in}}%
\pgfpathlineto{\pgfqpoint{0.625000in}{2.783222in}}%
\pgfusepath{stroke}%
\end{pgfscope}%
\begin{pgfscope}%
\pgfpathrectangle{\pgfqpoint{0.625000in}{0.550000in}}{\pgfqpoint{3.875000in}{3.850000in}} %
\pgfusepath{clip}%
\pgfsetbuttcap%
\pgfsetroundjoin%
\pgfsetlinewidth{0.250937pt}%
\definecolor{currentstroke}{rgb}{0.000000,0.000000,0.000000}%
\pgfsetstrokecolor{currentstroke}%
\pgfsetdash{}{0pt}%
\pgfpathmoveto{\pgfqpoint{0.625000in}{2.948534in}}%
\pgfpathlineto{\pgfqpoint{0.630603in}{2.942982in}}%
\pgfpathlineto{\pgfqpoint{0.625000in}{2.937384in}}%
\pgfusepath{stroke}%
\end{pgfscope}%
\begin{pgfscope}%
\pgfpathrectangle{\pgfqpoint{0.625000in}{0.550000in}}{\pgfqpoint{3.875000in}{3.850000in}} %
\pgfusepath{clip}%
\pgfsetbuttcap%
\pgfsetroundjoin%
\pgfsetlinewidth{0.250937pt}%
\definecolor{currentstroke}{rgb}{0.000000,0.000000,0.000000}%
\pgfsetstrokecolor{currentstroke}%
\pgfsetdash{}{0pt}%
\pgfpathmoveto{\pgfqpoint{0.625000in}{3.102717in}}%
\pgfpathlineto{\pgfqpoint{0.630673in}{3.097368in}}%
\pgfpathlineto{\pgfqpoint{0.625000in}{3.091905in}}%
\pgfusepath{stroke}%
\end{pgfscope}%
\begin{pgfscope}%
\pgfpathrectangle{\pgfqpoint{0.625000in}{0.550000in}}{\pgfqpoint{3.875000in}{3.850000in}} %
\pgfusepath{clip}%
\pgfsetbuttcap%
\pgfsetroundjoin%
\pgfsetlinewidth{0.250937pt}%
\definecolor{currentstroke}{rgb}{0.000000,0.000000,0.000000}%
\pgfsetstrokecolor{currentstroke}%
\pgfsetdash{}{0pt}%
\pgfpathmoveto{\pgfqpoint{0.625000in}{3.257105in}}%
\pgfpathlineto{\pgfqpoint{0.631942in}{3.251754in}}%
\pgfpathlineto{\pgfqpoint{0.625000in}{3.246439in}}%
\pgfusepath{stroke}%
\end{pgfscope}%
\begin{pgfscope}%
\pgfpathrectangle{\pgfqpoint{0.625000in}{0.550000in}}{\pgfqpoint{3.875000in}{3.850000in}} %
\pgfusepath{clip}%
\pgfsetbuttcap%
\pgfsetroundjoin%
\pgfsetlinewidth{0.250937pt}%
\definecolor{currentstroke}{rgb}{0.000000,0.000000,0.000000}%
\pgfsetstrokecolor{currentstroke}%
\pgfsetdash{}{0pt}%
\pgfpathmoveto{\pgfqpoint{0.625000in}{3.411730in}}%
\pgfpathlineto{\pgfqpoint{0.630684in}{3.406140in}}%
\pgfpathlineto{\pgfqpoint{0.625000in}{3.400732in}}%
\pgfusepath{stroke}%
\end{pgfscope}%
\begin{pgfscope}%
\pgfpathrectangle{\pgfqpoint{0.625000in}{0.550000in}}{\pgfqpoint{3.875000in}{3.850000in}} %
\pgfusepath{clip}%
\pgfsetbuttcap%
\pgfsetroundjoin%
\pgfsetlinewidth{0.250937pt}%
\definecolor{currentstroke}{rgb}{0.000000,0.000000,0.000000}%
\pgfsetstrokecolor{currentstroke}%
\pgfsetdash{}{0pt}%
\pgfpathmoveto{\pgfqpoint{0.625000in}{3.565889in}}%
\pgfpathlineto{\pgfqpoint{0.630644in}{3.560526in}}%
\pgfpathlineto{\pgfqpoint{0.625000in}{3.555167in}}%
\pgfusepath{stroke}%
\end{pgfscope}%
\begin{pgfscope}%
\pgfpathrectangle{\pgfqpoint{0.625000in}{0.550000in}}{\pgfqpoint{3.875000in}{3.850000in}} %
\pgfusepath{clip}%
\pgfsetbuttcap%
\pgfsetroundjoin%
\pgfsetlinewidth{0.250937pt}%
\definecolor{currentstroke}{rgb}{0.000000,0.000000,0.000000}%
\pgfsetstrokecolor{currentstroke}%
\pgfsetdash{}{0pt}%
\pgfpathmoveto{\pgfqpoint{0.625000in}{3.720336in}}%
\pgfpathlineto{\pgfqpoint{0.630664in}{3.714912in}}%
\pgfpathlineto{\pgfqpoint{0.625000in}{3.709325in}}%
\pgfusepath{stroke}%
\end{pgfscope}%
\begin{pgfscope}%
\pgfpathrectangle{\pgfqpoint{0.625000in}{0.550000in}}{\pgfqpoint{3.875000in}{3.850000in}} %
\pgfusepath{clip}%
\pgfsetbuttcap%
\pgfsetroundjoin%
\pgfsetlinewidth{0.250937pt}%
\definecolor{currentstroke}{rgb}{0.000000,0.000000,0.000000}%
\pgfsetstrokecolor{currentstroke}%
\pgfsetdash{}{0pt}%
\pgfpathmoveto{\pgfqpoint{0.625000in}{3.874562in}}%
\pgfpathlineto{\pgfqpoint{0.630531in}{3.869298in}}%
\pgfpathlineto{\pgfqpoint{0.625000in}{3.864013in}}%
\pgfusepath{stroke}%
\end{pgfscope}%
\begin{pgfscope}%
\pgfpathrectangle{\pgfqpoint{0.625000in}{0.550000in}}{\pgfqpoint{3.875000in}{3.850000in}} %
\pgfusepath{clip}%
\pgfsetbuttcap%
\pgfsetroundjoin%
\pgfsetlinewidth{0.250937pt}%
\definecolor{currentstroke}{rgb}{0.000000,0.000000,0.000000}%
\pgfsetstrokecolor{currentstroke}%
\pgfsetdash{}{0pt}%
\pgfpathmoveto{\pgfqpoint{0.625000in}{4.029134in}}%
\pgfpathlineto{\pgfqpoint{0.631999in}{4.023684in}}%
\pgfpathlineto{\pgfqpoint{0.625000in}{4.018287in}}%
\pgfusepath{stroke}%
\end{pgfscope}%
\begin{pgfscope}%
\pgfpathrectangle{\pgfqpoint{0.625000in}{0.550000in}}{\pgfqpoint{3.875000in}{3.850000in}} %
\pgfusepath{clip}%
\pgfsetbuttcap%
\pgfsetroundjoin%
\pgfsetlinewidth{0.250937pt}%
\definecolor{currentstroke}{rgb}{0.000000,0.000000,0.000000}%
\pgfsetstrokecolor{currentstroke}%
\pgfsetdash{}{0pt}%
\pgfpathmoveto{\pgfqpoint{0.625000in}{4.183365in}}%
\pgfpathlineto{\pgfqpoint{0.630501in}{4.178070in}}%
\pgfpathlineto{\pgfqpoint{0.625000in}{4.173023in}}%
\pgfusepath{stroke}%
\end{pgfscope}%
\begin{pgfscope}%
\pgfpathrectangle{\pgfqpoint{0.625000in}{0.550000in}}{\pgfqpoint{3.875000in}{3.850000in}} %
\pgfusepath{clip}%
\pgfsetbuttcap%
\pgfsetroundjoin%
\pgfsetlinewidth{0.250937pt}%
\definecolor{currentstroke}{rgb}{0.000000,0.000000,0.000000}%
\pgfsetstrokecolor{currentstroke}%
\pgfsetdash{}{0pt}%
\pgfpathmoveto{\pgfqpoint{0.625000in}{4.337749in}}%
\pgfpathlineto{\pgfqpoint{0.630498in}{4.332456in}}%
\pgfpathlineto{\pgfqpoint{0.625000in}{4.327145in}}%
\pgfusepath{stroke}%
\end{pgfscope}%
\begin{pgfscope}%
\pgfpathrectangle{\pgfqpoint{0.625000in}{0.550000in}}{\pgfqpoint{3.875000in}{3.850000in}} %
\pgfusepath{clip}%
\pgfsetbuttcap%
\pgfsetroundjoin%
\pgfsetlinewidth{0.250937pt}%
\definecolor{currentstroke}{rgb}{0.000000,0.000000,0.000000}%
\pgfsetstrokecolor{currentstroke}%
\pgfsetdash{}{0pt}%
\pgfpathmoveto{\pgfqpoint{0.634712in}{1.188518in}}%
\pgfpathlineto{\pgfqpoint{0.632658in}{1.196491in}}%
\pgfpathlineto{\pgfqpoint{0.634712in}{1.199578in}}%
\pgfpathlineto{\pgfqpoint{0.638858in}{1.196491in}}%
\pgfpathlineto{\pgfqpoint{0.634712in}{1.188518in}}%
\pgfusepath{stroke}%
\end{pgfscope}%
\begin{pgfscope}%
\pgfpathrectangle{\pgfqpoint{0.625000in}{0.550000in}}{\pgfqpoint{3.875000in}{3.850000in}} %
\pgfusepath{clip}%
\pgfsetbuttcap%
\pgfsetroundjoin%
\pgfsetlinewidth{0.250937pt}%
\definecolor{currentstroke}{rgb}{0.000000,0.000000,0.000000}%
\pgfsetstrokecolor{currentstroke}%
\pgfsetdash{}{0pt}%
\pgfpathmoveto{\pgfqpoint{0.634712in}{3.760072in}}%
\pgfpathlineto{\pgfqpoint{0.632658in}{3.763158in}}%
\pgfpathlineto{\pgfqpoint{0.634712in}{3.771132in}}%
\pgfpathlineto{\pgfqpoint{0.638858in}{3.763158in}}%
\pgfpathlineto{\pgfqpoint{0.634712in}{3.760072in}}%
\pgfusepath{stroke}%
\end{pgfscope}%
\begin{pgfscope}%
\pgfpathrectangle{\pgfqpoint{0.625000in}{0.550000in}}{\pgfqpoint{3.875000in}{3.850000in}} %
\pgfusepath{clip}%
\pgfsetbuttcap%
\pgfsetroundjoin%
\pgfsetlinewidth{0.250937pt}%
\definecolor{currentstroke}{rgb}{0.000000,0.000000,0.000000}%
\pgfsetstrokecolor{currentstroke}%
\pgfsetdash{}{0pt}%
\pgfpathmoveto{\pgfqpoint{0.625000in}{0.633153in}}%
\pgfpathlineto{\pgfqpoint{0.631147in}{0.627193in}}%
\pgfpathlineto{\pgfqpoint{0.625000in}{0.621251in}}%
\pgfusepath{stroke}%
\end{pgfscope}%
\begin{pgfscope}%
\pgfpathrectangle{\pgfqpoint{0.625000in}{0.550000in}}{\pgfqpoint{3.875000in}{3.850000in}} %
\pgfusepath{clip}%
\pgfsetbuttcap%
\pgfsetroundjoin%
\pgfsetlinewidth{0.250937pt}%
\definecolor{currentstroke}{rgb}{0.000000,0.000000,0.000000}%
\pgfsetstrokecolor{currentstroke}%
\pgfsetdash{}{0pt}%
\pgfpathmoveto{\pgfqpoint{0.625000in}{0.787114in}}%
\pgfpathlineto{\pgfqpoint{0.630988in}{0.781579in}}%
\pgfpathlineto{\pgfqpoint{0.625000in}{0.775797in}}%
\pgfusepath{stroke}%
\end{pgfscope}%
\begin{pgfscope}%
\pgfpathrectangle{\pgfqpoint{0.625000in}{0.550000in}}{\pgfqpoint{3.875000in}{3.850000in}} %
\pgfusepath{clip}%
\pgfsetbuttcap%
\pgfsetroundjoin%
\pgfsetlinewidth{0.250937pt}%
\definecolor{currentstroke}{rgb}{0.000000,0.000000,0.000000}%
\pgfsetstrokecolor{currentstroke}%
\pgfsetdash{}{0pt}%
\pgfpathmoveto{\pgfqpoint{0.625000in}{0.941799in}}%
\pgfpathlineto{\pgfqpoint{0.632386in}{0.935965in}}%
\pgfpathlineto{\pgfqpoint{0.625000in}{0.930078in}}%
\pgfusepath{stroke}%
\end{pgfscope}%
\begin{pgfscope}%
\pgfpathrectangle{\pgfqpoint{0.625000in}{0.550000in}}{\pgfqpoint{3.875000in}{3.850000in}} %
\pgfusepath{clip}%
\pgfsetbuttcap%
\pgfsetroundjoin%
\pgfsetlinewidth{0.250937pt}%
\definecolor{currentstroke}{rgb}{0.000000,0.000000,0.000000}%
\pgfsetstrokecolor{currentstroke}%
\pgfsetdash{}{0pt}%
\pgfpathmoveto{\pgfqpoint{0.625000in}{1.096115in}}%
\pgfpathlineto{\pgfqpoint{0.631011in}{1.090351in}}%
\pgfpathlineto{\pgfqpoint{0.625000in}{1.084608in}}%
\pgfusepath{stroke}%
\end{pgfscope}%
\begin{pgfscope}%
\pgfpathrectangle{\pgfqpoint{0.625000in}{0.550000in}}{\pgfqpoint{3.875000in}{3.850000in}} %
\pgfusepath{clip}%
\pgfsetbuttcap%
\pgfsetroundjoin%
\pgfsetlinewidth{0.250937pt}%
\definecolor{currentstroke}{rgb}{0.000000,0.000000,0.000000}%
\pgfsetstrokecolor{currentstroke}%
\pgfsetdash{}{0pt}%
\pgfpathmoveto{\pgfqpoint{0.625000in}{1.250647in}}%
\pgfpathlineto{\pgfqpoint{0.630988in}{1.244737in}}%
\pgfpathlineto{\pgfqpoint{0.625000in}{1.238990in}}%
\pgfusepath{stroke}%
\end{pgfscope}%
\begin{pgfscope}%
\pgfpathrectangle{\pgfqpoint{0.625000in}{0.550000in}}{\pgfqpoint{3.875000in}{3.850000in}} %
\pgfusepath{clip}%
\pgfsetbuttcap%
\pgfsetroundjoin%
\pgfsetlinewidth{0.250937pt}%
\definecolor{currentstroke}{rgb}{0.000000,0.000000,0.000000}%
\pgfsetstrokecolor{currentstroke}%
\pgfsetdash{}{0pt}%
\pgfpathmoveto{\pgfqpoint{0.625000in}{1.404727in}}%
\pgfpathlineto{\pgfqpoint{0.630891in}{1.399123in}}%
\pgfpathlineto{\pgfqpoint{0.625000in}{1.393515in}}%
\pgfusepath{stroke}%
\end{pgfscope}%
\begin{pgfscope}%
\pgfpathrectangle{\pgfqpoint{0.625000in}{0.550000in}}{\pgfqpoint{3.875000in}{3.850000in}} %
\pgfusepath{clip}%
\pgfsetbuttcap%
\pgfsetroundjoin%
\pgfsetlinewidth{0.250937pt}%
\definecolor{currentstroke}{rgb}{0.000000,0.000000,0.000000}%
\pgfsetstrokecolor{currentstroke}%
\pgfsetdash{}{0pt}%
\pgfpathmoveto{\pgfqpoint{0.625000in}{1.559133in}}%
\pgfpathlineto{\pgfqpoint{0.630904in}{1.553509in}}%
\pgfpathlineto{\pgfqpoint{0.625000in}{1.547701in}}%
\pgfusepath{stroke}%
\end{pgfscope}%
\begin{pgfscope}%
\pgfpathrectangle{\pgfqpoint{0.625000in}{0.550000in}}{\pgfqpoint{3.875000in}{3.850000in}} %
\pgfusepath{clip}%
\pgfsetbuttcap%
\pgfsetroundjoin%
\pgfsetlinewidth{0.250937pt}%
\definecolor{currentstroke}{rgb}{0.000000,0.000000,0.000000}%
\pgfsetstrokecolor{currentstroke}%
\pgfsetdash{}{0pt}%
\pgfpathmoveto{\pgfqpoint{0.625000in}{1.713489in}}%
\pgfpathlineto{\pgfqpoint{0.632208in}{1.707895in}}%
\pgfpathlineto{\pgfqpoint{0.625000in}{1.702265in}}%
\pgfusepath{stroke}%
\end{pgfscope}%
\begin{pgfscope}%
\pgfpathrectangle{\pgfqpoint{0.625000in}{0.550000in}}{\pgfqpoint{3.875000in}{3.850000in}} %
\pgfusepath{clip}%
\pgfsetbuttcap%
\pgfsetroundjoin%
\pgfsetlinewidth{0.250937pt}%
\definecolor{currentstroke}{rgb}{0.000000,0.000000,0.000000}%
\pgfsetstrokecolor{currentstroke}%
\pgfsetdash{}{0pt}%
\pgfpathmoveto{\pgfqpoint{0.625000in}{1.867882in}}%
\pgfpathlineto{\pgfqpoint{0.630813in}{1.862281in}}%
\pgfpathlineto{\pgfqpoint{0.625000in}{1.856795in}}%
\pgfusepath{stroke}%
\end{pgfscope}%
\begin{pgfscope}%
\pgfpathrectangle{\pgfqpoint{0.625000in}{0.550000in}}{\pgfqpoint{3.875000in}{3.850000in}} %
\pgfusepath{clip}%
\pgfsetbuttcap%
\pgfsetroundjoin%
\pgfsetlinewidth{0.250937pt}%
\definecolor{currentstroke}{rgb}{0.000000,0.000000,0.000000}%
\pgfsetstrokecolor{currentstroke}%
\pgfsetdash{}{0pt}%
\pgfpathmoveto{\pgfqpoint{0.625000in}{2.022422in}}%
\pgfpathlineto{\pgfqpoint{0.630760in}{2.016667in}}%
\pgfpathlineto{\pgfqpoint{0.625000in}{2.010959in}}%
\pgfusepath{stroke}%
\end{pgfscope}%
\begin{pgfscope}%
\pgfpathrectangle{\pgfqpoint{0.625000in}{0.550000in}}{\pgfqpoint{3.875000in}{3.850000in}} %
\pgfusepath{clip}%
\pgfsetbuttcap%
\pgfsetroundjoin%
\pgfsetlinewidth{0.250937pt}%
\definecolor{currentstroke}{rgb}{0.000000,0.000000,0.000000}%
\pgfsetstrokecolor{currentstroke}%
\pgfsetdash{}{0pt}%
\pgfpathmoveto{\pgfqpoint{0.625000in}{2.176535in}}%
\pgfpathlineto{\pgfqpoint{0.630798in}{2.171053in}}%
\pgfpathlineto{\pgfqpoint{0.625000in}{2.165413in}}%
\pgfusepath{stroke}%
\end{pgfscope}%
\begin{pgfscope}%
\pgfpathrectangle{\pgfqpoint{0.625000in}{0.550000in}}{\pgfqpoint{3.875000in}{3.850000in}} %
\pgfusepath{clip}%
\pgfsetbuttcap%
\pgfsetroundjoin%
\pgfsetlinewidth{0.250937pt}%
\definecolor{currentstroke}{rgb}{0.000000,0.000000,0.000000}%
\pgfsetstrokecolor{currentstroke}%
\pgfsetdash{}{0pt}%
\pgfpathmoveto{\pgfqpoint{0.625000in}{2.330866in}}%
\pgfpathlineto{\pgfqpoint{0.630794in}{2.325439in}}%
\pgfpathlineto{\pgfqpoint{0.625000in}{2.319631in}}%
\pgfusepath{stroke}%
\end{pgfscope}%
\begin{pgfscope}%
\pgfpathrectangle{\pgfqpoint{0.625000in}{0.550000in}}{\pgfqpoint{3.875000in}{3.850000in}} %
\pgfusepath{clip}%
\pgfsetbuttcap%
\pgfsetroundjoin%
\pgfsetlinewidth{0.250937pt}%
\definecolor{currentstroke}{rgb}{0.000000,0.000000,0.000000}%
\pgfsetstrokecolor{currentstroke}%
\pgfsetdash{}{0pt}%
\pgfpathmoveto{\pgfqpoint{0.625000in}{2.486202in}}%
\pgfpathlineto{\pgfqpoint{0.628069in}{2.489474in}}%
\pgfpathlineto{\pgfqpoint{0.628805in}{2.499123in}}%
\pgfpathlineto{\pgfqpoint{0.632740in}{2.508772in}}%
\pgfpathlineto{\pgfqpoint{0.634712in}{2.513632in}}%
\pgfpathlineto{\pgfqpoint{0.639403in}{2.518421in}}%
\pgfpathlineto{\pgfqpoint{0.644424in}{2.525210in}}%
\pgfpathlineto{\pgfqpoint{0.648912in}{2.528070in}}%
\pgfpathlineto{\pgfqpoint{0.654135in}{2.532867in}}%
\pgfpathlineto{\pgfqpoint{0.663847in}{2.537654in}}%
\pgfpathlineto{\pgfqpoint{0.664097in}{2.537719in}}%
\pgfpathlineto{\pgfqpoint{0.673559in}{2.541254in}}%
\pgfpathlineto{\pgfqpoint{0.683271in}{2.542970in}}%
\pgfpathlineto{\pgfqpoint{0.692982in}{2.543199in}}%
\pgfpathlineto{\pgfqpoint{0.702694in}{2.541983in}}%
\pgfpathlineto{\pgfqpoint{0.712406in}{2.539199in}}%
\pgfpathlineto{\pgfqpoint{0.715960in}{2.537719in}}%
\pgfpathlineto{\pgfqpoint{0.722118in}{2.534674in}}%
\pgfpathlineto{\pgfqpoint{0.731666in}{2.528070in}}%
\pgfpathlineto{\pgfqpoint{0.731830in}{2.527923in}}%
\pgfpathlineto{\pgfqpoint{0.741020in}{2.518421in}}%
\pgfpathlineto{\pgfqpoint{0.741541in}{2.517643in}}%
\pgfpathlineto{\pgfqpoint{0.747360in}{2.508772in}}%
\pgfpathlineto{\pgfqpoint{0.751250in}{2.499123in}}%
\pgfpathlineto{\pgfqpoint{0.751253in}{2.499108in}}%
\pgfpathlineto{\pgfqpoint{0.753688in}{2.489474in}}%
\pgfpathlineto{\pgfqpoint{0.754465in}{2.479825in}}%
\pgfpathlineto{\pgfqpoint{0.753688in}{2.470175in}}%
\pgfpathlineto{\pgfqpoint{0.751253in}{2.460541in}}%
\pgfpathlineto{\pgfqpoint{0.751250in}{2.460526in}}%
\pgfpathlineto{\pgfqpoint{0.747360in}{2.450877in}}%
\pgfpathlineto{\pgfqpoint{0.741541in}{2.442006in}}%
\pgfpathlineto{\pgfqpoint{0.741020in}{2.441228in}}%
\pgfpathlineto{\pgfqpoint{0.731830in}{2.431726in}}%
\pgfpathlineto{\pgfqpoint{0.731666in}{2.431579in}}%
\pgfpathlineto{\pgfqpoint{0.722118in}{2.424976in}}%
\pgfpathlineto{\pgfqpoint{0.715960in}{2.421930in}}%
\pgfpathlineto{\pgfqpoint{0.712406in}{2.420450in}}%
\pgfpathlineto{\pgfqpoint{0.702694in}{2.417666in}}%
\pgfpathlineto{\pgfqpoint{0.692982in}{2.416450in}}%
\pgfpathlineto{\pgfqpoint{0.683271in}{2.416679in}}%
\pgfpathlineto{\pgfqpoint{0.673559in}{2.418396in}}%
\pgfpathlineto{\pgfqpoint{0.664097in}{2.421930in}}%
\pgfpathlineto{\pgfqpoint{0.663847in}{2.421995in}}%
\pgfpathlineto{\pgfqpoint{0.654135in}{2.426783in}}%
\pgfpathlineto{\pgfqpoint{0.648912in}{2.431579in}}%
\pgfpathlineto{\pgfqpoint{0.644424in}{2.434439in}}%
\pgfpathlineto{\pgfqpoint{0.639403in}{2.441228in}}%
\pgfpathlineto{\pgfqpoint{0.634712in}{2.446017in}}%
\pgfpathlineto{\pgfqpoint{0.632740in}{2.450877in}}%
\pgfpathlineto{\pgfqpoint{0.628814in}{2.460526in}}%
\pgfpathlineto{\pgfqpoint{0.628069in}{2.470175in}}%
\pgfpathlineto{\pgfqpoint{0.625000in}{2.473447in}}%
\pgfusepath{stroke}%
\end{pgfscope}%
\begin{pgfscope}%
\pgfpathrectangle{\pgfqpoint{0.625000in}{0.550000in}}{\pgfqpoint{3.875000in}{3.850000in}} %
\pgfusepath{clip}%
\pgfsetbuttcap%
\pgfsetroundjoin%
\pgfsetlinewidth{0.250937pt}%
\definecolor{currentstroke}{rgb}{0.000000,0.000000,0.000000}%
\pgfsetstrokecolor{currentstroke}%
\pgfsetdash{}{0pt}%
\pgfpathmoveto{\pgfqpoint{0.625000in}{2.640011in}}%
\pgfpathlineto{\pgfqpoint{0.630787in}{2.634211in}}%
\pgfpathlineto{\pgfqpoint{0.625000in}{2.628791in}}%
\pgfusepath{stroke}%
\end{pgfscope}%
\begin{pgfscope}%
\pgfpathrectangle{\pgfqpoint{0.625000in}{0.550000in}}{\pgfqpoint{3.875000in}{3.850000in}} %
\pgfusepath{clip}%
\pgfsetbuttcap%
\pgfsetroundjoin%
\pgfsetlinewidth{0.250937pt}%
\definecolor{currentstroke}{rgb}{0.000000,0.000000,0.000000}%
\pgfsetstrokecolor{currentstroke}%
\pgfsetdash{}{0pt}%
\pgfpathmoveto{\pgfqpoint{0.625000in}{2.794211in}}%
\pgfpathlineto{\pgfqpoint{0.630773in}{2.788596in}}%
\pgfpathlineto{\pgfqpoint{0.625000in}{2.783139in}}%
\pgfusepath{stroke}%
\end{pgfscope}%
\begin{pgfscope}%
\pgfpathrectangle{\pgfqpoint{0.625000in}{0.550000in}}{\pgfqpoint{3.875000in}{3.850000in}} %
\pgfusepath{clip}%
\pgfsetbuttcap%
\pgfsetroundjoin%
\pgfsetlinewidth{0.250937pt}%
\definecolor{currentstroke}{rgb}{0.000000,0.000000,0.000000}%
\pgfsetstrokecolor{currentstroke}%
\pgfsetdash{}{0pt}%
\pgfpathmoveto{\pgfqpoint{0.625000in}{2.948622in}}%
\pgfpathlineto{\pgfqpoint{0.630692in}{2.942982in}}%
\pgfpathlineto{\pgfqpoint{0.625000in}{2.937296in}}%
\pgfusepath{stroke}%
\end{pgfscope}%
\begin{pgfscope}%
\pgfpathrectangle{\pgfqpoint{0.625000in}{0.550000in}}{\pgfqpoint{3.875000in}{3.850000in}} %
\pgfusepath{clip}%
\pgfsetbuttcap%
\pgfsetroundjoin%
\pgfsetlinewidth{0.250937pt}%
\definecolor{currentstroke}{rgb}{0.000000,0.000000,0.000000}%
\pgfsetstrokecolor{currentstroke}%
\pgfsetdash{}{0pt}%
\pgfpathmoveto{\pgfqpoint{0.625000in}{3.102800in}}%
\pgfpathlineto{\pgfqpoint{0.630760in}{3.097368in}}%
\pgfpathlineto{\pgfqpoint{0.625000in}{3.091821in}}%
\pgfusepath{stroke}%
\end{pgfscope}%
\begin{pgfscope}%
\pgfpathrectangle{\pgfqpoint{0.625000in}{0.550000in}}{\pgfqpoint{3.875000in}{3.850000in}} %
\pgfusepath{clip}%
\pgfsetbuttcap%
\pgfsetroundjoin%
\pgfsetlinewidth{0.250937pt}%
\definecolor{currentstroke}{rgb}{0.000000,0.000000,0.000000}%
\pgfsetstrokecolor{currentstroke}%
\pgfsetdash{}{0pt}%
\pgfpathmoveto{\pgfqpoint{0.625000in}{3.257190in}}%
\pgfpathlineto{\pgfqpoint{0.632052in}{3.251754in}}%
\pgfpathlineto{\pgfqpoint{0.625000in}{3.246354in}}%
\pgfusepath{stroke}%
\end{pgfscope}%
\begin{pgfscope}%
\pgfpathrectangle{\pgfqpoint{0.625000in}{0.550000in}}{\pgfqpoint{3.875000in}{3.850000in}} %
\pgfusepath{clip}%
\pgfsetbuttcap%
\pgfsetroundjoin%
\pgfsetlinewidth{0.250937pt}%
\definecolor{currentstroke}{rgb}{0.000000,0.000000,0.000000}%
\pgfsetstrokecolor{currentstroke}%
\pgfsetdash{}{0pt}%
\pgfpathmoveto{\pgfqpoint{0.625000in}{3.411815in}}%
\pgfpathlineto{\pgfqpoint{0.630771in}{3.406140in}}%
\pgfpathlineto{\pgfqpoint{0.625000in}{3.400650in}}%
\pgfusepath{stroke}%
\end{pgfscope}%
\begin{pgfscope}%
\pgfpathrectangle{\pgfqpoint{0.625000in}{0.550000in}}{\pgfqpoint{3.875000in}{3.850000in}} %
\pgfusepath{clip}%
\pgfsetbuttcap%
\pgfsetroundjoin%
\pgfsetlinewidth{0.250937pt}%
\definecolor{currentstroke}{rgb}{0.000000,0.000000,0.000000}%
\pgfsetstrokecolor{currentstroke}%
\pgfsetdash{}{0pt}%
\pgfpathmoveto{\pgfqpoint{0.625000in}{3.565974in}}%
\pgfpathlineto{\pgfqpoint{0.630733in}{3.560526in}}%
\pgfpathlineto{\pgfqpoint{0.625000in}{3.555082in}}%
\pgfusepath{stroke}%
\end{pgfscope}%
\begin{pgfscope}%
\pgfpathrectangle{\pgfqpoint{0.625000in}{0.550000in}}{\pgfqpoint{3.875000in}{3.850000in}} %
\pgfusepath{clip}%
\pgfsetbuttcap%
\pgfsetroundjoin%
\pgfsetlinewidth{0.250937pt}%
\definecolor{currentstroke}{rgb}{0.000000,0.000000,0.000000}%
\pgfsetstrokecolor{currentstroke}%
\pgfsetdash{}{0pt}%
\pgfpathmoveto{\pgfqpoint{0.625000in}{3.720422in}}%
\pgfpathlineto{\pgfqpoint{0.630753in}{3.714912in}}%
\pgfpathlineto{\pgfqpoint{0.625000in}{3.709237in}}%
\pgfusepath{stroke}%
\end{pgfscope}%
\begin{pgfscope}%
\pgfpathrectangle{\pgfqpoint{0.625000in}{0.550000in}}{\pgfqpoint{3.875000in}{3.850000in}} %
\pgfusepath{clip}%
\pgfsetbuttcap%
\pgfsetroundjoin%
\pgfsetlinewidth{0.250937pt}%
\definecolor{currentstroke}{rgb}{0.000000,0.000000,0.000000}%
\pgfsetstrokecolor{currentstroke}%
\pgfsetdash{}{0pt}%
\pgfpathmoveto{\pgfqpoint{0.625000in}{3.874649in}}%
\pgfpathlineto{\pgfqpoint{0.630623in}{3.869298in}}%
\pgfpathlineto{\pgfqpoint{0.625000in}{3.863926in}}%
\pgfusepath{stroke}%
\end{pgfscope}%
\begin{pgfscope}%
\pgfpathrectangle{\pgfqpoint{0.625000in}{0.550000in}}{\pgfqpoint{3.875000in}{3.850000in}} %
\pgfusepath{clip}%
\pgfsetbuttcap%
\pgfsetroundjoin%
\pgfsetlinewidth{0.250937pt}%
\definecolor{currentstroke}{rgb}{0.000000,0.000000,0.000000}%
\pgfsetstrokecolor{currentstroke}%
\pgfsetdash{}{0pt}%
\pgfpathmoveto{\pgfqpoint{0.625000in}{4.029219in}}%
\pgfpathlineto{\pgfqpoint{0.632108in}{4.023684in}}%
\pgfpathlineto{\pgfqpoint{0.625000in}{4.018204in}}%
\pgfusepath{stroke}%
\end{pgfscope}%
\begin{pgfscope}%
\pgfpathrectangle{\pgfqpoint{0.625000in}{0.550000in}}{\pgfqpoint{3.875000in}{3.850000in}} %
\pgfusepath{clip}%
\pgfsetbuttcap%
\pgfsetroundjoin%
\pgfsetlinewidth{0.250937pt}%
\definecolor{currentstroke}{rgb}{0.000000,0.000000,0.000000}%
\pgfsetstrokecolor{currentstroke}%
\pgfsetdash{}{0pt}%
\pgfpathmoveto{\pgfqpoint{0.625000in}{4.183454in}}%
\pgfpathlineto{\pgfqpoint{0.630593in}{4.178070in}}%
\pgfpathlineto{\pgfqpoint{0.625000in}{4.172938in}}%
\pgfusepath{stroke}%
\end{pgfscope}%
\begin{pgfscope}%
\pgfpathrectangle{\pgfqpoint{0.625000in}{0.550000in}}{\pgfqpoint{3.875000in}{3.850000in}} %
\pgfusepath{clip}%
\pgfsetbuttcap%
\pgfsetroundjoin%
\pgfsetlinewidth{0.250937pt}%
\definecolor{currentstroke}{rgb}{0.000000,0.000000,0.000000}%
\pgfsetstrokecolor{currentstroke}%
\pgfsetdash{}{0pt}%
\pgfpathmoveto{\pgfqpoint{0.625000in}{4.337836in}}%
\pgfpathlineto{\pgfqpoint{0.630589in}{4.332456in}}%
\pgfpathlineto{\pgfqpoint{0.625000in}{4.327058in}}%
\pgfusepath{stroke}%
\end{pgfscope}%
\begin{pgfscope}%
\pgfpathrectangle{\pgfqpoint{0.625000in}{0.550000in}}{\pgfqpoint{3.875000in}{3.850000in}} %
\pgfusepath{clip}%
\pgfsetbuttcap%
\pgfsetroundjoin%
\pgfsetlinewidth{0.250937pt}%
\definecolor{currentstroke}{rgb}{0.000000,0.000000,0.000000}%
\pgfsetstrokecolor{currentstroke}%
\pgfsetdash{}{0pt}%
\pgfpathmoveto{\pgfqpoint{0.634712in}{1.187840in}}%
\pgfpathlineto{\pgfqpoint{0.632483in}{1.196491in}}%
\pgfpathlineto{\pgfqpoint{0.634712in}{1.199840in}}%
\pgfpathlineto{\pgfqpoint{0.639210in}{1.196491in}}%
\pgfpathlineto{\pgfqpoint{0.634712in}{1.187840in}}%
\pgfusepath{stroke}%
\end{pgfscope}%
\begin{pgfscope}%
\pgfpathrectangle{\pgfqpoint{0.625000in}{0.550000in}}{\pgfqpoint{3.875000in}{3.850000in}} %
\pgfusepath{clip}%
\pgfsetbuttcap%
\pgfsetroundjoin%
\pgfsetlinewidth{0.250937pt}%
\definecolor{currentstroke}{rgb}{0.000000,0.000000,0.000000}%
\pgfsetstrokecolor{currentstroke}%
\pgfsetdash{}{0pt}%
\pgfpathmoveto{\pgfqpoint{0.634712in}{3.759809in}}%
\pgfpathlineto{\pgfqpoint{0.632483in}{3.763158in}}%
\pgfpathlineto{\pgfqpoint{0.634712in}{3.771809in}}%
\pgfpathlineto{\pgfqpoint{0.639210in}{3.763158in}}%
\pgfpathlineto{\pgfqpoint{0.634712in}{3.759809in}}%
\pgfusepath{stroke}%
\end{pgfscope}%
\begin{pgfscope}%
\pgfpathrectangle{\pgfqpoint{0.625000in}{0.550000in}}{\pgfqpoint{3.875000in}{3.850000in}} %
\pgfusepath{clip}%
\pgfsetbuttcap%
\pgfsetroundjoin%
\pgfsetlinewidth{0.250937pt}%
\definecolor{currentstroke}{rgb}{0.000000,0.000000,0.000000}%
\pgfsetstrokecolor{currentstroke}%
\pgfsetdash{}{0pt}%
\pgfpathmoveto{\pgfqpoint{0.625000in}{0.633229in}}%
\pgfpathlineto{\pgfqpoint{0.631225in}{0.627193in}}%
\pgfpathlineto{\pgfqpoint{0.625000in}{0.621175in}}%
\pgfusepath{stroke}%
\end{pgfscope}%
\begin{pgfscope}%
\pgfpathrectangle{\pgfqpoint{0.625000in}{0.550000in}}{\pgfqpoint{3.875000in}{3.850000in}} %
\pgfusepath{clip}%
\pgfsetbuttcap%
\pgfsetroundjoin%
\pgfsetlinewidth{0.250937pt}%
\definecolor{currentstroke}{rgb}{0.000000,0.000000,0.000000}%
\pgfsetstrokecolor{currentstroke}%
\pgfsetdash{}{0pt}%
\pgfpathmoveto{\pgfqpoint{0.625000in}{0.787191in}}%
\pgfpathlineto{\pgfqpoint{0.631072in}{0.781579in}}%
\pgfpathlineto{\pgfqpoint{0.625000in}{0.775717in}}%
\pgfusepath{stroke}%
\end{pgfscope}%
\begin{pgfscope}%
\pgfpathrectangle{\pgfqpoint{0.625000in}{0.550000in}}{\pgfqpoint{3.875000in}{3.850000in}} %
\pgfusepath{clip}%
\pgfsetbuttcap%
\pgfsetroundjoin%
\pgfsetlinewidth{0.250937pt}%
\definecolor{currentstroke}{rgb}{0.000000,0.000000,0.000000}%
\pgfsetstrokecolor{currentstroke}%
\pgfsetdash{}{0pt}%
\pgfpathmoveto{\pgfqpoint{0.625000in}{0.941876in}}%
\pgfpathlineto{\pgfqpoint{0.632483in}{0.935965in}}%
\pgfpathlineto{\pgfqpoint{0.625000in}{0.930001in}}%
\pgfusepath{stroke}%
\end{pgfscope}%
\begin{pgfscope}%
\pgfpathrectangle{\pgfqpoint{0.625000in}{0.550000in}}{\pgfqpoint{3.875000in}{3.850000in}} %
\pgfusepath{clip}%
\pgfsetbuttcap%
\pgfsetroundjoin%
\pgfsetlinewidth{0.250937pt}%
\definecolor{currentstroke}{rgb}{0.000000,0.000000,0.000000}%
\pgfsetstrokecolor{currentstroke}%
\pgfsetdash{}{0pt}%
\pgfpathmoveto{\pgfqpoint{0.625000in}{1.096194in}}%
\pgfpathlineto{\pgfqpoint{0.631093in}{1.090351in}}%
\pgfpathlineto{\pgfqpoint{0.625000in}{1.084529in}}%
\pgfusepath{stroke}%
\end{pgfscope}%
\begin{pgfscope}%
\pgfpathrectangle{\pgfqpoint{0.625000in}{0.550000in}}{\pgfqpoint{3.875000in}{3.850000in}} %
\pgfusepath{clip}%
\pgfsetbuttcap%
\pgfsetroundjoin%
\pgfsetlinewidth{0.250937pt}%
\definecolor{currentstroke}{rgb}{0.000000,0.000000,0.000000}%
\pgfsetstrokecolor{currentstroke}%
\pgfsetdash{}{0pt}%
\pgfpathmoveto{\pgfqpoint{0.625000in}{1.250730in}}%
\pgfpathlineto{\pgfqpoint{0.631072in}{1.244737in}}%
\pgfpathlineto{\pgfqpoint{0.625000in}{1.238909in}}%
\pgfusepath{stroke}%
\end{pgfscope}%
\begin{pgfscope}%
\pgfpathrectangle{\pgfqpoint{0.625000in}{0.550000in}}{\pgfqpoint{3.875000in}{3.850000in}} %
\pgfusepath{clip}%
\pgfsetbuttcap%
\pgfsetroundjoin%
\pgfsetlinewidth{0.250937pt}%
\definecolor{currentstroke}{rgb}{0.000000,0.000000,0.000000}%
\pgfsetstrokecolor{currentstroke}%
\pgfsetdash{}{0pt}%
\pgfpathmoveto{\pgfqpoint{0.625000in}{1.404809in}}%
\pgfpathlineto{\pgfqpoint{0.630977in}{1.399123in}}%
\pgfpathlineto{\pgfqpoint{0.625000in}{1.393434in}}%
\pgfusepath{stroke}%
\end{pgfscope}%
\begin{pgfscope}%
\pgfpathrectangle{\pgfqpoint{0.625000in}{0.550000in}}{\pgfqpoint{3.875000in}{3.850000in}} %
\pgfusepath{clip}%
\pgfsetbuttcap%
\pgfsetroundjoin%
\pgfsetlinewidth{0.250937pt}%
\definecolor{currentstroke}{rgb}{0.000000,0.000000,0.000000}%
\pgfsetstrokecolor{currentstroke}%
\pgfsetdash{}{0pt}%
\pgfpathmoveto{\pgfqpoint{0.625000in}{1.559213in}}%
\pgfpathlineto{\pgfqpoint{0.630987in}{1.553509in}}%
\pgfpathlineto{\pgfqpoint{0.625000in}{1.547619in}}%
\pgfusepath{stroke}%
\end{pgfscope}%
\begin{pgfscope}%
\pgfpathrectangle{\pgfqpoint{0.625000in}{0.550000in}}{\pgfqpoint{3.875000in}{3.850000in}} %
\pgfusepath{clip}%
\pgfsetbuttcap%
\pgfsetroundjoin%
\pgfsetlinewidth{0.250937pt}%
\definecolor{currentstroke}{rgb}{0.000000,0.000000,0.000000}%
\pgfsetstrokecolor{currentstroke}%
\pgfsetdash{}{0pt}%
\pgfpathmoveto{\pgfqpoint{0.625000in}{1.713570in}}%
\pgfpathlineto{\pgfqpoint{0.632312in}{1.707895in}}%
\pgfpathlineto{\pgfqpoint{0.625000in}{1.702184in}}%
\pgfusepath{stroke}%
\end{pgfscope}%
\begin{pgfscope}%
\pgfpathrectangle{\pgfqpoint{0.625000in}{0.550000in}}{\pgfqpoint{3.875000in}{3.850000in}} %
\pgfusepath{clip}%
\pgfsetbuttcap%
\pgfsetroundjoin%
\pgfsetlinewidth{0.250937pt}%
\definecolor{currentstroke}{rgb}{0.000000,0.000000,0.000000}%
\pgfsetstrokecolor{currentstroke}%
\pgfsetdash{}{0pt}%
\pgfpathmoveto{\pgfqpoint{0.625000in}{1.867965in}}%
\pgfpathlineto{\pgfqpoint{0.630899in}{1.862281in}}%
\pgfpathlineto{\pgfqpoint{0.625000in}{1.856714in}}%
\pgfusepath{stroke}%
\end{pgfscope}%
\begin{pgfscope}%
\pgfpathrectangle{\pgfqpoint{0.625000in}{0.550000in}}{\pgfqpoint{3.875000in}{3.850000in}} %
\pgfusepath{clip}%
\pgfsetbuttcap%
\pgfsetroundjoin%
\pgfsetlinewidth{0.250937pt}%
\definecolor{currentstroke}{rgb}{0.000000,0.000000,0.000000}%
\pgfsetstrokecolor{currentstroke}%
\pgfsetdash{}{0pt}%
\pgfpathmoveto{\pgfqpoint{0.625000in}{2.022509in}}%
\pgfpathlineto{\pgfqpoint{0.630848in}{2.016667in}}%
\pgfpathlineto{\pgfqpoint{0.625000in}{2.010872in}}%
\pgfusepath{stroke}%
\end{pgfscope}%
\begin{pgfscope}%
\pgfpathrectangle{\pgfqpoint{0.625000in}{0.550000in}}{\pgfqpoint{3.875000in}{3.850000in}} %
\pgfusepath{clip}%
\pgfsetbuttcap%
\pgfsetroundjoin%
\pgfsetlinewidth{0.250937pt}%
\definecolor{currentstroke}{rgb}{0.000000,0.000000,0.000000}%
\pgfsetstrokecolor{currentstroke}%
\pgfsetdash{}{0pt}%
\pgfpathmoveto{\pgfqpoint{0.625000in}{2.176617in}}%
\pgfpathlineto{\pgfqpoint{0.630885in}{2.171053in}}%
\pgfpathlineto{\pgfqpoint{0.625000in}{2.165329in}}%
\pgfusepath{stroke}%
\end{pgfscope}%
\begin{pgfscope}%
\pgfpathrectangle{\pgfqpoint{0.625000in}{0.550000in}}{\pgfqpoint{3.875000in}{3.850000in}} %
\pgfusepath{clip}%
\pgfsetbuttcap%
\pgfsetroundjoin%
\pgfsetlinewidth{0.250937pt}%
\definecolor{currentstroke}{rgb}{0.000000,0.000000,0.000000}%
\pgfsetstrokecolor{currentstroke}%
\pgfsetdash{}{0pt}%
\pgfpathmoveto{\pgfqpoint{0.625000in}{2.330949in}}%
\pgfpathlineto{\pgfqpoint{0.630882in}{2.325439in}}%
\pgfpathlineto{\pgfqpoint{0.625000in}{2.319543in}}%
\pgfusepath{stroke}%
\end{pgfscope}%
\begin{pgfscope}%
\pgfpathrectangle{\pgfqpoint{0.625000in}{0.550000in}}{\pgfqpoint{3.875000in}{3.850000in}} %
\pgfusepath{clip}%
\pgfsetbuttcap%
\pgfsetroundjoin%
\pgfsetlinewidth{0.250937pt}%
\definecolor{currentstroke}{rgb}{0.000000,0.000000,0.000000}%
\pgfsetstrokecolor{currentstroke}%
\pgfsetdash{}{0pt}%
\pgfpathmoveto{\pgfqpoint{0.625000in}{2.486286in}}%
\pgfpathlineto{\pgfqpoint{0.627990in}{2.489474in}}%
\pgfpathlineto{\pgfqpoint{0.628726in}{2.499123in}}%
\pgfpathlineto{\pgfqpoint{0.632585in}{2.508772in}}%
\pgfpathlineto{\pgfqpoint{0.634712in}{2.514013in}}%
\pgfpathlineto{\pgfqpoint{0.639030in}{2.518421in}}%
\pgfpathlineto{\pgfqpoint{0.644424in}{2.525715in}}%
\pgfpathlineto{\pgfqpoint{0.648121in}{2.528070in}}%
\pgfpathlineto{\pgfqpoint{0.654135in}{2.533594in}}%
\pgfpathlineto{\pgfqpoint{0.662500in}{2.537719in}}%
\pgfpathlineto{\pgfqpoint{0.663847in}{2.538653in}}%
\pgfpathlineto{\pgfqpoint{0.673559in}{2.542343in}}%
\pgfpathlineto{\pgfqpoint{0.683271in}{2.544200in}}%
\pgfpathlineto{\pgfqpoint{0.692982in}{2.544635in}}%
\pgfpathlineto{\pgfqpoint{0.702694in}{2.543688in}}%
\pgfpathlineto{\pgfqpoint{0.712406in}{2.541240in}}%
\pgfpathlineto{\pgfqpoint{0.720860in}{2.537719in}}%
\pgfpathlineto{\pgfqpoint{0.722118in}{2.537097in}}%
\pgfpathlineto{\pgfqpoint{0.731830in}{2.530906in}}%
\pgfpathlineto{\pgfqpoint{0.735421in}{2.528070in}}%
\pgfpathlineto{\pgfqpoint{0.741541in}{2.521683in}}%
\pgfpathlineto{\pgfqpoint{0.744393in}{2.518421in}}%
\pgfpathlineto{\pgfqpoint{0.750357in}{2.508772in}}%
\pgfpathlineto{\pgfqpoint{0.751253in}{2.506549in}}%
\pgfpathlineto{\pgfqpoint{0.754395in}{2.499123in}}%
\pgfpathlineto{\pgfqpoint{0.756678in}{2.489474in}}%
\pgfpathlineto{\pgfqpoint{0.757405in}{2.479825in}}%
\pgfpathlineto{\pgfqpoint{0.756678in}{2.470175in}}%
\pgfpathlineto{\pgfqpoint{0.754395in}{2.460526in}}%
\pgfpathlineto{\pgfqpoint{0.751253in}{2.453100in}}%
\pgfpathlineto{\pgfqpoint{0.750357in}{2.450877in}}%
\pgfpathlineto{\pgfqpoint{0.744393in}{2.441228in}}%
\pgfpathlineto{\pgfqpoint{0.741541in}{2.437966in}}%
\pgfpathlineto{\pgfqpoint{0.735421in}{2.431579in}}%
\pgfpathlineto{\pgfqpoint{0.731830in}{2.428743in}}%
\pgfpathlineto{\pgfqpoint{0.722118in}{2.422552in}}%
\pgfpathlineto{\pgfqpoint{0.720860in}{2.421930in}}%
\pgfpathlineto{\pgfqpoint{0.712406in}{2.418409in}}%
\pgfpathlineto{\pgfqpoint{0.702694in}{2.415961in}}%
\pgfpathlineto{\pgfqpoint{0.692982in}{2.415014in}}%
\pgfpathlineto{\pgfqpoint{0.683271in}{2.415449in}}%
\pgfpathlineto{\pgfqpoint{0.673559in}{2.417306in}}%
\pgfpathlineto{\pgfqpoint{0.663847in}{2.420996in}}%
\pgfpathlineto{\pgfqpoint{0.662500in}{2.421930in}}%
\pgfpathlineto{\pgfqpoint{0.654135in}{2.426056in}}%
\pgfpathlineto{\pgfqpoint{0.648121in}{2.431579in}}%
\pgfpathlineto{\pgfqpoint{0.644424in}{2.433935in}}%
\pgfpathlineto{\pgfqpoint{0.639030in}{2.441228in}}%
\pgfpathlineto{\pgfqpoint{0.634712in}{2.445636in}}%
\pgfpathlineto{\pgfqpoint{0.632585in}{2.450877in}}%
\pgfpathlineto{\pgfqpoint{0.628735in}{2.460526in}}%
\pgfpathlineto{\pgfqpoint{0.627990in}{2.470175in}}%
\pgfpathlineto{\pgfqpoint{0.625000in}{2.473363in}}%
\pgfusepath{stroke}%
\end{pgfscope}%
\begin{pgfscope}%
\pgfpathrectangle{\pgfqpoint{0.625000in}{0.550000in}}{\pgfqpoint{3.875000in}{3.850000in}} %
\pgfusepath{clip}%
\pgfsetbuttcap%
\pgfsetroundjoin%
\pgfsetlinewidth{0.250937pt}%
\definecolor{currentstroke}{rgb}{0.000000,0.000000,0.000000}%
\pgfsetstrokecolor{currentstroke}%
\pgfsetdash{}{0pt}%
\pgfpathmoveto{\pgfqpoint{0.625000in}{2.640100in}}%
\pgfpathlineto{\pgfqpoint{0.630875in}{2.634211in}}%
\pgfpathlineto{\pgfqpoint{0.625000in}{2.628708in}}%
\pgfusepath{stroke}%
\end{pgfscope}%
\begin{pgfscope}%
\pgfpathrectangle{\pgfqpoint{0.625000in}{0.550000in}}{\pgfqpoint{3.875000in}{3.850000in}} %
\pgfusepath{clip}%
\pgfsetbuttcap%
\pgfsetroundjoin%
\pgfsetlinewidth{0.250937pt}%
\definecolor{currentstroke}{rgb}{0.000000,0.000000,0.000000}%
\pgfsetstrokecolor{currentstroke}%
\pgfsetdash{}{0pt}%
\pgfpathmoveto{\pgfqpoint{0.625000in}{2.794297in}}%
\pgfpathlineto{\pgfqpoint{0.630861in}{2.788596in}}%
\pgfpathlineto{\pgfqpoint{0.625000in}{2.783056in}}%
\pgfusepath{stroke}%
\end{pgfscope}%
\begin{pgfscope}%
\pgfpathrectangle{\pgfqpoint{0.625000in}{0.550000in}}{\pgfqpoint{3.875000in}{3.850000in}} %
\pgfusepath{clip}%
\pgfsetbuttcap%
\pgfsetroundjoin%
\pgfsetlinewidth{0.250937pt}%
\definecolor{currentstroke}{rgb}{0.000000,0.000000,0.000000}%
\pgfsetstrokecolor{currentstroke}%
\pgfsetdash{}{0pt}%
\pgfpathmoveto{\pgfqpoint{0.625000in}{2.948710in}}%
\pgfpathlineto{\pgfqpoint{0.630781in}{2.942982in}}%
\pgfpathlineto{\pgfqpoint{0.625000in}{2.937207in}}%
\pgfusepath{stroke}%
\end{pgfscope}%
\begin{pgfscope}%
\pgfpathrectangle{\pgfqpoint{0.625000in}{0.550000in}}{\pgfqpoint{3.875000in}{3.850000in}} %
\pgfusepath{clip}%
\pgfsetbuttcap%
\pgfsetroundjoin%
\pgfsetlinewidth{0.250937pt}%
\definecolor{currentstroke}{rgb}{0.000000,0.000000,0.000000}%
\pgfsetstrokecolor{currentstroke}%
\pgfsetdash{}{0pt}%
\pgfpathmoveto{\pgfqpoint{0.625000in}{3.102882in}}%
\pgfpathlineto{\pgfqpoint{0.630847in}{3.097368in}}%
\pgfpathlineto{\pgfqpoint{0.625000in}{3.091737in}}%
\pgfusepath{stroke}%
\end{pgfscope}%
\begin{pgfscope}%
\pgfpathrectangle{\pgfqpoint{0.625000in}{0.550000in}}{\pgfqpoint{3.875000in}{3.850000in}} %
\pgfusepath{clip}%
\pgfsetbuttcap%
\pgfsetroundjoin%
\pgfsetlinewidth{0.250937pt}%
\definecolor{currentstroke}{rgb}{0.000000,0.000000,0.000000}%
\pgfsetstrokecolor{currentstroke}%
\pgfsetdash{}{0pt}%
\pgfpathmoveto{\pgfqpoint{0.625000in}{3.257276in}}%
\pgfpathlineto{\pgfqpoint{0.632163in}{3.251754in}}%
\pgfpathlineto{\pgfqpoint{0.625000in}{3.246269in}}%
\pgfusepath{stroke}%
\end{pgfscope}%
\begin{pgfscope}%
\pgfpathrectangle{\pgfqpoint{0.625000in}{0.550000in}}{\pgfqpoint{3.875000in}{3.850000in}} %
\pgfusepath{clip}%
\pgfsetbuttcap%
\pgfsetroundjoin%
\pgfsetlinewidth{0.250937pt}%
\definecolor{currentstroke}{rgb}{0.000000,0.000000,0.000000}%
\pgfsetstrokecolor{currentstroke}%
\pgfsetdash{}{0pt}%
\pgfpathmoveto{\pgfqpoint{0.625000in}{3.411901in}}%
\pgfpathlineto{\pgfqpoint{0.630858in}{3.406140in}}%
\pgfpathlineto{\pgfqpoint{0.625000in}{3.400567in}}%
\pgfusepath{stroke}%
\end{pgfscope}%
\begin{pgfscope}%
\pgfpathrectangle{\pgfqpoint{0.625000in}{0.550000in}}{\pgfqpoint{3.875000in}{3.850000in}} %
\pgfusepath{clip}%
\pgfsetbuttcap%
\pgfsetroundjoin%
\pgfsetlinewidth{0.250937pt}%
\definecolor{currentstroke}{rgb}{0.000000,0.000000,0.000000}%
\pgfsetstrokecolor{currentstroke}%
\pgfsetdash{}{0pt}%
\pgfpathmoveto{\pgfqpoint{0.625000in}{3.566058in}}%
\pgfpathlineto{\pgfqpoint{0.630822in}{3.560526in}}%
\pgfpathlineto{\pgfqpoint{0.625000in}{3.554998in}}%
\pgfusepath{stroke}%
\end{pgfscope}%
\begin{pgfscope}%
\pgfpathrectangle{\pgfqpoint{0.625000in}{0.550000in}}{\pgfqpoint{3.875000in}{3.850000in}} %
\pgfusepath{clip}%
\pgfsetbuttcap%
\pgfsetroundjoin%
\pgfsetlinewidth{0.250937pt}%
\definecolor{currentstroke}{rgb}{0.000000,0.000000,0.000000}%
\pgfsetstrokecolor{currentstroke}%
\pgfsetdash{}{0pt}%
\pgfpathmoveto{\pgfqpoint{0.625000in}{3.720507in}}%
\pgfpathlineto{\pgfqpoint{0.630842in}{3.714912in}}%
\pgfpathlineto{\pgfqpoint{0.625000in}{3.709149in}}%
\pgfusepath{stroke}%
\end{pgfscope}%
\begin{pgfscope}%
\pgfpathrectangle{\pgfqpoint{0.625000in}{0.550000in}}{\pgfqpoint{3.875000in}{3.850000in}} %
\pgfusepath{clip}%
\pgfsetbuttcap%
\pgfsetroundjoin%
\pgfsetlinewidth{0.250937pt}%
\definecolor{currentstroke}{rgb}{0.000000,0.000000,0.000000}%
\pgfsetstrokecolor{currentstroke}%
\pgfsetdash{}{0pt}%
\pgfpathmoveto{\pgfqpoint{0.625000in}{3.874735in}}%
\pgfpathlineto{\pgfqpoint{0.630714in}{3.869298in}}%
\pgfpathlineto{\pgfqpoint{0.625000in}{3.863839in}}%
\pgfusepath{stroke}%
\end{pgfscope}%
\begin{pgfscope}%
\pgfpathrectangle{\pgfqpoint{0.625000in}{0.550000in}}{\pgfqpoint{3.875000in}{3.850000in}} %
\pgfusepath{clip}%
\pgfsetbuttcap%
\pgfsetroundjoin%
\pgfsetlinewidth{0.250937pt}%
\definecolor{currentstroke}{rgb}{0.000000,0.000000,0.000000}%
\pgfsetstrokecolor{currentstroke}%
\pgfsetdash{}{0pt}%
\pgfpathmoveto{\pgfqpoint{0.625000in}{4.029303in}}%
\pgfpathlineto{\pgfqpoint{0.632216in}{4.023684in}}%
\pgfpathlineto{\pgfqpoint{0.625000in}{4.018120in}}%
\pgfusepath{stroke}%
\end{pgfscope}%
\begin{pgfscope}%
\pgfpathrectangle{\pgfqpoint{0.625000in}{0.550000in}}{\pgfqpoint{3.875000in}{3.850000in}} %
\pgfusepath{clip}%
\pgfsetbuttcap%
\pgfsetroundjoin%
\pgfsetlinewidth{0.250937pt}%
\definecolor{currentstroke}{rgb}{0.000000,0.000000,0.000000}%
\pgfsetstrokecolor{currentstroke}%
\pgfsetdash{}{0pt}%
\pgfpathmoveto{\pgfqpoint{0.625000in}{4.183543in}}%
\pgfpathlineto{\pgfqpoint{0.630685in}{4.178070in}}%
\pgfpathlineto{\pgfqpoint{0.625000in}{4.172854in}}%
\pgfusepath{stroke}%
\end{pgfscope}%
\begin{pgfscope}%
\pgfpathrectangle{\pgfqpoint{0.625000in}{0.550000in}}{\pgfqpoint{3.875000in}{3.850000in}} %
\pgfusepath{clip}%
\pgfsetbuttcap%
\pgfsetroundjoin%
\pgfsetlinewidth{0.250937pt}%
\definecolor{currentstroke}{rgb}{0.000000,0.000000,0.000000}%
\pgfsetstrokecolor{currentstroke}%
\pgfsetdash{}{0pt}%
\pgfpathmoveto{\pgfqpoint{0.625000in}{4.337924in}}%
\pgfpathlineto{\pgfqpoint{0.630680in}{4.332456in}}%
\pgfpathlineto{\pgfqpoint{0.625000in}{4.326970in}}%
\pgfusepath{stroke}%
\end{pgfscope}%
\begin{pgfscope}%
\pgfpathrectangle{\pgfqpoint{0.625000in}{0.550000in}}{\pgfqpoint{3.875000in}{3.850000in}} %
\pgfusepath{clip}%
\pgfsetbuttcap%
\pgfsetroundjoin%
\pgfsetlinewidth{0.250937pt}%
\definecolor{currentstroke}{rgb}{0.000000,0.000000,0.000000}%
\pgfsetstrokecolor{currentstroke}%
\pgfsetdash{}{0pt}%
\pgfpathmoveto{\pgfqpoint{0.634712in}{1.187162in}}%
\pgfpathlineto{\pgfqpoint{0.632308in}{1.196491in}}%
\pgfpathlineto{\pgfqpoint{0.634712in}{1.200102in}}%
\pgfpathlineto{\pgfqpoint{0.639562in}{1.196491in}}%
\pgfpathlineto{\pgfqpoint{0.634712in}{1.187162in}}%
\pgfusepath{stroke}%
\end{pgfscope}%
\begin{pgfscope}%
\pgfpathrectangle{\pgfqpoint{0.625000in}{0.550000in}}{\pgfqpoint{3.875000in}{3.850000in}} %
\pgfusepath{clip}%
\pgfsetbuttcap%
\pgfsetroundjoin%
\pgfsetlinewidth{0.250937pt}%
\definecolor{currentstroke}{rgb}{0.000000,0.000000,0.000000}%
\pgfsetstrokecolor{currentstroke}%
\pgfsetdash{}{0pt}%
\pgfpathmoveto{\pgfqpoint{0.634712in}{3.759547in}}%
\pgfpathlineto{\pgfqpoint{0.632308in}{3.763158in}}%
\pgfpathlineto{\pgfqpoint{0.634712in}{3.772487in}}%
\pgfpathlineto{\pgfqpoint{0.639562in}{3.763158in}}%
\pgfpathlineto{\pgfqpoint{0.634712in}{3.759547in}}%
\pgfusepath{stroke}%
\end{pgfscope}%
\begin{pgfscope}%
\pgfpathrectangle{\pgfqpoint{0.625000in}{0.550000in}}{\pgfqpoint{3.875000in}{3.850000in}} %
\pgfusepath{clip}%
\pgfsetbuttcap%
\pgfsetroundjoin%
\pgfsetlinewidth{0.250937pt}%
\definecolor{currentstroke}{rgb}{0.000000,0.000000,0.000000}%
\pgfsetstrokecolor{currentstroke}%
\pgfsetdash{}{0pt}%
\pgfpathmoveto{\pgfqpoint{0.625000in}{0.633305in}}%
\pgfpathlineto{\pgfqpoint{0.631304in}{0.627193in}}%
\pgfpathlineto{\pgfqpoint{0.625000in}{0.621099in}}%
\pgfusepath{stroke}%
\end{pgfscope}%
\begin{pgfscope}%
\pgfpathrectangle{\pgfqpoint{0.625000in}{0.550000in}}{\pgfqpoint{3.875000in}{3.850000in}} %
\pgfusepath{clip}%
\pgfsetbuttcap%
\pgfsetroundjoin%
\pgfsetlinewidth{0.250937pt}%
\definecolor{currentstroke}{rgb}{0.000000,0.000000,0.000000}%
\pgfsetstrokecolor{currentstroke}%
\pgfsetdash{}{0pt}%
\pgfpathmoveto{\pgfqpoint{0.625000in}{0.787268in}}%
\pgfpathlineto{\pgfqpoint{0.631155in}{0.781579in}}%
\pgfpathlineto{\pgfqpoint{0.625000in}{0.775636in}}%
\pgfusepath{stroke}%
\end{pgfscope}%
\begin{pgfscope}%
\pgfpathrectangle{\pgfqpoint{0.625000in}{0.550000in}}{\pgfqpoint{3.875000in}{3.850000in}} %
\pgfusepath{clip}%
\pgfsetbuttcap%
\pgfsetroundjoin%
\pgfsetlinewidth{0.250937pt}%
\definecolor{currentstroke}{rgb}{0.000000,0.000000,0.000000}%
\pgfsetstrokecolor{currentstroke}%
\pgfsetdash{}{0pt}%
\pgfpathmoveto{\pgfqpoint{0.625000in}{0.941952in}}%
\pgfpathlineto{\pgfqpoint{0.632580in}{0.935965in}}%
\pgfpathlineto{\pgfqpoint{0.625000in}{0.929924in}}%
\pgfusepath{stroke}%
\end{pgfscope}%
\begin{pgfscope}%
\pgfpathrectangle{\pgfqpoint{0.625000in}{0.550000in}}{\pgfqpoint{3.875000in}{3.850000in}} %
\pgfusepath{clip}%
\pgfsetbuttcap%
\pgfsetroundjoin%
\pgfsetlinewidth{0.250937pt}%
\definecolor{currentstroke}{rgb}{0.000000,0.000000,0.000000}%
\pgfsetstrokecolor{currentstroke}%
\pgfsetdash{}{0pt}%
\pgfpathmoveto{\pgfqpoint{0.625000in}{1.096273in}}%
\pgfpathlineto{\pgfqpoint{0.631176in}{1.090351in}}%
\pgfpathlineto{\pgfqpoint{0.625000in}{1.084450in}}%
\pgfusepath{stroke}%
\end{pgfscope}%
\begin{pgfscope}%
\pgfpathrectangle{\pgfqpoint{0.625000in}{0.550000in}}{\pgfqpoint{3.875000in}{3.850000in}} %
\pgfusepath{clip}%
\pgfsetbuttcap%
\pgfsetroundjoin%
\pgfsetlinewidth{0.250937pt}%
\definecolor{currentstroke}{rgb}{0.000000,0.000000,0.000000}%
\pgfsetstrokecolor{currentstroke}%
\pgfsetdash{}{0pt}%
\pgfpathmoveto{\pgfqpoint{0.625000in}{1.250812in}}%
\pgfpathlineto{\pgfqpoint{0.631156in}{1.244737in}}%
\pgfpathlineto{\pgfqpoint{0.625000in}{1.238829in}}%
\pgfusepath{stroke}%
\end{pgfscope}%
\begin{pgfscope}%
\pgfpathrectangle{\pgfqpoint{0.625000in}{0.550000in}}{\pgfqpoint{3.875000in}{3.850000in}} %
\pgfusepath{clip}%
\pgfsetbuttcap%
\pgfsetroundjoin%
\pgfsetlinewidth{0.250937pt}%
\definecolor{currentstroke}{rgb}{0.000000,0.000000,0.000000}%
\pgfsetstrokecolor{currentstroke}%
\pgfsetdash{}{0pt}%
\pgfpathmoveto{\pgfqpoint{0.625000in}{1.404890in}}%
\pgfpathlineto{\pgfqpoint{0.631063in}{1.399123in}}%
\pgfpathlineto{\pgfqpoint{0.625000in}{1.393352in}}%
\pgfusepath{stroke}%
\end{pgfscope}%
\begin{pgfscope}%
\pgfpathrectangle{\pgfqpoint{0.625000in}{0.550000in}}{\pgfqpoint{3.875000in}{3.850000in}} %
\pgfusepath{clip}%
\pgfsetbuttcap%
\pgfsetroundjoin%
\pgfsetlinewidth{0.250937pt}%
\definecolor{currentstroke}{rgb}{0.000000,0.000000,0.000000}%
\pgfsetstrokecolor{currentstroke}%
\pgfsetdash{}{0pt}%
\pgfpathmoveto{\pgfqpoint{0.625000in}{1.559293in}}%
\pgfpathlineto{\pgfqpoint{0.631071in}{1.553509in}}%
\pgfpathlineto{\pgfqpoint{0.625000in}{1.547536in}}%
\pgfusepath{stroke}%
\end{pgfscope}%
\begin{pgfscope}%
\pgfpathrectangle{\pgfqpoint{0.625000in}{0.550000in}}{\pgfqpoint{3.875000in}{3.850000in}} %
\pgfusepath{clip}%
\pgfsetbuttcap%
\pgfsetroundjoin%
\pgfsetlinewidth{0.250937pt}%
\definecolor{currentstroke}{rgb}{0.000000,0.000000,0.000000}%
\pgfsetstrokecolor{currentstroke}%
\pgfsetdash{}{0pt}%
\pgfpathmoveto{\pgfqpoint{0.625000in}{1.713651in}}%
\pgfpathlineto{\pgfqpoint{0.632417in}{1.707895in}}%
\pgfpathlineto{\pgfqpoint{0.625000in}{1.702102in}}%
\pgfusepath{stroke}%
\end{pgfscope}%
\begin{pgfscope}%
\pgfpathrectangle{\pgfqpoint{0.625000in}{0.550000in}}{\pgfqpoint{3.875000in}{3.850000in}} %
\pgfusepath{clip}%
\pgfsetbuttcap%
\pgfsetroundjoin%
\pgfsetlinewidth{0.250937pt}%
\definecolor{currentstroke}{rgb}{0.000000,0.000000,0.000000}%
\pgfsetstrokecolor{currentstroke}%
\pgfsetdash{}{0pt}%
\pgfpathmoveto{\pgfqpoint{0.625000in}{1.868048in}}%
\pgfpathlineto{\pgfqpoint{0.630985in}{1.862281in}}%
\pgfpathlineto{\pgfqpoint{0.625000in}{1.856633in}}%
\pgfusepath{stroke}%
\end{pgfscope}%
\begin{pgfscope}%
\pgfpathrectangle{\pgfqpoint{0.625000in}{0.550000in}}{\pgfqpoint{3.875000in}{3.850000in}} %
\pgfusepath{clip}%
\pgfsetbuttcap%
\pgfsetroundjoin%
\pgfsetlinewidth{0.250937pt}%
\definecolor{currentstroke}{rgb}{0.000000,0.000000,0.000000}%
\pgfsetstrokecolor{currentstroke}%
\pgfsetdash{}{0pt}%
\pgfpathmoveto{\pgfqpoint{0.625000in}{2.022596in}}%
\pgfpathlineto{\pgfqpoint{0.630935in}{2.016667in}}%
\pgfpathlineto{\pgfqpoint{0.625000in}{2.010786in}}%
\pgfusepath{stroke}%
\end{pgfscope}%
\begin{pgfscope}%
\pgfpathrectangle{\pgfqpoint{0.625000in}{0.550000in}}{\pgfqpoint{3.875000in}{3.850000in}} %
\pgfusepath{clip}%
\pgfsetbuttcap%
\pgfsetroundjoin%
\pgfsetlinewidth{0.250937pt}%
\definecolor{currentstroke}{rgb}{0.000000,0.000000,0.000000}%
\pgfsetstrokecolor{currentstroke}%
\pgfsetdash{}{0pt}%
\pgfpathmoveto{\pgfqpoint{0.625000in}{2.176699in}}%
\pgfpathlineto{\pgfqpoint{0.630972in}{2.171053in}}%
\pgfpathlineto{\pgfqpoint{0.625000in}{2.165244in}}%
\pgfusepath{stroke}%
\end{pgfscope}%
\begin{pgfscope}%
\pgfpathrectangle{\pgfqpoint{0.625000in}{0.550000in}}{\pgfqpoint{3.875000in}{3.850000in}} %
\pgfusepath{clip}%
\pgfsetbuttcap%
\pgfsetroundjoin%
\pgfsetlinewidth{0.250937pt}%
\definecolor{currentstroke}{rgb}{0.000000,0.000000,0.000000}%
\pgfsetstrokecolor{currentstroke}%
\pgfsetdash{}{0pt}%
\pgfpathmoveto{\pgfqpoint{0.625000in}{2.331031in}}%
\pgfpathlineto{\pgfqpoint{0.630971in}{2.325439in}}%
\pgfpathlineto{\pgfqpoint{0.625000in}{2.319454in}}%
\pgfusepath{stroke}%
\end{pgfscope}%
\begin{pgfscope}%
\pgfpathrectangle{\pgfqpoint{0.625000in}{0.550000in}}{\pgfqpoint{3.875000in}{3.850000in}} %
\pgfusepath{clip}%
\pgfsetbuttcap%
\pgfsetroundjoin%
\pgfsetlinewidth{0.250937pt}%
\definecolor{currentstroke}{rgb}{0.000000,0.000000,0.000000}%
\pgfsetstrokecolor{currentstroke}%
\pgfsetdash{}{0pt}%
\pgfpathmoveto{\pgfqpoint{0.625000in}{2.486370in}}%
\pgfpathlineto{\pgfqpoint{0.627911in}{2.489474in}}%
\pgfpathlineto{\pgfqpoint{0.628647in}{2.499123in}}%
\pgfpathlineto{\pgfqpoint{0.632431in}{2.508772in}}%
\pgfpathlineto{\pgfqpoint{0.634712in}{2.514394in}}%
\pgfpathlineto{\pgfqpoint{0.638657in}{2.518421in}}%
\pgfpathlineto{\pgfqpoint{0.644424in}{2.526219in}}%
\pgfpathlineto{\pgfqpoint{0.647329in}{2.528070in}}%
\pgfpathlineto{\pgfqpoint{0.654135in}{2.534321in}}%
\pgfpathlineto{\pgfqpoint{0.661026in}{2.537719in}}%
\pgfpathlineto{\pgfqpoint{0.663847in}{2.539675in}}%
\pgfpathlineto{\pgfqpoint{0.673559in}{2.543433in}}%
\pgfpathlineto{\pgfqpoint{0.683271in}{2.545431in}}%
\pgfpathlineto{\pgfqpoint{0.692982in}{2.546070in}}%
\pgfpathlineto{\pgfqpoint{0.702694in}{2.545393in}}%
\pgfpathlineto{\pgfqpoint{0.712406in}{2.543281in}}%
\pgfpathlineto{\pgfqpoint{0.722118in}{2.539540in}}%
\pgfpathlineto{\pgfqpoint{0.725667in}{2.537719in}}%
\pgfpathlineto{\pgfqpoint{0.731830in}{2.533880in}}%
\pgfpathlineto{\pgfqpoint{0.739187in}{2.528070in}}%
\pgfpathlineto{\pgfqpoint{0.741541in}{2.525614in}}%
\pgfpathlineto{\pgfqpoint{0.747828in}{2.518421in}}%
\pgfpathlineto{\pgfqpoint{0.751253in}{2.512756in}}%
\pgfpathlineto{\pgfqpoint{0.753652in}{2.508772in}}%
\pgfpathlineto{\pgfqpoint{0.757541in}{2.499123in}}%
\pgfpathlineto{\pgfqpoint{0.759668in}{2.489474in}}%
\pgfpathlineto{\pgfqpoint{0.760344in}{2.479825in}}%
\pgfpathlineto{\pgfqpoint{0.759668in}{2.470175in}}%
\pgfpathlineto{\pgfqpoint{0.757541in}{2.460526in}}%
\pgfpathlineto{\pgfqpoint{0.753652in}{2.450877in}}%
\pgfpathlineto{\pgfqpoint{0.751253in}{2.446893in}}%
\pgfpathlineto{\pgfqpoint{0.747828in}{2.441228in}}%
\pgfpathlineto{\pgfqpoint{0.741541in}{2.434036in}}%
\pgfpathlineto{\pgfqpoint{0.739187in}{2.431579in}}%
\pgfpathlineto{\pgfqpoint{0.731830in}{2.425769in}}%
\pgfpathlineto{\pgfqpoint{0.725667in}{2.421930in}}%
\pgfpathlineto{\pgfqpoint{0.722118in}{2.420109in}}%
\pgfpathlineto{\pgfqpoint{0.712406in}{2.416368in}}%
\pgfpathlineto{\pgfqpoint{0.702694in}{2.414257in}}%
\pgfpathlineto{\pgfqpoint{0.692982in}{2.413579in}}%
\pgfpathlineto{\pgfqpoint{0.683271in}{2.414218in}}%
\pgfpathlineto{\pgfqpoint{0.673559in}{2.416216in}}%
\pgfpathlineto{\pgfqpoint{0.663847in}{2.419974in}}%
\pgfpathlineto{\pgfqpoint{0.661026in}{2.421930in}}%
\pgfpathlineto{\pgfqpoint{0.654135in}{2.425329in}}%
\pgfpathlineto{\pgfqpoint{0.647329in}{2.431579in}}%
\pgfpathlineto{\pgfqpoint{0.644424in}{2.433430in}}%
\pgfpathlineto{\pgfqpoint{0.638657in}{2.441228in}}%
\pgfpathlineto{\pgfqpoint{0.634712in}{2.445255in}}%
\pgfpathlineto{\pgfqpoint{0.632431in}{2.450877in}}%
\pgfpathlineto{\pgfqpoint{0.628656in}{2.460526in}}%
\pgfpathlineto{\pgfqpoint{0.627911in}{2.470175in}}%
\pgfpathlineto{\pgfqpoint{0.625000in}{2.473280in}}%
\pgfusepath{stroke}%
\end{pgfscope}%
\begin{pgfscope}%
\pgfpathrectangle{\pgfqpoint{0.625000in}{0.550000in}}{\pgfqpoint{3.875000in}{3.850000in}} %
\pgfusepath{clip}%
\pgfsetbuttcap%
\pgfsetroundjoin%
\pgfsetlinewidth{0.250937pt}%
\definecolor{currentstroke}{rgb}{0.000000,0.000000,0.000000}%
\pgfsetstrokecolor{currentstroke}%
\pgfsetdash{}{0pt}%
\pgfpathmoveto{\pgfqpoint{0.625000in}{2.640188in}}%
\pgfpathlineto{\pgfqpoint{0.630964in}{2.634211in}}%
\pgfpathlineto{\pgfqpoint{0.625000in}{2.628625in}}%
\pgfusepath{stroke}%
\end{pgfscope}%
\begin{pgfscope}%
\pgfpathrectangle{\pgfqpoint{0.625000in}{0.550000in}}{\pgfqpoint{3.875000in}{3.850000in}} %
\pgfusepath{clip}%
\pgfsetbuttcap%
\pgfsetroundjoin%
\pgfsetlinewidth{0.250937pt}%
\definecolor{currentstroke}{rgb}{0.000000,0.000000,0.000000}%
\pgfsetstrokecolor{currentstroke}%
\pgfsetdash{}{0pt}%
\pgfpathmoveto{\pgfqpoint{0.625000in}{2.794382in}}%
\pgfpathlineto{\pgfqpoint{0.630948in}{2.788596in}}%
\pgfpathlineto{\pgfqpoint{0.625000in}{2.782973in}}%
\pgfusepath{stroke}%
\end{pgfscope}%
\begin{pgfscope}%
\pgfpathrectangle{\pgfqpoint{0.625000in}{0.550000in}}{\pgfqpoint{3.875000in}{3.850000in}} %
\pgfusepath{clip}%
\pgfsetbuttcap%
\pgfsetroundjoin%
\pgfsetlinewidth{0.250937pt}%
\definecolor{currentstroke}{rgb}{0.000000,0.000000,0.000000}%
\pgfsetstrokecolor{currentstroke}%
\pgfsetdash{}{0pt}%
\pgfpathmoveto{\pgfqpoint{0.625000in}{2.948798in}}%
\pgfpathlineto{\pgfqpoint{0.630870in}{2.942982in}}%
\pgfpathlineto{\pgfqpoint{0.625000in}{2.937118in}}%
\pgfusepath{stroke}%
\end{pgfscope}%
\begin{pgfscope}%
\pgfpathrectangle{\pgfqpoint{0.625000in}{0.550000in}}{\pgfqpoint{3.875000in}{3.850000in}} %
\pgfusepath{clip}%
\pgfsetbuttcap%
\pgfsetroundjoin%
\pgfsetlinewidth{0.250937pt}%
\definecolor{currentstroke}{rgb}{0.000000,0.000000,0.000000}%
\pgfsetstrokecolor{currentstroke}%
\pgfsetdash{}{0pt}%
\pgfpathmoveto{\pgfqpoint{0.625000in}{3.102964in}}%
\pgfpathlineto{\pgfqpoint{0.630934in}{3.097368in}}%
\pgfpathlineto{\pgfqpoint{0.625000in}{3.091653in}}%
\pgfusepath{stroke}%
\end{pgfscope}%
\begin{pgfscope}%
\pgfpathrectangle{\pgfqpoint{0.625000in}{0.550000in}}{\pgfqpoint{3.875000in}{3.850000in}} %
\pgfusepath{clip}%
\pgfsetbuttcap%
\pgfsetroundjoin%
\pgfsetlinewidth{0.250937pt}%
\definecolor{currentstroke}{rgb}{0.000000,0.000000,0.000000}%
\pgfsetstrokecolor{currentstroke}%
\pgfsetdash{}{0pt}%
\pgfpathmoveto{\pgfqpoint{0.625000in}{3.257361in}}%
\pgfpathlineto{\pgfqpoint{0.632274in}{3.251754in}}%
\pgfpathlineto{\pgfqpoint{0.625000in}{3.246185in}}%
\pgfusepath{stroke}%
\end{pgfscope}%
\begin{pgfscope}%
\pgfpathrectangle{\pgfqpoint{0.625000in}{0.550000in}}{\pgfqpoint{3.875000in}{3.850000in}} %
\pgfusepath{clip}%
\pgfsetbuttcap%
\pgfsetroundjoin%
\pgfsetlinewidth{0.250937pt}%
\definecolor{currentstroke}{rgb}{0.000000,0.000000,0.000000}%
\pgfsetstrokecolor{currentstroke}%
\pgfsetdash{}{0pt}%
\pgfpathmoveto{\pgfqpoint{0.625000in}{3.411986in}}%
\pgfpathlineto{\pgfqpoint{0.630945in}{3.406140in}}%
\pgfpathlineto{\pgfqpoint{0.625000in}{3.400485in}}%
\pgfusepath{stroke}%
\end{pgfscope}%
\begin{pgfscope}%
\pgfpathrectangle{\pgfqpoint{0.625000in}{0.550000in}}{\pgfqpoint{3.875000in}{3.850000in}} %
\pgfusepath{clip}%
\pgfsetbuttcap%
\pgfsetroundjoin%
\pgfsetlinewidth{0.250937pt}%
\definecolor{currentstroke}{rgb}{0.000000,0.000000,0.000000}%
\pgfsetstrokecolor{currentstroke}%
\pgfsetdash{}{0pt}%
\pgfpathmoveto{\pgfqpoint{0.625000in}{3.566143in}}%
\pgfpathlineto{\pgfqpoint{0.630911in}{3.560526in}}%
\pgfpathlineto{\pgfqpoint{0.625000in}{3.554913in}}%
\pgfusepath{stroke}%
\end{pgfscope}%
\begin{pgfscope}%
\pgfpathrectangle{\pgfqpoint{0.625000in}{0.550000in}}{\pgfqpoint{3.875000in}{3.850000in}} %
\pgfusepath{clip}%
\pgfsetbuttcap%
\pgfsetroundjoin%
\pgfsetlinewidth{0.250937pt}%
\definecolor{currentstroke}{rgb}{0.000000,0.000000,0.000000}%
\pgfsetstrokecolor{currentstroke}%
\pgfsetdash{}{0pt}%
\pgfpathmoveto{\pgfqpoint{0.625000in}{3.720592in}}%
\pgfpathlineto{\pgfqpoint{0.630931in}{3.714912in}}%
\pgfpathlineto{\pgfqpoint{0.625000in}{3.709061in}}%
\pgfusepath{stroke}%
\end{pgfscope}%
\begin{pgfscope}%
\pgfpathrectangle{\pgfqpoint{0.625000in}{0.550000in}}{\pgfqpoint{3.875000in}{3.850000in}} %
\pgfusepath{clip}%
\pgfsetbuttcap%
\pgfsetroundjoin%
\pgfsetlinewidth{0.250937pt}%
\definecolor{currentstroke}{rgb}{0.000000,0.000000,0.000000}%
\pgfsetstrokecolor{currentstroke}%
\pgfsetdash{}{0pt}%
\pgfpathmoveto{\pgfqpoint{0.625000in}{3.874822in}}%
\pgfpathlineto{\pgfqpoint{0.630805in}{3.869298in}}%
\pgfpathlineto{\pgfqpoint{0.625000in}{3.863752in}}%
\pgfusepath{stroke}%
\end{pgfscope}%
\begin{pgfscope}%
\pgfpathrectangle{\pgfqpoint{0.625000in}{0.550000in}}{\pgfqpoint{3.875000in}{3.850000in}} %
\pgfusepath{clip}%
\pgfsetbuttcap%
\pgfsetroundjoin%
\pgfsetlinewidth{0.250937pt}%
\definecolor{currentstroke}{rgb}{0.000000,0.000000,0.000000}%
\pgfsetstrokecolor{currentstroke}%
\pgfsetdash{}{0pt}%
\pgfpathmoveto{\pgfqpoint{0.625000in}{4.029387in}}%
\pgfpathlineto{\pgfqpoint{0.632325in}{4.023684in}}%
\pgfpathlineto{\pgfqpoint{0.625000in}{4.018037in}}%
\pgfusepath{stroke}%
\end{pgfscope}%
\begin{pgfscope}%
\pgfpathrectangle{\pgfqpoint{0.625000in}{0.550000in}}{\pgfqpoint{3.875000in}{3.850000in}} %
\pgfusepath{clip}%
\pgfsetbuttcap%
\pgfsetroundjoin%
\pgfsetlinewidth{0.250937pt}%
\definecolor{currentstroke}{rgb}{0.000000,0.000000,0.000000}%
\pgfsetstrokecolor{currentstroke}%
\pgfsetdash{}{0pt}%
\pgfpathmoveto{\pgfqpoint{0.625000in}{4.183631in}}%
\pgfpathlineto{\pgfqpoint{0.630777in}{4.178070in}}%
\pgfpathlineto{\pgfqpoint{0.625000in}{4.172769in}}%
\pgfusepath{stroke}%
\end{pgfscope}%
\begin{pgfscope}%
\pgfpathrectangle{\pgfqpoint{0.625000in}{0.550000in}}{\pgfqpoint{3.875000in}{3.850000in}} %
\pgfusepath{clip}%
\pgfsetbuttcap%
\pgfsetroundjoin%
\pgfsetlinewidth{0.250937pt}%
\definecolor{currentstroke}{rgb}{0.000000,0.000000,0.000000}%
\pgfsetstrokecolor{currentstroke}%
\pgfsetdash{}{0pt}%
\pgfpathmoveto{\pgfqpoint{0.625000in}{4.338011in}}%
\pgfpathlineto{\pgfqpoint{0.630770in}{4.332456in}}%
\pgfpathlineto{\pgfqpoint{0.625000in}{4.326882in}}%
\pgfusepath{stroke}%
\end{pgfscope}%
\begin{pgfscope}%
\pgfpathrectangle{\pgfqpoint{0.625000in}{0.550000in}}{\pgfqpoint{3.875000in}{3.850000in}} %
\pgfusepath{clip}%
\pgfsetbuttcap%
\pgfsetroundjoin%
\pgfsetlinewidth{0.250937pt}%
\definecolor{currentstroke}{rgb}{0.000000,0.000000,0.000000}%
\pgfsetstrokecolor{currentstroke}%
\pgfsetdash{}{0pt}%
\pgfpathmoveto{\pgfqpoint{0.634712in}{1.186666in}}%
\pgfpathlineto{\pgfqpoint{0.634602in}{1.186842in}}%
\pgfpathlineto{\pgfqpoint{0.632134in}{1.196491in}}%
\pgfpathlineto{\pgfqpoint{0.634712in}{1.200365in}}%
\pgfpathlineto{\pgfqpoint{0.639915in}{1.196491in}}%
\pgfpathlineto{\pgfqpoint{0.635036in}{1.186842in}}%
\pgfpathlineto{\pgfqpoint{0.634712in}{1.186666in}}%
\pgfusepath{stroke}%
\end{pgfscope}%
\begin{pgfscope}%
\pgfpathrectangle{\pgfqpoint{0.625000in}{0.550000in}}{\pgfqpoint{3.875000in}{3.850000in}} %
\pgfusepath{clip}%
\pgfsetbuttcap%
\pgfsetroundjoin%
\pgfsetlinewidth{0.250937pt}%
\definecolor{currentstroke}{rgb}{0.000000,0.000000,0.000000}%
\pgfsetstrokecolor{currentstroke}%
\pgfsetdash{}{0pt}%
\pgfpathmoveto{\pgfqpoint{0.634712in}{3.759285in}}%
\pgfpathlineto{\pgfqpoint{0.632134in}{3.763158in}}%
\pgfpathlineto{\pgfqpoint{0.634602in}{3.772807in}}%
\pgfpathlineto{\pgfqpoint{0.634712in}{3.772983in}}%
\pgfpathlineto{\pgfqpoint{0.635036in}{3.772807in}}%
\pgfpathlineto{\pgfqpoint{0.639915in}{3.763158in}}%
\pgfpathlineto{\pgfqpoint{0.634712in}{3.759285in}}%
\pgfusepath{stroke}%
\end{pgfscope}%
\begin{pgfscope}%
\pgfpathrectangle{\pgfqpoint{0.625000in}{0.550000in}}{\pgfqpoint{3.875000in}{3.850000in}} %
\pgfusepath{clip}%
\pgfsetbuttcap%
\pgfsetroundjoin%
\pgfsetlinewidth{0.250937pt}%
\definecolor{currentstroke}{rgb}{0.000000,0.000000,0.000000}%
\pgfsetstrokecolor{currentstroke}%
\pgfsetdash{}{0pt}%
\pgfpathmoveto{\pgfqpoint{0.625000in}{0.633381in}}%
\pgfpathlineto{\pgfqpoint{0.631382in}{0.627193in}}%
\pgfpathlineto{\pgfqpoint{0.625000in}{0.621023in}}%
\pgfusepath{stroke}%
\end{pgfscope}%
\begin{pgfscope}%
\pgfpathrectangle{\pgfqpoint{0.625000in}{0.550000in}}{\pgfqpoint{3.875000in}{3.850000in}} %
\pgfusepath{clip}%
\pgfsetbuttcap%
\pgfsetroundjoin%
\pgfsetlinewidth{0.250937pt}%
\definecolor{currentstroke}{rgb}{0.000000,0.000000,0.000000}%
\pgfsetstrokecolor{currentstroke}%
\pgfsetdash{}{0pt}%
\pgfpathmoveto{\pgfqpoint{0.625000in}{0.787345in}}%
\pgfpathlineto{\pgfqpoint{0.631238in}{0.781579in}}%
\pgfpathlineto{\pgfqpoint{0.625000in}{0.775556in}}%
\pgfusepath{stroke}%
\end{pgfscope}%
\begin{pgfscope}%
\pgfpathrectangle{\pgfqpoint{0.625000in}{0.550000in}}{\pgfqpoint{3.875000in}{3.850000in}} %
\pgfusepath{clip}%
\pgfsetbuttcap%
\pgfsetroundjoin%
\pgfsetlinewidth{0.250937pt}%
\definecolor{currentstroke}{rgb}{0.000000,0.000000,0.000000}%
\pgfsetstrokecolor{currentstroke}%
\pgfsetdash{}{0pt}%
\pgfpathmoveto{\pgfqpoint{0.625000in}{0.942029in}}%
\pgfpathlineto{\pgfqpoint{0.632676in}{0.935965in}}%
\pgfpathlineto{\pgfqpoint{0.625000in}{0.929846in}}%
\pgfusepath{stroke}%
\end{pgfscope}%
\begin{pgfscope}%
\pgfpathrectangle{\pgfqpoint{0.625000in}{0.550000in}}{\pgfqpoint{3.875000in}{3.850000in}} %
\pgfusepath{clip}%
\pgfsetbuttcap%
\pgfsetroundjoin%
\pgfsetlinewidth{0.250937pt}%
\definecolor{currentstroke}{rgb}{0.000000,0.000000,0.000000}%
\pgfsetstrokecolor{currentstroke}%
\pgfsetdash{}{0pt}%
\pgfpathmoveto{\pgfqpoint{0.625000in}{1.096352in}}%
\pgfpathlineto{\pgfqpoint{0.631258in}{1.090351in}}%
\pgfpathlineto{\pgfqpoint{0.625000in}{1.084372in}}%
\pgfusepath{stroke}%
\end{pgfscope}%
\begin{pgfscope}%
\pgfpathrectangle{\pgfqpoint{0.625000in}{0.550000in}}{\pgfqpoint{3.875000in}{3.850000in}} %
\pgfusepath{clip}%
\pgfsetbuttcap%
\pgfsetroundjoin%
\pgfsetlinewidth{0.250937pt}%
\definecolor{currentstroke}{rgb}{0.000000,0.000000,0.000000}%
\pgfsetstrokecolor{currentstroke}%
\pgfsetdash{}{0pt}%
\pgfpathmoveto{\pgfqpoint{0.625000in}{1.250895in}}%
\pgfpathlineto{\pgfqpoint{0.631240in}{1.244737in}}%
\pgfpathlineto{\pgfqpoint{0.625000in}{1.238748in}}%
\pgfusepath{stroke}%
\end{pgfscope}%
\begin{pgfscope}%
\pgfpathrectangle{\pgfqpoint{0.625000in}{0.550000in}}{\pgfqpoint{3.875000in}{3.850000in}} %
\pgfusepath{clip}%
\pgfsetbuttcap%
\pgfsetroundjoin%
\pgfsetlinewidth{0.250937pt}%
\definecolor{currentstroke}{rgb}{0.000000,0.000000,0.000000}%
\pgfsetstrokecolor{currentstroke}%
\pgfsetdash{}{0pt}%
\pgfpathmoveto{\pgfqpoint{0.625000in}{1.404972in}}%
\pgfpathlineto{\pgfqpoint{0.631148in}{1.399123in}}%
\pgfpathlineto{\pgfqpoint{0.625000in}{1.393271in}}%
\pgfusepath{stroke}%
\end{pgfscope}%
\begin{pgfscope}%
\pgfpathrectangle{\pgfqpoint{0.625000in}{0.550000in}}{\pgfqpoint{3.875000in}{3.850000in}} %
\pgfusepath{clip}%
\pgfsetbuttcap%
\pgfsetroundjoin%
\pgfsetlinewidth{0.250937pt}%
\definecolor{currentstroke}{rgb}{0.000000,0.000000,0.000000}%
\pgfsetstrokecolor{currentstroke}%
\pgfsetdash{}{0pt}%
\pgfpathmoveto{\pgfqpoint{0.625000in}{1.559373in}}%
\pgfpathlineto{\pgfqpoint{0.631155in}{1.553509in}}%
\pgfpathlineto{\pgfqpoint{0.625000in}{1.547454in}}%
\pgfusepath{stroke}%
\end{pgfscope}%
\begin{pgfscope}%
\pgfpathrectangle{\pgfqpoint{0.625000in}{0.550000in}}{\pgfqpoint{3.875000in}{3.850000in}} %
\pgfusepath{clip}%
\pgfsetbuttcap%
\pgfsetroundjoin%
\pgfsetlinewidth{0.250937pt}%
\definecolor{currentstroke}{rgb}{0.000000,0.000000,0.000000}%
\pgfsetstrokecolor{currentstroke}%
\pgfsetdash{}{0pt}%
\pgfpathmoveto{\pgfqpoint{0.625000in}{1.713732in}}%
\pgfpathlineto{\pgfqpoint{0.632521in}{1.707895in}}%
\pgfpathlineto{\pgfqpoint{0.625000in}{1.702021in}}%
\pgfusepath{stroke}%
\end{pgfscope}%
\begin{pgfscope}%
\pgfpathrectangle{\pgfqpoint{0.625000in}{0.550000in}}{\pgfqpoint{3.875000in}{3.850000in}} %
\pgfusepath{clip}%
\pgfsetbuttcap%
\pgfsetroundjoin%
\pgfsetlinewidth{0.250937pt}%
\definecolor{currentstroke}{rgb}{0.000000,0.000000,0.000000}%
\pgfsetstrokecolor{currentstroke}%
\pgfsetdash{}{0pt}%
\pgfpathmoveto{\pgfqpoint{0.625000in}{1.868131in}}%
\pgfpathlineto{\pgfqpoint{0.631071in}{1.862281in}}%
\pgfpathlineto{\pgfqpoint{0.625000in}{1.856552in}}%
\pgfusepath{stroke}%
\end{pgfscope}%
\begin{pgfscope}%
\pgfpathrectangle{\pgfqpoint{0.625000in}{0.550000in}}{\pgfqpoint{3.875000in}{3.850000in}} %
\pgfusepath{clip}%
\pgfsetbuttcap%
\pgfsetroundjoin%
\pgfsetlinewidth{0.250937pt}%
\definecolor{currentstroke}{rgb}{0.000000,0.000000,0.000000}%
\pgfsetstrokecolor{currentstroke}%
\pgfsetdash{}{0pt}%
\pgfpathmoveto{\pgfqpoint{0.625000in}{2.022683in}}%
\pgfpathlineto{\pgfqpoint{0.631023in}{2.016667in}}%
\pgfpathlineto{\pgfqpoint{0.625000in}{2.010699in}}%
\pgfusepath{stroke}%
\end{pgfscope}%
\begin{pgfscope}%
\pgfpathrectangle{\pgfqpoint{0.625000in}{0.550000in}}{\pgfqpoint{3.875000in}{3.850000in}} %
\pgfusepath{clip}%
\pgfsetbuttcap%
\pgfsetroundjoin%
\pgfsetlinewidth{0.250937pt}%
\definecolor{currentstroke}{rgb}{0.000000,0.000000,0.000000}%
\pgfsetstrokecolor{currentstroke}%
\pgfsetdash{}{0pt}%
\pgfpathmoveto{\pgfqpoint{0.625000in}{2.176782in}}%
\pgfpathlineto{\pgfqpoint{0.631059in}{2.171053in}}%
\pgfpathlineto{\pgfqpoint{0.625000in}{2.165160in}}%
\pgfusepath{stroke}%
\end{pgfscope}%
\begin{pgfscope}%
\pgfpathrectangle{\pgfqpoint{0.625000in}{0.550000in}}{\pgfqpoint{3.875000in}{3.850000in}} %
\pgfusepath{clip}%
\pgfsetbuttcap%
\pgfsetroundjoin%
\pgfsetlinewidth{0.250937pt}%
\definecolor{currentstroke}{rgb}{0.000000,0.000000,0.000000}%
\pgfsetstrokecolor{currentstroke}%
\pgfsetdash{}{0pt}%
\pgfpathmoveto{\pgfqpoint{0.625000in}{2.331114in}}%
\pgfpathlineto{\pgfqpoint{0.631059in}{2.325439in}}%
\pgfpathlineto{\pgfqpoint{0.625000in}{2.319365in}}%
\pgfusepath{stroke}%
\end{pgfscope}%
\begin{pgfscope}%
\pgfpathrectangle{\pgfqpoint{0.625000in}{0.550000in}}{\pgfqpoint{3.875000in}{3.850000in}} %
\pgfusepath{clip}%
\pgfsetbuttcap%
\pgfsetroundjoin%
\pgfsetlinewidth{0.250937pt}%
\definecolor{currentstroke}{rgb}{0.000000,0.000000,0.000000}%
\pgfsetstrokecolor{currentstroke}%
\pgfsetdash{}{0pt}%
\pgfpathmoveto{\pgfqpoint{0.625000in}{2.486453in}}%
\pgfpathlineto{\pgfqpoint{0.627833in}{2.489474in}}%
\pgfpathlineto{\pgfqpoint{0.628568in}{2.499123in}}%
\pgfpathlineto{\pgfqpoint{0.632276in}{2.508772in}}%
\pgfpathlineto{\pgfqpoint{0.634712in}{2.514775in}}%
\pgfpathlineto{\pgfqpoint{0.638284in}{2.518421in}}%
\pgfpathlineto{\pgfqpoint{0.644424in}{2.526723in}}%
\pgfpathlineto{\pgfqpoint{0.646537in}{2.528070in}}%
\pgfpathlineto{\pgfqpoint{0.654135in}{2.535048in}}%
\pgfpathlineto{\pgfqpoint{0.659552in}{2.537719in}}%
\pgfpathlineto{\pgfqpoint{0.663847in}{2.540697in}}%
\pgfpathlineto{\pgfqpoint{0.673559in}{2.544523in}}%
\pgfpathlineto{\pgfqpoint{0.683271in}{2.546661in}}%
\pgfpathlineto{\pgfqpoint{0.691597in}{2.547368in}}%
\pgfpathlineto{\pgfqpoint{0.692982in}{2.547527in}}%
\pgfpathlineto{\pgfqpoint{0.696629in}{2.547368in}}%
\pgfpathlineto{\pgfqpoint{0.702694in}{2.547097in}}%
\pgfpathlineto{\pgfqpoint{0.712406in}{2.545322in}}%
\pgfpathlineto{\pgfqpoint{0.722118in}{2.541989in}}%
\pgfpathlineto{\pgfqpoint{0.730442in}{2.537719in}}%
\pgfpathlineto{\pgfqpoint{0.731830in}{2.536855in}}%
\pgfpathlineto{\pgfqpoint{0.741541in}{2.529429in}}%
\pgfpathlineto{\pgfqpoint{0.743076in}{2.528070in}}%
\pgfpathlineto{\pgfqpoint{0.751253in}{2.518437in}}%
\pgfpathlineto{\pgfqpoint{0.751266in}{2.518421in}}%
\pgfpathlineto{\pgfqpoint{0.757075in}{2.508772in}}%
\pgfpathlineto{\pgfqpoint{0.760686in}{2.499123in}}%
\pgfpathlineto{\pgfqpoint{0.760965in}{2.497817in}}%
\pgfpathlineto{\pgfqpoint{0.762920in}{2.489474in}}%
\pgfpathlineto{\pgfqpoint{0.763649in}{2.479825in}}%
\pgfpathlineto{\pgfqpoint{0.762920in}{2.470175in}}%
\pgfpathlineto{\pgfqpoint{0.760965in}{2.461832in}}%
\pgfpathlineto{\pgfqpoint{0.760686in}{2.460526in}}%
\pgfpathlineto{\pgfqpoint{0.757075in}{2.450877in}}%
\pgfpathlineto{\pgfqpoint{0.751266in}{2.441228in}}%
\pgfpathlineto{\pgfqpoint{0.751253in}{2.441212in}}%
\pgfpathlineto{\pgfqpoint{0.743076in}{2.431579in}}%
\pgfpathlineto{\pgfqpoint{0.741541in}{2.430220in}}%
\pgfpathlineto{\pgfqpoint{0.731830in}{2.422794in}}%
\pgfpathlineto{\pgfqpoint{0.730442in}{2.421930in}}%
\pgfpathlineto{\pgfqpoint{0.722118in}{2.417660in}}%
\pgfpathlineto{\pgfqpoint{0.712406in}{2.414327in}}%
\pgfpathlineto{\pgfqpoint{0.702694in}{2.412552in}}%
\pgfpathlineto{\pgfqpoint{0.696629in}{2.412281in}}%
\pgfpathlineto{\pgfqpoint{0.692982in}{2.412122in}}%
\pgfpathlineto{\pgfqpoint{0.691597in}{2.412281in}}%
\pgfpathlineto{\pgfqpoint{0.683271in}{2.412988in}}%
\pgfpathlineto{\pgfqpoint{0.673559in}{2.415126in}}%
\pgfpathlineto{\pgfqpoint{0.663847in}{2.418952in}}%
\pgfpathlineto{\pgfqpoint{0.659552in}{2.421930in}}%
\pgfpathlineto{\pgfqpoint{0.654135in}{2.424601in}}%
\pgfpathlineto{\pgfqpoint{0.646537in}{2.431579in}}%
\pgfpathlineto{\pgfqpoint{0.644424in}{2.432926in}}%
\pgfpathlineto{\pgfqpoint{0.638284in}{2.441228in}}%
\pgfpathlineto{\pgfqpoint{0.634712in}{2.444874in}}%
\pgfpathlineto{\pgfqpoint{0.632276in}{2.450877in}}%
\pgfpathlineto{\pgfqpoint{0.628577in}{2.460526in}}%
\pgfpathlineto{\pgfqpoint{0.627833in}{2.470175in}}%
\pgfpathlineto{\pgfqpoint{0.625000in}{2.473196in}}%
\pgfusepath{stroke}%
\end{pgfscope}%
\begin{pgfscope}%
\pgfpathrectangle{\pgfqpoint{0.625000in}{0.550000in}}{\pgfqpoint{3.875000in}{3.850000in}} %
\pgfusepath{clip}%
\pgfsetbuttcap%
\pgfsetroundjoin%
\pgfsetlinewidth{0.250937pt}%
\definecolor{currentstroke}{rgb}{0.000000,0.000000,0.000000}%
\pgfsetstrokecolor{currentstroke}%
\pgfsetdash{}{0pt}%
\pgfpathmoveto{\pgfqpoint{0.625000in}{2.640277in}}%
\pgfpathlineto{\pgfqpoint{0.631052in}{2.634211in}}%
\pgfpathlineto{\pgfqpoint{0.625000in}{2.628542in}}%
\pgfusepath{stroke}%
\end{pgfscope}%
\begin{pgfscope}%
\pgfpathrectangle{\pgfqpoint{0.625000in}{0.550000in}}{\pgfqpoint{3.875000in}{3.850000in}} %
\pgfusepath{clip}%
\pgfsetbuttcap%
\pgfsetroundjoin%
\pgfsetlinewidth{0.250937pt}%
\definecolor{currentstroke}{rgb}{0.000000,0.000000,0.000000}%
\pgfsetstrokecolor{currentstroke}%
\pgfsetdash{}{0pt}%
\pgfpathmoveto{\pgfqpoint{0.625000in}{2.794467in}}%
\pgfpathlineto{\pgfqpoint{0.631036in}{2.788596in}}%
\pgfpathlineto{\pgfqpoint{0.625000in}{2.782891in}}%
\pgfusepath{stroke}%
\end{pgfscope}%
\begin{pgfscope}%
\pgfpathrectangle{\pgfqpoint{0.625000in}{0.550000in}}{\pgfqpoint{3.875000in}{3.850000in}} %
\pgfusepath{clip}%
\pgfsetbuttcap%
\pgfsetroundjoin%
\pgfsetlinewidth{0.250937pt}%
\definecolor{currentstroke}{rgb}{0.000000,0.000000,0.000000}%
\pgfsetstrokecolor{currentstroke}%
\pgfsetdash{}{0pt}%
\pgfpathmoveto{\pgfqpoint{0.625000in}{2.948886in}}%
\pgfpathlineto{\pgfqpoint{0.630958in}{2.942982in}}%
\pgfpathlineto{\pgfqpoint{0.625000in}{2.937029in}}%
\pgfusepath{stroke}%
\end{pgfscope}%
\begin{pgfscope}%
\pgfpathrectangle{\pgfqpoint{0.625000in}{0.550000in}}{\pgfqpoint{3.875000in}{3.850000in}} %
\pgfusepath{clip}%
\pgfsetbuttcap%
\pgfsetroundjoin%
\pgfsetlinewidth{0.250937pt}%
\definecolor{currentstroke}{rgb}{0.000000,0.000000,0.000000}%
\pgfsetstrokecolor{currentstroke}%
\pgfsetdash{}{0pt}%
\pgfpathmoveto{\pgfqpoint{0.625000in}{3.103046in}}%
\pgfpathlineto{\pgfqpoint{0.631021in}{3.097368in}}%
\pgfpathlineto{\pgfqpoint{0.625000in}{3.091569in}}%
\pgfusepath{stroke}%
\end{pgfscope}%
\begin{pgfscope}%
\pgfpathrectangle{\pgfqpoint{0.625000in}{0.550000in}}{\pgfqpoint{3.875000in}{3.850000in}} %
\pgfusepath{clip}%
\pgfsetbuttcap%
\pgfsetroundjoin%
\pgfsetlinewidth{0.250937pt}%
\definecolor{currentstroke}{rgb}{0.000000,0.000000,0.000000}%
\pgfsetstrokecolor{currentstroke}%
\pgfsetdash{}{0pt}%
\pgfpathmoveto{\pgfqpoint{0.625000in}{3.257446in}}%
\pgfpathlineto{\pgfqpoint{0.632385in}{3.251754in}}%
\pgfpathlineto{\pgfqpoint{0.625000in}{3.246100in}}%
\pgfusepath{stroke}%
\end{pgfscope}%
\begin{pgfscope}%
\pgfpathrectangle{\pgfqpoint{0.625000in}{0.550000in}}{\pgfqpoint{3.875000in}{3.850000in}} %
\pgfusepath{clip}%
\pgfsetbuttcap%
\pgfsetroundjoin%
\pgfsetlinewidth{0.250937pt}%
\definecolor{currentstroke}{rgb}{0.000000,0.000000,0.000000}%
\pgfsetstrokecolor{currentstroke}%
\pgfsetdash{}{0pt}%
\pgfpathmoveto{\pgfqpoint{0.625000in}{3.412072in}}%
\pgfpathlineto{\pgfqpoint{0.631031in}{3.406140in}}%
\pgfpathlineto{\pgfqpoint{0.625000in}{3.400402in}}%
\pgfusepath{stroke}%
\end{pgfscope}%
\begin{pgfscope}%
\pgfpathrectangle{\pgfqpoint{0.625000in}{0.550000in}}{\pgfqpoint{3.875000in}{3.850000in}} %
\pgfusepath{clip}%
\pgfsetbuttcap%
\pgfsetroundjoin%
\pgfsetlinewidth{0.250937pt}%
\definecolor{currentstroke}{rgb}{0.000000,0.000000,0.000000}%
\pgfsetstrokecolor{currentstroke}%
\pgfsetdash{}{0pt}%
\pgfpathmoveto{\pgfqpoint{0.625000in}{3.566228in}}%
\pgfpathlineto{\pgfqpoint{0.631001in}{3.560526in}}%
\pgfpathlineto{\pgfqpoint{0.625000in}{3.554828in}}%
\pgfusepath{stroke}%
\end{pgfscope}%
\begin{pgfscope}%
\pgfpathrectangle{\pgfqpoint{0.625000in}{0.550000in}}{\pgfqpoint{3.875000in}{3.850000in}} %
\pgfusepath{clip}%
\pgfsetbuttcap%
\pgfsetroundjoin%
\pgfsetlinewidth{0.250937pt}%
\definecolor{currentstroke}{rgb}{0.000000,0.000000,0.000000}%
\pgfsetstrokecolor{currentstroke}%
\pgfsetdash{}{0pt}%
\pgfpathmoveto{\pgfqpoint{0.625000in}{3.720678in}}%
\pgfpathlineto{\pgfqpoint{0.631020in}{3.714912in}}%
\pgfpathlineto{\pgfqpoint{0.625000in}{3.708973in}}%
\pgfusepath{stroke}%
\end{pgfscope}%
\begin{pgfscope}%
\pgfpathrectangle{\pgfqpoint{0.625000in}{0.550000in}}{\pgfqpoint{3.875000in}{3.850000in}} %
\pgfusepath{clip}%
\pgfsetbuttcap%
\pgfsetroundjoin%
\pgfsetlinewidth{0.250937pt}%
\definecolor{currentstroke}{rgb}{0.000000,0.000000,0.000000}%
\pgfsetstrokecolor{currentstroke}%
\pgfsetdash{}{0pt}%
\pgfpathmoveto{\pgfqpoint{0.625000in}{3.874909in}}%
\pgfpathlineto{\pgfqpoint{0.630896in}{3.869298in}}%
\pgfpathlineto{\pgfqpoint{0.625000in}{3.863665in}}%
\pgfusepath{stroke}%
\end{pgfscope}%
\begin{pgfscope}%
\pgfpathrectangle{\pgfqpoint{0.625000in}{0.550000in}}{\pgfqpoint{3.875000in}{3.850000in}} %
\pgfusepath{clip}%
\pgfsetbuttcap%
\pgfsetroundjoin%
\pgfsetlinewidth{0.250937pt}%
\definecolor{currentstroke}{rgb}{0.000000,0.000000,0.000000}%
\pgfsetstrokecolor{currentstroke}%
\pgfsetdash{}{0pt}%
\pgfpathmoveto{\pgfqpoint{0.625000in}{4.029472in}}%
\pgfpathlineto{\pgfqpoint{0.632433in}{4.023684in}}%
\pgfpathlineto{\pgfqpoint{0.625000in}{4.017953in}}%
\pgfusepath{stroke}%
\end{pgfscope}%
\begin{pgfscope}%
\pgfpathrectangle{\pgfqpoint{0.625000in}{0.550000in}}{\pgfqpoint{3.875000in}{3.850000in}} %
\pgfusepath{clip}%
\pgfsetbuttcap%
\pgfsetroundjoin%
\pgfsetlinewidth{0.250937pt}%
\definecolor{currentstroke}{rgb}{0.000000,0.000000,0.000000}%
\pgfsetstrokecolor{currentstroke}%
\pgfsetdash{}{0pt}%
\pgfpathmoveto{\pgfqpoint{0.625000in}{4.183720in}}%
\pgfpathlineto{\pgfqpoint{0.630869in}{4.178070in}}%
\pgfpathlineto{\pgfqpoint{0.625000in}{4.172685in}}%
\pgfusepath{stroke}%
\end{pgfscope}%
\begin{pgfscope}%
\pgfpathrectangle{\pgfqpoint{0.625000in}{0.550000in}}{\pgfqpoint{3.875000in}{3.850000in}} %
\pgfusepath{clip}%
\pgfsetbuttcap%
\pgfsetroundjoin%
\pgfsetlinewidth{0.250937pt}%
\definecolor{currentstroke}{rgb}{0.000000,0.000000,0.000000}%
\pgfsetstrokecolor{currentstroke}%
\pgfsetdash{}{0pt}%
\pgfpathmoveto{\pgfqpoint{0.625000in}{4.338098in}}%
\pgfpathlineto{\pgfqpoint{0.630861in}{4.332456in}}%
\pgfpathlineto{\pgfqpoint{0.625000in}{4.326795in}}%
\pgfusepath{stroke}%
\end{pgfscope}%
\begin{pgfscope}%
\pgfpathrectangle{\pgfqpoint{0.625000in}{0.550000in}}{\pgfqpoint{3.875000in}{3.850000in}} %
\pgfusepath{clip}%
\pgfsetbuttcap%
\pgfsetroundjoin%
\pgfsetlinewidth{0.250937pt}%
\definecolor{currentstroke}{rgb}{0.000000,0.000000,0.000000}%
\pgfsetstrokecolor{currentstroke}%
\pgfsetdash{}{0pt}%
\pgfpathmoveto{\pgfqpoint{0.634712in}{1.186332in}}%
\pgfpathlineto{\pgfqpoint{0.634394in}{1.186842in}}%
\pgfpathlineto{\pgfqpoint{0.631959in}{1.196491in}}%
\pgfpathlineto{\pgfqpoint{0.634712in}{1.200627in}}%
\pgfpathlineto{\pgfqpoint{0.640267in}{1.196491in}}%
\pgfpathlineto{\pgfqpoint{0.635651in}{1.186842in}}%
\pgfpathlineto{\pgfqpoint{0.634712in}{1.186332in}}%
\pgfusepath{stroke}%
\end{pgfscope}%
\begin{pgfscope}%
\pgfpathrectangle{\pgfqpoint{0.625000in}{0.550000in}}{\pgfqpoint{3.875000in}{3.850000in}} %
\pgfusepath{clip}%
\pgfsetbuttcap%
\pgfsetroundjoin%
\pgfsetlinewidth{0.250937pt}%
\definecolor{currentstroke}{rgb}{0.000000,0.000000,0.000000}%
\pgfsetstrokecolor{currentstroke}%
\pgfsetdash{}{0pt}%
\pgfpathmoveto{\pgfqpoint{0.634712in}{3.759022in}}%
\pgfpathlineto{\pgfqpoint{0.631959in}{3.763158in}}%
\pgfpathlineto{\pgfqpoint{0.634394in}{3.772807in}}%
\pgfpathlineto{\pgfqpoint{0.634712in}{3.773317in}}%
\pgfpathlineto{\pgfqpoint{0.635651in}{3.772807in}}%
\pgfpathlineto{\pgfqpoint{0.640267in}{3.763158in}}%
\pgfpathlineto{\pgfqpoint{0.634712in}{3.759022in}}%
\pgfusepath{stroke}%
\end{pgfscope}%
\begin{pgfscope}%
\pgfpathrectangle{\pgfqpoint{0.625000in}{0.550000in}}{\pgfqpoint{3.875000in}{3.850000in}} %
\pgfusepath{clip}%
\pgfsetbuttcap%
\pgfsetroundjoin%
\pgfsetlinewidth{0.250937pt}%
\definecolor{currentstroke}{rgb}{0.000000,0.000000,0.000000}%
\pgfsetstrokecolor{currentstroke}%
\pgfsetdash{}{0pt}%
\pgfpathmoveto{\pgfqpoint{0.625000in}{0.633457in}}%
\pgfpathlineto{\pgfqpoint{0.631461in}{0.627193in}}%
\pgfpathlineto{\pgfqpoint{0.625000in}{0.620947in}}%
\pgfusepath{stroke}%
\end{pgfscope}%
\begin{pgfscope}%
\pgfpathrectangle{\pgfqpoint{0.625000in}{0.550000in}}{\pgfqpoint{3.875000in}{3.850000in}} %
\pgfusepath{clip}%
\pgfsetbuttcap%
\pgfsetroundjoin%
\pgfsetlinewidth{0.250937pt}%
\definecolor{currentstroke}{rgb}{0.000000,0.000000,0.000000}%
\pgfsetstrokecolor{currentstroke}%
\pgfsetdash{}{0pt}%
\pgfpathmoveto{\pgfqpoint{0.625000in}{0.787422in}}%
\pgfpathlineto{\pgfqpoint{0.631321in}{0.781579in}}%
\pgfpathlineto{\pgfqpoint{0.625000in}{0.775476in}}%
\pgfusepath{stroke}%
\end{pgfscope}%
\begin{pgfscope}%
\pgfpathrectangle{\pgfqpoint{0.625000in}{0.550000in}}{\pgfqpoint{3.875000in}{3.850000in}} %
\pgfusepath{clip}%
\pgfsetbuttcap%
\pgfsetroundjoin%
\pgfsetlinewidth{0.250937pt}%
\definecolor{currentstroke}{rgb}{0.000000,0.000000,0.000000}%
\pgfsetstrokecolor{currentstroke}%
\pgfsetdash{}{0pt}%
\pgfpathmoveto{\pgfqpoint{0.625000in}{0.942105in}}%
\pgfpathlineto{\pgfqpoint{0.632773in}{0.935965in}}%
\pgfpathlineto{\pgfqpoint{0.625000in}{0.929769in}}%
\pgfusepath{stroke}%
\end{pgfscope}%
\begin{pgfscope}%
\pgfpathrectangle{\pgfqpoint{0.625000in}{0.550000in}}{\pgfqpoint{3.875000in}{3.850000in}} %
\pgfusepath{clip}%
\pgfsetbuttcap%
\pgfsetroundjoin%
\pgfsetlinewidth{0.250937pt}%
\definecolor{currentstroke}{rgb}{0.000000,0.000000,0.000000}%
\pgfsetstrokecolor{currentstroke}%
\pgfsetdash{}{0pt}%
\pgfpathmoveto{\pgfqpoint{0.625000in}{1.096432in}}%
\pgfpathlineto{\pgfqpoint{0.631341in}{1.090351in}}%
\pgfpathlineto{\pgfqpoint{0.625000in}{1.084293in}}%
\pgfusepath{stroke}%
\end{pgfscope}%
\begin{pgfscope}%
\pgfpathrectangle{\pgfqpoint{0.625000in}{0.550000in}}{\pgfqpoint{3.875000in}{3.850000in}} %
\pgfusepath{clip}%
\pgfsetbuttcap%
\pgfsetroundjoin%
\pgfsetlinewidth{0.250937pt}%
\definecolor{currentstroke}{rgb}{0.000000,0.000000,0.000000}%
\pgfsetstrokecolor{currentstroke}%
\pgfsetdash{}{0pt}%
\pgfpathmoveto{\pgfqpoint{0.625000in}{1.250978in}}%
\pgfpathlineto{\pgfqpoint{0.631324in}{1.244737in}}%
\pgfpathlineto{\pgfqpoint{0.625000in}{1.238668in}}%
\pgfusepath{stroke}%
\end{pgfscope}%
\begin{pgfscope}%
\pgfpathrectangle{\pgfqpoint{0.625000in}{0.550000in}}{\pgfqpoint{3.875000in}{3.850000in}} %
\pgfusepath{clip}%
\pgfsetbuttcap%
\pgfsetroundjoin%
\pgfsetlinewidth{0.250937pt}%
\definecolor{currentstroke}{rgb}{0.000000,0.000000,0.000000}%
\pgfsetstrokecolor{currentstroke}%
\pgfsetdash{}{0pt}%
\pgfpathmoveto{\pgfqpoint{0.625000in}{1.405053in}}%
\pgfpathlineto{\pgfqpoint{0.631234in}{1.399123in}}%
\pgfpathlineto{\pgfqpoint{0.625000in}{1.393189in}}%
\pgfusepath{stroke}%
\end{pgfscope}%
\begin{pgfscope}%
\pgfpathrectangle{\pgfqpoint{0.625000in}{0.550000in}}{\pgfqpoint{3.875000in}{3.850000in}} %
\pgfusepath{clip}%
\pgfsetbuttcap%
\pgfsetroundjoin%
\pgfsetlinewidth{0.250937pt}%
\definecolor{currentstroke}{rgb}{0.000000,0.000000,0.000000}%
\pgfsetstrokecolor{currentstroke}%
\pgfsetdash{}{0pt}%
\pgfpathmoveto{\pgfqpoint{0.625000in}{1.559453in}}%
\pgfpathlineto{\pgfqpoint{0.631239in}{1.553509in}}%
\pgfpathlineto{\pgfqpoint{0.625000in}{1.547371in}}%
\pgfusepath{stroke}%
\end{pgfscope}%
\begin{pgfscope}%
\pgfpathrectangle{\pgfqpoint{0.625000in}{0.550000in}}{\pgfqpoint{3.875000in}{3.850000in}} %
\pgfusepath{clip}%
\pgfsetbuttcap%
\pgfsetroundjoin%
\pgfsetlinewidth{0.250937pt}%
\definecolor{currentstroke}{rgb}{0.000000,0.000000,0.000000}%
\pgfsetstrokecolor{currentstroke}%
\pgfsetdash{}{0pt}%
\pgfpathmoveto{\pgfqpoint{0.625000in}{1.713813in}}%
\pgfpathlineto{\pgfqpoint{0.632625in}{1.707895in}}%
\pgfpathlineto{\pgfqpoint{0.625000in}{1.701939in}}%
\pgfusepath{stroke}%
\end{pgfscope}%
\begin{pgfscope}%
\pgfpathrectangle{\pgfqpoint{0.625000in}{0.550000in}}{\pgfqpoint{3.875000in}{3.850000in}} %
\pgfusepath{clip}%
\pgfsetbuttcap%
\pgfsetroundjoin%
\pgfsetlinewidth{0.250937pt}%
\definecolor{currentstroke}{rgb}{0.000000,0.000000,0.000000}%
\pgfsetstrokecolor{currentstroke}%
\pgfsetdash{}{0pt}%
\pgfpathmoveto{\pgfqpoint{0.625000in}{1.868214in}}%
\pgfpathlineto{\pgfqpoint{0.631157in}{1.862281in}}%
\pgfpathlineto{\pgfqpoint{0.625000in}{1.856470in}}%
\pgfusepath{stroke}%
\end{pgfscope}%
\begin{pgfscope}%
\pgfpathrectangle{\pgfqpoint{0.625000in}{0.550000in}}{\pgfqpoint{3.875000in}{3.850000in}} %
\pgfusepath{clip}%
\pgfsetbuttcap%
\pgfsetroundjoin%
\pgfsetlinewidth{0.250937pt}%
\definecolor{currentstroke}{rgb}{0.000000,0.000000,0.000000}%
\pgfsetstrokecolor{currentstroke}%
\pgfsetdash{}{0pt}%
\pgfpathmoveto{\pgfqpoint{0.625000in}{2.022771in}}%
\pgfpathlineto{\pgfqpoint{0.631110in}{2.016667in}}%
\pgfpathlineto{\pgfqpoint{0.625000in}{2.010612in}}%
\pgfusepath{stroke}%
\end{pgfscope}%
\begin{pgfscope}%
\pgfpathrectangle{\pgfqpoint{0.625000in}{0.550000in}}{\pgfqpoint{3.875000in}{3.850000in}} %
\pgfusepath{clip}%
\pgfsetbuttcap%
\pgfsetroundjoin%
\pgfsetlinewidth{0.250937pt}%
\definecolor{currentstroke}{rgb}{0.000000,0.000000,0.000000}%
\pgfsetstrokecolor{currentstroke}%
\pgfsetdash{}{0pt}%
\pgfpathmoveto{\pgfqpoint{0.625000in}{2.176864in}}%
\pgfpathlineto{\pgfqpoint{0.631146in}{2.171053in}}%
\pgfpathlineto{\pgfqpoint{0.625000in}{2.165075in}}%
\pgfusepath{stroke}%
\end{pgfscope}%
\begin{pgfscope}%
\pgfpathrectangle{\pgfqpoint{0.625000in}{0.550000in}}{\pgfqpoint{3.875000in}{3.850000in}} %
\pgfusepath{clip}%
\pgfsetbuttcap%
\pgfsetroundjoin%
\pgfsetlinewidth{0.250937pt}%
\definecolor{currentstroke}{rgb}{0.000000,0.000000,0.000000}%
\pgfsetstrokecolor{currentstroke}%
\pgfsetdash{}{0pt}%
\pgfpathmoveto{\pgfqpoint{0.625000in}{2.331197in}}%
\pgfpathlineto{\pgfqpoint{0.631147in}{2.325439in}}%
\pgfpathlineto{\pgfqpoint{0.625000in}{2.319277in}}%
\pgfusepath{stroke}%
\end{pgfscope}%
\begin{pgfscope}%
\pgfpathrectangle{\pgfqpoint{0.625000in}{0.550000in}}{\pgfqpoint{3.875000in}{3.850000in}} %
\pgfusepath{clip}%
\pgfsetbuttcap%
\pgfsetroundjoin%
\pgfsetlinewidth{0.250937pt}%
\definecolor{currentstroke}{rgb}{0.000000,0.000000,0.000000}%
\pgfsetstrokecolor{currentstroke}%
\pgfsetdash{}{0pt}%
\pgfpathmoveto{\pgfqpoint{0.625000in}{2.486537in}}%
\pgfpathlineto{\pgfqpoint{0.627754in}{2.489474in}}%
\pgfpathlineto{\pgfqpoint{0.628490in}{2.499123in}}%
\pgfpathlineto{\pgfqpoint{0.632122in}{2.508772in}}%
\pgfpathlineto{\pgfqpoint{0.634712in}{2.515156in}}%
\pgfpathlineto{\pgfqpoint{0.637911in}{2.518421in}}%
\pgfpathlineto{\pgfqpoint{0.644424in}{2.527228in}}%
\pgfpathlineto{\pgfqpoint{0.645745in}{2.528070in}}%
\pgfpathlineto{\pgfqpoint{0.654135in}{2.535775in}}%
\pgfpathlineto{\pgfqpoint{0.658078in}{2.537719in}}%
\pgfpathlineto{\pgfqpoint{0.663847in}{2.541719in}}%
\pgfpathlineto{\pgfqpoint{0.673559in}{2.545613in}}%
\pgfpathlineto{\pgfqpoint{0.681242in}{2.547368in}}%
\pgfpathlineto{\pgfqpoint{0.683271in}{2.548001in}}%
\pgfpathlineto{\pgfqpoint{0.692982in}{2.549186in}}%
\pgfpathlineto{\pgfqpoint{0.702694in}{2.548963in}}%
\pgfpathlineto{\pgfqpoint{0.712375in}{2.547368in}}%
\pgfpathlineto{\pgfqpoint{0.712406in}{2.547363in}}%
\pgfpathlineto{\pgfqpoint{0.722118in}{2.544438in}}%
\pgfpathlineto{\pgfqpoint{0.731830in}{2.539802in}}%
\pgfpathlineto{\pgfqpoint{0.735278in}{2.537719in}}%
\pgfpathlineto{\pgfqpoint{0.741541in}{2.533052in}}%
\pgfpathlineto{\pgfqpoint{0.747170in}{2.528070in}}%
\pgfpathlineto{\pgfqpoint{0.751253in}{2.523259in}}%
\pgfpathlineto{\pgfqpoint{0.755115in}{2.518421in}}%
\pgfpathlineto{\pgfqpoint{0.760497in}{2.508772in}}%
\pgfpathlineto{\pgfqpoint{0.760965in}{2.507514in}}%
\pgfpathlineto{\pgfqpoint{0.764255in}{2.499123in}}%
\pgfpathlineto{\pgfqpoint{0.766373in}{2.489474in}}%
\pgfpathlineto{\pgfqpoint{0.767052in}{2.479825in}}%
\pgfpathlineto{\pgfqpoint{0.766373in}{2.470175in}}%
\pgfpathlineto{\pgfqpoint{0.764255in}{2.460526in}}%
\pgfpathlineto{\pgfqpoint{0.760965in}{2.452135in}}%
\pgfpathlineto{\pgfqpoint{0.760497in}{2.450877in}}%
\pgfpathlineto{\pgfqpoint{0.755115in}{2.441228in}}%
\pgfpathlineto{\pgfqpoint{0.751253in}{2.436390in}}%
\pgfpathlineto{\pgfqpoint{0.747170in}{2.431579in}}%
\pgfpathlineto{\pgfqpoint{0.741541in}{2.426597in}}%
\pgfpathlineto{\pgfqpoint{0.735278in}{2.421930in}}%
\pgfpathlineto{\pgfqpoint{0.731830in}{2.419847in}}%
\pgfpathlineto{\pgfqpoint{0.722118in}{2.415211in}}%
\pgfpathlineto{\pgfqpoint{0.712406in}{2.412286in}}%
\pgfpathlineto{\pgfqpoint{0.712375in}{2.412281in}}%
\pgfpathlineto{\pgfqpoint{0.702694in}{2.410686in}}%
\pgfpathlineto{\pgfqpoint{0.692982in}{2.410463in}}%
\pgfpathlineto{\pgfqpoint{0.683271in}{2.411648in}}%
\pgfpathlineto{\pgfqpoint{0.681242in}{2.412281in}}%
\pgfpathlineto{\pgfqpoint{0.673559in}{2.414037in}}%
\pgfpathlineto{\pgfqpoint{0.663847in}{2.417930in}}%
\pgfpathlineto{\pgfqpoint{0.658078in}{2.421930in}}%
\pgfpathlineto{\pgfqpoint{0.654135in}{2.423874in}}%
\pgfpathlineto{\pgfqpoint{0.645745in}{2.431579in}}%
\pgfpathlineto{\pgfqpoint{0.644424in}{2.432421in}}%
\pgfpathlineto{\pgfqpoint{0.637911in}{2.441228in}}%
\pgfpathlineto{\pgfqpoint{0.634712in}{2.444494in}}%
\pgfpathlineto{\pgfqpoint{0.632122in}{2.450877in}}%
\pgfpathlineto{\pgfqpoint{0.628498in}{2.460526in}}%
\pgfpathlineto{\pgfqpoint{0.627754in}{2.470175in}}%
\pgfpathlineto{\pgfqpoint{0.625000in}{2.473112in}}%
\pgfusepath{stroke}%
\end{pgfscope}%
\begin{pgfscope}%
\pgfpathrectangle{\pgfqpoint{0.625000in}{0.550000in}}{\pgfqpoint{3.875000in}{3.850000in}} %
\pgfusepath{clip}%
\pgfsetbuttcap%
\pgfsetroundjoin%
\pgfsetlinewidth{0.250937pt}%
\definecolor{currentstroke}{rgb}{0.000000,0.000000,0.000000}%
\pgfsetstrokecolor{currentstroke}%
\pgfsetdash{}{0pt}%
\pgfpathmoveto{\pgfqpoint{0.625000in}{2.640366in}}%
\pgfpathlineto{\pgfqpoint{0.631141in}{2.634211in}}%
\pgfpathlineto{\pgfqpoint{0.625000in}{2.628459in}}%
\pgfusepath{stroke}%
\end{pgfscope}%
\begin{pgfscope}%
\pgfpathrectangle{\pgfqpoint{0.625000in}{0.550000in}}{\pgfqpoint{3.875000in}{3.850000in}} %
\pgfusepath{clip}%
\pgfsetbuttcap%
\pgfsetroundjoin%
\pgfsetlinewidth{0.250937pt}%
\definecolor{currentstroke}{rgb}{0.000000,0.000000,0.000000}%
\pgfsetstrokecolor{currentstroke}%
\pgfsetdash{}{0pt}%
\pgfpathmoveto{\pgfqpoint{0.625000in}{2.794552in}}%
\pgfpathlineto{\pgfqpoint{0.631123in}{2.788596in}}%
\pgfpathlineto{\pgfqpoint{0.625000in}{2.782808in}}%
\pgfusepath{stroke}%
\end{pgfscope}%
\begin{pgfscope}%
\pgfpathrectangle{\pgfqpoint{0.625000in}{0.550000in}}{\pgfqpoint{3.875000in}{3.850000in}} %
\pgfusepath{clip}%
\pgfsetbuttcap%
\pgfsetroundjoin%
\pgfsetlinewidth{0.250937pt}%
\definecolor{currentstroke}{rgb}{0.000000,0.000000,0.000000}%
\pgfsetstrokecolor{currentstroke}%
\pgfsetdash{}{0pt}%
\pgfpathmoveto{\pgfqpoint{0.625000in}{2.948974in}}%
\pgfpathlineto{\pgfqpoint{0.631047in}{2.942982in}}%
\pgfpathlineto{\pgfqpoint{0.625000in}{2.936940in}}%
\pgfusepath{stroke}%
\end{pgfscope}%
\begin{pgfscope}%
\pgfpathrectangle{\pgfqpoint{0.625000in}{0.550000in}}{\pgfqpoint{3.875000in}{3.850000in}} %
\pgfusepath{clip}%
\pgfsetbuttcap%
\pgfsetroundjoin%
\pgfsetlinewidth{0.250937pt}%
\definecolor{currentstroke}{rgb}{0.000000,0.000000,0.000000}%
\pgfsetstrokecolor{currentstroke}%
\pgfsetdash{}{0pt}%
\pgfpathmoveto{\pgfqpoint{0.625000in}{3.103129in}}%
\pgfpathlineto{\pgfqpoint{0.631109in}{3.097368in}}%
\pgfpathlineto{\pgfqpoint{0.625000in}{3.091485in}}%
\pgfusepath{stroke}%
\end{pgfscope}%
\begin{pgfscope}%
\pgfpathrectangle{\pgfqpoint{0.625000in}{0.550000in}}{\pgfqpoint{3.875000in}{3.850000in}} %
\pgfusepath{clip}%
\pgfsetbuttcap%
\pgfsetroundjoin%
\pgfsetlinewidth{0.250937pt}%
\definecolor{currentstroke}{rgb}{0.000000,0.000000,0.000000}%
\pgfsetstrokecolor{currentstroke}%
\pgfsetdash{}{0pt}%
\pgfpathmoveto{\pgfqpoint{0.625000in}{3.257532in}}%
\pgfpathlineto{\pgfqpoint{0.632495in}{3.251754in}}%
\pgfpathlineto{\pgfqpoint{0.625000in}{3.246015in}}%
\pgfusepath{stroke}%
\end{pgfscope}%
\begin{pgfscope}%
\pgfpathrectangle{\pgfqpoint{0.625000in}{0.550000in}}{\pgfqpoint{3.875000in}{3.850000in}} %
\pgfusepath{clip}%
\pgfsetbuttcap%
\pgfsetroundjoin%
\pgfsetlinewidth{0.250937pt}%
\definecolor{currentstroke}{rgb}{0.000000,0.000000,0.000000}%
\pgfsetstrokecolor{currentstroke}%
\pgfsetdash{}{0pt}%
\pgfpathmoveto{\pgfqpoint{0.625000in}{3.412157in}}%
\pgfpathlineto{\pgfqpoint{0.631118in}{3.406140in}}%
\pgfpathlineto{\pgfqpoint{0.625000in}{3.400319in}}%
\pgfusepath{stroke}%
\end{pgfscope}%
\begin{pgfscope}%
\pgfpathrectangle{\pgfqpoint{0.625000in}{0.550000in}}{\pgfqpoint{3.875000in}{3.850000in}} %
\pgfusepath{clip}%
\pgfsetbuttcap%
\pgfsetroundjoin%
\pgfsetlinewidth{0.250937pt}%
\definecolor{currentstroke}{rgb}{0.000000,0.000000,0.000000}%
\pgfsetstrokecolor{currentstroke}%
\pgfsetdash{}{0pt}%
\pgfpathmoveto{\pgfqpoint{0.625000in}{3.566313in}}%
\pgfpathlineto{\pgfqpoint{0.631090in}{3.560526in}}%
\pgfpathlineto{\pgfqpoint{0.625000in}{3.554743in}}%
\pgfusepath{stroke}%
\end{pgfscope}%
\begin{pgfscope}%
\pgfpathrectangle{\pgfqpoint{0.625000in}{0.550000in}}{\pgfqpoint{3.875000in}{3.850000in}} %
\pgfusepath{clip}%
\pgfsetbuttcap%
\pgfsetroundjoin%
\pgfsetlinewidth{0.250937pt}%
\definecolor{currentstroke}{rgb}{0.000000,0.000000,0.000000}%
\pgfsetstrokecolor{currentstroke}%
\pgfsetdash{}{0pt}%
\pgfpathmoveto{\pgfqpoint{0.625000in}{3.720763in}}%
\pgfpathlineto{\pgfqpoint{0.631110in}{3.714912in}}%
\pgfpathlineto{\pgfqpoint{0.625000in}{3.708885in}}%
\pgfusepath{stroke}%
\end{pgfscope}%
\begin{pgfscope}%
\pgfpathrectangle{\pgfqpoint{0.625000in}{0.550000in}}{\pgfqpoint{3.875000in}{3.850000in}} %
\pgfusepath{clip}%
\pgfsetbuttcap%
\pgfsetroundjoin%
\pgfsetlinewidth{0.250937pt}%
\definecolor{currentstroke}{rgb}{0.000000,0.000000,0.000000}%
\pgfsetstrokecolor{currentstroke}%
\pgfsetdash{}{0pt}%
\pgfpathmoveto{\pgfqpoint{0.625000in}{3.874995in}}%
\pgfpathlineto{\pgfqpoint{0.630987in}{3.869298in}}%
\pgfpathlineto{\pgfqpoint{0.625000in}{3.863578in}}%
\pgfusepath{stroke}%
\end{pgfscope}%
\begin{pgfscope}%
\pgfpathrectangle{\pgfqpoint{0.625000in}{0.550000in}}{\pgfqpoint{3.875000in}{3.850000in}} %
\pgfusepath{clip}%
\pgfsetbuttcap%
\pgfsetroundjoin%
\pgfsetlinewidth{0.250937pt}%
\definecolor{currentstroke}{rgb}{0.000000,0.000000,0.000000}%
\pgfsetstrokecolor{currentstroke}%
\pgfsetdash{}{0pt}%
\pgfpathmoveto{\pgfqpoint{0.625000in}{4.029556in}}%
\pgfpathlineto{\pgfqpoint{0.632542in}{4.023684in}}%
\pgfpathlineto{\pgfqpoint{0.625000in}{4.017869in}}%
\pgfusepath{stroke}%
\end{pgfscope}%
\begin{pgfscope}%
\pgfpathrectangle{\pgfqpoint{0.625000in}{0.550000in}}{\pgfqpoint{3.875000in}{3.850000in}} %
\pgfusepath{clip}%
\pgfsetbuttcap%
\pgfsetroundjoin%
\pgfsetlinewidth{0.250937pt}%
\definecolor{currentstroke}{rgb}{0.000000,0.000000,0.000000}%
\pgfsetstrokecolor{currentstroke}%
\pgfsetdash{}{0pt}%
\pgfpathmoveto{\pgfqpoint{0.625000in}{4.183808in}}%
\pgfpathlineto{\pgfqpoint{0.630961in}{4.178070in}}%
\pgfpathlineto{\pgfqpoint{0.625000in}{4.172601in}}%
\pgfusepath{stroke}%
\end{pgfscope}%
\begin{pgfscope}%
\pgfpathrectangle{\pgfqpoint{0.625000in}{0.550000in}}{\pgfqpoint{3.875000in}{3.850000in}} %
\pgfusepath{clip}%
\pgfsetbuttcap%
\pgfsetroundjoin%
\pgfsetlinewidth{0.250937pt}%
\definecolor{currentstroke}{rgb}{0.000000,0.000000,0.000000}%
\pgfsetstrokecolor{currentstroke}%
\pgfsetdash{}{0pt}%
\pgfpathmoveto{\pgfqpoint{0.625000in}{4.338186in}}%
\pgfpathlineto{\pgfqpoint{0.630952in}{4.332456in}}%
\pgfpathlineto{\pgfqpoint{0.625000in}{4.326707in}}%
\pgfusepath{stroke}%
\end{pgfscope}%
\begin{pgfscope}%
\pgfpathrectangle{\pgfqpoint{0.625000in}{0.550000in}}{\pgfqpoint{3.875000in}{3.850000in}} %
\pgfusepath{clip}%
\pgfsetbuttcap%
\pgfsetroundjoin%
\pgfsetlinewidth{0.250937pt}%
\definecolor{currentstroke}{rgb}{0.000000,0.000000,0.000000}%
\pgfsetstrokecolor{currentstroke}%
\pgfsetdash{}{0pt}%
\pgfpathmoveto{\pgfqpoint{0.634712in}{1.185998in}}%
\pgfpathlineto{\pgfqpoint{0.634187in}{1.186842in}}%
\pgfpathlineto{\pgfqpoint{0.631785in}{1.196491in}}%
\pgfpathlineto{\pgfqpoint{0.634712in}{1.200889in}}%
\pgfpathlineto{\pgfqpoint{0.640620in}{1.196491in}}%
\pgfpathlineto{\pgfqpoint{0.636266in}{1.186842in}}%
\pgfpathlineto{\pgfqpoint{0.634712in}{1.185998in}}%
\pgfusepath{stroke}%
\end{pgfscope}%
\begin{pgfscope}%
\pgfpathrectangle{\pgfqpoint{0.625000in}{0.550000in}}{\pgfqpoint{3.875000in}{3.850000in}} %
\pgfusepath{clip}%
\pgfsetbuttcap%
\pgfsetroundjoin%
\pgfsetlinewidth{0.250937pt}%
\definecolor{currentstroke}{rgb}{0.000000,0.000000,0.000000}%
\pgfsetstrokecolor{currentstroke}%
\pgfsetdash{}{0pt}%
\pgfpathmoveto{\pgfqpoint{0.634712in}{3.758760in}}%
\pgfpathlineto{\pgfqpoint{0.631785in}{3.763158in}}%
\pgfpathlineto{\pgfqpoint{0.634187in}{3.772807in}}%
\pgfpathlineto{\pgfqpoint{0.634712in}{3.773651in}}%
\pgfpathlineto{\pgfqpoint{0.636266in}{3.772807in}}%
\pgfpathlineto{\pgfqpoint{0.640620in}{3.763158in}}%
\pgfpathlineto{\pgfqpoint{0.634712in}{3.758760in}}%
\pgfusepath{stroke}%
\end{pgfscope}%
\begin{pgfscope}%
\pgfpathrectangle{\pgfqpoint{0.625000in}{0.550000in}}{\pgfqpoint{3.875000in}{3.850000in}} %
\pgfusepath{clip}%
\pgfsetbuttcap%
\pgfsetroundjoin%
\pgfsetlinewidth{0.250937pt}%
\definecolor{currentstroke}{rgb}{0.000000,0.000000,0.000000}%
\pgfsetstrokecolor{currentstroke}%
\pgfsetdash{}{0pt}%
\pgfpathmoveto{\pgfqpoint{0.625000in}{0.633533in}}%
\pgfpathlineto{\pgfqpoint{0.631539in}{0.627193in}}%
\pgfpathlineto{\pgfqpoint{0.625000in}{0.620871in}}%
\pgfusepath{stroke}%
\end{pgfscope}%
\begin{pgfscope}%
\pgfpathrectangle{\pgfqpoint{0.625000in}{0.550000in}}{\pgfqpoint{3.875000in}{3.850000in}} %
\pgfusepath{clip}%
\pgfsetbuttcap%
\pgfsetroundjoin%
\pgfsetlinewidth{0.250937pt}%
\definecolor{currentstroke}{rgb}{0.000000,0.000000,0.000000}%
\pgfsetstrokecolor{currentstroke}%
\pgfsetdash{}{0pt}%
\pgfpathmoveto{\pgfqpoint{0.625000in}{0.787499in}}%
\pgfpathlineto{\pgfqpoint{0.631405in}{0.781579in}}%
\pgfpathlineto{\pgfqpoint{0.625000in}{0.775395in}}%
\pgfusepath{stroke}%
\end{pgfscope}%
\begin{pgfscope}%
\pgfpathrectangle{\pgfqpoint{0.625000in}{0.550000in}}{\pgfqpoint{3.875000in}{3.850000in}} %
\pgfusepath{clip}%
\pgfsetbuttcap%
\pgfsetroundjoin%
\pgfsetlinewidth{0.250937pt}%
\definecolor{currentstroke}{rgb}{0.000000,0.000000,0.000000}%
\pgfsetstrokecolor{currentstroke}%
\pgfsetdash{}{0pt}%
\pgfpathmoveto{\pgfqpoint{0.625000in}{0.942182in}}%
\pgfpathlineto{\pgfqpoint{0.632870in}{0.935965in}}%
\pgfpathlineto{\pgfqpoint{0.625000in}{0.929692in}}%
\pgfusepath{stroke}%
\end{pgfscope}%
\begin{pgfscope}%
\pgfpathrectangle{\pgfqpoint{0.625000in}{0.550000in}}{\pgfqpoint{3.875000in}{3.850000in}} %
\pgfusepath{clip}%
\pgfsetbuttcap%
\pgfsetroundjoin%
\pgfsetlinewidth{0.250937pt}%
\definecolor{currentstroke}{rgb}{0.000000,0.000000,0.000000}%
\pgfsetstrokecolor{currentstroke}%
\pgfsetdash{}{0pt}%
\pgfpathmoveto{\pgfqpoint{0.625000in}{1.096511in}}%
\pgfpathlineto{\pgfqpoint{0.631423in}{1.090351in}}%
\pgfpathlineto{\pgfqpoint{0.625000in}{1.084214in}}%
\pgfusepath{stroke}%
\end{pgfscope}%
\begin{pgfscope}%
\pgfpathrectangle{\pgfqpoint{0.625000in}{0.550000in}}{\pgfqpoint{3.875000in}{3.850000in}} %
\pgfusepath{clip}%
\pgfsetbuttcap%
\pgfsetroundjoin%
\pgfsetlinewidth{0.250937pt}%
\definecolor{currentstroke}{rgb}{0.000000,0.000000,0.000000}%
\pgfsetstrokecolor{currentstroke}%
\pgfsetdash{}{0pt}%
\pgfpathmoveto{\pgfqpoint{0.625000in}{1.251061in}}%
\pgfpathlineto{\pgfqpoint{0.631408in}{1.244737in}}%
\pgfpathlineto{\pgfqpoint{0.625000in}{1.238587in}}%
\pgfusepath{stroke}%
\end{pgfscope}%
\begin{pgfscope}%
\pgfpathrectangle{\pgfqpoint{0.625000in}{0.550000in}}{\pgfqpoint{3.875000in}{3.850000in}} %
\pgfusepath{clip}%
\pgfsetbuttcap%
\pgfsetroundjoin%
\pgfsetlinewidth{0.250937pt}%
\definecolor{currentstroke}{rgb}{0.000000,0.000000,0.000000}%
\pgfsetstrokecolor{currentstroke}%
\pgfsetdash{}{0pt}%
\pgfpathmoveto{\pgfqpoint{0.625000in}{1.405134in}}%
\pgfpathlineto{\pgfqpoint{0.631320in}{1.399123in}}%
\pgfpathlineto{\pgfqpoint{0.625000in}{1.393108in}}%
\pgfusepath{stroke}%
\end{pgfscope}%
\begin{pgfscope}%
\pgfpathrectangle{\pgfqpoint{0.625000in}{0.550000in}}{\pgfqpoint{3.875000in}{3.850000in}} %
\pgfusepath{clip}%
\pgfsetbuttcap%
\pgfsetroundjoin%
\pgfsetlinewidth{0.250937pt}%
\definecolor{currentstroke}{rgb}{0.000000,0.000000,0.000000}%
\pgfsetstrokecolor{currentstroke}%
\pgfsetdash{}{0pt}%
\pgfpathmoveto{\pgfqpoint{0.625000in}{1.559533in}}%
\pgfpathlineto{\pgfqpoint{0.631323in}{1.553509in}}%
\pgfpathlineto{\pgfqpoint{0.625000in}{1.547289in}}%
\pgfusepath{stroke}%
\end{pgfscope}%
\begin{pgfscope}%
\pgfpathrectangle{\pgfqpoint{0.625000in}{0.550000in}}{\pgfqpoint{3.875000in}{3.850000in}} %
\pgfusepath{clip}%
\pgfsetbuttcap%
\pgfsetroundjoin%
\pgfsetlinewidth{0.250937pt}%
\definecolor{currentstroke}{rgb}{0.000000,0.000000,0.000000}%
\pgfsetstrokecolor{currentstroke}%
\pgfsetdash{}{0pt}%
\pgfpathmoveto{\pgfqpoint{0.625000in}{1.713894in}}%
\pgfpathlineto{\pgfqpoint{0.632729in}{1.707895in}}%
\pgfpathlineto{\pgfqpoint{0.625000in}{1.701858in}}%
\pgfusepath{stroke}%
\end{pgfscope}%
\begin{pgfscope}%
\pgfpathrectangle{\pgfqpoint{0.625000in}{0.550000in}}{\pgfqpoint{3.875000in}{3.850000in}} %
\pgfusepath{clip}%
\pgfsetbuttcap%
\pgfsetroundjoin%
\pgfsetlinewidth{0.250937pt}%
\definecolor{currentstroke}{rgb}{0.000000,0.000000,0.000000}%
\pgfsetstrokecolor{currentstroke}%
\pgfsetdash{}{0pt}%
\pgfpathmoveto{\pgfqpoint{0.625000in}{1.868297in}}%
\pgfpathlineto{\pgfqpoint{0.631244in}{1.862281in}}%
\pgfpathlineto{\pgfqpoint{0.625000in}{1.856389in}}%
\pgfusepath{stroke}%
\end{pgfscope}%
\begin{pgfscope}%
\pgfpathrectangle{\pgfqpoint{0.625000in}{0.550000in}}{\pgfqpoint{3.875000in}{3.850000in}} %
\pgfusepath{clip}%
\pgfsetbuttcap%
\pgfsetroundjoin%
\pgfsetlinewidth{0.250937pt}%
\definecolor{currentstroke}{rgb}{0.000000,0.000000,0.000000}%
\pgfsetstrokecolor{currentstroke}%
\pgfsetdash{}{0pt}%
\pgfpathmoveto{\pgfqpoint{0.625000in}{2.022858in}}%
\pgfpathlineto{\pgfqpoint{0.631197in}{2.016667in}}%
\pgfpathlineto{\pgfqpoint{0.625000in}{2.010526in}}%
\pgfusepath{stroke}%
\end{pgfscope}%
\begin{pgfscope}%
\pgfpathrectangle{\pgfqpoint{0.625000in}{0.550000in}}{\pgfqpoint{3.875000in}{3.850000in}} %
\pgfusepath{clip}%
\pgfsetbuttcap%
\pgfsetroundjoin%
\pgfsetlinewidth{0.250937pt}%
\definecolor{currentstroke}{rgb}{0.000000,0.000000,0.000000}%
\pgfsetstrokecolor{currentstroke}%
\pgfsetdash{}{0pt}%
\pgfpathmoveto{\pgfqpoint{0.625000in}{2.176946in}}%
\pgfpathlineto{\pgfqpoint{0.631233in}{2.171053in}}%
\pgfpathlineto{\pgfqpoint{0.625000in}{2.164990in}}%
\pgfusepath{stroke}%
\end{pgfscope}%
\begin{pgfscope}%
\pgfpathrectangle{\pgfqpoint{0.625000in}{0.550000in}}{\pgfqpoint{3.875000in}{3.850000in}} %
\pgfusepath{clip}%
\pgfsetbuttcap%
\pgfsetroundjoin%
\pgfsetlinewidth{0.250937pt}%
\definecolor{currentstroke}{rgb}{0.000000,0.000000,0.000000}%
\pgfsetstrokecolor{currentstroke}%
\pgfsetdash{}{0pt}%
\pgfpathmoveto{\pgfqpoint{0.625000in}{2.331280in}}%
\pgfpathlineto{\pgfqpoint{0.631236in}{2.325439in}}%
\pgfpathlineto{\pgfqpoint{0.625000in}{2.319188in}}%
\pgfusepath{stroke}%
\end{pgfscope}%
\begin{pgfscope}%
\pgfpathrectangle{\pgfqpoint{0.625000in}{0.550000in}}{\pgfqpoint{3.875000in}{3.850000in}} %
\pgfusepath{clip}%
\pgfsetbuttcap%
\pgfsetroundjoin%
\pgfsetlinewidth{0.250937pt}%
\definecolor{currentstroke}{rgb}{0.000000,0.000000,0.000000}%
\pgfsetstrokecolor{currentstroke}%
\pgfsetdash{}{0pt}%
\pgfpathmoveto{\pgfqpoint{0.625000in}{2.486621in}}%
\pgfpathlineto{\pgfqpoint{0.627675in}{2.489474in}}%
\pgfpathlineto{\pgfqpoint{0.628411in}{2.499123in}}%
\pgfpathlineto{\pgfqpoint{0.631967in}{2.508772in}}%
\pgfpathlineto{\pgfqpoint{0.634712in}{2.515536in}}%
\pgfpathlineto{\pgfqpoint{0.637538in}{2.518421in}}%
\pgfpathlineto{\pgfqpoint{0.644424in}{2.527732in}}%
\pgfpathlineto{\pgfqpoint{0.644954in}{2.528070in}}%
\pgfpathlineto{\pgfqpoint{0.654135in}{2.536502in}}%
\pgfpathlineto{\pgfqpoint{0.656604in}{2.537719in}}%
\pgfpathlineto{\pgfqpoint{0.663847in}{2.542741in}}%
\pgfpathlineto{\pgfqpoint{0.673559in}{2.546702in}}%
\pgfpathlineto{\pgfqpoint{0.676474in}{2.547368in}}%
\pgfpathlineto{\pgfqpoint{0.683271in}{2.549487in}}%
\pgfpathlineto{\pgfqpoint{0.692982in}{2.550846in}}%
\pgfpathlineto{\pgfqpoint{0.702694in}{2.550859in}}%
\pgfpathlineto{\pgfqpoint{0.712406in}{2.549560in}}%
\pgfpathlineto{\pgfqpoint{0.720521in}{2.547368in}}%
\pgfpathlineto{\pgfqpoint{0.722118in}{2.546888in}}%
\pgfpathlineto{\pgfqpoint{0.731830in}{2.542737in}}%
\pgfpathlineto{\pgfqpoint{0.740139in}{2.537719in}}%
\pgfpathlineto{\pgfqpoint{0.741541in}{2.536675in}}%
\pgfpathlineto{\pgfqpoint{0.751253in}{2.528081in}}%
\pgfpathlineto{\pgfqpoint{0.751264in}{2.528070in}}%
\pgfpathlineto{\pgfqpoint{0.758964in}{2.518421in}}%
\pgfpathlineto{\pgfqpoint{0.760965in}{2.514794in}}%
\pgfpathlineto{\pgfqpoint{0.764318in}{2.508772in}}%
\pgfpathlineto{\pgfqpoint{0.767864in}{2.499123in}}%
\pgfpathlineto{\pgfqpoint{0.769827in}{2.489474in}}%
\pgfpathlineto{\pgfqpoint{0.770455in}{2.479825in}}%
\pgfpathlineto{\pgfqpoint{0.769827in}{2.470175in}}%
\pgfpathlineto{\pgfqpoint{0.767864in}{2.460526in}}%
\pgfpathlineto{\pgfqpoint{0.764318in}{2.450877in}}%
\pgfpathlineto{\pgfqpoint{0.760965in}{2.444855in}}%
\pgfpathlineto{\pgfqpoint{0.758964in}{2.441228in}}%
\pgfpathlineto{\pgfqpoint{0.751264in}{2.431579in}}%
\pgfpathlineto{\pgfqpoint{0.751253in}{2.431568in}}%
\pgfpathlineto{\pgfqpoint{0.741541in}{2.422975in}}%
\pgfpathlineto{\pgfqpoint{0.740139in}{2.421930in}}%
\pgfpathlineto{\pgfqpoint{0.731830in}{2.416912in}}%
\pgfpathlineto{\pgfqpoint{0.722118in}{2.412761in}}%
\pgfpathlineto{\pgfqpoint{0.720521in}{2.412281in}}%
\pgfpathlineto{\pgfqpoint{0.712406in}{2.410089in}}%
\pgfpathlineto{\pgfqpoint{0.702694in}{2.408790in}}%
\pgfpathlineto{\pgfqpoint{0.692982in}{2.408803in}}%
\pgfpathlineto{\pgfqpoint{0.683271in}{2.410162in}}%
\pgfpathlineto{\pgfqpoint{0.676474in}{2.412281in}}%
\pgfpathlineto{\pgfqpoint{0.673559in}{2.412947in}}%
\pgfpathlineto{\pgfqpoint{0.663847in}{2.416908in}}%
\pgfpathlineto{\pgfqpoint{0.656604in}{2.421930in}}%
\pgfpathlineto{\pgfqpoint{0.654135in}{2.423147in}}%
\pgfpathlineto{\pgfqpoint{0.644954in}{2.431579in}}%
\pgfpathlineto{\pgfqpoint{0.644424in}{2.431917in}}%
\pgfpathlineto{\pgfqpoint{0.637538in}{2.441228in}}%
\pgfpathlineto{\pgfqpoint{0.634712in}{2.444113in}}%
\pgfpathlineto{\pgfqpoint{0.631967in}{2.450877in}}%
\pgfpathlineto{\pgfqpoint{0.628419in}{2.460526in}}%
\pgfpathlineto{\pgfqpoint{0.627675in}{2.470175in}}%
\pgfpathlineto{\pgfqpoint{0.625000in}{2.473028in}}%
\pgfusepath{stroke}%
\end{pgfscope}%
\begin{pgfscope}%
\pgfpathrectangle{\pgfqpoint{0.625000in}{0.550000in}}{\pgfqpoint{3.875000in}{3.850000in}} %
\pgfusepath{clip}%
\pgfsetbuttcap%
\pgfsetroundjoin%
\pgfsetlinewidth{0.250937pt}%
\definecolor{currentstroke}{rgb}{0.000000,0.000000,0.000000}%
\pgfsetstrokecolor{currentstroke}%
\pgfsetdash{}{0pt}%
\pgfpathmoveto{\pgfqpoint{0.625000in}{2.640454in}}%
\pgfpathlineto{\pgfqpoint{0.631229in}{2.634211in}}%
\pgfpathlineto{\pgfqpoint{0.625000in}{2.628376in}}%
\pgfusepath{stroke}%
\end{pgfscope}%
\begin{pgfscope}%
\pgfpathrectangle{\pgfqpoint{0.625000in}{0.550000in}}{\pgfqpoint{3.875000in}{3.850000in}} %
\pgfusepath{clip}%
\pgfsetbuttcap%
\pgfsetroundjoin%
\pgfsetlinewidth{0.250937pt}%
\definecolor{currentstroke}{rgb}{0.000000,0.000000,0.000000}%
\pgfsetstrokecolor{currentstroke}%
\pgfsetdash{}{0pt}%
\pgfpathmoveto{\pgfqpoint{0.625000in}{2.794637in}}%
\pgfpathlineto{\pgfqpoint{0.631211in}{2.788596in}}%
\pgfpathlineto{\pgfqpoint{0.625000in}{2.782725in}}%
\pgfusepath{stroke}%
\end{pgfscope}%
\begin{pgfscope}%
\pgfpathrectangle{\pgfqpoint{0.625000in}{0.550000in}}{\pgfqpoint{3.875000in}{3.850000in}} %
\pgfusepath{clip}%
\pgfsetbuttcap%
\pgfsetroundjoin%
\pgfsetlinewidth{0.250937pt}%
\definecolor{currentstroke}{rgb}{0.000000,0.000000,0.000000}%
\pgfsetstrokecolor{currentstroke}%
\pgfsetdash{}{0pt}%
\pgfpathmoveto{\pgfqpoint{0.625000in}{2.949062in}}%
\pgfpathlineto{\pgfqpoint{0.631136in}{2.942982in}}%
\pgfpathlineto{\pgfqpoint{0.625000in}{2.936852in}}%
\pgfusepath{stroke}%
\end{pgfscope}%
\begin{pgfscope}%
\pgfpathrectangle{\pgfqpoint{0.625000in}{0.550000in}}{\pgfqpoint{3.875000in}{3.850000in}} %
\pgfusepath{clip}%
\pgfsetbuttcap%
\pgfsetroundjoin%
\pgfsetlinewidth{0.250937pt}%
\definecolor{currentstroke}{rgb}{0.000000,0.000000,0.000000}%
\pgfsetstrokecolor{currentstroke}%
\pgfsetdash{}{0pt}%
\pgfpathmoveto{\pgfqpoint{0.625000in}{3.103211in}}%
\pgfpathlineto{\pgfqpoint{0.631196in}{3.097368in}}%
\pgfpathlineto{\pgfqpoint{0.625000in}{3.091401in}}%
\pgfusepath{stroke}%
\end{pgfscope}%
\begin{pgfscope}%
\pgfpathrectangle{\pgfqpoint{0.625000in}{0.550000in}}{\pgfqpoint{3.875000in}{3.850000in}} %
\pgfusepath{clip}%
\pgfsetbuttcap%
\pgfsetroundjoin%
\pgfsetlinewidth{0.250937pt}%
\definecolor{currentstroke}{rgb}{0.000000,0.000000,0.000000}%
\pgfsetstrokecolor{currentstroke}%
\pgfsetdash{}{0pt}%
\pgfpathmoveto{\pgfqpoint{0.625000in}{3.257617in}}%
\pgfpathlineto{\pgfqpoint{0.632606in}{3.251754in}}%
\pgfpathlineto{\pgfqpoint{0.625000in}{3.245930in}}%
\pgfusepath{stroke}%
\end{pgfscope}%
\begin{pgfscope}%
\pgfpathrectangle{\pgfqpoint{0.625000in}{0.550000in}}{\pgfqpoint{3.875000in}{3.850000in}} %
\pgfusepath{clip}%
\pgfsetbuttcap%
\pgfsetroundjoin%
\pgfsetlinewidth{0.250937pt}%
\definecolor{currentstroke}{rgb}{0.000000,0.000000,0.000000}%
\pgfsetstrokecolor{currentstroke}%
\pgfsetdash{}{0pt}%
\pgfpathmoveto{\pgfqpoint{0.625000in}{3.412242in}}%
\pgfpathlineto{\pgfqpoint{0.631205in}{3.406140in}}%
\pgfpathlineto{\pgfqpoint{0.625000in}{3.400237in}}%
\pgfusepath{stroke}%
\end{pgfscope}%
\begin{pgfscope}%
\pgfpathrectangle{\pgfqpoint{0.625000in}{0.550000in}}{\pgfqpoint{3.875000in}{3.850000in}} %
\pgfusepath{clip}%
\pgfsetbuttcap%
\pgfsetroundjoin%
\pgfsetlinewidth{0.250937pt}%
\definecolor{currentstroke}{rgb}{0.000000,0.000000,0.000000}%
\pgfsetstrokecolor{currentstroke}%
\pgfsetdash{}{0pt}%
\pgfpathmoveto{\pgfqpoint{0.625000in}{3.566397in}}%
\pgfpathlineto{\pgfqpoint{0.631179in}{3.560526in}}%
\pgfpathlineto{\pgfqpoint{0.625000in}{3.554659in}}%
\pgfusepath{stroke}%
\end{pgfscope}%
\begin{pgfscope}%
\pgfpathrectangle{\pgfqpoint{0.625000in}{0.550000in}}{\pgfqpoint{3.875000in}{3.850000in}} %
\pgfusepath{clip}%
\pgfsetbuttcap%
\pgfsetroundjoin%
\pgfsetlinewidth{0.250937pt}%
\definecolor{currentstroke}{rgb}{0.000000,0.000000,0.000000}%
\pgfsetstrokecolor{currentstroke}%
\pgfsetdash{}{0pt}%
\pgfpathmoveto{\pgfqpoint{0.625000in}{3.720849in}}%
\pgfpathlineto{\pgfqpoint{0.631199in}{3.714912in}}%
\pgfpathlineto{\pgfqpoint{0.625000in}{3.708797in}}%
\pgfusepath{stroke}%
\end{pgfscope}%
\begin{pgfscope}%
\pgfpathrectangle{\pgfqpoint{0.625000in}{0.550000in}}{\pgfqpoint{3.875000in}{3.850000in}} %
\pgfusepath{clip}%
\pgfsetbuttcap%
\pgfsetroundjoin%
\pgfsetlinewidth{0.250937pt}%
\definecolor{currentstroke}{rgb}{0.000000,0.000000,0.000000}%
\pgfsetstrokecolor{currentstroke}%
\pgfsetdash{}{0pt}%
\pgfpathmoveto{\pgfqpoint{0.625000in}{3.875082in}}%
\pgfpathlineto{\pgfqpoint{0.631078in}{3.869298in}}%
\pgfpathlineto{\pgfqpoint{0.625000in}{3.863491in}}%
\pgfusepath{stroke}%
\end{pgfscope}%
\begin{pgfscope}%
\pgfpathrectangle{\pgfqpoint{0.625000in}{0.550000in}}{\pgfqpoint{3.875000in}{3.850000in}} %
\pgfusepath{clip}%
\pgfsetbuttcap%
\pgfsetroundjoin%
\pgfsetlinewidth{0.250937pt}%
\definecolor{currentstroke}{rgb}{0.000000,0.000000,0.000000}%
\pgfsetstrokecolor{currentstroke}%
\pgfsetdash{}{0pt}%
\pgfpathmoveto{\pgfqpoint{0.625000in}{4.029641in}}%
\pgfpathlineto{\pgfqpoint{0.632650in}{4.023684in}}%
\pgfpathlineto{\pgfqpoint{0.625000in}{4.017786in}}%
\pgfusepath{stroke}%
\end{pgfscope}%
\begin{pgfscope}%
\pgfpathrectangle{\pgfqpoint{0.625000in}{0.550000in}}{\pgfqpoint{3.875000in}{3.850000in}} %
\pgfusepath{clip}%
\pgfsetbuttcap%
\pgfsetroundjoin%
\pgfsetlinewidth{0.250937pt}%
\definecolor{currentstroke}{rgb}{0.000000,0.000000,0.000000}%
\pgfsetstrokecolor{currentstroke}%
\pgfsetdash{}{0pt}%
\pgfpathmoveto{\pgfqpoint{0.625000in}{4.183897in}}%
\pgfpathlineto{\pgfqpoint{0.631053in}{4.178070in}}%
\pgfpathlineto{\pgfqpoint{0.625000in}{4.172516in}}%
\pgfusepath{stroke}%
\end{pgfscope}%
\begin{pgfscope}%
\pgfpathrectangle{\pgfqpoint{0.625000in}{0.550000in}}{\pgfqpoint{3.875000in}{3.850000in}} %
\pgfusepath{clip}%
\pgfsetbuttcap%
\pgfsetroundjoin%
\pgfsetlinewidth{0.250937pt}%
\definecolor{currentstroke}{rgb}{0.000000,0.000000,0.000000}%
\pgfsetstrokecolor{currentstroke}%
\pgfsetdash{}{0pt}%
\pgfpathmoveto{\pgfqpoint{0.625000in}{4.338273in}}%
\pgfpathlineto{\pgfqpoint{0.631043in}{4.332456in}}%
\pgfpathlineto{\pgfqpoint{0.625000in}{4.326619in}}%
\pgfusepath{stroke}%
\end{pgfscope}%
\begin{pgfscope}%
\pgfpathrectangle{\pgfqpoint{0.625000in}{0.550000in}}{\pgfqpoint{3.875000in}{3.850000in}} %
\pgfusepath{clip}%
\pgfsetbuttcap%
\pgfsetroundjoin%
\pgfsetlinewidth{0.250937pt}%
\definecolor{currentstroke}{rgb}{0.000000,0.000000,0.000000}%
\pgfsetstrokecolor{currentstroke}%
\pgfsetdash{}{0pt}%
\pgfpathmoveto{\pgfqpoint{0.634712in}{1.185664in}}%
\pgfpathlineto{\pgfqpoint{0.633979in}{1.186842in}}%
\pgfpathlineto{\pgfqpoint{0.631610in}{1.196491in}}%
\pgfpathlineto{\pgfqpoint{0.634712in}{1.201152in}}%
\pgfpathlineto{\pgfqpoint{0.640972in}{1.196491in}}%
\pgfpathlineto{\pgfqpoint{0.636881in}{1.186842in}}%
\pgfpathlineto{\pgfqpoint{0.634712in}{1.185664in}}%
\pgfusepath{stroke}%
\end{pgfscope}%
\begin{pgfscope}%
\pgfpathrectangle{\pgfqpoint{0.625000in}{0.550000in}}{\pgfqpoint{3.875000in}{3.850000in}} %
\pgfusepath{clip}%
\pgfsetbuttcap%
\pgfsetroundjoin%
\pgfsetlinewidth{0.250937pt}%
\definecolor{currentstroke}{rgb}{0.000000,0.000000,0.000000}%
\pgfsetstrokecolor{currentstroke}%
\pgfsetdash{}{0pt}%
\pgfpathmoveto{\pgfqpoint{0.634712in}{3.758498in}}%
\pgfpathlineto{\pgfqpoint{0.631610in}{3.763158in}}%
\pgfpathlineto{\pgfqpoint{0.633979in}{3.772807in}}%
\pgfpathlineto{\pgfqpoint{0.634712in}{3.773985in}}%
\pgfpathlineto{\pgfqpoint{0.636881in}{3.772807in}}%
\pgfpathlineto{\pgfqpoint{0.640972in}{3.763158in}}%
\pgfpathlineto{\pgfqpoint{0.634712in}{3.758498in}}%
\pgfusepath{stroke}%
\end{pgfscope}%
\begin{pgfscope}%
\pgfpathrectangle{\pgfqpoint{0.625000in}{0.550000in}}{\pgfqpoint{3.875000in}{3.850000in}} %
\pgfusepath{clip}%
\pgfsetbuttcap%
\pgfsetroundjoin%
\pgfsetlinewidth{0.250937pt}%
\definecolor{currentstroke}{rgb}{0.000000,0.000000,0.000000}%
\pgfsetstrokecolor{currentstroke}%
\pgfsetdash{}{0pt}%
\pgfpathmoveto{\pgfqpoint{0.625000in}{0.633609in}}%
\pgfpathlineto{\pgfqpoint{0.631618in}{0.627193in}}%
\pgfpathlineto{\pgfqpoint{0.625000in}{0.620795in}}%
\pgfusepath{stroke}%
\end{pgfscope}%
\begin{pgfscope}%
\pgfpathrectangle{\pgfqpoint{0.625000in}{0.550000in}}{\pgfqpoint{3.875000in}{3.850000in}} %
\pgfusepath{clip}%
\pgfsetbuttcap%
\pgfsetroundjoin%
\pgfsetlinewidth{0.250937pt}%
\definecolor{currentstroke}{rgb}{0.000000,0.000000,0.000000}%
\pgfsetstrokecolor{currentstroke}%
\pgfsetdash{}{0pt}%
\pgfpathmoveto{\pgfqpoint{0.625000in}{0.787576in}}%
\pgfpathlineto{\pgfqpoint{0.631488in}{0.781579in}}%
\pgfpathlineto{\pgfqpoint{0.625000in}{0.775315in}}%
\pgfusepath{stroke}%
\end{pgfscope}%
\begin{pgfscope}%
\pgfpathrectangle{\pgfqpoint{0.625000in}{0.550000in}}{\pgfqpoint{3.875000in}{3.850000in}} %
\pgfusepath{clip}%
\pgfsetbuttcap%
\pgfsetroundjoin%
\pgfsetlinewidth{0.250937pt}%
\definecolor{currentstroke}{rgb}{0.000000,0.000000,0.000000}%
\pgfsetstrokecolor{currentstroke}%
\pgfsetdash{}{0pt}%
\pgfpathmoveto{\pgfqpoint{0.625000in}{0.942258in}}%
\pgfpathlineto{\pgfqpoint{0.632967in}{0.935965in}}%
\pgfpathlineto{\pgfqpoint{0.625000in}{0.929615in}}%
\pgfusepath{stroke}%
\end{pgfscope}%
\begin{pgfscope}%
\pgfpathrectangle{\pgfqpoint{0.625000in}{0.550000in}}{\pgfqpoint{3.875000in}{3.850000in}} %
\pgfusepath{clip}%
\pgfsetbuttcap%
\pgfsetroundjoin%
\pgfsetlinewidth{0.250937pt}%
\definecolor{currentstroke}{rgb}{0.000000,0.000000,0.000000}%
\pgfsetstrokecolor{currentstroke}%
\pgfsetdash{}{0pt}%
\pgfpathmoveto{\pgfqpoint{0.625000in}{1.096590in}}%
\pgfpathlineto{\pgfqpoint{0.631506in}{1.090351in}}%
\pgfpathlineto{\pgfqpoint{0.625000in}{1.084135in}}%
\pgfusepath{stroke}%
\end{pgfscope}%
\begin{pgfscope}%
\pgfpathrectangle{\pgfqpoint{0.625000in}{0.550000in}}{\pgfqpoint{3.875000in}{3.850000in}} %
\pgfusepath{clip}%
\pgfsetbuttcap%
\pgfsetroundjoin%
\pgfsetlinewidth{0.250937pt}%
\definecolor{currentstroke}{rgb}{0.000000,0.000000,0.000000}%
\pgfsetstrokecolor{currentstroke}%
\pgfsetdash{}{0pt}%
\pgfpathmoveto{\pgfqpoint{0.625000in}{1.251144in}}%
\pgfpathlineto{\pgfqpoint{0.631491in}{1.244737in}}%
\pgfpathlineto{\pgfqpoint{0.625000in}{1.238507in}}%
\pgfusepath{stroke}%
\end{pgfscope}%
\begin{pgfscope}%
\pgfpathrectangle{\pgfqpoint{0.625000in}{0.550000in}}{\pgfqpoint{3.875000in}{3.850000in}} %
\pgfusepath{clip}%
\pgfsetbuttcap%
\pgfsetroundjoin%
\pgfsetlinewidth{0.250937pt}%
\definecolor{currentstroke}{rgb}{0.000000,0.000000,0.000000}%
\pgfsetstrokecolor{currentstroke}%
\pgfsetdash{}{0pt}%
\pgfpathmoveto{\pgfqpoint{0.625000in}{1.405216in}}%
\pgfpathlineto{\pgfqpoint{0.631405in}{1.399123in}}%
\pgfpathlineto{\pgfqpoint{0.625000in}{1.393026in}}%
\pgfusepath{stroke}%
\end{pgfscope}%
\begin{pgfscope}%
\pgfpathrectangle{\pgfqpoint{0.625000in}{0.550000in}}{\pgfqpoint{3.875000in}{3.850000in}} %
\pgfusepath{clip}%
\pgfsetbuttcap%
\pgfsetroundjoin%
\pgfsetlinewidth{0.250937pt}%
\definecolor{currentstroke}{rgb}{0.000000,0.000000,0.000000}%
\pgfsetstrokecolor{currentstroke}%
\pgfsetdash{}{0pt}%
\pgfpathmoveto{\pgfqpoint{0.625000in}{1.559613in}}%
\pgfpathlineto{\pgfqpoint{0.631407in}{1.553509in}}%
\pgfpathlineto{\pgfqpoint{0.625000in}{1.547206in}}%
\pgfusepath{stroke}%
\end{pgfscope}%
\begin{pgfscope}%
\pgfpathrectangle{\pgfqpoint{0.625000in}{0.550000in}}{\pgfqpoint{3.875000in}{3.850000in}} %
\pgfusepath{clip}%
\pgfsetbuttcap%
\pgfsetroundjoin%
\pgfsetlinewidth{0.250937pt}%
\definecolor{currentstroke}{rgb}{0.000000,0.000000,0.000000}%
\pgfsetstrokecolor{currentstroke}%
\pgfsetdash{}{0pt}%
\pgfpathmoveto{\pgfqpoint{0.625000in}{1.713975in}}%
\pgfpathlineto{\pgfqpoint{0.632834in}{1.707895in}}%
\pgfpathlineto{\pgfqpoint{0.625000in}{1.701777in}}%
\pgfusepath{stroke}%
\end{pgfscope}%
\begin{pgfscope}%
\pgfpathrectangle{\pgfqpoint{0.625000in}{0.550000in}}{\pgfqpoint{3.875000in}{3.850000in}} %
\pgfusepath{clip}%
\pgfsetbuttcap%
\pgfsetroundjoin%
\pgfsetlinewidth{0.250937pt}%
\definecolor{currentstroke}{rgb}{0.000000,0.000000,0.000000}%
\pgfsetstrokecolor{currentstroke}%
\pgfsetdash{}{0pt}%
\pgfpathmoveto{\pgfqpoint{0.625000in}{1.868379in}}%
\pgfpathlineto{\pgfqpoint{0.631330in}{1.862281in}}%
\pgfpathlineto{\pgfqpoint{0.625000in}{1.856308in}}%
\pgfusepath{stroke}%
\end{pgfscope}%
\begin{pgfscope}%
\pgfpathrectangle{\pgfqpoint{0.625000in}{0.550000in}}{\pgfqpoint{3.875000in}{3.850000in}} %
\pgfusepath{clip}%
\pgfsetbuttcap%
\pgfsetroundjoin%
\pgfsetlinewidth{0.250937pt}%
\definecolor{currentstroke}{rgb}{0.000000,0.000000,0.000000}%
\pgfsetstrokecolor{currentstroke}%
\pgfsetdash{}{0pt}%
\pgfpathmoveto{\pgfqpoint{0.625000in}{2.022945in}}%
\pgfpathlineto{\pgfqpoint{0.631285in}{2.016667in}}%
\pgfpathlineto{\pgfqpoint{0.625000in}{2.010439in}}%
\pgfusepath{stroke}%
\end{pgfscope}%
\begin{pgfscope}%
\pgfpathrectangle{\pgfqpoint{0.625000in}{0.550000in}}{\pgfqpoint{3.875000in}{3.850000in}} %
\pgfusepath{clip}%
\pgfsetbuttcap%
\pgfsetroundjoin%
\pgfsetlinewidth{0.250937pt}%
\definecolor{currentstroke}{rgb}{0.000000,0.000000,0.000000}%
\pgfsetstrokecolor{currentstroke}%
\pgfsetdash{}{0pt}%
\pgfpathmoveto{\pgfqpoint{0.625000in}{2.177028in}}%
\pgfpathlineto{\pgfqpoint{0.631320in}{2.171053in}}%
\pgfpathlineto{\pgfqpoint{0.625000in}{2.164906in}}%
\pgfusepath{stroke}%
\end{pgfscope}%
\begin{pgfscope}%
\pgfpathrectangle{\pgfqpoint{0.625000in}{0.550000in}}{\pgfqpoint{3.875000in}{3.850000in}} %
\pgfusepath{clip}%
\pgfsetbuttcap%
\pgfsetroundjoin%
\pgfsetlinewidth{0.250937pt}%
\definecolor{currentstroke}{rgb}{0.000000,0.000000,0.000000}%
\pgfsetstrokecolor{currentstroke}%
\pgfsetdash{}{0pt}%
\pgfpathmoveto{\pgfqpoint{0.625000in}{2.331362in}}%
\pgfpathlineto{\pgfqpoint{0.631324in}{2.325439in}}%
\pgfpathlineto{\pgfqpoint{0.625000in}{2.319100in}}%
\pgfusepath{stroke}%
\end{pgfscope}%
\begin{pgfscope}%
\pgfpathrectangle{\pgfqpoint{0.625000in}{0.550000in}}{\pgfqpoint{3.875000in}{3.850000in}} %
\pgfusepath{clip}%
\pgfsetbuttcap%
\pgfsetroundjoin%
\pgfsetlinewidth{0.250937pt}%
\definecolor{currentstroke}{rgb}{0.000000,0.000000,0.000000}%
\pgfsetstrokecolor{currentstroke}%
\pgfsetdash{}{0pt}%
\pgfpathmoveto{\pgfqpoint{0.625000in}{2.486705in}}%
\pgfpathlineto{\pgfqpoint{0.627597in}{2.489474in}}%
\pgfpathlineto{\pgfqpoint{0.628332in}{2.499123in}}%
\pgfpathlineto{\pgfqpoint{0.631813in}{2.508772in}}%
\pgfpathlineto{\pgfqpoint{0.634712in}{2.515917in}}%
\pgfpathlineto{\pgfqpoint{0.637165in}{2.518421in}}%
\pgfpathlineto{\pgfqpoint{0.644257in}{2.528070in}}%
\pgfpathlineto{\pgfqpoint{0.644424in}{2.528345in}}%
\pgfpathlineto{\pgfqpoint{0.654135in}{2.537229in}}%
\pgfpathlineto{\pgfqpoint{0.655130in}{2.537719in}}%
\pgfpathlineto{\pgfqpoint{0.663847in}{2.543762in}}%
\pgfpathlineto{\pgfqpoint{0.672595in}{2.547368in}}%
\pgfpathlineto{\pgfqpoint{0.673559in}{2.547907in}}%
\pgfpathlineto{\pgfqpoint{0.683271in}{2.550974in}}%
\pgfpathlineto{\pgfqpoint{0.692982in}{2.552505in}}%
\pgfpathlineto{\pgfqpoint{0.702694in}{2.552756in}}%
\pgfpathlineto{\pgfqpoint{0.712406in}{2.551758in}}%
\pgfpathlineto{\pgfqpoint{0.722118in}{2.549431in}}%
\pgfpathlineto{\pgfqpoint{0.727769in}{2.547368in}}%
\pgfpathlineto{\pgfqpoint{0.731830in}{2.545673in}}%
\pgfpathlineto{\pgfqpoint{0.741541in}{2.540215in}}%
\pgfpathlineto{\pgfqpoint{0.745154in}{2.537719in}}%
\pgfpathlineto{\pgfqpoint{0.751253in}{2.532459in}}%
\pgfpathlineto{\pgfqpoint{0.755738in}{2.528070in}}%
\pgfpathlineto{\pgfqpoint{0.760965in}{2.521232in}}%
\pgfpathlineto{\pgfqpoint{0.763030in}{2.518421in}}%
\pgfpathlineto{\pgfqpoint{0.768202in}{2.508772in}}%
\pgfpathlineto{\pgfqpoint{0.770677in}{2.501606in}}%
\pgfpathlineto{\pgfqpoint{0.771584in}{2.499123in}}%
\pgfpathlineto{\pgfqpoint{0.773657in}{2.489474in}}%
\pgfpathlineto{\pgfqpoint{0.774324in}{2.479825in}}%
\pgfpathlineto{\pgfqpoint{0.773657in}{2.470175in}}%
\pgfpathlineto{\pgfqpoint{0.771584in}{2.460526in}}%
\pgfpathlineto{\pgfqpoint{0.770677in}{2.458043in}}%
\pgfpathlineto{\pgfqpoint{0.768202in}{2.450877in}}%
\pgfpathlineto{\pgfqpoint{0.763030in}{2.441228in}}%
\pgfpathlineto{\pgfqpoint{0.760965in}{2.438417in}}%
\pgfpathlineto{\pgfqpoint{0.755738in}{2.431579in}}%
\pgfpathlineto{\pgfqpoint{0.751253in}{2.427190in}}%
\pgfpathlineto{\pgfqpoint{0.745154in}{2.421930in}}%
\pgfpathlineto{\pgfqpoint{0.741541in}{2.419434in}}%
\pgfpathlineto{\pgfqpoint{0.731830in}{2.413976in}}%
\pgfpathlineto{\pgfqpoint{0.727769in}{2.412281in}}%
\pgfpathlineto{\pgfqpoint{0.722118in}{2.410219in}}%
\pgfpathlineto{\pgfqpoint{0.712406in}{2.407891in}}%
\pgfpathlineto{\pgfqpoint{0.702694in}{2.406893in}}%
\pgfpathlineto{\pgfqpoint{0.692982in}{2.407144in}}%
\pgfpathlineto{\pgfqpoint{0.683271in}{2.408675in}}%
\pgfpathlineto{\pgfqpoint{0.673559in}{2.411743in}}%
\pgfpathlineto{\pgfqpoint{0.672595in}{2.412281in}}%
\pgfpathlineto{\pgfqpoint{0.663847in}{2.415887in}}%
\pgfpathlineto{\pgfqpoint{0.655130in}{2.421930in}}%
\pgfpathlineto{\pgfqpoint{0.654135in}{2.422420in}}%
\pgfpathlineto{\pgfqpoint{0.644424in}{2.431304in}}%
\pgfpathlineto{\pgfqpoint{0.644257in}{2.431579in}}%
\pgfpathlineto{\pgfqpoint{0.637165in}{2.441228in}}%
\pgfpathlineto{\pgfqpoint{0.634712in}{2.443732in}}%
\pgfpathlineto{\pgfqpoint{0.631813in}{2.450877in}}%
\pgfpathlineto{\pgfqpoint{0.628340in}{2.460526in}}%
\pgfpathlineto{\pgfqpoint{0.627596in}{2.470175in}}%
\pgfpathlineto{\pgfqpoint{0.625000in}{2.472944in}}%
\pgfusepath{stroke}%
\end{pgfscope}%
\begin{pgfscope}%
\pgfpathrectangle{\pgfqpoint{0.625000in}{0.550000in}}{\pgfqpoint{3.875000in}{3.850000in}} %
\pgfusepath{clip}%
\pgfsetbuttcap%
\pgfsetroundjoin%
\pgfsetlinewidth{0.250937pt}%
\definecolor{currentstroke}{rgb}{0.000000,0.000000,0.000000}%
\pgfsetstrokecolor{currentstroke}%
\pgfsetdash{}{0pt}%
\pgfpathmoveto{\pgfqpoint{0.625000in}{2.640543in}}%
\pgfpathlineto{\pgfqpoint{0.631318in}{2.634211in}}%
\pgfpathlineto{\pgfqpoint{0.625000in}{2.628293in}}%
\pgfusepath{stroke}%
\end{pgfscope}%
\begin{pgfscope}%
\pgfpathrectangle{\pgfqpoint{0.625000in}{0.550000in}}{\pgfqpoint{3.875000in}{3.850000in}} %
\pgfusepath{clip}%
\pgfsetbuttcap%
\pgfsetroundjoin%
\pgfsetlinewidth{0.250937pt}%
\definecolor{currentstroke}{rgb}{0.000000,0.000000,0.000000}%
\pgfsetstrokecolor{currentstroke}%
\pgfsetdash{}{0pt}%
\pgfpathmoveto{\pgfqpoint{0.625000in}{2.794722in}}%
\pgfpathlineto{\pgfqpoint{0.631299in}{2.788596in}}%
\pgfpathlineto{\pgfqpoint{0.625000in}{2.782642in}}%
\pgfusepath{stroke}%
\end{pgfscope}%
\begin{pgfscope}%
\pgfpathrectangle{\pgfqpoint{0.625000in}{0.550000in}}{\pgfqpoint{3.875000in}{3.850000in}} %
\pgfusepath{clip}%
\pgfsetbuttcap%
\pgfsetroundjoin%
\pgfsetlinewidth{0.250937pt}%
\definecolor{currentstroke}{rgb}{0.000000,0.000000,0.000000}%
\pgfsetstrokecolor{currentstroke}%
\pgfsetdash{}{0pt}%
\pgfpathmoveto{\pgfqpoint{0.625000in}{2.949151in}}%
\pgfpathlineto{\pgfqpoint{0.631225in}{2.942982in}}%
\pgfpathlineto{\pgfqpoint{0.625000in}{2.936763in}}%
\pgfusepath{stroke}%
\end{pgfscope}%
\begin{pgfscope}%
\pgfpathrectangle{\pgfqpoint{0.625000in}{0.550000in}}{\pgfqpoint{3.875000in}{3.850000in}} %
\pgfusepath{clip}%
\pgfsetbuttcap%
\pgfsetroundjoin%
\pgfsetlinewidth{0.250937pt}%
\definecolor{currentstroke}{rgb}{0.000000,0.000000,0.000000}%
\pgfsetstrokecolor{currentstroke}%
\pgfsetdash{}{0pt}%
\pgfpathmoveto{\pgfqpoint{0.625000in}{3.103293in}}%
\pgfpathlineto{\pgfqpoint{0.631283in}{3.097368in}}%
\pgfpathlineto{\pgfqpoint{0.625000in}{3.091317in}}%
\pgfusepath{stroke}%
\end{pgfscope}%
\begin{pgfscope}%
\pgfpathrectangle{\pgfqpoint{0.625000in}{0.550000in}}{\pgfqpoint{3.875000in}{3.850000in}} %
\pgfusepath{clip}%
\pgfsetbuttcap%
\pgfsetroundjoin%
\pgfsetlinewidth{0.250937pt}%
\definecolor{currentstroke}{rgb}{0.000000,0.000000,0.000000}%
\pgfsetstrokecolor{currentstroke}%
\pgfsetdash{}{0pt}%
\pgfpathmoveto{\pgfqpoint{0.625000in}{3.257702in}}%
\pgfpathlineto{\pgfqpoint{0.632717in}{3.251754in}}%
\pgfpathlineto{\pgfqpoint{0.625000in}{3.245846in}}%
\pgfusepath{stroke}%
\end{pgfscope}%
\begin{pgfscope}%
\pgfpathrectangle{\pgfqpoint{0.625000in}{0.550000in}}{\pgfqpoint{3.875000in}{3.850000in}} %
\pgfusepath{clip}%
\pgfsetbuttcap%
\pgfsetroundjoin%
\pgfsetlinewidth{0.250937pt}%
\definecolor{currentstroke}{rgb}{0.000000,0.000000,0.000000}%
\pgfsetstrokecolor{currentstroke}%
\pgfsetdash{}{0pt}%
\pgfpathmoveto{\pgfqpoint{0.625000in}{3.412328in}}%
\pgfpathlineto{\pgfqpoint{0.631292in}{3.406140in}}%
\pgfpathlineto{\pgfqpoint{0.625000in}{3.400154in}}%
\pgfusepath{stroke}%
\end{pgfscope}%
\begin{pgfscope}%
\pgfpathrectangle{\pgfqpoint{0.625000in}{0.550000in}}{\pgfqpoint{3.875000in}{3.850000in}} %
\pgfusepath{clip}%
\pgfsetbuttcap%
\pgfsetroundjoin%
\pgfsetlinewidth{0.250937pt}%
\definecolor{currentstroke}{rgb}{0.000000,0.000000,0.000000}%
\pgfsetstrokecolor{currentstroke}%
\pgfsetdash{}{0pt}%
\pgfpathmoveto{\pgfqpoint{0.625000in}{3.566482in}}%
\pgfpathlineto{\pgfqpoint{0.631268in}{3.560526in}}%
\pgfpathlineto{\pgfqpoint{0.625000in}{3.554574in}}%
\pgfusepath{stroke}%
\end{pgfscope}%
\begin{pgfscope}%
\pgfpathrectangle{\pgfqpoint{0.625000in}{0.550000in}}{\pgfqpoint{3.875000in}{3.850000in}} %
\pgfusepath{clip}%
\pgfsetbuttcap%
\pgfsetroundjoin%
\pgfsetlinewidth{0.250937pt}%
\definecolor{currentstroke}{rgb}{0.000000,0.000000,0.000000}%
\pgfsetstrokecolor{currentstroke}%
\pgfsetdash{}{0pt}%
\pgfpathmoveto{\pgfqpoint{0.625000in}{3.720934in}}%
\pgfpathlineto{\pgfqpoint{0.631288in}{3.714912in}}%
\pgfpathlineto{\pgfqpoint{0.625000in}{3.708709in}}%
\pgfusepath{stroke}%
\end{pgfscope}%
\begin{pgfscope}%
\pgfpathrectangle{\pgfqpoint{0.625000in}{0.550000in}}{\pgfqpoint{3.875000in}{3.850000in}} %
\pgfusepath{clip}%
\pgfsetbuttcap%
\pgfsetroundjoin%
\pgfsetlinewidth{0.250937pt}%
\definecolor{currentstroke}{rgb}{0.000000,0.000000,0.000000}%
\pgfsetstrokecolor{currentstroke}%
\pgfsetdash{}{0pt}%
\pgfpathmoveto{\pgfqpoint{0.625000in}{3.875169in}}%
\pgfpathlineto{\pgfqpoint{0.631169in}{3.869298in}}%
\pgfpathlineto{\pgfqpoint{0.625000in}{3.863403in}}%
\pgfusepath{stroke}%
\end{pgfscope}%
\begin{pgfscope}%
\pgfpathrectangle{\pgfqpoint{0.625000in}{0.550000in}}{\pgfqpoint{3.875000in}{3.850000in}} %
\pgfusepath{clip}%
\pgfsetbuttcap%
\pgfsetroundjoin%
\pgfsetlinewidth{0.250937pt}%
\definecolor{currentstroke}{rgb}{0.000000,0.000000,0.000000}%
\pgfsetstrokecolor{currentstroke}%
\pgfsetdash{}{0pt}%
\pgfpathmoveto{\pgfqpoint{0.625000in}{4.029725in}}%
\pgfpathlineto{\pgfqpoint{0.632758in}{4.023684in}}%
\pgfpathlineto{\pgfqpoint{0.625000in}{4.017702in}}%
\pgfusepath{stroke}%
\end{pgfscope}%
\begin{pgfscope}%
\pgfpathrectangle{\pgfqpoint{0.625000in}{0.550000in}}{\pgfqpoint{3.875000in}{3.850000in}} %
\pgfusepath{clip}%
\pgfsetbuttcap%
\pgfsetroundjoin%
\pgfsetlinewidth{0.250937pt}%
\definecolor{currentstroke}{rgb}{0.000000,0.000000,0.000000}%
\pgfsetstrokecolor{currentstroke}%
\pgfsetdash{}{0pt}%
\pgfpathmoveto{\pgfqpoint{0.625000in}{4.183986in}}%
\pgfpathlineto{\pgfqpoint{0.631145in}{4.178070in}}%
\pgfpathlineto{\pgfqpoint{0.625000in}{4.172432in}}%
\pgfusepath{stroke}%
\end{pgfscope}%
\begin{pgfscope}%
\pgfpathrectangle{\pgfqpoint{0.625000in}{0.550000in}}{\pgfqpoint{3.875000in}{3.850000in}} %
\pgfusepath{clip}%
\pgfsetbuttcap%
\pgfsetroundjoin%
\pgfsetlinewidth{0.250937pt}%
\definecolor{currentstroke}{rgb}{0.000000,0.000000,0.000000}%
\pgfsetstrokecolor{currentstroke}%
\pgfsetdash{}{0pt}%
\pgfpathmoveto{\pgfqpoint{0.625000in}{4.338361in}}%
\pgfpathlineto{\pgfqpoint{0.631133in}{4.332456in}}%
\pgfpathlineto{\pgfqpoint{0.625000in}{4.326532in}}%
\pgfusepath{stroke}%
\end{pgfscope}%
\begin{pgfscope}%
\pgfpathrectangle{\pgfqpoint{0.625000in}{0.550000in}}{\pgfqpoint{3.875000in}{3.850000in}} %
\pgfusepath{clip}%
\pgfsetbuttcap%
\pgfsetroundjoin%
\pgfsetlinewidth{0.250937pt}%
\definecolor{currentstroke}{rgb}{0.000000,0.000000,0.000000}%
\pgfsetstrokecolor{currentstroke}%
\pgfsetdash{}{0pt}%
\pgfpathmoveto{\pgfqpoint{0.634712in}{1.185330in}}%
\pgfpathlineto{\pgfqpoint{0.633771in}{1.186842in}}%
\pgfpathlineto{\pgfqpoint{0.631435in}{1.196491in}}%
\pgfpathlineto{\pgfqpoint{0.634712in}{1.201414in}}%
\pgfpathlineto{\pgfqpoint{0.641324in}{1.196491in}}%
\pgfpathlineto{\pgfqpoint{0.637496in}{1.186842in}}%
\pgfpathlineto{\pgfqpoint{0.634712in}{1.185330in}}%
\pgfusepath{stroke}%
\end{pgfscope}%
\begin{pgfscope}%
\pgfpathrectangle{\pgfqpoint{0.625000in}{0.550000in}}{\pgfqpoint{3.875000in}{3.850000in}} %
\pgfusepath{clip}%
\pgfsetbuttcap%
\pgfsetroundjoin%
\pgfsetlinewidth{0.250937pt}%
\definecolor{currentstroke}{rgb}{0.000000,0.000000,0.000000}%
\pgfsetstrokecolor{currentstroke}%
\pgfsetdash{}{0pt}%
\pgfpathmoveto{\pgfqpoint{0.634712in}{3.758235in}}%
\pgfpathlineto{\pgfqpoint{0.631435in}{3.763158in}}%
\pgfpathlineto{\pgfqpoint{0.633771in}{3.772807in}}%
\pgfpathlineto{\pgfqpoint{0.634712in}{3.774319in}}%
\pgfpathlineto{\pgfqpoint{0.637496in}{3.772807in}}%
\pgfpathlineto{\pgfqpoint{0.641324in}{3.763158in}}%
\pgfpathlineto{\pgfqpoint{0.634712in}{3.758235in}}%
\pgfusepath{stroke}%
\end{pgfscope}%
\begin{pgfscope}%
\pgfpathrectangle{\pgfqpoint{0.625000in}{0.550000in}}{\pgfqpoint{3.875000in}{3.850000in}} %
\pgfusepath{clip}%
\pgfsetbuttcap%
\pgfsetroundjoin%
\pgfsetlinewidth{0.250937pt}%
\definecolor{currentstroke}{rgb}{0.000000,0.000000,0.000000}%
\pgfsetstrokecolor{currentstroke}%
\pgfsetdash{}{0pt}%
\pgfpathmoveto{\pgfqpoint{0.625000in}{0.633686in}}%
\pgfpathlineto{\pgfqpoint{0.631696in}{0.627193in}}%
\pgfpathlineto{\pgfqpoint{0.625000in}{0.620719in}}%
\pgfusepath{stroke}%
\end{pgfscope}%
\begin{pgfscope}%
\pgfpathrectangle{\pgfqpoint{0.625000in}{0.550000in}}{\pgfqpoint{3.875000in}{3.850000in}} %
\pgfusepath{clip}%
\pgfsetbuttcap%
\pgfsetroundjoin%
\pgfsetlinewidth{0.250937pt}%
\definecolor{currentstroke}{rgb}{0.000000,0.000000,0.000000}%
\pgfsetstrokecolor{currentstroke}%
\pgfsetdash{}{0pt}%
\pgfpathmoveto{\pgfqpoint{0.625000in}{0.787653in}}%
\pgfpathlineto{\pgfqpoint{0.631571in}{0.781579in}}%
\pgfpathlineto{\pgfqpoint{0.625000in}{0.775234in}}%
\pgfusepath{stroke}%
\end{pgfscope}%
\begin{pgfscope}%
\pgfpathrectangle{\pgfqpoint{0.625000in}{0.550000in}}{\pgfqpoint{3.875000in}{3.850000in}} %
\pgfusepath{clip}%
\pgfsetbuttcap%
\pgfsetroundjoin%
\pgfsetlinewidth{0.250937pt}%
\definecolor{currentstroke}{rgb}{0.000000,0.000000,0.000000}%
\pgfsetstrokecolor{currentstroke}%
\pgfsetdash{}{0pt}%
\pgfpathmoveto{\pgfqpoint{0.625000in}{0.942335in}}%
\pgfpathlineto{\pgfqpoint{0.633064in}{0.935965in}}%
\pgfpathlineto{\pgfqpoint{0.625000in}{0.929538in}}%
\pgfusepath{stroke}%
\end{pgfscope}%
\begin{pgfscope}%
\pgfpathrectangle{\pgfqpoint{0.625000in}{0.550000in}}{\pgfqpoint{3.875000in}{3.850000in}} %
\pgfusepath{clip}%
\pgfsetbuttcap%
\pgfsetroundjoin%
\pgfsetlinewidth{0.250937pt}%
\definecolor{currentstroke}{rgb}{0.000000,0.000000,0.000000}%
\pgfsetstrokecolor{currentstroke}%
\pgfsetdash{}{0pt}%
\pgfpathmoveto{\pgfqpoint{0.625000in}{1.096669in}}%
\pgfpathlineto{\pgfqpoint{0.631588in}{1.090351in}}%
\pgfpathlineto{\pgfqpoint{0.625000in}{1.084056in}}%
\pgfusepath{stroke}%
\end{pgfscope}%
\begin{pgfscope}%
\pgfpathrectangle{\pgfqpoint{0.625000in}{0.550000in}}{\pgfqpoint{3.875000in}{3.850000in}} %
\pgfusepath{clip}%
\pgfsetbuttcap%
\pgfsetroundjoin%
\pgfsetlinewidth{0.250937pt}%
\definecolor{currentstroke}{rgb}{0.000000,0.000000,0.000000}%
\pgfsetstrokecolor{currentstroke}%
\pgfsetdash{}{0pt}%
\pgfpathmoveto{\pgfqpoint{0.625000in}{1.251226in}}%
\pgfpathlineto{\pgfqpoint{0.631575in}{1.244737in}}%
\pgfpathlineto{\pgfqpoint{0.625000in}{1.238426in}}%
\pgfusepath{stroke}%
\end{pgfscope}%
\begin{pgfscope}%
\pgfpathrectangle{\pgfqpoint{0.625000in}{0.550000in}}{\pgfqpoint{3.875000in}{3.850000in}} %
\pgfusepath{clip}%
\pgfsetbuttcap%
\pgfsetroundjoin%
\pgfsetlinewidth{0.250937pt}%
\definecolor{currentstroke}{rgb}{0.000000,0.000000,0.000000}%
\pgfsetstrokecolor{currentstroke}%
\pgfsetdash{}{0pt}%
\pgfpathmoveto{\pgfqpoint{0.625000in}{1.405297in}}%
\pgfpathlineto{\pgfqpoint{0.631491in}{1.399123in}}%
\pgfpathlineto{\pgfqpoint{0.625000in}{1.392945in}}%
\pgfusepath{stroke}%
\end{pgfscope}%
\begin{pgfscope}%
\pgfpathrectangle{\pgfqpoint{0.625000in}{0.550000in}}{\pgfqpoint{3.875000in}{3.850000in}} %
\pgfusepath{clip}%
\pgfsetbuttcap%
\pgfsetroundjoin%
\pgfsetlinewidth{0.250937pt}%
\definecolor{currentstroke}{rgb}{0.000000,0.000000,0.000000}%
\pgfsetstrokecolor{currentstroke}%
\pgfsetdash{}{0pt}%
\pgfpathmoveto{\pgfqpoint{0.625000in}{1.559693in}}%
\pgfpathlineto{\pgfqpoint{0.631491in}{1.553509in}}%
\pgfpathlineto{\pgfqpoint{0.625000in}{1.547123in}}%
\pgfusepath{stroke}%
\end{pgfscope}%
\begin{pgfscope}%
\pgfpathrectangle{\pgfqpoint{0.625000in}{0.550000in}}{\pgfqpoint{3.875000in}{3.850000in}} %
\pgfusepath{clip}%
\pgfsetbuttcap%
\pgfsetroundjoin%
\pgfsetlinewidth{0.250937pt}%
\definecolor{currentstroke}{rgb}{0.000000,0.000000,0.000000}%
\pgfsetstrokecolor{currentstroke}%
\pgfsetdash{}{0pt}%
\pgfpathmoveto{\pgfqpoint{0.625000in}{1.714055in}}%
\pgfpathlineto{\pgfqpoint{0.632938in}{1.707895in}}%
\pgfpathlineto{\pgfqpoint{0.625000in}{1.701695in}}%
\pgfusepath{stroke}%
\end{pgfscope}%
\begin{pgfscope}%
\pgfpathrectangle{\pgfqpoint{0.625000in}{0.550000in}}{\pgfqpoint{3.875000in}{3.850000in}} %
\pgfusepath{clip}%
\pgfsetbuttcap%
\pgfsetroundjoin%
\pgfsetlinewidth{0.250937pt}%
\definecolor{currentstroke}{rgb}{0.000000,0.000000,0.000000}%
\pgfsetstrokecolor{currentstroke}%
\pgfsetdash{}{0pt}%
\pgfpathmoveto{\pgfqpoint{0.625000in}{1.868462in}}%
\pgfpathlineto{\pgfqpoint{0.631416in}{1.862281in}}%
\pgfpathlineto{\pgfqpoint{0.625000in}{1.856227in}}%
\pgfusepath{stroke}%
\end{pgfscope}%
\begin{pgfscope}%
\pgfpathrectangle{\pgfqpoint{0.625000in}{0.550000in}}{\pgfqpoint{3.875000in}{3.850000in}} %
\pgfusepath{clip}%
\pgfsetbuttcap%
\pgfsetroundjoin%
\pgfsetlinewidth{0.250937pt}%
\definecolor{currentstroke}{rgb}{0.000000,0.000000,0.000000}%
\pgfsetstrokecolor{currentstroke}%
\pgfsetdash{}{0pt}%
\pgfpathmoveto{\pgfqpoint{0.625000in}{2.023033in}}%
\pgfpathlineto{\pgfqpoint{0.631372in}{2.016667in}}%
\pgfpathlineto{\pgfqpoint{0.625000in}{2.010353in}}%
\pgfusepath{stroke}%
\end{pgfscope}%
\begin{pgfscope}%
\pgfpathrectangle{\pgfqpoint{0.625000in}{0.550000in}}{\pgfqpoint{3.875000in}{3.850000in}} %
\pgfusepath{clip}%
\pgfsetbuttcap%
\pgfsetroundjoin%
\pgfsetlinewidth{0.250937pt}%
\definecolor{currentstroke}{rgb}{0.000000,0.000000,0.000000}%
\pgfsetstrokecolor{currentstroke}%
\pgfsetdash{}{0pt}%
\pgfpathmoveto{\pgfqpoint{0.625000in}{2.177111in}}%
\pgfpathlineto{\pgfqpoint{0.631407in}{2.171053in}}%
\pgfpathlineto{\pgfqpoint{0.625000in}{2.164821in}}%
\pgfusepath{stroke}%
\end{pgfscope}%
\begin{pgfscope}%
\pgfpathrectangle{\pgfqpoint{0.625000in}{0.550000in}}{\pgfqpoint{3.875000in}{3.850000in}} %
\pgfusepath{clip}%
\pgfsetbuttcap%
\pgfsetroundjoin%
\pgfsetlinewidth{0.250937pt}%
\definecolor{currentstroke}{rgb}{0.000000,0.000000,0.000000}%
\pgfsetstrokecolor{currentstroke}%
\pgfsetdash{}{0pt}%
\pgfpathmoveto{\pgfqpoint{0.625000in}{2.331445in}}%
\pgfpathlineto{\pgfqpoint{0.631412in}{2.325439in}}%
\pgfpathlineto{\pgfqpoint{0.625000in}{2.319011in}}%
\pgfusepath{stroke}%
\end{pgfscope}%
\begin{pgfscope}%
\pgfpathrectangle{\pgfqpoint{0.625000in}{0.550000in}}{\pgfqpoint{3.875000in}{3.850000in}} %
\pgfusepath{clip}%
\pgfsetbuttcap%
\pgfsetroundjoin%
\pgfsetlinewidth{0.250937pt}%
\definecolor{currentstroke}{rgb}{0.000000,0.000000,0.000000}%
\pgfsetstrokecolor{currentstroke}%
\pgfsetdash{}{0pt}%
\pgfpathmoveto{\pgfqpoint{0.625000in}{2.486789in}}%
\pgfpathlineto{\pgfqpoint{0.627518in}{2.489474in}}%
\pgfpathlineto{\pgfqpoint{0.628253in}{2.499123in}}%
\pgfpathlineto{\pgfqpoint{0.631659in}{2.508772in}}%
\pgfpathlineto{\pgfqpoint{0.634712in}{2.516298in}}%
\pgfpathlineto{\pgfqpoint{0.636792in}{2.518421in}}%
\pgfpathlineto{\pgfqpoint{0.643753in}{2.528070in}}%
\pgfpathlineto{\pgfqpoint{0.644424in}{2.529177in}}%
\pgfpathlineto{\pgfqpoint{0.653837in}{2.537719in}}%
\pgfpathlineto{\pgfqpoint{0.654135in}{2.538063in}}%
\pgfpathlineto{\pgfqpoint{0.663847in}{2.544784in}}%
\pgfpathlineto{\pgfqpoint{0.670116in}{2.547368in}}%
\pgfpathlineto{\pgfqpoint{0.673559in}{2.549291in}}%
\pgfpathlineto{\pgfqpoint{0.683271in}{2.552461in}}%
\pgfpathlineto{\pgfqpoint{0.692982in}{2.554164in}}%
\pgfpathlineto{\pgfqpoint{0.702694in}{2.554652in}}%
\pgfpathlineto{\pgfqpoint{0.712406in}{2.553955in}}%
\pgfpathlineto{\pgfqpoint{0.722118in}{2.551996in}}%
\pgfpathlineto{\pgfqpoint{0.731830in}{2.548638in}}%
\pgfpathlineto{\pgfqpoint{0.734635in}{2.547368in}}%
\pgfpathlineto{\pgfqpoint{0.741541in}{2.543721in}}%
\pgfpathlineto{\pgfqpoint{0.750230in}{2.537719in}}%
\pgfpathlineto{\pgfqpoint{0.751253in}{2.536837in}}%
\pgfpathlineto{\pgfqpoint{0.760213in}{2.528070in}}%
\pgfpathlineto{\pgfqpoint{0.760965in}{2.527086in}}%
\pgfpathlineto{\pgfqpoint{0.767332in}{2.518421in}}%
\pgfpathlineto{\pgfqpoint{0.770677in}{2.511843in}}%
\pgfpathlineto{\pgfqpoint{0.772267in}{2.508772in}}%
\pgfpathlineto{\pgfqpoint{0.775693in}{2.499123in}}%
\pgfpathlineto{\pgfqpoint{0.777609in}{2.489474in}}%
\pgfpathlineto{\pgfqpoint{0.778226in}{2.479825in}}%
\pgfpathlineto{\pgfqpoint{0.777609in}{2.470175in}}%
\pgfpathlineto{\pgfqpoint{0.775693in}{2.460526in}}%
\pgfpathlineto{\pgfqpoint{0.772267in}{2.450877in}}%
\pgfpathlineto{\pgfqpoint{0.770677in}{2.447806in}}%
\pgfpathlineto{\pgfqpoint{0.767332in}{2.441228in}}%
\pgfpathlineto{\pgfqpoint{0.760965in}{2.432563in}}%
\pgfpathlineto{\pgfqpoint{0.760213in}{2.431579in}}%
\pgfpathlineto{\pgfqpoint{0.751253in}{2.422813in}}%
\pgfpathlineto{\pgfqpoint{0.750230in}{2.421930in}}%
\pgfpathlineto{\pgfqpoint{0.741541in}{2.415928in}}%
\pgfpathlineto{\pgfqpoint{0.734635in}{2.412281in}}%
\pgfpathlineto{\pgfqpoint{0.731830in}{2.411011in}}%
\pgfpathlineto{\pgfqpoint{0.722118in}{2.407653in}}%
\pgfpathlineto{\pgfqpoint{0.712406in}{2.405694in}}%
\pgfpathlineto{\pgfqpoint{0.702694in}{2.404997in}}%
\pgfpathlineto{\pgfqpoint{0.692982in}{2.405485in}}%
\pgfpathlineto{\pgfqpoint{0.683271in}{2.407188in}}%
\pgfpathlineto{\pgfqpoint{0.673559in}{2.410358in}}%
\pgfpathlineto{\pgfqpoint{0.670116in}{2.412281in}}%
\pgfpathlineto{\pgfqpoint{0.663847in}{2.414865in}}%
\pgfpathlineto{\pgfqpoint{0.654135in}{2.421586in}}%
\pgfpathlineto{\pgfqpoint{0.653837in}{2.421930in}}%
\pgfpathlineto{\pgfqpoint{0.644424in}{2.430472in}}%
\pgfpathlineto{\pgfqpoint{0.643753in}{2.431579in}}%
\pgfpathlineto{\pgfqpoint{0.636792in}{2.441228in}}%
\pgfpathlineto{\pgfqpoint{0.634712in}{2.443351in}}%
\pgfpathlineto{\pgfqpoint{0.631659in}{2.450877in}}%
\pgfpathlineto{\pgfqpoint{0.628261in}{2.460526in}}%
\pgfpathlineto{\pgfqpoint{0.627518in}{2.470175in}}%
\pgfpathlineto{\pgfqpoint{0.625000in}{2.472860in}}%
\pgfusepath{stroke}%
\end{pgfscope}%
\begin{pgfscope}%
\pgfpathrectangle{\pgfqpoint{0.625000in}{0.550000in}}{\pgfqpoint{3.875000in}{3.850000in}} %
\pgfusepath{clip}%
\pgfsetbuttcap%
\pgfsetroundjoin%
\pgfsetlinewidth{0.250937pt}%
\definecolor{currentstroke}{rgb}{0.000000,0.000000,0.000000}%
\pgfsetstrokecolor{currentstroke}%
\pgfsetdash{}{0pt}%
\pgfpathmoveto{\pgfqpoint{0.625000in}{2.640632in}}%
\pgfpathlineto{\pgfqpoint{0.631406in}{2.634211in}}%
\pgfpathlineto{\pgfqpoint{0.625000in}{2.628210in}}%
\pgfusepath{stroke}%
\end{pgfscope}%
\begin{pgfscope}%
\pgfpathrectangle{\pgfqpoint{0.625000in}{0.550000in}}{\pgfqpoint{3.875000in}{3.850000in}} %
\pgfusepath{clip}%
\pgfsetbuttcap%
\pgfsetroundjoin%
\pgfsetlinewidth{0.250937pt}%
\definecolor{currentstroke}{rgb}{0.000000,0.000000,0.000000}%
\pgfsetstrokecolor{currentstroke}%
\pgfsetdash{}{0pt}%
\pgfpathmoveto{\pgfqpoint{0.625000in}{2.794807in}}%
\pgfpathlineto{\pgfqpoint{0.631386in}{2.788596in}}%
\pgfpathlineto{\pgfqpoint{0.625000in}{2.782560in}}%
\pgfusepath{stroke}%
\end{pgfscope}%
\begin{pgfscope}%
\pgfpathrectangle{\pgfqpoint{0.625000in}{0.550000in}}{\pgfqpoint{3.875000in}{3.850000in}} %
\pgfusepath{clip}%
\pgfsetbuttcap%
\pgfsetroundjoin%
\pgfsetlinewidth{0.250937pt}%
\definecolor{currentstroke}{rgb}{0.000000,0.000000,0.000000}%
\pgfsetstrokecolor{currentstroke}%
\pgfsetdash{}{0pt}%
\pgfpathmoveto{\pgfqpoint{0.625000in}{2.949239in}}%
\pgfpathlineto{\pgfqpoint{0.631314in}{2.942982in}}%
\pgfpathlineto{\pgfqpoint{0.625000in}{2.936674in}}%
\pgfusepath{stroke}%
\end{pgfscope}%
\begin{pgfscope}%
\pgfpathrectangle{\pgfqpoint{0.625000in}{0.550000in}}{\pgfqpoint{3.875000in}{3.850000in}} %
\pgfusepath{clip}%
\pgfsetbuttcap%
\pgfsetroundjoin%
\pgfsetlinewidth{0.250937pt}%
\definecolor{currentstroke}{rgb}{0.000000,0.000000,0.000000}%
\pgfsetstrokecolor{currentstroke}%
\pgfsetdash{}{0pt}%
\pgfpathmoveto{\pgfqpoint{0.625000in}{3.103375in}}%
\pgfpathlineto{\pgfqpoint{0.631370in}{3.097368in}}%
\pgfpathlineto{\pgfqpoint{0.625000in}{3.091233in}}%
\pgfusepath{stroke}%
\end{pgfscope}%
\begin{pgfscope}%
\pgfpathrectangle{\pgfqpoint{0.625000in}{0.550000in}}{\pgfqpoint{3.875000in}{3.850000in}} %
\pgfusepath{clip}%
\pgfsetbuttcap%
\pgfsetroundjoin%
\pgfsetlinewidth{0.250937pt}%
\definecolor{currentstroke}{rgb}{0.000000,0.000000,0.000000}%
\pgfsetstrokecolor{currentstroke}%
\pgfsetdash{}{0pt}%
\pgfpathmoveto{\pgfqpoint{0.625000in}{3.257788in}}%
\pgfpathlineto{\pgfqpoint{0.632828in}{3.251754in}}%
\pgfpathlineto{\pgfqpoint{0.625000in}{3.245761in}}%
\pgfusepath{stroke}%
\end{pgfscope}%
\begin{pgfscope}%
\pgfpathrectangle{\pgfqpoint{0.625000in}{0.550000in}}{\pgfqpoint{3.875000in}{3.850000in}} %
\pgfusepath{clip}%
\pgfsetbuttcap%
\pgfsetroundjoin%
\pgfsetlinewidth{0.250937pt}%
\definecolor{currentstroke}{rgb}{0.000000,0.000000,0.000000}%
\pgfsetstrokecolor{currentstroke}%
\pgfsetdash{}{0pt}%
\pgfpathmoveto{\pgfqpoint{0.625000in}{3.412413in}}%
\pgfpathlineto{\pgfqpoint{0.631379in}{3.406140in}}%
\pgfpathlineto{\pgfqpoint{0.625000in}{3.400071in}}%
\pgfusepath{stroke}%
\end{pgfscope}%
\begin{pgfscope}%
\pgfpathrectangle{\pgfqpoint{0.625000in}{0.550000in}}{\pgfqpoint{3.875000in}{3.850000in}} %
\pgfusepath{clip}%
\pgfsetbuttcap%
\pgfsetroundjoin%
\pgfsetlinewidth{0.250937pt}%
\definecolor{currentstroke}{rgb}{0.000000,0.000000,0.000000}%
\pgfsetstrokecolor{currentstroke}%
\pgfsetdash{}{0pt}%
\pgfpathmoveto{\pgfqpoint{0.625000in}{3.566567in}}%
\pgfpathlineto{\pgfqpoint{0.631357in}{3.560526in}}%
\pgfpathlineto{\pgfqpoint{0.625000in}{3.554489in}}%
\pgfusepath{stroke}%
\end{pgfscope}%
\begin{pgfscope}%
\pgfpathrectangle{\pgfqpoint{0.625000in}{0.550000in}}{\pgfqpoint{3.875000in}{3.850000in}} %
\pgfusepath{clip}%
\pgfsetbuttcap%
\pgfsetroundjoin%
\pgfsetlinewidth{0.250937pt}%
\definecolor{currentstroke}{rgb}{0.000000,0.000000,0.000000}%
\pgfsetstrokecolor{currentstroke}%
\pgfsetdash{}{0pt}%
\pgfpathmoveto{\pgfqpoint{0.625000in}{3.721019in}}%
\pgfpathlineto{\pgfqpoint{0.631377in}{3.714912in}}%
\pgfpathlineto{\pgfqpoint{0.625000in}{3.708621in}}%
\pgfusepath{stroke}%
\end{pgfscope}%
\begin{pgfscope}%
\pgfpathrectangle{\pgfqpoint{0.625000in}{0.550000in}}{\pgfqpoint{3.875000in}{3.850000in}} %
\pgfusepath{clip}%
\pgfsetbuttcap%
\pgfsetroundjoin%
\pgfsetlinewidth{0.250937pt}%
\definecolor{currentstroke}{rgb}{0.000000,0.000000,0.000000}%
\pgfsetstrokecolor{currentstroke}%
\pgfsetdash{}{0pt}%
\pgfpathmoveto{\pgfqpoint{0.625000in}{3.875256in}}%
\pgfpathlineto{\pgfqpoint{0.631261in}{3.869298in}}%
\pgfpathlineto{\pgfqpoint{0.625000in}{3.863316in}}%
\pgfusepath{stroke}%
\end{pgfscope}%
\begin{pgfscope}%
\pgfpathrectangle{\pgfqpoint{0.625000in}{0.550000in}}{\pgfqpoint{3.875000in}{3.850000in}} %
\pgfusepath{clip}%
\pgfsetbuttcap%
\pgfsetroundjoin%
\pgfsetlinewidth{0.250937pt}%
\definecolor{currentstroke}{rgb}{0.000000,0.000000,0.000000}%
\pgfsetstrokecolor{currentstroke}%
\pgfsetdash{}{0pt}%
\pgfpathmoveto{\pgfqpoint{0.625000in}{4.029810in}}%
\pgfpathlineto{\pgfqpoint{0.632867in}{4.023684in}}%
\pgfpathlineto{\pgfqpoint{0.625000in}{4.017619in}}%
\pgfusepath{stroke}%
\end{pgfscope}%
\begin{pgfscope}%
\pgfpathrectangle{\pgfqpoint{0.625000in}{0.550000in}}{\pgfqpoint{3.875000in}{3.850000in}} %
\pgfusepath{clip}%
\pgfsetbuttcap%
\pgfsetroundjoin%
\pgfsetlinewidth{0.250937pt}%
\definecolor{currentstroke}{rgb}{0.000000,0.000000,0.000000}%
\pgfsetstrokecolor{currentstroke}%
\pgfsetdash{}{0pt}%
\pgfpathmoveto{\pgfqpoint{0.625000in}{4.184074in}}%
\pgfpathlineto{\pgfqpoint{0.631237in}{4.178070in}}%
\pgfpathlineto{\pgfqpoint{0.625000in}{4.172347in}}%
\pgfusepath{stroke}%
\end{pgfscope}%
\begin{pgfscope}%
\pgfpathrectangle{\pgfqpoint{0.625000in}{0.550000in}}{\pgfqpoint{3.875000in}{3.850000in}} %
\pgfusepath{clip}%
\pgfsetbuttcap%
\pgfsetroundjoin%
\pgfsetlinewidth{0.250937pt}%
\definecolor{currentstroke}{rgb}{0.000000,0.000000,0.000000}%
\pgfsetstrokecolor{currentstroke}%
\pgfsetdash{}{0pt}%
\pgfpathmoveto{\pgfqpoint{0.625000in}{4.338448in}}%
\pgfpathlineto{\pgfqpoint{0.631224in}{4.332456in}}%
\pgfpathlineto{\pgfqpoint{0.625000in}{4.326444in}}%
\pgfusepath{stroke}%
\end{pgfscope}%
\begin{pgfscope}%
\pgfpathrectangle{\pgfqpoint{0.625000in}{0.550000in}}{\pgfqpoint{3.875000in}{3.850000in}} %
\pgfusepath{clip}%
\pgfsetbuttcap%
\pgfsetroundjoin%
\pgfsetlinewidth{0.250937pt}%
\definecolor{currentstroke}{rgb}{0.000000,0.000000,0.000000}%
\pgfsetstrokecolor{currentstroke}%
\pgfsetdash{}{0pt}%
\pgfpathmoveto{\pgfqpoint{0.634712in}{1.184996in}}%
\pgfpathlineto{\pgfqpoint{0.633563in}{1.186842in}}%
\pgfpathlineto{\pgfqpoint{0.631261in}{1.196491in}}%
\pgfpathlineto{\pgfqpoint{0.634712in}{1.201676in}}%
\pgfpathlineto{\pgfqpoint{0.641677in}{1.196491in}}%
\pgfpathlineto{\pgfqpoint{0.638111in}{1.186842in}}%
\pgfpathlineto{\pgfqpoint{0.634712in}{1.184996in}}%
\pgfusepath{stroke}%
\end{pgfscope}%
\begin{pgfscope}%
\pgfpathrectangle{\pgfqpoint{0.625000in}{0.550000in}}{\pgfqpoint{3.875000in}{3.850000in}} %
\pgfusepath{clip}%
\pgfsetbuttcap%
\pgfsetroundjoin%
\pgfsetlinewidth{0.250937pt}%
\definecolor{currentstroke}{rgb}{0.000000,0.000000,0.000000}%
\pgfsetstrokecolor{currentstroke}%
\pgfsetdash{}{0pt}%
\pgfpathmoveto{\pgfqpoint{0.634712in}{3.757973in}}%
\pgfpathlineto{\pgfqpoint{0.631261in}{3.763158in}}%
\pgfpathlineto{\pgfqpoint{0.633563in}{3.772807in}}%
\pgfpathlineto{\pgfqpoint{0.634712in}{3.774653in}}%
\pgfpathlineto{\pgfqpoint{0.638111in}{3.772807in}}%
\pgfpathlineto{\pgfqpoint{0.641677in}{3.763158in}}%
\pgfpathlineto{\pgfqpoint{0.634712in}{3.757973in}}%
\pgfusepath{stroke}%
\end{pgfscope}%
\begin{pgfscope}%
\pgfpathrectangle{\pgfqpoint{0.625000in}{0.550000in}}{\pgfqpoint{3.875000in}{3.850000in}} %
\pgfusepath{clip}%
\pgfsetbuttcap%
\pgfsetroundjoin%
\pgfsetlinewidth{0.250937pt}%
\definecolor{currentstroke}{rgb}{0.000000,0.000000,0.000000}%
\pgfsetstrokecolor{currentstroke}%
\pgfsetdash{}{0pt}%
\pgfpathmoveto{\pgfqpoint{0.625000in}{0.633762in}}%
\pgfpathlineto{\pgfqpoint{0.631775in}{0.627193in}}%
\pgfpathlineto{\pgfqpoint{0.625000in}{0.620644in}}%
\pgfusepath{stroke}%
\end{pgfscope}%
\begin{pgfscope}%
\pgfpathrectangle{\pgfqpoint{0.625000in}{0.550000in}}{\pgfqpoint{3.875000in}{3.850000in}} %
\pgfusepath{clip}%
\pgfsetbuttcap%
\pgfsetroundjoin%
\pgfsetlinewidth{0.250937pt}%
\definecolor{currentstroke}{rgb}{0.000000,0.000000,0.000000}%
\pgfsetstrokecolor{currentstroke}%
\pgfsetdash{}{0pt}%
\pgfpathmoveto{\pgfqpoint{0.625000in}{0.787730in}}%
\pgfpathlineto{\pgfqpoint{0.631654in}{0.781579in}}%
\pgfpathlineto{\pgfqpoint{0.625000in}{0.775154in}}%
\pgfusepath{stroke}%
\end{pgfscope}%
\begin{pgfscope}%
\pgfpathrectangle{\pgfqpoint{0.625000in}{0.550000in}}{\pgfqpoint{3.875000in}{3.850000in}} %
\pgfusepath{clip}%
\pgfsetbuttcap%
\pgfsetroundjoin%
\pgfsetlinewidth{0.250937pt}%
\definecolor{currentstroke}{rgb}{0.000000,0.000000,0.000000}%
\pgfsetstrokecolor{currentstroke}%
\pgfsetdash{}{0pt}%
\pgfpathmoveto{\pgfqpoint{0.625000in}{0.942411in}}%
\pgfpathlineto{\pgfqpoint{0.633161in}{0.935965in}}%
\pgfpathlineto{\pgfqpoint{0.625000in}{0.929461in}}%
\pgfusepath{stroke}%
\end{pgfscope}%
\begin{pgfscope}%
\pgfpathrectangle{\pgfqpoint{0.625000in}{0.550000in}}{\pgfqpoint{3.875000in}{3.850000in}} %
\pgfusepath{clip}%
\pgfsetbuttcap%
\pgfsetroundjoin%
\pgfsetlinewidth{0.250937pt}%
\definecolor{currentstroke}{rgb}{0.000000,0.000000,0.000000}%
\pgfsetstrokecolor{currentstroke}%
\pgfsetdash{}{0pt}%
\pgfpathmoveto{\pgfqpoint{0.625000in}{1.096748in}}%
\pgfpathlineto{\pgfqpoint{0.631671in}{1.090351in}}%
\pgfpathlineto{\pgfqpoint{0.625000in}{1.083977in}}%
\pgfusepath{stroke}%
\end{pgfscope}%
\begin{pgfscope}%
\pgfpathrectangle{\pgfqpoint{0.625000in}{0.550000in}}{\pgfqpoint{3.875000in}{3.850000in}} %
\pgfusepath{clip}%
\pgfsetbuttcap%
\pgfsetroundjoin%
\pgfsetlinewidth{0.250937pt}%
\definecolor{currentstroke}{rgb}{0.000000,0.000000,0.000000}%
\pgfsetstrokecolor{currentstroke}%
\pgfsetdash{}{0pt}%
\pgfpathmoveto{\pgfqpoint{0.625000in}{1.251309in}}%
\pgfpathlineto{\pgfqpoint{0.631659in}{1.244737in}}%
\pgfpathlineto{\pgfqpoint{0.625000in}{1.238346in}}%
\pgfusepath{stroke}%
\end{pgfscope}%
\begin{pgfscope}%
\pgfpathrectangle{\pgfqpoint{0.625000in}{0.550000in}}{\pgfqpoint{3.875000in}{3.850000in}} %
\pgfusepath{clip}%
\pgfsetbuttcap%
\pgfsetroundjoin%
\pgfsetlinewidth{0.250937pt}%
\definecolor{currentstroke}{rgb}{0.000000,0.000000,0.000000}%
\pgfsetstrokecolor{currentstroke}%
\pgfsetdash{}{0pt}%
\pgfpathmoveto{\pgfqpoint{0.625000in}{1.405379in}}%
\pgfpathlineto{\pgfqpoint{0.631577in}{1.399123in}}%
\pgfpathlineto{\pgfqpoint{0.625000in}{1.392863in}}%
\pgfusepath{stroke}%
\end{pgfscope}%
\begin{pgfscope}%
\pgfpathrectangle{\pgfqpoint{0.625000in}{0.550000in}}{\pgfqpoint{3.875000in}{3.850000in}} %
\pgfusepath{clip}%
\pgfsetbuttcap%
\pgfsetroundjoin%
\pgfsetlinewidth{0.250937pt}%
\definecolor{currentstroke}{rgb}{0.000000,0.000000,0.000000}%
\pgfsetstrokecolor{currentstroke}%
\pgfsetdash{}{0pt}%
\pgfpathmoveto{\pgfqpoint{0.625000in}{1.559773in}}%
\pgfpathlineto{\pgfqpoint{0.631575in}{1.553509in}}%
\pgfpathlineto{\pgfqpoint{0.625000in}{1.547041in}}%
\pgfusepath{stroke}%
\end{pgfscope}%
\begin{pgfscope}%
\pgfpathrectangle{\pgfqpoint{0.625000in}{0.550000in}}{\pgfqpoint{3.875000in}{3.850000in}} %
\pgfusepath{clip}%
\pgfsetbuttcap%
\pgfsetroundjoin%
\pgfsetlinewidth{0.250937pt}%
\definecolor{currentstroke}{rgb}{0.000000,0.000000,0.000000}%
\pgfsetstrokecolor{currentstroke}%
\pgfsetdash{}{0pt}%
\pgfpathmoveto{\pgfqpoint{0.625000in}{1.714136in}}%
\pgfpathlineto{\pgfqpoint{0.633042in}{1.707895in}}%
\pgfpathlineto{\pgfqpoint{0.625000in}{1.701614in}}%
\pgfusepath{stroke}%
\end{pgfscope}%
\begin{pgfscope}%
\pgfpathrectangle{\pgfqpoint{0.625000in}{0.550000in}}{\pgfqpoint{3.875000in}{3.850000in}} %
\pgfusepath{clip}%
\pgfsetbuttcap%
\pgfsetroundjoin%
\pgfsetlinewidth{0.250937pt}%
\definecolor{currentstroke}{rgb}{0.000000,0.000000,0.000000}%
\pgfsetstrokecolor{currentstroke}%
\pgfsetdash{}{0pt}%
\pgfpathmoveto{\pgfqpoint{0.625000in}{1.868545in}}%
\pgfpathlineto{\pgfqpoint{0.631502in}{1.862281in}}%
\pgfpathlineto{\pgfqpoint{0.625000in}{1.856146in}}%
\pgfusepath{stroke}%
\end{pgfscope}%
\begin{pgfscope}%
\pgfpathrectangle{\pgfqpoint{0.625000in}{0.550000in}}{\pgfqpoint{3.875000in}{3.850000in}} %
\pgfusepath{clip}%
\pgfsetbuttcap%
\pgfsetroundjoin%
\pgfsetlinewidth{0.250937pt}%
\definecolor{currentstroke}{rgb}{0.000000,0.000000,0.000000}%
\pgfsetstrokecolor{currentstroke}%
\pgfsetdash{}{0pt}%
\pgfpathmoveto{\pgfqpoint{0.625000in}{2.023120in}}%
\pgfpathlineto{\pgfqpoint{0.631460in}{2.016667in}}%
\pgfpathlineto{\pgfqpoint{0.625000in}{2.010266in}}%
\pgfusepath{stroke}%
\end{pgfscope}%
\begin{pgfscope}%
\pgfpathrectangle{\pgfqpoint{0.625000in}{0.550000in}}{\pgfqpoint{3.875000in}{3.850000in}} %
\pgfusepath{clip}%
\pgfsetbuttcap%
\pgfsetroundjoin%
\pgfsetlinewidth{0.250937pt}%
\definecolor{currentstroke}{rgb}{0.000000,0.000000,0.000000}%
\pgfsetstrokecolor{currentstroke}%
\pgfsetdash{}{0pt}%
\pgfpathmoveto{\pgfqpoint{0.625000in}{2.177193in}}%
\pgfpathlineto{\pgfqpoint{0.631494in}{2.171053in}}%
\pgfpathlineto{\pgfqpoint{0.625000in}{2.164736in}}%
\pgfusepath{stroke}%
\end{pgfscope}%
\begin{pgfscope}%
\pgfpathrectangle{\pgfqpoint{0.625000in}{0.550000in}}{\pgfqpoint{3.875000in}{3.850000in}} %
\pgfusepath{clip}%
\pgfsetbuttcap%
\pgfsetroundjoin%
\pgfsetlinewidth{0.250937pt}%
\definecolor{currentstroke}{rgb}{0.000000,0.000000,0.000000}%
\pgfsetstrokecolor{currentstroke}%
\pgfsetdash{}{0pt}%
\pgfpathmoveto{\pgfqpoint{0.625000in}{2.331528in}}%
\pgfpathlineto{\pgfqpoint{0.631501in}{2.325439in}}%
\pgfpathlineto{\pgfqpoint{0.625000in}{2.318923in}}%
\pgfusepath{stroke}%
\end{pgfscope}%
\begin{pgfscope}%
\pgfpathrectangle{\pgfqpoint{0.625000in}{0.550000in}}{\pgfqpoint{3.875000in}{3.850000in}} %
\pgfusepath{clip}%
\pgfsetbuttcap%
\pgfsetroundjoin%
\pgfsetlinewidth{0.250937pt}%
\definecolor{currentstroke}{rgb}{0.000000,0.000000,0.000000}%
\pgfsetstrokecolor{currentstroke}%
\pgfsetdash{}{0pt}%
\pgfpathmoveto{\pgfqpoint{0.625000in}{2.486873in}}%
\pgfpathlineto{\pgfqpoint{0.627439in}{2.489474in}}%
\pgfpathlineto{\pgfqpoint{0.628174in}{2.499123in}}%
\pgfpathlineto{\pgfqpoint{0.631504in}{2.508772in}}%
\pgfpathlineto{\pgfqpoint{0.634712in}{2.516679in}}%
\pgfpathlineto{\pgfqpoint{0.636419in}{2.518421in}}%
\pgfpathlineto{\pgfqpoint{0.643249in}{2.528070in}}%
\pgfpathlineto{\pgfqpoint{0.644424in}{2.530010in}}%
\pgfpathlineto{\pgfqpoint{0.652920in}{2.537719in}}%
\pgfpathlineto{\pgfqpoint{0.654135in}{2.539119in}}%
\pgfpathlineto{\pgfqpoint{0.663847in}{2.545806in}}%
\pgfpathlineto{\pgfqpoint{0.667637in}{2.547368in}}%
\pgfpathlineto{\pgfqpoint{0.673559in}{2.550675in}}%
\pgfpathlineto{\pgfqpoint{0.683271in}{2.553948in}}%
\pgfpathlineto{\pgfqpoint{0.692982in}{2.555824in}}%
\pgfpathlineto{\pgfqpoint{0.702694in}{2.556549in}}%
\pgfpathlineto{\pgfqpoint{0.712406in}{2.556153in}}%
\pgfpathlineto{\pgfqpoint{0.722118in}{2.554562in}}%
\pgfpathlineto{\pgfqpoint{0.731830in}{2.551642in}}%
\pgfpathlineto{\pgfqpoint{0.741275in}{2.547368in}}%
\pgfpathlineto{\pgfqpoint{0.741541in}{2.547228in}}%
\pgfpathlineto{\pgfqpoint{0.751253in}{2.541048in}}%
\pgfpathlineto{\pgfqpoint{0.755549in}{2.537719in}}%
\pgfpathlineto{\pgfqpoint{0.760965in}{2.532436in}}%
\pgfpathlineto{\pgfqpoint{0.765044in}{2.528070in}}%
\pgfpathlineto{\pgfqpoint{0.770677in}{2.519987in}}%
\pgfpathlineto{\pgfqpoint{0.771743in}{2.518421in}}%
\pgfpathlineto{\pgfqpoint{0.776649in}{2.508772in}}%
\pgfpathlineto{\pgfqpoint{0.779802in}{2.499123in}}%
\pgfpathlineto{\pgfqpoint{0.780388in}{2.495990in}}%
\pgfpathlineto{\pgfqpoint{0.781721in}{2.489474in}}%
\pgfpathlineto{\pgfqpoint{0.782367in}{2.479825in}}%
\pgfpathlineto{\pgfqpoint{0.781721in}{2.470175in}}%
\pgfpathlineto{\pgfqpoint{0.780388in}{2.463659in}}%
\pgfpathlineto{\pgfqpoint{0.779802in}{2.460526in}}%
\pgfpathlineto{\pgfqpoint{0.776649in}{2.450877in}}%
\pgfpathlineto{\pgfqpoint{0.771743in}{2.441228in}}%
\pgfpathlineto{\pgfqpoint{0.770677in}{2.439662in}}%
\pgfpathlineto{\pgfqpoint{0.765044in}{2.431579in}}%
\pgfpathlineto{\pgfqpoint{0.760965in}{2.427213in}}%
\pgfpathlineto{\pgfqpoint{0.755549in}{2.421930in}}%
\pgfpathlineto{\pgfqpoint{0.751253in}{2.418602in}}%
\pgfpathlineto{\pgfqpoint{0.741541in}{2.412421in}}%
\pgfpathlineto{\pgfqpoint{0.741275in}{2.412281in}}%
\pgfpathlineto{\pgfqpoint{0.731830in}{2.408007in}}%
\pgfpathlineto{\pgfqpoint{0.722118in}{2.405087in}}%
\pgfpathlineto{\pgfqpoint{0.712406in}{2.403496in}}%
\pgfpathlineto{\pgfqpoint{0.702694in}{2.403101in}}%
\pgfpathlineto{\pgfqpoint{0.692982in}{2.403825in}}%
\pgfpathlineto{\pgfqpoint{0.683271in}{2.405702in}}%
\pgfpathlineto{\pgfqpoint{0.673559in}{2.408974in}}%
\pgfpathlineto{\pgfqpoint{0.667637in}{2.412281in}}%
\pgfpathlineto{\pgfqpoint{0.663847in}{2.413843in}}%
\pgfpathlineto{\pgfqpoint{0.654135in}{2.420530in}}%
\pgfpathlineto{\pgfqpoint{0.652920in}{2.421930in}}%
\pgfpathlineto{\pgfqpoint{0.644424in}{2.429640in}}%
\pgfpathlineto{\pgfqpoint{0.643249in}{2.431579in}}%
\pgfpathlineto{\pgfqpoint{0.636419in}{2.441228in}}%
\pgfpathlineto{\pgfqpoint{0.634712in}{2.442970in}}%
\pgfpathlineto{\pgfqpoint{0.631504in}{2.450877in}}%
\pgfpathlineto{\pgfqpoint{0.628181in}{2.460526in}}%
\pgfpathlineto{\pgfqpoint{0.627439in}{2.470175in}}%
\pgfpathlineto{\pgfqpoint{0.625000in}{2.472776in}}%
\pgfusepath{stroke}%
\end{pgfscope}%
\begin{pgfscope}%
\pgfpathrectangle{\pgfqpoint{0.625000in}{0.550000in}}{\pgfqpoint{3.875000in}{3.850000in}} %
\pgfusepath{clip}%
\pgfsetbuttcap%
\pgfsetroundjoin%
\pgfsetlinewidth{0.250937pt}%
\definecolor{currentstroke}{rgb}{0.000000,0.000000,0.000000}%
\pgfsetstrokecolor{currentstroke}%
\pgfsetdash{}{0pt}%
\pgfpathmoveto{\pgfqpoint{0.625000in}{2.640721in}}%
\pgfpathlineto{\pgfqpoint{0.631495in}{2.634211in}}%
\pgfpathlineto{\pgfqpoint{0.625000in}{2.628128in}}%
\pgfusepath{stroke}%
\end{pgfscope}%
\begin{pgfscope}%
\pgfpathrectangle{\pgfqpoint{0.625000in}{0.550000in}}{\pgfqpoint{3.875000in}{3.850000in}} %
\pgfusepath{clip}%
\pgfsetbuttcap%
\pgfsetroundjoin%
\pgfsetlinewidth{0.250937pt}%
\definecolor{currentstroke}{rgb}{0.000000,0.000000,0.000000}%
\pgfsetstrokecolor{currentstroke}%
\pgfsetdash{}{0pt}%
\pgfpathmoveto{\pgfqpoint{0.625000in}{2.794893in}}%
\pgfpathlineto{\pgfqpoint{0.631474in}{2.788596in}}%
\pgfpathlineto{\pgfqpoint{0.625000in}{2.782477in}}%
\pgfusepath{stroke}%
\end{pgfscope}%
\begin{pgfscope}%
\pgfpathrectangle{\pgfqpoint{0.625000in}{0.550000in}}{\pgfqpoint{3.875000in}{3.850000in}} %
\pgfusepath{clip}%
\pgfsetbuttcap%
\pgfsetroundjoin%
\pgfsetlinewidth{0.250937pt}%
\definecolor{currentstroke}{rgb}{0.000000,0.000000,0.000000}%
\pgfsetstrokecolor{currentstroke}%
\pgfsetdash{}{0pt}%
\pgfpathmoveto{\pgfqpoint{0.625000in}{2.949327in}}%
\pgfpathlineto{\pgfqpoint{0.631403in}{2.942982in}}%
\pgfpathlineto{\pgfqpoint{0.625000in}{2.936585in}}%
\pgfusepath{stroke}%
\end{pgfscope}%
\begin{pgfscope}%
\pgfpathrectangle{\pgfqpoint{0.625000in}{0.550000in}}{\pgfqpoint{3.875000in}{3.850000in}} %
\pgfusepath{clip}%
\pgfsetbuttcap%
\pgfsetroundjoin%
\pgfsetlinewidth{0.250937pt}%
\definecolor{currentstroke}{rgb}{0.000000,0.000000,0.000000}%
\pgfsetstrokecolor{currentstroke}%
\pgfsetdash{}{0pt}%
\pgfpathmoveto{\pgfqpoint{0.625000in}{3.103458in}}%
\pgfpathlineto{\pgfqpoint{0.631458in}{3.097368in}}%
\pgfpathlineto{\pgfqpoint{0.625000in}{3.091149in}}%
\pgfusepath{stroke}%
\end{pgfscope}%
\begin{pgfscope}%
\pgfpathrectangle{\pgfqpoint{0.625000in}{0.550000in}}{\pgfqpoint{3.875000in}{3.850000in}} %
\pgfusepath{clip}%
\pgfsetbuttcap%
\pgfsetroundjoin%
\pgfsetlinewidth{0.250937pt}%
\definecolor{currentstroke}{rgb}{0.000000,0.000000,0.000000}%
\pgfsetstrokecolor{currentstroke}%
\pgfsetdash{}{0pt}%
\pgfpathmoveto{\pgfqpoint{0.625000in}{3.257873in}}%
\pgfpathlineto{\pgfqpoint{0.632938in}{3.251754in}}%
\pgfpathlineto{\pgfqpoint{0.625000in}{3.245676in}}%
\pgfusepath{stroke}%
\end{pgfscope}%
\begin{pgfscope}%
\pgfpathrectangle{\pgfqpoint{0.625000in}{0.550000in}}{\pgfqpoint{3.875000in}{3.850000in}} %
\pgfusepath{clip}%
\pgfsetbuttcap%
\pgfsetroundjoin%
\pgfsetlinewidth{0.250937pt}%
\definecolor{currentstroke}{rgb}{0.000000,0.000000,0.000000}%
\pgfsetstrokecolor{currentstroke}%
\pgfsetdash{}{0pt}%
\pgfpathmoveto{\pgfqpoint{0.625000in}{3.412499in}}%
\pgfpathlineto{\pgfqpoint{0.631466in}{3.406140in}}%
\pgfpathlineto{\pgfqpoint{0.625000in}{3.399989in}}%
\pgfusepath{stroke}%
\end{pgfscope}%
\begin{pgfscope}%
\pgfpathrectangle{\pgfqpoint{0.625000in}{0.550000in}}{\pgfqpoint{3.875000in}{3.850000in}} %
\pgfusepath{clip}%
\pgfsetbuttcap%
\pgfsetroundjoin%
\pgfsetlinewidth{0.250937pt}%
\definecolor{currentstroke}{rgb}{0.000000,0.000000,0.000000}%
\pgfsetstrokecolor{currentstroke}%
\pgfsetdash{}{0pt}%
\pgfpathmoveto{\pgfqpoint{0.625000in}{3.566652in}}%
\pgfpathlineto{\pgfqpoint{0.631447in}{3.560526in}}%
\pgfpathlineto{\pgfqpoint{0.625000in}{3.554405in}}%
\pgfusepath{stroke}%
\end{pgfscope}%
\begin{pgfscope}%
\pgfpathrectangle{\pgfqpoint{0.625000in}{0.550000in}}{\pgfqpoint{3.875000in}{3.850000in}} %
\pgfusepath{clip}%
\pgfsetbuttcap%
\pgfsetroundjoin%
\pgfsetlinewidth{0.250937pt}%
\definecolor{currentstroke}{rgb}{0.000000,0.000000,0.000000}%
\pgfsetstrokecolor{currentstroke}%
\pgfsetdash{}{0pt}%
\pgfpathmoveto{\pgfqpoint{0.625000in}{3.721105in}}%
\pgfpathlineto{\pgfqpoint{0.631466in}{3.714912in}}%
\pgfpathlineto{\pgfqpoint{0.625000in}{3.708533in}}%
\pgfusepath{stroke}%
\end{pgfscope}%
\begin{pgfscope}%
\pgfpathrectangle{\pgfqpoint{0.625000in}{0.550000in}}{\pgfqpoint{3.875000in}{3.850000in}} %
\pgfusepath{clip}%
\pgfsetbuttcap%
\pgfsetroundjoin%
\pgfsetlinewidth{0.250937pt}%
\definecolor{currentstroke}{rgb}{0.000000,0.000000,0.000000}%
\pgfsetstrokecolor{currentstroke}%
\pgfsetdash{}{0pt}%
\pgfpathmoveto{\pgfqpoint{0.625000in}{3.875342in}}%
\pgfpathlineto{\pgfqpoint{0.631352in}{3.869298in}}%
\pgfpathlineto{\pgfqpoint{0.625000in}{3.863229in}}%
\pgfusepath{stroke}%
\end{pgfscope}%
\begin{pgfscope}%
\pgfpathrectangle{\pgfqpoint{0.625000in}{0.550000in}}{\pgfqpoint{3.875000in}{3.850000in}} %
\pgfusepath{clip}%
\pgfsetbuttcap%
\pgfsetroundjoin%
\pgfsetlinewidth{0.250937pt}%
\definecolor{currentstroke}{rgb}{0.000000,0.000000,0.000000}%
\pgfsetstrokecolor{currentstroke}%
\pgfsetdash{}{0pt}%
\pgfpathmoveto{\pgfqpoint{0.625000in}{4.029894in}}%
\pgfpathlineto{\pgfqpoint{0.632975in}{4.023684in}}%
\pgfpathlineto{\pgfqpoint{0.625000in}{4.017535in}}%
\pgfusepath{stroke}%
\end{pgfscope}%
\begin{pgfscope}%
\pgfpathrectangle{\pgfqpoint{0.625000in}{0.550000in}}{\pgfqpoint{3.875000in}{3.850000in}} %
\pgfusepath{clip}%
\pgfsetbuttcap%
\pgfsetroundjoin%
\pgfsetlinewidth{0.250937pt}%
\definecolor{currentstroke}{rgb}{0.000000,0.000000,0.000000}%
\pgfsetstrokecolor{currentstroke}%
\pgfsetdash{}{0pt}%
\pgfpathmoveto{\pgfqpoint{0.625000in}{4.184163in}}%
\pgfpathlineto{\pgfqpoint{0.631329in}{4.178070in}}%
\pgfpathlineto{\pgfqpoint{0.625000in}{4.172263in}}%
\pgfusepath{stroke}%
\end{pgfscope}%
\begin{pgfscope}%
\pgfpathrectangle{\pgfqpoint{0.625000in}{0.550000in}}{\pgfqpoint{3.875000in}{3.850000in}} %
\pgfusepath{clip}%
\pgfsetbuttcap%
\pgfsetroundjoin%
\pgfsetlinewidth{0.250937pt}%
\definecolor{currentstroke}{rgb}{0.000000,0.000000,0.000000}%
\pgfsetstrokecolor{currentstroke}%
\pgfsetdash{}{0pt}%
\pgfpathmoveto{\pgfqpoint{0.625000in}{4.338535in}}%
\pgfpathlineto{\pgfqpoint{0.631315in}{4.332456in}}%
\pgfpathlineto{\pgfqpoint{0.625000in}{4.326356in}}%
\pgfusepath{stroke}%
\end{pgfscope}%
\begin{pgfscope}%
\pgfpathrectangle{\pgfqpoint{0.625000in}{0.550000in}}{\pgfqpoint{3.875000in}{3.850000in}} %
\pgfusepath{clip}%
\pgfsetbuttcap%
\pgfsetroundjoin%
\pgfsetlinewidth{0.250937pt}%
\definecolor{currentstroke}{rgb}{0.000000,0.000000,0.000000}%
\pgfsetstrokecolor{currentstroke}%
\pgfsetdash{}{0pt}%
\pgfpathmoveto{\pgfqpoint{0.634712in}{1.184662in}}%
\pgfpathlineto{\pgfqpoint{0.633355in}{1.186842in}}%
\pgfpathlineto{\pgfqpoint{0.631086in}{1.196491in}}%
\pgfpathlineto{\pgfqpoint{0.634712in}{1.201939in}}%
\pgfpathlineto{\pgfqpoint{0.642029in}{1.196491in}}%
\pgfpathlineto{\pgfqpoint{0.638726in}{1.186842in}}%
\pgfpathlineto{\pgfqpoint{0.634712in}{1.184662in}}%
\pgfusepath{stroke}%
\end{pgfscope}%
\begin{pgfscope}%
\pgfpathrectangle{\pgfqpoint{0.625000in}{0.550000in}}{\pgfqpoint{3.875000in}{3.850000in}} %
\pgfusepath{clip}%
\pgfsetbuttcap%
\pgfsetroundjoin%
\pgfsetlinewidth{0.250937pt}%
\definecolor{currentstroke}{rgb}{0.000000,0.000000,0.000000}%
\pgfsetstrokecolor{currentstroke}%
\pgfsetdash{}{0pt}%
\pgfpathmoveto{\pgfqpoint{0.634712in}{3.757711in}}%
\pgfpathlineto{\pgfqpoint{0.631086in}{3.763158in}}%
\pgfpathlineto{\pgfqpoint{0.633355in}{3.772807in}}%
\pgfpathlineto{\pgfqpoint{0.634712in}{3.774987in}}%
\pgfpathlineto{\pgfqpoint{0.638726in}{3.772807in}}%
\pgfpathlineto{\pgfqpoint{0.642029in}{3.763158in}}%
\pgfpathlineto{\pgfqpoint{0.634712in}{3.757711in}}%
\pgfusepath{stroke}%
\end{pgfscope}%
\begin{pgfscope}%
\pgfpathrectangle{\pgfqpoint{0.625000in}{0.550000in}}{\pgfqpoint{3.875000in}{3.850000in}} %
\pgfusepath{clip}%
\pgfsetbuttcap%
\pgfsetroundjoin%
\pgfsetlinewidth{0.250937pt}%
\definecolor{currentstroke}{rgb}{0.000000,0.000000,0.000000}%
\pgfsetstrokecolor{currentstroke}%
\pgfsetdash{}{0pt}%
\pgfpathmoveto{\pgfqpoint{0.625000in}{0.633838in}}%
\pgfpathlineto{\pgfqpoint{0.631853in}{0.627193in}}%
\pgfpathlineto{\pgfqpoint{0.625000in}{0.620568in}}%
\pgfusepath{stroke}%
\end{pgfscope}%
\begin{pgfscope}%
\pgfpathrectangle{\pgfqpoint{0.625000in}{0.550000in}}{\pgfqpoint{3.875000in}{3.850000in}} %
\pgfusepath{clip}%
\pgfsetbuttcap%
\pgfsetroundjoin%
\pgfsetlinewidth{0.250937pt}%
\definecolor{currentstroke}{rgb}{0.000000,0.000000,0.000000}%
\pgfsetstrokecolor{currentstroke}%
\pgfsetdash{}{0pt}%
\pgfpathmoveto{\pgfqpoint{0.625000in}{0.787807in}}%
\pgfpathlineto{\pgfqpoint{0.631738in}{0.781579in}}%
\pgfpathlineto{\pgfqpoint{0.625000in}{0.775074in}}%
\pgfusepath{stroke}%
\end{pgfscope}%
\begin{pgfscope}%
\pgfpathrectangle{\pgfqpoint{0.625000in}{0.550000in}}{\pgfqpoint{3.875000in}{3.850000in}} %
\pgfusepath{clip}%
\pgfsetbuttcap%
\pgfsetroundjoin%
\pgfsetlinewidth{0.250937pt}%
\definecolor{currentstroke}{rgb}{0.000000,0.000000,0.000000}%
\pgfsetstrokecolor{currentstroke}%
\pgfsetdash{}{0pt}%
\pgfpathmoveto{\pgfqpoint{0.625000in}{0.942488in}}%
\pgfpathlineto{\pgfqpoint{0.633258in}{0.935965in}}%
\pgfpathlineto{\pgfqpoint{0.625000in}{0.929383in}}%
\pgfusepath{stroke}%
\end{pgfscope}%
\begin{pgfscope}%
\pgfpathrectangle{\pgfqpoint{0.625000in}{0.550000in}}{\pgfqpoint{3.875000in}{3.850000in}} %
\pgfusepath{clip}%
\pgfsetbuttcap%
\pgfsetroundjoin%
\pgfsetlinewidth{0.250937pt}%
\definecolor{currentstroke}{rgb}{0.000000,0.000000,0.000000}%
\pgfsetstrokecolor{currentstroke}%
\pgfsetdash{}{0pt}%
\pgfpathmoveto{\pgfqpoint{0.625000in}{1.096827in}}%
\pgfpathlineto{\pgfqpoint{0.631753in}{1.090351in}}%
\pgfpathlineto{\pgfqpoint{0.625000in}{1.083899in}}%
\pgfusepath{stroke}%
\end{pgfscope}%
\begin{pgfscope}%
\pgfpathrectangle{\pgfqpoint{0.625000in}{0.550000in}}{\pgfqpoint{3.875000in}{3.850000in}} %
\pgfusepath{clip}%
\pgfsetbuttcap%
\pgfsetroundjoin%
\pgfsetlinewidth{0.250937pt}%
\definecolor{currentstroke}{rgb}{0.000000,0.000000,0.000000}%
\pgfsetstrokecolor{currentstroke}%
\pgfsetdash{}{0pt}%
\pgfpathmoveto{\pgfqpoint{0.625000in}{1.251392in}}%
\pgfpathlineto{\pgfqpoint{0.631743in}{1.244737in}}%
\pgfpathlineto{\pgfqpoint{0.625000in}{1.238265in}}%
\pgfusepath{stroke}%
\end{pgfscope}%
\begin{pgfscope}%
\pgfpathrectangle{\pgfqpoint{0.625000in}{0.550000in}}{\pgfqpoint{3.875000in}{3.850000in}} %
\pgfusepath{clip}%
\pgfsetbuttcap%
\pgfsetroundjoin%
\pgfsetlinewidth{0.250937pt}%
\definecolor{currentstroke}{rgb}{0.000000,0.000000,0.000000}%
\pgfsetstrokecolor{currentstroke}%
\pgfsetdash{}{0pt}%
\pgfpathmoveto{\pgfqpoint{0.625000in}{1.405460in}}%
\pgfpathlineto{\pgfqpoint{0.631662in}{1.399123in}}%
\pgfpathlineto{\pgfqpoint{0.625000in}{1.392782in}}%
\pgfusepath{stroke}%
\end{pgfscope}%
\begin{pgfscope}%
\pgfpathrectangle{\pgfqpoint{0.625000in}{0.550000in}}{\pgfqpoint{3.875000in}{3.850000in}} %
\pgfusepath{clip}%
\pgfsetbuttcap%
\pgfsetroundjoin%
\pgfsetlinewidth{0.250937pt}%
\definecolor{currentstroke}{rgb}{0.000000,0.000000,0.000000}%
\pgfsetstrokecolor{currentstroke}%
\pgfsetdash{}{0pt}%
\pgfpathmoveto{\pgfqpoint{0.625000in}{1.559853in}}%
\pgfpathlineto{\pgfqpoint{0.631659in}{1.553509in}}%
\pgfpathlineto{\pgfqpoint{0.625000in}{1.546958in}}%
\pgfusepath{stroke}%
\end{pgfscope}%
\begin{pgfscope}%
\pgfpathrectangle{\pgfqpoint{0.625000in}{0.550000in}}{\pgfqpoint{3.875000in}{3.850000in}} %
\pgfusepath{clip}%
\pgfsetbuttcap%
\pgfsetroundjoin%
\pgfsetlinewidth{0.250937pt}%
\definecolor{currentstroke}{rgb}{0.000000,0.000000,0.000000}%
\pgfsetstrokecolor{currentstroke}%
\pgfsetdash{}{0pt}%
\pgfpathmoveto{\pgfqpoint{0.625000in}{1.714217in}}%
\pgfpathlineto{\pgfqpoint{0.633146in}{1.707895in}}%
\pgfpathlineto{\pgfqpoint{0.625000in}{1.701532in}}%
\pgfusepath{stroke}%
\end{pgfscope}%
\begin{pgfscope}%
\pgfpathrectangle{\pgfqpoint{0.625000in}{0.550000in}}{\pgfqpoint{3.875000in}{3.850000in}} %
\pgfusepath{clip}%
\pgfsetbuttcap%
\pgfsetroundjoin%
\pgfsetlinewidth{0.250937pt}%
\definecolor{currentstroke}{rgb}{0.000000,0.000000,0.000000}%
\pgfsetstrokecolor{currentstroke}%
\pgfsetdash{}{0pt}%
\pgfpathmoveto{\pgfqpoint{0.625000in}{1.868628in}}%
\pgfpathlineto{\pgfqpoint{0.631588in}{1.862281in}}%
\pgfpathlineto{\pgfqpoint{0.625000in}{1.856064in}}%
\pgfusepath{stroke}%
\end{pgfscope}%
\begin{pgfscope}%
\pgfpathrectangle{\pgfqpoint{0.625000in}{0.550000in}}{\pgfqpoint{3.875000in}{3.850000in}} %
\pgfusepath{clip}%
\pgfsetbuttcap%
\pgfsetroundjoin%
\pgfsetlinewidth{0.250937pt}%
\definecolor{currentstroke}{rgb}{0.000000,0.000000,0.000000}%
\pgfsetstrokecolor{currentstroke}%
\pgfsetdash{}{0pt}%
\pgfpathmoveto{\pgfqpoint{0.625000in}{2.023207in}}%
\pgfpathlineto{\pgfqpoint{0.631547in}{2.016667in}}%
\pgfpathlineto{\pgfqpoint{0.625000in}{2.010179in}}%
\pgfusepath{stroke}%
\end{pgfscope}%
\begin{pgfscope}%
\pgfpathrectangle{\pgfqpoint{0.625000in}{0.550000in}}{\pgfqpoint{3.875000in}{3.850000in}} %
\pgfusepath{clip}%
\pgfsetbuttcap%
\pgfsetroundjoin%
\pgfsetlinewidth{0.250937pt}%
\definecolor{currentstroke}{rgb}{0.000000,0.000000,0.000000}%
\pgfsetstrokecolor{currentstroke}%
\pgfsetdash{}{0pt}%
\pgfpathmoveto{\pgfqpoint{0.625000in}{2.177275in}}%
\pgfpathlineto{\pgfqpoint{0.631581in}{2.171053in}}%
\pgfpathlineto{\pgfqpoint{0.625000in}{2.164652in}}%
\pgfusepath{stroke}%
\end{pgfscope}%
\begin{pgfscope}%
\pgfpathrectangle{\pgfqpoint{0.625000in}{0.550000in}}{\pgfqpoint{3.875000in}{3.850000in}} %
\pgfusepath{clip}%
\pgfsetbuttcap%
\pgfsetroundjoin%
\pgfsetlinewidth{0.250937pt}%
\definecolor{currentstroke}{rgb}{0.000000,0.000000,0.000000}%
\pgfsetstrokecolor{currentstroke}%
\pgfsetdash{}{0pt}%
\pgfpathmoveto{\pgfqpoint{0.625000in}{2.331611in}}%
\pgfpathlineto{\pgfqpoint{0.631589in}{2.325439in}}%
\pgfpathlineto{\pgfqpoint{0.625000in}{2.318834in}}%
\pgfusepath{stroke}%
\end{pgfscope}%
\begin{pgfscope}%
\pgfpathrectangle{\pgfqpoint{0.625000in}{0.550000in}}{\pgfqpoint{3.875000in}{3.850000in}} %
\pgfusepath{clip}%
\pgfsetbuttcap%
\pgfsetroundjoin%
\pgfsetlinewidth{0.250937pt}%
\definecolor{currentstroke}{rgb}{0.000000,0.000000,0.000000}%
\pgfsetstrokecolor{currentstroke}%
\pgfsetdash{}{0pt}%
\pgfpathmoveto{\pgfqpoint{0.625000in}{2.486957in}}%
\pgfpathlineto{\pgfqpoint{0.627360in}{2.489474in}}%
\pgfpathlineto{\pgfqpoint{0.628095in}{2.499123in}}%
\pgfpathlineto{\pgfqpoint{0.631350in}{2.508772in}}%
\pgfpathlineto{\pgfqpoint{0.634712in}{2.517059in}}%
\pgfpathlineto{\pgfqpoint{0.636046in}{2.518421in}}%
\pgfpathlineto{\pgfqpoint{0.642746in}{2.528070in}}%
\pgfpathlineto{\pgfqpoint{0.644424in}{2.530842in}}%
\pgfpathlineto{\pgfqpoint{0.652003in}{2.537719in}}%
\pgfpathlineto{\pgfqpoint{0.654135in}{2.540176in}}%
\pgfpathlineto{\pgfqpoint{0.663847in}{2.546828in}}%
\pgfpathlineto{\pgfqpoint{0.665159in}{2.547368in}}%
\pgfpathlineto{\pgfqpoint{0.673559in}{2.552059in}}%
\pgfpathlineto{\pgfqpoint{0.683271in}{2.555434in}}%
\pgfpathlineto{\pgfqpoint{0.690957in}{2.557018in}}%
\pgfpathlineto{\pgfqpoint{0.692982in}{2.557564in}}%
\pgfpathlineto{\pgfqpoint{0.702694in}{2.558639in}}%
\pgfpathlineto{\pgfqpoint{0.712406in}{2.558488in}}%
\pgfpathlineto{\pgfqpoint{0.722118in}{2.557135in}}%
\pgfpathlineto{\pgfqpoint{0.722617in}{2.557018in}}%
\pgfpathlineto{\pgfqpoint{0.731830in}{2.554646in}}%
\pgfpathlineto{\pgfqpoint{0.741541in}{2.550746in}}%
\pgfpathlineto{\pgfqpoint{0.747857in}{2.547368in}}%
\pgfpathlineto{\pgfqpoint{0.751253in}{2.545216in}}%
\pgfpathlineto{\pgfqpoint{0.760929in}{2.537719in}}%
\pgfpathlineto{\pgfqpoint{0.760965in}{2.537684in}}%
\pgfpathlineto{\pgfqpoint{0.769948in}{2.528070in}}%
\pgfpathlineto{\pgfqpoint{0.770677in}{2.527024in}}%
\pgfpathlineto{\pgfqpoint{0.776537in}{2.518421in}}%
\pgfpathlineto{\pgfqpoint{0.780388in}{2.510262in}}%
\pgfpathlineto{\pgfqpoint{0.781110in}{2.508772in}}%
\pgfpathlineto{\pgfqpoint{0.784370in}{2.499123in}}%
\pgfpathlineto{\pgfqpoint{0.786210in}{2.489474in}}%
\pgfpathlineto{\pgfqpoint{0.786804in}{2.479825in}}%
\pgfpathlineto{\pgfqpoint{0.786210in}{2.470175in}}%
\pgfpathlineto{\pgfqpoint{0.784370in}{2.460526in}}%
\pgfpathlineto{\pgfqpoint{0.781110in}{2.450877in}}%
\pgfpathlineto{\pgfqpoint{0.780388in}{2.449387in}}%
\pgfpathlineto{\pgfqpoint{0.776537in}{2.441228in}}%
\pgfpathlineto{\pgfqpoint{0.770677in}{2.432625in}}%
\pgfpathlineto{\pgfqpoint{0.769948in}{2.431579in}}%
\pgfpathlineto{\pgfqpoint{0.760965in}{2.421965in}}%
\pgfpathlineto{\pgfqpoint{0.760929in}{2.421930in}}%
\pgfpathlineto{\pgfqpoint{0.751253in}{2.414433in}}%
\pgfpathlineto{\pgfqpoint{0.747857in}{2.412281in}}%
\pgfpathlineto{\pgfqpoint{0.741541in}{2.408903in}}%
\pgfpathlineto{\pgfqpoint{0.731830in}{2.405003in}}%
\pgfpathlineto{\pgfqpoint{0.722617in}{2.402632in}}%
\pgfpathlineto{\pgfqpoint{0.722118in}{2.402514in}}%
\pgfpathlineto{\pgfqpoint{0.712406in}{2.401162in}}%
\pgfpathlineto{\pgfqpoint{0.702694in}{2.401010in}}%
\pgfpathlineto{\pgfqpoint{0.692982in}{2.402085in}}%
\pgfpathlineto{\pgfqpoint{0.690957in}{2.402632in}}%
\pgfpathlineto{\pgfqpoint{0.683271in}{2.404215in}}%
\pgfpathlineto{\pgfqpoint{0.673559in}{2.407590in}}%
\pgfpathlineto{\pgfqpoint{0.665159in}{2.412281in}}%
\pgfpathlineto{\pgfqpoint{0.663847in}{2.412821in}}%
\pgfpathlineto{\pgfqpoint{0.654135in}{2.419474in}}%
\pgfpathlineto{\pgfqpoint{0.652003in}{2.421930in}}%
\pgfpathlineto{\pgfqpoint{0.644424in}{2.428807in}}%
\pgfpathlineto{\pgfqpoint{0.642746in}{2.431579in}}%
\pgfpathlineto{\pgfqpoint{0.636046in}{2.441228in}}%
\pgfpathlineto{\pgfqpoint{0.634712in}{2.442590in}}%
\pgfpathlineto{\pgfqpoint{0.631350in}{2.450877in}}%
\pgfpathlineto{\pgfqpoint{0.628102in}{2.460526in}}%
\pgfpathlineto{\pgfqpoint{0.627360in}{2.470175in}}%
\pgfpathlineto{\pgfqpoint{0.625000in}{2.472692in}}%
\pgfusepath{stroke}%
\end{pgfscope}%
\begin{pgfscope}%
\pgfpathrectangle{\pgfqpoint{0.625000in}{0.550000in}}{\pgfqpoint{3.875000in}{3.850000in}} %
\pgfusepath{clip}%
\pgfsetbuttcap%
\pgfsetroundjoin%
\pgfsetlinewidth{0.250937pt}%
\definecolor{currentstroke}{rgb}{0.000000,0.000000,0.000000}%
\pgfsetstrokecolor{currentstroke}%
\pgfsetdash{}{0pt}%
\pgfpathmoveto{\pgfqpoint{0.625000in}{2.640809in}}%
\pgfpathlineto{\pgfqpoint{0.631583in}{2.634211in}}%
\pgfpathlineto{\pgfqpoint{0.625000in}{2.628045in}}%
\pgfusepath{stroke}%
\end{pgfscope}%
\begin{pgfscope}%
\pgfpathrectangle{\pgfqpoint{0.625000in}{0.550000in}}{\pgfqpoint{3.875000in}{3.850000in}} %
\pgfusepath{clip}%
\pgfsetbuttcap%
\pgfsetroundjoin%
\pgfsetlinewidth{0.250937pt}%
\definecolor{currentstroke}{rgb}{0.000000,0.000000,0.000000}%
\pgfsetstrokecolor{currentstroke}%
\pgfsetdash{}{0pt}%
\pgfpathmoveto{\pgfqpoint{0.625000in}{2.794978in}}%
\pgfpathlineto{\pgfqpoint{0.631561in}{2.788596in}}%
\pgfpathlineto{\pgfqpoint{0.625000in}{2.782394in}}%
\pgfusepath{stroke}%
\end{pgfscope}%
\begin{pgfscope}%
\pgfpathrectangle{\pgfqpoint{0.625000in}{0.550000in}}{\pgfqpoint{3.875000in}{3.850000in}} %
\pgfusepath{clip}%
\pgfsetbuttcap%
\pgfsetroundjoin%
\pgfsetlinewidth{0.250937pt}%
\definecolor{currentstroke}{rgb}{0.000000,0.000000,0.000000}%
\pgfsetstrokecolor{currentstroke}%
\pgfsetdash{}{0pt}%
\pgfpathmoveto{\pgfqpoint{0.625000in}{2.949415in}}%
\pgfpathlineto{\pgfqpoint{0.631492in}{2.942982in}}%
\pgfpathlineto{\pgfqpoint{0.625000in}{2.936496in}}%
\pgfusepath{stroke}%
\end{pgfscope}%
\begin{pgfscope}%
\pgfpathrectangle{\pgfqpoint{0.625000in}{0.550000in}}{\pgfqpoint{3.875000in}{3.850000in}} %
\pgfusepath{clip}%
\pgfsetbuttcap%
\pgfsetroundjoin%
\pgfsetlinewidth{0.250937pt}%
\definecolor{currentstroke}{rgb}{0.000000,0.000000,0.000000}%
\pgfsetstrokecolor{currentstroke}%
\pgfsetdash{}{0pt}%
\pgfpathmoveto{\pgfqpoint{0.625000in}{3.103540in}}%
\pgfpathlineto{\pgfqpoint{0.631545in}{3.097368in}}%
\pgfpathlineto{\pgfqpoint{0.625000in}{3.091065in}}%
\pgfusepath{stroke}%
\end{pgfscope}%
\begin{pgfscope}%
\pgfpathrectangle{\pgfqpoint{0.625000in}{0.550000in}}{\pgfqpoint{3.875000in}{3.850000in}} %
\pgfusepath{clip}%
\pgfsetbuttcap%
\pgfsetroundjoin%
\pgfsetlinewidth{0.250937pt}%
\definecolor{currentstroke}{rgb}{0.000000,0.000000,0.000000}%
\pgfsetstrokecolor{currentstroke}%
\pgfsetdash{}{0pt}%
\pgfpathmoveto{\pgfqpoint{0.625000in}{3.257958in}}%
\pgfpathlineto{\pgfqpoint{0.633049in}{3.251754in}}%
\pgfpathlineto{\pgfqpoint{0.625000in}{3.245591in}}%
\pgfusepath{stroke}%
\end{pgfscope}%
\begin{pgfscope}%
\pgfpathrectangle{\pgfqpoint{0.625000in}{0.550000in}}{\pgfqpoint{3.875000in}{3.850000in}} %
\pgfusepath{clip}%
\pgfsetbuttcap%
\pgfsetroundjoin%
\pgfsetlinewidth{0.250937pt}%
\definecolor{currentstroke}{rgb}{0.000000,0.000000,0.000000}%
\pgfsetstrokecolor{currentstroke}%
\pgfsetdash{}{0pt}%
\pgfpathmoveto{\pgfqpoint{0.625000in}{3.412584in}}%
\pgfpathlineto{\pgfqpoint{0.631553in}{3.406140in}}%
\pgfpathlineto{\pgfqpoint{0.625000in}{3.399906in}}%
\pgfusepath{stroke}%
\end{pgfscope}%
\begin{pgfscope}%
\pgfpathrectangle{\pgfqpoint{0.625000in}{0.550000in}}{\pgfqpoint{3.875000in}{3.850000in}} %
\pgfusepath{clip}%
\pgfsetbuttcap%
\pgfsetroundjoin%
\pgfsetlinewidth{0.250937pt}%
\definecolor{currentstroke}{rgb}{0.000000,0.000000,0.000000}%
\pgfsetstrokecolor{currentstroke}%
\pgfsetdash{}{0pt}%
\pgfpathmoveto{\pgfqpoint{0.625000in}{3.566736in}}%
\pgfpathlineto{\pgfqpoint{0.631536in}{3.560526in}}%
\pgfpathlineto{\pgfqpoint{0.625000in}{3.554320in}}%
\pgfusepath{stroke}%
\end{pgfscope}%
\begin{pgfscope}%
\pgfpathrectangle{\pgfqpoint{0.625000in}{0.550000in}}{\pgfqpoint{3.875000in}{3.850000in}} %
\pgfusepath{clip}%
\pgfsetbuttcap%
\pgfsetroundjoin%
\pgfsetlinewidth{0.250937pt}%
\definecolor{currentstroke}{rgb}{0.000000,0.000000,0.000000}%
\pgfsetstrokecolor{currentstroke}%
\pgfsetdash{}{0pt}%
\pgfpathmoveto{\pgfqpoint{0.625000in}{3.721190in}}%
\pgfpathlineto{\pgfqpoint{0.631555in}{3.714912in}}%
\pgfpathlineto{\pgfqpoint{0.625000in}{3.708445in}}%
\pgfusepath{stroke}%
\end{pgfscope}%
\begin{pgfscope}%
\pgfpathrectangle{\pgfqpoint{0.625000in}{0.550000in}}{\pgfqpoint{3.875000in}{3.850000in}} %
\pgfusepath{clip}%
\pgfsetbuttcap%
\pgfsetroundjoin%
\pgfsetlinewidth{0.250937pt}%
\definecolor{currentstroke}{rgb}{0.000000,0.000000,0.000000}%
\pgfsetstrokecolor{currentstroke}%
\pgfsetdash{}{0pt}%
\pgfpathmoveto{\pgfqpoint{0.625000in}{3.875429in}}%
\pgfpathlineto{\pgfqpoint{0.631443in}{3.869298in}}%
\pgfpathlineto{\pgfqpoint{0.625000in}{3.863142in}}%
\pgfusepath{stroke}%
\end{pgfscope}%
\begin{pgfscope}%
\pgfpathrectangle{\pgfqpoint{0.625000in}{0.550000in}}{\pgfqpoint{3.875000in}{3.850000in}} %
\pgfusepath{clip}%
\pgfsetbuttcap%
\pgfsetroundjoin%
\pgfsetlinewidth{0.250937pt}%
\definecolor{currentstroke}{rgb}{0.000000,0.000000,0.000000}%
\pgfsetstrokecolor{currentstroke}%
\pgfsetdash{}{0pt}%
\pgfpathmoveto{\pgfqpoint{0.625000in}{4.029978in}}%
\pgfpathlineto{\pgfqpoint{0.633084in}{4.023684in}}%
\pgfpathlineto{\pgfqpoint{0.625000in}{4.017451in}}%
\pgfusepath{stroke}%
\end{pgfscope}%
\begin{pgfscope}%
\pgfpathrectangle{\pgfqpoint{0.625000in}{0.550000in}}{\pgfqpoint{3.875000in}{3.850000in}} %
\pgfusepath{clip}%
\pgfsetbuttcap%
\pgfsetroundjoin%
\pgfsetlinewidth{0.250937pt}%
\definecolor{currentstroke}{rgb}{0.000000,0.000000,0.000000}%
\pgfsetstrokecolor{currentstroke}%
\pgfsetdash{}{0pt}%
\pgfpathmoveto{\pgfqpoint{0.625000in}{4.184252in}}%
\pgfpathlineto{\pgfqpoint{0.631421in}{4.178070in}}%
\pgfpathlineto{\pgfqpoint{0.625000in}{4.172178in}}%
\pgfusepath{stroke}%
\end{pgfscope}%
\begin{pgfscope}%
\pgfpathrectangle{\pgfqpoint{0.625000in}{0.550000in}}{\pgfqpoint{3.875000in}{3.850000in}} %
\pgfusepath{clip}%
\pgfsetbuttcap%
\pgfsetroundjoin%
\pgfsetlinewidth{0.250937pt}%
\definecolor{currentstroke}{rgb}{0.000000,0.000000,0.000000}%
\pgfsetstrokecolor{currentstroke}%
\pgfsetdash{}{0pt}%
\pgfpathmoveto{\pgfqpoint{0.625000in}{4.338623in}}%
\pgfpathlineto{\pgfqpoint{0.631406in}{4.332456in}}%
\pgfpathlineto{\pgfqpoint{0.625000in}{4.326269in}}%
\pgfusepath{stroke}%
\end{pgfscope}%
\begin{pgfscope}%
\pgfpathrectangle{\pgfqpoint{0.625000in}{0.550000in}}{\pgfqpoint{3.875000in}{3.850000in}} %
\pgfusepath{clip}%
\pgfsetbuttcap%
\pgfsetroundjoin%
\pgfsetlinewidth{0.250937pt}%
\definecolor{currentstroke}{rgb}{0.000000,0.000000,0.000000}%
\pgfsetstrokecolor{currentstroke}%
\pgfsetdash{}{0pt}%
\pgfpathmoveto{\pgfqpoint{0.634712in}{1.184328in}}%
\pgfpathlineto{\pgfqpoint{0.633148in}{1.186842in}}%
\pgfpathlineto{\pgfqpoint{0.630912in}{1.196491in}}%
\pgfpathlineto{\pgfqpoint{0.634712in}{1.202201in}}%
\pgfpathlineto{\pgfqpoint{0.642381in}{1.196491in}}%
\pgfpathlineto{\pgfqpoint{0.639341in}{1.186842in}}%
\pgfpathlineto{\pgfqpoint{0.634712in}{1.184328in}}%
\pgfusepath{stroke}%
\end{pgfscope}%
\begin{pgfscope}%
\pgfpathrectangle{\pgfqpoint{0.625000in}{0.550000in}}{\pgfqpoint{3.875000in}{3.850000in}} %
\pgfusepath{clip}%
\pgfsetbuttcap%
\pgfsetroundjoin%
\pgfsetlinewidth{0.250937pt}%
\definecolor{currentstroke}{rgb}{0.000000,0.000000,0.000000}%
\pgfsetstrokecolor{currentstroke}%
\pgfsetdash{}{0pt}%
\pgfpathmoveto{\pgfqpoint{0.634712in}{3.757448in}}%
\pgfpathlineto{\pgfqpoint{0.630912in}{3.763158in}}%
\pgfpathlineto{\pgfqpoint{0.633148in}{3.772807in}}%
\pgfpathlineto{\pgfqpoint{0.634712in}{3.775321in}}%
\pgfpathlineto{\pgfqpoint{0.639341in}{3.772807in}}%
\pgfpathlineto{\pgfqpoint{0.642381in}{3.763158in}}%
\pgfpathlineto{\pgfqpoint{0.634712in}{3.757448in}}%
\pgfusepath{stroke}%
\end{pgfscope}%
\begin{pgfscope}%
\pgfpathrectangle{\pgfqpoint{0.625000in}{0.550000in}}{\pgfqpoint{3.875000in}{3.850000in}} %
\pgfusepath{clip}%
\pgfsetbuttcap%
\pgfsetroundjoin%
\pgfsetlinewidth{0.250937pt}%
\definecolor{currentstroke}{rgb}{0.000000,0.000000,0.000000}%
\pgfsetstrokecolor{currentstroke}%
\pgfsetdash{}{0pt}%
\pgfpathmoveto{\pgfqpoint{0.625000in}{0.633914in}}%
\pgfpathlineto{\pgfqpoint{0.631932in}{0.627193in}}%
\pgfpathlineto{\pgfqpoint{0.625000in}{0.620492in}}%
\pgfusepath{stroke}%
\end{pgfscope}%
\begin{pgfscope}%
\pgfpathrectangle{\pgfqpoint{0.625000in}{0.550000in}}{\pgfqpoint{3.875000in}{3.850000in}} %
\pgfusepath{clip}%
\pgfsetbuttcap%
\pgfsetroundjoin%
\pgfsetlinewidth{0.250937pt}%
\definecolor{currentstroke}{rgb}{0.000000,0.000000,0.000000}%
\pgfsetstrokecolor{currentstroke}%
\pgfsetdash{}{0pt}%
\pgfpathmoveto{\pgfqpoint{0.625000in}{0.787884in}}%
\pgfpathlineto{\pgfqpoint{0.631821in}{0.781579in}}%
\pgfpathlineto{\pgfqpoint{0.625000in}{0.774993in}}%
\pgfusepath{stroke}%
\end{pgfscope}%
\begin{pgfscope}%
\pgfpathrectangle{\pgfqpoint{0.625000in}{0.550000in}}{\pgfqpoint{3.875000in}{3.850000in}} %
\pgfusepath{clip}%
\pgfsetbuttcap%
\pgfsetroundjoin%
\pgfsetlinewidth{0.250937pt}%
\definecolor{currentstroke}{rgb}{0.000000,0.000000,0.000000}%
\pgfsetstrokecolor{currentstroke}%
\pgfsetdash{}{0pt}%
\pgfpathmoveto{\pgfqpoint{0.625000in}{0.942564in}}%
\pgfpathlineto{\pgfqpoint{0.633354in}{0.935965in}}%
\pgfpathlineto{\pgfqpoint{0.625000in}{0.929306in}}%
\pgfusepath{stroke}%
\end{pgfscope}%
\begin{pgfscope}%
\pgfpathrectangle{\pgfqpoint{0.625000in}{0.550000in}}{\pgfqpoint{3.875000in}{3.850000in}} %
\pgfusepath{clip}%
\pgfsetbuttcap%
\pgfsetroundjoin%
\pgfsetlinewidth{0.250937pt}%
\definecolor{currentstroke}{rgb}{0.000000,0.000000,0.000000}%
\pgfsetstrokecolor{currentstroke}%
\pgfsetdash{}{0pt}%
\pgfpathmoveto{\pgfqpoint{0.625000in}{1.096906in}}%
\pgfpathlineto{\pgfqpoint{0.631836in}{1.090351in}}%
\pgfpathlineto{\pgfqpoint{0.625000in}{1.083820in}}%
\pgfusepath{stroke}%
\end{pgfscope}%
\begin{pgfscope}%
\pgfpathrectangle{\pgfqpoint{0.625000in}{0.550000in}}{\pgfqpoint{3.875000in}{3.850000in}} %
\pgfusepath{clip}%
\pgfsetbuttcap%
\pgfsetroundjoin%
\pgfsetlinewidth{0.250937pt}%
\definecolor{currentstroke}{rgb}{0.000000,0.000000,0.000000}%
\pgfsetstrokecolor{currentstroke}%
\pgfsetdash{}{0pt}%
\pgfpathmoveto{\pgfqpoint{0.625000in}{1.251475in}}%
\pgfpathlineto{\pgfqpoint{0.631827in}{1.244737in}}%
\pgfpathlineto{\pgfqpoint{0.625000in}{1.238185in}}%
\pgfusepath{stroke}%
\end{pgfscope}%
\begin{pgfscope}%
\pgfpathrectangle{\pgfqpoint{0.625000in}{0.550000in}}{\pgfqpoint{3.875000in}{3.850000in}} %
\pgfusepath{clip}%
\pgfsetbuttcap%
\pgfsetroundjoin%
\pgfsetlinewidth{0.250937pt}%
\definecolor{currentstroke}{rgb}{0.000000,0.000000,0.000000}%
\pgfsetstrokecolor{currentstroke}%
\pgfsetdash{}{0pt}%
\pgfpathmoveto{\pgfqpoint{0.625000in}{1.405542in}}%
\pgfpathlineto{\pgfqpoint{0.631748in}{1.399123in}}%
\pgfpathlineto{\pgfqpoint{0.625000in}{1.392700in}}%
\pgfusepath{stroke}%
\end{pgfscope}%
\begin{pgfscope}%
\pgfpathrectangle{\pgfqpoint{0.625000in}{0.550000in}}{\pgfqpoint{3.875000in}{3.850000in}} %
\pgfusepath{clip}%
\pgfsetbuttcap%
\pgfsetroundjoin%
\pgfsetlinewidth{0.250937pt}%
\definecolor{currentstroke}{rgb}{0.000000,0.000000,0.000000}%
\pgfsetstrokecolor{currentstroke}%
\pgfsetdash{}{0pt}%
\pgfpathmoveto{\pgfqpoint{0.625000in}{1.559933in}}%
\pgfpathlineto{\pgfqpoint{0.631743in}{1.553509in}}%
\pgfpathlineto{\pgfqpoint{0.625000in}{1.546876in}}%
\pgfusepath{stroke}%
\end{pgfscope}%
\begin{pgfscope}%
\pgfpathrectangle{\pgfqpoint{0.625000in}{0.550000in}}{\pgfqpoint{3.875000in}{3.850000in}} %
\pgfusepath{clip}%
\pgfsetbuttcap%
\pgfsetroundjoin%
\pgfsetlinewidth{0.250937pt}%
\definecolor{currentstroke}{rgb}{0.000000,0.000000,0.000000}%
\pgfsetstrokecolor{currentstroke}%
\pgfsetdash{}{0pt}%
\pgfpathmoveto{\pgfqpoint{0.625000in}{1.714298in}}%
\pgfpathlineto{\pgfqpoint{0.633250in}{1.707895in}}%
\pgfpathlineto{\pgfqpoint{0.625000in}{1.701451in}}%
\pgfusepath{stroke}%
\end{pgfscope}%
\begin{pgfscope}%
\pgfpathrectangle{\pgfqpoint{0.625000in}{0.550000in}}{\pgfqpoint{3.875000in}{3.850000in}} %
\pgfusepath{clip}%
\pgfsetbuttcap%
\pgfsetroundjoin%
\pgfsetlinewidth{0.250937pt}%
\definecolor{currentstroke}{rgb}{0.000000,0.000000,0.000000}%
\pgfsetstrokecolor{currentstroke}%
\pgfsetdash{}{0pt}%
\pgfpathmoveto{\pgfqpoint{0.625000in}{1.868711in}}%
\pgfpathlineto{\pgfqpoint{0.631674in}{1.862281in}}%
\pgfpathlineto{\pgfqpoint{0.625000in}{1.855983in}}%
\pgfusepath{stroke}%
\end{pgfscope}%
\begin{pgfscope}%
\pgfpathrectangle{\pgfqpoint{0.625000in}{0.550000in}}{\pgfqpoint{3.875000in}{3.850000in}} %
\pgfusepath{clip}%
\pgfsetbuttcap%
\pgfsetroundjoin%
\pgfsetlinewidth{0.250937pt}%
\definecolor{currentstroke}{rgb}{0.000000,0.000000,0.000000}%
\pgfsetstrokecolor{currentstroke}%
\pgfsetdash{}{0pt}%
\pgfpathmoveto{\pgfqpoint{0.625000in}{2.023295in}}%
\pgfpathlineto{\pgfqpoint{0.631634in}{2.016667in}}%
\pgfpathlineto{\pgfqpoint{0.625000in}{2.010093in}}%
\pgfusepath{stroke}%
\end{pgfscope}%
\begin{pgfscope}%
\pgfpathrectangle{\pgfqpoint{0.625000in}{0.550000in}}{\pgfqpoint{3.875000in}{3.850000in}} %
\pgfusepath{clip}%
\pgfsetbuttcap%
\pgfsetroundjoin%
\pgfsetlinewidth{0.250937pt}%
\definecolor{currentstroke}{rgb}{0.000000,0.000000,0.000000}%
\pgfsetstrokecolor{currentstroke}%
\pgfsetdash{}{0pt}%
\pgfpathmoveto{\pgfqpoint{0.625000in}{2.177358in}}%
\pgfpathlineto{\pgfqpoint{0.631668in}{2.171053in}}%
\pgfpathlineto{\pgfqpoint{0.625000in}{2.164567in}}%
\pgfusepath{stroke}%
\end{pgfscope}%
\begin{pgfscope}%
\pgfpathrectangle{\pgfqpoint{0.625000in}{0.550000in}}{\pgfqpoint{3.875000in}{3.850000in}} %
\pgfusepath{clip}%
\pgfsetbuttcap%
\pgfsetroundjoin%
\pgfsetlinewidth{0.250937pt}%
\definecolor{currentstroke}{rgb}{0.000000,0.000000,0.000000}%
\pgfsetstrokecolor{currentstroke}%
\pgfsetdash{}{0pt}%
\pgfpathmoveto{\pgfqpoint{0.625000in}{2.331693in}}%
\pgfpathlineto{\pgfqpoint{0.631677in}{2.325439in}}%
\pgfpathlineto{\pgfqpoint{0.625000in}{2.318746in}}%
\pgfusepath{stroke}%
\end{pgfscope}%
\begin{pgfscope}%
\pgfpathrectangle{\pgfqpoint{0.625000in}{0.550000in}}{\pgfqpoint{3.875000in}{3.850000in}} %
\pgfusepath{clip}%
\pgfsetbuttcap%
\pgfsetroundjoin%
\pgfsetlinewidth{0.250937pt}%
\definecolor{currentstroke}{rgb}{0.000000,0.000000,0.000000}%
\pgfsetstrokecolor{currentstroke}%
\pgfsetdash{}{0pt}%
\pgfpathmoveto{\pgfqpoint{0.625000in}{2.487041in}}%
\pgfpathlineto{\pgfqpoint{0.627282in}{2.489474in}}%
\pgfpathlineto{\pgfqpoint{0.628016in}{2.499123in}}%
\pgfpathlineto{\pgfqpoint{0.631195in}{2.508772in}}%
\pgfpathlineto{\pgfqpoint{0.634712in}{2.517440in}}%
\pgfpathlineto{\pgfqpoint{0.635673in}{2.518421in}}%
\pgfpathlineto{\pgfqpoint{0.642242in}{2.528070in}}%
\pgfpathlineto{\pgfqpoint{0.644424in}{2.531674in}}%
\pgfpathlineto{\pgfqpoint{0.651086in}{2.537719in}}%
\pgfpathlineto{\pgfqpoint{0.654135in}{2.541232in}}%
\pgfpathlineto{\pgfqpoint{0.663119in}{2.547368in}}%
\pgfpathlineto{\pgfqpoint{0.663847in}{2.548014in}}%
\pgfpathlineto{\pgfqpoint{0.673559in}{2.553444in}}%
\pgfpathlineto{\pgfqpoint{0.683271in}{2.556921in}}%
\pgfpathlineto{\pgfqpoint{0.683739in}{2.557018in}}%
\pgfpathlineto{\pgfqpoint{0.692982in}{2.559514in}}%
\pgfpathlineto{\pgfqpoint{0.702694in}{2.560793in}}%
\pgfpathlineto{\pgfqpoint{0.712406in}{2.560911in}}%
\pgfpathlineto{\pgfqpoint{0.722118in}{2.559892in}}%
\pgfpathlineto{\pgfqpoint{0.731830in}{2.557682in}}%
\pgfpathlineto{\pgfqpoint{0.733889in}{2.557018in}}%
\pgfpathlineto{\pgfqpoint{0.741541in}{2.554264in}}%
\pgfpathlineto{\pgfqpoint{0.751253in}{2.549359in}}%
\pgfpathlineto{\pgfqpoint{0.754498in}{2.547368in}}%
\pgfpathlineto{\pgfqpoint{0.760965in}{2.542616in}}%
\pgfpathlineto{\pgfqpoint{0.766681in}{2.537719in}}%
\pgfpathlineto{\pgfqpoint{0.770677in}{2.533384in}}%
\pgfpathlineto{\pgfqpoint{0.775255in}{2.528070in}}%
\pgfpathlineto{\pgfqpoint{0.780388in}{2.520068in}}%
\pgfpathlineto{\pgfqpoint{0.781434in}{2.518421in}}%
\pgfpathlineto{\pgfqpoint{0.786027in}{2.508772in}}%
\pgfpathlineto{\pgfqpoint{0.789015in}{2.499123in}}%
\pgfpathlineto{\pgfqpoint{0.790100in}{2.492975in}}%
\pgfpathlineto{\pgfqpoint{0.790774in}{2.489474in}}%
\pgfpathlineto{\pgfqpoint{0.791388in}{2.479825in}}%
\pgfpathlineto{\pgfqpoint{0.790774in}{2.470175in}}%
\pgfpathlineto{\pgfqpoint{0.790100in}{2.466674in}}%
\pgfpathlineto{\pgfqpoint{0.789015in}{2.460526in}}%
\pgfpathlineto{\pgfqpoint{0.786027in}{2.450877in}}%
\pgfpathlineto{\pgfqpoint{0.781434in}{2.441228in}}%
\pgfpathlineto{\pgfqpoint{0.780388in}{2.439581in}}%
\pgfpathlineto{\pgfqpoint{0.775255in}{2.431579in}}%
\pgfpathlineto{\pgfqpoint{0.770677in}{2.426265in}}%
\pgfpathlineto{\pgfqpoint{0.766681in}{2.421930in}}%
\pgfpathlineto{\pgfqpoint{0.760965in}{2.417033in}}%
\pgfpathlineto{\pgfqpoint{0.754498in}{2.412281in}}%
\pgfpathlineto{\pgfqpoint{0.751253in}{2.410290in}}%
\pgfpathlineto{\pgfqpoint{0.741541in}{2.405385in}}%
\pgfpathlineto{\pgfqpoint{0.733889in}{2.402632in}}%
\pgfpathlineto{\pgfqpoint{0.731830in}{2.401967in}}%
\pgfpathlineto{\pgfqpoint{0.722118in}{2.399757in}}%
\pgfpathlineto{\pgfqpoint{0.712406in}{2.398738in}}%
\pgfpathlineto{\pgfqpoint{0.702694in}{2.398856in}}%
\pgfpathlineto{\pgfqpoint{0.692982in}{2.400135in}}%
\pgfpathlineto{\pgfqpoint{0.683739in}{2.402632in}}%
\pgfpathlineto{\pgfqpoint{0.683271in}{2.402728in}}%
\pgfpathlineto{\pgfqpoint{0.673559in}{2.406205in}}%
\pgfpathlineto{\pgfqpoint{0.663847in}{2.411635in}}%
\pgfpathlineto{\pgfqpoint{0.663119in}{2.412281in}}%
\pgfpathlineto{\pgfqpoint{0.654135in}{2.418417in}}%
\pgfpathlineto{\pgfqpoint{0.651086in}{2.421930in}}%
\pgfpathlineto{\pgfqpoint{0.644424in}{2.427975in}}%
\pgfpathlineto{\pgfqpoint{0.642242in}{2.431579in}}%
\pgfpathlineto{\pgfqpoint{0.635673in}{2.441228in}}%
\pgfpathlineto{\pgfqpoint{0.634712in}{2.442209in}}%
\pgfpathlineto{\pgfqpoint{0.631195in}{2.450877in}}%
\pgfpathlineto{\pgfqpoint{0.628023in}{2.460526in}}%
\pgfpathlineto{\pgfqpoint{0.627282in}{2.470175in}}%
\pgfpathlineto{\pgfqpoint{0.625000in}{2.472608in}}%
\pgfusepath{stroke}%
\end{pgfscope}%
\begin{pgfscope}%
\pgfpathrectangle{\pgfqpoint{0.625000in}{0.550000in}}{\pgfqpoint{3.875000in}{3.850000in}} %
\pgfusepath{clip}%
\pgfsetbuttcap%
\pgfsetroundjoin%
\pgfsetlinewidth{0.250937pt}%
\definecolor{currentstroke}{rgb}{0.000000,0.000000,0.000000}%
\pgfsetstrokecolor{currentstroke}%
\pgfsetdash{}{0pt}%
\pgfpathmoveto{\pgfqpoint{0.625000in}{2.640898in}}%
\pgfpathlineto{\pgfqpoint{0.631672in}{2.634211in}}%
\pgfpathlineto{\pgfqpoint{0.625000in}{2.627962in}}%
\pgfusepath{stroke}%
\end{pgfscope}%
\begin{pgfscope}%
\pgfpathrectangle{\pgfqpoint{0.625000in}{0.550000in}}{\pgfqpoint{3.875000in}{3.850000in}} %
\pgfusepath{clip}%
\pgfsetbuttcap%
\pgfsetroundjoin%
\pgfsetlinewidth{0.250937pt}%
\definecolor{currentstroke}{rgb}{0.000000,0.000000,0.000000}%
\pgfsetstrokecolor{currentstroke}%
\pgfsetdash{}{0pt}%
\pgfpathmoveto{\pgfqpoint{0.625000in}{2.795063in}}%
\pgfpathlineto{\pgfqpoint{0.631649in}{2.788596in}}%
\pgfpathlineto{\pgfqpoint{0.625000in}{2.782311in}}%
\pgfusepath{stroke}%
\end{pgfscope}%
\begin{pgfscope}%
\pgfpathrectangle{\pgfqpoint{0.625000in}{0.550000in}}{\pgfqpoint{3.875000in}{3.850000in}} %
\pgfusepath{clip}%
\pgfsetbuttcap%
\pgfsetroundjoin%
\pgfsetlinewidth{0.250937pt}%
\definecolor{currentstroke}{rgb}{0.000000,0.000000,0.000000}%
\pgfsetstrokecolor{currentstroke}%
\pgfsetdash{}{0pt}%
\pgfpathmoveto{\pgfqpoint{0.625000in}{2.949503in}}%
\pgfpathlineto{\pgfqpoint{0.631581in}{2.942982in}}%
\pgfpathlineto{\pgfqpoint{0.625000in}{2.936407in}}%
\pgfusepath{stroke}%
\end{pgfscope}%
\begin{pgfscope}%
\pgfpathrectangle{\pgfqpoint{0.625000in}{0.550000in}}{\pgfqpoint{3.875000in}{3.850000in}} %
\pgfusepath{clip}%
\pgfsetbuttcap%
\pgfsetroundjoin%
\pgfsetlinewidth{0.250937pt}%
\definecolor{currentstroke}{rgb}{0.000000,0.000000,0.000000}%
\pgfsetstrokecolor{currentstroke}%
\pgfsetdash{}{0pt}%
\pgfpathmoveto{\pgfqpoint{0.625000in}{3.103622in}}%
\pgfpathlineto{\pgfqpoint{0.631632in}{3.097368in}}%
\pgfpathlineto{\pgfqpoint{0.625000in}{3.090981in}}%
\pgfusepath{stroke}%
\end{pgfscope}%
\begin{pgfscope}%
\pgfpathrectangle{\pgfqpoint{0.625000in}{0.550000in}}{\pgfqpoint{3.875000in}{3.850000in}} %
\pgfusepath{clip}%
\pgfsetbuttcap%
\pgfsetroundjoin%
\pgfsetlinewidth{0.250937pt}%
\definecolor{currentstroke}{rgb}{0.000000,0.000000,0.000000}%
\pgfsetstrokecolor{currentstroke}%
\pgfsetdash{}{0pt}%
\pgfpathmoveto{\pgfqpoint{0.625000in}{3.258044in}}%
\pgfpathlineto{\pgfqpoint{0.633160in}{3.251754in}}%
\pgfpathlineto{\pgfqpoint{0.625000in}{3.245506in}}%
\pgfusepath{stroke}%
\end{pgfscope}%
\begin{pgfscope}%
\pgfpathrectangle{\pgfqpoint{0.625000in}{0.550000in}}{\pgfqpoint{3.875000in}{3.850000in}} %
\pgfusepath{clip}%
\pgfsetbuttcap%
\pgfsetroundjoin%
\pgfsetlinewidth{0.250937pt}%
\definecolor{currentstroke}{rgb}{0.000000,0.000000,0.000000}%
\pgfsetstrokecolor{currentstroke}%
\pgfsetdash{}{0pt}%
\pgfpathmoveto{\pgfqpoint{0.625000in}{3.412669in}}%
\pgfpathlineto{\pgfqpoint{0.631639in}{3.406140in}}%
\pgfpathlineto{\pgfqpoint{0.625000in}{3.399824in}}%
\pgfusepath{stroke}%
\end{pgfscope}%
\begin{pgfscope}%
\pgfpathrectangle{\pgfqpoint{0.625000in}{0.550000in}}{\pgfqpoint{3.875000in}{3.850000in}} %
\pgfusepath{clip}%
\pgfsetbuttcap%
\pgfsetroundjoin%
\pgfsetlinewidth{0.250937pt}%
\definecolor{currentstroke}{rgb}{0.000000,0.000000,0.000000}%
\pgfsetstrokecolor{currentstroke}%
\pgfsetdash{}{0pt}%
\pgfpathmoveto{\pgfqpoint{0.625000in}{3.566821in}}%
\pgfpathlineto{\pgfqpoint{0.631625in}{3.560526in}}%
\pgfpathlineto{\pgfqpoint{0.625000in}{3.554235in}}%
\pgfusepath{stroke}%
\end{pgfscope}%
\begin{pgfscope}%
\pgfpathrectangle{\pgfqpoint{0.625000in}{0.550000in}}{\pgfqpoint{3.875000in}{3.850000in}} %
\pgfusepath{clip}%
\pgfsetbuttcap%
\pgfsetroundjoin%
\pgfsetlinewidth{0.250937pt}%
\definecolor{currentstroke}{rgb}{0.000000,0.000000,0.000000}%
\pgfsetstrokecolor{currentstroke}%
\pgfsetdash{}{0pt}%
\pgfpathmoveto{\pgfqpoint{0.625000in}{3.721276in}}%
\pgfpathlineto{\pgfqpoint{0.631645in}{3.714912in}}%
\pgfpathlineto{\pgfqpoint{0.625000in}{3.708357in}}%
\pgfusepath{stroke}%
\end{pgfscope}%
\begin{pgfscope}%
\pgfpathrectangle{\pgfqpoint{0.625000in}{0.550000in}}{\pgfqpoint{3.875000in}{3.850000in}} %
\pgfusepath{clip}%
\pgfsetbuttcap%
\pgfsetroundjoin%
\pgfsetlinewidth{0.250937pt}%
\definecolor{currentstroke}{rgb}{0.000000,0.000000,0.000000}%
\pgfsetstrokecolor{currentstroke}%
\pgfsetdash{}{0pt}%
\pgfpathmoveto{\pgfqpoint{0.625000in}{3.875516in}}%
\pgfpathlineto{\pgfqpoint{0.631534in}{3.869298in}}%
\pgfpathlineto{\pgfqpoint{0.625000in}{3.863055in}}%
\pgfusepath{stroke}%
\end{pgfscope}%
\begin{pgfscope}%
\pgfpathrectangle{\pgfqpoint{0.625000in}{0.550000in}}{\pgfqpoint{3.875000in}{3.850000in}} %
\pgfusepath{clip}%
\pgfsetbuttcap%
\pgfsetroundjoin%
\pgfsetlinewidth{0.250937pt}%
\definecolor{currentstroke}{rgb}{0.000000,0.000000,0.000000}%
\pgfsetstrokecolor{currentstroke}%
\pgfsetdash{}{0pt}%
\pgfpathmoveto{\pgfqpoint{0.625000in}{4.030063in}}%
\pgfpathlineto{\pgfqpoint{0.633192in}{4.023684in}}%
\pgfpathlineto{\pgfqpoint{0.625000in}{4.017368in}}%
\pgfusepath{stroke}%
\end{pgfscope}%
\begin{pgfscope}%
\pgfpathrectangle{\pgfqpoint{0.625000in}{0.550000in}}{\pgfqpoint{3.875000in}{3.850000in}} %
\pgfusepath{clip}%
\pgfsetbuttcap%
\pgfsetroundjoin%
\pgfsetlinewidth{0.250937pt}%
\definecolor{currentstroke}{rgb}{0.000000,0.000000,0.000000}%
\pgfsetstrokecolor{currentstroke}%
\pgfsetdash{}{0pt}%
\pgfpathmoveto{\pgfqpoint{0.625000in}{4.184340in}}%
\pgfpathlineto{\pgfqpoint{0.631514in}{4.178070in}}%
\pgfpathlineto{\pgfqpoint{0.625000in}{4.172094in}}%
\pgfusepath{stroke}%
\end{pgfscope}%
\begin{pgfscope}%
\pgfpathrectangle{\pgfqpoint{0.625000in}{0.550000in}}{\pgfqpoint{3.875000in}{3.850000in}} %
\pgfusepath{clip}%
\pgfsetbuttcap%
\pgfsetroundjoin%
\pgfsetlinewidth{0.250937pt}%
\definecolor{currentstroke}{rgb}{0.000000,0.000000,0.000000}%
\pgfsetstrokecolor{currentstroke}%
\pgfsetdash{}{0pt}%
\pgfpathmoveto{\pgfqpoint{0.625000in}{4.338710in}}%
\pgfpathlineto{\pgfqpoint{0.631496in}{4.332456in}}%
\pgfpathlineto{\pgfqpoint{0.625000in}{4.326181in}}%
\pgfusepath{stroke}%
\end{pgfscope}%
\begin{pgfscope}%
\pgfpathrectangle{\pgfqpoint{0.625000in}{0.550000in}}{\pgfqpoint{3.875000in}{3.850000in}} %
\pgfusepath{clip}%
\pgfsetbuttcap%
\pgfsetroundjoin%
\pgfsetlinewidth{0.250937pt}%
\definecolor{currentstroke}{rgb}{0.000000,0.000000,0.000000}%
\pgfsetstrokecolor{currentstroke}%
\pgfsetdash{}{0pt}%
\pgfpathmoveto{\pgfqpoint{0.634712in}{1.183994in}}%
\pgfpathlineto{\pgfqpoint{0.632940in}{1.186842in}}%
\pgfpathlineto{\pgfqpoint{0.630737in}{1.196491in}}%
\pgfpathlineto{\pgfqpoint{0.634712in}{1.202463in}}%
\pgfpathlineto{\pgfqpoint{0.642734in}{1.196491in}}%
\pgfpathlineto{\pgfqpoint{0.639956in}{1.186842in}}%
\pgfpathlineto{\pgfqpoint{0.634712in}{1.183994in}}%
\pgfusepath{stroke}%
\end{pgfscope}%
\begin{pgfscope}%
\pgfpathrectangle{\pgfqpoint{0.625000in}{0.550000in}}{\pgfqpoint{3.875000in}{3.850000in}} %
\pgfusepath{clip}%
\pgfsetbuttcap%
\pgfsetroundjoin%
\pgfsetlinewidth{0.250937pt}%
\definecolor{currentstroke}{rgb}{0.000000,0.000000,0.000000}%
\pgfsetstrokecolor{currentstroke}%
\pgfsetdash{}{0pt}%
\pgfpathmoveto{\pgfqpoint{0.634712in}{3.757186in}}%
\pgfpathlineto{\pgfqpoint{0.630737in}{3.763158in}}%
\pgfpathlineto{\pgfqpoint{0.632940in}{3.772807in}}%
\pgfpathlineto{\pgfqpoint{0.634712in}{3.775655in}}%
\pgfpathlineto{\pgfqpoint{0.639956in}{3.772807in}}%
\pgfpathlineto{\pgfqpoint{0.642734in}{3.763158in}}%
\pgfpathlineto{\pgfqpoint{0.634712in}{3.757186in}}%
\pgfusepath{stroke}%
\end{pgfscope}%
\begin{pgfscope}%
\pgfpathrectangle{\pgfqpoint{0.625000in}{0.550000in}}{\pgfqpoint{3.875000in}{3.850000in}} %
\pgfusepath{clip}%
\pgfsetbuttcap%
\pgfsetroundjoin%
\pgfsetlinewidth{0.250937pt}%
\definecolor{currentstroke}{rgb}{0.000000,0.000000,0.000000}%
\pgfsetstrokecolor{currentstroke}%
\pgfsetdash{}{0pt}%
\pgfpathmoveto{\pgfqpoint{0.625000in}{0.633990in}}%
\pgfpathlineto{\pgfqpoint{0.632010in}{0.627193in}}%
\pgfpathlineto{\pgfqpoint{0.625000in}{0.620416in}}%
\pgfusepath{stroke}%
\end{pgfscope}%
\begin{pgfscope}%
\pgfpathrectangle{\pgfqpoint{0.625000in}{0.550000in}}{\pgfqpoint{3.875000in}{3.850000in}} %
\pgfusepath{clip}%
\pgfsetbuttcap%
\pgfsetroundjoin%
\pgfsetlinewidth{0.250937pt}%
\definecolor{currentstroke}{rgb}{0.000000,0.000000,0.000000}%
\pgfsetstrokecolor{currentstroke}%
\pgfsetdash{}{0pt}%
\pgfpathmoveto{\pgfqpoint{0.625000in}{0.787961in}}%
\pgfpathlineto{\pgfqpoint{0.631904in}{0.781579in}}%
\pgfpathlineto{\pgfqpoint{0.625000in}{0.774913in}}%
\pgfusepath{stroke}%
\end{pgfscope}%
\begin{pgfscope}%
\pgfpathrectangle{\pgfqpoint{0.625000in}{0.550000in}}{\pgfqpoint{3.875000in}{3.850000in}} %
\pgfusepath{clip}%
\pgfsetbuttcap%
\pgfsetroundjoin%
\pgfsetlinewidth{0.250937pt}%
\definecolor{currentstroke}{rgb}{0.000000,0.000000,0.000000}%
\pgfsetstrokecolor{currentstroke}%
\pgfsetdash{}{0pt}%
\pgfpathmoveto{\pgfqpoint{0.625000in}{0.942641in}}%
\pgfpathlineto{\pgfqpoint{0.633451in}{0.935965in}}%
\pgfpathlineto{\pgfqpoint{0.625000in}{0.929229in}}%
\pgfusepath{stroke}%
\end{pgfscope}%
\begin{pgfscope}%
\pgfpathrectangle{\pgfqpoint{0.625000in}{0.550000in}}{\pgfqpoint{3.875000in}{3.850000in}} %
\pgfusepath{clip}%
\pgfsetbuttcap%
\pgfsetroundjoin%
\pgfsetlinewidth{0.250937pt}%
\definecolor{currentstroke}{rgb}{0.000000,0.000000,0.000000}%
\pgfsetstrokecolor{currentstroke}%
\pgfsetdash{}{0pt}%
\pgfpathmoveto{\pgfqpoint{0.625000in}{1.096985in}}%
\pgfpathlineto{\pgfqpoint{0.631918in}{1.090351in}}%
\pgfpathlineto{\pgfqpoint{0.625000in}{1.083741in}}%
\pgfusepath{stroke}%
\end{pgfscope}%
\begin{pgfscope}%
\pgfpathrectangle{\pgfqpoint{0.625000in}{0.550000in}}{\pgfqpoint{3.875000in}{3.850000in}} %
\pgfusepath{clip}%
\pgfsetbuttcap%
\pgfsetroundjoin%
\pgfsetlinewidth{0.250937pt}%
\definecolor{currentstroke}{rgb}{0.000000,0.000000,0.000000}%
\pgfsetstrokecolor{currentstroke}%
\pgfsetdash{}{0pt}%
\pgfpathmoveto{\pgfqpoint{0.625000in}{1.251557in}}%
\pgfpathlineto{\pgfqpoint{0.631911in}{1.244737in}}%
\pgfpathlineto{\pgfqpoint{0.625000in}{1.238104in}}%
\pgfusepath{stroke}%
\end{pgfscope}%
\begin{pgfscope}%
\pgfpathrectangle{\pgfqpoint{0.625000in}{0.550000in}}{\pgfqpoint{3.875000in}{3.850000in}} %
\pgfusepath{clip}%
\pgfsetbuttcap%
\pgfsetroundjoin%
\pgfsetlinewidth{0.250937pt}%
\definecolor{currentstroke}{rgb}{0.000000,0.000000,0.000000}%
\pgfsetstrokecolor{currentstroke}%
\pgfsetdash{}{0pt}%
\pgfpathmoveto{\pgfqpoint{0.625000in}{1.405623in}}%
\pgfpathlineto{\pgfqpoint{0.631834in}{1.399123in}}%
\pgfpathlineto{\pgfqpoint{0.625000in}{1.392619in}}%
\pgfusepath{stroke}%
\end{pgfscope}%
\begin{pgfscope}%
\pgfpathrectangle{\pgfqpoint{0.625000in}{0.550000in}}{\pgfqpoint{3.875000in}{3.850000in}} %
\pgfusepath{clip}%
\pgfsetbuttcap%
\pgfsetroundjoin%
\pgfsetlinewidth{0.250937pt}%
\definecolor{currentstroke}{rgb}{0.000000,0.000000,0.000000}%
\pgfsetstrokecolor{currentstroke}%
\pgfsetdash{}{0pt}%
\pgfpathmoveto{\pgfqpoint{0.625000in}{1.560013in}}%
\pgfpathlineto{\pgfqpoint{0.631827in}{1.553509in}}%
\pgfpathlineto{\pgfqpoint{0.625000in}{1.546793in}}%
\pgfusepath{stroke}%
\end{pgfscope}%
\begin{pgfscope}%
\pgfpathrectangle{\pgfqpoint{0.625000in}{0.550000in}}{\pgfqpoint{3.875000in}{3.850000in}} %
\pgfusepath{clip}%
\pgfsetbuttcap%
\pgfsetroundjoin%
\pgfsetlinewidth{0.250937pt}%
\definecolor{currentstroke}{rgb}{0.000000,0.000000,0.000000}%
\pgfsetstrokecolor{currentstroke}%
\pgfsetdash{}{0pt}%
\pgfpathmoveto{\pgfqpoint{0.625000in}{1.714379in}}%
\pgfpathlineto{\pgfqpoint{0.633355in}{1.707895in}}%
\pgfpathlineto{\pgfqpoint{0.625000in}{1.701369in}}%
\pgfusepath{stroke}%
\end{pgfscope}%
\begin{pgfscope}%
\pgfpathrectangle{\pgfqpoint{0.625000in}{0.550000in}}{\pgfqpoint{3.875000in}{3.850000in}} %
\pgfusepath{clip}%
\pgfsetbuttcap%
\pgfsetroundjoin%
\pgfsetlinewidth{0.250937pt}%
\definecolor{currentstroke}{rgb}{0.000000,0.000000,0.000000}%
\pgfsetstrokecolor{currentstroke}%
\pgfsetdash{}{0pt}%
\pgfpathmoveto{\pgfqpoint{0.625000in}{1.868794in}}%
\pgfpathlineto{\pgfqpoint{0.631760in}{1.862281in}}%
\pgfpathlineto{\pgfqpoint{0.625000in}{1.855902in}}%
\pgfusepath{stroke}%
\end{pgfscope}%
\begin{pgfscope}%
\pgfpathrectangle{\pgfqpoint{0.625000in}{0.550000in}}{\pgfqpoint{3.875000in}{3.850000in}} %
\pgfusepath{clip}%
\pgfsetbuttcap%
\pgfsetroundjoin%
\pgfsetlinewidth{0.250937pt}%
\definecolor{currentstroke}{rgb}{0.000000,0.000000,0.000000}%
\pgfsetstrokecolor{currentstroke}%
\pgfsetdash{}{0pt}%
\pgfpathmoveto{\pgfqpoint{0.625000in}{2.023382in}}%
\pgfpathlineto{\pgfqpoint{0.631722in}{2.016667in}}%
\pgfpathlineto{\pgfqpoint{0.625000in}{2.010006in}}%
\pgfusepath{stroke}%
\end{pgfscope}%
\begin{pgfscope}%
\pgfpathrectangle{\pgfqpoint{0.625000in}{0.550000in}}{\pgfqpoint{3.875000in}{3.850000in}} %
\pgfusepath{clip}%
\pgfsetbuttcap%
\pgfsetroundjoin%
\pgfsetlinewidth{0.250937pt}%
\definecolor{currentstroke}{rgb}{0.000000,0.000000,0.000000}%
\pgfsetstrokecolor{currentstroke}%
\pgfsetdash{}{0pt}%
\pgfpathmoveto{\pgfqpoint{0.625000in}{2.177440in}}%
\pgfpathlineto{\pgfqpoint{0.631755in}{2.171053in}}%
\pgfpathlineto{\pgfqpoint{0.625000in}{2.164483in}}%
\pgfusepath{stroke}%
\end{pgfscope}%
\begin{pgfscope}%
\pgfpathrectangle{\pgfqpoint{0.625000in}{0.550000in}}{\pgfqpoint{3.875000in}{3.850000in}} %
\pgfusepath{clip}%
\pgfsetbuttcap%
\pgfsetroundjoin%
\pgfsetlinewidth{0.250937pt}%
\definecolor{currentstroke}{rgb}{0.000000,0.000000,0.000000}%
\pgfsetstrokecolor{currentstroke}%
\pgfsetdash{}{0pt}%
\pgfpathmoveto{\pgfqpoint{0.625000in}{2.331776in}}%
\pgfpathlineto{\pgfqpoint{0.631766in}{2.325439in}}%
\pgfpathlineto{\pgfqpoint{0.625000in}{2.318657in}}%
\pgfusepath{stroke}%
\end{pgfscope}%
\begin{pgfscope}%
\pgfpathrectangle{\pgfqpoint{0.625000in}{0.550000in}}{\pgfqpoint{3.875000in}{3.850000in}} %
\pgfusepath{clip}%
\pgfsetbuttcap%
\pgfsetroundjoin%
\pgfsetlinewidth{0.250937pt}%
\definecolor{currentstroke}{rgb}{0.000000,0.000000,0.000000}%
\pgfsetstrokecolor{currentstroke}%
\pgfsetdash{}{0pt}%
\pgfpathmoveto{\pgfqpoint{0.625000in}{2.487125in}}%
\pgfpathlineto{\pgfqpoint{0.627203in}{2.489474in}}%
\pgfpathlineto{\pgfqpoint{0.627937in}{2.499123in}}%
\pgfpathlineto{\pgfqpoint{0.631041in}{2.508772in}}%
\pgfpathlineto{\pgfqpoint{0.634712in}{2.517821in}}%
\pgfpathlineto{\pgfqpoint{0.635300in}{2.518421in}}%
\pgfpathlineto{\pgfqpoint{0.641738in}{2.528070in}}%
\pgfpathlineto{\pgfqpoint{0.644424in}{2.532506in}}%
\pgfpathlineto{\pgfqpoint{0.650169in}{2.537719in}}%
\pgfpathlineto{\pgfqpoint{0.654135in}{2.542288in}}%
\pgfpathlineto{\pgfqpoint{0.661573in}{2.547368in}}%
\pgfpathlineto{\pgfqpoint{0.663847in}{2.549384in}}%
\pgfpathlineto{\pgfqpoint{0.673559in}{2.554828in}}%
\pgfpathlineto{\pgfqpoint{0.679663in}{2.557018in}}%
\pgfpathlineto{\pgfqpoint{0.683271in}{2.558713in}}%
\pgfpathlineto{\pgfqpoint{0.692982in}{2.561463in}}%
\pgfpathlineto{\pgfqpoint{0.702694in}{2.562947in}}%
\pgfpathlineto{\pgfqpoint{0.712406in}{2.563334in}}%
\pgfpathlineto{\pgfqpoint{0.722118in}{2.562650in}}%
\pgfpathlineto{\pgfqpoint{0.731830in}{2.560840in}}%
\pgfpathlineto{\pgfqpoint{0.741541in}{2.557806in}}%
\pgfpathlineto{\pgfqpoint{0.743498in}{2.557018in}}%
\pgfpathlineto{\pgfqpoint{0.751253in}{2.553472in}}%
\pgfpathlineto{\pgfqpoint{0.760965in}{2.547541in}}%
\pgfpathlineto{\pgfqpoint{0.761215in}{2.547368in}}%
\pgfpathlineto{\pgfqpoint{0.770677in}{2.539490in}}%
\pgfpathlineto{\pgfqpoint{0.772567in}{2.537719in}}%
\pgfpathlineto{\pgfqpoint{0.780388in}{2.528404in}}%
\pgfpathlineto{\pgfqpoint{0.780655in}{2.528070in}}%
\pgfpathlineto{\pgfqpoint{0.786758in}{2.518421in}}%
\pgfpathlineto{\pgfqpoint{0.790100in}{2.510842in}}%
\pgfpathlineto{\pgfqpoint{0.791042in}{2.508772in}}%
\pgfpathlineto{\pgfqpoint{0.794098in}{2.499123in}}%
\pgfpathlineto{\pgfqpoint{0.795834in}{2.489474in}}%
\pgfpathlineto{\pgfqpoint{0.796397in}{2.479825in}}%
\pgfpathlineto{\pgfqpoint{0.795834in}{2.470175in}}%
\pgfpathlineto{\pgfqpoint{0.794098in}{2.460526in}}%
\pgfpathlineto{\pgfqpoint{0.791042in}{2.450877in}}%
\pgfpathlineto{\pgfqpoint{0.790100in}{2.448807in}}%
\pgfpathlineto{\pgfqpoint{0.786758in}{2.441228in}}%
\pgfpathlineto{\pgfqpoint{0.780655in}{2.431579in}}%
\pgfpathlineto{\pgfqpoint{0.780388in}{2.431245in}}%
\pgfpathlineto{\pgfqpoint{0.772567in}{2.421930in}}%
\pgfpathlineto{\pgfqpoint{0.770677in}{2.420159in}}%
\pgfpathlineto{\pgfqpoint{0.761215in}{2.412281in}}%
\pgfpathlineto{\pgfqpoint{0.760965in}{2.412108in}}%
\pgfpathlineto{\pgfqpoint{0.751253in}{2.406177in}}%
\pgfpathlineto{\pgfqpoint{0.743498in}{2.402632in}}%
\pgfpathlineto{\pgfqpoint{0.741541in}{2.401843in}}%
\pgfpathlineto{\pgfqpoint{0.731830in}{2.398809in}}%
\pgfpathlineto{\pgfqpoint{0.722118in}{2.397000in}}%
\pgfpathlineto{\pgfqpoint{0.712406in}{2.396315in}}%
\pgfpathlineto{\pgfqpoint{0.702694in}{2.396702in}}%
\pgfpathlineto{\pgfqpoint{0.692982in}{2.398186in}}%
\pgfpathlineto{\pgfqpoint{0.683271in}{2.400936in}}%
\pgfpathlineto{\pgfqpoint{0.679663in}{2.402632in}}%
\pgfpathlineto{\pgfqpoint{0.673559in}{2.404821in}}%
\pgfpathlineto{\pgfqpoint{0.663847in}{2.410265in}}%
\pgfpathlineto{\pgfqpoint{0.661573in}{2.412281in}}%
\pgfpathlineto{\pgfqpoint{0.654135in}{2.417361in}}%
\pgfpathlineto{\pgfqpoint{0.650169in}{2.421930in}}%
\pgfpathlineto{\pgfqpoint{0.644424in}{2.427143in}}%
\pgfpathlineto{\pgfqpoint{0.641738in}{2.431579in}}%
\pgfpathlineto{\pgfqpoint{0.635300in}{2.441228in}}%
\pgfpathlineto{\pgfqpoint{0.634712in}{2.441828in}}%
\pgfpathlineto{\pgfqpoint{0.631041in}{2.450877in}}%
\pgfpathlineto{\pgfqpoint{0.627944in}{2.460526in}}%
\pgfpathlineto{\pgfqpoint{0.627203in}{2.470175in}}%
\pgfpathlineto{\pgfqpoint{0.625000in}{2.472524in}}%
\pgfusepath{stroke}%
\end{pgfscope}%
\begin{pgfscope}%
\pgfpathrectangle{\pgfqpoint{0.625000in}{0.550000in}}{\pgfqpoint{3.875000in}{3.850000in}} %
\pgfusepath{clip}%
\pgfsetbuttcap%
\pgfsetroundjoin%
\pgfsetlinewidth{0.250937pt}%
\definecolor{currentstroke}{rgb}{0.000000,0.000000,0.000000}%
\pgfsetstrokecolor{currentstroke}%
\pgfsetdash{}{0pt}%
\pgfpathmoveto{\pgfqpoint{0.625000in}{2.640987in}}%
\pgfpathlineto{\pgfqpoint{0.631760in}{2.634211in}}%
\pgfpathlineto{\pgfqpoint{0.625000in}{2.627879in}}%
\pgfusepath{stroke}%
\end{pgfscope}%
\begin{pgfscope}%
\pgfpathrectangle{\pgfqpoint{0.625000in}{0.550000in}}{\pgfqpoint{3.875000in}{3.850000in}} %
\pgfusepath{clip}%
\pgfsetbuttcap%
\pgfsetroundjoin%
\pgfsetlinewidth{0.250937pt}%
\definecolor{currentstroke}{rgb}{0.000000,0.000000,0.000000}%
\pgfsetstrokecolor{currentstroke}%
\pgfsetdash{}{0pt}%
\pgfpathmoveto{\pgfqpoint{0.625000in}{2.795148in}}%
\pgfpathlineto{\pgfqpoint{0.631736in}{2.788596in}}%
\pgfpathlineto{\pgfqpoint{0.625000in}{2.782229in}}%
\pgfusepath{stroke}%
\end{pgfscope}%
\begin{pgfscope}%
\pgfpathrectangle{\pgfqpoint{0.625000in}{0.550000in}}{\pgfqpoint{3.875000in}{3.850000in}} %
\pgfusepath{clip}%
\pgfsetbuttcap%
\pgfsetroundjoin%
\pgfsetlinewidth{0.250937pt}%
\definecolor{currentstroke}{rgb}{0.000000,0.000000,0.000000}%
\pgfsetstrokecolor{currentstroke}%
\pgfsetdash{}{0pt}%
\pgfpathmoveto{\pgfqpoint{0.625000in}{2.949591in}}%
\pgfpathlineto{\pgfqpoint{0.631670in}{2.942982in}}%
\pgfpathlineto{\pgfqpoint{0.625000in}{2.936319in}}%
\pgfusepath{stroke}%
\end{pgfscope}%
\begin{pgfscope}%
\pgfpathrectangle{\pgfqpoint{0.625000in}{0.550000in}}{\pgfqpoint{3.875000in}{3.850000in}} %
\pgfusepath{clip}%
\pgfsetbuttcap%
\pgfsetroundjoin%
\pgfsetlinewidth{0.250937pt}%
\definecolor{currentstroke}{rgb}{0.000000,0.000000,0.000000}%
\pgfsetstrokecolor{currentstroke}%
\pgfsetdash{}{0pt}%
\pgfpathmoveto{\pgfqpoint{0.625000in}{3.103704in}}%
\pgfpathlineto{\pgfqpoint{0.631719in}{3.097368in}}%
\pgfpathlineto{\pgfqpoint{0.625000in}{3.090897in}}%
\pgfusepath{stroke}%
\end{pgfscope}%
\begin{pgfscope}%
\pgfpathrectangle{\pgfqpoint{0.625000in}{0.550000in}}{\pgfqpoint{3.875000in}{3.850000in}} %
\pgfusepath{clip}%
\pgfsetbuttcap%
\pgfsetroundjoin%
\pgfsetlinewidth{0.250937pt}%
\definecolor{currentstroke}{rgb}{0.000000,0.000000,0.000000}%
\pgfsetstrokecolor{currentstroke}%
\pgfsetdash{}{0pt}%
\pgfpathmoveto{\pgfqpoint{0.625000in}{3.258129in}}%
\pgfpathlineto{\pgfqpoint{0.633270in}{3.251754in}}%
\pgfpathlineto{\pgfqpoint{0.625000in}{3.245422in}}%
\pgfusepath{stroke}%
\end{pgfscope}%
\begin{pgfscope}%
\pgfpathrectangle{\pgfqpoint{0.625000in}{0.550000in}}{\pgfqpoint{3.875000in}{3.850000in}} %
\pgfusepath{clip}%
\pgfsetbuttcap%
\pgfsetroundjoin%
\pgfsetlinewidth{0.250937pt}%
\definecolor{currentstroke}{rgb}{0.000000,0.000000,0.000000}%
\pgfsetstrokecolor{currentstroke}%
\pgfsetdash{}{0pt}%
\pgfpathmoveto{\pgfqpoint{0.625000in}{3.412755in}}%
\pgfpathlineto{\pgfqpoint{0.631726in}{3.406140in}}%
\pgfpathlineto{\pgfqpoint{0.625000in}{3.399741in}}%
\pgfusepath{stroke}%
\end{pgfscope}%
\begin{pgfscope}%
\pgfpathrectangle{\pgfqpoint{0.625000in}{0.550000in}}{\pgfqpoint{3.875000in}{3.850000in}} %
\pgfusepath{clip}%
\pgfsetbuttcap%
\pgfsetroundjoin%
\pgfsetlinewidth{0.250937pt}%
\definecolor{currentstroke}{rgb}{0.000000,0.000000,0.000000}%
\pgfsetstrokecolor{currentstroke}%
\pgfsetdash{}{0pt}%
\pgfpathmoveto{\pgfqpoint{0.625000in}{3.566906in}}%
\pgfpathlineto{\pgfqpoint{0.631714in}{3.560526in}}%
\pgfpathlineto{\pgfqpoint{0.625000in}{3.554151in}}%
\pgfusepath{stroke}%
\end{pgfscope}%
\begin{pgfscope}%
\pgfpathrectangle{\pgfqpoint{0.625000in}{0.550000in}}{\pgfqpoint{3.875000in}{3.850000in}} %
\pgfusepath{clip}%
\pgfsetbuttcap%
\pgfsetroundjoin%
\pgfsetlinewidth{0.250937pt}%
\definecolor{currentstroke}{rgb}{0.000000,0.000000,0.000000}%
\pgfsetstrokecolor{currentstroke}%
\pgfsetdash{}{0pt}%
\pgfpathmoveto{\pgfqpoint{0.625000in}{3.721361in}}%
\pgfpathlineto{\pgfqpoint{0.631734in}{3.714912in}}%
\pgfpathlineto{\pgfqpoint{0.625000in}{3.708269in}}%
\pgfusepath{stroke}%
\end{pgfscope}%
\begin{pgfscope}%
\pgfpathrectangle{\pgfqpoint{0.625000in}{0.550000in}}{\pgfqpoint{3.875000in}{3.850000in}} %
\pgfusepath{clip}%
\pgfsetbuttcap%
\pgfsetroundjoin%
\pgfsetlinewidth{0.250937pt}%
\definecolor{currentstroke}{rgb}{0.000000,0.000000,0.000000}%
\pgfsetstrokecolor{currentstroke}%
\pgfsetdash{}{0pt}%
\pgfpathmoveto{\pgfqpoint{0.625000in}{3.875603in}}%
\pgfpathlineto{\pgfqpoint{0.631625in}{3.869298in}}%
\pgfpathlineto{\pgfqpoint{0.625000in}{3.862968in}}%
\pgfusepath{stroke}%
\end{pgfscope}%
\begin{pgfscope}%
\pgfpathrectangle{\pgfqpoint{0.625000in}{0.550000in}}{\pgfqpoint{3.875000in}{3.850000in}} %
\pgfusepath{clip}%
\pgfsetbuttcap%
\pgfsetroundjoin%
\pgfsetlinewidth{0.250937pt}%
\definecolor{currentstroke}{rgb}{0.000000,0.000000,0.000000}%
\pgfsetstrokecolor{currentstroke}%
\pgfsetdash{}{0pt}%
\pgfpathmoveto{\pgfqpoint{0.625000in}{4.030147in}}%
\pgfpathlineto{\pgfqpoint{0.633300in}{4.023684in}}%
\pgfpathlineto{\pgfqpoint{0.625000in}{4.017284in}}%
\pgfusepath{stroke}%
\end{pgfscope}%
\begin{pgfscope}%
\pgfpathrectangle{\pgfqpoint{0.625000in}{0.550000in}}{\pgfqpoint{3.875000in}{3.850000in}} %
\pgfusepath{clip}%
\pgfsetbuttcap%
\pgfsetroundjoin%
\pgfsetlinewidth{0.250937pt}%
\definecolor{currentstroke}{rgb}{0.000000,0.000000,0.000000}%
\pgfsetstrokecolor{currentstroke}%
\pgfsetdash{}{0pt}%
\pgfpathmoveto{\pgfqpoint{0.625000in}{4.184429in}}%
\pgfpathlineto{\pgfqpoint{0.631606in}{4.178070in}}%
\pgfpathlineto{\pgfqpoint{0.625000in}{4.172009in}}%
\pgfusepath{stroke}%
\end{pgfscope}%
\begin{pgfscope}%
\pgfpathrectangle{\pgfqpoint{0.625000in}{0.550000in}}{\pgfqpoint{3.875000in}{3.850000in}} %
\pgfusepath{clip}%
\pgfsetbuttcap%
\pgfsetroundjoin%
\pgfsetlinewidth{0.250937pt}%
\definecolor{currentstroke}{rgb}{0.000000,0.000000,0.000000}%
\pgfsetstrokecolor{currentstroke}%
\pgfsetdash{}{0pt}%
\pgfpathmoveto{\pgfqpoint{0.625000in}{4.338798in}}%
\pgfpathlineto{\pgfqpoint{0.631587in}{4.332456in}}%
\pgfpathlineto{\pgfqpoint{0.625000in}{4.326093in}}%
\pgfusepath{stroke}%
\end{pgfscope}%
\begin{pgfscope}%
\pgfpathrectangle{\pgfqpoint{0.625000in}{0.550000in}}{\pgfqpoint{3.875000in}{3.850000in}} %
\pgfusepath{clip}%
\pgfsetbuttcap%
\pgfsetroundjoin%
\pgfsetlinewidth{0.250937pt}%
\definecolor{currentstroke}{rgb}{0.000000,0.000000,0.000000}%
\pgfsetstrokecolor{currentstroke}%
\pgfsetdash{}{0pt}%
\pgfpathmoveto{\pgfqpoint{0.634712in}{1.183660in}}%
\pgfpathlineto{\pgfqpoint{0.632732in}{1.186842in}}%
\pgfpathlineto{\pgfqpoint{0.630563in}{1.196491in}}%
\pgfpathlineto{\pgfqpoint{0.634712in}{1.202725in}}%
\pgfpathlineto{\pgfqpoint{0.643086in}{1.196491in}}%
\pgfpathlineto{\pgfqpoint{0.640571in}{1.186842in}}%
\pgfpathlineto{\pgfqpoint{0.634712in}{1.183660in}}%
\pgfusepath{stroke}%
\end{pgfscope}%
\begin{pgfscope}%
\pgfpathrectangle{\pgfqpoint{0.625000in}{0.550000in}}{\pgfqpoint{3.875000in}{3.850000in}} %
\pgfusepath{clip}%
\pgfsetbuttcap%
\pgfsetroundjoin%
\pgfsetlinewidth{0.250937pt}%
\definecolor{currentstroke}{rgb}{0.000000,0.000000,0.000000}%
\pgfsetstrokecolor{currentstroke}%
\pgfsetdash{}{0pt}%
\pgfpathmoveto{\pgfqpoint{0.634712in}{3.756924in}}%
\pgfpathlineto{\pgfqpoint{0.630563in}{3.763158in}}%
\pgfpathlineto{\pgfqpoint{0.632732in}{3.772807in}}%
\pgfpathlineto{\pgfqpoint{0.634712in}{3.775989in}}%
\pgfpathlineto{\pgfqpoint{0.640571in}{3.772807in}}%
\pgfpathlineto{\pgfqpoint{0.643086in}{3.763158in}}%
\pgfpathlineto{\pgfqpoint{0.634712in}{3.756924in}}%
\pgfusepath{stroke}%
\end{pgfscope}%
\begin{pgfscope}%
\pgfpathrectangle{\pgfqpoint{0.625000in}{0.550000in}}{\pgfqpoint{3.875000in}{3.850000in}} %
\pgfusepath{clip}%
\pgfsetbuttcap%
\pgfsetroundjoin%
\pgfsetlinewidth{0.250937pt}%
\definecolor{currentstroke}{rgb}{0.000000,0.000000,0.000000}%
\pgfsetstrokecolor{currentstroke}%
\pgfsetdash{}{0pt}%
\pgfpathmoveto{\pgfqpoint{0.625000in}{0.634066in}}%
\pgfpathlineto{\pgfqpoint{0.632089in}{0.627193in}}%
\pgfpathlineto{\pgfqpoint{0.625000in}{0.620340in}}%
\pgfusepath{stroke}%
\end{pgfscope}%
\begin{pgfscope}%
\pgfpathrectangle{\pgfqpoint{0.625000in}{0.550000in}}{\pgfqpoint{3.875000in}{3.850000in}} %
\pgfusepath{clip}%
\pgfsetbuttcap%
\pgfsetroundjoin%
\pgfsetlinewidth{0.250937pt}%
\definecolor{currentstroke}{rgb}{0.000000,0.000000,0.000000}%
\pgfsetstrokecolor{currentstroke}%
\pgfsetdash{}{0pt}%
\pgfpathmoveto{\pgfqpoint{0.625000in}{0.788038in}}%
\pgfpathlineto{\pgfqpoint{0.631987in}{0.781579in}}%
\pgfpathlineto{\pgfqpoint{0.625000in}{0.774833in}}%
\pgfusepath{stroke}%
\end{pgfscope}%
\begin{pgfscope}%
\pgfpathrectangle{\pgfqpoint{0.625000in}{0.550000in}}{\pgfqpoint{3.875000in}{3.850000in}} %
\pgfusepath{clip}%
\pgfsetbuttcap%
\pgfsetroundjoin%
\pgfsetlinewidth{0.250937pt}%
\definecolor{currentstroke}{rgb}{0.000000,0.000000,0.000000}%
\pgfsetstrokecolor{currentstroke}%
\pgfsetdash{}{0pt}%
\pgfpathmoveto{\pgfqpoint{0.625000in}{0.942717in}}%
\pgfpathlineto{\pgfqpoint{0.633548in}{0.935965in}}%
\pgfpathlineto{\pgfqpoint{0.625000in}{0.929152in}}%
\pgfusepath{stroke}%
\end{pgfscope}%
\begin{pgfscope}%
\pgfpathrectangle{\pgfqpoint{0.625000in}{0.550000in}}{\pgfqpoint{3.875000in}{3.850000in}} %
\pgfusepath{clip}%
\pgfsetbuttcap%
\pgfsetroundjoin%
\pgfsetlinewidth{0.250937pt}%
\definecolor{currentstroke}{rgb}{0.000000,0.000000,0.000000}%
\pgfsetstrokecolor{currentstroke}%
\pgfsetdash{}{0pt}%
\pgfpathmoveto{\pgfqpoint{0.625000in}{1.097064in}}%
\pgfpathlineto{\pgfqpoint{0.632001in}{1.090351in}}%
\pgfpathlineto{\pgfqpoint{0.625000in}{1.083662in}}%
\pgfusepath{stroke}%
\end{pgfscope}%
\begin{pgfscope}%
\pgfpathrectangle{\pgfqpoint{0.625000in}{0.550000in}}{\pgfqpoint{3.875000in}{3.850000in}} %
\pgfusepath{clip}%
\pgfsetbuttcap%
\pgfsetroundjoin%
\pgfsetlinewidth{0.250937pt}%
\definecolor{currentstroke}{rgb}{0.000000,0.000000,0.000000}%
\pgfsetstrokecolor{currentstroke}%
\pgfsetdash{}{0pt}%
\pgfpathmoveto{\pgfqpoint{0.625000in}{1.251640in}}%
\pgfpathlineto{\pgfqpoint{0.631995in}{1.244737in}}%
\pgfpathlineto{\pgfqpoint{0.625000in}{1.238024in}}%
\pgfusepath{stroke}%
\end{pgfscope}%
\begin{pgfscope}%
\pgfpathrectangle{\pgfqpoint{0.625000in}{0.550000in}}{\pgfqpoint{3.875000in}{3.850000in}} %
\pgfusepath{clip}%
\pgfsetbuttcap%
\pgfsetroundjoin%
\pgfsetlinewidth{0.250937pt}%
\definecolor{currentstroke}{rgb}{0.000000,0.000000,0.000000}%
\pgfsetstrokecolor{currentstroke}%
\pgfsetdash{}{0pt}%
\pgfpathmoveto{\pgfqpoint{0.625000in}{1.405705in}}%
\pgfpathlineto{\pgfqpoint{0.631919in}{1.399123in}}%
\pgfpathlineto{\pgfqpoint{0.625000in}{1.392537in}}%
\pgfusepath{stroke}%
\end{pgfscope}%
\begin{pgfscope}%
\pgfpathrectangle{\pgfqpoint{0.625000in}{0.550000in}}{\pgfqpoint{3.875000in}{3.850000in}} %
\pgfusepath{clip}%
\pgfsetbuttcap%
\pgfsetroundjoin%
\pgfsetlinewidth{0.250937pt}%
\definecolor{currentstroke}{rgb}{0.000000,0.000000,0.000000}%
\pgfsetstrokecolor{currentstroke}%
\pgfsetdash{}{0pt}%
\pgfpathmoveto{\pgfqpoint{0.625000in}{1.560093in}}%
\pgfpathlineto{\pgfqpoint{0.631911in}{1.553509in}}%
\pgfpathlineto{\pgfqpoint{0.625000in}{1.546711in}}%
\pgfusepath{stroke}%
\end{pgfscope}%
\begin{pgfscope}%
\pgfpathrectangle{\pgfqpoint{0.625000in}{0.550000in}}{\pgfqpoint{3.875000in}{3.850000in}} %
\pgfusepath{clip}%
\pgfsetbuttcap%
\pgfsetroundjoin%
\pgfsetlinewidth{0.250937pt}%
\definecolor{currentstroke}{rgb}{0.000000,0.000000,0.000000}%
\pgfsetstrokecolor{currentstroke}%
\pgfsetdash{}{0pt}%
\pgfpathmoveto{\pgfqpoint{0.625000in}{1.714460in}}%
\pgfpathlineto{\pgfqpoint{0.633459in}{1.707895in}}%
\pgfpathlineto{\pgfqpoint{0.625000in}{1.701288in}}%
\pgfusepath{stroke}%
\end{pgfscope}%
\begin{pgfscope}%
\pgfpathrectangle{\pgfqpoint{0.625000in}{0.550000in}}{\pgfqpoint{3.875000in}{3.850000in}} %
\pgfusepath{clip}%
\pgfsetbuttcap%
\pgfsetroundjoin%
\pgfsetlinewidth{0.250937pt}%
\definecolor{currentstroke}{rgb}{0.000000,0.000000,0.000000}%
\pgfsetstrokecolor{currentstroke}%
\pgfsetdash{}{0pt}%
\pgfpathmoveto{\pgfqpoint{0.625000in}{1.868877in}}%
\pgfpathlineto{\pgfqpoint{0.631846in}{1.862281in}}%
\pgfpathlineto{\pgfqpoint{0.625000in}{1.855821in}}%
\pgfusepath{stroke}%
\end{pgfscope}%
\begin{pgfscope}%
\pgfpathrectangle{\pgfqpoint{0.625000in}{0.550000in}}{\pgfqpoint{3.875000in}{3.850000in}} %
\pgfusepath{clip}%
\pgfsetbuttcap%
\pgfsetroundjoin%
\pgfsetlinewidth{0.250937pt}%
\definecolor{currentstroke}{rgb}{0.000000,0.000000,0.000000}%
\pgfsetstrokecolor{currentstroke}%
\pgfsetdash{}{0pt}%
\pgfpathmoveto{\pgfqpoint{0.625000in}{2.023469in}}%
\pgfpathlineto{\pgfqpoint{0.631809in}{2.016667in}}%
\pgfpathlineto{\pgfqpoint{0.625000in}{2.009920in}}%
\pgfusepath{stroke}%
\end{pgfscope}%
\begin{pgfscope}%
\pgfpathrectangle{\pgfqpoint{0.625000in}{0.550000in}}{\pgfqpoint{3.875000in}{3.850000in}} %
\pgfusepath{clip}%
\pgfsetbuttcap%
\pgfsetroundjoin%
\pgfsetlinewidth{0.250937pt}%
\definecolor{currentstroke}{rgb}{0.000000,0.000000,0.000000}%
\pgfsetstrokecolor{currentstroke}%
\pgfsetdash{}{0pt}%
\pgfpathmoveto{\pgfqpoint{0.625000in}{2.177522in}}%
\pgfpathlineto{\pgfqpoint{0.631842in}{2.171053in}}%
\pgfpathlineto{\pgfqpoint{0.625000in}{2.164398in}}%
\pgfusepath{stroke}%
\end{pgfscope}%
\begin{pgfscope}%
\pgfpathrectangle{\pgfqpoint{0.625000in}{0.550000in}}{\pgfqpoint{3.875000in}{3.850000in}} %
\pgfusepath{clip}%
\pgfsetbuttcap%
\pgfsetroundjoin%
\pgfsetlinewidth{0.250937pt}%
\definecolor{currentstroke}{rgb}{0.000000,0.000000,0.000000}%
\pgfsetstrokecolor{currentstroke}%
\pgfsetdash{}{0pt}%
\pgfpathmoveto{\pgfqpoint{0.625000in}{2.331859in}}%
\pgfpathlineto{\pgfqpoint{0.631854in}{2.325439in}}%
\pgfpathlineto{\pgfqpoint{0.625000in}{2.318569in}}%
\pgfusepath{stroke}%
\end{pgfscope}%
\begin{pgfscope}%
\pgfpathrectangle{\pgfqpoint{0.625000in}{0.550000in}}{\pgfqpoint{3.875000in}{3.850000in}} %
\pgfusepath{clip}%
\pgfsetbuttcap%
\pgfsetroundjoin%
\pgfsetlinewidth{0.250937pt}%
\definecolor{currentstroke}{rgb}{0.000000,0.000000,0.000000}%
\pgfsetstrokecolor{currentstroke}%
\pgfsetdash{}{0pt}%
\pgfpathmoveto{\pgfqpoint{0.625000in}{2.487209in}}%
\pgfpathlineto{\pgfqpoint{0.627124in}{2.489474in}}%
\pgfpathlineto{\pgfqpoint{0.627858in}{2.499123in}}%
\pgfpathlineto{\pgfqpoint{0.630886in}{2.508772in}}%
\pgfpathlineto{\pgfqpoint{0.634712in}{2.518202in}}%
\pgfpathlineto{\pgfqpoint{0.634927in}{2.518421in}}%
\pgfpathlineto{\pgfqpoint{0.641234in}{2.528070in}}%
\pgfpathlineto{\pgfqpoint{0.644424in}{2.533338in}}%
\pgfpathlineto{\pgfqpoint{0.649252in}{2.537719in}}%
\pgfpathlineto{\pgfqpoint{0.654135in}{2.543344in}}%
\pgfpathlineto{\pgfqpoint{0.660026in}{2.547368in}}%
\pgfpathlineto{\pgfqpoint{0.663847in}{2.550754in}}%
\pgfpathlineto{\pgfqpoint{0.673559in}{2.556212in}}%
\pgfpathlineto{\pgfqpoint{0.675804in}{2.557018in}}%
\pgfpathlineto{\pgfqpoint{0.683271in}{2.560527in}}%
\pgfpathlineto{\pgfqpoint{0.692982in}{2.563413in}}%
\pgfpathlineto{\pgfqpoint{0.702694in}{2.565101in}}%
\pgfpathlineto{\pgfqpoint{0.712406in}{2.565758in}}%
\pgfpathlineto{\pgfqpoint{0.722118in}{2.565407in}}%
\pgfpathlineto{\pgfqpoint{0.731830in}{2.563997in}}%
\pgfpathlineto{\pgfqpoint{0.741541in}{2.561433in}}%
\pgfpathlineto{\pgfqpoint{0.751253in}{2.557594in}}%
\pgfpathlineto{\pgfqpoint{0.752458in}{2.557018in}}%
\pgfpathlineto{\pgfqpoint{0.760965in}{2.552337in}}%
\pgfpathlineto{\pgfqpoint{0.768177in}{2.547368in}}%
\pgfpathlineto{\pgfqpoint{0.770677in}{2.545287in}}%
\pgfpathlineto{\pgfqpoint{0.778755in}{2.537719in}}%
\pgfpathlineto{\pgfqpoint{0.780388in}{2.535773in}}%
\pgfpathlineto{\pgfqpoint{0.786549in}{2.528070in}}%
\pgfpathlineto{\pgfqpoint{0.790100in}{2.522105in}}%
\pgfpathlineto{\pgfqpoint{0.792293in}{2.518421in}}%
\pgfpathlineto{\pgfqpoint{0.796530in}{2.508772in}}%
\pgfpathlineto{\pgfqpoint{0.799315in}{2.499123in}}%
\pgfpathlineto{\pgfqpoint{0.799812in}{2.496147in}}%
\pgfpathlineto{\pgfqpoint{0.801024in}{2.489474in}}%
\pgfpathlineto{\pgfqpoint{0.801599in}{2.479825in}}%
\pgfpathlineto{\pgfqpoint{0.801024in}{2.470175in}}%
\pgfpathlineto{\pgfqpoint{0.799812in}{2.463502in}}%
\pgfpathlineto{\pgfqpoint{0.799315in}{2.460526in}}%
\pgfpathlineto{\pgfqpoint{0.796530in}{2.450877in}}%
\pgfpathlineto{\pgfqpoint{0.792293in}{2.441228in}}%
\pgfpathlineto{\pgfqpoint{0.790100in}{2.437544in}}%
\pgfpathlineto{\pgfqpoint{0.786549in}{2.431579in}}%
\pgfpathlineto{\pgfqpoint{0.780388in}{2.423876in}}%
\pgfpathlineto{\pgfqpoint{0.778755in}{2.421930in}}%
\pgfpathlineto{\pgfqpoint{0.770677in}{2.414362in}}%
\pgfpathlineto{\pgfqpoint{0.768177in}{2.412281in}}%
\pgfpathlineto{\pgfqpoint{0.760965in}{2.407312in}}%
\pgfpathlineto{\pgfqpoint{0.752458in}{2.402632in}}%
\pgfpathlineto{\pgfqpoint{0.751253in}{2.402055in}}%
\pgfpathlineto{\pgfqpoint{0.741541in}{2.398216in}}%
\pgfpathlineto{\pgfqpoint{0.731830in}{2.395652in}}%
\pgfpathlineto{\pgfqpoint{0.722118in}{2.394242in}}%
\pgfpathlineto{\pgfqpoint{0.712406in}{2.393891in}}%
\pgfpathlineto{\pgfqpoint{0.702694in}{2.394548in}}%
\pgfpathlineto{\pgfqpoint{0.692982in}{2.396236in}}%
\pgfpathlineto{\pgfqpoint{0.683271in}{2.399122in}}%
\pgfpathlineto{\pgfqpoint{0.675804in}{2.402632in}}%
\pgfpathlineto{\pgfqpoint{0.673559in}{2.403437in}}%
\pgfpathlineto{\pgfqpoint{0.663847in}{2.408895in}}%
\pgfpathlineto{\pgfqpoint{0.660026in}{2.412281in}}%
\pgfpathlineto{\pgfqpoint{0.654135in}{2.416305in}}%
\pgfpathlineto{\pgfqpoint{0.649252in}{2.421930in}}%
\pgfpathlineto{\pgfqpoint{0.644424in}{2.426311in}}%
\pgfpathlineto{\pgfqpoint{0.641234in}{2.431579in}}%
\pgfpathlineto{\pgfqpoint{0.634927in}{2.441228in}}%
\pgfpathlineto{\pgfqpoint{0.634712in}{2.441447in}}%
\pgfpathlineto{\pgfqpoint{0.630886in}{2.450877in}}%
\pgfpathlineto{\pgfqpoint{0.627865in}{2.460526in}}%
\pgfpathlineto{\pgfqpoint{0.627124in}{2.470175in}}%
\pgfpathlineto{\pgfqpoint{0.625000in}{2.472441in}}%
\pgfusepath{stroke}%
\end{pgfscope}%
\begin{pgfscope}%
\pgfpathrectangle{\pgfqpoint{0.625000in}{0.550000in}}{\pgfqpoint{3.875000in}{3.850000in}} %
\pgfusepath{clip}%
\pgfsetbuttcap%
\pgfsetroundjoin%
\pgfsetlinewidth{0.250937pt}%
\definecolor{currentstroke}{rgb}{0.000000,0.000000,0.000000}%
\pgfsetstrokecolor{currentstroke}%
\pgfsetdash{}{0pt}%
\pgfpathmoveto{\pgfqpoint{0.625000in}{2.641075in}}%
\pgfpathlineto{\pgfqpoint{0.631849in}{2.634211in}}%
\pgfpathlineto{\pgfqpoint{0.625000in}{2.627796in}}%
\pgfusepath{stroke}%
\end{pgfscope}%
\begin{pgfscope}%
\pgfpathrectangle{\pgfqpoint{0.625000in}{0.550000in}}{\pgfqpoint{3.875000in}{3.850000in}} %
\pgfusepath{clip}%
\pgfsetbuttcap%
\pgfsetroundjoin%
\pgfsetlinewidth{0.250937pt}%
\definecolor{currentstroke}{rgb}{0.000000,0.000000,0.000000}%
\pgfsetstrokecolor{currentstroke}%
\pgfsetdash{}{0pt}%
\pgfpathmoveto{\pgfqpoint{0.625000in}{2.795233in}}%
\pgfpathlineto{\pgfqpoint{0.631824in}{2.788596in}}%
\pgfpathlineto{\pgfqpoint{0.625000in}{2.782146in}}%
\pgfusepath{stroke}%
\end{pgfscope}%
\begin{pgfscope}%
\pgfpathrectangle{\pgfqpoint{0.625000in}{0.550000in}}{\pgfqpoint{3.875000in}{3.850000in}} %
\pgfusepath{clip}%
\pgfsetbuttcap%
\pgfsetroundjoin%
\pgfsetlinewidth{0.250937pt}%
\definecolor{currentstroke}{rgb}{0.000000,0.000000,0.000000}%
\pgfsetstrokecolor{currentstroke}%
\pgfsetdash{}{0pt}%
\pgfpathmoveto{\pgfqpoint{0.625000in}{2.949679in}}%
\pgfpathlineto{\pgfqpoint{0.631759in}{2.942982in}}%
\pgfpathlineto{\pgfqpoint{0.625000in}{2.936230in}}%
\pgfusepath{stroke}%
\end{pgfscope}%
\begin{pgfscope}%
\pgfpathrectangle{\pgfqpoint{0.625000in}{0.550000in}}{\pgfqpoint{3.875000in}{3.850000in}} %
\pgfusepath{clip}%
\pgfsetbuttcap%
\pgfsetroundjoin%
\pgfsetlinewidth{0.250937pt}%
\definecolor{currentstroke}{rgb}{0.000000,0.000000,0.000000}%
\pgfsetstrokecolor{currentstroke}%
\pgfsetdash{}{0pt}%
\pgfpathmoveto{\pgfqpoint{0.625000in}{3.103787in}}%
\pgfpathlineto{\pgfqpoint{0.631806in}{3.097368in}}%
\pgfpathlineto{\pgfqpoint{0.625000in}{3.090813in}}%
\pgfusepath{stroke}%
\end{pgfscope}%
\begin{pgfscope}%
\pgfpathrectangle{\pgfqpoint{0.625000in}{0.550000in}}{\pgfqpoint{3.875000in}{3.850000in}} %
\pgfusepath{clip}%
\pgfsetbuttcap%
\pgfsetroundjoin%
\pgfsetlinewidth{0.250937pt}%
\definecolor{currentstroke}{rgb}{0.000000,0.000000,0.000000}%
\pgfsetstrokecolor{currentstroke}%
\pgfsetdash{}{0pt}%
\pgfpathmoveto{\pgfqpoint{0.625000in}{3.258214in}}%
\pgfpathlineto{\pgfqpoint{0.633381in}{3.251754in}}%
\pgfpathlineto{\pgfqpoint{0.625000in}{3.245337in}}%
\pgfusepath{stroke}%
\end{pgfscope}%
\begin{pgfscope}%
\pgfpathrectangle{\pgfqpoint{0.625000in}{0.550000in}}{\pgfqpoint{3.875000in}{3.850000in}} %
\pgfusepath{clip}%
\pgfsetbuttcap%
\pgfsetroundjoin%
\pgfsetlinewidth{0.250937pt}%
\definecolor{currentstroke}{rgb}{0.000000,0.000000,0.000000}%
\pgfsetstrokecolor{currentstroke}%
\pgfsetdash{}{0pt}%
\pgfpathmoveto{\pgfqpoint{0.625000in}{3.412840in}}%
\pgfpathlineto{\pgfqpoint{0.631813in}{3.406140in}}%
\pgfpathlineto{\pgfqpoint{0.625000in}{3.399658in}}%
\pgfusepath{stroke}%
\end{pgfscope}%
\begin{pgfscope}%
\pgfpathrectangle{\pgfqpoint{0.625000in}{0.550000in}}{\pgfqpoint{3.875000in}{3.850000in}} %
\pgfusepath{clip}%
\pgfsetbuttcap%
\pgfsetroundjoin%
\pgfsetlinewidth{0.250937pt}%
\definecolor{currentstroke}{rgb}{0.000000,0.000000,0.000000}%
\pgfsetstrokecolor{currentstroke}%
\pgfsetdash{}{0pt}%
\pgfpathmoveto{\pgfqpoint{0.625000in}{3.566991in}}%
\pgfpathlineto{\pgfqpoint{0.631803in}{3.560526in}}%
\pgfpathlineto{\pgfqpoint{0.625000in}{3.554066in}}%
\pgfusepath{stroke}%
\end{pgfscope}%
\begin{pgfscope}%
\pgfpathrectangle{\pgfqpoint{0.625000in}{0.550000in}}{\pgfqpoint{3.875000in}{3.850000in}} %
\pgfusepath{clip}%
\pgfsetbuttcap%
\pgfsetroundjoin%
\pgfsetlinewidth{0.250937pt}%
\definecolor{currentstroke}{rgb}{0.000000,0.000000,0.000000}%
\pgfsetstrokecolor{currentstroke}%
\pgfsetdash{}{0pt}%
\pgfpathmoveto{\pgfqpoint{0.625000in}{3.721446in}}%
\pgfpathlineto{\pgfqpoint{0.631823in}{3.714912in}}%
\pgfpathlineto{\pgfqpoint{0.625000in}{3.708181in}}%
\pgfusepath{stroke}%
\end{pgfscope}%
\begin{pgfscope}%
\pgfpathrectangle{\pgfqpoint{0.625000in}{0.550000in}}{\pgfqpoint{3.875000in}{3.850000in}} %
\pgfusepath{clip}%
\pgfsetbuttcap%
\pgfsetroundjoin%
\pgfsetlinewidth{0.250937pt}%
\definecolor{currentstroke}{rgb}{0.000000,0.000000,0.000000}%
\pgfsetstrokecolor{currentstroke}%
\pgfsetdash{}{0pt}%
\pgfpathmoveto{\pgfqpoint{0.625000in}{3.875689in}}%
\pgfpathlineto{\pgfqpoint{0.631716in}{3.869298in}}%
\pgfpathlineto{\pgfqpoint{0.625000in}{3.862881in}}%
\pgfusepath{stroke}%
\end{pgfscope}%
\begin{pgfscope}%
\pgfpathrectangle{\pgfqpoint{0.625000in}{0.550000in}}{\pgfqpoint{3.875000in}{3.850000in}} %
\pgfusepath{clip}%
\pgfsetbuttcap%
\pgfsetroundjoin%
\pgfsetlinewidth{0.250937pt}%
\definecolor{currentstroke}{rgb}{0.000000,0.000000,0.000000}%
\pgfsetstrokecolor{currentstroke}%
\pgfsetdash{}{0pt}%
\pgfpathmoveto{\pgfqpoint{0.625000in}{4.030232in}}%
\pgfpathlineto{\pgfqpoint{0.633409in}{4.023684in}}%
\pgfpathlineto{\pgfqpoint{0.625000in}{4.017201in}}%
\pgfusepath{stroke}%
\end{pgfscope}%
\begin{pgfscope}%
\pgfpathrectangle{\pgfqpoint{0.625000in}{0.550000in}}{\pgfqpoint{3.875000in}{3.850000in}} %
\pgfusepath{clip}%
\pgfsetbuttcap%
\pgfsetroundjoin%
\pgfsetlinewidth{0.250937pt}%
\definecolor{currentstroke}{rgb}{0.000000,0.000000,0.000000}%
\pgfsetstrokecolor{currentstroke}%
\pgfsetdash{}{0pt}%
\pgfpathmoveto{\pgfqpoint{0.625000in}{4.184518in}}%
\pgfpathlineto{\pgfqpoint{0.631698in}{4.178070in}}%
\pgfpathlineto{\pgfqpoint{0.625000in}{4.171925in}}%
\pgfusepath{stroke}%
\end{pgfscope}%
\begin{pgfscope}%
\pgfpathrectangle{\pgfqpoint{0.625000in}{0.550000in}}{\pgfqpoint{3.875000in}{3.850000in}} %
\pgfusepath{clip}%
\pgfsetbuttcap%
\pgfsetroundjoin%
\pgfsetlinewidth{0.250937pt}%
\definecolor{currentstroke}{rgb}{0.000000,0.000000,0.000000}%
\pgfsetstrokecolor{currentstroke}%
\pgfsetdash{}{0pt}%
\pgfpathmoveto{\pgfqpoint{0.625000in}{4.338885in}}%
\pgfpathlineto{\pgfqpoint{0.631678in}{4.332456in}}%
\pgfpathlineto{\pgfqpoint{0.625000in}{4.326006in}}%
\pgfusepath{stroke}%
\end{pgfscope}%
\begin{pgfscope}%
\pgfpathrectangle{\pgfqpoint{0.625000in}{0.550000in}}{\pgfqpoint{3.875000in}{3.850000in}} %
\pgfusepath{clip}%
\pgfsetbuttcap%
\pgfsetroundjoin%
\pgfsetlinewidth{0.250937pt}%
\definecolor{currentstroke}{rgb}{0.000000,0.000000,0.000000}%
\pgfsetstrokecolor{currentstroke}%
\pgfsetdash{}{0pt}%
\pgfpathmoveto{\pgfqpoint{0.634712in}{1.183326in}}%
\pgfpathlineto{\pgfqpoint{0.632524in}{1.186842in}}%
\pgfpathlineto{\pgfqpoint{0.630388in}{1.196491in}}%
\pgfpathlineto{\pgfqpoint{0.634712in}{1.202988in}}%
\pgfpathlineto{\pgfqpoint{0.643439in}{1.196491in}}%
\pgfpathlineto{\pgfqpoint{0.641186in}{1.186842in}}%
\pgfpathlineto{\pgfqpoint{0.634712in}{1.183326in}}%
\pgfusepath{stroke}%
\end{pgfscope}%
\begin{pgfscope}%
\pgfpathrectangle{\pgfqpoint{0.625000in}{0.550000in}}{\pgfqpoint{3.875000in}{3.850000in}} %
\pgfusepath{clip}%
\pgfsetbuttcap%
\pgfsetroundjoin%
\pgfsetlinewidth{0.250937pt}%
\definecolor{currentstroke}{rgb}{0.000000,0.000000,0.000000}%
\pgfsetstrokecolor{currentstroke}%
\pgfsetdash{}{0pt}%
\pgfpathmoveto{\pgfqpoint{0.634712in}{3.756661in}}%
\pgfpathlineto{\pgfqpoint{0.630388in}{3.763158in}}%
\pgfpathlineto{\pgfqpoint{0.632524in}{3.772807in}}%
\pgfpathlineto{\pgfqpoint{0.634712in}{3.776323in}}%
\pgfpathlineto{\pgfqpoint{0.641186in}{3.772807in}}%
\pgfpathlineto{\pgfqpoint{0.643439in}{3.763158in}}%
\pgfpathlineto{\pgfqpoint{0.634712in}{3.756661in}}%
\pgfusepath{stroke}%
\end{pgfscope}%
\begin{pgfscope}%
\pgfpathrectangle{\pgfqpoint{0.625000in}{0.550000in}}{\pgfqpoint{3.875000in}{3.850000in}} %
\pgfusepath{clip}%
\pgfsetbuttcap%
\pgfsetroundjoin%
\pgfsetlinewidth{0.250937pt}%
\definecolor{currentstroke}{rgb}{0.000000,0.000000,0.000000}%
\pgfsetstrokecolor{currentstroke}%
\pgfsetdash{}{0pt}%
\pgfpathmoveto{\pgfqpoint{0.625000in}{0.634142in}}%
\pgfpathlineto{\pgfqpoint{0.632167in}{0.627193in}}%
\pgfpathlineto{\pgfqpoint{0.625000in}{0.620264in}}%
\pgfusepath{stroke}%
\end{pgfscope}%
\begin{pgfscope}%
\pgfpathrectangle{\pgfqpoint{0.625000in}{0.550000in}}{\pgfqpoint{3.875000in}{3.850000in}} %
\pgfusepath{clip}%
\pgfsetbuttcap%
\pgfsetroundjoin%
\pgfsetlinewidth{0.250937pt}%
\definecolor{currentstroke}{rgb}{0.000000,0.000000,0.000000}%
\pgfsetstrokecolor{currentstroke}%
\pgfsetdash{}{0pt}%
\pgfpathmoveto{\pgfqpoint{0.625000in}{0.788115in}}%
\pgfpathlineto{\pgfqpoint{0.632070in}{0.781579in}}%
\pgfpathlineto{\pgfqpoint{0.625000in}{0.774752in}}%
\pgfusepath{stroke}%
\end{pgfscope}%
\begin{pgfscope}%
\pgfpathrectangle{\pgfqpoint{0.625000in}{0.550000in}}{\pgfqpoint{3.875000in}{3.850000in}} %
\pgfusepath{clip}%
\pgfsetbuttcap%
\pgfsetroundjoin%
\pgfsetlinewidth{0.250937pt}%
\definecolor{currentstroke}{rgb}{0.000000,0.000000,0.000000}%
\pgfsetstrokecolor{currentstroke}%
\pgfsetdash{}{0pt}%
\pgfpathmoveto{\pgfqpoint{0.625000in}{0.942794in}}%
\pgfpathlineto{\pgfqpoint{0.633645in}{0.935965in}}%
\pgfpathlineto{\pgfqpoint{0.625000in}{0.929075in}}%
\pgfusepath{stroke}%
\end{pgfscope}%
\begin{pgfscope}%
\pgfpathrectangle{\pgfqpoint{0.625000in}{0.550000in}}{\pgfqpoint{3.875000in}{3.850000in}} %
\pgfusepath{clip}%
\pgfsetbuttcap%
\pgfsetroundjoin%
\pgfsetlinewidth{0.250937pt}%
\definecolor{currentstroke}{rgb}{0.000000,0.000000,0.000000}%
\pgfsetstrokecolor{currentstroke}%
\pgfsetdash{}{0pt}%
\pgfpathmoveto{\pgfqpoint{0.625000in}{1.097144in}}%
\pgfpathlineto{\pgfqpoint{0.632083in}{1.090351in}}%
\pgfpathlineto{\pgfqpoint{0.625000in}{1.083583in}}%
\pgfusepath{stroke}%
\end{pgfscope}%
\begin{pgfscope}%
\pgfpathrectangle{\pgfqpoint{0.625000in}{0.550000in}}{\pgfqpoint{3.875000in}{3.850000in}} %
\pgfusepath{clip}%
\pgfsetbuttcap%
\pgfsetroundjoin%
\pgfsetlinewidth{0.250937pt}%
\definecolor{currentstroke}{rgb}{0.000000,0.000000,0.000000}%
\pgfsetstrokecolor{currentstroke}%
\pgfsetdash{}{0pt}%
\pgfpathmoveto{\pgfqpoint{0.625000in}{1.251723in}}%
\pgfpathlineto{\pgfqpoint{0.632079in}{1.244737in}}%
\pgfpathlineto{\pgfqpoint{0.625000in}{1.237943in}}%
\pgfusepath{stroke}%
\end{pgfscope}%
\begin{pgfscope}%
\pgfpathrectangle{\pgfqpoint{0.625000in}{0.550000in}}{\pgfqpoint{3.875000in}{3.850000in}} %
\pgfusepath{clip}%
\pgfsetbuttcap%
\pgfsetroundjoin%
\pgfsetlinewidth{0.250937pt}%
\definecolor{currentstroke}{rgb}{0.000000,0.000000,0.000000}%
\pgfsetstrokecolor{currentstroke}%
\pgfsetdash{}{0pt}%
\pgfpathmoveto{\pgfqpoint{0.625000in}{1.405786in}}%
\pgfpathlineto{\pgfqpoint{0.632005in}{1.399123in}}%
\pgfpathlineto{\pgfqpoint{0.625000in}{1.392456in}}%
\pgfusepath{stroke}%
\end{pgfscope}%
\begin{pgfscope}%
\pgfpathrectangle{\pgfqpoint{0.625000in}{0.550000in}}{\pgfqpoint{3.875000in}{3.850000in}} %
\pgfusepath{clip}%
\pgfsetbuttcap%
\pgfsetroundjoin%
\pgfsetlinewidth{0.250937pt}%
\definecolor{currentstroke}{rgb}{0.000000,0.000000,0.000000}%
\pgfsetstrokecolor{currentstroke}%
\pgfsetdash{}{0pt}%
\pgfpathmoveto{\pgfqpoint{0.625000in}{1.560173in}}%
\pgfpathlineto{\pgfqpoint{0.631995in}{1.553509in}}%
\pgfpathlineto{\pgfqpoint{0.625000in}{1.546628in}}%
\pgfusepath{stroke}%
\end{pgfscope}%
\begin{pgfscope}%
\pgfpathrectangle{\pgfqpoint{0.625000in}{0.550000in}}{\pgfqpoint{3.875000in}{3.850000in}} %
\pgfusepath{clip}%
\pgfsetbuttcap%
\pgfsetroundjoin%
\pgfsetlinewidth{0.250937pt}%
\definecolor{currentstroke}{rgb}{0.000000,0.000000,0.000000}%
\pgfsetstrokecolor{currentstroke}%
\pgfsetdash{}{0pt}%
\pgfpathmoveto{\pgfqpoint{0.625000in}{1.714541in}}%
\pgfpathlineto{\pgfqpoint{0.633563in}{1.707895in}}%
\pgfpathlineto{\pgfqpoint{0.625000in}{1.701207in}}%
\pgfusepath{stroke}%
\end{pgfscope}%
\begin{pgfscope}%
\pgfpathrectangle{\pgfqpoint{0.625000in}{0.550000in}}{\pgfqpoint{3.875000in}{3.850000in}} %
\pgfusepath{clip}%
\pgfsetbuttcap%
\pgfsetroundjoin%
\pgfsetlinewidth{0.250937pt}%
\definecolor{currentstroke}{rgb}{0.000000,0.000000,0.000000}%
\pgfsetstrokecolor{currentstroke}%
\pgfsetdash{}{0pt}%
\pgfpathmoveto{\pgfqpoint{0.625000in}{1.868960in}}%
\pgfpathlineto{\pgfqpoint{0.631932in}{1.862281in}}%
\pgfpathlineto{\pgfqpoint{0.625000in}{1.855740in}}%
\pgfusepath{stroke}%
\end{pgfscope}%
\begin{pgfscope}%
\pgfpathrectangle{\pgfqpoint{0.625000in}{0.550000in}}{\pgfqpoint{3.875000in}{3.850000in}} %
\pgfusepath{clip}%
\pgfsetbuttcap%
\pgfsetroundjoin%
\pgfsetlinewidth{0.250937pt}%
\definecolor{currentstroke}{rgb}{0.000000,0.000000,0.000000}%
\pgfsetstrokecolor{currentstroke}%
\pgfsetdash{}{0pt}%
\pgfpathmoveto{\pgfqpoint{0.625000in}{2.023557in}}%
\pgfpathlineto{\pgfqpoint{0.631896in}{2.016667in}}%
\pgfpathlineto{\pgfqpoint{0.625000in}{2.009833in}}%
\pgfusepath{stroke}%
\end{pgfscope}%
\begin{pgfscope}%
\pgfpathrectangle{\pgfqpoint{0.625000in}{0.550000in}}{\pgfqpoint{3.875000in}{3.850000in}} %
\pgfusepath{clip}%
\pgfsetbuttcap%
\pgfsetroundjoin%
\pgfsetlinewidth{0.250937pt}%
\definecolor{currentstroke}{rgb}{0.000000,0.000000,0.000000}%
\pgfsetstrokecolor{currentstroke}%
\pgfsetdash{}{0pt}%
\pgfpathmoveto{\pgfqpoint{0.625000in}{2.177604in}}%
\pgfpathlineto{\pgfqpoint{0.631929in}{2.171053in}}%
\pgfpathlineto{\pgfqpoint{0.625000in}{2.164313in}}%
\pgfusepath{stroke}%
\end{pgfscope}%
\begin{pgfscope}%
\pgfpathrectangle{\pgfqpoint{0.625000in}{0.550000in}}{\pgfqpoint{3.875000in}{3.850000in}} %
\pgfusepath{clip}%
\pgfsetbuttcap%
\pgfsetroundjoin%
\pgfsetlinewidth{0.250937pt}%
\definecolor{currentstroke}{rgb}{0.000000,0.000000,0.000000}%
\pgfsetstrokecolor{currentstroke}%
\pgfsetdash{}{0pt}%
\pgfpathmoveto{\pgfqpoint{0.625000in}{2.331942in}}%
\pgfpathlineto{\pgfqpoint{0.631942in}{2.325439in}}%
\pgfpathlineto{\pgfqpoint{0.625000in}{2.318480in}}%
\pgfusepath{stroke}%
\end{pgfscope}%
\begin{pgfscope}%
\pgfpathrectangle{\pgfqpoint{0.625000in}{0.550000in}}{\pgfqpoint{3.875000in}{3.850000in}} %
\pgfusepath{clip}%
\pgfsetbuttcap%
\pgfsetroundjoin%
\pgfsetlinewidth{0.250937pt}%
\definecolor{currentstroke}{rgb}{0.000000,0.000000,0.000000}%
\pgfsetstrokecolor{currentstroke}%
\pgfsetdash{}{0pt}%
\pgfpathmoveto{\pgfqpoint{0.625000in}{2.487292in}}%
\pgfpathlineto{\pgfqpoint{0.627046in}{2.489474in}}%
\pgfpathlineto{\pgfqpoint{0.627779in}{2.499123in}}%
\pgfpathlineto{\pgfqpoint{0.630732in}{2.508772in}}%
\pgfpathlineto{\pgfqpoint{0.634601in}{2.518421in}}%
\pgfpathlineto{\pgfqpoint{0.634712in}{2.518752in}}%
\pgfpathlineto{\pgfqpoint{0.640730in}{2.528070in}}%
\pgfpathlineto{\pgfqpoint{0.644424in}{2.534170in}}%
\pgfpathlineto{\pgfqpoint{0.648334in}{2.537719in}}%
\pgfpathlineto{\pgfqpoint{0.654135in}{2.544401in}}%
\pgfpathlineto{\pgfqpoint{0.658480in}{2.547368in}}%
\pgfpathlineto{\pgfqpoint{0.663847in}{2.552124in}}%
\pgfpathlineto{\pgfqpoint{0.672541in}{2.557018in}}%
\pgfpathlineto{\pgfqpoint{0.673559in}{2.557753in}}%
\pgfpathlineto{\pgfqpoint{0.683271in}{2.562340in}}%
\pgfpathlineto{\pgfqpoint{0.692982in}{2.565362in}}%
\pgfpathlineto{\pgfqpoint{0.699877in}{2.566667in}}%
\pgfpathlineto{\pgfqpoint{0.702694in}{2.567344in}}%
\pgfpathlineto{\pgfqpoint{0.712406in}{2.568364in}}%
\pgfpathlineto{\pgfqpoint{0.722118in}{2.568305in}}%
\pgfpathlineto{\pgfqpoint{0.731830in}{2.567189in}}%
\pgfpathlineto{\pgfqpoint{0.734341in}{2.566667in}}%
\pgfpathlineto{\pgfqpoint{0.741541in}{2.565061in}}%
\pgfpathlineto{\pgfqpoint{0.751253in}{2.561765in}}%
\pgfpathlineto{\pgfqpoint{0.760965in}{2.557133in}}%
\pgfpathlineto{\pgfqpoint{0.761174in}{2.557018in}}%
\pgfpathlineto{\pgfqpoint{0.770677in}{2.550939in}}%
\pgfpathlineto{\pgfqpoint{0.775361in}{2.547368in}}%
\pgfpathlineto{\pgfqpoint{0.780388in}{2.542707in}}%
\pgfpathlineto{\pgfqpoint{0.785299in}{2.537719in}}%
\pgfpathlineto{\pgfqpoint{0.790100in}{2.531503in}}%
\pgfpathlineto{\pgfqpoint{0.792664in}{2.528070in}}%
\pgfpathlineto{\pgfqpoint{0.798184in}{2.518421in}}%
\pgfpathlineto{\pgfqpoint{0.799812in}{2.514491in}}%
\pgfpathlineto{\pgfqpoint{0.802264in}{2.508772in}}%
\pgfpathlineto{\pgfqpoint{0.805083in}{2.499123in}}%
\pgfpathlineto{\pgfqpoint{0.806692in}{2.489474in}}%
\pgfpathlineto{\pgfqpoint{0.807216in}{2.479825in}}%
\pgfpathlineto{\pgfqpoint{0.806692in}{2.470175in}}%
\pgfpathlineto{\pgfqpoint{0.805083in}{2.460526in}}%
\pgfpathlineto{\pgfqpoint{0.802264in}{2.450877in}}%
\pgfpathlineto{\pgfqpoint{0.799812in}{2.445159in}}%
\pgfpathlineto{\pgfqpoint{0.798184in}{2.441228in}}%
\pgfpathlineto{\pgfqpoint{0.792664in}{2.431579in}}%
\pgfpathlineto{\pgfqpoint{0.790100in}{2.428146in}}%
\pgfpathlineto{\pgfqpoint{0.785299in}{2.421930in}}%
\pgfpathlineto{\pgfqpoint{0.780388in}{2.416943in}}%
\pgfpathlineto{\pgfqpoint{0.775361in}{2.412281in}}%
\pgfpathlineto{\pgfqpoint{0.770677in}{2.408710in}}%
\pgfpathlineto{\pgfqpoint{0.761174in}{2.402632in}}%
\pgfpathlineto{\pgfqpoint{0.760965in}{2.402517in}}%
\pgfpathlineto{\pgfqpoint{0.751253in}{2.397884in}}%
\pgfpathlineto{\pgfqpoint{0.741541in}{2.394589in}}%
\pgfpathlineto{\pgfqpoint{0.734341in}{2.392982in}}%
\pgfpathlineto{\pgfqpoint{0.731830in}{2.392460in}}%
\pgfpathlineto{\pgfqpoint{0.722118in}{2.391344in}}%
\pgfpathlineto{\pgfqpoint{0.712406in}{2.391285in}}%
\pgfpathlineto{\pgfqpoint{0.702694in}{2.392305in}}%
\pgfpathlineto{\pgfqpoint{0.699877in}{2.392982in}}%
\pgfpathlineto{\pgfqpoint{0.692982in}{2.394287in}}%
\pgfpathlineto{\pgfqpoint{0.683271in}{2.397309in}}%
\pgfpathlineto{\pgfqpoint{0.673559in}{2.401896in}}%
\pgfpathlineto{\pgfqpoint{0.672541in}{2.402632in}}%
\pgfpathlineto{\pgfqpoint{0.663847in}{2.407525in}}%
\pgfpathlineto{\pgfqpoint{0.658480in}{2.412281in}}%
\pgfpathlineto{\pgfqpoint{0.654135in}{2.415248in}}%
\pgfpathlineto{\pgfqpoint{0.648334in}{2.421930in}}%
\pgfpathlineto{\pgfqpoint{0.644424in}{2.425479in}}%
\pgfpathlineto{\pgfqpoint{0.640730in}{2.431579in}}%
\pgfpathlineto{\pgfqpoint{0.634712in}{2.440897in}}%
\pgfpathlineto{\pgfqpoint{0.634601in}{2.441228in}}%
\pgfpathlineto{\pgfqpoint{0.630732in}{2.450877in}}%
\pgfpathlineto{\pgfqpoint{0.627786in}{2.460526in}}%
\pgfpathlineto{\pgfqpoint{0.627046in}{2.470175in}}%
\pgfpathlineto{\pgfqpoint{0.625000in}{2.472357in}}%
\pgfusepath{stroke}%
\end{pgfscope}%
\begin{pgfscope}%
\pgfpathrectangle{\pgfqpoint{0.625000in}{0.550000in}}{\pgfqpoint{3.875000in}{3.850000in}} %
\pgfusepath{clip}%
\pgfsetbuttcap%
\pgfsetroundjoin%
\pgfsetlinewidth{0.250937pt}%
\definecolor{currentstroke}{rgb}{0.000000,0.000000,0.000000}%
\pgfsetstrokecolor{currentstroke}%
\pgfsetdash{}{0pt}%
\pgfpathmoveto{\pgfqpoint{0.625000in}{2.641164in}}%
\pgfpathlineto{\pgfqpoint{0.631937in}{2.634211in}}%
\pgfpathlineto{\pgfqpoint{0.625000in}{2.627713in}}%
\pgfusepath{stroke}%
\end{pgfscope}%
\begin{pgfscope}%
\pgfpathrectangle{\pgfqpoint{0.625000in}{0.550000in}}{\pgfqpoint{3.875000in}{3.850000in}} %
\pgfusepath{clip}%
\pgfsetbuttcap%
\pgfsetroundjoin%
\pgfsetlinewidth{0.250937pt}%
\definecolor{currentstroke}{rgb}{0.000000,0.000000,0.000000}%
\pgfsetstrokecolor{currentstroke}%
\pgfsetdash{}{0pt}%
\pgfpathmoveto{\pgfqpoint{0.625000in}{2.795318in}}%
\pgfpathlineto{\pgfqpoint{0.631911in}{2.788596in}}%
\pgfpathlineto{\pgfqpoint{0.625000in}{2.782063in}}%
\pgfusepath{stroke}%
\end{pgfscope}%
\begin{pgfscope}%
\pgfpathrectangle{\pgfqpoint{0.625000in}{0.550000in}}{\pgfqpoint{3.875000in}{3.850000in}} %
\pgfusepath{clip}%
\pgfsetbuttcap%
\pgfsetroundjoin%
\pgfsetlinewidth{0.250937pt}%
\definecolor{currentstroke}{rgb}{0.000000,0.000000,0.000000}%
\pgfsetstrokecolor{currentstroke}%
\pgfsetdash{}{0pt}%
\pgfpathmoveto{\pgfqpoint{0.625000in}{2.949767in}}%
\pgfpathlineto{\pgfqpoint{0.631848in}{2.942982in}}%
\pgfpathlineto{\pgfqpoint{0.625000in}{2.936141in}}%
\pgfusepath{stroke}%
\end{pgfscope}%
\begin{pgfscope}%
\pgfpathrectangle{\pgfqpoint{0.625000in}{0.550000in}}{\pgfqpoint{3.875000in}{3.850000in}} %
\pgfusepath{clip}%
\pgfsetbuttcap%
\pgfsetroundjoin%
\pgfsetlinewidth{0.250937pt}%
\definecolor{currentstroke}{rgb}{0.000000,0.000000,0.000000}%
\pgfsetstrokecolor{currentstroke}%
\pgfsetdash{}{0pt}%
\pgfpathmoveto{\pgfqpoint{0.625000in}{3.103869in}}%
\pgfpathlineto{\pgfqpoint{0.631894in}{3.097368in}}%
\pgfpathlineto{\pgfqpoint{0.625000in}{3.090729in}}%
\pgfusepath{stroke}%
\end{pgfscope}%
\begin{pgfscope}%
\pgfpathrectangle{\pgfqpoint{0.625000in}{0.550000in}}{\pgfqpoint{3.875000in}{3.850000in}} %
\pgfusepath{clip}%
\pgfsetbuttcap%
\pgfsetroundjoin%
\pgfsetlinewidth{0.250937pt}%
\definecolor{currentstroke}{rgb}{0.000000,0.000000,0.000000}%
\pgfsetstrokecolor{currentstroke}%
\pgfsetdash{}{0pt}%
\pgfpathmoveto{\pgfqpoint{0.625000in}{3.258300in}}%
\pgfpathlineto{\pgfqpoint{0.633492in}{3.251754in}}%
\pgfpathlineto{\pgfqpoint{0.625000in}{3.245252in}}%
\pgfusepath{stroke}%
\end{pgfscope}%
\begin{pgfscope}%
\pgfpathrectangle{\pgfqpoint{0.625000in}{0.550000in}}{\pgfqpoint{3.875000in}{3.850000in}} %
\pgfusepath{clip}%
\pgfsetbuttcap%
\pgfsetroundjoin%
\pgfsetlinewidth{0.250937pt}%
\definecolor{currentstroke}{rgb}{0.000000,0.000000,0.000000}%
\pgfsetstrokecolor{currentstroke}%
\pgfsetdash{}{0pt}%
\pgfpathmoveto{\pgfqpoint{0.625000in}{3.412926in}}%
\pgfpathlineto{\pgfqpoint{0.631900in}{3.406140in}}%
\pgfpathlineto{\pgfqpoint{0.625000in}{3.399576in}}%
\pgfusepath{stroke}%
\end{pgfscope}%
\begin{pgfscope}%
\pgfpathrectangle{\pgfqpoint{0.625000in}{0.550000in}}{\pgfqpoint{3.875000in}{3.850000in}} %
\pgfusepath{clip}%
\pgfsetbuttcap%
\pgfsetroundjoin%
\pgfsetlinewidth{0.250937pt}%
\definecolor{currentstroke}{rgb}{0.000000,0.000000,0.000000}%
\pgfsetstrokecolor{currentstroke}%
\pgfsetdash{}{0pt}%
\pgfpathmoveto{\pgfqpoint{0.625000in}{3.567075in}}%
\pgfpathlineto{\pgfqpoint{0.631893in}{3.560526in}}%
\pgfpathlineto{\pgfqpoint{0.625000in}{3.553981in}}%
\pgfusepath{stroke}%
\end{pgfscope}%
\begin{pgfscope}%
\pgfpathrectangle{\pgfqpoint{0.625000in}{0.550000in}}{\pgfqpoint{3.875000in}{3.850000in}} %
\pgfusepath{clip}%
\pgfsetbuttcap%
\pgfsetroundjoin%
\pgfsetlinewidth{0.250937pt}%
\definecolor{currentstroke}{rgb}{0.000000,0.000000,0.000000}%
\pgfsetstrokecolor{currentstroke}%
\pgfsetdash{}{0pt}%
\pgfpathmoveto{\pgfqpoint{0.625000in}{3.721532in}}%
\pgfpathlineto{\pgfqpoint{0.631912in}{3.714912in}}%
\pgfpathlineto{\pgfqpoint{0.625000in}{3.708093in}}%
\pgfusepath{stroke}%
\end{pgfscope}%
\begin{pgfscope}%
\pgfpathrectangle{\pgfqpoint{0.625000in}{0.550000in}}{\pgfqpoint{3.875000in}{3.850000in}} %
\pgfusepath{clip}%
\pgfsetbuttcap%
\pgfsetroundjoin%
\pgfsetlinewidth{0.250937pt}%
\definecolor{currentstroke}{rgb}{0.000000,0.000000,0.000000}%
\pgfsetstrokecolor{currentstroke}%
\pgfsetdash{}{0pt}%
\pgfpathmoveto{\pgfqpoint{0.625000in}{3.875776in}}%
\pgfpathlineto{\pgfqpoint{0.631807in}{3.869298in}}%
\pgfpathlineto{\pgfqpoint{0.625000in}{3.862794in}}%
\pgfusepath{stroke}%
\end{pgfscope}%
\begin{pgfscope}%
\pgfpathrectangle{\pgfqpoint{0.625000in}{0.550000in}}{\pgfqpoint{3.875000in}{3.850000in}} %
\pgfusepath{clip}%
\pgfsetbuttcap%
\pgfsetroundjoin%
\pgfsetlinewidth{0.250937pt}%
\definecolor{currentstroke}{rgb}{0.000000,0.000000,0.000000}%
\pgfsetstrokecolor{currentstroke}%
\pgfsetdash{}{0pt}%
\pgfpathmoveto{\pgfqpoint{0.625000in}{4.030316in}}%
\pgfpathlineto{\pgfqpoint{0.633517in}{4.023684in}}%
\pgfpathlineto{\pgfqpoint{0.625000in}{4.017117in}}%
\pgfusepath{stroke}%
\end{pgfscope}%
\begin{pgfscope}%
\pgfpathrectangle{\pgfqpoint{0.625000in}{0.550000in}}{\pgfqpoint{3.875000in}{3.850000in}} %
\pgfusepath{clip}%
\pgfsetbuttcap%
\pgfsetroundjoin%
\pgfsetlinewidth{0.250937pt}%
\definecolor{currentstroke}{rgb}{0.000000,0.000000,0.000000}%
\pgfsetstrokecolor{currentstroke}%
\pgfsetdash{}{0pt}%
\pgfpathmoveto{\pgfqpoint{0.625000in}{4.184606in}}%
\pgfpathlineto{\pgfqpoint{0.631790in}{4.178070in}}%
\pgfpathlineto{\pgfqpoint{0.625000in}{4.171840in}}%
\pgfusepath{stroke}%
\end{pgfscope}%
\begin{pgfscope}%
\pgfpathrectangle{\pgfqpoint{0.625000in}{0.550000in}}{\pgfqpoint{3.875000in}{3.850000in}} %
\pgfusepath{clip}%
\pgfsetbuttcap%
\pgfsetroundjoin%
\pgfsetlinewidth{0.250937pt}%
\definecolor{currentstroke}{rgb}{0.000000,0.000000,0.000000}%
\pgfsetstrokecolor{currentstroke}%
\pgfsetdash{}{0pt}%
\pgfpathmoveto{\pgfqpoint{0.625000in}{4.338972in}}%
\pgfpathlineto{\pgfqpoint{0.631769in}{4.332456in}}%
\pgfpathlineto{\pgfqpoint{0.625000in}{4.325918in}}%
\pgfusepath{stroke}%
\end{pgfscope}%
\begin{pgfscope}%
\pgfpathrectangle{\pgfqpoint{0.625000in}{0.550000in}}{\pgfqpoint{3.875000in}{3.850000in}} %
\pgfusepath{clip}%
\pgfsetbuttcap%
\pgfsetroundjoin%
\pgfsetlinewidth{0.250937pt}%
\definecolor{currentstroke}{rgb}{0.000000,0.000000,0.000000}%
\pgfsetstrokecolor{currentstroke}%
\pgfsetdash{}{0pt}%
\pgfpathmoveto{\pgfqpoint{0.634712in}{1.182992in}}%
\pgfpathlineto{\pgfqpoint{0.632316in}{1.186842in}}%
\pgfpathlineto{\pgfqpoint{0.630213in}{1.196491in}}%
\pgfpathlineto{\pgfqpoint{0.634712in}{1.203250in}}%
\pgfpathlineto{\pgfqpoint{0.643791in}{1.196491in}}%
\pgfpathlineto{\pgfqpoint{0.641801in}{1.186842in}}%
\pgfpathlineto{\pgfqpoint{0.634712in}{1.182992in}}%
\pgfusepath{stroke}%
\end{pgfscope}%
\begin{pgfscope}%
\pgfpathrectangle{\pgfqpoint{0.625000in}{0.550000in}}{\pgfqpoint{3.875000in}{3.850000in}} %
\pgfusepath{clip}%
\pgfsetbuttcap%
\pgfsetroundjoin%
\pgfsetlinewidth{0.250937pt}%
\definecolor{currentstroke}{rgb}{0.000000,0.000000,0.000000}%
\pgfsetstrokecolor{currentstroke}%
\pgfsetdash{}{0pt}%
\pgfpathmoveto{\pgfqpoint{0.634712in}{3.756399in}}%
\pgfpathlineto{\pgfqpoint{0.630213in}{3.763158in}}%
\pgfpathlineto{\pgfqpoint{0.632316in}{3.772807in}}%
\pgfpathlineto{\pgfqpoint{0.634712in}{3.776657in}}%
\pgfpathlineto{\pgfqpoint{0.641801in}{3.772807in}}%
\pgfpathlineto{\pgfqpoint{0.643791in}{3.763158in}}%
\pgfpathlineto{\pgfqpoint{0.634712in}{3.756399in}}%
\pgfusepath{stroke}%
\end{pgfscope}%
\begin{pgfscope}%
\pgfpathrectangle{\pgfqpoint{0.625000in}{0.550000in}}{\pgfqpoint{3.875000in}{3.850000in}} %
\pgfusepath{clip}%
\pgfsetbuttcap%
\pgfsetroundjoin%
\pgfsetlinewidth{0.250937pt}%
\definecolor{currentstroke}{rgb}{0.000000,0.000000,0.000000}%
\pgfsetstrokecolor{currentstroke}%
\pgfsetdash{}{0pt}%
\pgfpathmoveto{\pgfqpoint{0.625000in}{0.634218in}}%
\pgfpathlineto{\pgfqpoint{0.632246in}{0.627193in}}%
\pgfpathlineto{\pgfqpoint{0.625000in}{0.620188in}}%
\pgfusepath{stroke}%
\end{pgfscope}%
\begin{pgfscope}%
\pgfpathrectangle{\pgfqpoint{0.625000in}{0.550000in}}{\pgfqpoint{3.875000in}{3.850000in}} %
\pgfusepath{clip}%
\pgfsetbuttcap%
\pgfsetroundjoin%
\pgfsetlinewidth{0.250937pt}%
\definecolor{currentstroke}{rgb}{0.000000,0.000000,0.000000}%
\pgfsetstrokecolor{currentstroke}%
\pgfsetdash{}{0pt}%
\pgfpathmoveto{\pgfqpoint{0.625000in}{0.788192in}}%
\pgfpathlineto{\pgfqpoint{0.632154in}{0.781579in}}%
\pgfpathlineto{\pgfqpoint{0.625000in}{0.774672in}}%
\pgfusepath{stroke}%
\end{pgfscope}%
\begin{pgfscope}%
\pgfpathrectangle{\pgfqpoint{0.625000in}{0.550000in}}{\pgfqpoint{3.875000in}{3.850000in}} %
\pgfusepath{clip}%
\pgfsetbuttcap%
\pgfsetroundjoin%
\pgfsetlinewidth{0.250937pt}%
\definecolor{currentstroke}{rgb}{0.000000,0.000000,0.000000}%
\pgfsetstrokecolor{currentstroke}%
\pgfsetdash{}{0pt}%
\pgfpathmoveto{\pgfqpoint{0.625000in}{0.942870in}}%
\pgfpathlineto{\pgfqpoint{0.633742in}{0.935965in}}%
\pgfpathlineto{\pgfqpoint{0.625000in}{0.928997in}}%
\pgfusepath{stroke}%
\end{pgfscope}%
\begin{pgfscope}%
\pgfpathrectangle{\pgfqpoint{0.625000in}{0.550000in}}{\pgfqpoint{3.875000in}{3.850000in}} %
\pgfusepath{clip}%
\pgfsetbuttcap%
\pgfsetroundjoin%
\pgfsetlinewidth{0.250937pt}%
\definecolor{currentstroke}{rgb}{0.000000,0.000000,0.000000}%
\pgfsetstrokecolor{currentstroke}%
\pgfsetdash{}{0pt}%
\pgfpathmoveto{\pgfqpoint{0.625000in}{1.097223in}}%
\pgfpathlineto{\pgfqpoint{0.632166in}{1.090351in}}%
\pgfpathlineto{\pgfqpoint{0.625000in}{1.083505in}}%
\pgfusepath{stroke}%
\end{pgfscope}%
\begin{pgfscope}%
\pgfpathrectangle{\pgfqpoint{0.625000in}{0.550000in}}{\pgfqpoint{3.875000in}{3.850000in}} %
\pgfusepath{clip}%
\pgfsetbuttcap%
\pgfsetroundjoin%
\pgfsetlinewidth{0.250937pt}%
\definecolor{currentstroke}{rgb}{0.000000,0.000000,0.000000}%
\pgfsetstrokecolor{currentstroke}%
\pgfsetdash{}{0pt}%
\pgfpathmoveto{\pgfqpoint{0.625000in}{1.251806in}}%
\pgfpathlineto{\pgfqpoint{0.632162in}{1.244737in}}%
\pgfpathlineto{\pgfqpoint{0.625000in}{1.237863in}}%
\pgfusepath{stroke}%
\end{pgfscope}%
\begin{pgfscope}%
\pgfpathrectangle{\pgfqpoint{0.625000in}{0.550000in}}{\pgfqpoint{3.875000in}{3.850000in}} %
\pgfusepath{clip}%
\pgfsetbuttcap%
\pgfsetroundjoin%
\pgfsetlinewidth{0.250937pt}%
\definecolor{currentstroke}{rgb}{0.000000,0.000000,0.000000}%
\pgfsetstrokecolor{currentstroke}%
\pgfsetdash{}{0pt}%
\pgfpathmoveto{\pgfqpoint{0.625000in}{1.405868in}}%
\pgfpathlineto{\pgfqpoint{0.632090in}{1.399123in}}%
\pgfpathlineto{\pgfqpoint{0.625000in}{1.392374in}}%
\pgfusepath{stroke}%
\end{pgfscope}%
\begin{pgfscope}%
\pgfpathrectangle{\pgfqpoint{0.625000in}{0.550000in}}{\pgfqpoint{3.875000in}{3.850000in}} %
\pgfusepath{clip}%
\pgfsetbuttcap%
\pgfsetroundjoin%
\pgfsetlinewidth{0.250937pt}%
\definecolor{currentstroke}{rgb}{0.000000,0.000000,0.000000}%
\pgfsetstrokecolor{currentstroke}%
\pgfsetdash{}{0pt}%
\pgfpathmoveto{\pgfqpoint{0.625000in}{1.560253in}}%
\pgfpathlineto{\pgfqpoint{0.632078in}{1.553509in}}%
\pgfpathlineto{\pgfqpoint{0.625000in}{1.546546in}}%
\pgfusepath{stroke}%
\end{pgfscope}%
\begin{pgfscope}%
\pgfpathrectangle{\pgfqpoint{0.625000in}{0.550000in}}{\pgfqpoint{3.875000in}{3.850000in}} %
\pgfusepath{clip}%
\pgfsetbuttcap%
\pgfsetroundjoin%
\pgfsetlinewidth{0.250937pt}%
\definecolor{currentstroke}{rgb}{0.000000,0.000000,0.000000}%
\pgfsetstrokecolor{currentstroke}%
\pgfsetdash{}{0pt}%
\pgfpathmoveto{\pgfqpoint{0.625000in}{1.714622in}}%
\pgfpathlineto{\pgfqpoint{0.633667in}{1.707895in}}%
\pgfpathlineto{\pgfqpoint{0.625000in}{1.701125in}}%
\pgfusepath{stroke}%
\end{pgfscope}%
\begin{pgfscope}%
\pgfpathrectangle{\pgfqpoint{0.625000in}{0.550000in}}{\pgfqpoint{3.875000in}{3.850000in}} %
\pgfusepath{clip}%
\pgfsetbuttcap%
\pgfsetroundjoin%
\pgfsetlinewidth{0.250937pt}%
\definecolor{currentstroke}{rgb}{0.000000,0.000000,0.000000}%
\pgfsetstrokecolor{currentstroke}%
\pgfsetdash{}{0pt}%
\pgfpathmoveto{\pgfqpoint{0.625000in}{1.869043in}}%
\pgfpathlineto{\pgfqpoint{0.632018in}{1.862281in}}%
\pgfpathlineto{\pgfqpoint{0.625000in}{1.855659in}}%
\pgfusepath{stroke}%
\end{pgfscope}%
\begin{pgfscope}%
\pgfpathrectangle{\pgfqpoint{0.625000in}{0.550000in}}{\pgfqpoint{3.875000in}{3.850000in}} %
\pgfusepath{clip}%
\pgfsetbuttcap%
\pgfsetroundjoin%
\pgfsetlinewidth{0.250937pt}%
\definecolor{currentstroke}{rgb}{0.000000,0.000000,0.000000}%
\pgfsetstrokecolor{currentstroke}%
\pgfsetdash{}{0pt}%
\pgfpathmoveto{\pgfqpoint{0.625000in}{2.023644in}}%
\pgfpathlineto{\pgfqpoint{0.631984in}{2.016667in}}%
\pgfpathlineto{\pgfqpoint{0.625000in}{2.009746in}}%
\pgfusepath{stroke}%
\end{pgfscope}%
\begin{pgfscope}%
\pgfpathrectangle{\pgfqpoint{0.625000in}{0.550000in}}{\pgfqpoint{3.875000in}{3.850000in}} %
\pgfusepath{clip}%
\pgfsetbuttcap%
\pgfsetroundjoin%
\pgfsetlinewidth{0.250937pt}%
\definecolor{currentstroke}{rgb}{0.000000,0.000000,0.000000}%
\pgfsetstrokecolor{currentstroke}%
\pgfsetdash{}{0pt}%
\pgfpathmoveto{\pgfqpoint{0.625000in}{2.177687in}}%
\pgfpathlineto{\pgfqpoint{0.632016in}{2.171053in}}%
\pgfpathlineto{\pgfqpoint{0.625000in}{2.164229in}}%
\pgfusepath{stroke}%
\end{pgfscope}%
\begin{pgfscope}%
\pgfpathrectangle{\pgfqpoint{0.625000in}{0.550000in}}{\pgfqpoint{3.875000in}{3.850000in}} %
\pgfusepath{clip}%
\pgfsetbuttcap%
\pgfsetroundjoin%
\pgfsetlinewidth{0.250937pt}%
\definecolor{currentstroke}{rgb}{0.000000,0.000000,0.000000}%
\pgfsetstrokecolor{currentstroke}%
\pgfsetdash{}{0pt}%
\pgfpathmoveto{\pgfqpoint{0.625000in}{2.332024in}}%
\pgfpathlineto{\pgfqpoint{0.632031in}{2.325439in}}%
\pgfpathlineto{\pgfqpoint{0.625000in}{2.318391in}}%
\pgfusepath{stroke}%
\end{pgfscope}%
\begin{pgfscope}%
\pgfpathrectangle{\pgfqpoint{0.625000in}{0.550000in}}{\pgfqpoint{3.875000in}{3.850000in}} %
\pgfusepath{clip}%
\pgfsetbuttcap%
\pgfsetroundjoin%
\pgfsetlinewidth{0.250937pt}%
\definecolor{currentstroke}{rgb}{0.000000,0.000000,0.000000}%
\pgfsetstrokecolor{currentstroke}%
\pgfsetdash{}{0pt}%
\pgfpathmoveto{\pgfqpoint{0.625000in}{2.487376in}}%
\pgfpathlineto{\pgfqpoint{0.626967in}{2.489474in}}%
\pgfpathlineto{\pgfqpoint{0.627700in}{2.499123in}}%
\pgfpathlineto{\pgfqpoint{0.630577in}{2.508772in}}%
\pgfpathlineto{\pgfqpoint{0.634338in}{2.518421in}}%
\pgfpathlineto{\pgfqpoint{0.634712in}{2.519532in}}%
\pgfpathlineto{\pgfqpoint{0.640226in}{2.528070in}}%
\pgfpathlineto{\pgfqpoint{0.644424in}{2.535003in}}%
\pgfpathlineto{\pgfqpoint{0.647417in}{2.537719in}}%
\pgfpathlineto{\pgfqpoint{0.654135in}{2.545457in}}%
\pgfpathlineto{\pgfqpoint{0.656934in}{2.547368in}}%
\pgfpathlineto{\pgfqpoint{0.663847in}{2.553494in}}%
\pgfpathlineto{\pgfqpoint{0.670106in}{2.557018in}}%
\pgfpathlineto{\pgfqpoint{0.673559in}{2.559512in}}%
\pgfpathlineto{\pgfqpoint{0.683271in}{2.564154in}}%
\pgfpathlineto{\pgfqpoint{0.691110in}{2.566667in}}%
\pgfpathlineto{\pgfqpoint{0.692982in}{2.567431in}}%
\pgfpathlineto{\pgfqpoint{0.702694in}{2.569823in}}%
\pgfpathlineto{\pgfqpoint{0.712406in}{2.571079in}}%
\pgfpathlineto{\pgfqpoint{0.722118in}{2.571322in}}%
\pgfpathlineto{\pgfqpoint{0.731830in}{2.570573in}}%
\pgfpathlineto{\pgfqpoint{0.741541in}{2.568793in}}%
\pgfpathlineto{\pgfqpoint{0.748930in}{2.566667in}}%
\pgfpathlineto{\pgfqpoint{0.751253in}{2.565936in}}%
\pgfpathlineto{\pgfqpoint{0.760965in}{2.561926in}}%
\pgfpathlineto{\pgfqpoint{0.769883in}{2.557018in}}%
\pgfpathlineto{\pgfqpoint{0.770677in}{2.556510in}}%
\pgfpathlineto{\pgfqpoint{0.780388in}{2.549380in}}%
\pgfpathlineto{\pgfqpoint{0.782801in}{2.547368in}}%
\pgfpathlineto{\pgfqpoint{0.790100in}{2.539929in}}%
\pgfpathlineto{\pgfqpoint{0.792122in}{2.537719in}}%
\pgfpathlineto{\pgfqpoint{0.799115in}{2.528070in}}%
\pgfpathlineto{\pgfqpoint{0.799812in}{2.526817in}}%
\pgfpathlineto{\pgfqpoint{0.804514in}{2.518421in}}%
\pgfpathlineto{\pgfqpoint{0.808361in}{2.508772in}}%
\pgfpathlineto{\pgfqpoint{0.809524in}{2.504476in}}%
\pgfpathlineto{\pgfqpoint{0.811062in}{2.499123in}}%
\pgfpathlineto{\pgfqpoint{0.812684in}{2.489474in}}%
\pgfpathlineto{\pgfqpoint{0.813213in}{2.479825in}}%
\pgfpathlineto{\pgfqpoint{0.812684in}{2.470175in}}%
\pgfpathlineto{\pgfqpoint{0.811062in}{2.460526in}}%
\pgfpathlineto{\pgfqpoint{0.809524in}{2.455173in}}%
\pgfpathlineto{\pgfqpoint{0.808361in}{2.450877in}}%
\pgfpathlineto{\pgfqpoint{0.804514in}{2.441228in}}%
\pgfpathlineto{\pgfqpoint{0.799812in}{2.432832in}}%
\pgfpathlineto{\pgfqpoint{0.799115in}{2.431579in}}%
\pgfpathlineto{\pgfqpoint{0.792122in}{2.421930in}}%
\pgfpathlineto{\pgfqpoint{0.790100in}{2.419720in}}%
\pgfpathlineto{\pgfqpoint{0.782801in}{2.412281in}}%
\pgfpathlineto{\pgfqpoint{0.780388in}{2.410269in}}%
\pgfpathlineto{\pgfqpoint{0.770677in}{2.403139in}}%
\pgfpathlineto{\pgfqpoint{0.769883in}{2.402632in}}%
\pgfpathlineto{\pgfqpoint{0.760965in}{2.397723in}}%
\pgfpathlineto{\pgfqpoint{0.751253in}{2.393713in}}%
\pgfpathlineto{\pgfqpoint{0.748930in}{2.392982in}}%
\pgfpathlineto{\pgfqpoint{0.741541in}{2.390856in}}%
\pgfpathlineto{\pgfqpoint{0.731830in}{2.389076in}}%
\pgfpathlineto{\pgfqpoint{0.722118in}{2.388327in}}%
\pgfpathlineto{\pgfqpoint{0.712406in}{2.388570in}}%
\pgfpathlineto{\pgfqpoint{0.702694in}{2.389826in}}%
\pgfpathlineto{\pgfqpoint{0.692982in}{2.392218in}}%
\pgfpathlineto{\pgfqpoint{0.691110in}{2.392982in}}%
\pgfpathlineto{\pgfqpoint{0.683271in}{2.395495in}}%
\pgfpathlineto{\pgfqpoint{0.673559in}{2.400137in}}%
\pgfpathlineto{\pgfqpoint{0.670106in}{2.402632in}}%
\pgfpathlineto{\pgfqpoint{0.663847in}{2.406155in}}%
\pgfpathlineto{\pgfqpoint{0.656934in}{2.412281in}}%
\pgfpathlineto{\pgfqpoint{0.654135in}{2.414192in}}%
\pgfpathlineto{\pgfqpoint{0.647417in}{2.421930in}}%
\pgfpathlineto{\pgfqpoint{0.644424in}{2.424646in}}%
\pgfpathlineto{\pgfqpoint{0.640226in}{2.431579in}}%
\pgfpathlineto{\pgfqpoint{0.634712in}{2.440117in}}%
\pgfpathlineto{\pgfqpoint{0.634338in}{2.441228in}}%
\pgfpathlineto{\pgfqpoint{0.630577in}{2.450877in}}%
\pgfpathlineto{\pgfqpoint{0.627707in}{2.460526in}}%
\pgfpathlineto{\pgfqpoint{0.626967in}{2.470175in}}%
\pgfpathlineto{\pgfqpoint{0.625000in}{2.472273in}}%
\pgfusepath{stroke}%
\end{pgfscope}%
\begin{pgfscope}%
\pgfpathrectangle{\pgfqpoint{0.625000in}{0.550000in}}{\pgfqpoint{3.875000in}{3.850000in}} %
\pgfusepath{clip}%
\pgfsetbuttcap%
\pgfsetroundjoin%
\pgfsetlinewidth{0.250937pt}%
\definecolor{currentstroke}{rgb}{0.000000,0.000000,0.000000}%
\pgfsetstrokecolor{currentstroke}%
\pgfsetdash{}{0pt}%
\pgfpathmoveto{\pgfqpoint{0.625000in}{2.641253in}}%
\pgfpathlineto{\pgfqpoint{0.632026in}{2.634211in}}%
\pgfpathlineto{\pgfqpoint{0.625000in}{2.627630in}}%
\pgfusepath{stroke}%
\end{pgfscope}%
\begin{pgfscope}%
\pgfpathrectangle{\pgfqpoint{0.625000in}{0.550000in}}{\pgfqpoint{3.875000in}{3.850000in}} %
\pgfusepath{clip}%
\pgfsetbuttcap%
\pgfsetroundjoin%
\pgfsetlinewidth{0.250937pt}%
\definecolor{currentstroke}{rgb}{0.000000,0.000000,0.000000}%
\pgfsetstrokecolor{currentstroke}%
\pgfsetdash{}{0pt}%
\pgfpathmoveto{\pgfqpoint{0.625000in}{2.795403in}}%
\pgfpathlineto{\pgfqpoint{0.631999in}{2.788596in}}%
\pgfpathlineto{\pgfqpoint{0.625000in}{2.781980in}}%
\pgfusepath{stroke}%
\end{pgfscope}%
\begin{pgfscope}%
\pgfpathrectangle{\pgfqpoint{0.625000in}{0.550000in}}{\pgfqpoint{3.875000in}{3.850000in}} %
\pgfusepath{clip}%
\pgfsetbuttcap%
\pgfsetroundjoin%
\pgfsetlinewidth{0.250937pt}%
\definecolor{currentstroke}{rgb}{0.000000,0.000000,0.000000}%
\pgfsetstrokecolor{currentstroke}%
\pgfsetdash{}{0pt}%
\pgfpathmoveto{\pgfqpoint{0.625000in}{2.949855in}}%
\pgfpathlineto{\pgfqpoint{0.631936in}{2.942982in}}%
\pgfpathlineto{\pgfqpoint{0.625000in}{2.936052in}}%
\pgfusepath{stroke}%
\end{pgfscope}%
\begin{pgfscope}%
\pgfpathrectangle{\pgfqpoint{0.625000in}{0.550000in}}{\pgfqpoint{3.875000in}{3.850000in}} %
\pgfusepath{clip}%
\pgfsetbuttcap%
\pgfsetroundjoin%
\pgfsetlinewidth{0.250937pt}%
\definecolor{currentstroke}{rgb}{0.000000,0.000000,0.000000}%
\pgfsetstrokecolor{currentstroke}%
\pgfsetdash{}{0pt}%
\pgfpathmoveto{\pgfqpoint{0.625000in}{3.103951in}}%
\pgfpathlineto{\pgfqpoint{0.631981in}{3.097368in}}%
\pgfpathlineto{\pgfqpoint{0.625000in}{3.090645in}}%
\pgfusepath{stroke}%
\end{pgfscope}%
\begin{pgfscope}%
\pgfpathrectangle{\pgfqpoint{0.625000in}{0.550000in}}{\pgfqpoint{3.875000in}{3.850000in}} %
\pgfusepath{clip}%
\pgfsetbuttcap%
\pgfsetroundjoin%
\pgfsetlinewidth{0.250937pt}%
\definecolor{currentstroke}{rgb}{0.000000,0.000000,0.000000}%
\pgfsetstrokecolor{currentstroke}%
\pgfsetdash{}{0pt}%
\pgfpathmoveto{\pgfqpoint{0.625000in}{3.258385in}}%
\pgfpathlineto{\pgfqpoint{0.633603in}{3.251754in}}%
\pgfpathlineto{\pgfqpoint{0.625000in}{3.245167in}}%
\pgfusepath{stroke}%
\end{pgfscope}%
\begin{pgfscope}%
\pgfpathrectangle{\pgfqpoint{0.625000in}{0.550000in}}{\pgfqpoint{3.875000in}{3.850000in}} %
\pgfusepath{clip}%
\pgfsetbuttcap%
\pgfsetroundjoin%
\pgfsetlinewidth{0.250937pt}%
\definecolor{currentstroke}{rgb}{0.000000,0.000000,0.000000}%
\pgfsetstrokecolor{currentstroke}%
\pgfsetdash{}{0pt}%
\pgfpathmoveto{\pgfqpoint{0.625000in}{3.413011in}}%
\pgfpathlineto{\pgfqpoint{0.631987in}{3.406140in}}%
\pgfpathlineto{\pgfqpoint{0.625000in}{3.399493in}}%
\pgfusepath{stroke}%
\end{pgfscope}%
\begin{pgfscope}%
\pgfpathrectangle{\pgfqpoint{0.625000in}{0.550000in}}{\pgfqpoint{3.875000in}{3.850000in}} %
\pgfusepath{clip}%
\pgfsetbuttcap%
\pgfsetroundjoin%
\pgfsetlinewidth{0.250937pt}%
\definecolor{currentstroke}{rgb}{0.000000,0.000000,0.000000}%
\pgfsetstrokecolor{currentstroke}%
\pgfsetdash{}{0pt}%
\pgfpathmoveto{\pgfqpoint{0.625000in}{3.567160in}}%
\pgfpathlineto{\pgfqpoint{0.631982in}{3.560526in}}%
\pgfpathlineto{\pgfqpoint{0.625000in}{3.553896in}}%
\pgfusepath{stroke}%
\end{pgfscope}%
\begin{pgfscope}%
\pgfpathrectangle{\pgfqpoint{0.625000in}{0.550000in}}{\pgfqpoint{3.875000in}{3.850000in}} %
\pgfusepath{clip}%
\pgfsetbuttcap%
\pgfsetroundjoin%
\pgfsetlinewidth{0.250937pt}%
\definecolor{currentstroke}{rgb}{0.000000,0.000000,0.000000}%
\pgfsetstrokecolor{currentstroke}%
\pgfsetdash{}{0pt}%
\pgfpathmoveto{\pgfqpoint{0.625000in}{3.721617in}}%
\pgfpathlineto{\pgfqpoint{0.632001in}{3.714912in}}%
\pgfpathlineto{\pgfqpoint{0.625000in}{3.708005in}}%
\pgfusepath{stroke}%
\end{pgfscope}%
\begin{pgfscope}%
\pgfpathrectangle{\pgfqpoint{0.625000in}{0.550000in}}{\pgfqpoint{3.875000in}{3.850000in}} %
\pgfusepath{clip}%
\pgfsetbuttcap%
\pgfsetroundjoin%
\pgfsetlinewidth{0.250937pt}%
\definecolor{currentstroke}{rgb}{0.000000,0.000000,0.000000}%
\pgfsetstrokecolor{currentstroke}%
\pgfsetdash{}{0pt}%
\pgfpathmoveto{\pgfqpoint{0.625000in}{3.875863in}}%
\pgfpathlineto{\pgfqpoint{0.631899in}{3.869298in}}%
\pgfpathlineto{\pgfqpoint{0.625000in}{3.862707in}}%
\pgfusepath{stroke}%
\end{pgfscope}%
\begin{pgfscope}%
\pgfpathrectangle{\pgfqpoint{0.625000in}{0.550000in}}{\pgfqpoint{3.875000in}{3.850000in}} %
\pgfusepath{clip}%
\pgfsetbuttcap%
\pgfsetroundjoin%
\pgfsetlinewidth{0.250937pt}%
\definecolor{currentstroke}{rgb}{0.000000,0.000000,0.000000}%
\pgfsetstrokecolor{currentstroke}%
\pgfsetdash{}{0pt}%
\pgfpathmoveto{\pgfqpoint{0.625000in}{4.030400in}}%
\pgfpathlineto{\pgfqpoint{0.633626in}{4.023684in}}%
\pgfpathlineto{\pgfqpoint{0.625000in}{4.017033in}}%
\pgfusepath{stroke}%
\end{pgfscope}%
\begin{pgfscope}%
\pgfpathrectangle{\pgfqpoint{0.625000in}{0.550000in}}{\pgfqpoint{3.875000in}{3.850000in}} %
\pgfusepath{clip}%
\pgfsetbuttcap%
\pgfsetroundjoin%
\pgfsetlinewidth{0.250937pt}%
\definecolor{currentstroke}{rgb}{0.000000,0.000000,0.000000}%
\pgfsetstrokecolor{currentstroke}%
\pgfsetdash{}{0pt}%
\pgfpathmoveto{\pgfqpoint{0.625000in}{4.184695in}}%
\pgfpathlineto{\pgfqpoint{0.631882in}{4.178070in}}%
\pgfpathlineto{\pgfqpoint{0.625000in}{4.171756in}}%
\pgfusepath{stroke}%
\end{pgfscope}%
\begin{pgfscope}%
\pgfpathrectangle{\pgfqpoint{0.625000in}{0.550000in}}{\pgfqpoint{3.875000in}{3.850000in}} %
\pgfusepath{clip}%
\pgfsetbuttcap%
\pgfsetroundjoin%
\pgfsetlinewidth{0.250937pt}%
\definecolor{currentstroke}{rgb}{0.000000,0.000000,0.000000}%
\pgfsetstrokecolor{currentstroke}%
\pgfsetdash{}{0pt}%
\pgfpathmoveto{\pgfqpoint{0.625000in}{4.339060in}}%
\pgfpathlineto{\pgfqpoint{0.631860in}{4.332456in}}%
\pgfpathlineto{\pgfqpoint{0.625000in}{4.325830in}}%
\pgfusepath{stroke}%
\end{pgfscope}%
\begin{pgfscope}%
\pgfpathrectangle{\pgfqpoint{0.625000in}{0.550000in}}{\pgfqpoint{3.875000in}{3.850000in}} %
\pgfusepath{clip}%
\pgfsetbuttcap%
\pgfsetroundjoin%
\pgfsetlinewidth{0.250937pt}%
\definecolor{currentstroke}{rgb}{0.000000,0.000000,0.000000}%
\pgfsetstrokecolor{currentstroke}%
\pgfsetdash{}{0pt}%
\pgfpathmoveto{\pgfqpoint{0.634712in}{1.182658in}}%
\pgfpathlineto{\pgfqpoint{0.632109in}{1.186842in}}%
\pgfpathlineto{\pgfqpoint{0.630039in}{1.196491in}}%
\pgfpathlineto{\pgfqpoint{0.634712in}{1.203512in}}%
\pgfpathlineto{\pgfqpoint{0.644143in}{1.196491in}}%
\pgfpathlineto{\pgfqpoint{0.642416in}{1.186842in}}%
\pgfpathlineto{\pgfqpoint{0.634712in}{1.182658in}}%
\pgfusepath{stroke}%
\end{pgfscope}%
\begin{pgfscope}%
\pgfpathrectangle{\pgfqpoint{0.625000in}{0.550000in}}{\pgfqpoint{3.875000in}{3.850000in}} %
\pgfusepath{clip}%
\pgfsetbuttcap%
\pgfsetroundjoin%
\pgfsetlinewidth{0.250937pt}%
\definecolor{currentstroke}{rgb}{0.000000,0.000000,0.000000}%
\pgfsetstrokecolor{currentstroke}%
\pgfsetdash{}{0pt}%
\pgfpathmoveto{\pgfqpoint{0.634712in}{3.756137in}}%
\pgfpathlineto{\pgfqpoint{0.630039in}{3.763158in}}%
\pgfpathlineto{\pgfqpoint{0.632109in}{3.772807in}}%
\pgfpathlineto{\pgfqpoint{0.634712in}{3.776991in}}%
\pgfpathlineto{\pgfqpoint{0.642416in}{3.772807in}}%
\pgfpathlineto{\pgfqpoint{0.644143in}{3.763158in}}%
\pgfpathlineto{\pgfqpoint{0.634712in}{3.756137in}}%
\pgfusepath{stroke}%
\end{pgfscope}%
\begin{pgfscope}%
\pgfpathrectangle{\pgfqpoint{0.625000in}{0.550000in}}{\pgfqpoint{3.875000in}{3.850000in}} %
\pgfusepath{clip}%
\pgfsetbuttcap%
\pgfsetroundjoin%
\pgfsetlinewidth{0.250937pt}%
\definecolor{currentstroke}{rgb}{0.000000,0.000000,0.000000}%
\pgfsetstrokecolor{currentstroke}%
\pgfsetdash{}{0pt}%
\pgfpathmoveto{\pgfqpoint{0.625000in}{0.634294in}}%
\pgfpathlineto{\pgfqpoint{0.632324in}{0.627193in}}%
\pgfpathlineto{\pgfqpoint{0.625000in}{0.620112in}}%
\pgfusepath{stroke}%
\end{pgfscope}%
\begin{pgfscope}%
\pgfpathrectangle{\pgfqpoint{0.625000in}{0.550000in}}{\pgfqpoint{3.875000in}{3.850000in}} %
\pgfusepath{clip}%
\pgfsetbuttcap%
\pgfsetroundjoin%
\pgfsetlinewidth{0.250937pt}%
\definecolor{currentstroke}{rgb}{0.000000,0.000000,0.000000}%
\pgfsetstrokecolor{currentstroke}%
\pgfsetdash{}{0pt}%
\pgfpathmoveto{\pgfqpoint{0.625000in}{0.788269in}}%
\pgfpathlineto{\pgfqpoint{0.632237in}{0.781579in}}%
\pgfpathlineto{\pgfqpoint{0.625000in}{0.774592in}}%
\pgfusepath{stroke}%
\end{pgfscope}%
\begin{pgfscope}%
\pgfpathrectangle{\pgfqpoint{0.625000in}{0.550000in}}{\pgfqpoint{3.875000in}{3.850000in}} %
\pgfusepath{clip}%
\pgfsetbuttcap%
\pgfsetroundjoin%
\pgfsetlinewidth{0.250937pt}%
\definecolor{currentstroke}{rgb}{0.000000,0.000000,0.000000}%
\pgfsetstrokecolor{currentstroke}%
\pgfsetdash{}{0pt}%
\pgfpathmoveto{\pgfqpoint{0.625000in}{0.942947in}}%
\pgfpathlineto{\pgfqpoint{0.633839in}{0.935965in}}%
\pgfpathlineto{\pgfqpoint{0.625000in}{0.928920in}}%
\pgfusepath{stroke}%
\end{pgfscope}%
\begin{pgfscope}%
\pgfpathrectangle{\pgfqpoint{0.625000in}{0.550000in}}{\pgfqpoint{3.875000in}{3.850000in}} %
\pgfusepath{clip}%
\pgfsetbuttcap%
\pgfsetroundjoin%
\pgfsetlinewidth{0.250937pt}%
\definecolor{currentstroke}{rgb}{0.000000,0.000000,0.000000}%
\pgfsetstrokecolor{currentstroke}%
\pgfsetdash{}{0pt}%
\pgfpathmoveto{\pgfqpoint{0.625000in}{1.097302in}}%
\pgfpathlineto{\pgfqpoint{0.632248in}{1.090351in}}%
\pgfpathlineto{\pgfqpoint{0.625000in}{1.083426in}}%
\pgfusepath{stroke}%
\end{pgfscope}%
\begin{pgfscope}%
\pgfpathrectangle{\pgfqpoint{0.625000in}{0.550000in}}{\pgfqpoint{3.875000in}{3.850000in}} %
\pgfusepath{clip}%
\pgfsetbuttcap%
\pgfsetroundjoin%
\pgfsetlinewidth{0.250937pt}%
\definecolor{currentstroke}{rgb}{0.000000,0.000000,0.000000}%
\pgfsetstrokecolor{currentstroke}%
\pgfsetdash{}{0pt}%
\pgfpathmoveto{\pgfqpoint{0.625000in}{1.251888in}}%
\pgfpathlineto{\pgfqpoint{0.632246in}{1.244737in}}%
\pgfpathlineto{\pgfqpoint{0.625000in}{1.237782in}}%
\pgfusepath{stroke}%
\end{pgfscope}%
\begin{pgfscope}%
\pgfpathrectangle{\pgfqpoint{0.625000in}{0.550000in}}{\pgfqpoint{3.875000in}{3.850000in}} %
\pgfusepath{clip}%
\pgfsetbuttcap%
\pgfsetroundjoin%
\pgfsetlinewidth{0.250937pt}%
\definecolor{currentstroke}{rgb}{0.000000,0.000000,0.000000}%
\pgfsetstrokecolor{currentstroke}%
\pgfsetdash{}{0pt}%
\pgfpathmoveto{\pgfqpoint{0.625000in}{1.405949in}}%
\pgfpathlineto{\pgfqpoint{0.632176in}{1.399123in}}%
\pgfpathlineto{\pgfqpoint{0.625000in}{1.392293in}}%
\pgfusepath{stroke}%
\end{pgfscope}%
\begin{pgfscope}%
\pgfpathrectangle{\pgfqpoint{0.625000in}{0.550000in}}{\pgfqpoint{3.875000in}{3.850000in}} %
\pgfusepath{clip}%
\pgfsetbuttcap%
\pgfsetroundjoin%
\pgfsetlinewidth{0.250937pt}%
\definecolor{currentstroke}{rgb}{0.000000,0.000000,0.000000}%
\pgfsetstrokecolor{currentstroke}%
\pgfsetdash{}{0pt}%
\pgfpathmoveto{\pgfqpoint{0.625000in}{1.560333in}}%
\pgfpathlineto{\pgfqpoint{0.632162in}{1.553509in}}%
\pgfpathlineto{\pgfqpoint{0.625000in}{1.546463in}}%
\pgfusepath{stroke}%
\end{pgfscope}%
\begin{pgfscope}%
\pgfpathrectangle{\pgfqpoint{0.625000in}{0.550000in}}{\pgfqpoint{3.875000in}{3.850000in}} %
\pgfusepath{clip}%
\pgfsetbuttcap%
\pgfsetroundjoin%
\pgfsetlinewidth{0.250937pt}%
\definecolor{currentstroke}{rgb}{0.000000,0.000000,0.000000}%
\pgfsetstrokecolor{currentstroke}%
\pgfsetdash{}{0pt}%
\pgfpathmoveto{\pgfqpoint{0.625000in}{1.714703in}}%
\pgfpathlineto{\pgfqpoint{0.633772in}{1.707895in}}%
\pgfpathlineto{\pgfqpoint{0.625000in}{1.701044in}}%
\pgfusepath{stroke}%
\end{pgfscope}%
\begin{pgfscope}%
\pgfpathrectangle{\pgfqpoint{0.625000in}{0.550000in}}{\pgfqpoint{3.875000in}{3.850000in}} %
\pgfusepath{clip}%
\pgfsetbuttcap%
\pgfsetroundjoin%
\pgfsetlinewidth{0.250937pt}%
\definecolor{currentstroke}{rgb}{0.000000,0.000000,0.000000}%
\pgfsetstrokecolor{currentstroke}%
\pgfsetdash{}{0pt}%
\pgfpathmoveto{\pgfqpoint{0.625000in}{1.869126in}}%
\pgfpathlineto{\pgfqpoint{0.632104in}{1.862281in}}%
\pgfpathlineto{\pgfqpoint{0.625000in}{1.855577in}}%
\pgfusepath{stroke}%
\end{pgfscope}%
\begin{pgfscope}%
\pgfpathrectangle{\pgfqpoint{0.625000in}{0.550000in}}{\pgfqpoint{3.875000in}{3.850000in}} %
\pgfusepath{clip}%
\pgfsetbuttcap%
\pgfsetroundjoin%
\pgfsetlinewidth{0.250937pt}%
\definecolor{currentstroke}{rgb}{0.000000,0.000000,0.000000}%
\pgfsetstrokecolor{currentstroke}%
\pgfsetdash{}{0pt}%
\pgfpathmoveto{\pgfqpoint{0.625000in}{2.023731in}}%
\pgfpathlineto{\pgfqpoint{0.632071in}{2.016667in}}%
\pgfpathlineto{\pgfqpoint{0.625000in}{2.009660in}}%
\pgfusepath{stroke}%
\end{pgfscope}%
\begin{pgfscope}%
\pgfpathrectangle{\pgfqpoint{0.625000in}{0.550000in}}{\pgfqpoint{3.875000in}{3.850000in}} %
\pgfusepath{clip}%
\pgfsetbuttcap%
\pgfsetroundjoin%
\pgfsetlinewidth{0.250937pt}%
\definecolor{currentstroke}{rgb}{0.000000,0.000000,0.000000}%
\pgfsetstrokecolor{currentstroke}%
\pgfsetdash{}{0pt}%
\pgfpathmoveto{\pgfqpoint{0.625000in}{2.177769in}}%
\pgfpathlineto{\pgfqpoint{0.632103in}{2.171053in}}%
\pgfpathlineto{\pgfqpoint{0.625000in}{2.164144in}}%
\pgfusepath{stroke}%
\end{pgfscope}%
\begin{pgfscope}%
\pgfpathrectangle{\pgfqpoint{0.625000in}{0.550000in}}{\pgfqpoint{3.875000in}{3.850000in}} %
\pgfusepath{clip}%
\pgfsetbuttcap%
\pgfsetroundjoin%
\pgfsetlinewidth{0.250937pt}%
\definecolor{currentstroke}{rgb}{0.000000,0.000000,0.000000}%
\pgfsetstrokecolor{currentstroke}%
\pgfsetdash{}{0pt}%
\pgfpathmoveto{\pgfqpoint{0.625000in}{2.332107in}}%
\pgfpathlineto{\pgfqpoint{0.632119in}{2.325439in}}%
\pgfpathlineto{\pgfqpoint{0.625000in}{2.318303in}}%
\pgfusepath{stroke}%
\end{pgfscope}%
\begin{pgfscope}%
\pgfpathrectangle{\pgfqpoint{0.625000in}{0.550000in}}{\pgfqpoint{3.875000in}{3.850000in}} %
\pgfusepath{clip}%
\pgfsetbuttcap%
\pgfsetroundjoin%
\pgfsetlinewidth{0.250937pt}%
\definecolor{currentstroke}{rgb}{0.000000,0.000000,0.000000}%
\pgfsetstrokecolor{currentstroke}%
\pgfsetdash{}{0pt}%
\pgfpathmoveto{\pgfqpoint{0.625000in}{2.487460in}}%
\pgfpathlineto{\pgfqpoint{0.626888in}{2.489474in}}%
\pgfpathlineto{\pgfqpoint{0.627621in}{2.499123in}}%
\pgfpathlineto{\pgfqpoint{0.630423in}{2.508772in}}%
\pgfpathlineto{\pgfqpoint{0.634076in}{2.518421in}}%
\pgfpathlineto{\pgfqpoint{0.634712in}{2.520312in}}%
\pgfpathlineto{\pgfqpoint{0.639723in}{2.528070in}}%
\pgfpathlineto{\pgfqpoint{0.644424in}{2.535835in}}%
\pgfpathlineto{\pgfqpoint{0.646500in}{2.537719in}}%
\pgfpathlineto{\pgfqpoint{0.654135in}{2.546513in}}%
\pgfpathlineto{\pgfqpoint{0.655387in}{2.547368in}}%
\pgfpathlineto{\pgfqpoint{0.663847in}{2.554865in}}%
\pgfpathlineto{\pgfqpoint{0.667672in}{2.557018in}}%
\pgfpathlineto{\pgfqpoint{0.673559in}{2.561271in}}%
\pgfpathlineto{\pgfqpoint{0.683271in}{2.565967in}}%
\pgfpathlineto{\pgfqpoint{0.685452in}{2.566667in}}%
\pgfpathlineto{\pgfqpoint{0.692982in}{2.569740in}}%
\pgfpathlineto{\pgfqpoint{0.702694in}{2.572301in}}%
\pgfpathlineto{\pgfqpoint{0.712406in}{2.573795in}}%
\pgfpathlineto{\pgfqpoint{0.722118in}{2.574340in}}%
\pgfpathlineto{\pgfqpoint{0.731830in}{2.573956in}}%
\pgfpathlineto{\pgfqpoint{0.741541in}{2.572610in}}%
\pgfpathlineto{\pgfqpoint{0.751253in}{2.570228in}}%
\pgfpathlineto{\pgfqpoint{0.760965in}{2.566720in}}%
\pgfpathlineto{\pgfqpoint{0.761089in}{2.566667in}}%
\pgfpathlineto{\pgfqpoint{0.770677in}{2.562016in}}%
\pgfpathlineto{\pgfqpoint{0.778746in}{2.557018in}}%
\pgfpathlineto{\pgfqpoint{0.780388in}{2.555825in}}%
\pgfpathlineto{\pgfqpoint{0.790100in}{2.547782in}}%
\pgfpathlineto{\pgfqpoint{0.790557in}{2.547368in}}%
\pgfpathlineto{\pgfqpoint{0.799328in}{2.537719in}}%
\pgfpathlineto{\pgfqpoint{0.799812in}{2.537043in}}%
\pgfpathlineto{\pgfqpoint{0.806099in}{2.528070in}}%
\pgfpathlineto{\pgfqpoint{0.809524in}{2.521512in}}%
\pgfpathlineto{\pgfqpoint{0.811159in}{2.518421in}}%
\pgfpathlineto{\pgfqpoint{0.814981in}{2.508772in}}%
\pgfpathlineto{\pgfqpoint{0.817534in}{2.499123in}}%
\pgfpathlineto{\pgfqpoint{0.818998in}{2.489474in}}%
\pgfpathlineto{\pgfqpoint{0.819236in}{2.484680in}}%
\pgfpathlineto{\pgfqpoint{0.819501in}{2.479825in}}%
\pgfpathlineto{\pgfqpoint{0.819236in}{2.474969in}}%
\pgfpathlineto{\pgfqpoint{0.818998in}{2.470175in}}%
\pgfpathlineto{\pgfqpoint{0.817534in}{2.460526in}}%
\pgfpathlineto{\pgfqpoint{0.814981in}{2.450877in}}%
\pgfpathlineto{\pgfqpoint{0.811159in}{2.441228in}}%
\pgfpathlineto{\pgfqpoint{0.809524in}{2.438137in}}%
\pgfpathlineto{\pgfqpoint{0.806099in}{2.431579in}}%
\pgfpathlineto{\pgfqpoint{0.799812in}{2.422606in}}%
\pgfpathlineto{\pgfqpoint{0.799328in}{2.421930in}}%
\pgfpathlineto{\pgfqpoint{0.790557in}{2.412281in}}%
\pgfpathlineto{\pgfqpoint{0.790100in}{2.411868in}}%
\pgfpathlineto{\pgfqpoint{0.780388in}{2.403824in}}%
\pgfpathlineto{\pgfqpoint{0.778746in}{2.402632in}}%
\pgfpathlineto{\pgfqpoint{0.770677in}{2.397633in}}%
\pgfpathlineto{\pgfqpoint{0.761089in}{2.392982in}}%
\pgfpathlineto{\pgfqpoint{0.760965in}{2.392929in}}%
\pgfpathlineto{\pgfqpoint{0.751253in}{2.389421in}}%
\pgfpathlineto{\pgfqpoint{0.741541in}{2.387040in}}%
\pgfpathlineto{\pgfqpoint{0.731830in}{2.385693in}}%
\pgfpathlineto{\pgfqpoint{0.722118in}{2.385310in}}%
\pgfpathlineto{\pgfqpoint{0.712406in}{2.385854in}}%
\pgfpathlineto{\pgfqpoint{0.702694in}{2.387348in}}%
\pgfpathlineto{\pgfqpoint{0.692982in}{2.389909in}}%
\pgfpathlineto{\pgfqpoint{0.685452in}{2.392982in}}%
\pgfpathlineto{\pgfqpoint{0.683271in}{2.393682in}}%
\pgfpathlineto{\pgfqpoint{0.673559in}{2.398379in}}%
\pgfpathlineto{\pgfqpoint{0.667672in}{2.402632in}}%
\pgfpathlineto{\pgfqpoint{0.663847in}{2.404785in}}%
\pgfpathlineto{\pgfqpoint{0.655387in}{2.412281in}}%
\pgfpathlineto{\pgfqpoint{0.654135in}{2.413136in}}%
\pgfpathlineto{\pgfqpoint{0.646500in}{2.421930in}}%
\pgfpathlineto{\pgfqpoint{0.644424in}{2.423814in}}%
\pgfpathlineto{\pgfqpoint{0.639723in}{2.431579in}}%
\pgfpathlineto{\pgfqpoint{0.634712in}{2.439337in}}%
\pgfpathlineto{\pgfqpoint{0.634076in}{2.441228in}}%
\pgfpathlineto{\pgfqpoint{0.630423in}{2.450877in}}%
\pgfpathlineto{\pgfqpoint{0.627628in}{2.460526in}}%
\pgfpathlineto{\pgfqpoint{0.626888in}{2.470175in}}%
\pgfpathlineto{\pgfqpoint{0.625000in}{2.472189in}}%
\pgfusepath{stroke}%
\end{pgfscope}%
\begin{pgfscope}%
\pgfpathrectangle{\pgfqpoint{0.625000in}{0.550000in}}{\pgfqpoint{3.875000in}{3.850000in}} %
\pgfusepath{clip}%
\pgfsetbuttcap%
\pgfsetroundjoin%
\pgfsetlinewidth{0.250937pt}%
\definecolor{currentstroke}{rgb}{0.000000,0.000000,0.000000}%
\pgfsetstrokecolor{currentstroke}%
\pgfsetdash{}{0pt}%
\pgfpathmoveto{\pgfqpoint{0.625000in}{2.641342in}}%
\pgfpathlineto{\pgfqpoint{0.632114in}{2.634211in}}%
\pgfpathlineto{\pgfqpoint{0.625000in}{2.627547in}}%
\pgfusepath{stroke}%
\end{pgfscope}%
\begin{pgfscope}%
\pgfpathrectangle{\pgfqpoint{0.625000in}{0.550000in}}{\pgfqpoint{3.875000in}{3.850000in}} %
\pgfusepath{clip}%
\pgfsetbuttcap%
\pgfsetroundjoin%
\pgfsetlinewidth{0.250937pt}%
\definecolor{currentstroke}{rgb}{0.000000,0.000000,0.000000}%
\pgfsetstrokecolor{currentstroke}%
\pgfsetdash{}{0pt}%
\pgfpathmoveto{\pgfqpoint{0.625000in}{2.795489in}}%
\pgfpathlineto{\pgfqpoint{0.632086in}{2.788596in}}%
\pgfpathlineto{\pgfqpoint{0.625000in}{2.781898in}}%
\pgfusepath{stroke}%
\end{pgfscope}%
\begin{pgfscope}%
\pgfpathrectangle{\pgfqpoint{0.625000in}{0.550000in}}{\pgfqpoint{3.875000in}{3.850000in}} %
\pgfusepath{clip}%
\pgfsetbuttcap%
\pgfsetroundjoin%
\pgfsetlinewidth{0.250937pt}%
\definecolor{currentstroke}{rgb}{0.000000,0.000000,0.000000}%
\pgfsetstrokecolor{currentstroke}%
\pgfsetdash{}{0pt}%
\pgfpathmoveto{\pgfqpoint{0.625000in}{2.949943in}}%
\pgfpathlineto{\pgfqpoint{0.632025in}{2.942982in}}%
\pgfpathlineto{\pgfqpoint{0.625000in}{2.935963in}}%
\pgfusepath{stroke}%
\end{pgfscope}%
\begin{pgfscope}%
\pgfpathrectangle{\pgfqpoint{0.625000in}{0.550000in}}{\pgfqpoint{3.875000in}{3.850000in}} %
\pgfusepath{clip}%
\pgfsetbuttcap%
\pgfsetroundjoin%
\pgfsetlinewidth{0.250937pt}%
\definecolor{currentstroke}{rgb}{0.000000,0.000000,0.000000}%
\pgfsetstrokecolor{currentstroke}%
\pgfsetdash{}{0pt}%
\pgfpathmoveto{\pgfqpoint{0.625000in}{3.104033in}}%
\pgfpathlineto{\pgfqpoint{0.632068in}{3.097368in}}%
\pgfpathlineto{\pgfqpoint{0.625000in}{3.090561in}}%
\pgfusepath{stroke}%
\end{pgfscope}%
\begin{pgfscope}%
\pgfpathrectangle{\pgfqpoint{0.625000in}{0.550000in}}{\pgfqpoint{3.875000in}{3.850000in}} %
\pgfusepath{clip}%
\pgfsetbuttcap%
\pgfsetroundjoin%
\pgfsetlinewidth{0.250937pt}%
\definecolor{currentstroke}{rgb}{0.000000,0.000000,0.000000}%
\pgfsetstrokecolor{currentstroke}%
\pgfsetdash{}{0pt}%
\pgfpathmoveto{\pgfqpoint{0.625000in}{3.258471in}}%
\pgfpathlineto{\pgfqpoint{0.633713in}{3.251754in}}%
\pgfpathlineto{\pgfqpoint{0.625000in}{3.245082in}}%
\pgfusepath{stroke}%
\end{pgfscope}%
\begin{pgfscope}%
\pgfpathrectangle{\pgfqpoint{0.625000in}{0.550000in}}{\pgfqpoint{3.875000in}{3.850000in}} %
\pgfusepath{clip}%
\pgfsetbuttcap%
\pgfsetroundjoin%
\pgfsetlinewidth{0.250937pt}%
\definecolor{currentstroke}{rgb}{0.000000,0.000000,0.000000}%
\pgfsetstrokecolor{currentstroke}%
\pgfsetdash{}{0pt}%
\pgfpathmoveto{\pgfqpoint{0.625000in}{3.413096in}}%
\pgfpathlineto{\pgfqpoint{0.632074in}{3.406140in}}%
\pgfpathlineto{\pgfqpoint{0.625000in}{3.399410in}}%
\pgfusepath{stroke}%
\end{pgfscope}%
\begin{pgfscope}%
\pgfpathrectangle{\pgfqpoint{0.625000in}{0.550000in}}{\pgfqpoint{3.875000in}{3.850000in}} %
\pgfusepath{clip}%
\pgfsetbuttcap%
\pgfsetroundjoin%
\pgfsetlinewidth{0.250937pt}%
\definecolor{currentstroke}{rgb}{0.000000,0.000000,0.000000}%
\pgfsetstrokecolor{currentstroke}%
\pgfsetdash{}{0pt}%
\pgfpathmoveto{\pgfqpoint{0.625000in}{3.567245in}}%
\pgfpathlineto{\pgfqpoint{0.632071in}{3.560526in}}%
\pgfpathlineto{\pgfqpoint{0.625000in}{3.553812in}}%
\pgfusepath{stroke}%
\end{pgfscope}%
\begin{pgfscope}%
\pgfpathrectangle{\pgfqpoint{0.625000in}{0.550000in}}{\pgfqpoint{3.875000in}{3.850000in}} %
\pgfusepath{clip}%
\pgfsetbuttcap%
\pgfsetroundjoin%
\pgfsetlinewidth{0.250937pt}%
\definecolor{currentstroke}{rgb}{0.000000,0.000000,0.000000}%
\pgfsetstrokecolor{currentstroke}%
\pgfsetdash{}{0pt}%
\pgfpathmoveto{\pgfqpoint{0.625000in}{3.721703in}}%
\pgfpathlineto{\pgfqpoint{0.632090in}{3.714912in}}%
\pgfpathlineto{\pgfqpoint{0.625000in}{3.707917in}}%
\pgfusepath{stroke}%
\end{pgfscope}%
\begin{pgfscope}%
\pgfpathrectangle{\pgfqpoint{0.625000in}{0.550000in}}{\pgfqpoint{3.875000in}{3.850000in}} %
\pgfusepath{clip}%
\pgfsetbuttcap%
\pgfsetroundjoin%
\pgfsetlinewidth{0.250937pt}%
\definecolor{currentstroke}{rgb}{0.000000,0.000000,0.000000}%
\pgfsetstrokecolor{currentstroke}%
\pgfsetdash{}{0pt}%
\pgfpathmoveto{\pgfqpoint{0.625000in}{3.875950in}}%
\pgfpathlineto{\pgfqpoint{0.631990in}{3.869298in}}%
\pgfpathlineto{\pgfqpoint{0.625000in}{3.862620in}}%
\pgfusepath{stroke}%
\end{pgfscope}%
\begin{pgfscope}%
\pgfpathrectangle{\pgfqpoint{0.625000in}{0.550000in}}{\pgfqpoint{3.875000in}{3.850000in}} %
\pgfusepath{clip}%
\pgfsetbuttcap%
\pgfsetroundjoin%
\pgfsetlinewidth{0.250937pt}%
\definecolor{currentstroke}{rgb}{0.000000,0.000000,0.000000}%
\pgfsetstrokecolor{currentstroke}%
\pgfsetdash{}{0pt}%
\pgfpathmoveto{\pgfqpoint{0.625000in}{4.030485in}}%
\pgfpathlineto{\pgfqpoint{0.633734in}{4.023684in}}%
\pgfpathlineto{\pgfqpoint{0.625000in}{4.016950in}}%
\pgfusepath{stroke}%
\end{pgfscope}%
\begin{pgfscope}%
\pgfpathrectangle{\pgfqpoint{0.625000in}{0.550000in}}{\pgfqpoint{3.875000in}{3.850000in}} %
\pgfusepath{clip}%
\pgfsetbuttcap%
\pgfsetroundjoin%
\pgfsetlinewidth{0.250937pt}%
\definecolor{currentstroke}{rgb}{0.000000,0.000000,0.000000}%
\pgfsetstrokecolor{currentstroke}%
\pgfsetdash{}{0pt}%
\pgfpathmoveto{\pgfqpoint{0.625000in}{4.184783in}}%
\pgfpathlineto{\pgfqpoint{0.631974in}{4.178070in}}%
\pgfpathlineto{\pgfqpoint{0.625000in}{4.171671in}}%
\pgfusepath{stroke}%
\end{pgfscope}%
\begin{pgfscope}%
\pgfpathrectangle{\pgfqpoint{0.625000in}{0.550000in}}{\pgfqpoint{3.875000in}{3.850000in}} %
\pgfusepath{clip}%
\pgfsetbuttcap%
\pgfsetroundjoin%
\pgfsetlinewidth{0.250937pt}%
\definecolor{currentstroke}{rgb}{0.000000,0.000000,0.000000}%
\pgfsetstrokecolor{currentstroke}%
\pgfsetdash{}{0pt}%
\pgfpathmoveto{\pgfqpoint{0.625000in}{4.339147in}}%
\pgfpathlineto{\pgfqpoint{0.631950in}{4.332456in}}%
\pgfpathlineto{\pgfqpoint{0.625000in}{4.325743in}}%
\pgfusepath{stroke}%
\end{pgfscope}%
\begin{pgfscope}%
\pgfpathrectangle{\pgfqpoint{0.625000in}{0.550000in}}{\pgfqpoint{3.875000in}{3.850000in}} %
\pgfusepath{clip}%
\pgfsetbuttcap%
\pgfsetroundjoin%
\pgfsetlinewidth{0.250937pt}%
\definecolor{currentstroke}{rgb}{0.000000,0.000000,0.000000}%
\pgfsetstrokecolor{currentstroke}%
\pgfsetdash{}{0pt}%
\pgfpathmoveto{\pgfqpoint{0.634712in}{1.182324in}}%
\pgfpathlineto{\pgfqpoint{0.631901in}{1.186842in}}%
\pgfpathlineto{\pgfqpoint{0.629864in}{1.196491in}}%
\pgfpathlineto{\pgfqpoint{0.634712in}{1.203775in}}%
\pgfpathlineto{\pgfqpoint{0.644424in}{1.196700in}}%
\pgfpathlineto{\pgfqpoint{0.644619in}{1.196491in}}%
\pgfpathlineto{\pgfqpoint{0.644424in}{1.195691in}}%
\pgfpathlineto{\pgfqpoint{0.643031in}{1.186842in}}%
\pgfpathlineto{\pgfqpoint{0.634712in}{1.182324in}}%
\pgfusepath{stroke}%
\end{pgfscope}%
\begin{pgfscope}%
\pgfpathrectangle{\pgfqpoint{0.625000in}{0.550000in}}{\pgfqpoint{3.875000in}{3.850000in}} %
\pgfusepath{clip}%
\pgfsetbuttcap%
\pgfsetroundjoin%
\pgfsetlinewidth{0.250937pt}%
\definecolor{currentstroke}{rgb}{0.000000,0.000000,0.000000}%
\pgfsetstrokecolor{currentstroke}%
\pgfsetdash{}{0pt}%
\pgfpathmoveto{\pgfqpoint{0.634712in}{3.755874in}}%
\pgfpathlineto{\pgfqpoint{0.629864in}{3.763158in}}%
\pgfpathlineto{\pgfqpoint{0.631901in}{3.772807in}}%
\pgfpathlineto{\pgfqpoint{0.634712in}{3.777325in}}%
\pgfpathlineto{\pgfqpoint{0.643031in}{3.772807in}}%
\pgfpathlineto{\pgfqpoint{0.644424in}{3.763958in}}%
\pgfpathlineto{\pgfqpoint{0.644619in}{3.763158in}}%
\pgfpathlineto{\pgfqpoint{0.644424in}{3.762949in}}%
\pgfpathlineto{\pgfqpoint{0.634712in}{3.755874in}}%
\pgfusepath{stroke}%
\end{pgfscope}%
\begin{pgfscope}%
\pgfpathrectangle{\pgfqpoint{0.625000in}{0.550000in}}{\pgfqpoint{3.875000in}{3.850000in}} %
\pgfusepath{clip}%
\pgfsetbuttcap%
\pgfsetroundjoin%
\pgfsetlinewidth{0.250937pt}%
\definecolor{currentstroke}{rgb}{0.000000,0.000000,0.000000}%
\pgfsetstrokecolor{currentstroke}%
\pgfsetdash{}{0pt}%
\pgfpathmoveto{\pgfqpoint{0.625000in}{0.634370in}}%
\pgfpathlineto{\pgfqpoint{0.632403in}{0.627193in}}%
\pgfpathlineto{\pgfqpoint{0.625000in}{0.620037in}}%
\pgfusepath{stroke}%
\end{pgfscope}%
\begin{pgfscope}%
\pgfpathrectangle{\pgfqpoint{0.625000in}{0.550000in}}{\pgfqpoint{3.875000in}{3.850000in}} %
\pgfusepath{clip}%
\pgfsetbuttcap%
\pgfsetroundjoin%
\pgfsetlinewidth{0.250937pt}%
\definecolor{currentstroke}{rgb}{0.000000,0.000000,0.000000}%
\pgfsetstrokecolor{currentstroke}%
\pgfsetdash{}{0pt}%
\pgfpathmoveto{\pgfqpoint{0.625000in}{0.788345in}}%
\pgfpathlineto{\pgfqpoint{0.632320in}{0.781579in}}%
\pgfpathlineto{\pgfqpoint{0.625000in}{0.774511in}}%
\pgfusepath{stroke}%
\end{pgfscope}%
\begin{pgfscope}%
\pgfpathrectangle{\pgfqpoint{0.625000in}{0.550000in}}{\pgfqpoint{3.875000in}{3.850000in}} %
\pgfusepath{clip}%
\pgfsetbuttcap%
\pgfsetroundjoin%
\pgfsetlinewidth{0.250937pt}%
\definecolor{currentstroke}{rgb}{0.000000,0.000000,0.000000}%
\pgfsetstrokecolor{currentstroke}%
\pgfsetdash{}{0pt}%
\pgfpathmoveto{\pgfqpoint{0.625000in}{0.943023in}}%
\pgfpathlineto{\pgfqpoint{0.633935in}{0.935965in}}%
\pgfpathlineto{\pgfqpoint{0.625000in}{0.928843in}}%
\pgfusepath{stroke}%
\end{pgfscope}%
\begin{pgfscope}%
\pgfpathrectangle{\pgfqpoint{0.625000in}{0.550000in}}{\pgfqpoint{3.875000in}{3.850000in}} %
\pgfusepath{clip}%
\pgfsetbuttcap%
\pgfsetroundjoin%
\pgfsetlinewidth{0.250937pt}%
\definecolor{currentstroke}{rgb}{0.000000,0.000000,0.000000}%
\pgfsetstrokecolor{currentstroke}%
\pgfsetdash{}{0pt}%
\pgfpathmoveto{\pgfqpoint{0.625000in}{1.097381in}}%
\pgfpathlineto{\pgfqpoint{0.632331in}{1.090351in}}%
\pgfpathlineto{\pgfqpoint{0.625000in}{1.083347in}}%
\pgfusepath{stroke}%
\end{pgfscope}%
\begin{pgfscope}%
\pgfpathrectangle{\pgfqpoint{0.625000in}{0.550000in}}{\pgfqpoint{3.875000in}{3.850000in}} %
\pgfusepath{clip}%
\pgfsetbuttcap%
\pgfsetroundjoin%
\pgfsetlinewidth{0.250937pt}%
\definecolor{currentstroke}{rgb}{0.000000,0.000000,0.000000}%
\pgfsetstrokecolor{currentstroke}%
\pgfsetdash{}{0pt}%
\pgfpathmoveto{\pgfqpoint{0.625000in}{1.251971in}}%
\pgfpathlineto{\pgfqpoint{0.632330in}{1.244737in}}%
\pgfpathlineto{\pgfqpoint{0.625000in}{1.237702in}}%
\pgfusepath{stroke}%
\end{pgfscope}%
\begin{pgfscope}%
\pgfpathrectangle{\pgfqpoint{0.625000in}{0.550000in}}{\pgfqpoint{3.875000in}{3.850000in}} %
\pgfusepath{clip}%
\pgfsetbuttcap%
\pgfsetroundjoin%
\pgfsetlinewidth{0.250937pt}%
\definecolor{currentstroke}{rgb}{0.000000,0.000000,0.000000}%
\pgfsetstrokecolor{currentstroke}%
\pgfsetdash{}{0pt}%
\pgfpathmoveto{\pgfqpoint{0.625000in}{1.406031in}}%
\pgfpathlineto{\pgfqpoint{0.632262in}{1.399123in}}%
\pgfpathlineto{\pgfqpoint{0.625000in}{1.392211in}}%
\pgfusepath{stroke}%
\end{pgfscope}%
\begin{pgfscope}%
\pgfpathrectangle{\pgfqpoint{0.625000in}{0.550000in}}{\pgfqpoint{3.875000in}{3.850000in}} %
\pgfusepath{clip}%
\pgfsetbuttcap%
\pgfsetroundjoin%
\pgfsetlinewidth{0.250937pt}%
\definecolor{currentstroke}{rgb}{0.000000,0.000000,0.000000}%
\pgfsetstrokecolor{currentstroke}%
\pgfsetdash{}{0pt}%
\pgfpathmoveto{\pgfqpoint{0.625000in}{1.560413in}}%
\pgfpathlineto{\pgfqpoint{0.632246in}{1.553509in}}%
\pgfpathlineto{\pgfqpoint{0.625000in}{1.546380in}}%
\pgfusepath{stroke}%
\end{pgfscope}%
\begin{pgfscope}%
\pgfpathrectangle{\pgfqpoint{0.625000in}{0.550000in}}{\pgfqpoint{3.875000in}{3.850000in}} %
\pgfusepath{clip}%
\pgfsetbuttcap%
\pgfsetroundjoin%
\pgfsetlinewidth{0.250937pt}%
\definecolor{currentstroke}{rgb}{0.000000,0.000000,0.000000}%
\pgfsetstrokecolor{currentstroke}%
\pgfsetdash{}{0pt}%
\pgfpathmoveto{\pgfqpoint{0.625000in}{1.714784in}}%
\pgfpathlineto{\pgfqpoint{0.633876in}{1.707895in}}%
\pgfpathlineto{\pgfqpoint{0.625000in}{1.700962in}}%
\pgfusepath{stroke}%
\end{pgfscope}%
\begin{pgfscope}%
\pgfpathrectangle{\pgfqpoint{0.625000in}{0.550000in}}{\pgfqpoint{3.875000in}{3.850000in}} %
\pgfusepath{clip}%
\pgfsetbuttcap%
\pgfsetroundjoin%
\pgfsetlinewidth{0.250937pt}%
\definecolor{currentstroke}{rgb}{0.000000,0.000000,0.000000}%
\pgfsetstrokecolor{currentstroke}%
\pgfsetdash{}{0pt}%
\pgfpathmoveto{\pgfqpoint{0.625000in}{1.869209in}}%
\pgfpathlineto{\pgfqpoint{0.632190in}{1.862281in}}%
\pgfpathlineto{\pgfqpoint{0.625000in}{1.855496in}}%
\pgfusepath{stroke}%
\end{pgfscope}%
\begin{pgfscope}%
\pgfpathrectangle{\pgfqpoint{0.625000in}{0.550000in}}{\pgfqpoint{3.875000in}{3.850000in}} %
\pgfusepath{clip}%
\pgfsetbuttcap%
\pgfsetroundjoin%
\pgfsetlinewidth{0.250937pt}%
\definecolor{currentstroke}{rgb}{0.000000,0.000000,0.000000}%
\pgfsetstrokecolor{currentstroke}%
\pgfsetdash{}{0pt}%
\pgfpathmoveto{\pgfqpoint{0.625000in}{2.023818in}}%
\pgfpathlineto{\pgfqpoint{0.632159in}{2.016667in}}%
\pgfpathlineto{\pgfqpoint{0.625000in}{2.009573in}}%
\pgfusepath{stroke}%
\end{pgfscope}%
\begin{pgfscope}%
\pgfpathrectangle{\pgfqpoint{0.625000in}{0.550000in}}{\pgfqpoint{3.875000in}{3.850000in}} %
\pgfusepath{clip}%
\pgfsetbuttcap%
\pgfsetroundjoin%
\pgfsetlinewidth{0.250937pt}%
\definecolor{currentstroke}{rgb}{0.000000,0.000000,0.000000}%
\pgfsetstrokecolor{currentstroke}%
\pgfsetdash{}{0pt}%
\pgfpathmoveto{\pgfqpoint{0.625000in}{2.177851in}}%
\pgfpathlineto{\pgfqpoint{0.632190in}{2.171053in}}%
\pgfpathlineto{\pgfqpoint{0.625000in}{2.164059in}}%
\pgfusepath{stroke}%
\end{pgfscope}%
\begin{pgfscope}%
\pgfpathrectangle{\pgfqpoint{0.625000in}{0.550000in}}{\pgfqpoint{3.875000in}{3.850000in}} %
\pgfusepath{clip}%
\pgfsetbuttcap%
\pgfsetroundjoin%
\pgfsetlinewidth{0.250937pt}%
\definecolor{currentstroke}{rgb}{0.000000,0.000000,0.000000}%
\pgfsetstrokecolor{currentstroke}%
\pgfsetdash{}{0pt}%
\pgfpathmoveto{\pgfqpoint{0.625000in}{2.332190in}}%
\pgfpathlineto{\pgfqpoint{0.632208in}{2.325439in}}%
\pgfpathlineto{\pgfqpoint{0.625000in}{2.318214in}}%
\pgfusepath{stroke}%
\end{pgfscope}%
\begin{pgfscope}%
\pgfpathrectangle{\pgfqpoint{0.625000in}{0.550000in}}{\pgfqpoint{3.875000in}{3.850000in}} %
\pgfusepath{clip}%
\pgfsetbuttcap%
\pgfsetroundjoin%
\pgfsetlinewidth{0.250937pt}%
\definecolor{currentstroke}{rgb}{0.000000,0.000000,0.000000}%
\pgfsetstrokecolor{currentstroke}%
\pgfsetdash{}{0pt}%
\pgfpathmoveto{\pgfqpoint{0.625000in}{2.487544in}}%
\pgfpathlineto{\pgfqpoint{0.626810in}{2.489474in}}%
\pgfpathlineto{\pgfqpoint{0.627542in}{2.499123in}}%
\pgfpathlineto{\pgfqpoint{0.630268in}{2.508772in}}%
\pgfpathlineto{\pgfqpoint{0.633814in}{2.518421in}}%
\pgfpathlineto{\pgfqpoint{0.634712in}{2.521092in}}%
\pgfpathlineto{\pgfqpoint{0.639219in}{2.528070in}}%
\pgfpathlineto{\pgfqpoint{0.644424in}{2.536667in}}%
\pgfpathlineto{\pgfqpoint{0.645583in}{2.537719in}}%
\pgfpathlineto{\pgfqpoint{0.653927in}{2.547368in}}%
\pgfpathlineto{\pgfqpoint{0.654135in}{2.547653in}}%
\pgfpathlineto{\pgfqpoint{0.663847in}{2.556235in}}%
\pgfpathlineto{\pgfqpoint{0.665238in}{2.557018in}}%
\pgfpathlineto{\pgfqpoint{0.673559in}{2.563029in}}%
\pgfpathlineto{\pgfqpoint{0.681046in}{2.566667in}}%
\pgfpathlineto{\pgfqpoint{0.683271in}{2.568029in}}%
\pgfpathlineto{\pgfqpoint{0.692982in}{2.572050in}}%
\pgfpathlineto{\pgfqpoint{0.702694in}{2.574780in}}%
\pgfpathlineto{\pgfqpoint{0.711400in}{2.576316in}}%
\pgfpathlineto{\pgfqpoint{0.712406in}{2.576536in}}%
\pgfpathlineto{\pgfqpoint{0.722118in}{2.577470in}}%
\pgfpathlineto{\pgfqpoint{0.731830in}{2.577429in}}%
\pgfpathlineto{\pgfqpoint{0.741541in}{2.576433in}}%
\pgfpathlineto{\pgfqpoint{0.742180in}{2.576316in}}%
\pgfpathlineto{\pgfqpoint{0.751253in}{2.574547in}}%
\pgfpathlineto{\pgfqpoint{0.760965in}{2.571613in}}%
\pgfpathlineto{\pgfqpoint{0.770677in}{2.567522in}}%
\pgfpathlineto{\pgfqpoint{0.772395in}{2.566667in}}%
\pgfpathlineto{\pgfqpoint{0.780388in}{2.562147in}}%
\pgfpathlineto{\pgfqpoint{0.787864in}{2.557018in}}%
\pgfpathlineto{\pgfqpoint{0.790100in}{2.555207in}}%
\pgfpathlineto{\pgfqpoint{0.798772in}{2.547368in}}%
\pgfpathlineto{\pgfqpoint{0.799812in}{2.546215in}}%
\pgfpathlineto{\pgfqpoint{0.807074in}{2.537719in}}%
\pgfpathlineto{\pgfqpoint{0.809524in}{2.534056in}}%
\pgfpathlineto{\pgfqpoint{0.813477in}{2.528070in}}%
\pgfpathlineto{\pgfqpoint{0.818301in}{2.518421in}}%
\pgfpathlineto{\pgfqpoint{0.819236in}{2.515895in}}%
\pgfpathlineto{\pgfqpoint{0.821977in}{2.508772in}}%
\pgfpathlineto{\pgfqpoint{0.824510in}{2.499123in}}%
\pgfpathlineto{\pgfqpoint{0.825969in}{2.489474in}}%
\pgfpathlineto{\pgfqpoint{0.826445in}{2.479825in}}%
\pgfpathlineto{\pgfqpoint{0.825969in}{2.470175in}}%
\pgfpathlineto{\pgfqpoint{0.824510in}{2.460526in}}%
\pgfpathlineto{\pgfqpoint{0.821977in}{2.450877in}}%
\pgfpathlineto{\pgfqpoint{0.819236in}{2.443754in}}%
\pgfpathlineto{\pgfqpoint{0.818301in}{2.441228in}}%
\pgfpathlineto{\pgfqpoint{0.813477in}{2.431579in}}%
\pgfpathlineto{\pgfqpoint{0.809524in}{2.425593in}}%
\pgfpathlineto{\pgfqpoint{0.807074in}{2.421930in}}%
\pgfpathlineto{\pgfqpoint{0.799812in}{2.413434in}}%
\pgfpathlineto{\pgfqpoint{0.798772in}{2.412281in}}%
\pgfpathlineto{\pgfqpoint{0.790100in}{2.404442in}}%
\pgfpathlineto{\pgfqpoint{0.787864in}{2.402632in}}%
\pgfpathlineto{\pgfqpoint{0.780388in}{2.397502in}}%
\pgfpathlineto{\pgfqpoint{0.772395in}{2.392982in}}%
\pgfpathlineto{\pgfqpoint{0.770677in}{2.392127in}}%
\pgfpathlineto{\pgfqpoint{0.760965in}{2.388036in}}%
\pgfpathlineto{\pgfqpoint{0.751253in}{2.385102in}}%
\pgfpathlineto{\pgfqpoint{0.742180in}{2.383333in}}%
\pgfpathlineto{\pgfqpoint{0.741541in}{2.383216in}}%
\pgfpathlineto{\pgfqpoint{0.731830in}{2.382220in}}%
\pgfpathlineto{\pgfqpoint{0.722118in}{2.382180in}}%
\pgfpathlineto{\pgfqpoint{0.712406in}{2.383113in}}%
\pgfpathlineto{\pgfqpoint{0.711400in}{2.383333in}}%
\pgfpathlineto{\pgfqpoint{0.702694in}{2.384869in}}%
\pgfpathlineto{\pgfqpoint{0.692982in}{2.387599in}}%
\pgfpathlineto{\pgfqpoint{0.683271in}{2.391620in}}%
\pgfpathlineto{\pgfqpoint{0.681046in}{2.392982in}}%
\pgfpathlineto{\pgfqpoint{0.673559in}{2.396620in}}%
\pgfpathlineto{\pgfqpoint{0.665238in}{2.402632in}}%
\pgfpathlineto{\pgfqpoint{0.663847in}{2.403414in}}%
\pgfpathlineto{\pgfqpoint{0.654135in}{2.411996in}}%
\pgfpathlineto{\pgfqpoint{0.653927in}{2.412281in}}%
\pgfpathlineto{\pgfqpoint{0.645583in}{2.421930in}}%
\pgfpathlineto{\pgfqpoint{0.644424in}{2.422982in}}%
\pgfpathlineto{\pgfqpoint{0.639219in}{2.431579in}}%
\pgfpathlineto{\pgfqpoint{0.634712in}{2.438557in}}%
\pgfpathlineto{\pgfqpoint{0.633814in}{2.441228in}}%
\pgfpathlineto{\pgfqpoint{0.630268in}{2.450877in}}%
\pgfpathlineto{\pgfqpoint{0.627548in}{2.460526in}}%
\pgfpathlineto{\pgfqpoint{0.626810in}{2.470175in}}%
\pgfpathlineto{\pgfqpoint{0.625000in}{2.472105in}}%
\pgfusepath{stroke}%
\end{pgfscope}%
\begin{pgfscope}%
\pgfpathrectangle{\pgfqpoint{0.625000in}{0.550000in}}{\pgfqpoint{3.875000in}{3.850000in}} %
\pgfusepath{clip}%
\pgfsetbuttcap%
\pgfsetroundjoin%
\pgfsetlinewidth{0.250937pt}%
\definecolor{currentstroke}{rgb}{0.000000,0.000000,0.000000}%
\pgfsetstrokecolor{currentstroke}%
\pgfsetdash{}{0pt}%
\pgfpathmoveto{\pgfqpoint{0.625000in}{2.641430in}}%
\pgfpathlineto{\pgfqpoint{0.632203in}{2.634211in}}%
\pgfpathlineto{\pgfqpoint{0.625000in}{2.627464in}}%
\pgfusepath{stroke}%
\end{pgfscope}%
\begin{pgfscope}%
\pgfpathrectangle{\pgfqpoint{0.625000in}{0.550000in}}{\pgfqpoint{3.875000in}{3.850000in}} %
\pgfusepath{clip}%
\pgfsetbuttcap%
\pgfsetroundjoin%
\pgfsetlinewidth{0.250937pt}%
\definecolor{currentstroke}{rgb}{0.000000,0.000000,0.000000}%
\pgfsetstrokecolor{currentstroke}%
\pgfsetdash{}{0pt}%
\pgfpathmoveto{\pgfqpoint{0.625000in}{2.795574in}}%
\pgfpathlineto{\pgfqpoint{0.632174in}{2.788596in}}%
\pgfpathlineto{\pgfqpoint{0.625000in}{2.781815in}}%
\pgfusepath{stroke}%
\end{pgfscope}%
\begin{pgfscope}%
\pgfpathrectangle{\pgfqpoint{0.625000in}{0.550000in}}{\pgfqpoint{3.875000in}{3.850000in}} %
\pgfusepath{clip}%
\pgfsetbuttcap%
\pgfsetroundjoin%
\pgfsetlinewidth{0.250937pt}%
\definecolor{currentstroke}{rgb}{0.000000,0.000000,0.000000}%
\pgfsetstrokecolor{currentstroke}%
\pgfsetdash{}{0pt}%
\pgfpathmoveto{\pgfqpoint{0.625000in}{2.950031in}}%
\pgfpathlineto{\pgfqpoint{0.632114in}{2.942982in}}%
\pgfpathlineto{\pgfqpoint{0.625000in}{2.935874in}}%
\pgfusepath{stroke}%
\end{pgfscope}%
\begin{pgfscope}%
\pgfpathrectangle{\pgfqpoint{0.625000in}{0.550000in}}{\pgfqpoint{3.875000in}{3.850000in}} %
\pgfusepath{clip}%
\pgfsetbuttcap%
\pgfsetroundjoin%
\pgfsetlinewidth{0.250937pt}%
\definecolor{currentstroke}{rgb}{0.000000,0.000000,0.000000}%
\pgfsetstrokecolor{currentstroke}%
\pgfsetdash{}{0pt}%
\pgfpathmoveto{\pgfqpoint{0.625000in}{3.104116in}}%
\pgfpathlineto{\pgfqpoint{0.632155in}{3.097368in}}%
\pgfpathlineto{\pgfqpoint{0.625000in}{3.090477in}}%
\pgfusepath{stroke}%
\end{pgfscope}%
\begin{pgfscope}%
\pgfpathrectangle{\pgfqpoint{0.625000in}{0.550000in}}{\pgfqpoint{3.875000in}{3.850000in}} %
\pgfusepath{clip}%
\pgfsetbuttcap%
\pgfsetroundjoin%
\pgfsetlinewidth{0.250937pt}%
\definecolor{currentstroke}{rgb}{0.000000,0.000000,0.000000}%
\pgfsetstrokecolor{currentstroke}%
\pgfsetdash{}{0pt}%
\pgfpathmoveto{\pgfqpoint{0.625000in}{3.258556in}}%
\pgfpathlineto{\pgfqpoint{0.633824in}{3.251754in}}%
\pgfpathlineto{\pgfqpoint{0.625000in}{3.244998in}}%
\pgfusepath{stroke}%
\end{pgfscope}%
\begin{pgfscope}%
\pgfpathrectangle{\pgfqpoint{0.625000in}{0.550000in}}{\pgfqpoint{3.875000in}{3.850000in}} %
\pgfusepath{clip}%
\pgfsetbuttcap%
\pgfsetroundjoin%
\pgfsetlinewidth{0.250937pt}%
\definecolor{currentstroke}{rgb}{0.000000,0.000000,0.000000}%
\pgfsetstrokecolor{currentstroke}%
\pgfsetdash{}{0pt}%
\pgfpathmoveto{\pgfqpoint{0.625000in}{3.413182in}}%
\pgfpathlineto{\pgfqpoint{0.632160in}{3.406140in}}%
\pgfpathlineto{\pgfqpoint{0.625000in}{3.399328in}}%
\pgfusepath{stroke}%
\end{pgfscope}%
\begin{pgfscope}%
\pgfpathrectangle{\pgfqpoint{0.625000in}{0.550000in}}{\pgfqpoint{3.875000in}{3.850000in}} %
\pgfusepath{clip}%
\pgfsetbuttcap%
\pgfsetroundjoin%
\pgfsetlinewidth{0.250937pt}%
\definecolor{currentstroke}{rgb}{0.000000,0.000000,0.000000}%
\pgfsetstrokecolor{currentstroke}%
\pgfsetdash{}{0pt}%
\pgfpathmoveto{\pgfqpoint{0.625000in}{3.567330in}}%
\pgfpathlineto{\pgfqpoint{0.632160in}{3.560526in}}%
\pgfpathlineto{\pgfqpoint{0.625000in}{3.553727in}}%
\pgfusepath{stroke}%
\end{pgfscope}%
\begin{pgfscope}%
\pgfpathrectangle{\pgfqpoint{0.625000in}{0.550000in}}{\pgfqpoint{3.875000in}{3.850000in}} %
\pgfusepath{clip}%
\pgfsetbuttcap%
\pgfsetroundjoin%
\pgfsetlinewidth{0.250937pt}%
\definecolor{currentstroke}{rgb}{0.000000,0.000000,0.000000}%
\pgfsetstrokecolor{currentstroke}%
\pgfsetdash{}{0pt}%
\pgfpathmoveto{\pgfqpoint{0.625000in}{3.721788in}}%
\pgfpathlineto{\pgfqpoint{0.632180in}{3.714912in}}%
\pgfpathlineto{\pgfqpoint{0.625000in}{3.707829in}}%
\pgfusepath{stroke}%
\end{pgfscope}%
\begin{pgfscope}%
\pgfpathrectangle{\pgfqpoint{0.625000in}{0.550000in}}{\pgfqpoint{3.875000in}{3.850000in}} %
\pgfusepath{clip}%
\pgfsetbuttcap%
\pgfsetroundjoin%
\pgfsetlinewidth{0.250937pt}%
\definecolor{currentstroke}{rgb}{0.000000,0.000000,0.000000}%
\pgfsetstrokecolor{currentstroke}%
\pgfsetdash{}{0pt}%
\pgfpathmoveto{\pgfqpoint{0.625000in}{3.876036in}}%
\pgfpathlineto{\pgfqpoint{0.632081in}{3.869298in}}%
\pgfpathlineto{\pgfqpoint{0.625000in}{3.862533in}}%
\pgfusepath{stroke}%
\end{pgfscope}%
\begin{pgfscope}%
\pgfpathrectangle{\pgfqpoint{0.625000in}{0.550000in}}{\pgfqpoint{3.875000in}{3.850000in}} %
\pgfusepath{clip}%
\pgfsetbuttcap%
\pgfsetroundjoin%
\pgfsetlinewidth{0.250937pt}%
\definecolor{currentstroke}{rgb}{0.000000,0.000000,0.000000}%
\pgfsetstrokecolor{currentstroke}%
\pgfsetdash{}{0pt}%
\pgfpathmoveto{\pgfqpoint{0.625000in}{4.030569in}}%
\pgfpathlineto{\pgfqpoint{0.633843in}{4.023684in}}%
\pgfpathlineto{\pgfqpoint{0.625000in}{4.016866in}}%
\pgfusepath{stroke}%
\end{pgfscope}%
\begin{pgfscope}%
\pgfpathrectangle{\pgfqpoint{0.625000in}{0.550000in}}{\pgfqpoint{3.875000in}{3.850000in}} %
\pgfusepath{clip}%
\pgfsetbuttcap%
\pgfsetroundjoin%
\pgfsetlinewidth{0.250937pt}%
\definecolor{currentstroke}{rgb}{0.000000,0.000000,0.000000}%
\pgfsetstrokecolor{currentstroke}%
\pgfsetdash{}{0pt}%
\pgfpathmoveto{\pgfqpoint{0.625000in}{4.184872in}}%
\pgfpathlineto{\pgfqpoint{0.632066in}{4.178070in}}%
\pgfpathlineto{\pgfqpoint{0.625000in}{4.171587in}}%
\pgfusepath{stroke}%
\end{pgfscope}%
\begin{pgfscope}%
\pgfpathrectangle{\pgfqpoint{0.625000in}{0.550000in}}{\pgfqpoint{3.875000in}{3.850000in}} %
\pgfusepath{clip}%
\pgfsetbuttcap%
\pgfsetroundjoin%
\pgfsetlinewidth{0.250937pt}%
\definecolor{currentstroke}{rgb}{0.000000,0.000000,0.000000}%
\pgfsetstrokecolor{currentstroke}%
\pgfsetdash{}{0pt}%
\pgfpathmoveto{\pgfqpoint{0.625000in}{4.339234in}}%
\pgfpathlineto{\pgfqpoint{0.632041in}{4.332456in}}%
\pgfpathlineto{\pgfqpoint{0.625000in}{4.325655in}}%
\pgfusepath{stroke}%
\end{pgfscope}%
\begin{pgfscope}%
\pgfpathrectangle{\pgfqpoint{0.625000in}{0.550000in}}{\pgfqpoint{3.875000in}{3.850000in}} %
\pgfusepath{clip}%
\pgfsetbuttcap%
\pgfsetroundjoin%
\pgfsetlinewidth{0.250937pt}%
\definecolor{currentstroke}{rgb}{0.000000,0.000000,0.000000}%
\pgfsetstrokecolor{currentstroke}%
\pgfsetdash{}{0pt}%
\pgfpathmoveto{\pgfqpoint{0.634712in}{1.181990in}}%
\pgfpathlineto{\pgfqpoint{0.631693in}{1.186842in}}%
\pgfpathlineto{\pgfqpoint{0.629690in}{1.196491in}}%
\pgfpathlineto{\pgfqpoint{0.634712in}{1.204037in}}%
\pgfpathlineto{\pgfqpoint{0.644424in}{1.197721in}}%
\pgfpathlineto{\pgfqpoint{0.645573in}{1.196491in}}%
\pgfpathlineto{\pgfqpoint{0.644424in}{1.191783in}}%
\pgfpathlineto{\pgfqpoint{0.643646in}{1.186842in}}%
\pgfpathlineto{\pgfqpoint{0.634712in}{1.181990in}}%
\pgfusepath{stroke}%
\end{pgfscope}%
\begin{pgfscope}%
\pgfpathrectangle{\pgfqpoint{0.625000in}{0.550000in}}{\pgfqpoint{3.875000in}{3.850000in}} %
\pgfusepath{clip}%
\pgfsetbuttcap%
\pgfsetroundjoin%
\pgfsetlinewidth{0.250937pt}%
\definecolor{currentstroke}{rgb}{0.000000,0.000000,0.000000}%
\pgfsetstrokecolor{currentstroke}%
\pgfsetdash{}{0pt}%
\pgfpathmoveto{\pgfqpoint{0.634712in}{3.755612in}}%
\pgfpathlineto{\pgfqpoint{0.629690in}{3.763158in}}%
\pgfpathlineto{\pgfqpoint{0.631693in}{3.772807in}}%
\pgfpathlineto{\pgfqpoint{0.634712in}{3.777659in}}%
\pgfpathlineto{\pgfqpoint{0.643646in}{3.772807in}}%
\pgfpathlineto{\pgfqpoint{0.644424in}{3.767866in}}%
\pgfpathlineto{\pgfqpoint{0.645573in}{3.763158in}}%
\pgfpathlineto{\pgfqpoint{0.644424in}{3.761929in}}%
\pgfpathlineto{\pgfqpoint{0.634712in}{3.755612in}}%
\pgfusepath{stroke}%
\end{pgfscope}%
\begin{pgfscope}%
\pgfpathrectangle{\pgfqpoint{0.625000in}{0.550000in}}{\pgfqpoint{3.875000in}{3.850000in}} %
\pgfusepath{clip}%
\pgfsetbuttcap%
\pgfsetroundjoin%
\pgfsetlinewidth{0.250937pt}%
\definecolor{currentstroke}{rgb}{0.000000,0.000000,0.000000}%
\pgfsetstrokecolor{currentstroke}%
\pgfsetdash{}{0pt}%
\pgfpathmoveto{\pgfqpoint{0.625000in}{0.634446in}}%
\pgfpathlineto{\pgfqpoint{0.632481in}{0.627193in}}%
\pgfpathlineto{\pgfqpoint{0.625000in}{0.619961in}}%
\pgfusepath{stroke}%
\end{pgfscope}%
\begin{pgfscope}%
\pgfpathrectangle{\pgfqpoint{0.625000in}{0.550000in}}{\pgfqpoint{3.875000in}{3.850000in}} %
\pgfusepath{clip}%
\pgfsetbuttcap%
\pgfsetroundjoin%
\pgfsetlinewidth{0.250937pt}%
\definecolor{currentstroke}{rgb}{0.000000,0.000000,0.000000}%
\pgfsetstrokecolor{currentstroke}%
\pgfsetdash{}{0pt}%
\pgfpathmoveto{\pgfqpoint{0.625000in}{0.788422in}}%
\pgfpathlineto{\pgfqpoint{0.632403in}{0.781579in}}%
\pgfpathlineto{\pgfqpoint{0.625000in}{0.774431in}}%
\pgfusepath{stroke}%
\end{pgfscope}%
\begin{pgfscope}%
\pgfpathrectangle{\pgfqpoint{0.625000in}{0.550000in}}{\pgfqpoint{3.875000in}{3.850000in}} %
\pgfusepath{clip}%
\pgfsetbuttcap%
\pgfsetroundjoin%
\pgfsetlinewidth{0.250937pt}%
\definecolor{currentstroke}{rgb}{0.000000,0.000000,0.000000}%
\pgfsetstrokecolor{currentstroke}%
\pgfsetdash{}{0pt}%
\pgfpathmoveto{\pgfqpoint{0.625000in}{0.943100in}}%
\pgfpathlineto{\pgfqpoint{0.634032in}{0.935965in}}%
\pgfpathlineto{\pgfqpoint{0.625000in}{0.928766in}}%
\pgfusepath{stroke}%
\end{pgfscope}%
\begin{pgfscope}%
\pgfpathrectangle{\pgfqpoint{0.625000in}{0.550000in}}{\pgfqpoint{3.875000in}{3.850000in}} %
\pgfusepath{clip}%
\pgfsetbuttcap%
\pgfsetroundjoin%
\pgfsetlinewidth{0.250937pt}%
\definecolor{currentstroke}{rgb}{0.000000,0.000000,0.000000}%
\pgfsetstrokecolor{currentstroke}%
\pgfsetdash{}{0pt}%
\pgfpathmoveto{\pgfqpoint{0.625000in}{1.097460in}}%
\pgfpathlineto{\pgfqpoint{0.632413in}{1.090351in}}%
\pgfpathlineto{\pgfqpoint{0.625000in}{1.083268in}}%
\pgfusepath{stroke}%
\end{pgfscope}%
\begin{pgfscope}%
\pgfpathrectangle{\pgfqpoint{0.625000in}{0.550000in}}{\pgfqpoint{3.875000in}{3.850000in}} %
\pgfusepath{clip}%
\pgfsetbuttcap%
\pgfsetroundjoin%
\pgfsetlinewidth{0.250937pt}%
\definecolor{currentstroke}{rgb}{0.000000,0.000000,0.000000}%
\pgfsetstrokecolor{currentstroke}%
\pgfsetdash{}{0pt}%
\pgfpathmoveto{\pgfqpoint{0.625000in}{1.252054in}}%
\pgfpathlineto{\pgfqpoint{0.632414in}{1.244737in}}%
\pgfpathlineto{\pgfqpoint{0.625000in}{1.237621in}}%
\pgfusepath{stroke}%
\end{pgfscope}%
\begin{pgfscope}%
\pgfpathrectangle{\pgfqpoint{0.625000in}{0.550000in}}{\pgfqpoint{3.875000in}{3.850000in}} %
\pgfusepath{clip}%
\pgfsetbuttcap%
\pgfsetroundjoin%
\pgfsetlinewidth{0.250937pt}%
\definecolor{currentstroke}{rgb}{0.000000,0.000000,0.000000}%
\pgfsetstrokecolor{currentstroke}%
\pgfsetdash{}{0pt}%
\pgfpathmoveto{\pgfqpoint{0.625000in}{1.406112in}}%
\pgfpathlineto{\pgfqpoint{0.632347in}{1.399123in}}%
\pgfpathlineto{\pgfqpoint{0.625000in}{1.392129in}}%
\pgfusepath{stroke}%
\end{pgfscope}%
\begin{pgfscope}%
\pgfpathrectangle{\pgfqpoint{0.625000in}{0.550000in}}{\pgfqpoint{3.875000in}{3.850000in}} %
\pgfusepath{clip}%
\pgfsetbuttcap%
\pgfsetroundjoin%
\pgfsetlinewidth{0.250937pt}%
\definecolor{currentstroke}{rgb}{0.000000,0.000000,0.000000}%
\pgfsetstrokecolor{currentstroke}%
\pgfsetdash{}{0pt}%
\pgfpathmoveto{\pgfqpoint{0.625000in}{1.560493in}}%
\pgfpathlineto{\pgfqpoint{0.632330in}{1.553509in}}%
\pgfpathlineto{\pgfqpoint{0.625000in}{1.546298in}}%
\pgfusepath{stroke}%
\end{pgfscope}%
\begin{pgfscope}%
\pgfpathrectangle{\pgfqpoint{0.625000in}{0.550000in}}{\pgfqpoint{3.875000in}{3.850000in}} %
\pgfusepath{clip}%
\pgfsetbuttcap%
\pgfsetroundjoin%
\pgfsetlinewidth{0.250937pt}%
\definecolor{currentstroke}{rgb}{0.000000,0.000000,0.000000}%
\pgfsetstrokecolor{currentstroke}%
\pgfsetdash{}{0pt}%
\pgfpathmoveto{\pgfqpoint{0.625000in}{1.714865in}}%
\pgfpathlineto{\pgfqpoint{0.633980in}{1.707895in}}%
\pgfpathlineto{\pgfqpoint{0.625000in}{1.700881in}}%
\pgfusepath{stroke}%
\end{pgfscope}%
\begin{pgfscope}%
\pgfpathrectangle{\pgfqpoint{0.625000in}{0.550000in}}{\pgfqpoint{3.875000in}{3.850000in}} %
\pgfusepath{clip}%
\pgfsetbuttcap%
\pgfsetroundjoin%
\pgfsetlinewidth{0.250937pt}%
\definecolor{currentstroke}{rgb}{0.000000,0.000000,0.000000}%
\pgfsetstrokecolor{currentstroke}%
\pgfsetdash{}{0pt}%
\pgfpathmoveto{\pgfqpoint{0.625000in}{1.869291in}}%
\pgfpathlineto{\pgfqpoint{0.632276in}{1.862281in}}%
\pgfpathlineto{\pgfqpoint{0.625000in}{1.855415in}}%
\pgfusepath{stroke}%
\end{pgfscope}%
\begin{pgfscope}%
\pgfpathrectangle{\pgfqpoint{0.625000in}{0.550000in}}{\pgfqpoint{3.875000in}{3.850000in}} %
\pgfusepath{clip}%
\pgfsetbuttcap%
\pgfsetroundjoin%
\pgfsetlinewidth{0.250937pt}%
\definecolor{currentstroke}{rgb}{0.000000,0.000000,0.000000}%
\pgfsetstrokecolor{currentstroke}%
\pgfsetdash{}{0pt}%
\pgfpathmoveto{\pgfqpoint{0.625000in}{2.023906in}}%
\pgfpathlineto{\pgfqpoint{0.632246in}{2.016667in}}%
\pgfpathlineto{\pgfqpoint{0.625000in}{2.009487in}}%
\pgfusepath{stroke}%
\end{pgfscope}%
\begin{pgfscope}%
\pgfpathrectangle{\pgfqpoint{0.625000in}{0.550000in}}{\pgfqpoint{3.875000in}{3.850000in}} %
\pgfusepath{clip}%
\pgfsetbuttcap%
\pgfsetroundjoin%
\pgfsetlinewidth{0.250937pt}%
\definecolor{currentstroke}{rgb}{0.000000,0.000000,0.000000}%
\pgfsetstrokecolor{currentstroke}%
\pgfsetdash{}{0pt}%
\pgfpathmoveto{\pgfqpoint{0.625000in}{2.177933in}}%
\pgfpathlineto{\pgfqpoint{0.632277in}{2.171053in}}%
\pgfpathlineto{\pgfqpoint{0.625000in}{2.163975in}}%
\pgfusepath{stroke}%
\end{pgfscope}%
\begin{pgfscope}%
\pgfpathrectangle{\pgfqpoint{0.625000in}{0.550000in}}{\pgfqpoint{3.875000in}{3.850000in}} %
\pgfusepath{clip}%
\pgfsetbuttcap%
\pgfsetroundjoin%
\pgfsetlinewidth{0.250937pt}%
\definecolor{currentstroke}{rgb}{0.000000,0.000000,0.000000}%
\pgfsetstrokecolor{currentstroke}%
\pgfsetdash{}{0pt}%
\pgfpathmoveto{\pgfqpoint{0.625000in}{2.332273in}}%
\pgfpathlineto{\pgfqpoint{0.632296in}{2.325439in}}%
\pgfpathlineto{\pgfqpoint{0.625000in}{2.318126in}}%
\pgfusepath{stroke}%
\end{pgfscope}%
\begin{pgfscope}%
\pgfpathrectangle{\pgfqpoint{0.625000in}{0.550000in}}{\pgfqpoint{3.875000in}{3.850000in}} %
\pgfusepath{clip}%
\pgfsetbuttcap%
\pgfsetroundjoin%
\pgfsetlinewidth{0.250937pt}%
\definecolor{currentstroke}{rgb}{0.000000,0.000000,0.000000}%
\pgfsetstrokecolor{currentstroke}%
\pgfsetdash{}{0pt}%
\pgfpathmoveto{\pgfqpoint{0.625000in}{2.487628in}}%
\pgfpathlineto{\pgfqpoint{0.626731in}{2.489474in}}%
\pgfpathlineto{\pgfqpoint{0.627463in}{2.499123in}}%
\pgfpathlineto{\pgfqpoint{0.630114in}{2.508772in}}%
\pgfpathlineto{\pgfqpoint{0.633551in}{2.518421in}}%
\pgfpathlineto{\pgfqpoint{0.634712in}{2.521872in}}%
\pgfpathlineto{\pgfqpoint{0.638715in}{2.528070in}}%
\pgfpathlineto{\pgfqpoint{0.644424in}{2.537499in}}%
\pgfpathlineto{\pgfqpoint{0.644666in}{2.537719in}}%
\pgfpathlineto{\pgfqpoint{0.652833in}{2.547368in}}%
\pgfpathlineto{\pgfqpoint{0.654135in}{2.549150in}}%
\pgfpathlineto{\pgfqpoint{0.663115in}{2.557018in}}%
\pgfpathlineto{\pgfqpoint{0.663847in}{2.557796in}}%
\pgfpathlineto{\pgfqpoint{0.673559in}{2.564788in}}%
\pgfpathlineto{\pgfqpoint{0.677426in}{2.566667in}}%
\pgfpathlineto{\pgfqpoint{0.683271in}{2.570247in}}%
\pgfpathlineto{\pgfqpoint{0.692982in}{2.574360in}}%
\pgfpathlineto{\pgfqpoint{0.699685in}{2.576316in}}%
\pgfpathlineto{\pgfqpoint{0.702694in}{2.577408in}}%
\pgfpathlineto{\pgfqpoint{0.712406in}{2.579611in}}%
\pgfpathlineto{\pgfqpoint{0.722118in}{2.580814in}}%
\pgfpathlineto{\pgfqpoint{0.731830in}{2.581108in}}%
\pgfpathlineto{\pgfqpoint{0.741541in}{2.580511in}}%
\pgfpathlineto{\pgfqpoint{0.751253in}{2.578998in}}%
\pgfpathlineto{\pgfqpoint{0.760965in}{2.576513in}}%
\pgfpathlineto{\pgfqpoint{0.761569in}{2.576316in}}%
\pgfpathlineto{\pgfqpoint{0.770677in}{2.573066in}}%
\pgfpathlineto{\pgfqpoint{0.780388in}{2.568437in}}%
\pgfpathlineto{\pgfqpoint{0.783540in}{2.566667in}}%
\pgfpathlineto{\pgfqpoint{0.790100in}{2.562450in}}%
\pgfpathlineto{\pgfqpoint{0.797331in}{2.557018in}}%
\pgfpathlineto{\pgfqpoint{0.799812in}{2.554808in}}%
\pgfpathlineto{\pgfqpoint{0.807459in}{2.547368in}}%
\pgfpathlineto{\pgfqpoint{0.809524in}{2.544898in}}%
\pgfpathlineto{\pgfqpoint{0.815287in}{2.537719in}}%
\pgfpathlineto{\pgfqpoint{0.819236in}{2.531429in}}%
\pgfpathlineto{\pgfqpoint{0.821333in}{2.528070in}}%
\pgfpathlineto{\pgfqpoint{0.826036in}{2.518421in}}%
\pgfpathlineto{\pgfqpoint{0.828947in}{2.510141in}}%
\pgfpathlineto{\pgfqpoint{0.829449in}{2.508772in}}%
\pgfpathlineto{\pgfqpoint{0.831939in}{2.499123in}}%
\pgfpathlineto{\pgfqpoint{0.833380in}{2.489474in}}%
\pgfpathlineto{\pgfqpoint{0.833851in}{2.479825in}}%
\pgfpathlineto{\pgfqpoint{0.833380in}{2.470175in}}%
\pgfpathlineto{\pgfqpoint{0.831939in}{2.460526in}}%
\pgfpathlineto{\pgfqpoint{0.829449in}{2.450877in}}%
\pgfpathlineto{\pgfqpoint{0.828947in}{2.449508in}}%
\pgfpathlineto{\pgfqpoint{0.826036in}{2.441228in}}%
\pgfpathlineto{\pgfqpoint{0.821333in}{2.431579in}}%
\pgfpathlineto{\pgfqpoint{0.819236in}{2.428220in}}%
\pgfpathlineto{\pgfqpoint{0.815287in}{2.421930in}}%
\pgfpathlineto{\pgfqpoint{0.809524in}{2.414751in}}%
\pgfpathlineto{\pgfqpoint{0.807459in}{2.412281in}}%
\pgfpathlineto{\pgfqpoint{0.799812in}{2.404841in}}%
\pgfpathlineto{\pgfqpoint{0.797331in}{2.402632in}}%
\pgfpathlineto{\pgfqpoint{0.790100in}{2.397199in}}%
\pgfpathlineto{\pgfqpoint{0.783540in}{2.392982in}}%
\pgfpathlineto{\pgfqpoint{0.780388in}{2.391212in}}%
\pgfpathlineto{\pgfqpoint{0.770677in}{2.386583in}}%
\pgfpathlineto{\pgfqpoint{0.761569in}{2.383333in}}%
\pgfpathlineto{\pgfqpoint{0.760965in}{2.383136in}}%
\pgfpathlineto{\pgfqpoint{0.751253in}{2.380651in}}%
\pgfpathlineto{\pgfqpoint{0.741541in}{2.379138in}}%
\pgfpathlineto{\pgfqpoint{0.731830in}{2.378541in}}%
\pgfpathlineto{\pgfqpoint{0.722118in}{2.378835in}}%
\pgfpathlineto{\pgfqpoint{0.712406in}{2.380038in}}%
\pgfpathlineto{\pgfqpoint{0.702694in}{2.382241in}}%
\pgfpathlineto{\pgfqpoint{0.699685in}{2.383333in}}%
\pgfpathlineto{\pgfqpoint{0.692982in}{2.385289in}}%
\pgfpathlineto{\pgfqpoint{0.683271in}{2.389402in}}%
\pgfpathlineto{\pgfqpoint{0.677426in}{2.392982in}}%
\pgfpathlineto{\pgfqpoint{0.673559in}{2.394861in}}%
\pgfpathlineto{\pgfqpoint{0.663847in}{2.401853in}}%
\pgfpathlineto{\pgfqpoint{0.663115in}{2.402632in}}%
\pgfpathlineto{\pgfqpoint{0.654135in}{2.410499in}}%
\pgfpathlineto{\pgfqpoint{0.652833in}{2.412281in}}%
\pgfpathlineto{\pgfqpoint{0.644666in}{2.421930in}}%
\pgfpathlineto{\pgfqpoint{0.644424in}{2.422150in}}%
\pgfpathlineto{\pgfqpoint{0.638715in}{2.431579in}}%
\pgfpathlineto{\pgfqpoint{0.634712in}{2.437777in}}%
\pgfpathlineto{\pgfqpoint{0.633551in}{2.441228in}}%
\pgfpathlineto{\pgfqpoint{0.630114in}{2.450877in}}%
\pgfpathlineto{\pgfqpoint{0.627469in}{2.460526in}}%
\pgfpathlineto{\pgfqpoint{0.626731in}{2.470175in}}%
\pgfpathlineto{\pgfqpoint{0.625000in}{2.472021in}}%
\pgfusepath{stroke}%
\end{pgfscope}%
\begin{pgfscope}%
\pgfpathrectangle{\pgfqpoint{0.625000in}{0.550000in}}{\pgfqpoint{3.875000in}{3.850000in}} %
\pgfusepath{clip}%
\pgfsetbuttcap%
\pgfsetroundjoin%
\pgfsetlinewidth{0.250937pt}%
\definecolor{currentstroke}{rgb}{0.000000,0.000000,0.000000}%
\pgfsetstrokecolor{currentstroke}%
\pgfsetdash{}{0pt}%
\pgfpathmoveto{\pgfqpoint{0.625000in}{2.641519in}}%
\pgfpathlineto{\pgfqpoint{0.632291in}{2.634211in}}%
\pgfpathlineto{\pgfqpoint{0.625000in}{2.627381in}}%
\pgfusepath{stroke}%
\end{pgfscope}%
\begin{pgfscope}%
\pgfpathrectangle{\pgfqpoint{0.625000in}{0.550000in}}{\pgfqpoint{3.875000in}{3.850000in}} %
\pgfusepath{clip}%
\pgfsetbuttcap%
\pgfsetroundjoin%
\pgfsetlinewidth{0.250937pt}%
\definecolor{currentstroke}{rgb}{0.000000,0.000000,0.000000}%
\pgfsetstrokecolor{currentstroke}%
\pgfsetdash{}{0pt}%
\pgfpathmoveto{\pgfqpoint{0.625000in}{2.795659in}}%
\pgfpathlineto{\pgfqpoint{0.632261in}{2.788596in}}%
\pgfpathlineto{\pgfqpoint{0.625000in}{2.781732in}}%
\pgfusepath{stroke}%
\end{pgfscope}%
\begin{pgfscope}%
\pgfpathrectangle{\pgfqpoint{0.625000in}{0.550000in}}{\pgfqpoint{3.875000in}{3.850000in}} %
\pgfusepath{clip}%
\pgfsetbuttcap%
\pgfsetroundjoin%
\pgfsetlinewidth{0.250937pt}%
\definecolor{currentstroke}{rgb}{0.000000,0.000000,0.000000}%
\pgfsetstrokecolor{currentstroke}%
\pgfsetdash{}{0pt}%
\pgfpathmoveto{\pgfqpoint{0.625000in}{2.950120in}}%
\pgfpathlineto{\pgfqpoint{0.632203in}{2.942982in}}%
\pgfpathlineto{\pgfqpoint{0.625000in}{2.935786in}}%
\pgfusepath{stroke}%
\end{pgfscope}%
\begin{pgfscope}%
\pgfpathrectangle{\pgfqpoint{0.625000in}{0.550000in}}{\pgfqpoint{3.875000in}{3.850000in}} %
\pgfusepath{clip}%
\pgfsetbuttcap%
\pgfsetroundjoin%
\pgfsetlinewidth{0.250937pt}%
\definecolor{currentstroke}{rgb}{0.000000,0.000000,0.000000}%
\pgfsetstrokecolor{currentstroke}%
\pgfsetdash{}{0pt}%
\pgfpathmoveto{\pgfqpoint{0.625000in}{3.104198in}}%
\pgfpathlineto{\pgfqpoint{0.632243in}{3.097368in}}%
\pgfpathlineto{\pgfqpoint{0.625000in}{3.090393in}}%
\pgfusepath{stroke}%
\end{pgfscope}%
\begin{pgfscope}%
\pgfpathrectangle{\pgfqpoint{0.625000in}{0.550000in}}{\pgfqpoint{3.875000in}{3.850000in}} %
\pgfusepath{clip}%
\pgfsetbuttcap%
\pgfsetroundjoin%
\pgfsetlinewidth{0.250937pt}%
\definecolor{currentstroke}{rgb}{0.000000,0.000000,0.000000}%
\pgfsetstrokecolor{currentstroke}%
\pgfsetdash{}{0pt}%
\pgfpathmoveto{\pgfqpoint{0.625000in}{3.258641in}}%
\pgfpathlineto{\pgfqpoint{0.633935in}{3.251754in}}%
\pgfpathlineto{\pgfqpoint{0.625000in}{3.244913in}}%
\pgfusepath{stroke}%
\end{pgfscope}%
\begin{pgfscope}%
\pgfpathrectangle{\pgfqpoint{0.625000in}{0.550000in}}{\pgfqpoint{3.875000in}{3.850000in}} %
\pgfusepath{clip}%
\pgfsetbuttcap%
\pgfsetroundjoin%
\pgfsetlinewidth{0.250937pt}%
\definecolor{currentstroke}{rgb}{0.000000,0.000000,0.000000}%
\pgfsetstrokecolor{currentstroke}%
\pgfsetdash{}{0pt}%
\pgfpathmoveto{\pgfqpoint{0.625000in}{3.413267in}}%
\pgfpathlineto{\pgfqpoint{0.632247in}{3.406140in}}%
\pgfpathlineto{\pgfqpoint{0.625000in}{3.399245in}}%
\pgfusepath{stroke}%
\end{pgfscope}%
\begin{pgfscope}%
\pgfpathrectangle{\pgfqpoint{0.625000in}{0.550000in}}{\pgfqpoint{3.875000in}{3.850000in}} %
\pgfusepath{clip}%
\pgfsetbuttcap%
\pgfsetroundjoin%
\pgfsetlinewidth{0.250937pt}%
\definecolor{currentstroke}{rgb}{0.000000,0.000000,0.000000}%
\pgfsetstrokecolor{currentstroke}%
\pgfsetdash{}{0pt}%
\pgfpathmoveto{\pgfqpoint{0.625000in}{3.567414in}}%
\pgfpathlineto{\pgfqpoint{0.632249in}{3.560526in}}%
\pgfpathlineto{\pgfqpoint{0.625000in}{3.553642in}}%
\pgfusepath{stroke}%
\end{pgfscope}%
\begin{pgfscope}%
\pgfpathrectangle{\pgfqpoint{0.625000in}{0.550000in}}{\pgfqpoint{3.875000in}{3.850000in}} %
\pgfusepath{clip}%
\pgfsetbuttcap%
\pgfsetroundjoin%
\pgfsetlinewidth{0.250937pt}%
\definecolor{currentstroke}{rgb}{0.000000,0.000000,0.000000}%
\pgfsetstrokecolor{currentstroke}%
\pgfsetdash{}{0pt}%
\pgfpathmoveto{\pgfqpoint{0.625000in}{3.721873in}}%
\pgfpathlineto{\pgfqpoint{0.632269in}{3.714912in}}%
\pgfpathlineto{\pgfqpoint{0.625000in}{3.707741in}}%
\pgfusepath{stroke}%
\end{pgfscope}%
\begin{pgfscope}%
\pgfpathrectangle{\pgfqpoint{0.625000in}{0.550000in}}{\pgfqpoint{3.875000in}{3.850000in}} %
\pgfusepath{clip}%
\pgfsetbuttcap%
\pgfsetroundjoin%
\pgfsetlinewidth{0.250937pt}%
\definecolor{currentstroke}{rgb}{0.000000,0.000000,0.000000}%
\pgfsetstrokecolor{currentstroke}%
\pgfsetdash{}{0pt}%
\pgfpathmoveto{\pgfqpoint{0.625000in}{3.876123in}}%
\pgfpathlineto{\pgfqpoint{0.632172in}{3.869298in}}%
\pgfpathlineto{\pgfqpoint{0.625000in}{3.862445in}}%
\pgfusepath{stroke}%
\end{pgfscope}%
\begin{pgfscope}%
\pgfpathrectangle{\pgfqpoint{0.625000in}{0.550000in}}{\pgfqpoint{3.875000in}{3.850000in}} %
\pgfusepath{clip}%
\pgfsetbuttcap%
\pgfsetroundjoin%
\pgfsetlinewidth{0.250937pt}%
\definecolor{currentstroke}{rgb}{0.000000,0.000000,0.000000}%
\pgfsetstrokecolor{currentstroke}%
\pgfsetdash{}{0pt}%
\pgfpathmoveto{\pgfqpoint{0.625000in}{4.030654in}}%
\pgfpathlineto{\pgfqpoint{0.633951in}{4.023684in}}%
\pgfpathlineto{\pgfqpoint{0.625000in}{4.016783in}}%
\pgfusepath{stroke}%
\end{pgfscope}%
\begin{pgfscope}%
\pgfpathrectangle{\pgfqpoint{0.625000in}{0.550000in}}{\pgfqpoint{3.875000in}{3.850000in}} %
\pgfusepath{clip}%
\pgfsetbuttcap%
\pgfsetroundjoin%
\pgfsetlinewidth{0.250937pt}%
\definecolor{currentstroke}{rgb}{0.000000,0.000000,0.000000}%
\pgfsetstrokecolor{currentstroke}%
\pgfsetdash{}{0pt}%
\pgfpathmoveto{\pgfqpoint{0.625000in}{4.184961in}}%
\pgfpathlineto{\pgfqpoint{0.632158in}{4.178070in}}%
\pgfpathlineto{\pgfqpoint{0.625000in}{4.171502in}}%
\pgfusepath{stroke}%
\end{pgfscope}%
\begin{pgfscope}%
\pgfpathrectangle{\pgfqpoint{0.625000in}{0.550000in}}{\pgfqpoint{3.875000in}{3.850000in}} %
\pgfusepath{clip}%
\pgfsetbuttcap%
\pgfsetroundjoin%
\pgfsetlinewidth{0.250937pt}%
\definecolor{currentstroke}{rgb}{0.000000,0.000000,0.000000}%
\pgfsetstrokecolor{currentstroke}%
\pgfsetdash{}{0pt}%
\pgfpathmoveto{\pgfqpoint{0.625000in}{4.339322in}}%
\pgfpathlineto{\pgfqpoint{0.632132in}{4.332456in}}%
\pgfpathlineto{\pgfqpoint{0.625000in}{4.325567in}}%
\pgfusepath{stroke}%
\end{pgfscope}%
\begin{pgfscope}%
\pgfpathrectangle{\pgfqpoint{0.625000in}{0.550000in}}{\pgfqpoint{3.875000in}{3.850000in}} %
\pgfusepath{clip}%
\pgfsetbuttcap%
\pgfsetroundjoin%
\pgfsetlinewidth{0.250937pt}%
\definecolor{currentstroke}{rgb}{0.000000,0.000000,0.000000}%
\pgfsetstrokecolor{currentstroke}%
\pgfsetdash{}{0pt}%
\pgfpathmoveto{\pgfqpoint{0.634712in}{1.181656in}}%
\pgfpathlineto{\pgfqpoint{0.631485in}{1.186842in}}%
\pgfpathlineto{\pgfqpoint{0.629515in}{1.196491in}}%
\pgfpathlineto{\pgfqpoint{0.634712in}{1.204299in}}%
\pgfpathlineto{\pgfqpoint{0.644424in}{1.198741in}}%
\pgfpathlineto{\pgfqpoint{0.646527in}{1.196491in}}%
\pgfpathlineto{\pgfqpoint{0.644424in}{1.187875in}}%
\pgfpathlineto{\pgfqpoint{0.644261in}{1.186842in}}%
\pgfpathlineto{\pgfqpoint{0.634712in}{1.181656in}}%
\pgfusepath{stroke}%
\end{pgfscope}%
\begin{pgfscope}%
\pgfpathrectangle{\pgfqpoint{0.625000in}{0.550000in}}{\pgfqpoint{3.875000in}{3.850000in}} %
\pgfusepath{clip}%
\pgfsetbuttcap%
\pgfsetroundjoin%
\pgfsetlinewidth{0.250937pt}%
\definecolor{currentstroke}{rgb}{0.000000,0.000000,0.000000}%
\pgfsetstrokecolor{currentstroke}%
\pgfsetdash{}{0pt}%
\pgfpathmoveto{\pgfqpoint{0.634712in}{3.755350in}}%
\pgfpathlineto{\pgfqpoint{0.629515in}{3.763158in}}%
\pgfpathlineto{\pgfqpoint{0.631485in}{3.772807in}}%
\pgfpathlineto{\pgfqpoint{0.634712in}{3.777993in}}%
\pgfpathlineto{\pgfqpoint{0.644261in}{3.772807in}}%
\pgfpathlineto{\pgfqpoint{0.644424in}{3.771774in}}%
\pgfpathlineto{\pgfqpoint{0.646527in}{3.763158in}}%
\pgfpathlineto{\pgfqpoint{0.644424in}{3.760908in}}%
\pgfpathlineto{\pgfqpoint{0.634712in}{3.755350in}}%
\pgfusepath{stroke}%
\end{pgfscope}%
\begin{pgfscope}%
\pgfpathrectangle{\pgfqpoint{0.625000in}{0.550000in}}{\pgfqpoint{3.875000in}{3.850000in}} %
\pgfusepath{clip}%
\pgfsetbuttcap%
\pgfsetroundjoin%
\pgfsetlinewidth{0.250937pt}%
\definecolor{currentstroke}{rgb}{0.000000,0.000000,0.000000}%
\pgfsetstrokecolor{currentstroke}%
\pgfsetdash{}{0pt}%
\pgfpathmoveto{\pgfqpoint{0.625000in}{0.634523in}}%
\pgfpathlineto{\pgfqpoint{0.632560in}{0.627193in}}%
\pgfpathlineto{\pgfqpoint{0.625000in}{0.619885in}}%
\pgfusepath{stroke}%
\end{pgfscope}%
\begin{pgfscope}%
\pgfpathrectangle{\pgfqpoint{0.625000in}{0.550000in}}{\pgfqpoint{3.875000in}{3.850000in}} %
\pgfusepath{clip}%
\pgfsetbuttcap%
\pgfsetroundjoin%
\pgfsetlinewidth{0.250937pt}%
\definecolor{currentstroke}{rgb}{0.000000,0.000000,0.000000}%
\pgfsetstrokecolor{currentstroke}%
\pgfsetdash{}{0pt}%
\pgfpathmoveto{\pgfqpoint{0.625000in}{0.788499in}}%
\pgfpathlineto{\pgfqpoint{0.632487in}{0.781579in}}%
\pgfpathlineto{\pgfqpoint{0.625000in}{0.774350in}}%
\pgfusepath{stroke}%
\end{pgfscope}%
\begin{pgfscope}%
\pgfpathrectangle{\pgfqpoint{0.625000in}{0.550000in}}{\pgfqpoint{3.875000in}{3.850000in}} %
\pgfusepath{clip}%
\pgfsetbuttcap%
\pgfsetroundjoin%
\pgfsetlinewidth{0.250937pt}%
\definecolor{currentstroke}{rgb}{0.000000,0.000000,0.000000}%
\pgfsetstrokecolor{currentstroke}%
\pgfsetdash{}{0pt}%
\pgfpathmoveto{\pgfqpoint{0.625000in}{0.943176in}}%
\pgfpathlineto{\pgfqpoint{0.634129in}{0.935965in}}%
\pgfpathlineto{\pgfqpoint{0.625000in}{0.928689in}}%
\pgfusepath{stroke}%
\end{pgfscope}%
\begin{pgfscope}%
\pgfpathrectangle{\pgfqpoint{0.625000in}{0.550000in}}{\pgfqpoint{3.875000in}{3.850000in}} %
\pgfusepath{clip}%
\pgfsetbuttcap%
\pgfsetroundjoin%
\pgfsetlinewidth{0.250937pt}%
\definecolor{currentstroke}{rgb}{0.000000,0.000000,0.000000}%
\pgfsetstrokecolor{currentstroke}%
\pgfsetdash{}{0pt}%
\pgfpathmoveto{\pgfqpoint{0.625000in}{1.097539in}}%
\pgfpathlineto{\pgfqpoint{0.632496in}{1.090351in}}%
\pgfpathlineto{\pgfqpoint{0.625000in}{1.083189in}}%
\pgfusepath{stroke}%
\end{pgfscope}%
\begin{pgfscope}%
\pgfpathrectangle{\pgfqpoint{0.625000in}{0.550000in}}{\pgfqpoint{3.875000in}{3.850000in}} %
\pgfusepath{clip}%
\pgfsetbuttcap%
\pgfsetroundjoin%
\pgfsetlinewidth{0.250937pt}%
\definecolor{currentstroke}{rgb}{0.000000,0.000000,0.000000}%
\pgfsetstrokecolor{currentstroke}%
\pgfsetdash{}{0pt}%
\pgfpathmoveto{\pgfqpoint{0.625000in}{1.252137in}}%
\pgfpathlineto{\pgfqpoint{0.632498in}{1.244737in}}%
\pgfpathlineto{\pgfqpoint{0.625000in}{1.237541in}}%
\pgfusepath{stroke}%
\end{pgfscope}%
\begin{pgfscope}%
\pgfpathrectangle{\pgfqpoint{0.625000in}{0.550000in}}{\pgfqpoint{3.875000in}{3.850000in}} %
\pgfusepath{clip}%
\pgfsetbuttcap%
\pgfsetroundjoin%
\pgfsetlinewidth{0.250937pt}%
\definecolor{currentstroke}{rgb}{0.000000,0.000000,0.000000}%
\pgfsetstrokecolor{currentstroke}%
\pgfsetdash{}{0pt}%
\pgfpathmoveto{\pgfqpoint{0.625000in}{1.406194in}}%
\pgfpathlineto{\pgfqpoint{0.632433in}{1.399123in}}%
\pgfpathlineto{\pgfqpoint{0.625000in}{1.392048in}}%
\pgfusepath{stroke}%
\end{pgfscope}%
\begin{pgfscope}%
\pgfpathrectangle{\pgfqpoint{0.625000in}{0.550000in}}{\pgfqpoint{3.875000in}{3.850000in}} %
\pgfusepath{clip}%
\pgfsetbuttcap%
\pgfsetroundjoin%
\pgfsetlinewidth{0.250937pt}%
\definecolor{currentstroke}{rgb}{0.000000,0.000000,0.000000}%
\pgfsetstrokecolor{currentstroke}%
\pgfsetdash{}{0pt}%
\pgfpathmoveto{\pgfqpoint{0.625000in}{1.560573in}}%
\pgfpathlineto{\pgfqpoint{0.632414in}{1.553509in}}%
\pgfpathlineto{\pgfqpoint{0.625000in}{1.546215in}}%
\pgfusepath{stroke}%
\end{pgfscope}%
\begin{pgfscope}%
\pgfpathrectangle{\pgfqpoint{0.625000in}{0.550000in}}{\pgfqpoint{3.875000in}{3.850000in}} %
\pgfusepath{clip}%
\pgfsetbuttcap%
\pgfsetroundjoin%
\pgfsetlinewidth{0.250937pt}%
\definecolor{currentstroke}{rgb}{0.000000,0.000000,0.000000}%
\pgfsetstrokecolor{currentstroke}%
\pgfsetdash{}{0pt}%
\pgfpathmoveto{\pgfqpoint{0.625000in}{1.714945in}}%
\pgfpathlineto{\pgfqpoint{0.634084in}{1.707895in}}%
\pgfpathlineto{\pgfqpoint{0.625000in}{1.700799in}}%
\pgfusepath{stroke}%
\end{pgfscope}%
\begin{pgfscope}%
\pgfpathrectangle{\pgfqpoint{0.625000in}{0.550000in}}{\pgfqpoint{3.875000in}{3.850000in}} %
\pgfusepath{clip}%
\pgfsetbuttcap%
\pgfsetroundjoin%
\pgfsetlinewidth{0.250937pt}%
\definecolor{currentstroke}{rgb}{0.000000,0.000000,0.000000}%
\pgfsetstrokecolor{currentstroke}%
\pgfsetdash{}{0pt}%
\pgfpathmoveto{\pgfqpoint{0.625000in}{1.869374in}}%
\pgfpathlineto{\pgfqpoint{0.632362in}{1.862281in}}%
\pgfpathlineto{\pgfqpoint{0.625000in}{1.855334in}}%
\pgfusepath{stroke}%
\end{pgfscope}%
\begin{pgfscope}%
\pgfpathrectangle{\pgfqpoint{0.625000in}{0.550000in}}{\pgfqpoint{3.875000in}{3.850000in}} %
\pgfusepath{clip}%
\pgfsetbuttcap%
\pgfsetroundjoin%
\pgfsetlinewidth{0.250937pt}%
\definecolor{currentstroke}{rgb}{0.000000,0.000000,0.000000}%
\pgfsetstrokecolor{currentstroke}%
\pgfsetdash{}{0pt}%
\pgfpathmoveto{\pgfqpoint{0.625000in}{2.023993in}}%
\pgfpathlineto{\pgfqpoint{0.632333in}{2.016667in}}%
\pgfpathlineto{\pgfqpoint{0.625000in}{2.009400in}}%
\pgfusepath{stroke}%
\end{pgfscope}%
\begin{pgfscope}%
\pgfpathrectangle{\pgfqpoint{0.625000in}{0.550000in}}{\pgfqpoint{3.875000in}{3.850000in}} %
\pgfusepath{clip}%
\pgfsetbuttcap%
\pgfsetroundjoin%
\pgfsetlinewidth{0.250937pt}%
\definecolor{currentstroke}{rgb}{0.000000,0.000000,0.000000}%
\pgfsetstrokecolor{currentstroke}%
\pgfsetdash{}{0pt}%
\pgfpathmoveto{\pgfqpoint{0.625000in}{2.178016in}}%
\pgfpathlineto{\pgfqpoint{0.632364in}{2.171053in}}%
\pgfpathlineto{\pgfqpoint{0.625000in}{2.163890in}}%
\pgfusepath{stroke}%
\end{pgfscope}%
\begin{pgfscope}%
\pgfpathrectangle{\pgfqpoint{0.625000in}{0.550000in}}{\pgfqpoint{3.875000in}{3.850000in}} %
\pgfusepath{clip}%
\pgfsetbuttcap%
\pgfsetroundjoin%
\pgfsetlinewidth{0.250937pt}%
\definecolor{currentstroke}{rgb}{0.000000,0.000000,0.000000}%
\pgfsetstrokecolor{currentstroke}%
\pgfsetdash{}{0pt}%
\pgfpathmoveto{\pgfqpoint{0.625000in}{2.332355in}}%
\pgfpathlineto{\pgfqpoint{0.632384in}{2.325439in}}%
\pgfpathlineto{\pgfqpoint{0.625000in}{2.318037in}}%
\pgfusepath{stroke}%
\end{pgfscope}%
\begin{pgfscope}%
\pgfpathrectangle{\pgfqpoint{0.625000in}{0.550000in}}{\pgfqpoint{3.875000in}{3.850000in}} %
\pgfusepath{clip}%
\pgfsetbuttcap%
\pgfsetroundjoin%
\pgfsetlinewidth{0.250937pt}%
\definecolor{currentstroke}{rgb}{0.000000,0.000000,0.000000}%
\pgfsetstrokecolor{currentstroke}%
\pgfsetdash{}{0pt}%
\pgfpathmoveto{\pgfqpoint{0.625000in}{2.487712in}}%
\pgfpathlineto{\pgfqpoint{0.626652in}{2.489474in}}%
\pgfpathlineto{\pgfqpoint{0.627385in}{2.499123in}}%
\pgfpathlineto{\pgfqpoint{0.629959in}{2.508772in}}%
\pgfpathlineto{\pgfqpoint{0.633289in}{2.518421in}}%
\pgfpathlineto{\pgfqpoint{0.634712in}{2.522652in}}%
\pgfpathlineto{\pgfqpoint{0.638211in}{2.528070in}}%
\pgfpathlineto{\pgfqpoint{0.643931in}{2.537719in}}%
\pgfpathlineto{\pgfqpoint{0.644424in}{2.538675in}}%
\pgfpathlineto{\pgfqpoint{0.651740in}{2.547368in}}%
\pgfpathlineto{\pgfqpoint{0.654135in}{2.550646in}}%
\pgfpathlineto{\pgfqpoint{0.661407in}{2.557018in}}%
\pgfpathlineto{\pgfqpoint{0.663847in}{2.559612in}}%
\pgfpathlineto{\pgfqpoint{0.673559in}{2.566547in}}%
\pgfpathlineto{\pgfqpoint{0.673806in}{2.566667in}}%
\pgfpathlineto{\pgfqpoint{0.683271in}{2.572465in}}%
\pgfpathlineto{\pgfqpoint{0.692196in}{2.576316in}}%
\pgfpathlineto{\pgfqpoint{0.692982in}{2.576736in}}%
\pgfpathlineto{\pgfqpoint{0.702694in}{2.580281in}}%
\pgfpathlineto{\pgfqpoint{0.712406in}{2.582686in}}%
\pgfpathlineto{\pgfqpoint{0.722118in}{2.584158in}}%
\pgfpathlineto{\pgfqpoint{0.731830in}{2.584786in}}%
\pgfpathlineto{\pgfqpoint{0.741541in}{2.584589in}}%
\pgfpathlineto{\pgfqpoint{0.751253in}{2.583542in}}%
\pgfpathlineto{\pgfqpoint{0.760965in}{2.581591in}}%
\pgfpathlineto{\pgfqpoint{0.770677in}{2.578669in}}%
\pgfpathlineto{\pgfqpoint{0.776684in}{2.576316in}}%
\pgfpathlineto{\pgfqpoint{0.780388in}{2.574714in}}%
\pgfpathlineto{\pgfqpoint{0.790100in}{2.569597in}}%
\pgfpathlineto{\pgfqpoint{0.794802in}{2.566667in}}%
\pgfpathlineto{\pgfqpoint{0.799812in}{2.563070in}}%
\pgfpathlineto{\pgfqpoint{0.807241in}{2.557018in}}%
\pgfpathlineto{\pgfqpoint{0.809524in}{2.554803in}}%
\pgfpathlineto{\pgfqpoint{0.816667in}{2.547368in}}%
\pgfpathlineto{\pgfqpoint{0.819236in}{2.544076in}}%
\pgfpathlineto{\pgfqpoint{0.824048in}{2.537719in}}%
\pgfpathlineto{\pgfqpoint{0.828947in}{2.529446in}}%
\pgfpathlineto{\pgfqpoint{0.829762in}{2.528070in}}%
\pgfpathlineto{\pgfqpoint{0.834313in}{2.518421in}}%
\pgfpathlineto{\pgfqpoint{0.837598in}{2.508772in}}%
\pgfpathlineto{\pgfqpoint{0.838659in}{2.504239in}}%
\pgfpathlineto{\pgfqpoint{0.839928in}{2.499123in}}%
\pgfpathlineto{\pgfqpoint{0.841338in}{2.489474in}}%
\pgfpathlineto{\pgfqpoint{0.841800in}{2.479825in}}%
\pgfpathlineto{\pgfqpoint{0.841338in}{2.470175in}}%
\pgfpathlineto{\pgfqpoint{0.839928in}{2.460526in}}%
\pgfpathlineto{\pgfqpoint{0.838659in}{2.455410in}}%
\pgfpathlineto{\pgfqpoint{0.837598in}{2.450877in}}%
\pgfpathlineto{\pgfqpoint{0.834313in}{2.441228in}}%
\pgfpathlineto{\pgfqpoint{0.829762in}{2.431579in}}%
\pgfpathlineto{\pgfqpoint{0.828947in}{2.430204in}}%
\pgfpathlineto{\pgfqpoint{0.824048in}{2.421930in}}%
\pgfpathlineto{\pgfqpoint{0.819236in}{2.415573in}}%
\pgfpathlineto{\pgfqpoint{0.816667in}{2.412281in}}%
\pgfpathlineto{\pgfqpoint{0.809524in}{2.404846in}}%
\pgfpathlineto{\pgfqpoint{0.807241in}{2.402632in}}%
\pgfpathlineto{\pgfqpoint{0.799812in}{2.396579in}}%
\pgfpathlineto{\pgfqpoint{0.794802in}{2.392982in}}%
\pgfpathlineto{\pgfqpoint{0.790100in}{2.390052in}}%
\pgfpathlineto{\pgfqpoint{0.780388in}{2.384936in}}%
\pgfpathlineto{\pgfqpoint{0.776684in}{2.383333in}}%
\pgfpathlineto{\pgfqpoint{0.770677in}{2.380980in}}%
\pgfpathlineto{\pgfqpoint{0.760965in}{2.378058in}}%
\pgfpathlineto{\pgfqpoint{0.751253in}{2.376108in}}%
\pgfpathlineto{\pgfqpoint{0.741541in}{2.375060in}}%
\pgfpathlineto{\pgfqpoint{0.731830in}{2.374863in}}%
\pgfpathlineto{\pgfqpoint{0.722118in}{2.375491in}}%
\pgfpathlineto{\pgfqpoint{0.712406in}{2.376964in}}%
\pgfpathlineto{\pgfqpoint{0.702694in}{2.379368in}}%
\pgfpathlineto{\pgfqpoint{0.692982in}{2.382913in}}%
\pgfpathlineto{\pgfqpoint{0.692196in}{2.383333in}}%
\pgfpathlineto{\pgfqpoint{0.683271in}{2.387185in}}%
\pgfpathlineto{\pgfqpoint{0.673806in}{2.392982in}}%
\pgfpathlineto{\pgfqpoint{0.673559in}{2.393102in}}%
\pgfpathlineto{\pgfqpoint{0.663847in}{2.400037in}}%
\pgfpathlineto{\pgfqpoint{0.661407in}{2.402632in}}%
\pgfpathlineto{\pgfqpoint{0.654135in}{2.409003in}}%
\pgfpathlineto{\pgfqpoint{0.651740in}{2.412281in}}%
\pgfpathlineto{\pgfqpoint{0.644424in}{2.420974in}}%
\pgfpathlineto{\pgfqpoint{0.643931in}{2.421930in}}%
\pgfpathlineto{\pgfqpoint{0.638211in}{2.431579in}}%
\pgfpathlineto{\pgfqpoint{0.634712in}{2.436997in}}%
\pgfpathlineto{\pgfqpoint{0.633289in}{2.441228in}}%
\pgfpathlineto{\pgfqpoint{0.629959in}{2.450877in}}%
\pgfpathlineto{\pgfqpoint{0.627390in}{2.460526in}}%
\pgfpathlineto{\pgfqpoint{0.626652in}{2.470175in}}%
\pgfpathlineto{\pgfqpoint{0.625000in}{2.471937in}}%
\pgfusepath{stroke}%
\end{pgfscope}%
\begin{pgfscope}%
\pgfpathrectangle{\pgfqpoint{0.625000in}{0.550000in}}{\pgfqpoint{3.875000in}{3.850000in}} %
\pgfusepath{clip}%
\pgfsetbuttcap%
\pgfsetroundjoin%
\pgfsetlinewidth{0.250937pt}%
\definecolor{currentstroke}{rgb}{0.000000,0.000000,0.000000}%
\pgfsetstrokecolor{currentstroke}%
\pgfsetdash{}{0pt}%
\pgfpathmoveto{\pgfqpoint{0.625000in}{2.641608in}}%
\pgfpathlineto{\pgfqpoint{0.632380in}{2.634211in}}%
\pgfpathlineto{\pgfqpoint{0.625000in}{2.627299in}}%
\pgfusepath{stroke}%
\end{pgfscope}%
\begin{pgfscope}%
\pgfpathrectangle{\pgfqpoint{0.625000in}{0.550000in}}{\pgfqpoint{3.875000in}{3.850000in}} %
\pgfusepath{clip}%
\pgfsetbuttcap%
\pgfsetroundjoin%
\pgfsetlinewidth{0.250937pt}%
\definecolor{currentstroke}{rgb}{0.000000,0.000000,0.000000}%
\pgfsetstrokecolor{currentstroke}%
\pgfsetdash{}{0pt}%
\pgfpathmoveto{\pgfqpoint{0.625000in}{2.795744in}}%
\pgfpathlineto{\pgfqpoint{0.632349in}{2.788596in}}%
\pgfpathlineto{\pgfqpoint{0.625000in}{2.781649in}}%
\pgfusepath{stroke}%
\end{pgfscope}%
\begin{pgfscope}%
\pgfpathrectangle{\pgfqpoint{0.625000in}{0.550000in}}{\pgfqpoint{3.875000in}{3.850000in}} %
\pgfusepath{clip}%
\pgfsetbuttcap%
\pgfsetroundjoin%
\pgfsetlinewidth{0.250937pt}%
\definecolor{currentstroke}{rgb}{0.000000,0.000000,0.000000}%
\pgfsetstrokecolor{currentstroke}%
\pgfsetdash{}{0pt}%
\pgfpathmoveto{\pgfqpoint{0.625000in}{2.950208in}}%
\pgfpathlineto{\pgfqpoint{0.632292in}{2.942982in}}%
\pgfpathlineto{\pgfqpoint{0.625000in}{2.935697in}}%
\pgfusepath{stroke}%
\end{pgfscope}%
\begin{pgfscope}%
\pgfpathrectangle{\pgfqpoint{0.625000in}{0.550000in}}{\pgfqpoint{3.875000in}{3.850000in}} %
\pgfusepath{clip}%
\pgfsetbuttcap%
\pgfsetroundjoin%
\pgfsetlinewidth{0.250937pt}%
\definecolor{currentstroke}{rgb}{0.000000,0.000000,0.000000}%
\pgfsetstrokecolor{currentstroke}%
\pgfsetdash{}{0pt}%
\pgfpathmoveto{\pgfqpoint{0.625000in}{3.104280in}}%
\pgfpathlineto{\pgfqpoint{0.632330in}{3.097368in}}%
\pgfpathlineto{\pgfqpoint{0.625000in}{3.090309in}}%
\pgfusepath{stroke}%
\end{pgfscope}%
\begin{pgfscope}%
\pgfpathrectangle{\pgfqpoint{0.625000in}{0.550000in}}{\pgfqpoint{3.875000in}{3.850000in}} %
\pgfusepath{clip}%
\pgfsetbuttcap%
\pgfsetroundjoin%
\pgfsetlinewidth{0.250937pt}%
\definecolor{currentstroke}{rgb}{0.000000,0.000000,0.000000}%
\pgfsetstrokecolor{currentstroke}%
\pgfsetdash{}{0pt}%
\pgfpathmoveto{\pgfqpoint{0.625000in}{3.258727in}}%
\pgfpathlineto{\pgfqpoint{0.634045in}{3.251754in}}%
\pgfpathlineto{\pgfqpoint{0.625000in}{3.244828in}}%
\pgfusepath{stroke}%
\end{pgfscope}%
\begin{pgfscope}%
\pgfpathrectangle{\pgfqpoint{0.625000in}{0.550000in}}{\pgfqpoint{3.875000in}{3.850000in}} %
\pgfusepath{clip}%
\pgfsetbuttcap%
\pgfsetroundjoin%
\pgfsetlinewidth{0.250937pt}%
\definecolor{currentstroke}{rgb}{0.000000,0.000000,0.000000}%
\pgfsetstrokecolor{currentstroke}%
\pgfsetdash{}{0pt}%
\pgfpathmoveto{\pgfqpoint{0.625000in}{3.413353in}}%
\pgfpathlineto{\pgfqpoint{0.632334in}{3.406140in}}%
\pgfpathlineto{\pgfqpoint{0.625000in}{3.399163in}}%
\pgfusepath{stroke}%
\end{pgfscope}%
\begin{pgfscope}%
\pgfpathrectangle{\pgfqpoint{0.625000in}{0.550000in}}{\pgfqpoint{3.875000in}{3.850000in}} %
\pgfusepath{clip}%
\pgfsetbuttcap%
\pgfsetroundjoin%
\pgfsetlinewidth{0.250937pt}%
\definecolor{currentstroke}{rgb}{0.000000,0.000000,0.000000}%
\pgfsetstrokecolor{currentstroke}%
\pgfsetdash{}{0pt}%
\pgfpathmoveto{\pgfqpoint{0.625000in}{3.567499in}}%
\pgfpathlineto{\pgfqpoint{0.632339in}{3.560526in}}%
\pgfpathlineto{\pgfqpoint{0.625000in}{3.553558in}}%
\pgfusepath{stroke}%
\end{pgfscope}%
\begin{pgfscope}%
\pgfpathrectangle{\pgfqpoint{0.625000in}{0.550000in}}{\pgfqpoint{3.875000in}{3.850000in}} %
\pgfusepath{clip}%
\pgfsetbuttcap%
\pgfsetroundjoin%
\pgfsetlinewidth{0.250937pt}%
\definecolor{currentstroke}{rgb}{0.000000,0.000000,0.000000}%
\pgfsetstrokecolor{currentstroke}%
\pgfsetdash{}{0pt}%
\pgfpathmoveto{\pgfqpoint{0.625000in}{3.721959in}}%
\pgfpathlineto{\pgfqpoint{0.632358in}{3.714912in}}%
\pgfpathlineto{\pgfqpoint{0.625000in}{3.707653in}}%
\pgfusepath{stroke}%
\end{pgfscope}%
\begin{pgfscope}%
\pgfpathrectangle{\pgfqpoint{0.625000in}{0.550000in}}{\pgfqpoint{3.875000in}{3.850000in}} %
\pgfusepath{clip}%
\pgfsetbuttcap%
\pgfsetroundjoin%
\pgfsetlinewidth{0.250937pt}%
\definecolor{currentstroke}{rgb}{0.000000,0.000000,0.000000}%
\pgfsetstrokecolor{currentstroke}%
\pgfsetdash{}{0pt}%
\pgfpathmoveto{\pgfqpoint{0.625000in}{3.876210in}}%
\pgfpathlineto{\pgfqpoint{0.632263in}{3.869298in}}%
\pgfpathlineto{\pgfqpoint{0.625000in}{3.862358in}}%
\pgfusepath{stroke}%
\end{pgfscope}%
\begin{pgfscope}%
\pgfpathrectangle{\pgfqpoint{0.625000in}{0.550000in}}{\pgfqpoint{3.875000in}{3.850000in}} %
\pgfusepath{clip}%
\pgfsetbuttcap%
\pgfsetroundjoin%
\pgfsetlinewidth{0.250937pt}%
\definecolor{currentstroke}{rgb}{0.000000,0.000000,0.000000}%
\pgfsetstrokecolor{currentstroke}%
\pgfsetdash{}{0pt}%
\pgfpathmoveto{\pgfqpoint{0.625000in}{4.030738in}}%
\pgfpathlineto{\pgfqpoint{0.634059in}{4.023684in}}%
\pgfpathlineto{\pgfqpoint{0.625000in}{4.016699in}}%
\pgfusepath{stroke}%
\end{pgfscope}%
\begin{pgfscope}%
\pgfpathrectangle{\pgfqpoint{0.625000in}{0.550000in}}{\pgfqpoint{3.875000in}{3.850000in}} %
\pgfusepath{clip}%
\pgfsetbuttcap%
\pgfsetroundjoin%
\pgfsetlinewidth{0.250937pt}%
\definecolor{currentstroke}{rgb}{0.000000,0.000000,0.000000}%
\pgfsetstrokecolor{currentstroke}%
\pgfsetdash{}{0pt}%
\pgfpathmoveto{\pgfqpoint{0.625000in}{4.185049in}}%
\pgfpathlineto{\pgfqpoint{0.632250in}{4.178070in}}%
\pgfpathlineto{\pgfqpoint{0.625000in}{4.171418in}}%
\pgfusepath{stroke}%
\end{pgfscope}%
\begin{pgfscope}%
\pgfpathrectangle{\pgfqpoint{0.625000in}{0.550000in}}{\pgfqpoint{3.875000in}{3.850000in}} %
\pgfusepath{clip}%
\pgfsetbuttcap%
\pgfsetroundjoin%
\pgfsetlinewidth{0.250937pt}%
\definecolor{currentstroke}{rgb}{0.000000,0.000000,0.000000}%
\pgfsetstrokecolor{currentstroke}%
\pgfsetdash{}{0pt}%
\pgfpathmoveto{\pgfqpoint{0.625000in}{4.339409in}}%
\pgfpathlineto{\pgfqpoint{0.632223in}{4.332456in}}%
\pgfpathlineto{\pgfqpoint{0.625000in}{4.325480in}}%
\pgfusepath{stroke}%
\end{pgfscope}%
\begin{pgfscope}%
\pgfpathrectangle{\pgfqpoint{0.625000in}{0.550000in}}{\pgfqpoint{3.875000in}{3.850000in}} %
\pgfusepath{clip}%
\pgfsetbuttcap%
\pgfsetroundjoin%
\pgfsetlinewidth{0.250937pt}%
\definecolor{currentstroke}{rgb}{0.000000,0.000000,0.000000}%
\pgfsetstrokecolor{currentstroke}%
\pgfsetdash{}{0pt}%
\pgfpathmoveto{\pgfqpoint{0.634712in}{1.181322in}}%
\pgfpathlineto{\pgfqpoint{0.631277in}{1.186842in}}%
\pgfpathlineto{\pgfqpoint{0.629340in}{1.196491in}}%
\pgfpathlineto{\pgfqpoint{0.634712in}{1.204562in}}%
\pgfpathlineto{\pgfqpoint{0.644424in}{1.199761in}}%
\pgfpathlineto{\pgfqpoint{0.647481in}{1.196491in}}%
\pgfpathlineto{\pgfqpoint{0.645265in}{1.186842in}}%
\pgfpathlineto{\pgfqpoint{0.644424in}{1.186159in}}%
\pgfpathlineto{\pgfqpoint{0.634712in}{1.181322in}}%
\pgfusepath{stroke}%
\end{pgfscope}%
\begin{pgfscope}%
\pgfpathrectangle{\pgfqpoint{0.625000in}{0.550000in}}{\pgfqpoint{3.875000in}{3.850000in}} %
\pgfusepath{clip}%
\pgfsetbuttcap%
\pgfsetroundjoin%
\pgfsetlinewidth{0.250937pt}%
\definecolor{currentstroke}{rgb}{0.000000,0.000000,0.000000}%
\pgfsetstrokecolor{currentstroke}%
\pgfsetdash{}{0pt}%
\pgfpathmoveto{\pgfqpoint{0.634712in}{3.755087in}}%
\pgfpathlineto{\pgfqpoint{0.629340in}{3.763158in}}%
\pgfpathlineto{\pgfqpoint{0.631277in}{3.772807in}}%
\pgfpathlineto{\pgfqpoint{0.634712in}{3.778327in}}%
\pgfpathlineto{\pgfqpoint{0.644424in}{3.773490in}}%
\pgfpathlineto{\pgfqpoint{0.645265in}{3.772807in}}%
\pgfpathlineto{\pgfqpoint{0.647481in}{3.763158in}}%
\pgfpathlineto{\pgfqpoint{0.644424in}{3.759888in}}%
\pgfpathlineto{\pgfqpoint{0.634712in}{3.755087in}}%
\pgfusepath{stroke}%
\end{pgfscope}%
\begin{pgfscope}%
\pgfpathrectangle{\pgfqpoint{0.625000in}{0.550000in}}{\pgfqpoint{3.875000in}{3.850000in}} %
\pgfusepath{clip}%
\pgfsetbuttcap%
\pgfsetroundjoin%
\pgfsetlinewidth{0.250937pt}%
\definecolor{currentstroke}{rgb}{0.000000,0.000000,0.000000}%
\pgfsetstrokecolor{currentstroke}%
\pgfsetdash{}{0pt}%
\pgfpathmoveto{\pgfqpoint{0.625000in}{0.634599in}}%
\pgfpathlineto{\pgfqpoint{0.632638in}{0.627193in}}%
\pgfpathlineto{\pgfqpoint{0.625000in}{0.619809in}}%
\pgfusepath{stroke}%
\end{pgfscope}%
\begin{pgfscope}%
\pgfpathrectangle{\pgfqpoint{0.625000in}{0.550000in}}{\pgfqpoint{3.875000in}{3.850000in}} %
\pgfusepath{clip}%
\pgfsetbuttcap%
\pgfsetroundjoin%
\pgfsetlinewidth{0.250937pt}%
\definecolor{currentstroke}{rgb}{0.000000,0.000000,0.000000}%
\pgfsetstrokecolor{currentstroke}%
\pgfsetdash{}{0pt}%
\pgfpathmoveto{\pgfqpoint{0.625000in}{0.788576in}}%
\pgfpathlineto{\pgfqpoint{0.632570in}{0.781579in}}%
\pgfpathlineto{\pgfqpoint{0.625000in}{0.774270in}}%
\pgfusepath{stroke}%
\end{pgfscope}%
\begin{pgfscope}%
\pgfpathrectangle{\pgfqpoint{0.625000in}{0.550000in}}{\pgfqpoint{3.875000in}{3.850000in}} %
\pgfusepath{clip}%
\pgfsetbuttcap%
\pgfsetroundjoin%
\pgfsetlinewidth{0.250937pt}%
\definecolor{currentstroke}{rgb}{0.000000,0.000000,0.000000}%
\pgfsetstrokecolor{currentstroke}%
\pgfsetdash{}{0pt}%
\pgfpathmoveto{\pgfqpoint{0.625000in}{0.943253in}}%
\pgfpathlineto{\pgfqpoint{0.634226in}{0.935965in}}%
\pgfpathlineto{\pgfqpoint{0.625000in}{0.928611in}}%
\pgfusepath{stroke}%
\end{pgfscope}%
\begin{pgfscope}%
\pgfpathrectangle{\pgfqpoint{0.625000in}{0.550000in}}{\pgfqpoint{3.875000in}{3.850000in}} %
\pgfusepath{clip}%
\pgfsetbuttcap%
\pgfsetroundjoin%
\pgfsetlinewidth{0.250937pt}%
\definecolor{currentstroke}{rgb}{0.000000,0.000000,0.000000}%
\pgfsetstrokecolor{currentstroke}%
\pgfsetdash{}{0pt}%
\pgfpathmoveto{\pgfqpoint{0.625000in}{1.097618in}}%
\pgfpathlineto{\pgfqpoint{0.632578in}{1.090351in}}%
\pgfpathlineto{\pgfqpoint{0.625000in}{1.083111in}}%
\pgfusepath{stroke}%
\end{pgfscope}%
\begin{pgfscope}%
\pgfpathrectangle{\pgfqpoint{0.625000in}{0.550000in}}{\pgfqpoint{3.875000in}{3.850000in}} %
\pgfusepath{clip}%
\pgfsetbuttcap%
\pgfsetroundjoin%
\pgfsetlinewidth{0.250937pt}%
\definecolor{currentstroke}{rgb}{0.000000,0.000000,0.000000}%
\pgfsetstrokecolor{currentstroke}%
\pgfsetdash{}{0pt}%
\pgfpathmoveto{\pgfqpoint{0.625000in}{1.252220in}}%
\pgfpathlineto{\pgfqpoint{0.632582in}{1.244737in}}%
\pgfpathlineto{\pgfqpoint{0.625000in}{1.237460in}}%
\pgfusepath{stroke}%
\end{pgfscope}%
\begin{pgfscope}%
\pgfpathrectangle{\pgfqpoint{0.625000in}{0.550000in}}{\pgfqpoint{3.875000in}{3.850000in}} %
\pgfusepath{clip}%
\pgfsetbuttcap%
\pgfsetroundjoin%
\pgfsetlinewidth{0.250937pt}%
\definecolor{currentstroke}{rgb}{0.000000,0.000000,0.000000}%
\pgfsetstrokecolor{currentstroke}%
\pgfsetdash{}{0pt}%
\pgfpathmoveto{\pgfqpoint{0.625000in}{1.406275in}}%
\pgfpathlineto{\pgfqpoint{0.632519in}{1.399123in}}%
\pgfpathlineto{\pgfqpoint{0.625000in}{1.391966in}}%
\pgfusepath{stroke}%
\end{pgfscope}%
\begin{pgfscope}%
\pgfpathrectangle{\pgfqpoint{0.625000in}{0.550000in}}{\pgfqpoint{3.875000in}{3.850000in}} %
\pgfusepath{clip}%
\pgfsetbuttcap%
\pgfsetroundjoin%
\pgfsetlinewidth{0.250937pt}%
\definecolor{currentstroke}{rgb}{0.000000,0.000000,0.000000}%
\pgfsetstrokecolor{currentstroke}%
\pgfsetdash{}{0pt}%
\pgfpathmoveto{\pgfqpoint{0.625000in}{1.560653in}}%
\pgfpathlineto{\pgfqpoint{0.632498in}{1.553509in}}%
\pgfpathlineto{\pgfqpoint{0.625000in}{1.546133in}}%
\pgfusepath{stroke}%
\end{pgfscope}%
\begin{pgfscope}%
\pgfpathrectangle{\pgfqpoint{0.625000in}{0.550000in}}{\pgfqpoint{3.875000in}{3.850000in}} %
\pgfusepath{clip}%
\pgfsetbuttcap%
\pgfsetroundjoin%
\pgfsetlinewidth{0.250937pt}%
\definecolor{currentstroke}{rgb}{0.000000,0.000000,0.000000}%
\pgfsetstrokecolor{currentstroke}%
\pgfsetdash{}{0pt}%
\pgfpathmoveto{\pgfqpoint{0.625000in}{1.715026in}}%
\pgfpathlineto{\pgfqpoint{0.634189in}{1.707895in}}%
\pgfpathlineto{\pgfqpoint{0.625000in}{1.700718in}}%
\pgfusepath{stroke}%
\end{pgfscope}%
\begin{pgfscope}%
\pgfpathrectangle{\pgfqpoint{0.625000in}{0.550000in}}{\pgfqpoint{3.875000in}{3.850000in}} %
\pgfusepath{clip}%
\pgfsetbuttcap%
\pgfsetroundjoin%
\pgfsetlinewidth{0.250937pt}%
\definecolor{currentstroke}{rgb}{0.000000,0.000000,0.000000}%
\pgfsetstrokecolor{currentstroke}%
\pgfsetdash{}{0pt}%
\pgfpathmoveto{\pgfqpoint{0.625000in}{1.869457in}}%
\pgfpathlineto{\pgfqpoint{0.632448in}{1.862281in}}%
\pgfpathlineto{\pgfqpoint{0.625000in}{1.855253in}}%
\pgfusepath{stroke}%
\end{pgfscope}%
\begin{pgfscope}%
\pgfpathrectangle{\pgfqpoint{0.625000in}{0.550000in}}{\pgfqpoint{3.875000in}{3.850000in}} %
\pgfusepath{clip}%
\pgfsetbuttcap%
\pgfsetroundjoin%
\pgfsetlinewidth{0.250937pt}%
\definecolor{currentstroke}{rgb}{0.000000,0.000000,0.000000}%
\pgfsetstrokecolor{currentstroke}%
\pgfsetdash{}{0pt}%
\pgfpathmoveto{\pgfqpoint{0.625000in}{2.024080in}}%
\pgfpathlineto{\pgfqpoint{0.632421in}{2.016667in}}%
\pgfpathlineto{\pgfqpoint{0.625000in}{2.009314in}}%
\pgfusepath{stroke}%
\end{pgfscope}%
\begin{pgfscope}%
\pgfpathrectangle{\pgfqpoint{0.625000in}{0.550000in}}{\pgfqpoint{3.875000in}{3.850000in}} %
\pgfusepath{clip}%
\pgfsetbuttcap%
\pgfsetroundjoin%
\pgfsetlinewidth{0.250937pt}%
\definecolor{currentstroke}{rgb}{0.000000,0.000000,0.000000}%
\pgfsetstrokecolor{currentstroke}%
\pgfsetdash{}{0pt}%
\pgfpathmoveto{\pgfqpoint{0.625000in}{2.178098in}}%
\pgfpathlineto{\pgfqpoint{0.632451in}{2.171053in}}%
\pgfpathlineto{\pgfqpoint{0.625000in}{2.163806in}}%
\pgfusepath{stroke}%
\end{pgfscope}%
\begin{pgfscope}%
\pgfpathrectangle{\pgfqpoint{0.625000in}{0.550000in}}{\pgfqpoint{3.875000in}{3.850000in}} %
\pgfusepath{clip}%
\pgfsetbuttcap%
\pgfsetroundjoin%
\pgfsetlinewidth{0.250937pt}%
\definecolor{currentstroke}{rgb}{0.000000,0.000000,0.000000}%
\pgfsetstrokecolor{currentstroke}%
\pgfsetdash{}{0pt}%
\pgfpathmoveto{\pgfqpoint{0.625000in}{2.332438in}}%
\pgfpathlineto{\pgfqpoint{0.632473in}{2.325439in}}%
\pgfpathlineto{\pgfqpoint{0.625000in}{2.317949in}}%
\pgfusepath{stroke}%
\end{pgfscope}%
\begin{pgfscope}%
\pgfpathrectangle{\pgfqpoint{0.625000in}{0.550000in}}{\pgfqpoint{3.875000in}{3.850000in}} %
\pgfusepath{clip}%
\pgfsetbuttcap%
\pgfsetroundjoin%
\pgfsetlinewidth{0.250937pt}%
\definecolor{currentstroke}{rgb}{0.000000,0.000000,0.000000}%
\pgfsetstrokecolor{currentstroke}%
\pgfsetdash{}{0pt}%
\pgfpathmoveto{\pgfqpoint{0.625000in}{2.487796in}}%
\pgfpathlineto{\pgfqpoint{0.626574in}{2.489474in}}%
\pgfpathlineto{\pgfqpoint{0.627306in}{2.499123in}}%
\pgfpathlineto{\pgfqpoint{0.629805in}{2.508772in}}%
\pgfpathlineto{\pgfqpoint{0.633027in}{2.518421in}}%
\pgfpathlineto{\pgfqpoint{0.634712in}{2.523432in}}%
\pgfpathlineto{\pgfqpoint{0.637707in}{2.528070in}}%
\pgfpathlineto{\pgfqpoint{0.643262in}{2.537719in}}%
\pgfpathlineto{\pgfqpoint{0.644424in}{2.539975in}}%
\pgfpathlineto{\pgfqpoint{0.650646in}{2.547368in}}%
\pgfpathlineto{\pgfqpoint{0.654135in}{2.552143in}}%
\pgfpathlineto{\pgfqpoint{0.659699in}{2.557018in}}%
\pgfpathlineto{\pgfqpoint{0.663847in}{2.561428in}}%
\pgfpathlineto{\pgfqpoint{0.671187in}{2.566667in}}%
\pgfpathlineto{\pgfqpoint{0.673559in}{2.568738in}}%
\pgfpathlineto{\pgfqpoint{0.683271in}{2.574682in}}%
\pgfpathlineto{\pgfqpoint{0.687057in}{2.576316in}}%
\pgfpathlineto{\pgfqpoint{0.692982in}{2.579481in}}%
\pgfpathlineto{\pgfqpoint{0.702694in}{2.583154in}}%
\pgfpathlineto{\pgfqpoint{0.712406in}{2.585760in}}%
\pgfpathlineto{\pgfqpoint{0.713635in}{2.585965in}}%
\pgfpathlineto{\pgfqpoint{0.722118in}{2.587683in}}%
\pgfpathlineto{\pgfqpoint{0.731830in}{2.588711in}}%
\pgfpathlineto{\pgfqpoint{0.741541in}{2.588886in}}%
\pgfpathlineto{\pgfqpoint{0.751253in}{2.588224in}}%
\pgfpathlineto{\pgfqpoint{0.760965in}{2.586706in}}%
\pgfpathlineto{\pgfqpoint{0.764156in}{2.585965in}}%
\pgfpathlineto{\pgfqpoint{0.770677in}{2.584353in}}%
\pgfpathlineto{\pgfqpoint{0.780388in}{2.581057in}}%
\pgfpathlineto{\pgfqpoint{0.790100in}{2.576694in}}%
\pgfpathlineto{\pgfqpoint{0.790837in}{2.576316in}}%
\pgfpathlineto{\pgfqpoint{0.799812in}{2.571148in}}%
\pgfpathlineto{\pgfqpoint{0.806374in}{2.566667in}}%
\pgfpathlineto{\pgfqpoint{0.809524in}{2.564176in}}%
\pgfpathlineto{\pgfqpoint{0.817686in}{2.557018in}}%
\pgfpathlineto{\pgfqpoint{0.819236in}{2.555395in}}%
\pgfpathlineto{\pgfqpoint{0.826481in}{2.547368in}}%
\pgfpathlineto{\pgfqpoint{0.828947in}{2.543999in}}%
\pgfpathlineto{\pgfqpoint{0.833448in}{2.537719in}}%
\pgfpathlineto{\pgfqpoint{0.838659in}{2.528426in}}%
\pgfpathlineto{\pgfqpoint{0.838860in}{2.528070in}}%
\pgfpathlineto{\pgfqpoint{0.843232in}{2.518421in}}%
\pgfpathlineto{\pgfqpoint{0.846409in}{2.508772in}}%
\pgfpathlineto{\pgfqpoint{0.848371in}{2.500017in}}%
\pgfpathlineto{\pgfqpoint{0.848583in}{2.499123in}}%
\pgfpathlineto{\pgfqpoint{0.849950in}{2.489474in}}%
\pgfpathlineto{\pgfqpoint{0.850399in}{2.479825in}}%
\pgfpathlineto{\pgfqpoint{0.849950in}{2.470175in}}%
\pgfpathlineto{\pgfqpoint{0.848583in}{2.460526in}}%
\pgfpathlineto{\pgfqpoint{0.848371in}{2.459632in}}%
\pgfpathlineto{\pgfqpoint{0.846409in}{2.450877in}}%
\pgfpathlineto{\pgfqpoint{0.843232in}{2.441228in}}%
\pgfpathlineto{\pgfqpoint{0.838860in}{2.431579in}}%
\pgfpathlineto{\pgfqpoint{0.838659in}{2.431223in}}%
\pgfpathlineto{\pgfqpoint{0.833448in}{2.421930in}}%
\pgfpathlineto{\pgfqpoint{0.828947in}{2.415650in}}%
\pgfpathlineto{\pgfqpoint{0.826481in}{2.412281in}}%
\pgfpathlineto{\pgfqpoint{0.819236in}{2.404254in}}%
\pgfpathlineto{\pgfqpoint{0.817686in}{2.402632in}}%
\pgfpathlineto{\pgfqpoint{0.809524in}{2.395473in}}%
\pgfpathlineto{\pgfqpoint{0.806374in}{2.392982in}}%
\pgfpathlineto{\pgfqpoint{0.799812in}{2.388501in}}%
\pgfpathlineto{\pgfqpoint{0.790837in}{2.383333in}}%
\pgfpathlineto{\pgfqpoint{0.790100in}{2.382956in}}%
\pgfpathlineto{\pgfqpoint{0.780388in}{2.378592in}}%
\pgfpathlineto{\pgfqpoint{0.770677in}{2.375296in}}%
\pgfpathlineto{\pgfqpoint{0.764156in}{2.373684in}}%
\pgfpathlineto{\pgfqpoint{0.760965in}{2.372943in}}%
\pgfpathlineto{\pgfqpoint{0.751253in}{2.371425in}}%
\pgfpathlineto{\pgfqpoint{0.741541in}{2.370763in}}%
\pgfpathlineto{\pgfqpoint{0.731830in}{2.370938in}}%
\pgfpathlineto{\pgfqpoint{0.722118in}{2.371966in}}%
\pgfpathlineto{\pgfqpoint{0.713635in}{2.373684in}}%
\pgfpathlineto{\pgfqpoint{0.712406in}{2.373889in}}%
\pgfpathlineto{\pgfqpoint{0.702694in}{2.376495in}}%
\pgfpathlineto{\pgfqpoint{0.692982in}{2.380168in}}%
\pgfpathlineto{\pgfqpoint{0.687057in}{2.383333in}}%
\pgfpathlineto{\pgfqpoint{0.683271in}{2.384967in}}%
\pgfpathlineto{\pgfqpoint{0.673559in}{2.390911in}}%
\pgfpathlineto{\pgfqpoint{0.671187in}{2.392982in}}%
\pgfpathlineto{\pgfqpoint{0.663847in}{2.398221in}}%
\pgfpathlineto{\pgfqpoint{0.659699in}{2.402632in}}%
\pgfpathlineto{\pgfqpoint{0.654135in}{2.407506in}}%
\pgfpathlineto{\pgfqpoint{0.650646in}{2.412281in}}%
\pgfpathlineto{\pgfqpoint{0.644424in}{2.419674in}}%
\pgfpathlineto{\pgfqpoint{0.643262in}{2.421930in}}%
\pgfpathlineto{\pgfqpoint{0.637707in}{2.431579in}}%
\pgfpathlineto{\pgfqpoint{0.634712in}{2.436217in}}%
\pgfpathlineto{\pgfqpoint{0.633027in}{2.441228in}}%
\pgfpathlineto{\pgfqpoint{0.629805in}{2.450877in}}%
\pgfpathlineto{\pgfqpoint{0.627311in}{2.460526in}}%
\pgfpathlineto{\pgfqpoint{0.626574in}{2.470175in}}%
\pgfpathlineto{\pgfqpoint{0.625000in}{2.471853in}}%
\pgfusepath{stroke}%
\end{pgfscope}%
\begin{pgfscope}%
\pgfpathrectangle{\pgfqpoint{0.625000in}{0.550000in}}{\pgfqpoint{3.875000in}{3.850000in}} %
\pgfusepath{clip}%
\pgfsetbuttcap%
\pgfsetroundjoin%
\pgfsetlinewidth{0.250937pt}%
\definecolor{currentstroke}{rgb}{0.000000,0.000000,0.000000}%
\pgfsetstrokecolor{currentstroke}%
\pgfsetdash{}{0pt}%
\pgfpathmoveto{\pgfqpoint{0.625000in}{2.641696in}}%
\pgfpathlineto{\pgfqpoint{0.632468in}{2.634211in}}%
\pgfpathlineto{\pgfqpoint{0.625000in}{2.627216in}}%
\pgfusepath{stroke}%
\end{pgfscope}%
\begin{pgfscope}%
\pgfpathrectangle{\pgfqpoint{0.625000in}{0.550000in}}{\pgfqpoint{3.875000in}{3.850000in}} %
\pgfusepath{clip}%
\pgfsetbuttcap%
\pgfsetroundjoin%
\pgfsetlinewidth{0.250937pt}%
\definecolor{currentstroke}{rgb}{0.000000,0.000000,0.000000}%
\pgfsetstrokecolor{currentstroke}%
\pgfsetdash{}{0pt}%
\pgfpathmoveto{\pgfqpoint{0.625000in}{2.795829in}}%
\pgfpathlineto{\pgfqpoint{0.632437in}{2.788596in}}%
\pgfpathlineto{\pgfqpoint{0.625000in}{2.781567in}}%
\pgfusepath{stroke}%
\end{pgfscope}%
\begin{pgfscope}%
\pgfpathrectangle{\pgfqpoint{0.625000in}{0.550000in}}{\pgfqpoint{3.875000in}{3.850000in}} %
\pgfusepath{clip}%
\pgfsetbuttcap%
\pgfsetroundjoin%
\pgfsetlinewidth{0.250937pt}%
\definecolor{currentstroke}{rgb}{0.000000,0.000000,0.000000}%
\pgfsetstrokecolor{currentstroke}%
\pgfsetdash{}{0pt}%
\pgfpathmoveto{\pgfqpoint{0.625000in}{2.950296in}}%
\pgfpathlineto{\pgfqpoint{0.632381in}{2.942982in}}%
\pgfpathlineto{\pgfqpoint{0.625000in}{2.935608in}}%
\pgfusepath{stroke}%
\end{pgfscope}%
\begin{pgfscope}%
\pgfpathrectangle{\pgfqpoint{0.625000in}{0.550000in}}{\pgfqpoint{3.875000in}{3.850000in}} %
\pgfusepath{clip}%
\pgfsetbuttcap%
\pgfsetroundjoin%
\pgfsetlinewidth{0.250937pt}%
\definecolor{currentstroke}{rgb}{0.000000,0.000000,0.000000}%
\pgfsetstrokecolor{currentstroke}%
\pgfsetdash{}{0pt}%
\pgfpathmoveto{\pgfqpoint{0.625000in}{3.104362in}}%
\pgfpathlineto{\pgfqpoint{0.632417in}{3.097368in}}%
\pgfpathlineto{\pgfqpoint{0.625000in}{3.090225in}}%
\pgfusepath{stroke}%
\end{pgfscope}%
\begin{pgfscope}%
\pgfpathrectangle{\pgfqpoint{0.625000in}{0.550000in}}{\pgfqpoint{3.875000in}{3.850000in}} %
\pgfusepath{clip}%
\pgfsetbuttcap%
\pgfsetroundjoin%
\pgfsetlinewidth{0.250937pt}%
\definecolor{currentstroke}{rgb}{0.000000,0.000000,0.000000}%
\pgfsetstrokecolor{currentstroke}%
\pgfsetdash{}{0pt}%
\pgfpathmoveto{\pgfqpoint{0.625000in}{3.258812in}}%
\pgfpathlineto{\pgfqpoint{0.634156in}{3.251754in}}%
\pgfpathlineto{\pgfqpoint{0.625000in}{3.244743in}}%
\pgfusepath{stroke}%
\end{pgfscope}%
\begin{pgfscope}%
\pgfpathrectangle{\pgfqpoint{0.625000in}{0.550000in}}{\pgfqpoint{3.875000in}{3.850000in}} %
\pgfusepath{clip}%
\pgfsetbuttcap%
\pgfsetroundjoin%
\pgfsetlinewidth{0.250937pt}%
\definecolor{currentstroke}{rgb}{0.000000,0.000000,0.000000}%
\pgfsetstrokecolor{currentstroke}%
\pgfsetdash{}{0pt}%
\pgfpathmoveto{\pgfqpoint{0.625000in}{3.413438in}}%
\pgfpathlineto{\pgfqpoint{0.632421in}{3.406140in}}%
\pgfpathlineto{\pgfqpoint{0.625000in}{3.399080in}}%
\pgfusepath{stroke}%
\end{pgfscope}%
\begin{pgfscope}%
\pgfpathrectangle{\pgfqpoint{0.625000in}{0.550000in}}{\pgfqpoint{3.875000in}{3.850000in}} %
\pgfusepath{clip}%
\pgfsetbuttcap%
\pgfsetroundjoin%
\pgfsetlinewidth{0.250937pt}%
\definecolor{currentstroke}{rgb}{0.000000,0.000000,0.000000}%
\pgfsetstrokecolor{currentstroke}%
\pgfsetdash{}{0pt}%
\pgfpathmoveto{\pgfqpoint{0.625000in}{3.567584in}}%
\pgfpathlineto{\pgfqpoint{0.632428in}{3.560526in}}%
\pgfpathlineto{\pgfqpoint{0.625000in}{3.553473in}}%
\pgfusepath{stroke}%
\end{pgfscope}%
\begin{pgfscope}%
\pgfpathrectangle{\pgfqpoint{0.625000in}{0.550000in}}{\pgfqpoint{3.875000in}{3.850000in}} %
\pgfusepath{clip}%
\pgfsetbuttcap%
\pgfsetroundjoin%
\pgfsetlinewidth{0.250937pt}%
\definecolor{currentstroke}{rgb}{0.000000,0.000000,0.000000}%
\pgfsetstrokecolor{currentstroke}%
\pgfsetdash{}{0pt}%
\pgfpathmoveto{\pgfqpoint{0.625000in}{3.722044in}}%
\pgfpathlineto{\pgfqpoint{0.632447in}{3.714912in}}%
\pgfpathlineto{\pgfqpoint{0.625000in}{3.707565in}}%
\pgfusepath{stroke}%
\end{pgfscope}%
\begin{pgfscope}%
\pgfpathrectangle{\pgfqpoint{0.625000in}{0.550000in}}{\pgfqpoint{3.875000in}{3.850000in}} %
\pgfusepath{clip}%
\pgfsetbuttcap%
\pgfsetroundjoin%
\pgfsetlinewidth{0.250937pt}%
\definecolor{currentstroke}{rgb}{0.000000,0.000000,0.000000}%
\pgfsetstrokecolor{currentstroke}%
\pgfsetdash{}{0pt}%
\pgfpathmoveto{\pgfqpoint{0.625000in}{3.876296in}}%
\pgfpathlineto{\pgfqpoint{0.632354in}{3.869298in}}%
\pgfpathlineto{\pgfqpoint{0.625000in}{3.862271in}}%
\pgfusepath{stroke}%
\end{pgfscope}%
\begin{pgfscope}%
\pgfpathrectangle{\pgfqpoint{0.625000in}{0.550000in}}{\pgfqpoint{3.875000in}{3.850000in}} %
\pgfusepath{clip}%
\pgfsetbuttcap%
\pgfsetroundjoin%
\pgfsetlinewidth{0.250937pt}%
\definecolor{currentstroke}{rgb}{0.000000,0.000000,0.000000}%
\pgfsetstrokecolor{currentstroke}%
\pgfsetdash{}{0pt}%
\pgfpathmoveto{\pgfqpoint{0.625000in}{4.030823in}}%
\pgfpathlineto{\pgfqpoint{0.634168in}{4.023684in}}%
\pgfpathlineto{\pgfqpoint{0.625000in}{4.016615in}}%
\pgfusepath{stroke}%
\end{pgfscope}%
\begin{pgfscope}%
\pgfpathrectangle{\pgfqpoint{0.625000in}{0.550000in}}{\pgfqpoint{3.875000in}{3.850000in}} %
\pgfusepath{clip}%
\pgfsetbuttcap%
\pgfsetroundjoin%
\pgfsetlinewidth{0.250937pt}%
\definecolor{currentstroke}{rgb}{0.000000,0.000000,0.000000}%
\pgfsetstrokecolor{currentstroke}%
\pgfsetdash{}{0pt}%
\pgfpathmoveto{\pgfqpoint{0.625000in}{4.185138in}}%
\pgfpathlineto{\pgfqpoint{0.632342in}{4.178070in}}%
\pgfpathlineto{\pgfqpoint{0.625000in}{4.171333in}}%
\pgfusepath{stroke}%
\end{pgfscope}%
\begin{pgfscope}%
\pgfpathrectangle{\pgfqpoint{0.625000in}{0.550000in}}{\pgfqpoint{3.875000in}{3.850000in}} %
\pgfusepath{clip}%
\pgfsetbuttcap%
\pgfsetroundjoin%
\pgfsetlinewidth{0.250937pt}%
\definecolor{currentstroke}{rgb}{0.000000,0.000000,0.000000}%
\pgfsetstrokecolor{currentstroke}%
\pgfsetdash{}{0pt}%
\pgfpathmoveto{\pgfqpoint{0.625000in}{4.339497in}}%
\pgfpathlineto{\pgfqpoint{0.632313in}{4.332456in}}%
\pgfpathlineto{\pgfqpoint{0.625000in}{4.325392in}}%
\pgfusepath{stroke}%
\end{pgfscope}%
\begin{pgfscope}%
\pgfpathrectangle{\pgfqpoint{0.625000in}{0.550000in}}{\pgfqpoint{3.875000in}{3.850000in}} %
\pgfusepath{clip}%
\pgfsetbuttcap%
\pgfsetroundjoin%
\pgfsetlinewidth{0.250937pt}%
\definecolor{currentstroke}{rgb}{0.000000,0.000000,0.000000}%
\pgfsetstrokecolor{currentstroke}%
\pgfsetdash{}{0pt}%
\pgfpathmoveto{\pgfqpoint{0.634712in}{1.180988in}}%
\pgfpathlineto{\pgfqpoint{0.631070in}{1.186842in}}%
\pgfpathlineto{\pgfqpoint{0.629166in}{1.196491in}}%
\pgfpathlineto{\pgfqpoint{0.634712in}{1.204824in}}%
\pgfpathlineto{\pgfqpoint{0.644424in}{1.200782in}}%
\pgfpathlineto{\pgfqpoint{0.648435in}{1.196491in}}%
\pgfpathlineto{\pgfqpoint{0.646410in}{1.186842in}}%
\pgfpathlineto{\pgfqpoint{0.644424in}{1.185231in}}%
\pgfpathlineto{\pgfqpoint{0.634712in}{1.180988in}}%
\pgfusepath{stroke}%
\end{pgfscope}%
\begin{pgfscope}%
\pgfpathrectangle{\pgfqpoint{0.625000in}{0.550000in}}{\pgfqpoint{3.875000in}{3.850000in}} %
\pgfusepath{clip}%
\pgfsetbuttcap%
\pgfsetroundjoin%
\pgfsetlinewidth{0.250937pt}%
\definecolor{currentstroke}{rgb}{0.000000,0.000000,0.000000}%
\pgfsetstrokecolor{currentstroke}%
\pgfsetdash{}{0pt}%
\pgfpathmoveto{\pgfqpoint{0.634712in}{3.754825in}}%
\pgfpathlineto{\pgfqpoint{0.629166in}{3.763158in}}%
\pgfpathlineto{\pgfqpoint{0.631070in}{3.772807in}}%
\pgfpathlineto{\pgfqpoint{0.634712in}{3.778661in}}%
\pgfpathlineto{\pgfqpoint{0.644424in}{3.774418in}}%
\pgfpathlineto{\pgfqpoint{0.646410in}{3.772807in}}%
\pgfpathlineto{\pgfqpoint{0.648435in}{3.763158in}}%
\pgfpathlineto{\pgfqpoint{0.644424in}{3.758867in}}%
\pgfpathlineto{\pgfqpoint{0.634712in}{3.754825in}}%
\pgfusepath{stroke}%
\end{pgfscope}%
\begin{pgfscope}%
\pgfpathrectangle{\pgfqpoint{0.625000in}{0.550000in}}{\pgfqpoint{3.875000in}{3.850000in}} %
\pgfusepath{clip}%
\pgfsetbuttcap%
\pgfsetroundjoin%
\pgfsetlinewidth{0.250937pt}%
\definecolor{currentstroke}{rgb}{0.000000,0.000000,0.000000}%
\pgfsetstrokecolor{currentstroke}%
\pgfsetdash{}{0pt}%
\pgfpathmoveto{\pgfqpoint{0.625000in}{0.634675in}}%
\pgfpathlineto{\pgfqpoint{0.632717in}{0.627193in}}%
\pgfpathlineto{\pgfqpoint{0.625000in}{0.619733in}}%
\pgfusepath{stroke}%
\end{pgfscope}%
\begin{pgfscope}%
\pgfpathrectangle{\pgfqpoint{0.625000in}{0.550000in}}{\pgfqpoint{3.875000in}{3.850000in}} %
\pgfusepath{clip}%
\pgfsetbuttcap%
\pgfsetroundjoin%
\pgfsetlinewidth{0.250937pt}%
\definecolor{currentstroke}{rgb}{0.000000,0.000000,0.000000}%
\pgfsetstrokecolor{currentstroke}%
\pgfsetdash{}{0pt}%
\pgfpathmoveto{\pgfqpoint{0.625000in}{0.788653in}}%
\pgfpathlineto{\pgfqpoint{0.632653in}{0.781579in}}%
\pgfpathlineto{\pgfqpoint{0.625000in}{0.774190in}}%
\pgfusepath{stroke}%
\end{pgfscope}%
\begin{pgfscope}%
\pgfpathrectangle{\pgfqpoint{0.625000in}{0.550000in}}{\pgfqpoint{3.875000in}{3.850000in}} %
\pgfusepath{clip}%
\pgfsetbuttcap%
\pgfsetroundjoin%
\pgfsetlinewidth{0.250937pt}%
\definecolor{currentstroke}{rgb}{0.000000,0.000000,0.000000}%
\pgfsetstrokecolor{currentstroke}%
\pgfsetdash{}{0pt}%
\pgfpathmoveto{\pgfqpoint{0.625000in}{0.943329in}}%
\pgfpathlineto{\pgfqpoint{0.634323in}{0.935965in}}%
\pgfpathlineto{\pgfqpoint{0.625000in}{0.928534in}}%
\pgfusepath{stroke}%
\end{pgfscope}%
\begin{pgfscope}%
\pgfpathrectangle{\pgfqpoint{0.625000in}{0.550000in}}{\pgfqpoint{3.875000in}{3.850000in}} %
\pgfusepath{clip}%
\pgfsetbuttcap%
\pgfsetroundjoin%
\pgfsetlinewidth{0.250937pt}%
\definecolor{currentstroke}{rgb}{0.000000,0.000000,0.000000}%
\pgfsetstrokecolor{currentstroke}%
\pgfsetdash{}{0pt}%
\pgfpathmoveto{\pgfqpoint{0.625000in}{1.097697in}}%
\pgfpathlineto{\pgfqpoint{0.632661in}{1.090351in}}%
\pgfpathlineto{\pgfqpoint{0.625000in}{1.083032in}}%
\pgfusepath{stroke}%
\end{pgfscope}%
\begin{pgfscope}%
\pgfpathrectangle{\pgfqpoint{0.625000in}{0.550000in}}{\pgfqpoint{3.875000in}{3.850000in}} %
\pgfusepath{clip}%
\pgfsetbuttcap%
\pgfsetroundjoin%
\pgfsetlinewidth{0.250937pt}%
\definecolor{currentstroke}{rgb}{0.000000,0.000000,0.000000}%
\pgfsetstrokecolor{currentstroke}%
\pgfsetdash{}{0pt}%
\pgfpathmoveto{\pgfqpoint{0.625000in}{1.252302in}}%
\pgfpathlineto{\pgfqpoint{0.632666in}{1.244737in}}%
\pgfpathlineto{\pgfqpoint{0.625000in}{1.237380in}}%
\pgfusepath{stroke}%
\end{pgfscope}%
\begin{pgfscope}%
\pgfpathrectangle{\pgfqpoint{0.625000in}{0.550000in}}{\pgfqpoint{3.875000in}{3.850000in}} %
\pgfusepath{clip}%
\pgfsetbuttcap%
\pgfsetroundjoin%
\pgfsetlinewidth{0.250937pt}%
\definecolor{currentstroke}{rgb}{0.000000,0.000000,0.000000}%
\pgfsetstrokecolor{currentstroke}%
\pgfsetdash{}{0pt}%
\pgfpathmoveto{\pgfqpoint{0.625000in}{1.406357in}}%
\pgfpathlineto{\pgfqpoint{0.632604in}{1.399123in}}%
\pgfpathlineto{\pgfqpoint{0.625000in}{1.391885in}}%
\pgfusepath{stroke}%
\end{pgfscope}%
\begin{pgfscope}%
\pgfpathrectangle{\pgfqpoint{0.625000in}{0.550000in}}{\pgfqpoint{3.875000in}{3.850000in}} %
\pgfusepath{clip}%
\pgfsetbuttcap%
\pgfsetroundjoin%
\pgfsetlinewidth{0.250937pt}%
\definecolor{currentstroke}{rgb}{0.000000,0.000000,0.000000}%
\pgfsetstrokecolor{currentstroke}%
\pgfsetdash{}{0pt}%
\pgfpathmoveto{\pgfqpoint{0.625000in}{1.560733in}}%
\pgfpathlineto{\pgfqpoint{0.632582in}{1.553509in}}%
\pgfpathlineto{\pgfqpoint{0.625000in}{1.546050in}}%
\pgfusepath{stroke}%
\end{pgfscope}%
\begin{pgfscope}%
\pgfpathrectangle{\pgfqpoint{0.625000in}{0.550000in}}{\pgfqpoint{3.875000in}{3.850000in}} %
\pgfusepath{clip}%
\pgfsetbuttcap%
\pgfsetroundjoin%
\pgfsetlinewidth{0.250937pt}%
\definecolor{currentstroke}{rgb}{0.000000,0.000000,0.000000}%
\pgfsetstrokecolor{currentstroke}%
\pgfsetdash{}{0pt}%
\pgfpathmoveto{\pgfqpoint{0.625000in}{1.715107in}}%
\pgfpathlineto{\pgfqpoint{0.634293in}{1.707895in}}%
\pgfpathlineto{\pgfqpoint{0.625000in}{1.700637in}}%
\pgfusepath{stroke}%
\end{pgfscope}%
\begin{pgfscope}%
\pgfpathrectangle{\pgfqpoint{0.625000in}{0.550000in}}{\pgfqpoint{3.875000in}{3.850000in}} %
\pgfusepath{clip}%
\pgfsetbuttcap%
\pgfsetroundjoin%
\pgfsetlinewidth{0.250937pt}%
\definecolor{currentstroke}{rgb}{0.000000,0.000000,0.000000}%
\pgfsetstrokecolor{currentstroke}%
\pgfsetdash{}{0pt}%
\pgfpathmoveto{\pgfqpoint{0.625000in}{1.869540in}}%
\pgfpathlineto{\pgfqpoint{0.632534in}{1.862281in}}%
\pgfpathlineto{\pgfqpoint{0.625000in}{1.855171in}}%
\pgfusepath{stroke}%
\end{pgfscope}%
\begin{pgfscope}%
\pgfpathrectangle{\pgfqpoint{0.625000in}{0.550000in}}{\pgfqpoint{3.875000in}{3.850000in}} %
\pgfusepath{clip}%
\pgfsetbuttcap%
\pgfsetroundjoin%
\pgfsetlinewidth{0.250937pt}%
\definecolor{currentstroke}{rgb}{0.000000,0.000000,0.000000}%
\pgfsetstrokecolor{currentstroke}%
\pgfsetdash{}{0pt}%
\pgfpathmoveto{\pgfqpoint{0.625000in}{2.024168in}}%
\pgfpathlineto{\pgfqpoint{0.632508in}{2.016667in}}%
\pgfpathlineto{\pgfqpoint{0.625000in}{2.009227in}}%
\pgfusepath{stroke}%
\end{pgfscope}%
\begin{pgfscope}%
\pgfpathrectangle{\pgfqpoint{0.625000in}{0.550000in}}{\pgfqpoint{3.875000in}{3.850000in}} %
\pgfusepath{clip}%
\pgfsetbuttcap%
\pgfsetroundjoin%
\pgfsetlinewidth{0.250937pt}%
\definecolor{currentstroke}{rgb}{0.000000,0.000000,0.000000}%
\pgfsetstrokecolor{currentstroke}%
\pgfsetdash{}{0pt}%
\pgfpathmoveto{\pgfqpoint{0.625000in}{2.178180in}}%
\pgfpathlineto{\pgfqpoint{0.632538in}{2.171053in}}%
\pgfpathlineto{\pgfqpoint{0.625000in}{2.163721in}}%
\pgfusepath{stroke}%
\end{pgfscope}%
\begin{pgfscope}%
\pgfpathrectangle{\pgfqpoint{0.625000in}{0.550000in}}{\pgfqpoint{3.875000in}{3.850000in}} %
\pgfusepath{clip}%
\pgfsetbuttcap%
\pgfsetroundjoin%
\pgfsetlinewidth{0.250937pt}%
\definecolor{currentstroke}{rgb}{0.000000,0.000000,0.000000}%
\pgfsetstrokecolor{currentstroke}%
\pgfsetdash{}{0pt}%
\pgfpathmoveto{\pgfqpoint{0.625000in}{2.332521in}}%
\pgfpathlineto{\pgfqpoint{0.632561in}{2.325439in}}%
\pgfpathlineto{\pgfqpoint{0.625000in}{2.317860in}}%
\pgfusepath{stroke}%
\end{pgfscope}%
\begin{pgfscope}%
\pgfpathrectangle{\pgfqpoint{0.625000in}{0.550000in}}{\pgfqpoint{3.875000in}{3.850000in}} %
\pgfusepath{clip}%
\pgfsetbuttcap%
\pgfsetroundjoin%
\pgfsetlinewidth{0.250937pt}%
\definecolor{currentstroke}{rgb}{0.000000,0.000000,0.000000}%
\pgfsetstrokecolor{currentstroke}%
\pgfsetdash{}{0pt}%
\pgfpathmoveto{\pgfqpoint{0.625000in}{2.487880in}}%
\pgfpathlineto{\pgfqpoint{0.626495in}{2.489474in}}%
\pgfpathlineto{\pgfqpoint{0.627227in}{2.499123in}}%
\pgfpathlineto{\pgfqpoint{0.629650in}{2.508772in}}%
\pgfpathlineto{\pgfqpoint{0.632764in}{2.518421in}}%
\pgfpathlineto{\pgfqpoint{0.634712in}{2.524213in}}%
\pgfpathlineto{\pgfqpoint{0.637203in}{2.528070in}}%
\pgfpathlineto{\pgfqpoint{0.642593in}{2.537719in}}%
\pgfpathlineto{\pgfqpoint{0.644424in}{2.541275in}}%
\pgfpathlineto{\pgfqpoint{0.649552in}{2.547368in}}%
\pgfpathlineto{\pgfqpoint{0.654135in}{2.553639in}}%
\pgfpathlineto{\pgfqpoint{0.657991in}{2.557018in}}%
\pgfpathlineto{\pgfqpoint{0.663847in}{2.563244in}}%
\pgfpathlineto{\pgfqpoint{0.668643in}{2.566667in}}%
\pgfpathlineto{\pgfqpoint{0.673559in}{2.570962in}}%
\pgfpathlineto{\pgfqpoint{0.682313in}{2.576316in}}%
\pgfpathlineto{\pgfqpoint{0.683271in}{2.577029in}}%
\pgfpathlineto{\pgfqpoint{0.692982in}{2.582227in}}%
\pgfpathlineto{\pgfqpoint{0.702542in}{2.585965in}}%
\pgfpathlineto{\pgfqpoint{0.702694in}{2.586037in}}%
\pgfpathlineto{\pgfqpoint{0.712406in}{2.589236in}}%
\pgfpathlineto{\pgfqpoint{0.722118in}{2.591422in}}%
\pgfpathlineto{\pgfqpoint{0.731830in}{2.592752in}}%
\pgfpathlineto{\pgfqpoint{0.741541in}{2.593294in}}%
\pgfpathlineto{\pgfqpoint{0.751253in}{2.593064in}}%
\pgfpathlineto{\pgfqpoint{0.760965in}{2.592045in}}%
\pgfpathlineto{\pgfqpoint{0.770677in}{2.590197in}}%
\pgfpathlineto{\pgfqpoint{0.780388in}{2.587464in}}%
\pgfpathlineto{\pgfqpoint{0.784587in}{2.585965in}}%
\pgfpathlineto{\pgfqpoint{0.790100in}{2.583821in}}%
\pgfpathlineto{\pgfqpoint{0.799812in}{2.579141in}}%
\pgfpathlineto{\pgfqpoint{0.804770in}{2.576316in}}%
\pgfpathlineto{\pgfqpoint{0.809524in}{2.573254in}}%
\pgfpathlineto{\pgfqpoint{0.818411in}{2.566667in}}%
\pgfpathlineto{\pgfqpoint{0.819236in}{2.565956in}}%
\pgfpathlineto{\pgfqpoint{0.828762in}{2.557018in}}%
\pgfpathlineto{\pgfqpoint{0.828947in}{2.556810in}}%
\pgfpathlineto{\pgfqpoint{0.836990in}{2.547368in}}%
\pgfpathlineto{\pgfqpoint{0.838659in}{2.544949in}}%
\pgfpathlineto{\pgfqpoint{0.843581in}{2.537719in}}%
\pgfpathlineto{\pgfqpoint{0.848371in}{2.528730in}}%
\pgfpathlineto{\pgfqpoint{0.848726in}{2.528070in}}%
\pgfpathlineto{\pgfqpoint{0.852895in}{2.518421in}}%
\pgfpathlineto{\pgfqpoint{0.855942in}{2.508772in}}%
\pgfpathlineto{\pgfqpoint{0.858018in}{2.499123in}}%
\pgfpathlineto{\pgfqpoint{0.858083in}{2.498617in}}%
\pgfpathlineto{\pgfqpoint{0.859328in}{2.489474in}}%
\pgfpathlineto{\pgfqpoint{0.859761in}{2.479825in}}%
\pgfpathlineto{\pgfqpoint{0.859328in}{2.470175in}}%
\pgfpathlineto{\pgfqpoint{0.858083in}{2.461032in}}%
\pgfpathlineto{\pgfqpoint{0.858018in}{2.460526in}}%
\pgfpathlineto{\pgfqpoint{0.855942in}{2.450877in}}%
\pgfpathlineto{\pgfqpoint{0.852895in}{2.441228in}}%
\pgfpathlineto{\pgfqpoint{0.848726in}{2.431579in}}%
\pgfpathlineto{\pgfqpoint{0.848371in}{2.430919in}}%
\pgfpathlineto{\pgfqpoint{0.843581in}{2.421930in}}%
\pgfpathlineto{\pgfqpoint{0.838659in}{2.414700in}}%
\pgfpathlineto{\pgfqpoint{0.836990in}{2.412281in}}%
\pgfpathlineto{\pgfqpoint{0.828947in}{2.402839in}}%
\pgfpathlineto{\pgfqpoint{0.828762in}{2.402632in}}%
\pgfpathlineto{\pgfqpoint{0.819236in}{2.393693in}}%
\pgfpathlineto{\pgfqpoint{0.818411in}{2.392982in}}%
\pgfpathlineto{\pgfqpoint{0.809524in}{2.386395in}}%
\pgfpathlineto{\pgfqpoint{0.804770in}{2.383333in}}%
\pgfpathlineto{\pgfqpoint{0.799812in}{2.380508in}}%
\pgfpathlineto{\pgfqpoint{0.790100in}{2.375828in}}%
\pgfpathlineto{\pgfqpoint{0.784587in}{2.373684in}}%
\pgfpathlineto{\pgfqpoint{0.780388in}{2.372185in}}%
\pgfpathlineto{\pgfqpoint{0.770677in}{2.369452in}}%
\pgfpathlineto{\pgfqpoint{0.760965in}{2.367604in}}%
\pgfpathlineto{\pgfqpoint{0.751253in}{2.366585in}}%
\pgfpathlineto{\pgfqpoint{0.741541in}{2.366355in}}%
\pgfpathlineto{\pgfqpoint{0.731830in}{2.366897in}}%
\pgfpathlineto{\pgfqpoint{0.722118in}{2.368227in}}%
\pgfpathlineto{\pgfqpoint{0.712406in}{2.370414in}}%
\pgfpathlineto{\pgfqpoint{0.702694in}{2.373612in}}%
\pgfpathlineto{\pgfqpoint{0.702542in}{2.373684in}}%
\pgfpathlineto{\pgfqpoint{0.692982in}{2.377422in}}%
\pgfpathlineto{\pgfqpoint{0.683271in}{2.382620in}}%
\pgfpathlineto{\pgfqpoint{0.682313in}{2.383333in}}%
\pgfpathlineto{\pgfqpoint{0.673559in}{2.388687in}}%
\pgfpathlineto{\pgfqpoint{0.668643in}{2.392982in}}%
\pgfpathlineto{\pgfqpoint{0.663847in}{2.396405in}}%
\pgfpathlineto{\pgfqpoint{0.657991in}{2.402632in}}%
\pgfpathlineto{\pgfqpoint{0.654135in}{2.406010in}}%
\pgfpathlineto{\pgfqpoint{0.649552in}{2.412281in}}%
\pgfpathlineto{\pgfqpoint{0.644424in}{2.418375in}}%
\pgfpathlineto{\pgfqpoint{0.642593in}{2.421930in}}%
\pgfpathlineto{\pgfqpoint{0.637203in}{2.431579in}}%
\pgfpathlineto{\pgfqpoint{0.634712in}{2.435437in}}%
\pgfpathlineto{\pgfqpoint{0.632764in}{2.441228in}}%
\pgfpathlineto{\pgfqpoint{0.629650in}{2.450877in}}%
\pgfpathlineto{\pgfqpoint{0.627232in}{2.460526in}}%
\pgfpathlineto{\pgfqpoint{0.626495in}{2.470175in}}%
\pgfpathlineto{\pgfqpoint{0.625000in}{2.471769in}}%
\pgfusepath{stroke}%
\end{pgfscope}%
\begin{pgfscope}%
\pgfpathrectangle{\pgfqpoint{0.625000in}{0.550000in}}{\pgfqpoint{3.875000in}{3.850000in}} %
\pgfusepath{clip}%
\pgfsetbuttcap%
\pgfsetroundjoin%
\pgfsetlinewidth{0.250937pt}%
\definecolor{currentstroke}{rgb}{0.000000,0.000000,0.000000}%
\pgfsetstrokecolor{currentstroke}%
\pgfsetdash{}{0pt}%
\pgfpathmoveto{\pgfqpoint{0.625000in}{2.641785in}}%
\pgfpathlineto{\pgfqpoint{0.632557in}{2.634211in}}%
\pgfpathlineto{\pgfqpoint{0.625000in}{2.627133in}}%
\pgfusepath{stroke}%
\end{pgfscope}%
\begin{pgfscope}%
\pgfpathrectangle{\pgfqpoint{0.625000in}{0.550000in}}{\pgfqpoint{3.875000in}{3.850000in}} %
\pgfusepath{clip}%
\pgfsetbuttcap%
\pgfsetroundjoin%
\pgfsetlinewidth{0.250937pt}%
\definecolor{currentstroke}{rgb}{0.000000,0.000000,0.000000}%
\pgfsetstrokecolor{currentstroke}%
\pgfsetdash{}{0pt}%
\pgfpathmoveto{\pgfqpoint{0.625000in}{2.795914in}}%
\pgfpathlineto{\pgfqpoint{0.632524in}{2.788596in}}%
\pgfpathlineto{\pgfqpoint{0.625000in}{2.781484in}}%
\pgfusepath{stroke}%
\end{pgfscope}%
\begin{pgfscope}%
\pgfpathrectangle{\pgfqpoint{0.625000in}{0.550000in}}{\pgfqpoint{3.875000in}{3.850000in}} %
\pgfusepath{clip}%
\pgfsetbuttcap%
\pgfsetroundjoin%
\pgfsetlinewidth{0.250937pt}%
\definecolor{currentstroke}{rgb}{0.000000,0.000000,0.000000}%
\pgfsetstrokecolor{currentstroke}%
\pgfsetdash{}{0pt}%
\pgfpathmoveto{\pgfqpoint{0.625000in}{2.950384in}}%
\pgfpathlineto{\pgfqpoint{0.632470in}{2.942982in}}%
\pgfpathlineto{\pgfqpoint{0.625000in}{2.935519in}}%
\pgfusepath{stroke}%
\end{pgfscope}%
\begin{pgfscope}%
\pgfpathrectangle{\pgfqpoint{0.625000in}{0.550000in}}{\pgfqpoint{3.875000in}{3.850000in}} %
\pgfusepath{clip}%
\pgfsetbuttcap%
\pgfsetroundjoin%
\pgfsetlinewidth{0.250937pt}%
\definecolor{currentstroke}{rgb}{0.000000,0.000000,0.000000}%
\pgfsetstrokecolor{currentstroke}%
\pgfsetdash{}{0pt}%
\pgfpathmoveto{\pgfqpoint{0.625000in}{3.104445in}}%
\pgfpathlineto{\pgfqpoint{0.632504in}{3.097368in}}%
\pgfpathlineto{\pgfqpoint{0.625000in}{3.090141in}}%
\pgfusepath{stroke}%
\end{pgfscope}%
\begin{pgfscope}%
\pgfpathrectangle{\pgfqpoint{0.625000in}{0.550000in}}{\pgfqpoint{3.875000in}{3.850000in}} %
\pgfusepath{clip}%
\pgfsetbuttcap%
\pgfsetroundjoin%
\pgfsetlinewidth{0.250937pt}%
\definecolor{currentstroke}{rgb}{0.000000,0.000000,0.000000}%
\pgfsetstrokecolor{currentstroke}%
\pgfsetdash{}{0pt}%
\pgfpathmoveto{\pgfqpoint{0.625000in}{3.258897in}}%
\pgfpathlineto{\pgfqpoint{0.634267in}{3.251754in}}%
\pgfpathlineto{\pgfqpoint{0.625000in}{3.244659in}}%
\pgfusepath{stroke}%
\end{pgfscope}%
\begin{pgfscope}%
\pgfpathrectangle{\pgfqpoint{0.625000in}{0.550000in}}{\pgfqpoint{3.875000in}{3.850000in}} %
\pgfusepath{clip}%
\pgfsetbuttcap%
\pgfsetroundjoin%
\pgfsetlinewidth{0.250937pt}%
\definecolor{currentstroke}{rgb}{0.000000,0.000000,0.000000}%
\pgfsetstrokecolor{currentstroke}%
\pgfsetdash{}{0pt}%
\pgfpathmoveto{\pgfqpoint{0.625000in}{3.413523in}}%
\pgfpathlineto{\pgfqpoint{0.632508in}{3.406140in}}%
\pgfpathlineto{\pgfqpoint{0.625000in}{3.398997in}}%
\pgfusepath{stroke}%
\end{pgfscope}%
\begin{pgfscope}%
\pgfpathrectangle{\pgfqpoint{0.625000in}{0.550000in}}{\pgfqpoint{3.875000in}{3.850000in}} %
\pgfusepath{clip}%
\pgfsetbuttcap%
\pgfsetroundjoin%
\pgfsetlinewidth{0.250937pt}%
\definecolor{currentstroke}{rgb}{0.000000,0.000000,0.000000}%
\pgfsetstrokecolor{currentstroke}%
\pgfsetdash{}{0pt}%
\pgfpathmoveto{\pgfqpoint{0.625000in}{3.567669in}}%
\pgfpathlineto{\pgfqpoint{0.632517in}{3.560526in}}%
\pgfpathlineto{\pgfqpoint{0.625000in}{3.553388in}}%
\pgfusepath{stroke}%
\end{pgfscope}%
\begin{pgfscope}%
\pgfpathrectangle{\pgfqpoint{0.625000in}{0.550000in}}{\pgfqpoint{3.875000in}{3.850000in}} %
\pgfusepath{clip}%
\pgfsetbuttcap%
\pgfsetroundjoin%
\pgfsetlinewidth{0.250937pt}%
\definecolor{currentstroke}{rgb}{0.000000,0.000000,0.000000}%
\pgfsetstrokecolor{currentstroke}%
\pgfsetdash{}{0pt}%
\pgfpathmoveto{\pgfqpoint{0.625000in}{3.722130in}}%
\pgfpathlineto{\pgfqpoint{0.632536in}{3.714912in}}%
\pgfpathlineto{\pgfqpoint{0.625000in}{3.707478in}}%
\pgfusepath{stroke}%
\end{pgfscope}%
\begin{pgfscope}%
\pgfpathrectangle{\pgfqpoint{0.625000in}{0.550000in}}{\pgfqpoint{3.875000in}{3.850000in}} %
\pgfusepath{clip}%
\pgfsetbuttcap%
\pgfsetroundjoin%
\pgfsetlinewidth{0.250937pt}%
\definecolor{currentstroke}{rgb}{0.000000,0.000000,0.000000}%
\pgfsetstrokecolor{currentstroke}%
\pgfsetdash{}{0pt}%
\pgfpathmoveto{\pgfqpoint{0.625000in}{3.876383in}}%
\pgfpathlineto{\pgfqpoint{0.632445in}{3.869298in}}%
\pgfpathlineto{\pgfqpoint{0.625000in}{3.862184in}}%
\pgfusepath{stroke}%
\end{pgfscope}%
\begin{pgfscope}%
\pgfpathrectangle{\pgfqpoint{0.625000in}{0.550000in}}{\pgfqpoint{3.875000in}{3.850000in}} %
\pgfusepath{clip}%
\pgfsetbuttcap%
\pgfsetroundjoin%
\pgfsetlinewidth{0.250937pt}%
\definecolor{currentstroke}{rgb}{0.000000,0.000000,0.000000}%
\pgfsetstrokecolor{currentstroke}%
\pgfsetdash{}{0pt}%
\pgfpathmoveto{\pgfqpoint{0.625000in}{4.030907in}}%
\pgfpathlineto{\pgfqpoint{0.634276in}{4.023684in}}%
\pgfpathlineto{\pgfqpoint{0.625000in}{4.016532in}}%
\pgfusepath{stroke}%
\end{pgfscope}%
\begin{pgfscope}%
\pgfpathrectangle{\pgfqpoint{0.625000in}{0.550000in}}{\pgfqpoint{3.875000in}{3.850000in}} %
\pgfusepath{clip}%
\pgfsetbuttcap%
\pgfsetroundjoin%
\pgfsetlinewidth{0.250937pt}%
\definecolor{currentstroke}{rgb}{0.000000,0.000000,0.000000}%
\pgfsetstrokecolor{currentstroke}%
\pgfsetdash{}{0pt}%
\pgfpathmoveto{\pgfqpoint{0.625000in}{4.185227in}}%
\pgfpathlineto{\pgfqpoint{0.632434in}{4.178070in}}%
\pgfpathlineto{\pgfqpoint{0.625000in}{4.171249in}}%
\pgfusepath{stroke}%
\end{pgfscope}%
\begin{pgfscope}%
\pgfpathrectangle{\pgfqpoint{0.625000in}{0.550000in}}{\pgfqpoint{3.875000in}{3.850000in}} %
\pgfusepath{clip}%
\pgfsetbuttcap%
\pgfsetroundjoin%
\pgfsetlinewidth{0.250937pt}%
\definecolor{currentstroke}{rgb}{0.000000,0.000000,0.000000}%
\pgfsetstrokecolor{currentstroke}%
\pgfsetdash{}{0pt}%
\pgfpathmoveto{\pgfqpoint{0.625000in}{4.339584in}}%
\pgfpathlineto{\pgfqpoint{0.632404in}{4.332456in}}%
\pgfpathlineto{\pgfqpoint{0.625000in}{4.325304in}}%
\pgfusepath{stroke}%
\end{pgfscope}%
\begin{pgfscope}%
\pgfpathrectangle{\pgfqpoint{0.625000in}{0.550000in}}{\pgfqpoint{3.875000in}{3.850000in}} %
\pgfusepath{clip}%
\pgfsetbuttcap%
\pgfsetroundjoin%
\pgfsetlinewidth{0.250937pt}%
\definecolor{currentstroke}{rgb}{0.000000,0.000000,0.000000}%
\pgfsetstrokecolor{currentstroke}%
\pgfsetdash{}{0pt}%
\pgfpathmoveto{\pgfqpoint{0.634712in}{1.180654in}}%
\pgfpathlineto{\pgfqpoint{0.630862in}{1.186842in}}%
\pgfpathlineto{\pgfqpoint{0.628991in}{1.196491in}}%
\pgfpathlineto{\pgfqpoint{0.634712in}{1.205086in}}%
\pgfpathlineto{\pgfqpoint{0.644424in}{1.201802in}}%
\pgfpathlineto{\pgfqpoint{0.649389in}{1.196491in}}%
\pgfpathlineto{\pgfqpoint{0.647554in}{1.186842in}}%
\pgfpathlineto{\pgfqpoint{0.644424in}{1.184302in}}%
\pgfpathlineto{\pgfqpoint{0.634712in}{1.180654in}}%
\pgfusepath{stroke}%
\end{pgfscope}%
\begin{pgfscope}%
\pgfpathrectangle{\pgfqpoint{0.625000in}{0.550000in}}{\pgfqpoint{3.875000in}{3.850000in}} %
\pgfusepath{clip}%
\pgfsetbuttcap%
\pgfsetroundjoin%
\pgfsetlinewidth{0.250937pt}%
\definecolor{currentstroke}{rgb}{0.000000,0.000000,0.000000}%
\pgfsetstrokecolor{currentstroke}%
\pgfsetdash{}{0pt}%
\pgfpathmoveto{\pgfqpoint{0.634712in}{3.754563in}}%
\pgfpathlineto{\pgfqpoint{0.628991in}{3.763158in}}%
\pgfpathlineto{\pgfqpoint{0.630862in}{3.772807in}}%
\pgfpathlineto{\pgfqpoint{0.634712in}{3.778995in}}%
\pgfpathlineto{\pgfqpoint{0.644424in}{3.775347in}}%
\pgfpathlineto{\pgfqpoint{0.647554in}{3.772807in}}%
\pgfpathlineto{\pgfqpoint{0.649389in}{3.763158in}}%
\pgfpathlineto{\pgfqpoint{0.644424in}{3.757847in}}%
\pgfpathlineto{\pgfqpoint{0.634712in}{3.754563in}}%
\pgfusepath{stroke}%
\end{pgfscope}%
\begin{pgfscope}%
\pgfpathrectangle{\pgfqpoint{0.625000in}{0.550000in}}{\pgfqpoint{3.875000in}{3.850000in}} %
\pgfusepath{clip}%
\pgfsetbuttcap%
\pgfsetroundjoin%
\pgfsetlinewidth{0.250937pt}%
\definecolor{currentstroke}{rgb}{0.000000,0.000000,0.000000}%
\pgfsetstrokecolor{currentstroke}%
\pgfsetdash{}{0pt}%
\pgfpathmoveto{\pgfqpoint{0.625000in}{0.634751in}}%
\pgfpathlineto{\pgfqpoint{0.632795in}{0.627193in}}%
\pgfpathlineto{\pgfqpoint{0.625000in}{0.619657in}}%
\pgfusepath{stroke}%
\end{pgfscope}%
\begin{pgfscope}%
\pgfpathrectangle{\pgfqpoint{0.625000in}{0.550000in}}{\pgfqpoint{3.875000in}{3.850000in}} %
\pgfusepath{clip}%
\pgfsetbuttcap%
\pgfsetroundjoin%
\pgfsetlinewidth{0.250937pt}%
\definecolor{currentstroke}{rgb}{0.000000,0.000000,0.000000}%
\pgfsetstrokecolor{currentstroke}%
\pgfsetdash{}{0pt}%
\pgfpathmoveto{\pgfqpoint{0.625000in}{0.788730in}}%
\pgfpathlineto{\pgfqpoint{0.632736in}{0.781579in}}%
\pgfpathlineto{\pgfqpoint{0.625000in}{0.774109in}}%
\pgfusepath{stroke}%
\end{pgfscope}%
\begin{pgfscope}%
\pgfpathrectangle{\pgfqpoint{0.625000in}{0.550000in}}{\pgfqpoint{3.875000in}{3.850000in}} %
\pgfusepath{clip}%
\pgfsetbuttcap%
\pgfsetroundjoin%
\pgfsetlinewidth{0.250937pt}%
\definecolor{currentstroke}{rgb}{0.000000,0.000000,0.000000}%
\pgfsetstrokecolor{currentstroke}%
\pgfsetdash{}{0pt}%
\pgfpathmoveto{\pgfqpoint{0.625000in}{0.943406in}}%
\pgfpathlineto{\pgfqpoint{0.634420in}{0.935965in}}%
\pgfpathlineto{\pgfqpoint{0.625000in}{0.928457in}}%
\pgfusepath{stroke}%
\end{pgfscope}%
\begin{pgfscope}%
\pgfpathrectangle{\pgfqpoint{0.625000in}{0.550000in}}{\pgfqpoint{3.875000in}{3.850000in}} %
\pgfusepath{clip}%
\pgfsetbuttcap%
\pgfsetroundjoin%
\pgfsetlinewidth{0.250937pt}%
\definecolor{currentstroke}{rgb}{0.000000,0.000000,0.000000}%
\pgfsetstrokecolor{currentstroke}%
\pgfsetdash{}{0pt}%
\pgfpathmoveto{\pgfqpoint{0.625000in}{1.097776in}}%
\pgfpathlineto{\pgfqpoint{0.632743in}{1.090351in}}%
\pgfpathlineto{\pgfqpoint{0.625000in}{1.082953in}}%
\pgfusepath{stroke}%
\end{pgfscope}%
\begin{pgfscope}%
\pgfpathrectangle{\pgfqpoint{0.625000in}{0.550000in}}{\pgfqpoint{3.875000in}{3.850000in}} %
\pgfusepath{clip}%
\pgfsetbuttcap%
\pgfsetroundjoin%
\pgfsetlinewidth{0.250937pt}%
\definecolor{currentstroke}{rgb}{0.000000,0.000000,0.000000}%
\pgfsetstrokecolor{currentstroke}%
\pgfsetdash{}{0pt}%
\pgfpathmoveto{\pgfqpoint{0.625000in}{1.252385in}}%
\pgfpathlineto{\pgfqpoint{0.632749in}{1.244737in}}%
\pgfpathlineto{\pgfqpoint{0.625000in}{1.237299in}}%
\pgfusepath{stroke}%
\end{pgfscope}%
\begin{pgfscope}%
\pgfpathrectangle{\pgfqpoint{0.625000in}{0.550000in}}{\pgfqpoint{3.875000in}{3.850000in}} %
\pgfusepath{clip}%
\pgfsetbuttcap%
\pgfsetroundjoin%
\pgfsetlinewidth{0.250937pt}%
\definecolor{currentstroke}{rgb}{0.000000,0.000000,0.000000}%
\pgfsetstrokecolor{currentstroke}%
\pgfsetdash{}{0pt}%
\pgfpathmoveto{\pgfqpoint{0.625000in}{1.406438in}}%
\pgfpathlineto{\pgfqpoint{0.632690in}{1.399123in}}%
\pgfpathlineto{\pgfqpoint{0.625000in}{1.391803in}}%
\pgfusepath{stroke}%
\end{pgfscope}%
\begin{pgfscope}%
\pgfpathrectangle{\pgfqpoint{0.625000in}{0.550000in}}{\pgfqpoint{3.875000in}{3.850000in}} %
\pgfusepath{clip}%
\pgfsetbuttcap%
\pgfsetroundjoin%
\pgfsetlinewidth{0.250937pt}%
\definecolor{currentstroke}{rgb}{0.000000,0.000000,0.000000}%
\pgfsetstrokecolor{currentstroke}%
\pgfsetdash{}{0pt}%
\pgfpathmoveto{\pgfqpoint{0.625000in}{1.560813in}}%
\pgfpathlineto{\pgfqpoint{0.632666in}{1.553509in}}%
\pgfpathlineto{\pgfqpoint{0.625000in}{1.545968in}}%
\pgfusepath{stroke}%
\end{pgfscope}%
\begin{pgfscope}%
\pgfpathrectangle{\pgfqpoint{0.625000in}{0.550000in}}{\pgfqpoint{3.875000in}{3.850000in}} %
\pgfusepath{clip}%
\pgfsetbuttcap%
\pgfsetroundjoin%
\pgfsetlinewidth{0.250937pt}%
\definecolor{currentstroke}{rgb}{0.000000,0.000000,0.000000}%
\pgfsetstrokecolor{currentstroke}%
\pgfsetdash{}{0pt}%
\pgfpathmoveto{\pgfqpoint{0.625000in}{1.715188in}}%
\pgfpathlineto{\pgfqpoint{0.634397in}{1.707895in}}%
\pgfpathlineto{\pgfqpoint{0.625000in}{1.700555in}}%
\pgfusepath{stroke}%
\end{pgfscope}%
\begin{pgfscope}%
\pgfpathrectangle{\pgfqpoint{0.625000in}{0.550000in}}{\pgfqpoint{3.875000in}{3.850000in}} %
\pgfusepath{clip}%
\pgfsetbuttcap%
\pgfsetroundjoin%
\pgfsetlinewidth{0.250937pt}%
\definecolor{currentstroke}{rgb}{0.000000,0.000000,0.000000}%
\pgfsetstrokecolor{currentstroke}%
\pgfsetdash{}{0pt}%
\pgfpathmoveto{\pgfqpoint{0.625000in}{1.869623in}}%
\pgfpathlineto{\pgfqpoint{0.632620in}{1.862281in}}%
\pgfpathlineto{\pgfqpoint{0.625000in}{1.855090in}}%
\pgfusepath{stroke}%
\end{pgfscope}%
\begin{pgfscope}%
\pgfpathrectangle{\pgfqpoint{0.625000in}{0.550000in}}{\pgfqpoint{3.875000in}{3.850000in}} %
\pgfusepath{clip}%
\pgfsetbuttcap%
\pgfsetroundjoin%
\pgfsetlinewidth{0.250937pt}%
\definecolor{currentstroke}{rgb}{0.000000,0.000000,0.000000}%
\pgfsetstrokecolor{currentstroke}%
\pgfsetdash{}{0pt}%
\pgfpathmoveto{\pgfqpoint{0.625000in}{2.024255in}}%
\pgfpathlineto{\pgfqpoint{0.632596in}{2.016667in}}%
\pgfpathlineto{\pgfqpoint{0.625000in}{2.009140in}}%
\pgfusepath{stroke}%
\end{pgfscope}%
\begin{pgfscope}%
\pgfpathrectangle{\pgfqpoint{0.625000in}{0.550000in}}{\pgfqpoint{3.875000in}{3.850000in}} %
\pgfusepath{clip}%
\pgfsetbuttcap%
\pgfsetroundjoin%
\pgfsetlinewidth{0.250937pt}%
\definecolor{currentstroke}{rgb}{0.000000,0.000000,0.000000}%
\pgfsetstrokecolor{currentstroke}%
\pgfsetdash{}{0pt}%
\pgfpathmoveto{\pgfqpoint{0.625000in}{2.178262in}}%
\pgfpathlineto{\pgfqpoint{0.632625in}{2.171053in}}%
\pgfpathlineto{\pgfqpoint{0.625000in}{2.163636in}}%
\pgfusepath{stroke}%
\end{pgfscope}%
\begin{pgfscope}%
\pgfpathrectangle{\pgfqpoint{0.625000in}{0.550000in}}{\pgfqpoint{3.875000in}{3.850000in}} %
\pgfusepath{clip}%
\pgfsetbuttcap%
\pgfsetroundjoin%
\pgfsetlinewidth{0.250937pt}%
\definecolor{currentstroke}{rgb}{0.000000,0.000000,0.000000}%
\pgfsetstrokecolor{currentstroke}%
\pgfsetdash{}{0pt}%
\pgfpathmoveto{\pgfqpoint{0.625000in}{2.332604in}}%
\pgfpathlineto{\pgfqpoint{0.632649in}{2.325439in}}%
\pgfpathlineto{\pgfqpoint{0.625000in}{2.317772in}}%
\pgfusepath{stroke}%
\end{pgfscope}%
\begin{pgfscope}%
\pgfpathrectangle{\pgfqpoint{0.625000in}{0.550000in}}{\pgfqpoint{3.875000in}{3.850000in}} %
\pgfusepath{clip}%
\pgfsetbuttcap%
\pgfsetroundjoin%
\pgfsetlinewidth{0.250937pt}%
\definecolor{currentstroke}{rgb}{0.000000,0.000000,0.000000}%
\pgfsetstrokecolor{currentstroke}%
\pgfsetdash{}{0pt}%
\pgfpathmoveto{\pgfqpoint{0.625000in}{2.487964in}}%
\pgfpathlineto{\pgfqpoint{0.626416in}{2.489474in}}%
\pgfpathlineto{\pgfqpoint{0.627148in}{2.499123in}}%
\pgfpathlineto{\pgfqpoint{0.629496in}{2.508772in}}%
\pgfpathlineto{\pgfqpoint{0.632502in}{2.518421in}}%
\pgfpathlineto{\pgfqpoint{0.634712in}{2.524993in}}%
\pgfpathlineto{\pgfqpoint{0.636699in}{2.528070in}}%
\pgfpathlineto{\pgfqpoint{0.641924in}{2.537719in}}%
\pgfpathlineto{\pgfqpoint{0.644424in}{2.542574in}}%
\pgfpathlineto{\pgfqpoint{0.648458in}{2.547368in}}%
\pgfpathlineto{\pgfqpoint{0.654135in}{2.555136in}}%
\pgfpathlineto{\pgfqpoint{0.656283in}{2.557018in}}%
\pgfpathlineto{\pgfqpoint{0.663847in}{2.565060in}}%
\pgfpathlineto{\pgfqpoint{0.666098in}{2.566667in}}%
\pgfpathlineto{\pgfqpoint{0.673559in}{2.573185in}}%
\pgfpathlineto{\pgfqpoint{0.678678in}{2.576316in}}%
\pgfpathlineto{\pgfqpoint{0.683271in}{2.579736in}}%
\pgfpathlineto{\pgfqpoint{0.692982in}{2.584972in}}%
\pgfpathlineto{\pgfqpoint{0.695521in}{2.585965in}}%
\pgfpathlineto{\pgfqpoint{0.702694in}{2.589379in}}%
\pgfpathlineto{\pgfqpoint{0.712406in}{2.592739in}}%
\pgfpathlineto{\pgfqpoint{0.722118in}{2.595161in}}%
\pgfpathlineto{\pgfqpoint{0.724970in}{2.595614in}}%
\pgfpathlineto{\pgfqpoint{0.731830in}{2.596918in}}%
\pgfpathlineto{\pgfqpoint{0.741541in}{2.597890in}}%
\pgfpathlineto{\pgfqpoint{0.751253in}{2.598078in}}%
\pgfpathlineto{\pgfqpoint{0.760965in}{2.597495in}}%
\pgfpathlineto{\pgfqpoint{0.770677in}{2.596129in}}%
\pgfpathlineto{\pgfqpoint{0.773134in}{2.595614in}}%
\pgfpathlineto{\pgfqpoint{0.780388in}{2.594010in}}%
\pgfpathlineto{\pgfqpoint{0.790100in}{2.591040in}}%
\pgfpathlineto{\pgfqpoint{0.799812in}{2.587124in}}%
\pgfpathlineto{\pgfqpoint{0.802276in}{2.585965in}}%
\pgfpathlineto{\pgfqpoint{0.809524in}{2.582194in}}%
\pgfpathlineto{\pgfqpoint{0.818924in}{2.576316in}}%
\pgfpathlineto{\pgfqpoint{0.819236in}{2.576095in}}%
\pgfpathlineto{\pgfqpoint{0.828947in}{2.568542in}}%
\pgfpathlineto{\pgfqpoint{0.831143in}{2.566667in}}%
\pgfpathlineto{\pgfqpoint{0.838659in}{2.559157in}}%
\pgfpathlineto{\pgfqpoint{0.840681in}{2.557018in}}%
\pgfpathlineto{\pgfqpoint{0.848287in}{2.547368in}}%
\pgfpathlineto{\pgfqpoint{0.848371in}{2.547240in}}%
\pgfpathlineto{\pgfqpoint{0.854544in}{2.537719in}}%
\pgfpathlineto{\pgfqpoint{0.858083in}{2.530752in}}%
\pgfpathlineto{\pgfqpoint{0.859463in}{2.528070in}}%
\pgfpathlineto{\pgfqpoint{0.863408in}{2.518421in}}%
\pgfpathlineto{\pgfqpoint{0.866305in}{2.508772in}}%
\pgfpathlineto{\pgfqpoint{0.867794in}{2.501564in}}%
\pgfpathlineto{\pgfqpoint{0.868328in}{2.499123in}}%
\pgfpathlineto{\pgfqpoint{0.869582in}{2.489474in}}%
\pgfpathlineto{\pgfqpoint{0.869995in}{2.479825in}}%
\pgfpathlineto{\pgfqpoint{0.869582in}{2.470175in}}%
\pgfpathlineto{\pgfqpoint{0.868328in}{2.460526in}}%
\pgfpathlineto{\pgfqpoint{0.867794in}{2.458085in}}%
\pgfpathlineto{\pgfqpoint{0.866305in}{2.450877in}}%
\pgfpathlineto{\pgfqpoint{0.863408in}{2.441228in}}%
\pgfpathlineto{\pgfqpoint{0.859463in}{2.431579in}}%
\pgfpathlineto{\pgfqpoint{0.858083in}{2.428897in}}%
\pgfpathlineto{\pgfqpoint{0.854544in}{2.421930in}}%
\pgfpathlineto{\pgfqpoint{0.848371in}{2.412409in}}%
\pgfpathlineto{\pgfqpoint{0.848287in}{2.412281in}}%
\pgfpathlineto{\pgfqpoint{0.840681in}{2.402632in}}%
\pgfpathlineto{\pgfqpoint{0.838659in}{2.400492in}}%
\pgfpathlineto{\pgfqpoint{0.831143in}{2.392982in}}%
\pgfpathlineto{\pgfqpoint{0.828947in}{2.391107in}}%
\pgfpathlineto{\pgfqpoint{0.819236in}{2.383555in}}%
\pgfpathlineto{\pgfqpoint{0.818924in}{2.383333in}}%
\pgfpathlineto{\pgfqpoint{0.809524in}{2.377455in}}%
\pgfpathlineto{\pgfqpoint{0.802276in}{2.373684in}}%
\pgfpathlineto{\pgfqpoint{0.799812in}{2.372525in}}%
\pgfpathlineto{\pgfqpoint{0.790100in}{2.368609in}}%
\pgfpathlineto{\pgfqpoint{0.780388in}{2.365639in}}%
\pgfpathlineto{\pgfqpoint{0.773134in}{2.364035in}}%
\pgfpathlineto{\pgfqpoint{0.770677in}{2.363520in}}%
\pgfpathlineto{\pgfqpoint{0.760965in}{2.362154in}}%
\pgfpathlineto{\pgfqpoint{0.751253in}{2.361571in}}%
\pgfpathlineto{\pgfqpoint{0.741541in}{2.361759in}}%
\pgfpathlineto{\pgfqpoint{0.731830in}{2.362731in}}%
\pgfpathlineto{\pgfqpoint{0.724970in}{2.364035in}}%
\pgfpathlineto{\pgfqpoint{0.722118in}{2.364489in}}%
\pgfpathlineto{\pgfqpoint{0.712406in}{2.366910in}}%
\pgfpathlineto{\pgfqpoint{0.702694in}{2.370270in}}%
\pgfpathlineto{\pgfqpoint{0.695521in}{2.373684in}}%
\pgfpathlineto{\pgfqpoint{0.692982in}{2.374677in}}%
\pgfpathlineto{\pgfqpoint{0.683271in}{2.379913in}}%
\pgfpathlineto{\pgfqpoint{0.678678in}{2.383333in}}%
\pgfpathlineto{\pgfqpoint{0.673559in}{2.386464in}}%
\pgfpathlineto{\pgfqpoint{0.666098in}{2.392982in}}%
\pgfpathlineto{\pgfqpoint{0.663847in}{2.394589in}}%
\pgfpathlineto{\pgfqpoint{0.656283in}{2.402632in}}%
\pgfpathlineto{\pgfqpoint{0.654135in}{2.404513in}}%
\pgfpathlineto{\pgfqpoint{0.648458in}{2.412281in}}%
\pgfpathlineto{\pgfqpoint{0.644424in}{2.417075in}}%
\pgfpathlineto{\pgfqpoint{0.641924in}{2.421930in}}%
\pgfpathlineto{\pgfqpoint{0.636699in}{2.431579in}}%
\pgfpathlineto{\pgfqpoint{0.634712in}{2.434657in}}%
\pgfpathlineto{\pgfqpoint{0.632502in}{2.441228in}}%
\pgfpathlineto{\pgfqpoint{0.629496in}{2.450877in}}%
\pgfpathlineto{\pgfqpoint{0.627153in}{2.460526in}}%
\pgfpathlineto{\pgfqpoint{0.626416in}{2.470175in}}%
\pgfpathlineto{\pgfqpoint{0.625000in}{2.471685in}}%
\pgfusepath{stroke}%
\end{pgfscope}%
\begin{pgfscope}%
\pgfpathrectangle{\pgfqpoint{0.625000in}{0.550000in}}{\pgfqpoint{3.875000in}{3.850000in}} %
\pgfusepath{clip}%
\pgfsetbuttcap%
\pgfsetroundjoin%
\pgfsetlinewidth{0.250937pt}%
\definecolor{currentstroke}{rgb}{0.000000,0.000000,0.000000}%
\pgfsetstrokecolor{currentstroke}%
\pgfsetdash{}{0pt}%
\pgfpathmoveto{\pgfqpoint{0.625000in}{2.641874in}}%
\pgfpathlineto{\pgfqpoint{0.632645in}{2.634211in}}%
\pgfpathlineto{\pgfqpoint{0.625000in}{2.627050in}}%
\pgfusepath{stroke}%
\end{pgfscope}%
\begin{pgfscope}%
\pgfpathrectangle{\pgfqpoint{0.625000in}{0.550000in}}{\pgfqpoint{3.875000in}{3.850000in}} %
\pgfusepath{clip}%
\pgfsetbuttcap%
\pgfsetroundjoin%
\pgfsetlinewidth{0.250937pt}%
\definecolor{currentstroke}{rgb}{0.000000,0.000000,0.000000}%
\pgfsetstrokecolor{currentstroke}%
\pgfsetdash{}{0pt}%
\pgfpathmoveto{\pgfqpoint{0.625000in}{2.795999in}}%
\pgfpathlineto{\pgfqpoint{0.632612in}{2.788596in}}%
\pgfpathlineto{\pgfqpoint{0.625000in}{2.781401in}}%
\pgfusepath{stroke}%
\end{pgfscope}%
\begin{pgfscope}%
\pgfpathrectangle{\pgfqpoint{0.625000in}{0.550000in}}{\pgfqpoint{3.875000in}{3.850000in}} %
\pgfusepath{clip}%
\pgfsetbuttcap%
\pgfsetroundjoin%
\pgfsetlinewidth{0.250937pt}%
\definecolor{currentstroke}{rgb}{0.000000,0.000000,0.000000}%
\pgfsetstrokecolor{currentstroke}%
\pgfsetdash{}{0pt}%
\pgfpathmoveto{\pgfqpoint{0.625000in}{2.950472in}}%
\pgfpathlineto{\pgfqpoint{0.632559in}{2.942982in}}%
\pgfpathlineto{\pgfqpoint{0.625000in}{2.935430in}}%
\pgfusepath{stroke}%
\end{pgfscope}%
\begin{pgfscope}%
\pgfpathrectangle{\pgfqpoint{0.625000in}{0.550000in}}{\pgfqpoint{3.875000in}{3.850000in}} %
\pgfusepath{clip}%
\pgfsetbuttcap%
\pgfsetroundjoin%
\pgfsetlinewidth{0.250937pt}%
\definecolor{currentstroke}{rgb}{0.000000,0.000000,0.000000}%
\pgfsetstrokecolor{currentstroke}%
\pgfsetdash{}{0pt}%
\pgfpathmoveto{\pgfqpoint{0.625000in}{3.104527in}}%
\pgfpathlineto{\pgfqpoint{0.632592in}{3.097368in}}%
\pgfpathlineto{\pgfqpoint{0.625000in}{3.090057in}}%
\pgfusepath{stroke}%
\end{pgfscope}%
\begin{pgfscope}%
\pgfpathrectangle{\pgfqpoint{0.625000in}{0.550000in}}{\pgfqpoint{3.875000in}{3.850000in}} %
\pgfusepath{clip}%
\pgfsetbuttcap%
\pgfsetroundjoin%
\pgfsetlinewidth{0.250937pt}%
\definecolor{currentstroke}{rgb}{0.000000,0.000000,0.000000}%
\pgfsetstrokecolor{currentstroke}%
\pgfsetdash{}{0pt}%
\pgfpathmoveto{\pgfqpoint{0.625000in}{3.258983in}}%
\pgfpathlineto{\pgfqpoint{0.634378in}{3.251754in}}%
\pgfpathlineto{\pgfqpoint{0.625000in}{3.244574in}}%
\pgfusepath{stroke}%
\end{pgfscope}%
\begin{pgfscope}%
\pgfpathrectangle{\pgfqpoint{0.625000in}{0.550000in}}{\pgfqpoint{3.875000in}{3.850000in}} %
\pgfusepath{clip}%
\pgfsetbuttcap%
\pgfsetroundjoin%
\pgfsetlinewidth{0.250937pt}%
\definecolor{currentstroke}{rgb}{0.000000,0.000000,0.000000}%
\pgfsetstrokecolor{currentstroke}%
\pgfsetdash{}{0pt}%
\pgfpathmoveto{\pgfqpoint{0.625000in}{3.413609in}}%
\pgfpathlineto{\pgfqpoint{0.632595in}{3.406140in}}%
\pgfpathlineto{\pgfqpoint{0.625000in}{3.398915in}}%
\pgfusepath{stroke}%
\end{pgfscope}%
\begin{pgfscope}%
\pgfpathrectangle{\pgfqpoint{0.625000in}{0.550000in}}{\pgfqpoint{3.875000in}{3.850000in}} %
\pgfusepath{clip}%
\pgfsetbuttcap%
\pgfsetroundjoin%
\pgfsetlinewidth{0.250937pt}%
\definecolor{currentstroke}{rgb}{0.000000,0.000000,0.000000}%
\pgfsetstrokecolor{currentstroke}%
\pgfsetdash{}{0pt}%
\pgfpathmoveto{\pgfqpoint{0.625000in}{3.567753in}}%
\pgfpathlineto{\pgfqpoint{0.632606in}{3.560526in}}%
\pgfpathlineto{\pgfqpoint{0.625000in}{3.553303in}}%
\pgfusepath{stroke}%
\end{pgfscope}%
\begin{pgfscope}%
\pgfpathrectangle{\pgfqpoint{0.625000in}{0.550000in}}{\pgfqpoint{3.875000in}{3.850000in}} %
\pgfusepath{clip}%
\pgfsetbuttcap%
\pgfsetroundjoin%
\pgfsetlinewidth{0.250937pt}%
\definecolor{currentstroke}{rgb}{0.000000,0.000000,0.000000}%
\pgfsetstrokecolor{currentstroke}%
\pgfsetdash{}{0pt}%
\pgfpathmoveto{\pgfqpoint{0.625000in}{3.722215in}}%
\pgfpathlineto{\pgfqpoint{0.632625in}{3.714912in}}%
\pgfpathlineto{\pgfqpoint{0.625000in}{3.707390in}}%
\pgfusepath{stroke}%
\end{pgfscope}%
\begin{pgfscope}%
\pgfpathrectangle{\pgfqpoint{0.625000in}{0.550000in}}{\pgfqpoint{3.875000in}{3.850000in}} %
\pgfusepath{clip}%
\pgfsetbuttcap%
\pgfsetroundjoin%
\pgfsetlinewidth{0.250937pt}%
\definecolor{currentstroke}{rgb}{0.000000,0.000000,0.000000}%
\pgfsetstrokecolor{currentstroke}%
\pgfsetdash{}{0pt}%
\pgfpathmoveto{\pgfqpoint{0.625000in}{3.876470in}}%
\pgfpathlineto{\pgfqpoint{0.632537in}{3.869298in}}%
\pgfpathlineto{\pgfqpoint{0.625000in}{3.862097in}}%
\pgfusepath{stroke}%
\end{pgfscope}%
\begin{pgfscope}%
\pgfpathrectangle{\pgfqpoint{0.625000in}{0.550000in}}{\pgfqpoint{3.875000in}{3.850000in}} %
\pgfusepath{clip}%
\pgfsetbuttcap%
\pgfsetroundjoin%
\pgfsetlinewidth{0.250937pt}%
\definecolor{currentstroke}{rgb}{0.000000,0.000000,0.000000}%
\pgfsetstrokecolor{currentstroke}%
\pgfsetdash{}{0pt}%
\pgfpathmoveto{\pgfqpoint{0.625000in}{4.030991in}}%
\pgfpathlineto{\pgfqpoint{0.634385in}{4.023684in}}%
\pgfpathlineto{\pgfqpoint{0.625000in}{4.016448in}}%
\pgfusepath{stroke}%
\end{pgfscope}%
\begin{pgfscope}%
\pgfpathrectangle{\pgfqpoint{0.625000in}{0.550000in}}{\pgfqpoint{3.875000in}{3.850000in}} %
\pgfusepath{clip}%
\pgfsetbuttcap%
\pgfsetroundjoin%
\pgfsetlinewidth{0.250937pt}%
\definecolor{currentstroke}{rgb}{0.000000,0.000000,0.000000}%
\pgfsetstrokecolor{currentstroke}%
\pgfsetdash{}{0pt}%
\pgfpathmoveto{\pgfqpoint{0.625000in}{4.185315in}}%
\pgfpathlineto{\pgfqpoint{0.632526in}{4.178070in}}%
\pgfpathlineto{\pgfqpoint{0.625000in}{4.171164in}}%
\pgfusepath{stroke}%
\end{pgfscope}%
\begin{pgfscope}%
\pgfpathrectangle{\pgfqpoint{0.625000in}{0.550000in}}{\pgfqpoint{3.875000in}{3.850000in}} %
\pgfusepath{clip}%
\pgfsetbuttcap%
\pgfsetroundjoin%
\pgfsetlinewidth{0.250937pt}%
\definecolor{currentstroke}{rgb}{0.000000,0.000000,0.000000}%
\pgfsetstrokecolor{currentstroke}%
\pgfsetdash{}{0pt}%
\pgfpathmoveto{\pgfqpoint{0.625000in}{4.339671in}}%
\pgfpathlineto{\pgfqpoint{0.632495in}{4.332456in}}%
\pgfpathlineto{\pgfqpoint{0.625000in}{4.325217in}}%
\pgfusepath{stroke}%
\end{pgfscope}%
\begin{pgfscope}%
\pgfpathrectangle{\pgfqpoint{0.625000in}{0.550000in}}{\pgfqpoint{3.875000in}{3.850000in}} %
\pgfusepath{clip}%
\pgfsetbuttcap%
\pgfsetroundjoin%
\pgfsetlinewidth{0.250937pt}%
\definecolor{currentstroke}{rgb}{0.000000,0.000000,0.000000}%
\pgfsetstrokecolor{currentstroke}%
\pgfsetdash{}{0pt}%
\pgfpathmoveto{\pgfqpoint{0.634712in}{1.180320in}}%
\pgfpathlineto{\pgfqpoint{0.630654in}{1.186842in}}%
\pgfpathlineto{\pgfqpoint{0.628817in}{1.196491in}}%
\pgfpathlineto{\pgfqpoint{0.634712in}{1.205349in}}%
\pgfpathlineto{\pgfqpoint{0.644424in}{1.202822in}}%
\pgfpathlineto{\pgfqpoint{0.650343in}{1.196491in}}%
\pgfpathlineto{\pgfqpoint{0.648699in}{1.186842in}}%
\pgfpathlineto{\pgfqpoint{0.644424in}{1.183374in}}%
\pgfpathlineto{\pgfqpoint{0.634712in}{1.180320in}}%
\pgfusepath{stroke}%
\end{pgfscope}%
\begin{pgfscope}%
\pgfpathrectangle{\pgfqpoint{0.625000in}{0.550000in}}{\pgfqpoint{3.875000in}{3.850000in}} %
\pgfusepath{clip}%
\pgfsetbuttcap%
\pgfsetroundjoin%
\pgfsetlinewidth{0.250937pt}%
\definecolor{currentstroke}{rgb}{0.000000,0.000000,0.000000}%
\pgfsetstrokecolor{currentstroke}%
\pgfsetdash{}{0pt}%
\pgfpathmoveto{\pgfqpoint{0.634712in}{3.754300in}}%
\pgfpathlineto{\pgfqpoint{0.628817in}{3.763158in}}%
\pgfpathlineto{\pgfqpoint{0.630654in}{3.772807in}}%
\pgfpathlineto{\pgfqpoint{0.634712in}{3.779329in}}%
\pgfpathlineto{\pgfqpoint{0.644424in}{3.776275in}}%
\pgfpathlineto{\pgfqpoint{0.648699in}{3.772807in}}%
\pgfpathlineto{\pgfqpoint{0.650343in}{3.763158in}}%
\pgfpathlineto{\pgfqpoint{0.644424in}{3.756827in}}%
\pgfpathlineto{\pgfqpoint{0.634712in}{3.754300in}}%
\pgfusepath{stroke}%
\end{pgfscope}%
\begin{pgfscope}%
\pgfpathrectangle{\pgfqpoint{0.625000in}{0.550000in}}{\pgfqpoint{3.875000in}{3.850000in}} %
\pgfusepath{clip}%
\pgfsetbuttcap%
\pgfsetroundjoin%
\pgfsetlinewidth{0.250937pt}%
\definecolor{currentstroke}{rgb}{0.000000,0.000000,0.000000}%
\pgfsetstrokecolor{currentstroke}%
\pgfsetdash{}{0pt}%
\pgfpathmoveto{\pgfqpoint{0.625000in}{0.634827in}}%
\pgfpathlineto{\pgfqpoint{0.632874in}{0.627193in}}%
\pgfpathlineto{\pgfqpoint{0.625000in}{0.619581in}}%
\pgfusepath{stroke}%
\end{pgfscope}%
\begin{pgfscope}%
\pgfpathrectangle{\pgfqpoint{0.625000in}{0.550000in}}{\pgfqpoint{3.875000in}{3.850000in}} %
\pgfusepath{clip}%
\pgfsetbuttcap%
\pgfsetroundjoin%
\pgfsetlinewidth{0.250937pt}%
\definecolor{currentstroke}{rgb}{0.000000,0.000000,0.000000}%
\pgfsetstrokecolor{currentstroke}%
\pgfsetdash{}{0pt}%
\pgfpathmoveto{\pgfqpoint{0.625000in}{0.788807in}}%
\pgfpathlineto{\pgfqpoint{0.632820in}{0.781579in}}%
\pgfpathlineto{\pgfqpoint{0.625000in}{0.774029in}}%
\pgfusepath{stroke}%
\end{pgfscope}%
\begin{pgfscope}%
\pgfpathrectangle{\pgfqpoint{0.625000in}{0.550000in}}{\pgfqpoint{3.875000in}{3.850000in}} %
\pgfusepath{clip}%
\pgfsetbuttcap%
\pgfsetroundjoin%
\pgfsetlinewidth{0.250937pt}%
\definecolor{currentstroke}{rgb}{0.000000,0.000000,0.000000}%
\pgfsetstrokecolor{currentstroke}%
\pgfsetdash{}{0pt}%
\pgfpathmoveto{\pgfqpoint{0.625000in}{0.943482in}}%
\pgfpathlineto{\pgfqpoint{0.634516in}{0.935965in}}%
\pgfpathlineto{\pgfqpoint{0.625000in}{0.928380in}}%
\pgfusepath{stroke}%
\end{pgfscope}%
\begin{pgfscope}%
\pgfpathrectangle{\pgfqpoint{0.625000in}{0.550000in}}{\pgfqpoint{3.875000in}{3.850000in}} %
\pgfusepath{clip}%
\pgfsetbuttcap%
\pgfsetroundjoin%
\pgfsetlinewidth{0.250937pt}%
\definecolor{currentstroke}{rgb}{0.000000,0.000000,0.000000}%
\pgfsetstrokecolor{currentstroke}%
\pgfsetdash{}{0pt}%
\pgfpathmoveto{\pgfqpoint{0.625000in}{1.097856in}}%
\pgfpathlineto{\pgfqpoint{0.632826in}{1.090351in}}%
\pgfpathlineto{\pgfqpoint{0.625000in}{1.082874in}}%
\pgfusepath{stroke}%
\end{pgfscope}%
\begin{pgfscope}%
\pgfpathrectangle{\pgfqpoint{0.625000in}{0.550000in}}{\pgfqpoint{3.875000in}{3.850000in}} %
\pgfusepath{clip}%
\pgfsetbuttcap%
\pgfsetroundjoin%
\pgfsetlinewidth{0.250937pt}%
\definecolor{currentstroke}{rgb}{0.000000,0.000000,0.000000}%
\pgfsetstrokecolor{currentstroke}%
\pgfsetdash{}{0pt}%
\pgfpathmoveto{\pgfqpoint{0.625000in}{1.252468in}}%
\pgfpathlineto{\pgfqpoint{0.632833in}{1.244737in}}%
\pgfpathlineto{\pgfqpoint{0.625000in}{1.237219in}}%
\pgfusepath{stroke}%
\end{pgfscope}%
\begin{pgfscope}%
\pgfpathrectangle{\pgfqpoint{0.625000in}{0.550000in}}{\pgfqpoint{3.875000in}{3.850000in}} %
\pgfusepath{clip}%
\pgfsetbuttcap%
\pgfsetroundjoin%
\pgfsetlinewidth{0.250937pt}%
\definecolor{currentstroke}{rgb}{0.000000,0.000000,0.000000}%
\pgfsetstrokecolor{currentstroke}%
\pgfsetdash{}{0pt}%
\pgfpathmoveto{\pgfqpoint{0.625000in}{1.406520in}}%
\pgfpathlineto{\pgfqpoint{0.632776in}{1.399123in}}%
\pgfpathlineto{\pgfqpoint{0.625000in}{1.391722in}}%
\pgfusepath{stroke}%
\end{pgfscope}%
\begin{pgfscope}%
\pgfpathrectangle{\pgfqpoint{0.625000in}{0.550000in}}{\pgfqpoint{3.875000in}{3.850000in}} %
\pgfusepath{clip}%
\pgfsetbuttcap%
\pgfsetroundjoin%
\pgfsetlinewidth{0.250937pt}%
\definecolor{currentstroke}{rgb}{0.000000,0.000000,0.000000}%
\pgfsetstrokecolor{currentstroke}%
\pgfsetdash{}{0pt}%
\pgfpathmoveto{\pgfqpoint{0.625000in}{1.560893in}}%
\pgfpathlineto{\pgfqpoint{0.632750in}{1.553509in}}%
\pgfpathlineto{\pgfqpoint{0.625000in}{1.545885in}}%
\pgfusepath{stroke}%
\end{pgfscope}%
\begin{pgfscope}%
\pgfpathrectangle{\pgfqpoint{0.625000in}{0.550000in}}{\pgfqpoint{3.875000in}{3.850000in}} %
\pgfusepath{clip}%
\pgfsetbuttcap%
\pgfsetroundjoin%
\pgfsetlinewidth{0.250937pt}%
\definecolor{currentstroke}{rgb}{0.000000,0.000000,0.000000}%
\pgfsetstrokecolor{currentstroke}%
\pgfsetdash{}{0pt}%
\pgfpathmoveto{\pgfqpoint{0.625000in}{1.715269in}}%
\pgfpathlineto{\pgfqpoint{0.634501in}{1.707895in}}%
\pgfpathlineto{\pgfqpoint{0.625000in}{1.700474in}}%
\pgfusepath{stroke}%
\end{pgfscope}%
\begin{pgfscope}%
\pgfpathrectangle{\pgfqpoint{0.625000in}{0.550000in}}{\pgfqpoint{3.875000in}{3.850000in}} %
\pgfusepath{clip}%
\pgfsetbuttcap%
\pgfsetroundjoin%
\pgfsetlinewidth{0.250937pt}%
\definecolor{currentstroke}{rgb}{0.000000,0.000000,0.000000}%
\pgfsetstrokecolor{currentstroke}%
\pgfsetdash{}{0pt}%
\pgfpathmoveto{\pgfqpoint{0.625000in}{1.869706in}}%
\pgfpathlineto{\pgfqpoint{0.632706in}{1.862281in}}%
\pgfpathlineto{\pgfqpoint{0.625000in}{1.855009in}}%
\pgfusepath{stroke}%
\end{pgfscope}%
\begin{pgfscope}%
\pgfpathrectangle{\pgfqpoint{0.625000in}{0.550000in}}{\pgfqpoint{3.875000in}{3.850000in}} %
\pgfusepath{clip}%
\pgfsetbuttcap%
\pgfsetroundjoin%
\pgfsetlinewidth{0.250937pt}%
\definecolor{currentstroke}{rgb}{0.000000,0.000000,0.000000}%
\pgfsetstrokecolor{currentstroke}%
\pgfsetdash{}{0pt}%
\pgfpathmoveto{\pgfqpoint{0.625000in}{2.024342in}}%
\pgfpathlineto{\pgfqpoint{0.632683in}{2.016667in}}%
\pgfpathlineto{\pgfqpoint{0.625000in}{2.009054in}}%
\pgfusepath{stroke}%
\end{pgfscope}%
\begin{pgfscope}%
\pgfpathrectangle{\pgfqpoint{0.625000in}{0.550000in}}{\pgfqpoint{3.875000in}{3.850000in}} %
\pgfusepath{clip}%
\pgfsetbuttcap%
\pgfsetroundjoin%
\pgfsetlinewidth{0.250937pt}%
\definecolor{currentstroke}{rgb}{0.000000,0.000000,0.000000}%
\pgfsetstrokecolor{currentstroke}%
\pgfsetdash{}{0pt}%
\pgfpathmoveto{\pgfqpoint{0.625000in}{2.178345in}}%
\pgfpathlineto{\pgfqpoint{0.632712in}{2.171053in}}%
\pgfpathlineto{\pgfqpoint{0.625000in}{2.163552in}}%
\pgfusepath{stroke}%
\end{pgfscope}%
\begin{pgfscope}%
\pgfpathrectangle{\pgfqpoint{0.625000in}{0.550000in}}{\pgfqpoint{3.875000in}{3.850000in}} %
\pgfusepath{clip}%
\pgfsetbuttcap%
\pgfsetroundjoin%
\pgfsetlinewidth{0.250937pt}%
\definecolor{currentstroke}{rgb}{0.000000,0.000000,0.000000}%
\pgfsetstrokecolor{currentstroke}%
\pgfsetdash{}{0pt}%
\pgfpathmoveto{\pgfqpoint{0.625000in}{2.332686in}}%
\pgfpathlineto{\pgfqpoint{0.632738in}{2.325439in}}%
\pgfpathlineto{\pgfqpoint{0.625000in}{2.317683in}}%
\pgfusepath{stroke}%
\end{pgfscope}%
\begin{pgfscope}%
\pgfpathrectangle{\pgfqpoint{0.625000in}{0.550000in}}{\pgfqpoint{3.875000in}{3.850000in}} %
\pgfusepath{clip}%
\pgfsetbuttcap%
\pgfsetroundjoin%
\pgfsetlinewidth{0.250937pt}%
\definecolor{currentstroke}{rgb}{0.000000,0.000000,0.000000}%
\pgfsetstrokecolor{currentstroke}%
\pgfsetdash{}{0pt}%
\pgfpathmoveto{\pgfqpoint{0.625000in}{2.488048in}}%
\pgfpathlineto{\pgfqpoint{0.626338in}{2.489474in}}%
\pgfpathlineto{\pgfqpoint{0.627069in}{2.499123in}}%
\pgfpathlineto{\pgfqpoint{0.629341in}{2.508772in}}%
\pgfpathlineto{\pgfqpoint{0.632240in}{2.518421in}}%
\pgfpathlineto{\pgfqpoint{0.634712in}{2.525773in}}%
\pgfpathlineto{\pgfqpoint{0.636196in}{2.528070in}}%
\pgfpathlineto{\pgfqpoint{0.641254in}{2.537719in}}%
\pgfpathlineto{\pgfqpoint{0.644424in}{2.543874in}}%
\pgfpathlineto{\pgfqpoint{0.647364in}{2.547368in}}%
\pgfpathlineto{\pgfqpoint{0.654135in}{2.556632in}}%
\pgfpathlineto{\pgfqpoint{0.654575in}{2.557018in}}%
\pgfpathlineto{\pgfqpoint{0.663625in}{2.566667in}}%
\pgfpathlineto{\pgfqpoint{0.663847in}{2.566941in}}%
\pgfpathlineto{\pgfqpoint{0.673559in}{2.575408in}}%
\pgfpathlineto{\pgfqpoint{0.675043in}{2.576316in}}%
\pgfpathlineto{\pgfqpoint{0.683271in}{2.582443in}}%
\pgfpathlineto{\pgfqpoint{0.689785in}{2.585965in}}%
\pgfpathlineto{\pgfqpoint{0.692982in}{2.588049in}}%
\pgfpathlineto{\pgfqpoint{0.702694in}{2.592721in}}%
\pgfpathlineto{\pgfqpoint{0.710737in}{2.595614in}}%
\pgfpathlineto{\pgfqpoint{0.712406in}{2.596334in}}%
\pgfpathlineto{\pgfqpoint{0.722118in}{2.599307in}}%
\pgfpathlineto{\pgfqpoint{0.731830in}{2.601389in}}%
\pgfpathlineto{\pgfqpoint{0.741541in}{2.602696in}}%
\pgfpathlineto{\pgfqpoint{0.751253in}{2.603284in}}%
\pgfpathlineto{\pgfqpoint{0.760965in}{2.603168in}}%
\pgfpathlineto{\pgfqpoint{0.770677in}{2.602334in}}%
\pgfpathlineto{\pgfqpoint{0.780388in}{2.600752in}}%
\pgfpathlineto{\pgfqpoint{0.790100in}{2.598380in}}%
\pgfpathlineto{\pgfqpoint{0.798562in}{2.595614in}}%
\pgfpathlineto{\pgfqpoint{0.799812in}{2.595174in}}%
\pgfpathlineto{\pgfqpoint{0.809524in}{2.591108in}}%
\pgfpathlineto{\pgfqpoint{0.819236in}{2.586027in}}%
\pgfpathlineto{\pgfqpoint{0.819343in}{2.585965in}}%
\pgfpathlineto{\pgfqpoint{0.828947in}{2.579781in}}%
\pgfpathlineto{\pgfqpoint{0.833681in}{2.576316in}}%
\pgfpathlineto{\pgfqpoint{0.838659in}{2.572145in}}%
\pgfpathlineto{\pgfqpoint{0.844684in}{2.566667in}}%
\pgfpathlineto{\pgfqpoint{0.848371in}{2.562737in}}%
\pgfpathlineto{\pgfqpoint{0.853492in}{2.557018in}}%
\pgfpathlineto{\pgfqpoint{0.858083in}{2.550859in}}%
\pgfpathlineto{\pgfqpoint{0.860633in}{2.547368in}}%
\pgfpathlineto{\pgfqpoint{0.866439in}{2.537719in}}%
\pgfpathlineto{\pgfqpoint{0.867794in}{2.534927in}}%
\pgfpathlineto{\pgfqpoint{0.871177in}{2.528070in}}%
\pgfpathlineto{\pgfqpoint{0.874878in}{2.518421in}}%
\pgfpathlineto{\pgfqpoint{0.877506in}{2.509139in}}%
\pgfpathlineto{\pgfqpoint{0.877615in}{2.508772in}}%
\pgfpathlineto{\pgfqpoint{0.879640in}{2.499123in}}%
\pgfpathlineto{\pgfqpoint{0.880826in}{2.489474in}}%
\pgfpathlineto{\pgfqpoint{0.881216in}{2.479825in}}%
\pgfpathlineto{\pgfqpoint{0.880826in}{2.470175in}}%
\pgfpathlineto{\pgfqpoint{0.879640in}{2.460526in}}%
\pgfpathlineto{\pgfqpoint{0.877615in}{2.450877in}}%
\pgfpathlineto{\pgfqpoint{0.877506in}{2.450510in}}%
\pgfpathlineto{\pgfqpoint{0.874878in}{2.441228in}}%
\pgfpathlineto{\pgfqpoint{0.871177in}{2.431579in}}%
\pgfpathlineto{\pgfqpoint{0.867794in}{2.424722in}}%
\pgfpathlineto{\pgfqpoint{0.866439in}{2.421930in}}%
\pgfpathlineto{\pgfqpoint{0.860633in}{2.412281in}}%
\pgfpathlineto{\pgfqpoint{0.858083in}{2.408791in}}%
\pgfpathlineto{\pgfqpoint{0.853492in}{2.402632in}}%
\pgfpathlineto{\pgfqpoint{0.848371in}{2.396912in}}%
\pgfpathlineto{\pgfqpoint{0.844684in}{2.392982in}}%
\pgfpathlineto{\pgfqpoint{0.838659in}{2.387505in}}%
\pgfpathlineto{\pgfqpoint{0.833681in}{2.383333in}}%
\pgfpathlineto{\pgfqpoint{0.828947in}{2.379868in}}%
\pgfpathlineto{\pgfqpoint{0.819343in}{2.373684in}}%
\pgfpathlineto{\pgfqpoint{0.819236in}{2.373623in}}%
\pgfpathlineto{\pgfqpoint{0.809524in}{2.368541in}}%
\pgfpathlineto{\pgfqpoint{0.799812in}{2.364475in}}%
\pgfpathlineto{\pgfqpoint{0.798562in}{2.364035in}}%
\pgfpathlineto{\pgfqpoint{0.790100in}{2.361269in}}%
\pgfpathlineto{\pgfqpoint{0.780388in}{2.358897in}}%
\pgfpathlineto{\pgfqpoint{0.770677in}{2.357315in}}%
\pgfpathlineto{\pgfqpoint{0.760965in}{2.356482in}}%
\pgfpathlineto{\pgfqpoint{0.751253in}{2.356365in}}%
\pgfpathlineto{\pgfqpoint{0.741541in}{2.356953in}}%
\pgfpathlineto{\pgfqpoint{0.731830in}{2.358260in}}%
\pgfpathlineto{\pgfqpoint{0.722118in}{2.360342in}}%
\pgfpathlineto{\pgfqpoint{0.712406in}{2.363315in}}%
\pgfpathlineto{\pgfqpoint{0.710737in}{2.364035in}}%
\pgfpathlineto{\pgfqpoint{0.702694in}{2.366928in}}%
\pgfpathlineto{\pgfqpoint{0.692982in}{2.371600in}}%
\pgfpathlineto{\pgfqpoint{0.689785in}{2.373684in}}%
\pgfpathlineto{\pgfqpoint{0.683271in}{2.377206in}}%
\pgfpathlineto{\pgfqpoint{0.675043in}{2.383333in}}%
\pgfpathlineto{\pgfqpoint{0.673559in}{2.384241in}}%
\pgfpathlineto{\pgfqpoint{0.663847in}{2.392708in}}%
\pgfpathlineto{\pgfqpoint{0.663625in}{2.392982in}}%
\pgfpathlineto{\pgfqpoint{0.654575in}{2.402632in}}%
\pgfpathlineto{\pgfqpoint{0.654135in}{2.403017in}}%
\pgfpathlineto{\pgfqpoint{0.647364in}{2.412281in}}%
\pgfpathlineto{\pgfqpoint{0.644424in}{2.415775in}}%
\pgfpathlineto{\pgfqpoint{0.641254in}{2.421930in}}%
\pgfpathlineto{\pgfqpoint{0.636196in}{2.431579in}}%
\pgfpathlineto{\pgfqpoint{0.634712in}{2.433876in}}%
\pgfpathlineto{\pgfqpoint{0.632240in}{2.441228in}}%
\pgfpathlineto{\pgfqpoint{0.629341in}{2.450877in}}%
\pgfpathlineto{\pgfqpoint{0.627074in}{2.460526in}}%
\pgfpathlineto{\pgfqpoint{0.626338in}{2.470175in}}%
\pgfpathlineto{\pgfqpoint{0.625000in}{2.471602in}}%
\pgfusepath{stroke}%
\end{pgfscope}%
\begin{pgfscope}%
\pgfpathrectangle{\pgfqpoint{0.625000in}{0.550000in}}{\pgfqpoint{3.875000in}{3.850000in}} %
\pgfusepath{clip}%
\pgfsetbuttcap%
\pgfsetroundjoin%
\pgfsetlinewidth{0.250937pt}%
\definecolor{currentstroke}{rgb}{0.000000,0.000000,0.000000}%
\pgfsetstrokecolor{currentstroke}%
\pgfsetdash{}{0pt}%
\pgfpathmoveto{\pgfqpoint{0.625000in}{2.641963in}}%
\pgfpathlineto{\pgfqpoint{0.632734in}{2.634211in}}%
\pgfpathlineto{\pgfqpoint{0.625000in}{2.626967in}}%
\pgfusepath{stroke}%
\end{pgfscope}%
\begin{pgfscope}%
\pgfpathrectangle{\pgfqpoint{0.625000in}{0.550000in}}{\pgfqpoint{3.875000in}{3.850000in}} %
\pgfusepath{clip}%
\pgfsetbuttcap%
\pgfsetroundjoin%
\pgfsetlinewidth{0.250937pt}%
\definecolor{currentstroke}{rgb}{0.000000,0.000000,0.000000}%
\pgfsetstrokecolor{currentstroke}%
\pgfsetdash{}{0pt}%
\pgfpathmoveto{\pgfqpoint{0.625000in}{2.796084in}}%
\pgfpathlineto{\pgfqpoint{0.632699in}{2.788596in}}%
\pgfpathlineto{\pgfqpoint{0.625000in}{2.781318in}}%
\pgfusepath{stroke}%
\end{pgfscope}%
\begin{pgfscope}%
\pgfpathrectangle{\pgfqpoint{0.625000in}{0.550000in}}{\pgfqpoint{3.875000in}{3.850000in}} %
\pgfusepath{clip}%
\pgfsetbuttcap%
\pgfsetroundjoin%
\pgfsetlinewidth{0.250937pt}%
\definecolor{currentstroke}{rgb}{0.000000,0.000000,0.000000}%
\pgfsetstrokecolor{currentstroke}%
\pgfsetdash{}{0pt}%
\pgfpathmoveto{\pgfqpoint{0.625000in}{2.950560in}}%
\pgfpathlineto{\pgfqpoint{0.632648in}{2.942982in}}%
\pgfpathlineto{\pgfqpoint{0.625000in}{2.935341in}}%
\pgfusepath{stroke}%
\end{pgfscope}%
\begin{pgfscope}%
\pgfpathrectangle{\pgfqpoint{0.625000in}{0.550000in}}{\pgfqpoint{3.875000in}{3.850000in}} %
\pgfusepath{clip}%
\pgfsetbuttcap%
\pgfsetroundjoin%
\pgfsetlinewidth{0.250937pt}%
\definecolor{currentstroke}{rgb}{0.000000,0.000000,0.000000}%
\pgfsetstrokecolor{currentstroke}%
\pgfsetdash{}{0pt}%
\pgfpathmoveto{\pgfqpoint{0.625000in}{3.104609in}}%
\pgfpathlineto{\pgfqpoint{0.632679in}{3.097368in}}%
\pgfpathlineto{\pgfqpoint{0.625000in}{3.089973in}}%
\pgfusepath{stroke}%
\end{pgfscope}%
\begin{pgfscope}%
\pgfpathrectangle{\pgfqpoint{0.625000in}{0.550000in}}{\pgfqpoint{3.875000in}{3.850000in}} %
\pgfusepath{clip}%
\pgfsetbuttcap%
\pgfsetroundjoin%
\pgfsetlinewidth{0.250937pt}%
\definecolor{currentstroke}{rgb}{0.000000,0.000000,0.000000}%
\pgfsetstrokecolor{currentstroke}%
\pgfsetdash{}{0pt}%
\pgfpathmoveto{\pgfqpoint{0.625000in}{3.259068in}}%
\pgfpathlineto{\pgfqpoint{0.634488in}{3.251754in}}%
\pgfpathlineto{\pgfqpoint{0.625000in}{3.244489in}}%
\pgfusepath{stroke}%
\end{pgfscope}%
\begin{pgfscope}%
\pgfpathrectangle{\pgfqpoint{0.625000in}{0.550000in}}{\pgfqpoint{3.875000in}{3.850000in}} %
\pgfusepath{clip}%
\pgfsetbuttcap%
\pgfsetroundjoin%
\pgfsetlinewidth{0.250937pt}%
\definecolor{currentstroke}{rgb}{0.000000,0.000000,0.000000}%
\pgfsetstrokecolor{currentstroke}%
\pgfsetdash{}{0pt}%
\pgfpathmoveto{\pgfqpoint{0.625000in}{3.413694in}}%
\pgfpathlineto{\pgfqpoint{0.632682in}{3.406140in}}%
\pgfpathlineto{\pgfqpoint{0.625000in}{3.398832in}}%
\pgfusepath{stroke}%
\end{pgfscope}%
\begin{pgfscope}%
\pgfpathrectangle{\pgfqpoint{0.625000in}{0.550000in}}{\pgfqpoint{3.875000in}{3.850000in}} %
\pgfusepath{clip}%
\pgfsetbuttcap%
\pgfsetroundjoin%
\pgfsetlinewidth{0.250937pt}%
\definecolor{currentstroke}{rgb}{0.000000,0.000000,0.000000}%
\pgfsetstrokecolor{currentstroke}%
\pgfsetdash{}{0pt}%
\pgfpathmoveto{\pgfqpoint{0.625000in}{3.567838in}}%
\pgfpathlineto{\pgfqpoint{0.632695in}{3.560526in}}%
\pgfpathlineto{\pgfqpoint{0.625000in}{3.553219in}}%
\pgfusepath{stroke}%
\end{pgfscope}%
\begin{pgfscope}%
\pgfpathrectangle{\pgfqpoint{0.625000in}{0.550000in}}{\pgfqpoint{3.875000in}{3.850000in}} %
\pgfusepath{clip}%
\pgfsetbuttcap%
\pgfsetroundjoin%
\pgfsetlinewidth{0.250937pt}%
\definecolor{currentstroke}{rgb}{0.000000,0.000000,0.000000}%
\pgfsetstrokecolor{currentstroke}%
\pgfsetdash{}{0pt}%
\pgfpathmoveto{\pgfqpoint{0.625000in}{3.722300in}}%
\pgfpathlineto{\pgfqpoint{0.632715in}{3.714912in}}%
\pgfpathlineto{\pgfqpoint{0.625000in}{3.707302in}}%
\pgfusepath{stroke}%
\end{pgfscope}%
\begin{pgfscope}%
\pgfpathrectangle{\pgfqpoint{0.625000in}{0.550000in}}{\pgfqpoint{3.875000in}{3.850000in}} %
\pgfusepath{clip}%
\pgfsetbuttcap%
\pgfsetroundjoin%
\pgfsetlinewidth{0.250937pt}%
\definecolor{currentstroke}{rgb}{0.000000,0.000000,0.000000}%
\pgfsetstrokecolor{currentstroke}%
\pgfsetdash{}{0pt}%
\pgfpathmoveto{\pgfqpoint{0.625000in}{3.876557in}}%
\pgfpathlineto{\pgfqpoint{0.632628in}{3.869298in}}%
\pgfpathlineto{\pgfqpoint{0.625000in}{3.862010in}}%
\pgfusepath{stroke}%
\end{pgfscope}%
\begin{pgfscope}%
\pgfpathrectangle{\pgfqpoint{0.625000in}{0.550000in}}{\pgfqpoint{3.875000in}{3.850000in}} %
\pgfusepath{clip}%
\pgfsetbuttcap%
\pgfsetroundjoin%
\pgfsetlinewidth{0.250937pt}%
\definecolor{currentstroke}{rgb}{0.000000,0.000000,0.000000}%
\pgfsetstrokecolor{currentstroke}%
\pgfsetdash{}{0pt}%
\pgfpathmoveto{\pgfqpoint{0.625000in}{4.031076in}}%
\pgfpathlineto{\pgfqpoint{0.634493in}{4.023684in}}%
\pgfpathlineto{\pgfqpoint{0.625000in}{4.016365in}}%
\pgfusepath{stroke}%
\end{pgfscope}%
\begin{pgfscope}%
\pgfpathrectangle{\pgfqpoint{0.625000in}{0.550000in}}{\pgfqpoint{3.875000in}{3.850000in}} %
\pgfusepath{clip}%
\pgfsetbuttcap%
\pgfsetroundjoin%
\pgfsetlinewidth{0.250937pt}%
\definecolor{currentstroke}{rgb}{0.000000,0.000000,0.000000}%
\pgfsetstrokecolor{currentstroke}%
\pgfsetdash{}{0pt}%
\pgfpathmoveto{\pgfqpoint{0.625000in}{4.185404in}}%
\pgfpathlineto{\pgfqpoint{0.632618in}{4.178070in}}%
\pgfpathlineto{\pgfqpoint{0.625000in}{4.171080in}}%
\pgfusepath{stroke}%
\end{pgfscope}%
\begin{pgfscope}%
\pgfpathrectangle{\pgfqpoint{0.625000in}{0.550000in}}{\pgfqpoint{3.875000in}{3.850000in}} %
\pgfusepath{clip}%
\pgfsetbuttcap%
\pgfsetroundjoin%
\pgfsetlinewidth{0.250937pt}%
\definecolor{currentstroke}{rgb}{0.000000,0.000000,0.000000}%
\pgfsetstrokecolor{currentstroke}%
\pgfsetdash{}{0pt}%
\pgfpathmoveto{\pgfqpoint{0.625000in}{4.339759in}}%
\pgfpathlineto{\pgfqpoint{0.632586in}{4.332456in}}%
\pgfpathlineto{\pgfqpoint{0.625000in}{4.325129in}}%
\pgfusepath{stroke}%
\end{pgfscope}%
\begin{pgfscope}%
\pgfpathrectangle{\pgfqpoint{0.625000in}{0.550000in}}{\pgfqpoint{3.875000in}{3.850000in}} %
\pgfusepath{clip}%
\pgfsetbuttcap%
\pgfsetroundjoin%
\pgfsetlinewidth{0.250937pt}%
\definecolor{currentstroke}{rgb}{0.000000,0.000000,0.000000}%
\pgfsetstrokecolor{currentstroke}%
\pgfsetdash{}{0pt}%
\pgfpathmoveto{\pgfqpoint{0.634712in}{1.179986in}}%
\pgfpathlineto{\pgfqpoint{0.630446in}{1.186842in}}%
\pgfpathlineto{\pgfqpoint{0.628642in}{1.196491in}}%
\pgfpathlineto{\pgfqpoint{0.634712in}{1.205611in}}%
\pgfpathlineto{\pgfqpoint{0.644424in}{1.203843in}}%
\pgfpathlineto{\pgfqpoint{0.651297in}{1.196491in}}%
\pgfpathlineto{\pgfqpoint{0.649843in}{1.186842in}}%
\pgfpathlineto{\pgfqpoint{0.644424in}{1.182446in}}%
\pgfpathlineto{\pgfqpoint{0.634712in}{1.179986in}}%
\pgfusepath{stroke}%
\end{pgfscope}%
\begin{pgfscope}%
\pgfpathrectangle{\pgfqpoint{0.625000in}{0.550000in}}{\pgfqpoint{3.875000in}{3.850000in}} %
\pgfusepath{clip}%
\pgfsetbuttcap%
\pgfsetroundjoin%
\pgfsetlinewidth{0.250937pt}%
\definecolor{currentstroke}{rgb}{0.000000,0.000000,0.000000}%
\pgfsetstrokecolor{currentstroke}%
\pgfsetdash{}{0pt}%
\pgfpathmoveto{\pgfqpoint{0.634712in}{3.754038in}}%
\pgfpathlineto{\pgfqpoint{0.628642in}{3.763158in}}%
\pgfpathlineto{\pgfqpoint{0.630446in}{3.772807in}}%
\pgfpathlineto{\pgfqpoint{0.634712in}{3.779663in}}%
\pgfpathlineto{\pgfqpoint{0.644424in}{3.777204in}}%
\pgfpathlineto{\pgfqpoint{0.649843in}{3.772807in}}%
\pgfpathlineto{\pgfqpoint{0.651297in}{3.763158in}}%
\pgfpathlineto{\pgfqpoint{0.644424in}{3.755806in}}%
\pgfpathlineto{\pgfqpoint{0.634712in}{3.754038in}}%
\pgfusepath{stroke}%
\end{pgfscope}%
\begin{pgfscope}%
\pgfpathrectangle{\pgfqpoint{0.625000in}{0.550000in}}{\pgfqpoint{3.875000in}{3.850000in}} %
\pgfusepath{clip}%
\pgfsetbuttcap%
\pgfsetroundjoin%
\pgfsetlinewidth{0.250937pt}%
\definecolor{currentstroke}{rgb}{0.000000,0.000000,0.000000}%
\pgfsetstrokecolor{currentstroke}%
\pgfsetdash{}{0pt}%
\pgfpathmoveto{\pgfqpoint{0.625000in}{0.634903in}}%
\pgfpathlineto{\pgfqpoint{0.632952in}{0.627193in}}%
\pgfpathlineto{\pgfqpoint{0.625000in}{0.619505in}}%
\pgfusepath{stroke}%
\end{pgfscope}%
\begin{pgfscope}%
\pgfpathrectangle{\pgfqpoint{0.625000in}{0.550000in}}{\pgfqpoint{3.875000in}{3.850000in}} %
\pgfusepath{clip}%
\pgfsetbuttcap%
\pgfsetroundjoin%
\pgfsetlinewidth{0.250937pt}%
\definecolor{currentstroke}{rgb}{0.000000,0.000000,0.000000}%
\pgfsetstrokecolor{currentstroke}%
\pgfsetdash{}{0pt}%
\pgfpathmoveto{\pgfqpoint{0.625000in}{0.788884in}}%
\pgfpathlineto{\pgfqpoint{0.632903in}{0.781579in}}%
\pgfpathlineto{\pgfqpoint{0.625000in}{0.773949in}}%
\pgfusepath{stroke}%
\end{pgfscope}%
\begin{pgfscope}%
\pgfpathrectangle{\pgfqpoint{0.625000in}{0.550000in}}{\pgfqpoint{3.875000in}{3.850000in}} %
\pgfusepath{clip}%
\pgfsetbuttcap%
\pgfsetroundjoin%
\pgfsetlinewidth{0.250937pt}%
\definecolor{currentstroke}{rgb}{0.000000,0.000000,0.000000}%
\pgfsetstrokecolor{currentstroke}%
\pgfsetdash{}{0pt}%
\pgfpathmoveto{\pgfqpoint{0.625000in}{0.943559in}}%
\pgfpathlineto{\pgfqpoint{0.634613in}{0.935965in}}%
\pgfpathlineto{\pgfqpoint{0.625000in}{0.928303in}}%
\pgfusepath{stroke}%
\end{pgfscope}%
\begin{pgfscope}%
\pgfpathrectangle{\pgfqpoint{0.625000in}{0.550000in}}{\pgfqpoint{3.875000in}{3.850000in}} %
\pgfusepath{clip}%
\pgfsetbuttcap%
\pgfsetroundjoin%
\pgfsetlinewidth{0.250937pt}%
\definecolor{currentstroke}{rgb}{0.000000,0.000000,0.000000}%
\pgfsetstrokecolor{currentstroke}%
\pgfsetdash{}{0pt}%
\pgfpathmoveto{\pgfqpoint{0.625000in}{1.097935in}}%
\pgfpathlineto{\pgfqpoint{0.632908in}{1.090351in}}%
\pgfpathlineto{\pgfqpoint{0.625000in}{1.082795in}}%
\pgfusepath{stroke}%
\end{pgfscope}%
\begin{pgfscope}%
\pgfpathrectangle{\pgfqpoint{0.625000in}{0.550000in}}{\pgfqpoint{3.875000in}{3.850000in}} %
\pgfusepath{clip}%
\pgfsetbuttcap%
\pgfsetroundjoin%
\pgfsetlinewidth{0.250937pt}%
\definecolor{currentstroke}{rgb}{0.000000,0.000000,0.000000}%
\pgfsetstrokecolor{currentstroke}%
\pgfsetdash{}{0pt}%
\pgfpathmoveto{\pgfqpoint{0.625000in}{1.252551in}}%
\pgfpathlineto{\pgfqpoint{0.632917in}{1.244737in}}%
\pgfpathlineto{\pgfqpoint{0.625000in}{1.237138in}}%
\pgfusepath{stroke}%
\end{pgfscope}%
\begin{pgfscope}%
\pgfpathrectangle{\pgfqpoint{0.625000in}{0.550000in}}{\pgfqpoint{3.875000in}{3.850000in}} %
\pgfusepath{clip}%
\pgfsetbuttcap%
\pgfsetroundjoin%
\pgfsetlinewidth{0.250937pt}%
\definecolor{currentstroke}{rgb}{0.000000,0.000000,0.000000}%
\pgfsetstrokecolor{currentstroke}%
\pgfsetdash{}{0pt}%
\pgfpathmoveto{\pgfqpoint{0.625000in}{1.406601in}}%
\pgfpathlineto{\pgfqpoint{0.632861in}{1.399123in}}%
\pgfpathlineto{\pgfqpoint{0.625000in}{1.391640in}}%
\pgfusepath{stroke}%
\end{pgfscope}%
\begin{pgfscope}%
\pgfpathrectangle{\pgfqpoint{0.625000in}{0.550000in}}{\pgfqpoint{3.875000in}{3.850000in}} %
\pgfusepath{clip}%
\pgfsetbuttcap%
\pgfsetroundjoin%
\pgfsetlinewidth{0.250937pt}%
\definecolor{currentstroke}{rgb}{0.000000,0.000000,0.000000}%
\pgfsetstrokecolor{currentstroke}%
\pgfsetdash{}{0pt}%
\pgfpathmoveto{\pgfqpoint{0.625000in}{1.560973in}}%
\pgfpathlineto{\pgfqpoint{0.632834in}{1.553509in}}%
\pgfpathlineto{\pgfqpoint{0.625000in}{1.545803in}}%
\pgfusepath{stroke}%
\end{pgfscope}%
\begin{pgfscope}%
\pgfpathrectangle{\pgfqpoint{0.625000in}{0.550000in}}{\pgfqpoint{3.875000in}{3.850000in}} %
\pgfusepath{clip}%
\pgfsetbuttcap%
\pgfsetroundjoin%
\pgfsetlinewidth{0.250937pt}%
\definecolor{currentstroke}{rgb}{0.000000,0.000000,0.000000}%
\pgfsetstrokecolor{currentstroke}%
\pgfsetdash{}{0pt}%
\pgfpathmoveto{\pgfqpoint{0.625000in}{1.715350in}}%
\pgfpathlineto{\pgfqpoint{0.634606in}{1.707895in}}%
\pgfpathlineto{\pgfqpoint{0.625000in}{1.700392in}}%
\pgfusepath{stroke}%
\end{pgfscope}%
\begin{pgfscope}%
\pgfpathrectangle{\pgfqpoint{0.625000in}{0.550000in}}{\pgfqpoint{3.875000in}{3.850000in}} %
\pgfusepath{clip}%
\pgfsetbuttcap%
\pgfsetroundjoin%
\pgfsetlinewidth{0.250937pt}%
\definecolor{currentstroke}{rgb}{0.000000,0.000000,0.000000}%
\pgfsetstrokecolor{currentstroke}%
\pgfsetdash{}{0pt}%
\pgfpathmoveto{\pgfqpoint{0.625000in}{1.869789in}}%
\pgfpathlineto{\pgfqpoint{0.632792in}{1.862281in}}%
\pgfpathlineto{\pgfqpoint{0.625000in}{1.854928in}}%
\pgfusepath{stroke}%
\end{pgfscope}%
\begin{pgfscope}%
\pgfpathrectangle{\pgfqpoint{0.625000in}{0.550000in}}{\pgfqpoint{3.875000in}{3.850000in}} %
\pgfusepath{clip}%
\pgfsetbuttcap%
\pgfsetroundjoin%
\pgfsetlinewidth{0.250937pt}%
\definecolor{currentstroke}{rgb}{0.000000,0.000000,0.000000}%
\pgfsetstrokecolor{currentstroke}%
\pgfsetdash{}{0pt}%
\pgfpathmoveto{\pgfqpoint{0.625000in}{2.024430in}}%
\pgfpathlineto{\pgfqpoint{0.632770in}{2.016667in}}%
\pgfpathlineto{\pgfqpoint{0.625000in}{2.008967in}}%
\pgfusepath{stroke}%
\end{pgfscope}%
\begin{pgfscope}%
\pgfpathrectangle{\pgfqpoint{0.625000in}{0.550000in}}{\pgfqpoint{3.875000in}{3.850000in}} %
\pgfusepath{clip}%
\pgfsetbuttcap%
\pgfsetroundjoin%
\pgfsetlinewidth{0.250937pt}%
\definecolor{currentstroke}{rgb}{0.000000,0.000000,0.000000}%
\pgfsetstrokecolor{currentstroke}%
\pgfsetdash{}{0pt}%
\pgfpathmoveto{\pgfqpoint{0.625000in}{2.178427in}}%
\pgfpathlineto{\pgfqpoint{0.632799in}{2.171053in}}%
\pgfpathlineto{\pgfqpoint{0.625000in}{2.163467in}}%
\pgfusepath{stroke}%
\end{pgfscope}%
\begin{pgfscope}%
\pgfpathrectangle{\pgfqpoint{0.625000in}{0.550000in}}{\pgfqpoint{3.875000in}{3.850000in}} %
\pgfusepath{clip}%
\pgfsetbuttcap%
\pgfsetroundjoin%
\pgfsetlinewidth{0.250937pt}%
\definecolor{currentstroke}{rgb}{0.000000,0.000000,0.000000}%
\pgfsetstrokecolor{currentstroke}%
\pgfsetdash{}{0pt}%
\pgfpathmoveto{\pgfqpoint{0.625000in}{2.332769in}}%
\pgfpathlineto{\pgfqpoint{0.632826in}{2.325439in}}%
\pgfpathlineto{\pgfqpoint{0.625000in}{2.317595in}}%
\pgfusepath{stroke}%
\end{pgfscope}%
\begin{pgfscope}%
\pgfpathrectangle{\pgfqpoint{0.625000in}{0.550000in}}{\pgfqpoint{3.875000in}{3.850000in}} %
\pgfusepath{clip}%
\pgfsetbuttcap%
\pgfsetroundjoin%
\pgfsetlinewidth{0.250937pt}%
\definecolor{currentstroke}{rgb}{0.000000,0.000000,0.000000}%
\pgfsetstrokecolor{currentstroke}%
\pgfsetdash{}{0pt}%
\pgfpathmoveto{\pgfqpoint{0.625000in}{2.488131in}}%
\pgfpathlineto{\pgfqpoint{0.626259in}{2.489474in}}%
\pgfpathlineto{\pgfqpoint{0.626990in}{2.499123in}}%
\pgfpathlineto{\pgfqpoint{0.629187in}{2.508772in}}%
\pgfpathlineto{\pgfqpoint{0.631977in}{2.518421in}}%
\pgfpathlineto{\pgfqpoint{0.634712in}{2.526553in}}%
\pgfpathlineto{\pgfqpoint{0.635692in}{2.528070in}}%
\pgfpathlineto{\pgfqpoint{0.640585in}{2.537719in}}%
\pgfpathlineto{\pgfqpoint{0.644424in}{2.545174in}}%
\pgfpathlineto{\pgfqpoint{0.646271in}{2.547368in}}%
\pgfpathlineto{\pgfqpoint{0.653162in}{2.557018in}}%
\pgfpathlineto{\pgfqpoint{0.654135in}{2.558552in}}%
\pgfpathlineto{\pgfqpoint{0.661698in}{2.566667in}}%
\pgfpathlineto{\pgfqpoint{0.663847in}{2.569315in}}%
\pgfpathlineto{\pgfqpoint{0.671936in}{2.576316in}}%
\pgfpathlineto{\pgfqpoint{0.673559in}{2.577966in}}%
\pgfpathlineto{\pgfqpoint{0.683271in}{2.585151in}}%
\pgfpathlineto{\pgfqpoint{0.684777in}{2.585965in}}%
\pgfpathlineto{\pgfqpoint{0.692982in}{2.591313in}}%
\pgfpathlineto{\pgfqpoint{0.701795in}{2.595614in}}%
\pgfpathlineto{\pgfqpoint{0.702694in}{2.596137in}}%
\pgfpathlineto{\pgfqpoint{0.712406in}{2.600341in}}%
\pgfpathlineto{\pgfqpoint{0.722118in}{2.603511in}}%
\pgfpathlineto{\pgfqpoint{0.729475in}{2.605263in}}%
\pgfpathlineto{\pgfqpoint{0.731830in}{2.605926in}}%
\pgfpathlineto{\pgfqpoint{0.741541in}{2.607719in}}%
\pgfpathlineto{\pgfqpoint{0.751253in}{2.608760in}}%
\pgfpathlineto{\pgfqpoint{0.760965in}{2.609093in}}%
\pgfpathlineto{\pgfqpoint{0.770677in}{2.608733in}}%
\pgfpathlineto{\pgfqpoint{0.780388in}{2.607671in}}%
\pgfpathlineto{\pgfqpoint{0.790100in}{2.605881in}}%
\pgfpathlineto{\pgfqpoint{0.792611in}{2.605263in}}%
\pgfpathlineto{\pgfqpoint{0.799812in}{2.603388in}}%
\pgfpathlineto{\pgfqpoint{0.809524in}{2.600092in}}%
\pgfpathlineto{\pgfqpoint{0.819236in}{2.595909in}}%
\pgfpathlineto{\pgfqpoint{0.819846in}{2.595614in}}%
\pgfpathlineto{\pgfqpoint{0.828947in}{2.590777in}}%
\pgfpathlineto{\pgfqpoint{0.836634in}{2.585965in}}%
\pgfpathlineto{\pgfqpoint{0.838659in}{2.584541in}}%
\pgfpathlineto{\pgfqpoint{0.848371in}{2.576996in}}%
\pgfpathlineto{\pgfqpoint{0.849186in}{2.576316in}}%
\pgfpathlineto{\pgfqpoint{0.858083in}{2.567774in}}%
\pgfpathlineto{\pgfqpoint{0.859171in}{2.566667in}}%
\pgfpathlineto{\pgfqpoint{0.867291in}{2.557018in}}%
\pgfpathlineto{\pgfqpoint{0.867794in}{2.556306in}}%
\pgfpathlineto{\pgfqpoint{0.874034in}{2.547368in}}%
\pgfpathlineto{\pgfqpoint{0.877506in}{2.541229in}}%
\pgfpathlineto{\pgfqpoint{0.879501in}{2.537719in}}%
\pgfpathlineto{\pgfqpoint{0.883976in}{2.528070in}}%
\pgfpathlineto{\pgfqpoint{0.887218in}{2.518994in}}%
\pgfpathlineto{\pgfqpoint{0.887430in}{2.518421in}}%
\pgfpathlineto{\pgfqpoint{0.890171in}{2.508772in}}%
\pgfpathlineto{\pgfqpoint{0.892063in}{2.499123in}}%
\pgfpathlineto{\pgfqpoint{0.893172in}{2.489474in}}%
\pgfpathlineto{\pgfqpoint{0.893538in}{2.479825in}}%
\pgfpathlineto{\pgfqpoint{0.893172in}{2.470175in}}%
\pgfpathlineto{\pgfqpoint{0.892063in}{2.460526in}}%
\pgfpathlineto{\pgfqpoint{0.890171in}{2.450877in}}%
\pgfpathlineto{\pgfqpoint{0.887430in}{2.441228in}}%
\pgfpathlineto{\pgfqpoint{0.887218in}{2.440655in}}%
\pgfpathlineto{\pgfqpoint{0.883976in}{2.431579in}}%
\pgfpathlineto{\pgfqpoint{0.879501in}{2.421930in}}%
\pgfpathlineto{\pgfqpoint{0.877506in}{2.418420in}}%
\pgfpathlineto{\pgfqpoint{0.874034in}{2.412281in}}%
\pgfpathlineto{\pgfqpoint{0.867794in}{2.403343in}}%
\pgfpathlineto{\pgfqpoint{0.867291in}{2.402632in}}%
\pgfpathlineto{\pgfqpoint{0.859171in}{2.392982in}}%
\pgfpathlineto{\pgfqpoint{0.858083in}{2.391876in}}%
\pgfpathlineto{\pgfqpoint{0.849186in}{2.383333in}}%
\pgfpathlineto{\pgfqpoint{0.848371in}{2.382653in}}%
\pgfpathlineto{\pgfqpoint{0.838659in}{2.375108in}}%
\pgfpathlineto{\pgfqpoint{0.836634in}{2.373684in}}%
\pgfpathlineto{\pgfqpoint{0.828947in}{2.368872in}}%
\pgfpathlineto{\pgfqpoint{0.819846in}{2.364035in}}%
\pgfpathlineto{\pgfqpoint{0.819236in}{2.363740in}}%
\pgfpathlineto{\pgfqpoint{0.809524in}{2.359557in}}%
\pgfpathlineto{\pgfqpoint{0.799812in}{2.356261in}}%
\pgfpathlineto{\pgfqpoint{0.792611in}{2.354386in}}%
\pgfpathlineto{\pgfqpoint{0.790100in}{2.353768in}}%
\pgfpathlineto{\pgfqpoint{0.780388in}{2.351978in}}%
\pgfpathlineto{\pgfqpoint{0.770677in}{2.350916in}}%
\pgfpathlineto{\pgfqpoint{0.760965in}{2.350556in}}%
\pgfpathlineto{\pgfqpoint{0.751253in}{2.350889in}}%
\pgfpathlineto{\pgfqpoint{0.741541in}{2.351930in}}%
\pgfpathlineto{\pgfqpoint{0.731830in}{2.353723in}}%
\pgfpathlineto{\pgfqpoint{0.729475in}{2.354386in}}%
\pgfpathlineto{\pgfqpoint{0.722118in}{2.356138in}}%
\pgfpathlineto{\pgfqpoint{0.712406in}{2.359308in}}%
\pgfpathlineto{\pgfqpoint{0.702694in}{2.363512in}}%
\pgfpathlineto{\pgfqpoint{0.701795in}{2.364035in}}%
\pgfpathlineto{\pgfqpoint{0.692982in}{2.368336in}}%
\pgfpathlineto{\pgfqpoint{0.684777in}{2.373684in}}%
\pgfpathlineto{\pgfqpoint{0.683271in}{2.374499in}}%
\pgfpathlineto{\pgfqpoint{0.673559in}{2.381683in}}%
\pgfpathlineto{\pgfqpoint{0.671936in}{2.383333in}}%
\pgfpathlineto{\pgfqpoint{0.663847in}{2.390335in}}%
\pgfpathlineto{\pgfqpoint{0.661698in}{2.392982in}}%
\pgfpathlineto{\pgfqpoint{0.654135in}{2.401097in}}%
\pgfpathlineto{\pgfqpoint{0.653162in}{2.402632in}}%
\pgfpathlineto{\pgfqpoint{0.646271in}{2.412281in}}%
\pgfpathlineto{\pgfqpoint{0.644424in}{2.414476in}}%
\pgfpathlineto{\pgfqpoint{0.640585in}{2.421930in}}%
\pgfpathlineto{\pgfqpoint{0.635692in}{2.431579in}}%
\pgfpathlineto{\pgfqpoint{0.634712in}{2.433096in}}%
\pgfpathlineto{\pgfqpoint{0.631977in}{2.441228in}}%
\pgfpathlineto{\pgfqpoint{0.629187in}{2.450877in}}%
\pgfpathlineto{\pgfqpoint{0.626995in}{2.460526in}}%
\pgfpathlineto{\pgfqpoint{0.626259in}{2.470175in}}%
\pgfpathlineto{\pgfqpoint{0.625000in}{2.471518in}}%
\pgfusepath{stroke}%
\end{pgfscope}%
\begin{pgfscope}%
\pgfpathrectangle{\pgfqpoint{0.625000in}{0.550000in}}{\pgfqpoint{3.875000in}{3.850000in}} %
\pgfusepath{clip}%
\pgfsetbuttcap%
\pgfsetroundjoin%
\pgfsetlinewidth{0.250937pt}%
\definecolor{currentstroke}{rgb}{0.000000,0.000000,0.000000}%
\pgfsetstrokecolor{currentstroke}%
\pgfsetdash{}{0pt}%
\pgfpathmoveto{\pgfqpoint{0.625000in}{2.642051in}}%
\pgfpathlineto{\pgfqpoint{0.632822in}{2.634211in}}%
\pgfpathlineto{\pgfqpoint{0.625000in}{2.626884in}}%
\pgfusepath{stroke}%
\end{pgfscope}%
\begin{pgfscope}%
\pgfpathrectangle{\pgfqpoint{0.625000in}{0.550000in}}{\pgfqpoint{3.875000in}{3.850000in}} %
\pgfusepath{clip}%
\pgfsetbuttcap%
\pgfsetroundjoin%
\pgfsetlinewidth{0.250937pt}%
\definecolor{currentstroke}{rgb}{0.000000,0.000000,0.000000}%
\pgfsetstrokecolor{currentstroke}%
\pgfsetdash{}{0pt}%
\pgfpathmoveto{\pgfqpoint{0.625000in}{2.796170in}}%
\pgfpathlineto{\pgfqpoint{0.632787in}{2.788596in}}%
\pgfpathlineto{\pgfqpoint{0.625000in}{2.781236in}}%
\pgfusepath{stroke}%
\end{pgfscope}%
\begin{pgfscope}%
\pgfpathrectangle{\pgfqpoint{0.625000in}{0.550000in}}{\pgfqpoint{3.875000in}{3.850000in}} %
\pgfusepath{clip}%
\pgfsetbuttcap%
\pgfsetroundjoin%
\pgfsetlinewidth{0.250937pt}%
\definecolor{currentstroke}{rgb}{0.000000,0.000000,0.000000}%
\pgfsetstrokecolor{currentstroke}%
\pgfsetdash{}{0pt}%
\pgfpathmoveto{\pgfqpoint{0.625000in}{2.950648in}}%
\pgfpathlineto{\pgfqpoint{0.632737in}{2.942982in}}%
\pgfpathlineto{\pgfqpoint{0.625000in}{2.935253in}}%
\pgfusepath{stroke}%
\end{pgfscope}%
\begin{pgfscope}%
\pgfpathrectangle{\pgfqpoint{0.625000in}{0.550000in}}{\pgfqpoint{3.875000in}{3.850000in}} %
\pgfusepath{clip}%
\pgfsetbuttcap%
\pgfsetroundjoin%
\pgfsetlinewidth{0.250937pt}%
\definecolor{currentstroke}{rgb}{0.000000,0.000000,0.000000}%
\pgfsetstrokecolor{currentstroke}%
\pgfsetdash{}{0pt}%
\pgfpathmoveto{\pgfqpoint{0.625000in}{3.104691in}}%
\pgfpathlineto{\pgfqpoint{0.632766in}{3.097368in}}%
\pgfpathlineto{\pgfqpoint{0.625000in}{3.089889in}}%
\pgfusepath{stroke}%
\end{pgfscope}%
\begin{pgfscope}%
\pgfpathrectangle{\pgfqpoint{0.625000in}{0.550000in}}{\pgfqpoint{3.875000in}{3.850000in}} %
\pgfusepath{clip}%
\pgfsetbuttcap%
\pgfsetroundjoin%
\pgfsetlinewidth{0.250937pt}%
\definecolor{currentstroke}{rgb}{0.000000,0.000000,0.000000}%
\pgfsetstrokecolor{currentstroke}%
\pgfsetdash{}{0pt}%
\pgfpathmoveto{\pgfqpoint{0.625000in}{3.259153in}}%
\pgfpathlineto{\pgfqpoint{0.634599in}{3.251754in}}%
\pgfpathlineto{\pgfqpoint{0.625000in}{3.244404in}}%
\pgfusepath{stroke}%
\end{pgfscope}%
\begin{pgfscope}%
\pgfpathrectangle{\pgfqpoint{0.625000in}{0.550000in}}{\pgfqpoint{3.875000in}{3.850000in}} %
\pgfusepath{clip}%
\pgfsetbuttcap%
\pgfsetroundjoin%
\pgfsetlinewidth{0.250937pt}%
\definecolor{currentstroke}{rgb}{0.000000,0.000000,0.000000}%
\pgfsetstrokecolor{currentstroke}%
\pgfsetdash{}{0pt}%
\pgfpathmoveto{\pgfqpoint{0.625000in}{3.413780in}}%
\pgfpathlineto{\pgfqpoint{0.632768in}{3.406140in}}%
\pgfpathlineto{\pgfqpoint{0.625000in}{3.398749in}}%
\pgfusepath{stroke}%
\end{pgfscope}%
\begin{pgfscope}%
\pgfpathrectangle{\pgfqpoint{0.625000in}{0.550000in}}{\pgfqpoint{3.875000in}{3.850000in}} %
\pgfusepath{clip}%
\pgfsetbuttcap%
\pgfsetroundjoin%
\pgfsetlinewidth{0.250937pt}%
\definecolor{currentstroke}{rgb}{0.000000,0.000000,0.000000}%
\pgfsetstrokecolor{currentstroke}%
\pgfsetdash{}{0pt}%
\pgfpathmoveto{\pgfqpoint{0.625000in}{3.567923in}}%
\pgfpathlineto{\pgfqpoint{0.632785in}{3.560526in}}%
\pgfpathlineto{\pgfqpoint{0.625000in}{3.553134in}}%
\pgfusepath{stroke}%
\end{pgfscope}%
\begin{pgfscope}%
\pgfpathrectangle{\pgfqpoint{0.625000in}{0.550000in}}{\pgfqpoint{3.875000in}{3.850000in}} %
\pgfusepath{clip}%
\pgfsetbuttcap%
\pgfsetroundjoin%
\pgfsetlinewidth{0.250937pt}%
\definecolor{currentstroke}{rgb}{0.000000,0.000000,0.000000}%
\pgfsetstrokecolor{currentstroke}%
\pgfsetdash{}{0pt}%
\pgfpathmoveto{\pgfqpoint{0.625000in}{3.722386in}}%
\pgfpathlineto{\pgfqpoint{0.632804in}{3.714912in}}%
\pgfpathlineto{\pgfqpoint{0.625000in}{3.707214in}}%
\pgfusepath{stroke}%
\end{pgfscope}%
\begin{pgfscope}%
\pgfpathrectangle{\pgfqpoint{0.625000in}{0.550000in}}{\pgfqpoint{3.875000in}{3.850000in}} %
\pgfusepath{clip}%
\pgfsetbuttcap%
\pgfsetroundjoin%
\pgfsetlinewidth{0.250937pt}%
\definecolor{currentstroke}{rgb}{0.000000,0.000000,0.000000}%
\pgfsetstrokecolor{currentstroke}%
\pgfsetdash{}{0pt}%
\pgfpathmoveto{\pgfqpoint{0.625000in}{3.876643in}}%
\pgfpathlineto{\pgfqpoint{0.632719in}{3.869298in}}%
\pgfpathlineto{\pgfqpoint{0.625000in}{3.861923in}}%
\pgfusepath{stroke}%
\end{pgfscope}%
\begin{pgfscope}%
\pgfpathrectangle{\pgfqpoint{0.625000in}{0.550000in}}{\pgfqpoint{3.875000in}{3.850000in}} %
\pgfusepath{clip}%
\pgfsetbuttcap%
\pgfsetroundjoin%
\pgfsetlinewidth{0.250937pt}%
\definecolor{currentstroke}{rgb}{0.000000,0.000000,0.000000}%
\pgfsetstrokecolor{currentstroke}%
\pgfsetdash{}{0pt}%
\pgfpathmoveto{\pgfqpoint{0.625000in}{4.031160in}}%
\pgfpathlineto{\pgfqpoint{0.634601in}{4.023684in}}%
\pgfpathlineto{\pgfqpoint{0.625000in}{4.016281in}}%
\pgfusepath{stroke}%
\end{pgfscope}%
\begin{pgfscope}%
\pgfpathrectangle{\pgfqpoint{0.625000in}{0.550000in}}{\pgfqpoint{3.875000in}{3.850000in}} %
\pgfusepath{clip}%
\pgfsetbuttcap%
\pgfsetroundjoin%
\pgfsetlinewidth{0.250937pt}%
\definecolor{currentstroke}{rgb}{0.000000,0.000000,0.000000}%
\pgfsetstrokecolor{currentstroke}%
\pgfsetdash{}{0pt}%
\pgfpathmoveto{\pgfqpoint{0.625000in}{4.185493in}}%
\pgfpathlineto{\pgfqpoint{0.632711in}{4.178070in}}%
\pgfpathlineto{\pgfqpoint{0.625000in}{4.170995in}}%
\pgfusepath{stroke}%
\end{pgfscope}%
\begin{pgfscope}%
\pgfpathrectangle{\pgfqpoint{0.625000in}{0.550000in}}{\pgfqpoint{3.875000in}{3.850000in}} %
\pgfusepath{clip}%
\pgfsetbuttcap%
\pgfsetroundjoin%
\pgfsetlinewidth{0.250937pt}%
\definecolor{currentstroke}{rgb}{0.000000,0.000000,0.000000}%
\pgfsetstrokecolor{currentstroke}%
\pgfsetdash{}{0pt}%
\pgfpathmoveto{\pgfqpoint{0.625000in}{4.339846in}}%
\pgfpathlineto{\pgfqpoint{0.632676in}{4.332456in}}%
\pgfpathlineto{\pgfqpoint{0.625000in}{4.325041in}}%
\pgfusepath{stroke}%
\end{pgfscope}%
\begin{pgfscope}%
\pgfpathrectangle{\pgfqpoint{0.625000in}{0.550000in}}{\pgfqpoint{3.875000in}{3.850000in}} %
\pgfusepath{clip}%
\pgfsetbuttcap%
\pgfsetroundjoin%
\pgfsetlinewidth{0.250937pt}%
\definecolor{currentstroke}{rgb}{0.000000,0.000000,0.000000}%
\pgfsetstrokecolor{currentstroke}%
\pgfsetdash{}{0pt}%
\pgfpathmoveto{\pgfqpoint{0.634712in}{1.179652in}}%
\pgfpathlineto{\pgfqpoint{0.630238in}{1.186842in}}%
\pgfpathlineto{\pgfqpoint{0.628467in}{1.196491in}}%
\pgfpathlineto{\pgfqpoint{0.634712in}{1.205873in}}%
\pgfpathlineto{\pgfqpoint{0.644424in}{1.204863in}}%
\pgfpathlineto{\pgfqpoint{0.652251in}{1.196491in}}%
\pgfpathlineto{\pgfqpoint{0.650987in}{1.186842in}}%
\pgfpathlineto{\pgfqpoint{0.644424in}{1.181517in}}%
\pgfpathlineto{\pgfqpoint{0.634712in}{1.179652in}}%
\pgfusepath{stroke}%
\end{pgfscope}%
\begin{pgfscope}%
\pgfpathrectangle{\pgfqpoint{0.625000in}{0.550000in}}{\pgfqpoint{3.875000in}{3.850000in}} %
\pgfusepath{clip}%
\pgfsetbuttcap%
\pgfsetroundjoin%
\pgfsetlinewidth{0.250937pt}%
\definecolor{currentstroke}{rgb}{0.000000,0.000000,0.000000}%
\pgfsetstrokecolor{currentstroke}%
\pgfsetdash{}{0pt}%
\pgfpathmoveto{\pgfqpoint{0.634712in}{3.753776in}}%
\pgfpathlineto{\pgfqpoint{0.628467in}{3.763158in}}%
\pgfpathlineto{\pgfqpoint{0.630238in}{3.772807in}}%
\pgfpathlineto{\pgfqpoint{0.634712in}{3.779997in}}%
\pgfpathlineto{\pgfqpoint{0.644424in}{3.778132in}}%
\pgfpathlineto{\pgfqpoint{0.650987in}{3.772807in}}%
\pgfpathlineto{\pgfqpoint{0.652251in}{3.763158in}}%
\pgfpathlineto{\pgfqpoint{0.644424in}{3.754786in}}%
\pgfpathlineto{\pgfqpoint{0.634712in}{3.753776in}}%
\pgfusepath{stroke}%
\end{pgfscope}%
\begin{pgfscope}%
\pgfpathrectangle{\pgfqpoint{0.625000in}{0.550000in}}{\pgfqpoint{3.875000in}{3.850000in}} %
\pgfusepath{clip}%
\pgfsetbuttcap%
\pgfsetroundjoin%
\pgfsetlinewidth{0.250937pt}%
\definecolor{currentstroke}{rgb}{0.000000,0.000000,0.000000}%
\pgfsetstrokecolor{currentstroke}%
\pgfsetdash{}{0pt}%
\pgfpathmoveto{\pgfqpoint{0.625000in}{0.634979in}}%
\pgfpathlineto{\pgfqpoint{0.633030in}{0.627193in}}%
\pgfpathlineto{\pgfqpoint{0.625000in}{0.619430in}}%
\pgfusepath{stroke}%
\end{pgfscope}%
\begin{pgfscope}%
\pgfpathrectangle{\pgfqpoint{0.625000in}{0.550000in}}{\pgfqpoint{3.875000in}{3.850000in}} %
\pgfusepath{clip}%
\pgfsetbuttcap%
\pgfsetroundjoin%
\pgfsetlinewidth{0.250937pt}%
\definecolor{currentstroke}{rgb}{0.000000,0.000000,0.000000}%
\pgfsetstrokecolor{currentstroke}%
\pgfsetdash{}{0pt}%
\pgfpathmoveto{\pgfqpoint{0.625000in}{0.788961in}}%
\pgfpathlineto{\pgfqpoint{0.632986in}{0.781579in}}%
\pgfpathlineto{\pgfqpoint{0.625000in}{0.773868in}}%
\pgfusepath{stroke}%
\end{pgfscope}%
\begin{pgfscope}%
\pgfpathrectangle{\pgfqpoint{0.625000in}{0.550000in}}{\pgfqpoint{3.875000in}{3.850000in}} %
\pgfusepath{clip}%
\pgfsetbuttcap%
\pgfsetroundjoin%
\pgfsetlinewidth{0.250937pt}%
\definecolor{currentstroke}{rgb}{0.000000,0.000000,0.000000}%
\pgfsetstrokecolor{currentstroke}%
\pgfsetdash{}{0pt}%
\pgfpathmoveto{\pgfqpoint{0.625000in}{0.943635in}}%
\pgfpathlineto{\pgfqpoint{0.634710in}{0.935965in}}%
\pgfpathlineto{\pgfqpoint{0.625000in}{0.928226in}}%
\pgfusepath{stroke}%
\end{pgfscope}%
\begin{pgfscope}%
\pgfpathrectangle{\pgfqpoint{0.625000in}{0.550000in}}{\pgfqpoint{3.875000in}{3.850000in}} %
\pgfusepath{clip}%
\pgfsetbuttcap%
\pgfsetroundjoin%
\pgfsetlinewidth{0.250937pt}%
\definecolor{currentstroke}{rgb}{0.000000,0.000000,0.000000}%
\pgfsetstrokecolor{currentstroke}%
\pgfsetdash{}{0pt}%
\pgfpathmoveto{\pgfqpoint{0.625000in}{1.098014in}}%
\pgfpathlineto{\pgfqpoint{0.632991in}{1.090351in}}%
\pgfpathlineto{\pgfqpoint{0.625000in}{1.082716in}}%
\pgfusepath{stroke}%
\end{pgfscope}%
\begin{pgfscope}%
\pgfpathrectangle{\pgfqpoint{0.625000in}{0.550000in}}{\pgfqpoint{3.875000in}{3.850000in}} %
\pgfusepath{clip}%
\pgfsetbuttcap%
\pgfsetroundjoin%
\pgfsetlinewidth{0.250937pt}%
\definecolor{currentstroke}{rgb}{0.000000,0.000000,0.000000}%
\pgfsetstrokecolor{currentstroke}%
\pgfsetdash{}{0pt}%
\pgfpathmoveto{\pgfqpoint{0.625000in}{1.252633in}}%
\pgfpathlineto{\pgfqpoint{0.633001in}{1.244737in}}%
\pgfpathlineto{\pgfqpoint{0.625000in}{1.237058in}}%
\pgfusepath{stroke}%
\end{pgfscope}%
\begin{pgfscope}%
\pgfpathrectangle{\pgfqpoint{0.625000in}{0.550000in}}{\pgfqpoint{3.875000in}{3.850000in}} %
\pgfusepath{clip}%
\pgfsetbuttcap%
\pgfsetroundjoin%
\pgfsetlinewidth{0.250937pt}%
\definecolor{currentstroke}{rgb}{0.000000,0.000000,0.000000}%
\pgfsetstrokecolor{currentstroke}%
\pgfsetdash{}{0pt}%
\pgfpathmoveto{\pgfqpoint{0.625000in}{1.406683in}}%
\pgfpathlineto{\pgfqpoint{0.632947in}{1.399123in}}%
\pgfpathlineto{\pgfqpoint{0.625000in}{1.391559in}}%
\pgfusepath{stroke}%
\end{pgfscope}%
\begin{pgfscope}%
\pgfpathrectangle{\pgfqpoint{0.625000in}{0.550000in}}{\pgfqpoint{3.875000in}{3.850000in}} %
\pgfusepath{clip}%
\pgfsetbuttcap%
\pgfsetroundjoin%
\pgfsetlinewidth{0.250937pt}%
\definecolor{currentstroke}{rgb}{0.000000,0.000000,0.000000}%
\pgfsetstrokecolor{currentstroke}%
\pgfsetdash{}{0pt}%
\pgfpathmoveto{\pgfqpoint{0.625000in}{1.561052in}}%
\pgfpathlineto{\pgfqpoint{0.632918in}{1.553509in}}%
\pgfpathlineto{\pgfqpoint{0.625000in}{1.545720in}}%
\pgfusepath{stroke}%
\end{pgfscope}%
\begin{pgfscope}%
\pgfpathrectangle{\pgfqpoint{0.625000in}{0.550000in}}{\pgfqpoint{3.875000in}{3.850000in}} %
\pgfusepath{clip}%
\pgfsetbuttcap%
\pgfsetroundjoin%
\pgfsetlinewidth{0.250937pt}%
\definecolor{currentstroke}{rgb}{0.000000,0.000000,0.000000}%
\pgfsetstrokecolor{currentstroke}%
\pgfsetdash{}{0pt}%
\pgfpathmoveto{\pgfqpoint{0.625000in}{1.715431in}}%
\pgfpathlineto{\pgfqpoint{0.634710in}{1.707895in}}%
\pgfpathlineto{\pgfqpoint{0.625000in}{1.700311in}}%
\pgfusepath{stroke}%
\end{pgfscope}%
\begin{pgfscope}%
\pgfpathrectangle{\pgfqpoint{0.625000in}{0.550000in}}{\pgfqpoint{3.875000in}{3.850000in}} %
\pgfusepath{clip}%
\pgfsetbuttcap%
\pgfsetroundjoin%
\pgfsetlinewidth{0.250937pt}%
\definecolor{currentstroke}{rgb}{0.000000,0.000000,0.000000}%
\pgfsetstrokecolor{currentstroke}%
\pgfsetdash{}{0pt}%
\pgfpathmoveto{\pgfqpoint{0.625000in}{1.869872in}}%
\pgfpathlineto{\pgfqpoint{0.632878in}{1.862281in}}%
\pgfpathlineto{\pgfqpoint{0.625000in}{1.854847in}}%
\pgfusepath{stroke}%
\end{pgfscope}%
\begin{pgfscope}%
\pgfpathrectangle{\pgfqpoint{0.625000in}{0.550000in}}{\pgfqpoint{3.875000in}{3.850000in}} %
\pgfusepath{clip}%
\pgfsetbuttcap%
\pgfsetroundjoin%
\pgfsetlinewidth{0.250937pt}%
\definecolor{currentstroke}{rgb}{0.000000,0.000000,0.000000}%
\pgfsetstrokecolor{currentstroke}%
\pgfsetdash{}{0pt}%
\pgfpathmoveto{\pgfqpoint{0.625000in}{2.024517in}}%
\pgfpathlineto{\pgfqpoint{0.632858in}{2.016667in}}%
\pgfpathlineto{\pgfqpoint{0.625000in}{2.008881in}}%
\pgfusepath{stroke}%
\end{pgfscope}%
\begin{pgfscope}%
\pgfpathrectangle{\pgfqpoint{0.625000in}{0.550000in}}{\pgfqpoint{3.875000in}{3.850000in}} %
\pgfusepath{clip}%
\pgfsetbuttcap%
\pgfsetroundjoin%
\pgfsetlinewidth{0.250937pt}%
\definecolor{currentstroke}{rgb}{0.000000,0.000000,0.000000}%
\pgfsetstrokecolor{currentstroke}%
\pgfsetdash{}{0pt}%
\pgfpathmoveto{\pgfqpoint{0.625000in}{2.178509in}}%
\pgfpathlineto{\pgfqpoint{0.632886in}{2.171053in}}%
\pgfpathlineto{\pgfqpoint{0.625000in}{2.163382in}}%
\pgfusepath{stroke}%
\end{pgfscope}%
\begin{pgfscope}%
\pgfpathrectangle{\pgfqpoint{0.625000in}{0.550000in}}{\pgfqpoint{3.875000in}{3.850000in}} %
\pgfusepath{clip}%
\pgfsetbuttcap%
\pgfsetroundjoin%
\pgfsetlinewidth{0.250937pt}%
\definecolor{currentstroke}{rgb}{0.000000,0.000000,0.000000}%
\pgfsetstrokecolor{currentstroke}%
\pgfsetdash{}{0pt}%
\pgfpathmoveto{\pgfqpoint{0.625000in}{2.332852in}}%
\pgfpathlineto{\pgfqpoint{0.632914in}{2.325439in}}%
\pgfpathlineto{\pgfqpoint{0.625000in}{2.317506in}}%
\pgfusepath{stroke}%
\end{pgfscope}%
\begin{pgfscope}%
\pgfpathrectangle{\pgfqpoint{0.625000in}{0.550000in}}{\pgfqpoint{3.875000in}{3.850000in}} %
\pgfusepath{clip}%
\pgfsetbuttcap%
\pgfsetroundjoin%
\pgfsetlinewidth{0.250937pt}%
\definecolor{currentstroke}{rgb}{0.000000,0.000000,0.000000}%
\pgfsetstrokecolor{currentstroke}%
\pgfsetdash{}{0pt}%
\pgfpathmoveto{\pgfqpoint{0.625000in}{2.488215in}}%
\pgfpathlineto{\pgfqpoint{0.626180in}{2.489474in}}%
\pgfpathlineto{\pgfqpoint{0.626911in}{2.499123in}}%
\pgfpathlineto{\pgfqpoint{0.629032in}{2.508772in}}%
\pgfpathlineto{\pgfqpoint{0.631715in}{2.518421in}}%
\pgfpathlineto{\pgfqpoint{0.634712in}{2.527333in}}%
\pgfpathlineto{\pgfqpoint{0.635188in}{2.528070in}}%
\pgfpathlineto{\pgfqpoint{0.639916in}{2.537719in}}%
\pgfpathlineto{\pgfqpoint{0.644424in}{2.546473in}}%
\pgfpathlineto{\pgfqpoint{0.645177in}{2.547368in}}%
\pgfpathlineto{\pgfqpoint{0.651852in}{2.557018in}}%
\pgfpathlineto{\pgfqpoint{0.654135in}{2.560620in}}%
\pgfpathlineto{\pgfqpoint{0.659771in}{2.566667in}}%
\pgfpathlineto{\pgfqpoint{0.663847in}{2.571688in}}%
\pgfpathlineto{\pgfqpoint{0.669193in}{2.576316in}}%
\pgfpathlineto{\pgfqpoint{0.673559in}{2.580755in}}%
\pgfpathlineto{\pgfqpoint{0.680625in}{2.585965in}}%
\pgfpathlineto{\pgfqpoint{0.683271in}{2.588266in}}%
\pgfpathlineto{\pgfqpoint{0.692982in}{2.594577in}}%
\pgfpathlineto{\pgfqpoint{0.695108in}{2.595614in}}%
\pgfpathlineto{\pgfqpoint{0.702694in}{2.600029in}}%
\pgfpathlineto{\pgfqpoint{0.712406in}{2.604348in}}%
\pgfpathlineto{\pgfqpoint{0.715139in}{2.605263in}}%
\pgfpathlineto{\pgfqpoint{0.722118in}{2.608028in}}%
\pgfpathlineto{\pgfqpoint{0.731830in}{2.610898in}}%
\pgfpathlineto{\pgfqpoint{0.741541in}{2.612991in}}%
\pgfpathlineto{\pgfqpoint{0.751253in}{2.614400in}}%
\pgfpathlineto{\pgfqpoint{0.757895in}{2.614912in}}%
\pgfpathlineto{\pgfqpoint{0.760965in}{2.615187in}}%
\pgfpathlineto{\pgfqpoint{0.770677in}{2.615333in}}%
\pgfpathlineto{\pgfqpoint{0.778534in}{2.614912in}}%
\pgfpathlineto{\pgfqpoint{0.780388in}{2.614811in}}%
\pgfpathlineto{\pgfqpoint{0.790100in}{2.613656in}}%
\pgfpathlineto{\pgfqpoint{0.799812in}{2.611809in}}%
\pgfpathlineto{\pgfqpoint{0.809524in}{2.609228in}}%
\pgfpathlineto{\pgfqpoint{0.819236in}{2.605867in}}%
\pgfpathlineto{\pgfqpoint{0.820742in}{2.605263in}}%
\pgfpathlineto{\pgfqpoint{0.828947in}{2.601708in}}%
\pgfpathlineto{\pgfqpoint{0.838659in}{2.596628in}}%
\pgfpathlineto{\pgfqpoint{0.840409in}{2.595614in}}%
\pgfpathlineto{\pgfqpoint{0.848371in}{2.590497in}}%
\pgfpathlineto{\pgfqpoint{0.854636in}{2.585965in}}%
\pgfpathlineto{\pgfqpoint{0.858083in}{2.583147in}}%
\pgfpathlineto{\pgfqpoint{0.865806in}{2.576316in}}%
\pgfpathlineto{\pgfqpoint{0.867794in}{2.574289in}}%
\pgfpathlineto{\pgfqpoint{0.874920in}{2.566667in}}%
\pgfpathlineto{\pgfqpoint{0.877506in}{2.563412in}}%
\pgfpathlineto{\pgfqpoint{0.882465in}{2.557018in}}%
\pgfpathlineto{\pgfqpoint{0.887218in}{2.549647in}}%
\pgfpathlineto{\pgfqpoint{0.888680in}{2.547368in}}%
\pgfpathlineto{\pgfqpoint{0.893873in}{2.537719in}}%
\pgfpathlineto{\pgfqpoint{0.896930in}{2.530605in}}%
\pgfpathlineto{\pgfqpoint{0.898043in}{2.528070in}}%
\pgfpathlineto{\pgfqpoint{0.901436in}{2.518421in}}%
\pgfpathlineto{\pgfqpoint{0.903964in}{2.508772in}}%
\pgfpathlineto{\pgfqpoint{0.905712in}{2.499123in}}%
\pgfpathlineto{\pgfqpoint{0.906642in}{2.490393in}}%
\pgfpathlineto{\pgfqpoint{0.906746in}{2.489474in}}%
\pgfpathlineto{\pgfqpoint{0.907109in}{2.479825in}}%
\pgfpathlineto{\pgfqpoint{0.906746in}{2.470175in}}%
\pgfpathlineto{\pgfqpoint{0.906642in}{2.469256in}}%
\pgfpathlineto{\pgfqpoint{0.905712in}{2.460526in}}%
\pgfpathlineto{\pgfqpoint{0.903964in}{2.450877in}}%
\pgfpathlineto{\pgfqpoint{0.901436in}{2.441228in}}%
\pgfpathlineto{\pgfqpoint{0.898043in}{2.431579in}}%
\pgfpathlineto{\pgfqpoint{0.896930in}{2.429045in}}%
\pgfpathlineto{\pgfqpoint{0.893873in}{2.421930in}}%
\pgfpathlineto{\pgfqpoint{0.888680in}{2.412281in}}%
\pgfpathlineto{\pgfqpoint{0.887218in}{2.410002in}}%
\pgfpathlineto{\pgfqpoint{0.882465in}{2.402632in}}%
\pgfpathlineto{\pgfqpoint{0.877506in}{2.396237in}}%
\pgfpathlineto{\pgfqpoint{0.874920in}{2.392982in}}%
\pgfpathlineto{\pgfqpoint{0.867794in}{2.385360in}}%
\pgfpathlineto{\pgfqpoint{0.865806in}{2.383333in}}%
\pgfpathlineto{\pgfqpoint{0.858083in}{2.376502in}}%
\pgfpathlineto{\pgfqpoint{0.854636in}{2.373684in}}%
\pgfpathlineto{\pgfqpoint{0.848371in}{2.369152in}}%
\pgfpathlineto{\pgfqpoint{0.840409in}{2.364035in}}%
\pgfpathlineto{\pgfqpoint{0.838659in}{2.363021in}}%
\pgfpathlineto{\pgfqpoint{0.828947in}{2.357941in}}%
\pgfpathlineto{\pgfqpoint{0.820742in}{2.354386in}}%
\pgfpathlineto{\pgfqpoint{0.819236in}{2.353783in}}%
\pgfpathlineto{\pgfqpoint{0.809524in}{2.350421in}}%
\pgfpathlineto{\pgfqpoint{0.799812in}{2.347840in}}%
\pgfpathlineto{\pgfqpoint{0.790100in}{2.345993in}}%
\pgfpathlineto{\pgfqpoint{0.780388in}{2.344838in}}%
\pgfpathlineto{\pgfqpoint{0.778534in}{2.344737in}}%
\pgfpathlineto{\pgfqpoint{0.770677in}{2.344316in}}%
\pgfpathlineto{\pgfqpoint{0.760965in}{2.344462in}}%
\pgfpathlineto{\pgfqpoint{0.757895in}{2.344737in}}%
\pgfpathlineto{\pgfqpoint{0.751253in}{2.345249in}}%
\pgfpathlineto{\pgfqpoint{0.741541in}{2.346658in}}%
\pgfpathlineto{\pgfqpoint{0.731830in}{2.348751in}}%
\pgfpathlineto{\pgfqpoint{0.722118in}{2.351621in}}%
\pgfpathlineto{\pgfqpoint{0.715139in}{2.354386in}}%
\pgfpathlineto{\pgfqpoint{0.712406in}{2.355301in}}%
\pgfpathlineto{\pgfqpoint{0.702694in}{2.359620in}}%
\pgfpathlineto{\pgfqpoint{0.695108in}{2.364035in}}%
\pgfpathlineto{\pgfqpoint{0.692982in}{2.365072in}}%
\pgfpathlineto{\pgfqpoint{0.683271in}{2.371383in}}%
\pgfpathlineto{\pgfqpoint{0.680625in}{2.373684in}}%
\pgfpathlineto{\pgfqpoint{0.673559in}{2.378894in}}%
\pgfpathlineto{\pgfqpoint{0.669193in}{2.383333in}}%
\pgfpathlineto{\pgfqpoint{0.663847in}{2.387961in}}%
\pgfpathlineto{\pgfqpoint{0.659771in}{2.392982in}}%
\pgfpathlineto{\pgfqpoint{0.654135in}{2.399030in}}%
\pgfpathlineto{\pgfqpoint{0.651852in}{2.402632in}}%
\pgfpathlineto{\pgfqpoint{0.645177in}{2.412281in}}%
\pgfpathlineto{\pgfqpoint{0.644424in}{2.413176in}}%
\pgfpathlineto{\pgfqpoint{0.639916in}{2.421930in}}%
\pgfpathlineto{\pgfqpoint{0.635188in}{2.431579in}}%
\pgfpathlineto{\pgfqpoint{0.634712in}{2.432316in}}%
\pgfpathlineto{\pgfqpoint{0.631715in}{2.441228in}}%
\pgfpathlineto{\pgfqpoint{0.629032in}{2.450877in}}%
\pgfpathlineto{\pgfqpoint{0.626916in}{2.460526in}}%
\pgfpathlineto{\pgfqpoint{0.626180in}{2.470175in}}%
\pgfpathlineto{\pgfqpoint{0.625000in}{2.471434in}}%
\pgfusepath{stroke}%
\end{pgfscope}%
\begin{pgfscope}%
\pgfpathrectangle{\pgfqpoint{0.625000in}{0.550000in}}{\pgfqpoint{3.875000in}{3.850000in}} %
\pgfusepath{clip}%
\pgfsetbuttcap%
\pgfsetroundjoin%
\pgfsetlinewidth{0.250937pt}%
\definecolor{currentstroke}{rgb}{0.000000,0.000000,0.000000}%
\pgfsetstrokecolor{currentstroke}%
\pgfsetdash{}{0pt}%
\pgfpathmoveto{\pgfqpoint{0.625000in}{2.642140in}}%
\pgfpathlineto{\pgfqpoint{0.632911in}{2.634211in}}%
\pgfpathlineto{\pgfqpoint{0.625000in}{2.626801in}}%
\pgfusepath{stroke}%
\end{pgfscope}%
\begin{pgfscope}%
\pgfpathrectangle{\pgfqpoint{0.625000in}{0.550000in}}{\pgfqpoint{3.875000in}{3.850000in}} %
\pgfusepath{clip}%
\pgfsetbuttcap%
\pgfsetroundjoin%
\pgfsetlinewidth{0.250937pt}%
\definecolor{currentstroke}{rgb}{0.000000,0.000000,0.000000}%
\pgfsetstrokecolor{currentstroke}%
\pgfsetdash{}{0pt}%
\pgfpathmoveto{\pgfqpoint{0.625000in}{2.796255in}}%
\pgfpathlineto{\pgfqpoint{0.632874in}{2.788596in}}%
\pgfpathlineto{\pgfqpoint{0.625000in}{2.781153in}}%
\pgfusepath{stroke}%
\end{pgfscope}%
\begin{pgfscope}%
\pgfpathrectangle{\pgfqpoint{0.625000in}{0.550000in}}{\pgfqpoint{3.875000in}{3.850000in}} %
\pgfusepath{clip}%
\pgfsetbuttcap%
\pgfsetroundjoin%
\pgfsetlinewidth{0.250937pt}%
\definecolor{currentstroke}{rgb}{0.000000,0.000000,0.000000}%
\pgfsetstrokecolor{currentstroke}%
\pgfsetdash{}{0pt}%
\pgfpathmoveto{\pgfqpoint{0.625000in}{2.950736in}}%
\pgfpathlineto{\pgfqpoint{0.632825in}{2.942982in}}%
\pgfpathlineto{\pgfqpoint{0.625000in}{2.935164in}}%
\pgfusepath{stroke}%
\end{pgfscope}%
\begin{pgfscope}%
\pgfpathrectangle{\pgfqpoint{0.625000in}{0.550000in}}{\pgfqpoint{3.875000in}{3.850000in}} %
\pgfusepath{clip}%
\pgfsetbuttcap%
\pgfsetroundjoin%
\pgfsetlinewidth{0.250937pt}%
\definecolor{currentstroke}{rgb}{0.000000,0.000000,0.000000}%
\pgfsetstrokecolor{currentstroke}%
\pgfsetdash{}{0pt}%
\pgfpathmoveto{\pgfqpoint{0.625000in}{3.104774in}}%
\pgfpathlineto{\pgfqpoint{0.632853in}{3.097368in}}%
\pgfpathlineto{\pgfqpoint{0.625000in}{3.089805in}}%
\pgfusepath{stroke}%
\end{pgfscope}%
\begin{pgfscope}%
\pgfpathrectangle{\pgfqpoint{0.625000in}{0.550000in}}{\pgfqpoint{3.875000in}{3.850000in}} %
\pgfusepath{clip}%
\pgfsetbuttcap%
\pgfsetroundjoin%
\pgfsetlinewidth{0.250937pt}%
\definecolor{currentstroke}{rgb}{0.000000,0.000000,0.000000}%
\pgfsetstrokecolor{currentstroke}%
\pgfsetdash{}{0pt}%
\pgfpathmoveto{\pgfqpoint{0.625000in}{3.259239in}}%
\pgfpathlineto{\pgfqpoint{0.634710in}{3.251754in}}%
\pgfpathlineto{\pgfqpoint{0.625000in}{3.244319in}}%
\pgfusepath{stroke}%
\end{pgfscope}%
\begin{pgfscope}%
\pgfpathrectangle{\pgfqpoint{0.625000in}{0.550000in}}{\pgfqpoint{3.875000in}{3.850000in}} %
\pgfusepath{clip}%
\pgfsetbuttcap%
\pgfsetroundjoin%
\pgfsetlinewidth{0.250937pt}%
\definecolor{currentstroke}{rgb}{0.000000,0.000000,0.000000}%
\pgfsetstrokecolor{currentstroke}%
\pgfsetdash{}{0pt}%
\pgfpathmoveto{\pgfqpoint{0.625000in}{3.413865in}}%
\pgfpathlineto{\pgfqpoint{0.632855in}{3.406140in}}%
\pgfpathlineto{\pgfqpoint{0.625000in}{3.398667in}}%
\pgfusepath{stroke}%
\end{pgfscope}%
\begin{pgfscope}%
\pgfpathrectangle{\pgfqpoint{0.625000in}{0.550000in}}{\pgfqpoint{3.875000in}{3.850000in}} %
\pgfusepath{clip}%
\pgfsetbuttcap%
\pgfsetroundjoin%
\pgfsetlinewidth{0.250937pt}%
\definecolor{currentstroke}{rgb}{0.000000,0.000000,0.000000}%
\pgfsetstrokecolor{currentstroke}%
\pgfsetdash{}{0pt}%
\pgfpathmoveto{\pgfqpoint{0.625000in}{3.568008in}}%
\pgfpathlineto{\pgfqpoint{0.632874in}{3.560526in}}%
\pgfpathlineto{\pgfqpoint{0.625000in}{3.553049in}}%
\pgfusepath{stroke}%
\end{pgfscope}%
\begin{pgfscope}%
\pgfpathrectangle{\pgfqpoint{0.625000in}{0.550000in}}{\pgfqpoint{3.875000in}{3.850000in}} %
\pgfusepath{clip}%
\pgfsetbuttcap%
\pgfsetroundjoin%
\pgfsetlinewidth{0.250937pt}%
\definecolor{currentstroke}{rgb}{0.000000,0.000000,0.000000}%
\pgfsetstrokecolor{currentstroke}%
\pgfsetdash{}{0pt}%
\pgfpathmoveto{\pgfqpoint{0.625000in}{3.722471in}}%
\pgfpathlineto{\pgfqpoint{0.632893in}{3.714912in}}%
\pgfpathlineto{\pgfqpoint{0.625000in}{3.707126in}}%
\pgfusepath{stroke}%
\end{pgfscope}%
\begin{pgfscope}%
\pgfpathrectangle{\pgfqpoint{0.625000in}{0.550000in}}{\pgfqpoint{3.875000in}{3.850000in}} %
\pgfusepath{clip}%
\pgfsetbuttcap%
\pgfsetroundjoin%
\pgfsetlinewidth{0.250937pt}%
\definecolor{currentstroke}{rgb}{0.000000,0.000000,0.000000}%
\pgfsetstrokecolor{currentstroke}%
\pgfsetdash{}{0pt}%
\pgfpathmoveto{\pgfqpoint{0.625000in}{3.876730in}}%
\pgfpathlineto{\pgfqpoint{0.632810in}{3.869298in}}%
\pgfpathlineto{\pgfqpoint{0.625000in}{3.861836in}}%
\pgfusepath{stroke}%
\end{pgfscope}%
\begin{pgfscope}%
\pgfpathrectangle{\pgfqpoint{0.625000in}{0.550000in}}{\pgfqpoint{3.875000in}{3.850000in}} %
\pgfusepath{clip}%
\pgfsetbuttcap%
\pgfsetroundjoin%
\pgfsetlinewidth{0.250937pt}%
\definecolor{currentstroke}{rgb}{0.000000,0.000000,0.000000}%
\pgfsetstrokecolor{currentstroke}%
\pgfsetdash{}{0pt}%
\pgfpathmoveto{\pgfqpoint{0.625000in}{4.031245in}}%
\pgfpathlineto{\pgfqpoint{0.634710in}{4.023684in}}%
\pgfpathlineto{\pgfqpoint{0.625000in}{4.016198in}}%
\pgfusepath{stroke}%
\end{pgfscope}%
\begin{pgfscope}%
\pgfpathrectangle{\pgfqpoint{0.625000in}{0.550000in}}{\pgfqpoint{3.875000in}{3.850000in}} %
\pgfusepath{clip}%
\pgfsetbuttcap%
\pgfsetroundjoin%
\pgfsetlinewidth{0.250937pt}%
\definecolor{currentstroke}{rgb}{0.000000,0.000000,0.000000}%
\pgfsetstrokecolor{currentstroke}%
\pgfsetdash{}{0pt}%
\pgfpathmoveto{\pgfqpoint{0.625000in}{4.185581in}}%
\pgfpathlineto{\pgfqpoint{0.632803in}{4.178070in}}%
\pgfpathlineto{\pgfqpoint{0.625000in}{4.170911in}}%
\pgfusepath{stroke}%
\end{pgfscope}%
\begin{pgfscope}%
\pgfpathrectangle{\pgfqpoint{0.625000in}{0.550000in}}{\pgfqpoint{3.875000in}{3.850000in}} %
\pgfusepath{clip}%
\pgfsetbuttcap%
\pgfsetroundjoin%
\pgfsetlinewidth{0.250937pt}%
\definecolor{currentstroke}{rgb}{0.000000,0.000000,0.000000}%
\pgfsetstrokecolor{currentstroke}%
\pgfsetdash{}{0pt}%
\pgfpathmoveto{\pgfqpoint{0.625000in}{4.339934in}}%
\pgfpathlineto{\pgfqpoint{0.632767in}{4.332456in}}%
\pgfpathlineto{\pgfqpoint{0.625000in}{4.324954in}}%
\pgfusepath{stroke}%
\end{pgfscope}%
\begin{pgfscope}%
\pgfpathrectangle{\pgfqpoint{0.625000in}{0.550000in}}{\pgfqpoint{3.875000in}{3.850000in}} %
\pgfusepath{clip}%
\pgfsetbuttcap%
\pgfsetroundjoin%
\pgfsetlinewidth{0.250937pt}%
\definecolor{currentstroke}{rgb}{0.000000,0.000000,0.000000}%
\pgfsetstrokecolor{currentstroke}%
\pgfsetdash{}{0pt}%
\pgfpathmoveto{\pgfqpoint{0.634712in}{1.179318in}}%
\pgfpathlineto{\pgfqpoint{0.630031in}{1.186842in}}%
\pgfpathlineto{\pgfqpoint{0.628293in}{1.196491in}}%
\pgfpathlineto{\pgfqpoint{0.634712in}{1.206136in}}%
\pgfpathlineto{\pgfqpoint{0.644424in}{1.205883in}}%
\pgfpathlineto{\pgfqpoint{0.653205in}{1.196491in}}%
\pgfpathlineto{\pgfqpoint{0.652132in}{1.186842in}}%
\pgfpathlineto{\pgfqpoint{0.644424in}{1.180589in}}%
\pgfpathlineto{\pgfqpoint{0.634712in}{1.179318in}}%
\pgfusepath{stroke}%
\end{pgfscope}%
\begin{pgfscope}%
\pgfpathrectangle{\pgfqpoint{0.625000in}{0.550000in}}{\pgfqpoint{3.875000in}{3.850000in}} %
\pgfusepath{clip}%
\pgfsetbuttcap%
\pgfsetroundjoin%
\pgfsetlinewidth{0.250937pt}%
\definecolor{currentstroke}{rgb}{0.000000,0.000000,0.000000}%
\pgfsetstrokecolor{currentstroke}%
\pgfsetdash{}{0pt}%
\pgfpathmoveto{\pgfqpoint{0.634712in}{3.753513in}}%
\pgfpathlineto{\pgfqpoint{0.628293in}{3.763158in}}%
\pgfpathlineto{\pgfqpoint{0.630031in}{3.772807in}}%
\pgfpathlineto{\pgfqpoint{0.634712in}{3.780331in}}%
\pgfpathlineto{\pgfqpoint{0.644424in}{3.779060in}}%
\pgfpathlineto{\pgfqpoint{0.652132in}{3.772807in}}%
\pgfpathlineto{\pgfqpoint{0.653205in}{3.763158in}}%
\pgfpathlineto{\pgfqpoint{0.644424in}{3.753766in}}%
\pgfpathlineto{\pgfqpoint{0.634712in}{3.753513in}}%
\pgfusepath{stroke}%
\end{pgfscope}%
\begin{pgfscope}%
\pgfpathrectangle{\pgfqpoint{0.625000in}{0.550000in}}{\pgfqpoint{3.875000in}{3.850000in}} %
\pgfusepath{clip}%
\pgfsetbuttcap%
\pgfsetroundjoin%
\pgfsetlinewidth{0.250937pt}%
\definecolor{currentstroke}{rgb}{0.000000,0.000000,0.000000}%
\pgfsetstrokecolor{currentstroke}%
\pgfsetdash{}{0pt}%
\pgfpathmoveto{\pgfqpoint{0.625000in}{0.635055in}}%
\pgfpathlineto{\pgfqpoint{0.633109in}{0.627193in}}%
\pgfpathlineto{\pgfqpoint{0.625000in}{0.619354in}}%
\pgfusepath{stroke}%
\end{pgfscope}%
\begin{pgfscope}%
\pgfpathrectangle{\pgfqpoint{0.625000in}{0.550000in}}{\pgfqpoint{3.875000in}{3.850000in}} %
\pgfusepath{clip}%
\pgfsetbuttcap%
\pgfsetroundjoin%
\pgfsetlinewidth{0.250937pt}%
\definecolor{currentstroke}{rgb}{0.000000,0.000000,0.000000}%
\pgfsetstrokecolor{currentstroke}%
\pgfsetdash{}{0pt}%
\pgfpathmoveto{\pgfqpoint{0.625000in}{0.789038in}}%
\pgfpathlineto{\pgfqpoint{0.633069in}{0.781579in}}%
\pgfpathlineto{\pgfqpoint{0.625000in}{0.773788in}}%
\pgfusepath{stroke}%
\end{pgfscope}%
\begin{pgfscope}%
\pgfpathrectangle{\pgfqpoint{0.625000in}{0.550000in}}{\pgfqpoint{3.875000in}{3.850000in}} %
\pgfusepath{clip}%
\pgfsetbuttcap%
\pgfsetroundjoin%
\pgfsetlinewidth{0.250937pt}%
\definecolor{currentstroke}{rgb}{0.000000,0.000000,0.000000}%
\pgfsetstrokecolor{currentstroke}%
\pgfsetdash{}{0pt}%
\pgfpathmoveto{\pgfqpoint{0.625000in}{0.943712in}}%
\pgfpathlineto{\pgfqpoint{0.634712in}{0.936763in}}%
\pgfpathlineto{\pgfqpoint{0.635565in}{0.935965in}}%
\pgfpathlineto{\pgfqpoint{0.634712in}{0.935167in}}%
\pgfpathlineto{\pgfqpoint{0.625000in}{0.928148in}}%
\pgfusepath{stroke}%
\end{pgfscope}%
\begin{pgfscope}%
\pgfpathrectangle{\pgfqpoint{0.625000in}{0.550000in}}{\pgfqpoint{3.875000in}{3.850000in}} %
\pgfusepath{clip}%
\pgfsetbuttcap%
\pgfsetroundjoin%
\pgfsetlinewidth{0.250937pt}%
\definecolor{currentstroke}{rgb}{0.000000,0.000000,0.000000}%
\pgfsetstrokecolor{currentstroke}%
\pgfsetdash{}{0pt}%
\pgfpathmoveto{\pgfqpoint{0.625000in}{1.098093in}}%
\pgfpathlineto{\pgfqpoint{0.633073in}{1.090351in}}%
\pgfpathlineto{\pgfqpoint{0.625000in}{1.082638in}}%
\pgfusepath{stroke}%
\end{pgfscope}%
\begin{pgfscope}%
\pgfpathrectangle{\pgfqpoint{0.625000in}{0.550000in}}{\pgfqpoint{3.875000in}{3.850000in}} %
\pgfusepath{clip}%
\pgfsetbuttcap%
\pgfsetroundjoin%
\pgfsetlinewidth{0.250937pt}%
\definecolor{currentstroke}{rgb}{0.000000,0.000000,0.000000}%
\pgfsetstrokecolor{currentstroke}%
\pgfsetdash{}{0pt}%
\pgfpathmoveto{\pgfqpoint{0.625000in}{1.252716in}}%
\pgfpathlineto{\pgfqpoint{0.633085in}{1.244737in}}%
\pgfpathlineto{\pgfqpoint{0.625000in}{1.236978in}}%
\pgfusepath{stroke}%
\end{pgfscope}%
\begin{pgfscope}%
\pgfpathrectangle{\pgfqpoint{0.625000in}{0.550000in}}{\pgfqpoint{3.875000in}{3.850000in}} %
\pgfusepath{clip}%
\pgfsetbuttcap%
\pgfsetroundjoin%
\pgfsetlinewidth{0.250937pt}%
\definecolor{currentstroke}{rgb}{0.000000,0.000000,0.000000}%
\pgfsetstrokecolor{currentstroke}%
\pgfsetdash{}{0pt}%
\pgfpathmoveto{\pgfqpoint{0.625000in}{1.406764in}}%
\pgfpathlineto{\pgfqpoint{0.633033in}{1.399123in}}%
\pgfpathlineto{\pgfqpoint{0.625000in}{1.391477in}}%
\pgfusepath{stroke}%
\end{pgfscope}%
\begin{pgfscope}%
\pgfpathrectangle{\pgfqpoint{0.625000in}{0.550000in}}{\pgfqpoint{3.875000in}{3.850000in}} %
\pgfusepath{clip}%
\pgfsetbuttcap%
\pgfsetroundjoin%
\pgfsetlinewidth{0.250937pt}%
\definecolor{currentstroke}{rgb}{0.000000,0.000000,0.000000}%
\pgfsetstrokecolor{currentstroke}%
\pgfsetdash{}{0pt}%
\pgfpathmoveto{\pgfqpoint{0.625000in}{1.561132in}}%
\pgfpathlineto{\pgfqpoint{0.633002in}{1.553509in}}%
\pgfpathlineto{\pgfqpoint{0.625000in}{1.545637in}}%
\pgfusepath{stroke}%
\end{pgfscope}%
\begin{pgfscope}%
\pgfpathrectangle{\pgfqpoint{0.625000in}{0.550000in}}{\pgfqpoint{3.875000in}{3.850000in}} %
\pgfusepath{clip}%
\pgfsetbuttcap%
\pgfsetroundjoin%
\pgfsetlinewidth{0.250937pt}%
\definecolor{currentstroke}{rgb}{0.000000,0.000000,0.000000}%
\pgfsetstrokecolor{currentstroke}%
\pgfsetdash{}{0pt}%
\pgfpathmoveto{\pgfqpoint{0.625000in}{1.715512in}}%
\pgfpathlineto{\pgfqpoint{0.634712in}{1.708693in}}%
\pgfpathlineto{\pgfqpoint{0.635565in}{1.707895in}}%
\pgfpathlineto{\pgfqpoint{0.634712in}{1.707096in}}%
\pgfpathlineto{\pgfqpoint{0.625000in}{1.700230in}}%
\pgfusepath{stroke}%
\end{pgfscope}%
\begin{pgfscope}%
\pgfpathrectangle{\pgfqpoint{0.625000in}{0.550000in}}{\pgfqpoint{3.875000in}{3.850000in}} %
\pgfusepath{clip}%
\pgfsetbuttcap%
\pgfsetroundjoin%
\pgfsetlinewidth{0.250937pt}%
\definecolor{currentstroke}{rgb}{0.000000,0.000000,0.000000}%
\pgfsetstrokecolor{currentstroke}%
\pgfsetdash{}{0pt}%
\pgfpathmoveto{\pgfqpoint{0.625000in}{1.869955in}}%
\pgfpathlineto{\pgfqpoint{0.632964in}{1.862281in}}%
\pgfpathlineto{\pgfqpoint{0.625000in}{1.854765in}}%
\pgfusepath{stroke}%
\end{pgfscope}%
\begin{pgfscope}%
\pgfpathrectangle{\pgfqpoint{0.625000in}{0.550000in}}{\pgfqpoint{3.875000in}{3.850000in}} %
\pgfusepath{clip}%
\pgfsetbuttcap%
\pgfsetroundjoin%
\pgfsetlinewidth{0.250937pt}%
\definecolor{currentstroke}{rgb}{0.000000,0.000000,0.000000}%
\pgfsetstrokecolor{currentstroke}%
\pgfsetdash{}{0pt}%
\pgfpathmoveto{\pgfqpoint{0.625000in}{2.024604in}}%
\pgfpathlineto{\pgfqpoint{0.632945in}{2.016667in}}%
\pgfpathlineto{\pgfqpoint{0.625000in}{2.008794in}}%
\pgfusepath{stroke}%
\end{pgfscope}%
\begin{pgfscope}%
\pgfpathrectangle{\pgfqpoint{0.625000in}{0.550000in}}{\pgfqpoint{3.875000in}{3.850000in}} %
\pgfusepath{clip}%
\pgfsetbuttcap%
\pgfsetroundjoin%
\pgfsetlinewidth{0.250937pt}%
\definecolor{currentstroke}{rgb}{0.000000,0.000000,0.000000}%
\pgfsetstrokecolor{currentstroke}%
\pgfsetdash{}{0pt}%
\pgfpathmoveto{\pgfqpoint{0.625000in}{2.178592in}}%
\pgfpathlineto{\pgfqpoint{0.632973in}{2.171053in}}%
\pgfpathlineto{\pgfqpoint{0.625000in}{2.163298in}}%
\pgfusepath{stroke}%
\end{pgfscope}%
\begin{pgfscope}%
\pgfpathrectangle{\pgfqpoint{0.625000in}{0.550000in}}{\pgfqpoint{3.875000in}{3.850000in}} %
\pgfusepath{clip}%
\pgfsetbuttcap%
\pgfsetroundjoin%
\pgfsetlinewidth{0.250937pt}%
\definecolor{currentstroke}{rgb}{0.000000,0.000000,0.000000}%
\pgfsetstrokecolor{currentstroke}%
\pgfsetdash{}{0pt}%
\pgfpathmoveto{\pgfqpoint{0.625000in}{2.332935in}}%
\pgfpathlineto{\pgfqpoint{0.633003in}{2.325439in}}%
\pgfpathlineto{\pgfqpoint{0.625000in}{2.317417in}}%
\pgfusepath{stroke}%
\end{pgfscope}%
\begin{pgfscope}%
\pgfpathrectangle{\pgfqpoint{0.625000in}{0.550000in}}{\pgfqpoint{3.875000in}{3.850000in}} %
\pgfusepath{clip}%
\pgfsetbuttcap%
\pgfsetroundjoin%
\pgfsetlinewidth{0.250937pt}%
\definecolor{currentstroke}{rgb}{0.000000,0.000000,0.000000}%
\pgfsetstrokecolor{currentstroke}%
\pgfsetdash{}{0pt}%
\pgfpathmoveto{\pgfqpoint{0.625000in}{2.488299in}}%
\pgfpathlineto{\pgfqpoint{0.626101in}{2.489474in}}%
\pgfpathlineto{\pgfqpoint{0.626832in}{2.499123in}}%
\pgfpathlineto{\pgfqpoint{0.628878in}{2.508772in}}%
\pgfpathlineto{\pgfqpoint{0.631453in}{2.518421in}}%
\pgfpathlineto{\pgfqpoint{0.634690in}{2.528070in}}%
\pgfpathlineto{\pgfqpoint{0.634712in}{2.528148in}}%
\pgfpathlineto{\pgfqpoint{0.639247in}{2.537719in}}%
\pgfpathlineto{\pgfqpoint{0.644154in}{2.547368in}}%
\pgfpathlineto{\pgfqpoint{0.644424in}{2.547971in}}%
\pgfpathlineto{\pgfqpoint{0.650542in}{2.557018in}}%
\pgfpathlineto{\pgfqpoint{0.654135in}{2.562687in}}%
\pgfpathlineto{\pgfqpoint{0.657845in}{2.566667in}}%
\pgfpathlineto{\pgfqpoint{0.663847in}{2.574062in}}%
\pgfpathlineto{\pgfqpoint{0.666451in}{2.576316in}}%
\pgfpathlineto{\pgfqpoint{0.673559in}{2.583544in}}%
\pgfpathlineto{\pgfqpoint{0.676842in}{2.585965in}}%
\pgfpathlineto{\pgfqpoint{0.683271in}{2.591558in}}%
\pgfpathlineto{\pgfqpoint{0.689522in}{2.595614in}}%
\pgfpathlineto{\pgfqpoint{0.692982in}{2.598256in}}%
\pgfpathlineto{\pgfqpoint{0.702694in}{2.603920in}}%
\pgfpathlineto{\pgfqpoint{0.705695in}{2.605263in}}%
\pgfpathlineto{\pgfqpoint{0.712406in}{2.608805in}}%
\pgfpathlineto{\pgfqpoint{0.722118in}{2.612770in}}%
\pgfpathlineto{\pgfqpoint{0.728928in}{2.614912in}}%
\pgfpathlineto{\pgfqpoint{0.731830in}{2.615980in}}%
\pgfpathlineto{\pgfqpoint{0.741541in}{2.618605in}}%
\pgfpathlineto{\pgfqpoint{0.751253in}{2.620497in}}%
\pgfpathlineto{\pgfqpoint{0.760965in}{2.621732in}}%
\pgfpathlineto{\pgfqpoint{0.770677in}{2.622345in}}%
\pgfpathlineto{\pgfqpoint{0.780388in}{2.622351in}}%
\pgfpathlineto{\pgfqpoint{0.790100in}{2.621743in}}%
\pgfpathlineto{\pgfqpoint{0.799812in}{2.620502in}}%
\pgfpathlineto{\pgfqpoint{0.809524in}{2.618601in}}%
\pgfpathlineto{\pgfqpoint{0.819236in}{2.616006in}}%
\pgfpathlineto{\pgfqpoint{0.822605in}{2.614912in}}%
\pgfpathlineto{\pgfqpoint{0.828947in}{2.612718in}}%
\pgfpathlineto{\pgfqpoint{0.838659in}{2.608650in}}%
\pgfpathlineto{\pgfqpoint{0.845513in}{2.605263in}}%
\pgfpathlineto{\pgfqpoint{0.848371in}{2.603721in}}%
\pgfpathlineto{\pgfqpoint{0.858083in}{2.597836in}}%
\pgfpathlineto{\pgfqpoint{0.861394in}{2.595614in}}%
\pgfpathlineto{\pgfqpoint{0.867794in}{2.590822in}}%
\pgfpathlineto{\pgfqpoint{0.873740in}{2.585965in}}%
\pgfpathlineto{\pgfqpoint{0.877506in}{2.582472in}}%
\pgfpathlineto{\pgfqpoint{0.883774in}{2.576316in}}%
\pgfpathlineto{\pgfqpoint{0.887218in}{2.572407in}}%
\pgfpathlineto{\pgfqpoint{0.892109in}{2.566667in}}%
\pgfpathlineto{\pgfqpoint{0.896930in}{2.560003in}}%
\pgfpathlineto{\pgfqpoint{0.899059in}{2.557018in}}%
\pgfpathlineto{\pgfqpoint{0.904890in}{2.547368in}}%
\pgfpathlineto{\pgfqpoint{0.906642in}{2.543870in}}%
\pgfpathlineto{\pgfqpoint{0.909760in}{2.537719in}}%
\pgfpathlineto{\pgfqpoint{0.913725in}{2.528070in}}%
\pgfpathlineto{\pgfqpoint{0.916353in}{2.519886in}}%
\pgfpathlineto{\pgfqpoint{0.916840in}{2.518421in}}%
\pgfpathlineto{\pgfqpoint{0.919305in}{2.508772in}}%
\pgfpathlineto{\pgfqpoint{0.921017in}{2.499123in}}%
\pgfpathlineto{\pgfqpoint{0.922024in}{2.489474in}}%
\pgfpathlineto{\pgfqpoint{0.922357in}{2.479825in}}%
\pgfpathlineto{\pgfqpoint{0.922024in}{2.470175in}}%
\pgfpathlineto{\pgfqpoint{0.921017in}{2.460526in}}%
\pgfpathlineto{\pgfqpoint{0.919305in}{2.450877in}}%
\pgfpathlineto{\pgfqpoint{0.916840in}{2.441228in}}%
\pgfpathlineto{\pgfqpoint{0.916353in}{2.439763in}}%
\pgfpathlineto{\pgfqpoint{0.913725in}{2.431579in}}%
\pgfpathlineto{\pgfqpoint{0.909760in}{2.421930in}}%
\pgfpathlineto{\pgfqpoint{0.906642in}{2.415779in}}%
\pgfpathlineto{\pgfqpoint{0.904890in}{2.412281in}}%
\pgfpathlineto{\pgfqpoint{0.899059in}{2.402632in}}%
\pgfpathlineto{\pgfqpoint{0.896930in}{2.399646in}}%
\pgfpathlineto{\pgfqpoint{0.892109in}{2.392982in}}%
\pgfpathlineto{\pgfqpoint{0.887218in}{2.387242in}}%
\pgfpathlineto{\pgfqpoint{0.883774in}{2.383333in}}%
\pgfpathlineto{\pgfqpoint{0.877506in}{2.377178in}}%
\pgfpathlineto{\pgfqpoint{0.873740in}{2.373684in}}%
\pgfpathlineto{\pgfqpoint{0.867794in}{2.368827in}}%
\pgfpathlineto{\pgfqpoint{0.861394in}{2.364035in}}%
\pgfpathlineto{\pgfqpoint{0.858083in}{2.361813in}}%
\pgfpathlineto{\pgfqpoint{0.848371in}{2.355928in}}%
\pgfpathlineto{\pgfqpoint{0.845513in}{2.354386in}}%
\pgfpathlineto{\pgfqpoint{0.838659in}{2.350999in}}%
\pgfpathlineto{\pgfqpoint{0.828947in}{2.346931in}}%
\pgfpathlineto{\pgfqpoint{0.822605in}{2.344737in}}%
\pgfpathlineto{\pgfqpoint{0.819236in}{2.343643in}}%
\pgfpathlineto{\pgfqpoint{0.809524in}{2.341048in}}%
\pgfpathlineto{\pgfqpoint{0.799812in}{2.339147in}}%
\pgfpathlineto{\pgfqpoint{0.790100in}{2.337907in}}%
\pgfpathlineto{\pgfqpoint{0.780388in}{2.337298in}}%
\pgfpathlineto{\pgfqpoint{0.770677in}{2.337304in}}%
\pgfpathlineto{\pgfqpoint{0.760965in}{2.337918in}}%
\pgfpathlineto{\pgfqpoint{0.751253in}{2.339152in}}%
\pgfpathlineto{\pgfqpoint{0.741541in}{2.341044in}}%
\pgfpathlineto{\pgfqpoint{0.731830in}{2.343669in}}%
\pgfpathlineto{\pgfqpoint{0.728928in}{2.344737in}}%
\pgfpathlineto{\pgfqpoint{0.722118in}{2.346880in}}%
\pgfpathlineto{\pgfqpoint{0.712406in}{2.350844in}}%
\pgfpathlineto{\pgfqpoint{0.705695in}{2.354386in}}%
\pgfpathlineto{\pgfqpoint{0.702694in}{2.355729in}}%
\pgfpathlineto{\pgfqpoint{0.692982in}{2.361393in}}%
\pgfpathlineto{\pgfqpoint{0.689522in}{2.364035in}}%
\pgfpathlineto{\pgfqpoint{0.683271in}{2.368091in}}%
\pgfpathlineto{\pgfqpoint{0.676842in}{2.373684in}}%
\pgfpathlineto{\pgfqpoint{0.673559in}{2.376105in}}%
\pgfpathlineto{\pgfqpoint{0.666451in}{2.383333in}}%
\pgfpathlineto{\pgfqpoint{0.663847in}{2.385587in}}%
\pgfpathlineto{\pgfqpoint{0.657845in}{2.392982in}}%
\pgfpathlineto{\pgfqpoint{0.654135in}{2.396963in}}%
\pgfpathlineto{\pgfqpoint{0.650542in}{2.402632in}}%
\pgfpathlineto{\pgfqpoint{0.644424in}{2.411678in}}%
\pgfpathlineto{\pgfqpoint{0.644154in}{2.412281in}}%
\pgfpathlineto{\pgfqpoint{0.639247in}{2.421930in}}%
\pgfpathlineto{\pgfqpoint{0.634712in}{2.431501in}}%
\pgfpathlineto{\pgfqpoint{0.634690in}{2.431579in}}%
\pgfpathlineto{\pgfqpoint{0.631453in}{2.441228in}}%
\pgfpathlineto{\pgfqpoint{0.628878in}{2.450877in}}%
\pgfpathlineto{\pgfqpoint{0.626836in}{2.460526in}}%
\pgfpathlineto{\pgfqpoint{0.626101in}{2.470175in}}%
\pgfpathlineto{\pgfqpoint{0.625000in}{2.471350in}}%
\pgfusepath{stroke}%
\end{pgfscope}%
\begin{pgfscope}%
\pgfpathrectangle{\pgfqpoint{0.625000in}{0.550000in}}{\pgfqpoint{3.875000in}{3.850000in}} %
\pgfusepath{clip}%
\pgfsetbuttcap%
\pgfsetroundjoin%
\pgfsetlinewidth{0.250937pt}%
\definecolor{currentstroke}{rgb}{0.000000,0.000000,0.000000}%
\pgfsetstrokecolor{currentstroke}%
\pgfsetdash{}{0pt}%
\pgfpathmoveto{\pgfqpoint{0.625000in}{2.642229in}}%
\pgfpathlineto{\pgfqpoint{0.632999in}{2.634211in}}%
\pgfpathlineto{\pgfqpoint{0.625000in}{2.626718in}}%
\pgfusepath{stroke}%
\end{pgfscope}%
\begin{pgfscope}%
\pgfpathrectangle{\pgfqpoint{0.625000in}{0.550000in}}{\pgfqpoint{3.875000in}{3.850000in}} %
\pgfusepath{clip}%
\pgfsetbuttcap%
\pgfsetroundjoin%
\pgfsetlinewidth{0.250937pt}%
\definecolor{currentstroke}{rgb}{0.000000,0.000000,0.000000}%
\pgfsetstrokecolor{currentstroke}%
\pgfsetdash{}{0pt}%
\pgfpathmoveto{\pgfqpoint{0.625000in}{2.796340in}}%
\pgfpathlineto{\pgfqpoint{0.632962in}{2.788596in}}%
\pgfpathlineto{\pgfqpoint{0.625000in}{2.781070in}}%
\pgfusepath{stroke}%
\end{pgfscope}%
\begin{pgfscope}%
\pgfpathrectangle{\pgfqpoint{0.625000in}{0.550000in}}{\pgfqpoint{3.875000in}{3.850000in}} %
\pgfusepath{clip}%
\pgfsetbuttcap%
\pgfsetroundjoin%
\pgfsetlinewidth{0.250937pt}%
\definecolor{currentstroke}{rgb}{0.000000,0.000000,0.000000}%
\pgfsetstrokecolor{currentstroke}%
\pgfsetdash{}{0pt}%
\pgfpathmoveto{\pgfqpoint{0.625000in}{2.950824in}}%
\pgfpathlineto{\pgfqpoint{0.632914in}{2.942982in}}%
\pgfpathlineto{\pgfqpoint{0.625000in}{2.935075in}}%
\pgfusepath{stroke}%
\end{pgfscope}%
\begin{pgfscope}%
\pgfpathrectangle{\pgfqpoint{0.625000in}{0.550000in}}{\pgfqpoint{3.875000in}{3.850000in}} %
\pgfusepath{clip}%
\pgfsetbuttcap%
\pgfsetroundjoin%
\pgfsetlinewidth{0.250937pt}%
\definecolor{currentstroke}{rgb}{0.000000,0.000000,0.000000}%
\pgfsetstrokecolor{currentstroke}%
\pgfsetdash{}{0pt}%
\pgfpathmoveto{\pgfqpoint{0.625000in}{3.104856in}}%
\pgfpathlineto{\pgfqpoint{0.632940in}{3.097368in}}%
\pgfpathlineto{\pgfqpoint{0.625000in}{3.089721in}}%
\pgfusepath{stroke}%
\end{pgfscope}%
\begin{pgfscope}%
\pgfpathrectangle{\pgfqpoint{0.625000in}{0.550000in}}{\pgfqpoint{3.875000in}{3.850000in}} %
\pgfusepath{clip}%
\pgfsetbuttcap%
\pgfsetroundjoin%
\pgfsetlinewidth{0.250937pt}%
\definecolor{currentstroke}{rgb}{0.000000,0.000000,0.000000}%
\pgfsetstrokecolor{currentstroke}%
\pgfsetdash{}{0pt}%
\pgfpathmoveto{\pgfqpoint{0.625000in}{3.259324in}}%
\pgfpathlineto{\pgfqpoint{0.634712in}{3.252553in}}%
\pgfpathlineto{\pgfqpoint{0.635565in}{3.251754in}}%
\pgfpathlineto{\pgfqpoint{0.634712in}{3.250956in}}%
\pgfpathlineto{\pgfqpoint{0.625000in}{3.244235in}}%
\pgfusepath{stroke}%
\end{pgfscope}%
\begin{pgfscope}%
\pgfpathrectangle{\pgfqpoint{0.625000in}{0.550000in}}{\pgfqpoint{3.875000in}{3.850000in}} %
\pgfusepath{clip}%
\pgfsetbuttcap%
\pgfsetroundjoin%
\pgfsetlinewidth{0.250937pt}%
\definecolor{currentstroke}{rgb}{0.000000,0.000000,0.000000}%
\pgfsetstrokecolor{currentstroke}%
\pgfsetdash{}{0pt}%
\pgfpathmoveto{\pgfqpoint{0.625000in}{3.413950in}}%
\pgfpathlineto{\pgfqpoint{0.632942in}{3.406140in}}%
\pgfpathlineto{\pgfqpoint{0.625000in}{3.398584in}}%
\pgfusepath{stroke}%
\end{pgfscope}%
\begin{pgfscope}%
\pgfpathrectangle{\pgfqpoint{0.625000in}{0.550000in}}{\pgfqpoint{3.875000in}{3.850000in}} %
\pgfusepath{clip}%
\pgfsetbuttcap%
\pgfsetroundjoin%
\pgfsetlinewidth{0.250937pt}%
\definecolor{currentstroke}{rgb}{0.000000,0.000000,0.000000}%
\pgfsetstrokecolor{currentstroke}%
\pgfsetdash{}{0pt}%
\pgfpathmoveto{\pgfqpoint{0.625000in}{3.568092in}}%
\pgfpathlineto{\pgfqpoint{0.632963in}{3.560526in}}%
\pgfpathlineto{\pgfqpoint{0.625000in}{3.552965in}}%
\pgfusepath{stroke}%
\end{pgfscope}%
\begin{pgfscope}%
\pgfpathrectangle{\pgfqpoint{0.625000in}{0.550000in}}{\pgfqpoint{3.875000in}{3.850000in}} %
\pgfusepath{clip}%
\pgfsetbuttcap%
\pgfsetroundjoin%
\pgfsetlinewidth{0.250937pt}%
\definecolor{currentstroke}{rgb}{0.000000,0.000000,0.000000}%
\pgfsetstrokecolor{currentstroke}%
\pgfsetdash{}{0pt}%
\pgfpathmoveto{\pgfqpoint{0.625000in}{3.722556in}}%
\pgfpathlineto{\pgfqpoint{0.632982in}{3.714912in}}%
\pgfpathlineto{\pgfqpoint{0.625000in}{3.707038in}}%
\pgfusepath{stroke}%
\end{pgfscope}%
\begin{pgfscope}%
\pgfpathrectangle{\pgfqpoint{0.625000in}{0.550000in}}{\pgfqpoint{3.875000in}{3.850000in}} %
\pgfusepath{clip}%
\pgfsetbuttcap%
\pgfsetroundjoin%
\pgfsetlinewidth{0.250937pt}%
\definecolor{currentstroke}{rgb}{0.000000,0.000000,0.000000}%
\pgfsetstrokecolor{currentstroke}%
\pgfsetdash{}{0pt}%
\pgfpathmoveto{\pgfqpoint{0.625000in}{3.876817in}}%
\pgfpathlineto{\pgfqpoint{0.632901in}{3.869298in}}%
\pgfpathlineto{\pgfqpoint{0.625000in}{3.861749in}}%
\pgfusepath{stroke}%
\end{pgfscope}%
\begin{pgfscope}%
\pgfpathrectangle{\pgfqpoint{0.625000in}{0.550000in}}{\pgfqpoint{3.875000in}{3.850000in}} %
\pgfusepath{clip}%
\pgfsetbuttcap%
\pgfsetroundjoin%
\pgfsetlinewidth{0.250937pt}%
\definecolor{currentstroke}{rgb}{0.000000,0.000000,0.000000}%
\pgfsetstrokecolor{currentstroke}%
\pgfsetdash{}{0pt}%
\pgfpathmoveto{\pgfqpoint{0.625000in}{4.031329in}}%
\pgfpathlineto{\pgfqpoint{0.634712in}{4.024483in}}%
\pgfpathlineto{\pgfqpoint{0.635565in}{4.023684in}}%
\pgfpathlineto{\pgfqpoint{0.634712in}{4.022886in}}%
\pgfpathlineto{\pgfqpoint{0.625000in}{4.016114in}}%
\pgfusepath{stroke}%
\end{pgfscope}%
\begin{pgfscope}%
\pgfpathrectangle{\pgfqpoint{0.625000in}{0.550000in}}{\pgfqpoint{3.875000in}{3.850000in}} %
\pgfusepath{clip}%
\pgfsetbuttcap%
\pgfsetroundjoin%
\pgfsetlinewidth{0.250937pt}%
\definecolor{currentstroke}{rgb}{0.000000,0.000000,0.000000}%
\pgfsetstrokecolor{currentstroke}%
\pgfsetdash{}{0pt}%
\pgfpathmoveto{\pgfqpoint{0.625000in}{4.185670in}}%
\pgfpathlineto{\pgfqpoint{0.632895in}{4.178070in}}%
\pgfpathlineto{\pgfqpoint{0.625000in}{4.170826in}}%
\pgfusepath{stroke}%
\end{pgfscope}%
\begin{pgfscope}%
\pgfpathrectangle{\pgfqpoint{0.625000in}{0.550000in}}{\pgfqpoint{3.875000in}{3.850000in}} %
\pgfusepath{clip}%
\pgfsetbuttcap%
\pgfsetroundjoin%
\pgfsetlinewidth{0.250937pt}%
\definecolor{currentstroke}{rgb}{0.000000,0.000000,0.000000}%
\pgfsetstrokecolor{currentstroke}%
\pgfsetdash{}{0pt}%
\pgfpathmoveto{\pgfqpoint{0.625000in}{4.340021in}}%
\pgfpathlineto{\pgfqpoint{0.632858in}{4.332456in}}%
\pgfpathlineto{\pgfqpoint{0.625000in}{4.324866in}}%
\pgfusepath{stroke}%
\end{pgfscope}%
\begin{pgfscope}%
\pgfpathrectangle{\pgfqpoint{0.625000in}{0.550000in}}{\pgfqpoint{3.875000in}{3.850000in}} %
\pgfusepath{clip}%
\pgfsetbuttcap%
\pgfsetroundjoin%
\pgfsetlinewidth{0.250937pt}%
\definecolor{currentstroke}{rgb}{0.000000,0.000000,0.000000}%
\pgfsetstrokecolor{currentstroke}%
\pgfsetdash{}{0pt}%
\pgfpathmoveto{\pgfqpoint{0.634712in}{1.178984in}}%
\pgfpathlineto{\pgfqpoint{0.629823in}{1.186842in}}%
\pgfpathlineto{\pgfqpoint{0.628118in}{1.196491in}}%
\pgfpathlineto{\pgfqpoint{0.634339in}{1.206140in}}%
\pgfpathlineto{\pgfqpoint{0.634712in}{1.206845in}}%
\pgfpathlineto{\pgfqpoint{0.644424in}{1.206952in}}%
\pgfpathlineto{\pgfqpoint{0.646191in}{1.206140in}}%
\pgfpathlineto{\pgfqpoint{0.654135in}{1.196562in}}%
\pgfpathlineto{\pgfqpoint{0.654183in}{1.196491in}}%
\pgfpathlineto{\pgfqpoint{0.654135in}{1.196183in}}%
\pgfpathlineto{\pgfqpoint{0.653276in}{1.186842in}}%
\pgfpathlineto{\pgfqpoint{0.644424in}{1.179660in}}%
\pgfpathlineto{\pgfqpoint{0.634712in}{1.178984in}}%
\pgfusepath{stroke}%
\end{pgfscope}%
\begin{pgfscope}%
\pgfpathrectangle{\pgfqpoint{0.625000in}{0.550000in}}{\pgfqpoint{3.875000in}{3.850000in}} %
\pgfusepath{clip}%
\pgfsetbuttcap%
\pgfsetroundjoin%
\pgfsetlinewidth{0.250937pt}%
\definecolor{currentstroke}{rgb}{0.000000,0.000000,0.000000}%
\pgfsetstrokecolor{currentstroke}%
\pgfsetdash{}{0pt}%
\pgfpathmoveto{\pgfqpoint{0.634712in}{3.752804in}}%
\pgfpathlineto{\pgfqpoint{0.634339in}{3.753509in}}%
\pgfpathlineto{\pgfqpoint{0.628118in}{3.763158in}}%
\pgfpathlineto{\pgfqpoint{0.629823in}{3.772807in}}%
\pgfpathlineto{\pgfqpoint{0.634712in}{3.780665in}}%
\pgfpathlineto{\pgfqpoint{0.644424in}{3.779989in}}%
\pgfpathlineto{\pgfqpoint{0.653276in}{3.772807in}}%
\pgfpathlineto{\pgfqpoint{0.654135in}{3.763466in}}%
\pgfpathlineto{\pgfqpoint{0.654183in}{3.763158in}}%
\pgfpathlineto{\pgfqpoint{0.654135in}{3.763087in}}%
\pgfpathlineto{\pgfqpoint{0.646191in}{3.753509in}}%
\pgfpathlineto{\pgfqpoint{0.644424in}{3.752697in}}%
\pgfpathlineto{\pgfqpoint{0.634712in}{3.752804in}}%
\pgfusepath{stroke}%
\end{pgfscope}%
\begin{pgfscope}%
\pgfpathrectangle{\pgfqpoint{0.625000in}{0.550000in}}{\pgfqpoint{3.875000in}{3.850000in}} %
\pgfusepath{clip}%
\pgfsetbuttcap%
\pgfsetroundjoin%
\pgfsetlinewidth{0.250937pt}%
\definecolor{currentstroke}{rgb}{0.000000,0.000000,0.000000}%
\pgfsetstrokecolor{currentstroke}%
\pgfsetdash{}{0pt}%
\pgfpathmoveto{\pgfqpoint{0.625000in}{0.635131in}}%
\pgfpathlineto{\pgfqpoint{0.633187in}{0.627193in}}%
\pgfpathlineto{\pgfqpoint{0.625000in}{0.619278in}}%
\pgfusepath{stroke}%
\end{pgfscope}%
\begin{pgfscope}%
\pgfpathrectangle{\pgfqpoint{0.625000in}{0.550000in}}{\pgfqpoint{3.875000in}{3.850000in}} %
\pgfusepath{clip}%
\pgfsetbuttcap%
\pgfsetroundjoin%
\pgfsetlinewidth{0.250937pt}%
\definecolor{currentstroke}{rgb}{0.000000,0.000000,0.000000}%
\pgfsetstrokecolor{currentstroke}%
\pgfsetdash{}{0pt}%
\pgfpathmoveto{\pgfqpoint{0.625000in}{0.789115in}}%
\pgfpathlineto{\pgfqpoint{0.633153in}{0.781579in}}%
\pgfpathlineto{\pgfqpoint{0.625000in}{0.773708in}}%
\pgfusepath{stroke}%
\end{pgfscope}%
\begin{pgfscope}%
\pgfpathrectangle{\pgfqpoint{0.625000in}{0.550000in}}{\pgfqpoint{3.875000in}{3.850000in}} %
\pgfusepath{clip}%
\pgfsetbuttcap%
\pgfsetroundjoin%
\pgfsetlinewidth{0.250937pt}%
\definecolor{currentstroke}{rgb}{0.000000,0.000000,0.000000}%
\pgfsetstrokecolor{currentstroke}%
\pgfsetdash{}{0pt}%
\pgfpathmoveto{\pgfqpoint{0.625000in}{0.943788in}}%
\pgfpathlineto{\pgfqpoint{0.634712in}{0.937576in}}%
\pgfpathlineto{\pgfqpoint{0.636434in}{0.935965in}}%
\pgfpathlineto{\pgfqpoint{0.634712in}{0.934354in}}%
\pgfpathlineto{\pgfqpoint{0.625000in}{0.928071in}}%
\pgfusepath{stroke}%
\end{pgfscope}%
\begin{pgfscope}%
\pgfpathrectangle{\pgfqpoint{0.625000in}{0.550000in}}{\pgfqpoint{3.875000in}{3.850000in}} %
\pgfusepath{clip}%
\pgfsetbuttcap%
\pgfsetroundjoin%
\pgfsetlinewidth{0.250937pt}%
\definecolor{currentstroke}{rgb}{0.000000,0.000000,0.000000}%
\pgfsetstrokecolor{currentstroke}%
\pgfsetdash{}{0pt}%
\pgfpathmoveto{\pgfqpoint{0.625000in}{1.098172in}}%
\pgfpathlineto{\pgfqpoint{0.633156in}{1.090351in}}%
\pgfpathlineto{\pgfqpoint{0.625000in}{1.082559in}}%
\pgfusepath{stroke}%
\end{pgfscope}%
\begin{pgfscope}%
\pgfpathrectangle{\pgfqpoint{0.625000in}{0.550000in}}{\pgfqpoint{3.875000in}{3.850000in}} %
\pgfusepath{clip}%
\pgfsetbuttcap%
\pgfsetroundjoin%
\pgfsetlinewidth{0.250937pt}%
\definecolor{currentstroke}{rgb}{0.000000,0.000000,0.000000}%
\pgfsetstrokecolor{currentstroke}%
\pgfsetdash{}{0pt}%
\pgfpathmoveto{\pgfqpoint{0.625000in}{1.252799in}}%
\pgfpathlineto{\pgfqpoint{0.633169in}{1.244737in}}%
\pgfpathlineto{\pgfqpoint{0.625000in}{1.236897in}}%
\pgfusepath{stroke}%
\end{pgfscope}%
\begin{pgfscope}%
\pgfpathrectangle{\pgfqpoint{0.625000in}{0.550000in}}{\pgfqpoint{3.875000in}{3.850000in}} %
\pgfusepath{clip}%
\pgfsetbuttcap%
\pgfsetroundjoin%
\pgfsetlinewidth{0.250937pt}%
\definecolor{currentstroke}{rgb}{0.000000,0.000000,0.000000}%
\pgfsetstrokecolor{currentstroke}%
\pgfsetdash{}{0pt}%
\pgfpathmoveto{\pgfqpoint{0.625000in}{1.406846in}}%
\pgfpathlineto{\pgfqpoint{0.633118in}{1.399123in}}%
\pgfpathlineto{\pgfqpoint{0.625000in}{1.391396in}}%
\pgfusepath{stroke}%
\end{pgfscope}%
\begin{pgfscope}%
\pgfpathrectangle{\pgfqpoint{0.625000in}{0.550000in}}{\pgfqpoint{3.875000in}{3.850000in}} %
\pgfusepath{clip}%
\pgfsetbuttcap%
\pgfsetroundjoin%
\pgfsetlinewidth{0.250937pt}%
\definecolor{currentstroke}{rgb}{0.000000,0.000000,0.000000}%
\pgfsetstrokecolor{currentstroke}%
\pgfsetdash{}{0pt}%
\pgfpathmoveto{\pgfqpoint{0.625000in}{1.561212in}}%
\pgfpathlineto{\pgfqpoint{0.633086in}{1.553509in}}%
\pgfpathlineto{\pgfqpoint{0.625000in}{1.545555in}}%
\pgfusepath{stroke}%
\end{pgfscope}%
\begin{pgfscope}%
\pgfpathrectangle{\pgfqpoint{0.625000in}{0.550000in}}{\pgfqpoint{3.875000in}{3.850000in}} %
\pgfusepath{clip}%
\pgfsetbuttcap%
\pgfsetroundjoin%
\pgfsetlinewidth{0.250937pt}%
\definecolor{currentstroke}{rgb}{0.000000,0.000000,0.000000}%
\pgfsetstrokecolor{currentstroke}%
\pgfsetdash{}{0pt}%
\pgfpathmoveto{\pgfqpoint{0.625000in}{1.715593in}}%
\pgfpathlineto{\pgfqpoint{0.634712in}{1.709506in}}%
\pgfpathlineto{\pgfqpoint{0.636434in}{1.707895in}}%
\pgfpathlineto{\pgfqpoint{0.634712in}{1.706284in}}%
\pgfpathlineto{\pgfqpoint{0.625000in}{1.700148in}}%
\pgfusepath{stroke}%
\end{pgfscope}%
\begin{pgfscope}%
\pgfpathrectangle{\pgfqpoint{0.625000in}{0.550000in}}{\pgfqpoint{3.875000in}{3.850000in}} %
\pgfusepath{clip}%
\pgfsetbuttcap%
\pgfsetroundjoin%
\pgfsetlinewidth{0.250937pt}%
\definecolor{currentstroke}{rgb}{0.000000,0.000000,0.000000}%
\pgfsetstrokecolor{currentstroke}%
\pgfsetdash{}{0pt}%
\pgfpathmoveto{\pgfqpoint{0.625000in}{1.870038in}}%
\pgfpathlineto{\pgfqpoint{0.633050in}{1.862281in}}%
\pgfpathlineto{\pgfqpoint{0.625000in}{1.854684in}}%
\pgfusepath{stroke}%
\end{pgfscope}%
\begin{pgfscope}%
\pgfpathrectangle{\pgfqpoint{0.625000in}{0.550000in}}{\pgfqpoint{3.875000in}{3.850000in}} %
\pgfusepath{clip}%
\pgfsetbuttcap%
\pgfsetroundjoin%
\pgfsetlinewidth{0.250937pt}%
\definecolor{currentstroke}{rgb}{0.000000,0.000000,0.000000}%
\pgfsetstrokecolor{currentstroke}%
\pgfsetdash{}{0pt}%
\pgfpathmoveto{\pgfqpoint{0.625000in}{2.024691in}}%
\pgfpathlineto{\pgfqpoint{0.633032in}{2.016667in}}%
\pgfpathlineto{\pgfqpoint{0.625000in}{2.008707in}}%
\pgfusepath{stroke}%
\end{pgfscope}%
\begin{pgfscope}%
\pgfpathrectangle{\pgfqpoint{0.625000in}{0.550000in}}{\pgfqpoint{3.875000in}{3.850000in}} %
\pgfusepath{clip}%
\pgfsetbuttcap%
\pgfsetroundjoin%
\pgfsetlinewidth{0.250937pt}%
\definecolor{currentstroke}{rgb}{0.000000,0.000000,0.000000}%
\pgfsetstrokecolor{currentstroke}%
\pgfsetdash{}{0pt}%
\pgfpathmoveto{\pgfqpoint{0.625000in}{2.178674in}}%
\pgfpathlineto{\pgfqpoint{0.633060in}{2.171053in}}%
\pgfpathlineto{\pgfqpoint{0.625000in}{2.163213in}}%
\pgfusepath{stroke}%
\end{pgfscope}%
\begin{pgfscope}%
\pgfpathrectangle{\pgfqpoint{0.625000in}{0.550000in}}{\pgfqpoint{3.875000in}{3.850000in}} %
\pgfusepath{clip}%
\pgfsetbuttcap%
\pgfsetroundjoin%
\pgfsetlinewidth{0.250937pt}%
\definecolor{currentstroke}{rgb}{0.000000,0.000000,0.000000}%
\pgfsetstrokecolor{currentstroke}%
\pgfsetdash{}{0pt}%
\pgfpathmoveto{\pgfqpoint{0.625000in}{2.333017in}}%
\pgfpathlineto{\pgfqpoint{0.633091in}{2.325439in}}%
\pgfpathlineto{\pgfqpoint{0.625000in}{2.317329in}}%
\pgfusepath{stroke}%
\end{pgfscope}%
\begin{pgfscope}%
\pgfpathrectangle{\pgfqpoint{0.625000in}{0.550000in}}{\pgfqpoint{3.875000in}{3.850000in}} %
\pgfusepath{clip}%
\pgfsetbuttcap%
\pgfsetroundjoin%
\pgfsetlinewidth{0.250937pt}%
\definecolor{currentstroke}{rgb}{0.000000,0.000000,0.000000}%
\pgfsetstrokecolor{currentstroke}%
\pgfsetdash{}{0pt}%
\pgfpathmoveto{\pgfqpoint{0.625000in}{2.488383in}}%
\pgfpathlineto{\pgfqpoint{0.626023in}{2.489474in}}%
\pgfpathlineto{\pgfqpoint{0.626753in}{2.499123in}}%
\pgfpathlineto{\pgfqpoint{0.628724in}{2.508772in}}%
\pgfpathlineto{\pgfqpoint{0.631190in}{2.518421in}}%
\pgfpathlineto{\pgfqpoint{0.634285in}{2.528070in}}%
\pgfpathlineto{\pgfqpoint{0.634712in}{2.529560in}}%
\pgfpathlineto{\pgfqpoint{0.638578in}{2.537719in}}%
\pgfpathlineto{\pgfqpoint{0.643287in}{2.547368in}}%
\pgfpathlineto{\pgfqpoint{0.644424in}{2.549908in}}%
\pgfpathlineto{\pgfqpoint{0.649232in}{2.557018in}}%
\pgfpathlineto{\pgfqpoint{0.654135in}{2.564754in}}%
\pgfpathlineto{\pgfqpoint{0.655918in}{2.566667in}}%
\pgfpathlineto{\pgfqpoint{0.663736in}{2.576316in}}%
\pgfpathlineto{\pgfqpoint{0.663847in}{2.576470in}}%
\pgfpathlineto{\pgfqpoint{0.673162in}{2.585965in}}%
\pgfpathlineto{\pgfqpoint{0.673559in}{2.586423in}}%
\pgfpathlineto{\pgfqpoint{0.683271in}{2.594849in}}%
\pgfpathlineto{\pgfqpoint{0.684449in}{2.595614in}}%
\pgfpathlineto{\pgfqpoint{0.692982in}{2.602129in}}%
\pgfpathlineto{\pgfqpoint{0.698350in}{2.605263in}}%
\pgfpathlineto{\pgfqpoint{0.702694in}{2.608229in}}%
\pgfpathlineto{\pgfqpoint{0.712406in}{2.613395in}}%
\pgfpathlineto{\pgfqpoint{0.716059in}{2.614912in}}%
\pgfpathlineto{\pgfqpoint{0.722118in}{2.617849in}}%
\pgfpathlineto{\pgfqpoint{0.731830in}{2.621526in}}%
\pgfpathlineto{\pgfqpoint{0.741541in}{2.624414in}}%
\pgfpathlineto{\pgfqpoint{0.742217in}{2.624561in}}%
\pgfpathlineto{\pgfqpoint{0.751253in}{2.626835in}}%
\pgfpathlineto{\pgfqpoint{0.760965in}{2.628583in}}%
\pgfpathlineto{\pgfqpoint{0.770677in}{2.629705in}}%
\pgfpathlineto{\pgfqpoint{0.780388in}{2.630233in}}%
\pgfpathlineto{\pgfqpoint{0.790100in}{2.630178in}}%
\pgfpathlineto{\pgfqpoint{0.799812in}{2.629537in}}%
\pgfpathlineto{\pgfqpoint{0.809524in}{2.628296in}}%
\pgfpathlineto{\pgfqpoint{0.819236in}{2.626431in}}%
\pgfpathlineto{\pgfqpoint{0.826587in}{2.624561in}}%
\pgfpathlineto{\pgfqpoint{0.828947in}{2.623930in}}%
\pgfpathlineto{\pgfqpoint{0.838659in}{2.620787in}}%
\pgfpathlineto{\pgfqpoint{0.848371in}{2.616899in}}%
\pgfpathlineto{\pgfqpoint{0.852690in}{2.614912in}}%
\pgfpathlineto{\pgfqpoint{0.858083in}{2.612228in}}%
\pgfpathlineto{\pgfqpoint{0.867794in}{2.606689in}}%
\pgfpathlineto{\pgfqpoint{0.870076in}{2.605263in}}%
\pgfpathlineto{\pgfqpoint{0.877506in}{2.600140in}}%
\pgfpathlineto{\pgfqpoint{0.883472in}{2.595614in}}%
\pgfpathlineto{\pgfqpoint{0.887218in}{2.592429in}}%
\pgfpathlineto{\pgfqpoint{0.894340in}{2.585965in}}%
\pgfpathlineto{\pgfqpoint{0.896930in}{2.583288in}}%
\pgfpathlineto{\pgfqpoint{0.903401in}{2.576316in}}%
\pgfpathlineto{\pgfqpoint{0.906642in}{2.572274in}}%
\pgfpathlineto{\pgfqpoint{0.911038in}{2.566667in}}%
\pgfpathlineto{\pgfqpoint{0.916353in}{2.558679in}}%
\pgfpathlineto{\pgfqpoint{0.917452in}{2.557018in}}%
\pgfpathlineto{\pgfqpoint{0.922927in}{2.547368in}}%
\pgfpathlineto{\pgfqpoint{0.926065in}{2.540633in}}%
\pgfpathlineto{\pgfqpoint{0.927444in}{2.537719in}}%
\pgfpathlineto{\pgfqpoint{0.931219in}{2.528070in}}%
\pgfpathlineto{\pgfqpoint{0.934173in}{2.518421in}}%
\pgfpathlineto{\pgfqpoint{0.935777in}{2.511481in}}%
\pgfpathlineto{\pgfqpoint{0.936430in}{2.508772in}}%
\pgfpathlineto{\pgfqpoint{0.938075in}{2.499123in}}%
\pgfpathlineto{\pgfqpoint{0.939046in}{2.489474in}}%
\pgfpathlineto{\pgfqpoint{0.939367in}{2.479825in}}%
\pgfpathlineto{\pgfqpoint{0.939046in}{2.470175in}}%
\pgfpathlineto{\pgfqpoint{0.938075in}{2.460526in}}%
\pgfpathlineto{\pgfqpoint{0.936430in}{2.450877in}}%
\pgfpathlineto{\pgfqpoint{0.935777in}{2.448168in}}%
\pgfpathlineto{\pgfqpoint{0.934173in}{2.441228in}}%
\pgfpathlineto{\pgfqpoint{0.931219in}{2.431579in}}%
\pgfpathlineto{\pgfqpoint{0.927444in}{2.421930in}}%
\pgfpathlineto{\pgfqpoint{0.926065in}{2.419016in}}%
\pgfpathlineto{\pgfqpoint{0.922927in}{2.412281in}}%
\pgfpathlineto{\pgfqpoint{0.917452in}{2.402632in}}%
\pgfpathlineto{\pgfqpoint{0.916353in}{2.400970in}}%
\pgfpathlineto{\pgfqpoint{0.911038in}{2.392982in}}%
\pgfpathlineto{\pgfqpoint{0.906642in}{2.387375in}}%
\pgfpathlineto{\pgfqpoint{0.903401in}{2.383333in}}%
\pgfpathlineto{\pgfqpoint{0.896930in}{2.376361in}}%
\pgfpathlineto{\pgfqpoint{0.894340in}{2.373684in}}%
\pgfpathlineto{\pgfqpoint{0.887218in}{2.367220in}}%
\pgfpathlineto{\pgfqpoint{0.883472in}{2.364035in}}%
\pgfpathlineto{\pgfqpoint{0.877506in}{2.359509in}}%
\pgfpathlineto{\pgfqpoint{0.870076in}{2.354386in}}%
\pgfpathlineto{\pgfqpoint{0.867794in}{2.352960in}}%
\pgfpathlineto{\pgfqpoint{0.858083in}{2.347421in}}%
\pgfpathlineto{\pgfqpoint{0.852690in}{2.344737in}}%
\pgfpathlineto{\pgfqpoint{0.848371in}{2.342750in}}%
\pgfpathlineto{\pgfqpoint{0.838659in}{2.338862in}}%
\pgfpathlineto{\pgfqpoint{0.828947in}{2.335719in}}%
\pgfpathlineto{\pgfqpoint{0.826587in}{2.335088in}}%
\pgfpathlineto{\pgfqpoint{0.819236in}{2.333218in}}%
\pgfpathlineto{\pgfqpoint{0.809524in}{2.331353in}}%
\pgfpathlineto{\pgfqpoint{0.799812in}{2.330112in}}%
\pgfpathlineto{\pgfqpoint{0.790100in}{2.329471in}}%
\pgfpathlineto{\pgfqpoint{0.780388in}{2.329417in}}%
\pgfpathlineto{\pgfqpoint{0.770677in}{2.329945in}}%
\pgfpathlineto{\pgfqpoint{0.760965in}{2.331066in}}%
\pgfpathlineto{\pgfqpoint{0.751253in}{2.332814in}}%
\pgfpathlineto{\pgfqpoint{0.742217in}{2.335088in}}%
\pgfpathlineto{\pgfqpoint{0.741541in}{2.335235in}}%
\pgfpathlineto{\pgfqpoint{0.731830in}{2.338123in}}%
\pgfpathlineto{\pgfqpoint{0.722118in}{2.341800in}}%
\pgfpathlineto{\pgfqpoint{0.716059in}{2.344737in}}%
\pgfpathlineto{\pgfqpoint{0.712406in}{2.346254in}}%
\pgfpathlineto{\pgfqpoint{0.702694in}{2.351420in}}%
\pgfpathlineto{\pgfqpoint{0.698350in}{2.354386in}}%
\pgfpathlineto{\pgfqpoint{0.692982in}{2.357520in}}%
\pgfpathlineto{\pgfqpoint{0.684449in}{2.364035in}}%
\pgfpathlineto{\pgfqpoint{0.683271in}{2.364800in}}%
\pgfpathlineto{\pgfqpoint{0.673559in}{2.373226in}}%
\pgfpathlineto{\pgfqpoint{0.673162in}{2.373684in}}%
\pgfpathlineto{\pgfqpoint{0.663847in}{2.383179in}}%
\pgfpathlineto{\pgfqpoint{0.663736in}{2.383333in}}%
\pgfpathlineto{\pgfqpoint{0.655918in}{2.392982in}}%
\pgfpathlineto{\pgfqpoint{0.654135in}{2.394895in}}%
\pgfpathlineto{\pgfqpoint{0.649232in}{2.402632in}}%
\pgfpathlineto{\pgfqpoint{0.644424in}{2.409741in}}%
\pgfpathlineto{\pgfqpoint{0.643287in}{2.412281in}}%
\pgfpathlineto{\pgfqpoint{0.638578in}{2.421930in}}%
\pgfpathlineto{\pgfqpoint{0.634712in}{2.430089in}}%
\pgfpathlineto{\pgfqpoint{0.634285in}{2.431579in}}%
\pgfpathlineto{\pgfqpoint{0.631190in}{2.441228in}}%
\pgfpathlineto{\pgfqpoint{0.628724in}{2.450877in}}%
\pgfpathlineto{\pgfqpoint{0.626757in}{2.460526in}}%
\pgfpathlineto{\pgfqpoint{0.626023in}{2.470175in}}%
\pgfpathlineto{\pgfqpoint{0.625000in}{2.471266in}}%
\pgfusepath{stroke}%
\end{pgfscope}%
\begin{pgfscope}%
\pgfpathrectangle{\pgfqpoint{0.625000in}{0.550000in}}{\pgfqpoint{3.875000in}{3.850000in}} %
\pgfusepath{clip}%
\pgfsetbuttcap%
\pgfsetroundjoin%
\pgfsetlinewidth{0.250937pt}%
\definecolor{currentstroke}{rgb}{0.000000,0.000000,0.000000}%
\pgfsetstrokecolor{currentstroke}%
\pgfsetdash{}{0pt}%
\pgfpathmoveto{\pgfqpoint{0.625000in}{2.642317in}}%
\pgfpathlineto{\pgfqpoint{0.633088in}{2.634211in}}%
\pgfpathlineto{\pgfqpoint{0.625000in}{2.626635in}}%
\pgfusepath{stroke}%
\end{pgfscope}%
\begin{pgfscope}%
\pgfpathrectangle{\pgfqpoint{0.625000in}{0.550000in}}{\pgfqpoint{3.875000in}{3.850000in}} %
\pgfusepath{clip}%
\pgfsetbuttcap%
\pgfsetroundjoin%
\pgfsetlinewidth{0.250937pt}%
\definecolor{currentstroke}{rgb}{0.000000,0.000000,0.000000}%
\pgfsetstrokecolor{currentstroke}%
\pgfsetdash{}{0pt}%
\pgfpathmoveto{\pgfqpoint{0.625000in}{2.796425in}}%
\pgfpathlineto{\pgfqpoint{0.633049in}{2.788596in}}%
\pgfpathlineto{\pgfqpoint{0.625000in}{2.780987in}}%
\pgfusepath{stroke}%
\end{pgfscope}%
\begin{pgfscope}%
\pgfpathrectangle{\pgfqpoint{0.625000in}{0.550000in}}{\pgfqpoint{3.875000in}{3.850000in}} %
\pgfusepath{clip}%
\pgfsetbuttcap%
\pgfsetroundjoin%
\pgfsetlinewidth{0.250937pt}%
\definecolor{currentstroke}{rgb}{0.000000,0.000000,0.000000}%
\pgfsetstrokecolor{currentstroke}%
\pgfsetdash{}{0pt}%
\pgfpathmoveto{\pgfqpoint{0.625000in}{2.950912in}}%
\pgfpathlineto{\pgfqpoint{0.633003in}{2.942982in}}%
\pgfpathlineto{\pgfqpoint{0.625000in}{2.934986in}}%
\pgfusepath{stroke}%
\end{pgfscope}%
\begin{pgfscope}%
\pgfpathrectangle{\pgfqpoint{0.625000in}{0.550000in}}{\pgfqpoint{3.875000in}{3.850000in}} %
\pgfusepath{clip}%
\pgfsetbuttcap%
\pgfsetroundjoin%
\pgfsetlinewidth{0.250937pt}%
\definecolor{currentstroke}{rgb}{0.000000,0.000000,0.000000}%
\pgfsetstrokecolor{currentstroke}%
\pgfsetdash{}{0pt}%
\pgfpathmoveto{\pgfqpoint{0.625000in}{3.104938in}}%
\pgfpathlineto{\pgfqpoint{0.633028in}{3.097368in}}%
\pgfpathlineto{\pgfqpoint{0.625000in}{3.089637in}}%
\pgfusepath{stroke}%
\end{pgfscope}%
\begin{pgfscope}%
\pgfpathrectangle{\pgfqpoint{0.625000in}{0.550000in}}{\pgfqpoint{3.875000in}{3.850000in}} %
\pgfusepath{clip}%
\pgfsetbuttcap%
\pgfsetroundjoin%
\pgfsetlinewidth{0.250937pt}%
\definecolor{currentstroke}{rgb}{0.000000,0.000000,0.000000}%
\pgfsetstrokecolor{currentstroke}%
\pgfsetdash{}{0pt}%
\pgfpathmoveto{\pgfqpoint{0.625000in}{3.259409in}}%
\pgfpathlineto{\pgfqpoint{0.634712in}{3.253365in}}%
\pgfpathlineto{\pgfqpoint{0.636434in}{3.251754in}}%
\pgfpathlineto{\pgfqpoint{0.634712in}{3.250143in}}%
\pgfpathlineto{\pgfqpoint{0.625000in}{3.244150in}}%
\pgfusepath{stroke}%
\end{pgfscope}%
\begin{pgfscope}%
\pgfpathrectangle{\pgfqpoint{0.625000in}{0.550000in}}{\pgfqpoint{3.875000in}{3.850000in}} %
\pgfusepath{clip}%
\pgfsetbuttcap%
\pgfsetroundjoin%
\pgfsetlinewidth{0.250937pt}%
\definecolor{currentstroke}{rgb}{0.000000,0.000000,0.000000}%
\pgfsetstrokecolor{currentstroke}%
\pgfsetdash{}{0pt}%
\pgfpathmoveto{\pgfqpoint{0.625000in}{3.414036in}}%
\pgfpathlineto{\pgfqpoint{0.633029in}{3.406140in}}%
\pgfpathlineto{\pgfqpoint{0.625000in}{3.398502in}}%
\pgfusepath{stroke}%
\end{pgfscope}%
\begin{pgfscope}%
\pgfpathrectangle{\pgfqpoint{0.625000in}{0.550000in}}{\pgfqpoint{3.875000in}{3.850000in}} %
\pgfusepath{clip}%
\pgfsetbuttcap%
\pgfsetroundjoin%
\pgfsetlinewidth{0.250937pt}%
\definecolor{currentstroke}{rgb}{0.000000,0.000000,0.000000}%
\pgfsetstrokecolor{currentstroke}%
\pgfsetdash{}{0pt}%
\pgfpathmoveto{\pgfqpoint{0.625000in}{3.568177in}}%
\pgfpathlineto{\pgfqpoint{0.633052in}{3.560526in}}%
\pgfpathlineto{\pgfqpoint{0.625000in}{3.552880in}}%
\pgfusepath{stroke}%
\end{pgfscope}%
\begin{pgfscope}%
\pgfpathrectangle{\pgfqpoint{0.625000in}{0.550000in}}{\pgfqpoint{3.875000in}{3.850000in}} %
\pgfusepath{clip}%
\pgfsetbuttcap%
\pgfsetroundjoin%
\pgfsetlinewidth{0.250937pt}%
\definecolor{currentstroke}{rgb}{0.000000,0.000000,0.000000}%
\pgfsetstrokecolor{currentstroke}%
\pgfsetdash{}{0pt}%
\pgfpathmoveto{\pgfqpoint{0.625000in}{3.722642in}}%
\pgfpathlineto{\pgfqpoint{0.633071in}{3.714912in}}%
\pgfpathlineto{\pgfqpoint{0.625000in}{3.706950in}}%
\pgfusepath{stroke}%
\end{pgfscope}%
\begin{pgfscope}%
\pgfpathrectangle{\pgfqpoint{0.625000in}{0.550000in}}{\pgfqpoint{3.875000in}{3.850000in}} %
\pgfusepath{clip}%
\pgfsetbuttcap%
\pgfsetroundjoin%
\pgfsetlinewidth{0.250937pt}%
\definecolor{currentstroke}{rgb}{0.000000,0.000000,0.000000}%
\pgfsetstrokecolor{currentstroke}%
\pgfsetdash{}{0pt}%
\pgfpathmoveto{\pgfqpoint{0.625000in}{3.876904in}}%
\pgfpathlineto{\pgfqpoint{0.632992in}{3.869298in}}%
\pgfpathlineto{\pgfqpoint{0.625000in}{3.861662in}}%
\pgfusepath{stroke}%
\end{pgfscope}%
\begin{pgfscope}%
\pgfpathrectangle{\pgfqpoint{0.625000in}{0.550000in}}{\pgfqpoint{3.875000in}{3.850000in}} %
\pgfusepath{clip}%
\pgfsetbuttcap%
\pgfsetroundjoin%
\pgfsetlinewidth{0.250937pt}%
\definecolor{currentstroke}{rgb}{0.000000,0.000000,0.000000}%
\pgfsetstrokecolor{currentstroke}%
\pgfsetdash{}{0pt}%
\pgfpathmoveto{\pgfqpoint{0.625000in}{4.031413in}}%
\pgfpathlineto{\pgfqpoint{0.634712in}{4.025295in}}%
\pgfpathlineto{\pgfqpoint{0.636434in}{4.023684in}}%
\pgfpathlineto{\pgfqpoint{0.634712in}{4.022073in}}%
\pgfpathlineto{\pgfqpoint{0.625000in}{4.016030in}}%
\pgfusepath{stroke}%
\end{pgfscope}%
\begin{pgfscope}%
\pgfpathrectangle{\pgfqpoint{0.625000in}{0.550000in}}{\pgfqpoint{3.875000in}{3.850000in}} %
\pgfusepath{clip}%
\pgfsetbuttcap%
\pgfsetroundjoin%
\pgfsetlinewidth{0.250937pt}%
\definecolor{currentstroke}{rgb}{0.000000,0.000000,0.000000}%
\pgfsetstrokecolor{currentstroke}%
\pgfsetdash{}{0pt}%
\pgfpathmoveto{\pgfqpoint{0.625000in}{4.185759in}}%
\pgfpathlineto{\pgfqpoint{0.632987in}{4.178070in}}%
\pgfpathlineto{\pgfqpoint{0.625000in}{4.170742in}}%
\pgfusepath{stroke}%
\end{pgfscope}%
\begin{pgfscope}%
\pgfpathrectangle{\pgfqpoint{0.625000in}{0.550000in}}{\pgfqpoint{3.875000in}{3.850000in}} %
\pgfusepath{clip}%
\pgfsetbuttcap%
\pgfsetroundjoin%
\pgfsetlinewidth{0.250937pt}%
\definecolor{currentstroke}{rgb}{0.000000,0.000000,0.000000}%
\pgfsetstrokecolor{currentstroke}%
\pgfsetdash{}{0pt}%
\pgfpathmoveto{\pgfqpoint{0.625000in}{4.340108in}}%
\pgfpathlineto{\pgfqpoint{0.632949in}{4.332456in}}%
\pgfpathlineto{\pgfqpoint{0.625000in}{4.324778in}}%
\pgfusepath{stroke}%
\end{pgfscope}%
\begin{pgfscope}%
\pgfpathrectangle{\pgfqpoint{0.625000in}{0.550000in}}{\pgfqpoint{3.875000in}{3.850000in}} %
\pgfusepath{clip}%
\pgfsetbuttcap%
\pgfsetroundjoin%
\pgfsetlinewidth{0.250937pt}%
\definecolor{currentstroke}{rgb}{0.000000,0.000000,0.000000}%
\pgfsetstrokecolor{currentstroke}%
\pgfsetdash{}{0pt}%
\pgfpathmoveto{\pgfqpoint{0.634712in}{1.178650in}}%
\pgfpathlineto{\pgfqpoint{0.629615in}{1.186842in}}%
\pgfpathlineto{\pgfqpoint{0.627944in}{1.196491in}}%
\pgfpathlineto{\pgfqpoint{0.633959in}{1.206140in}}%
\pgfpathlineto{\pgfqpoint{0.634712in}{1.207562in}}%
\pgfpathlineto{\pgfqpoint{0.644424in}{1.208038in}}%
\pgfpathlineto{\pgfqpoint{0.648554in}{1.206140in}}%
\pgfpathlineto{\pgfqpoint{0.654135in}{1.199411in}}%
\pgfpathlineto{\pgfqpoint{0.656093in}{1.196491in}}%
\pgfpathlineto{\pgfqpoint{0.654657in}{1.186842in}}%
\pgfpathlineto{\pgfqpoint{0.654135in}{1.186265in}}%
\pgfpathlineto{\pgfqpoint{0.644424in}{1.178732in}}%
\pgfpathlineto{\pgfqpoint{0.634712in}{1.178650in}}%
\pgfusepath{stroke}%
\end{pgfscope}%
\begin{pgfscope}%
\pgfpathrectangle{\pgfqpoint{0.625000in}{0.550000in}}{\pgfqpoint{3.875000in}{3.850000in}} %
\pgfusepath{clip}%
\pgfsetbuttcap%
\pgfsetroundjoin%
\pgfsetlinewidth{0.250937pt}%
\definecolor{currentstroke}{rgb}{0.000000,0.000000,0.000000}%
\pgfsetstrokecolor{currentstroke}%
\pgfsetdash{}{0pt}%
\pgfpathmoveto{\pgfqpoint{0.634712in}{3.752087in}}%
\pgfpathlineto{\pgfqpoint{0.633959in}{3.753509in}}%
\pgfpathlineto{\pgfqpoint{0.627944in}{3.763158in}}%
\pgfpathlineto{\pgfqpoint{0.629615in}{3.772807in}}%
\pgfpathlineto{\pgfqpoint{0.634712in}{3.780999in}}%
\pgfpathlineto{\pgfqpoint{0.644424in}{3.780917in}}%
\pgfpathlineto{\pgfqpoint{0.654135in}{3.773384in}}%
\pgfpathlineto{\pgfqpoint{0.654657in}{3.772807in}}%
\pgfpathlineto{\pgfqpoint{0.656093in}{3.763158in}}%
\pgfpathlineto{\pgfqpoint{0.654135in}{3.760238in}}%
\pgfpathlineto{\pgfqpoint{0.648554in}{3.753509in}}%
\pgfpathlineto{\pgfqpoint{0.644424in}{3.751612in}}%
\pgfpathlineto{\pgfqpoint{0.634712in}{3.752087in}}%
\pgfusepath{stroke}%
\end{pgfscope}%
\begin{pgfscope}%
\pgfpathrectangle{\pgfqpoint{0.625000in}{0.550000in}}{\pgfqpoint{3.875000in}{3.850000in}} %
\pgfusepath{clip}%
\pgfsetbuttcap%
\pgfsetroundjoin%
\pgfsetlinewidth{0.250937pt}%
\definecolor{currentstroke}{rgb}{0.000000,0.000000,0.000000}%
\pgfsetstrokecolor{currentstroke}%
\pgfsetdash{}{0pt}%
\pgfpathmoveto{\pgfqpoint{0.625000in}{0.635207in}}%
\pgfpathlineto{\pgfqpoint{0.633266in}{0.627193in}}%
\pgfpathlineto{\pgfqpoint{0.625000in}{0.619202in}}%
\pgfusepath{stroke}%
\end{pgfscope}%
\begin{pgfscope}%
\pgfpathrectangle{\pgfqpoint{0.625000in}{0.550000in}}{\pgfqpoint{3.875000in}{3.850000in}} %
\pgfusepath{clip}%
\pgfsetbuttcap%
\pgfsetroundjoin%
\pgfsetlinewidth{0.250937pt}%
\definecolor{currentstroke}{rgb}{0.000000,0.000000,0.000000}%
\pgfsetstrokecolor{currentstroke}%
\pgfsetdash{}{0pt}%
\pgfpathmoveto{\pgfqpoint{0.625000in}{0.789192in}}%
\pgfpathlineto{\pgfqpoint{0.633236in}{0.781579in}}%
\pgfpathlineto{\pgfqpoint{0.625000in}{0.773627in}}%
\pgfusepath{stroke}%
\end{pgfscope}%
\begin{pgfscope}%
\pgfpathrectangle{\pgfqpoint{0.625000in}{0.550000in}}{\pgfqpoint{3.875000in}{3.850000in}} %
\pgfusepath{clip}%
\pgfsetbuttcap%
\pgfsetroundjoin%
\pgfsetlinewidth{0.250937pt}%
\definecolor{currentstroke}{rgb}{0.000000,0.000000,0.000000}%
\pgfsetstrokecolor{currentstroke}%
\pgfsetdash{}{0pt}%
\pgfpathmoveto{\pgfqpoint{0.625000in}{0.943865in}}%
\pgfpathlineto{\pgfqpoint{0.634712in}{0.938389in}}%
\pgfpathlineto{\pgfqpoint{0.637302in}{0.935965in}}%
\pgfpathlineto{\pgfqpoint{0.634712in}{0.933541in}}%
\pgfpathlineto{\pgfqpoint{0.625000in}{0.927994in}}%
\pgfusepath{stroke}%
\end{pgfscope}%
\begin{pgfscope}%
\pgfpathrectangle{\pgfqpoint{0.625000in}{0.550000in}}{\pgfqpoint{3.875000in}{3.850000in}} %
\pgfusepath{clip}%
\pgfsetbuttcap%
\pgfsetroundjoin%
\pgfsetlinewidth{0.250937pt}%
\definecolor{currentstroke}{rgb}{0.000000,0.000000,0.000000}%
\pgfsetstrokecolor{currentstroke}%
\pgfsetdash{}{0pt}%
\pgfpathmoveto{\pgfqpoint{0.625000in}{1.098251in}}%
\pgfpathlineto{\pgfqpoint{0.633238in}{1.090351in}}%
\pgfpathlineto{\pgfqpoint{0.625000in}{1.082480in}}%
\pgfusepath{stroke}%
\end{pgfscope}%
\begin{pgfscope}%
\pgfpathrectangle{\pgfqpoint{0.625000in}{0.550000in}}{\pgfqpoint{3.875000in}{3.850000in}} %
\pgfusepath{clip}%
\pgfsetbuttcap%
\pgfsetroundjoin%
\pgfsetlinewidth{0.250937pt}%
\definecolor{currentstroke}{rgb}{0.000000,0.000000,0.000000}%
\pgfsetstrokecolor{currentstroke}%
\pgfsetdash{}{0pt}%
\pgfpathmoveto{\pgfqpoint{0.625000in}{1.252882in}}%
\pgfpathlineto{\pgfqpoint{0.633253in}{1.244737in}}%
\pgfpathlineto{\pgfqpoint{0.625000in}{1.236817in}}%
\pgfusepath{stroke}%
\end{pgfscope}%
\begin{pgfscope}%
\pgfpathrectangle{\pgfqpoint{0.625000in}{0.550000in}}{\pgfqpoint{3.875000in}{3.850000in}} %
\pgfusepath{clip}%
\pgfsetbuttcap%
\pgfsetroundjoin%
\pgfsetlinewidth{0.250937pt}%
\definecolor{currentstroke}{rgb}{0.000000,0.000000,0.000000}%
\pgfsetstrokecolor{currentstroke}%
\pgfsetdash{}{0pt}%
\pgfpathmoveto{\pgfqpoint{0.625000in}{1.406927in}}%
\pgfpathlineto{\pgfqpoint{0.633204in}{1.399123in}}%
\pgfpathlineto{\pgfqpoint{0.625000in}{1.391314in}}%
\pgfusepath{stroke}%
\end{pgfscope}%
\begin{pgfscope}%
\pgfpathrectangle{\pgfqpoint{0.625000in}{0.550000in}}{\pgfqpoint{3.875000in}{3.850000in}} %
\pgfusepath{clip}%
\pgfsetbuttcap%
\pgfsetroundjoin%
\pgfsetlinewidth{0.250937pt}%
\definecolor{currentstroke}{rgb}{0.000000,0.000000,0.000000}%
\pgfsetstrokecolor{currentstroke}%
\pgfsetdash{}{0pt}%
\pgfpathmoveto{\pgfqpoint{0.625000in}{1.561292in}}%
\pgfpathlineto{\pgfqpoint{0.633169in}{1.553509in}}%
\pgfpathlineto{\pgfqpoint{0.625000in}{1.545472in}}%
\pgfusepath{stroke}%
\end{pgfscope}%
\begin{pgfscope}%
\pgfpathrectangle{\pgfqpoint{0.625000in}{0.550000in}}{\pgfqpoint{3.875000in}{3.850000in}} %
\pgfusepath{clip}%
\pgfsetbuttcap%
\pgfsetroundjoin%
\pgfsetlinewidth{0.250937pt}%
\definecolor{currentstroke}{rgb}{0.000000,0.000000,0.000000}%
\pgfsetstrokecolor{currentstroke}%
\pgfsetdash{}{0pt}%
\pgfpathmoveto{\pgfqpoint{0.625000in}{1.715674in}}%
\pgfpathlineto{\pgfqpoint{0.634712in}{1.710318in}}%
\pgfpathlineto{\pgfqpoint{0.637302in}{1.707895in}}%
\pgfpathlineto{\pgfqpoint{0.634712in}{1.705471in}}%
\pgfpathlineto{\pgfqpoint{0.625000in}{1.700067in}}%
\pgfusepath{stroke}%
\end{pgfscope}%
\begin{pgfscope}%
\pgfpathrectangle{\pgfqpoint{0.625000in}{0.550000in}}{\pgfqpoint{3.875000in}{3.850000in}} %
\pgfusepath{clip}%
\pgfsetbuttcap%
\pgfsetroundjoin%
\pgfsetlinewidth{0.250937pt}%
\definecolor{currentstroke}{rgb}{0.000000,0.000000,0.000000}%
\pgfsetstrokecolor{currentstroke}%
\pgfsetdash{}{0pt}%
\pgfpathmoveto{\pgfqpoint{0.625000in}{1.870120in}}%
\pgfpathlineto{\pgfqpoint{0.633137in}{1.862281in}}%
\pgfpathlineto{\pgfqpoint{0.625000in}{1.854603in}}%
\pgfusepath{stroke}%
\end{pgfscope}%
\begin{pgfscope}%
\pgfpathrectangle{\pgfqpoint{0.625000in}{0.550000in}}{\pgfqpoint{3.875000in}{3.850000in}} %
\pgfusepath{clip}%
\pgfsetbuttcap%
\pgfsetroundjoin%
\pgfsetlinewidth{0.250937pt}%
\definecolor{currentstroke}{rgb}{0.000000,0.000000,0.000000}%
\pgfsetstrokecolor{currentstroke}%
\pgfsetdash{}{0pt}%
\pgfpathmoveto{\pgfqpoint{0.625000in}{2.024779in}}%
\pgfpathlineto{\pgfqpoint{0.633120in}{2.016667in}}%
\pgfpathlineto{\pgfqpoint{0.625000in}{2.008621in}}%
\pgfusepath{stroke}%
\end{pgfscope}%
\begin{pgfscope}%
\pgfpathrectangle{\pgfqpoint{0.625000in}{0.550000in}}{\pgfqpoint{3.875000in}{3.850000in}} %
\pgfusepath{clip}%
\pgfsetbuttcap%
\pgfsetroundjoin%
\pgfsetlinewidth{0.250937pt}%
\definecolor{currentstroke}{rgb}{0.000000,0.000000,0.000000}%
\pgfsetstrokecolor{currentstroke}%
\pgfsetdash{}{0pt}%
\pgfpathmoveto{\pgfqpoint{0.625000in}{2.178756in}}%
\pgfpathlineto{\pgfqpoint{0.633147in}{2.171053in}}%
\pgfpathlineto{\pgfqpoint{0.625000in}{2.163129in}}%
\pgfusepath{stroke}%
\end{pgfscope}%
\begin{pgfscope}%
\pgfpathrectangle{\pgfqpoint{0.625000in}{0.550000in}}{\pgfqpoint{3.875000in}{3.850000in}} %
\pgfusepath{clip}%
\pgfsetbuttcap%
\pgfsetroundjoin%
\pgfsetlinewidth{0.250937pt}%
\definecolor{currentstroke}{rgb}{0.000000,0.000000,0.000000}%
\pgfsetstrokecolor{currentstroke}%
\pgfsetdash{}{0pt}%
\pgfpathmoveto{\pgfqpoint{0.625000in}{2.333100in}}%
\pgfpathlineto{\pgfqpoint{0.633179in}{2.325439in}}%
\pgfpathlineto{\pgfqpoint{0.625000in}{2.317240in}}%
\pgfusepath{stroke}%
\end{pgfscope}%
\begin{pgfscope}%
\pgfpathrectangle{\pgfqpoint{0.625000in}{0.550000in}}{\pgfqpoint{3.875000in}{3.850000in}} %
\pgfusepath{clip}%
\pgfsetbuttcap%
\pgfsetroundjoin%
\pgfsetlinewidth{0.250937pt}%
\definecolor{currentstroke}{rgb}{0.000000,0.000000,0.000000}%
\pgfsetstrokecolor{currentstroke}%
\pgfsetdash{}{0pt}%
\pgfpathmoveto{\pgfqpoint{0.625000in}{2.488467in}}%
\pgfpathlineto{\pgfqpoint{0.625944in}{2.489474in}}%
\pgfpathlineto{\pgfqpoint{0.626674in}{2.499123in}}%
\pgfpathlineto{\pgfqpoint{0.630928in}{2.518421in}}%
\pgfpathlineto{\pgfqpoint{0.637908in}{2.537719in}}%
\pgfpathlineto{\pgfqpoint{0.647922in}{2.557018in}}%
\pgfpathlineto{\pgfqpoint{0.654135in}{2.566875in}}%
\pgfpathlineto{\pgfqpoint{0.673559in}{2.589891in}}%
\pgfpathlineto{\pgfqpoint{0.692982in}{2.606136in}}%
\pgfpathlineto{\pgfqpoint{0.712406in}{2.618432in}}%
\pgfpathlineto{\pgfqpoint{0.731830in}{2.627367in}}%
\pgfpathlineto{\pgfqpoint{0.754387in}{2.634211in}}%
\pgfpathlineto{\pgfqpoint{0.770677in}{2.637456in}}%
\pgfpathlineto{\pgfqpoint{0.790100in}{2.639039in}}%
\pgfpathlineto{\pgfqpoint{0.809524in}{2.638403in}}%
\pgfpathlineto{\pgfqpoint{0.828947in}{2.635501in}}%
\pgfpathlineto{\pgfqpoint{0.848371in}{2.630249in}}%
\pgfpathlineto{\pgfqpoint{0.867794in}{2.622313in}}%
\pgfpathlineto{\pgfqpoint{0.887218in}{2.611226in}}%
\pgfpathlineto{\pgfqpoint{0.896930in}{2.604258in}}%
\pgfpathlineto{\pgfqpoint{0.907199in}{2.595614in}}%
\pgfpathlineto{\pgfqpoint{0.925072in}{2.576316in}}%
\pgfpathlineto{\pgfqpoint{0.938053in}{2.557018in}}%
\pgfpathlineto{\pgfqpoint{0.947362in}{2.537719in}}%
\pgfpathlineto{\pgfqpoint{0.953674in}{2.518421in}}%
\pgfpathlineto{\pgfqpoint{0.957356in}{2.499123in}}%
\pgfpathlineto{\pgfqpoint{0.958578in}{2.479825in}}%
\pgfpathlineto{\pgfqpoint{0.957356in}{2.460526in}}%
\pgfpathlineto{\pgfqpoint{0.953674in}{2.441228in}}%
\pgfpathlineto{\pgfqpoint{0.947362in}{2.421930in}}%
\pgfpathlineto{\pgfqpoint{0.938053in}{2.402632in}}%
\pgfpathlineto{\pgfqpoint{0.925072in}{2.383333in}}%
\pgfpathlineto{\pgfqpoint{0.907199in}{2.364035in}}%
\pgfpathlineto{\pgfqpoint{0.895648in}{2.354386in}}%
\pgfpathlineto{\pgfqpoint{0.877506in}{2.342439in}}%
\pgfpathlineto{\pgfqpoint{0.858083in}{2.333013in}}%
\pgfpathlineto{\pgfqpoint{0.838659in}{2.326471in}}%
\pgfpathlineto{\pgfqpoint{0.819236in}{2.322405in}}%
\pgfpathlineto{\pgfqpoint{0.799812in}{2.320651in}}%
\pgfpathlineto{\pgfqpoint{0.780388in}{2.321121in}}%
\pgfpathlineto{\pgfqpoint{0.760965in}{2.323856in}}%
\pgfpathlineto{\pgfqpoint{0.741541in}{2.328830in}}%
\pgfpathlineto{\pgfqpoint{0.722118in}{2.336441in}}%
\pgfpathlineto{\pgfqpoint{0.702694in}{2.346891in}}%
\pgfpathlineto{\pgfqpoint{0.683271in}{2.360979in}}%
\pgfpathlineto{\pgfqpoint{0.661547in}{2.383333in}}%
\pgfpathlineto{\pgfqpoint{0.654019in}{2.392982in}}%
\pgfpathlineto{\pgfqpoint{0.642420in}{2.412281in}}%
\pgfpathlineto{\pgfqpoint{0.633880in}{2.431579in}}%
\pgfpathlineto{\pgfqpoint{0.628569in}{2.450877in}}%
\pgfpathlineto{\pgfqpoint{0.626678in}{2.460526in}}%
\pgfpathlineto{\pgfqpoint{0.625000in}{2.471182in}}%
\pgfpathlineto{\pgfqpoint{0.625000in}{2.471182in}}%
\pgfusepath{stroke}%
\end{pgfscope}%
\begin{pgfscope}%
\pgfpathrectangle{\pgfqpoint{0.625000in}{0.550000in}}{\pgfqpoint{3.875000in}{3.850000in}} %
\pgfusepath{clip}%
\pgfsetbuttcap%
\pgfsetroundjoin%
\pgfsetlinewidth{0.250937pt}%
\definecolor{currentstroke}{rgb}{0.000000,0.000000,0.000000}%
\pgfsetstrokecolor{currentstroke}%
\pgfsetdash{}{0pt}%
\pgfpathmoveto{\pgfqpoint{0.625000in}{2.642406in}}%
\pgfpathlineto{\pgfqpoint{0.633176in}{2.634211in}}%
\pgfpathlineto{\pgfqpoint{0.625000in}{2.626553in}}%
\pgfusepath{stroke}%
\end{pgfscope}%
\begin{pgfscope}%
\pgfpathrectangle{\pgfqpoint{0.625000in}{0.550000in}}{\pgfqpoint{3.875000in}{3.850000in}} %
\pgfusepath{clip}%
\pgfsetbuttcap%
\pgfsetroundjoin%
\pgfsetlinewidth{0.250937pt}%
\definecolor{currentstroke}{rgb}{0.000000,0.000000,0.000000}%
\pgfsetstrokecolor{currentstroke}%
\pgfsetdash{}{0pt}%
\pgfpathmoveto{\pgfqpoint{0.625000in}{2.796510in}}%
\pgfpathlineto{\pgfqpoint{0.633137in}{2.788596in}}%
\pgfpathlineto{\pgfqpoint{0.625000in}{2.780905in}}%
\pgfusepath{stroke}%
\end{pgfscope}%
\begin{pgfscope}%
\pgfpathrectangle{\pgfqpoint{0.625000in}{0.550000in}}{\pgfqpoint{3.875000in}{3.850000in}} %
\pgfusepath{clip}%
\pgfsetbuttcap%
\pgfsetroundjoin%
\pgfsetlinewidth{0.250937pt}%
\definecolor{currentstroke}{rgb}{0.000000,0.000000,0.000000}%
\pgfsetstrokecolor{currentstroke}%
\pgfsetdash{}{0pt}%
\pgfpathmoveto{\pgfqpoint{0.625000in}{2.951001in}}%
\pgfpathlineto{\pgfqpoint{0.633092in}{2.942982in}}%
\pgfpathlineto{\pgfqpoint{0.625000in}{2.934897in}}%
\pgfusepath{stroke}%
\end{pgfscope}%
\begin{pgfscope}%
\pgfpathrectangle{\pgfqpoint{0.625000in}{0.550000in}}{\pgfqpoint{3.875000in}{3.850000in}} %
\pgfusepath{clip}%
\pgfsetbuttcap%
\pgfsetroundjoin%
\pgfsetlinewidth{0.250937pt}%
\definecolor{currentstroke}{rgb}{0.000000,0.000000,0.000000}%
\pgfsetstrokecolor{currentstroke}%
\pgfsetdash{}{0pt}%
\pgfpathmoveto{\pgfqpoint{0.625000in}{3.105020in}}%
\pgfpathlineto{\pgfqpoint{0.633115in}{3.097368in}}%
\pgfpathlineto{\pgfqpoint{0.625000in}{3.089553in}}%
\pgfusepath{stroke}%
\end{pgfscope}%
\begin{pgfscope}%
\pgfpathrectangle{\pgfqpoint{0.625000in}{0.550000in}}{\pgfqpoint{3.875000in}{3.850000in}} %
\pgfusepath{clip}%
\pgfsetbuttcap%
\pgfsetroundjoin%
\pgfsetlinewidth{0.250937pt}%
\definecolor{currentstroke}{rgb}{0.000000,0.000000,0.000000}%
\pgfsetstrokecolor{currentstroke}%
\pgfsetdash{}{0pt}%
\pgfpathmoveto{\pgfqpoint{0.625000in}{3.259495in}}%
\pgfpathlineto{\pgfqpoint{0.634712in}{3.254178in}}%
\pgfpathlineto{\pgfqpoint{0.637302in}{3.251754in}}%
\pgfpathlineto{\pgfqpoint{0.634712in}{3.249331in}}%
\pgfpathlineto{\pgfqpoint{0.625000in}{3.244065in}}%
\pgfusepath{stroke}%
\end{pgfscope}%
\begin{pgfscope}%
\pgfpathrectangle{\pgfqpoint{0.625000in}{0.550000in}}{\pgfqpoint{3.875000in}{3.850000in}} %
\pgfusepath{clip}%
\pgfsetbuttcap%
\pgfsetroundjoin%
\pgfsetlinewidth{0.250937pt}%
\definecolor{currentstroke}{rgb}{0.000000,0.000000,0.000000}%
\pgfsetstrokecolor{currentstroke}%
\pgfsetdash{}{0pt}%
\pgfpathmoveto{\pgfqpoint{0.625000in}{3.414121in}}%
\pgfpathlineto{\pgfqpoint{0.633116in}{3.406140in}}%
\pgfpathlineto{\pgfqpoint{0.625000in}{3.398419in}}%
\pgfusepath{stroke}%
\end{pgfscope}%
\begin{pgfscope}%
\pgfpathrectangle{\pgfqpoint{0.625000in}{0.550000in}}{\pgfqpoint{3.875000in}{3.850000in}} %
\pgfusepath{clip}%
\pgfsetbuttcap%
\pgfsetroundjoin%
\pgfsetlinewidth{0.250937pt}%
\definecolor{currentstroke}{rgb}{0.000000,0.000000,0.000000}%
\pgfsetstrokecolor{currentstroke}%
\pgfsetdash{}{0pt}%
\pgfpathmoveto{\pgfqpoint{0.625000in}{3.568262in}}%
\pgfpathlineto{\pgfqpoint{0.633141in}{3.560526in}}%
\pgfpathlineto{\pgfqpoint{0.625000in}{3.552795in}}%
\pgfusepath{stroke}%
\end{pgfscope}%
\begin{pgfscope}%
\pgfpathrectangle{\pgfqpoint{0.625000in}{0.550000in}}{\pgfqpoint{3.875000in}{3.850000in}} %
\pgfusepath{clip}%
\pgfsetbuttcap%
\pgfsetroundjoin%
\pgfsetlinewidth{0.250937pt}%
\definecolor{currentstroke}{rgb}{0.000000,0.000000,0.000000}%
\pgfsetstrokecolor{currentstroke}%
\pgfsetdash{}{0pt}%
\pgfpathmoveto{\pgfqpoint{0.625000in}{3.722727in}}%
\pgfpathlineto{\pgfqpoint{0.633160in}{3.714912in}}%
\pgfpathlineto{\pgfqpoint{0.625000in}{3.706862in}}%
\pgfusepath{stroke}%
\end{pgfscope}%
\begin{pgfscope}%
\pgfpathrectangle{\pgfqpoint{0.625000in}{0.550000in}}{\pgfqpoint{3.875000in}{3.850000in}} %
\pgfusepath{clip}%
\pgfsetbuttcap%
\pgfsetroundjoin%
\pgfsetlinewidth{0.250937pt}%
\definecolor{currentstroke}{rgb}{0.000000,0.000000,0.000000}%
\pgfsetstrokecolor{currentstroke}%
\pgfsetdash{}{0pt}%
\pgfpathmoveto{\pgfqpoint{0.625000in}{3.876990in}}%
\pgfpathlineto{\pgfqpoint{0.633083in}{3.869298in}}%
\pgfpathlineto{\pgfqpoint{0.625000in}{3.861575in}}%
\pgfusepath{stroke}%
\end{pgfscope}%
\begin{pgfscope}%
\pgfpathrectangle{\pgfqpoint{0.625000in}{0.550000in}}{\pgfqpoint{3.875000in}{3.850000in}} %
\pgfusepath{clip}%
\pgfsetbuttcap%
\pgfsetroundjoin%
\pgfsetlinewidth{0.250937pt}%
\definecolor{currentstroke}{rgb}{0.000000,0.000000,0.000000}%
\pgfsetstrokecolor{currentstroke}%
\pgfsetdash{}{0pt}%
\pgfpathmoveto{\pgfqpoint{0.625000in}{4.031498in}}%
\pgfpathlineto{\pgfqpoint{0.634712in}{4.026108in}}%
\pgfpathlineto{\pgfqpoint{0.637302in}{4.023684in}}%
\pgfpathlineto{\pgfqpoint{0.634712in}{4.021260in}}%
\pgfpathlineto{\pgfqpoint{0.625000in}{4.015947in}}%
\pgfusepath{stroke}%
\end{pgfscope}%
\begin{pgfscope}%
\pgfpathrectangle{\pgfqpoint{0.625000in}{0.550000in}}{\pgfqpoint{3.875000in}{3.850000in}} %
\pgfusepath{clip}%
\pgfsetbuttcap%
\pgfsetroundjoin%
\pgfsetlinewidth{0.250937pt}%
\definecolor{currentstroke}{rgb}{0.000000,0.000000,0.000000}%
\pgfsetstrokecolor{currentstroke}%
\pgfsetdash{}{0pt}%
\pgfpathmoveto{\pgfqpoint{0.625000in}{4.185847in}}%
\pgfpathlineto{\pgfqpoint{0.633079in}{4.178070in}}%
\pgfpathlineto{\pgfqpoint{0.625000in}{4.170657in}}%
\pgfusepath{stroke}%
\end{pgfscope}%
\begin{pgfscope}%
\pgfpathrectangle{\pgfqpoint{0.625000in}{0.550000in}}{\pgfqpoint{3.875000in}{3.850000in}} %
\pgfusepath{clip}%
\pgfsetbuttcap%
\pgfsetroundjoin%
\pgfsetlinewidth{0.250937pt}%
\definecolor{currentstroke}{rgb}{0.000000,0.000000,0.000000}%
\pgfsetstrokecolor{currentstroke}%
\pgfsetdash{}{0pt}%
\pgfpathmoveto{\pgfqpoint{0.625000in}{4.340196in}}%
\pgfpathlineto{\pgfqpoint{0.633040in}{4.332456in}}%
\pgfpathlineto{\pgfqpoint{0.625000in}{4.324690in}}%
\pgfusepath{stroke}%
\end{pgfscope}%
\begin{pgfscope}%
\pgfpathrectangle{\pgfqpoint{0.625000in}{0.550000in}}{\pgfqpoint{3.875000in}{3.850000in}} %
\pgfusepath{clip}%
\pgfsetbuttcap%
\pgfsetroundjoin%
\pgfsetlinewidth{0.250937pt}%
\definecolor{currentstroke}{rgb}{0.000000,0.000000,0.000000}%
\pgfsetstrokecolor{currentstroke}%
\pgfsetdash{}{0pt}%
\pgfpathmoveto{\pgfqpoint{0.634712in}{1.178316in}}%
\pgfpathlineto{\pgfqpoint{0.629407in}{1.186842in}}%
\pgfpathlineto{\pgfqpoint{0.627769in}{1.196491in}}%
\pgfpathlineto{\pgfqpoint{0.633579in}{1.206140in}}%
\pgfpathlineto{\pgfqpoint{0.634712in}{1.208279in}}%
\pgfpathlineto{\pgfqpoint{0.644424in}{1.209123in}}%
\pgfpathlineto{\pgfqpoint{0.650917in}{1.206140in}}%
\pgfpathlineto{\pgfqpoint{0.654135in}{1.202260in}}%
\pgfpathlineto{\pgfqpoint{0.658003in}{1.196491in}}%
\pgfpathlineto{\pgfqpoint{0.656750in}{1.186842in}}%
\pgfpathlineto{\pgfqpoint{0.654135in}{1.183953in}}%
\pgfpathlineto{\pgfqpoint{0.644424in}{1.177804in}}%
\pgfpathlineto{\pgfqpoint{0.634712in}{1.178316in}}%
\pgfusepath{stroke}%
\end{pgfscope}%
\begin{pgfscope}%
\pgfpathrectangle{\pgfqpoint{0.625000in}{0.550000in}}{\pgfqpoint{3.875000in}{3.850000in}} %
\pgfusepath{clip}%
\pgfsetbuttcap%
\pgfsetroundjoin%
\pgfsetlinewidth{0.250937pt}%
\definecolor{currentstroke}{rgb}{0.000000,0.000000,0.000000}%
\pgfsetstrokecolor{currentstroke}%
\pgfsetdash{}{0pt}%
\pgfpathmoveto{\pgfqpoint{0.634712in}{3.751370in}}%
\pgfpathlineto{\pgfqpoint{0.633579in}{3.753509in}}%
\pgfpathlineto{\pgfqpoint{0.627769in}{3.763158in}}%
\pgfpathlineto{\pgfqpoint{0.629407in}{3.772807in}}%
\pgfpathlineto{\pgfqpoint{0.634712in}{3.781333in}}%
\pgfpathlineto{\pgfqpoint{0.644424in}{3.781845in}}%
\pgfpathlineto{\pgfqpoint{0.654135in}{3.775696in}}%
\pgfpathlineto{\pgfqpoint{0.656750in}{3.772807in}}%
\pgfpathlineto{\pgfqpoint{0.658003in}{3.763158in}}%
\pgfpathlineto{\pgfqpoint{0.654135in}{3.757389in}}%
\pgfpathlineto{\pgfqpoint{0.650917in}{3.753509in}}%
\pgfpathlineto{\pgfqpoint{0.644424in}{3.750526in}}%
\pgfpathlineto{\pgfqpoint{0.634712in}{3.751370in}}%
\pgfusepath{stroke}%
\end{pgfscope}%
\begin{pgfscope}%
\pgfpathrectangle{\pgfqpoint{0.625000in}{0.550000in}}{\pgfqpoint{3.875000in}{3.850000in}} %
\pgfusepath{clip}%
\pgfsetbuttcap%
\pgfsetroundjoin%
\pgfsetlinewidth{0.250937pt}%
\definecolor{currentstroke}{rgb}{0.000000,0.000000,0.000000}%
\pgfsetstrokecolor{currentstroke}%
\pgfsetdash{}{0pt}%
\pgfpathmoveto{\pgfqpoint{0.625000in}{0.635284in}}%
\pgfpathlineto{\pgfqpoint{0.633344in}{0.627193in}}%
\pgfpathlineto{\pgfqpoint{0.625000in}{0.619126in}}%
\pgfusepath{stroke}%
\end{pgfscope}%
\begin{pgfscope}%
\pgfpathrectangle{\pgfqpoint{0.625000in}{0.550000in}}{\pgfqpoint{3.875000in}{3.850000in}} %
\pgfusepath{clip}%
\pgfsetbuttcap%
\pgfsetroundjoin%
\pgfsetlinewidth{0.250937pt}%
\definecolor{currentstroke}{rgb}{0.000000,0.000000,0.000000}%
\pgfsetstrokecolor{currentstroke}%
\pgfsetdash{}{0pt}%
\pgfpathmoveto{\pgfqpoint{0.625000in}{0.789269in}}%
\pgfpathlineto{\pgfqpoint{0.633319in}{0.781579in}}%
\pgfpathlineto{\pgfqpoint{0.625000in}{0.773547in}}%
\pgfusepath{stroke}%
\end{pgfscope}%
\begin{pgfscope}%
\pgfpathrectangle{\pgfqpoint{0.625000in}{0.550000in}}{\pgfqpoint{3.875000in}{3.850000in}} %
\pgfusepath{clip}%
\pgfsetbuttcap%
\pgfsetroundjoin%
\pgfsetlinewidth{0.250937pt}%
\definecolor{currentstroke}{rgb}{0.000000,0.000000,0.000000}%
\pgfsetstrokecolor{currentstroke}%
\pgfsetdash{}{0pt}%
\pgfpathmoveto{\pgfqpoint{0.625000in}{0.943941in}}%
\pgfpathlineto{\pgfqpoint{0.634712in}{0.939201in}}%
\pgfpathlineto{\pgfqpoint{0.638171in}{0.935965in}}%
\pgfpathlineto{\pgfqpoint{0.634712in}{0.932729in}}%
\pgfpathlineto{\pgfqpoint{0.625000in}{0.927917in}}%
\pgfusepath{stroke}%
\end{pgfscope}%
\begin{pgfscope}%
\pgfpathrectangle{\pgfqpoint{0.625000in}{0.550000in}}{\pgfqpoint{3.875000in}{3.850000in}} %
\pgfusepath{clip}%
\pgfsetbuttcap%
\pgfsetroundjoin%
\pgfsetlinewidth{0.250937pt}%
\definecolor{currentstroke}{rgb}{0.000000,0.000000,0.000000}%
\pgfsetstrokecolor{currentstroke}%
\pgfsetdash{}{0pt}%
\pgfpathmoveto{\pgfqpoint{0.625000in}{1.098330in}}%
\pgfpathlineto{\pgfqpoint{0.633321in}{1.090351in}}%
\pgfpathlineto{\pgfqpoint{0.625000in}{1.082401in}}%
\pgfusepath{stroke}%
\end{pgfscope}%
\begin{pgfscope}%
\pgfpathrectangle{\pgfqpoint{0.625000in}{0.550000in}}{\pgfqpoint{3.875000in}{3.850000in}} %
\pgfusepath{clip}%
\pgfsetbuttcap%
\pgfsetroundjoin%
\pgfsetlinewidth{0.250937pt}%
\definecolor{currentstroke}{rgb}{0.000000,0.000000,0.000000}%
\pgfsetstrokecolor{currentstroke}%
\pgfsetdash{}{0pt}%
\pgfpathmoveto{\pgfqpoint{0.625000in}{1.252965in}}%
\pgfpathlineto{\pgfqpoint{0.633337in}{1.244737in}}%
\pgfpathlineto{\pgfqpoint{0.625000in}{1.236736in}}%
\pgfusepath{stroke}%
\end{pgfscope}%
\begin{pgfscope}%
\pgfpathrectangle{\pgfqpoint{0.625000in}{0.550000in}}{\pgfqpoint{3.875000in}{3.850000in}} %
\pgfusepath{clip}%
\pgfsetbuttcap%
\pgfsetroundjoin%
\pgfsetlinewidth{0.250937pt}%
\definecolor{currentstroke}{rgb}{0.000000,0.000000,0.000000}%
\pgfsetstrokecolor{currentstroke}%
\pgfsetdash{}{0pt}%
\pgfpathmoveto{\pgfqpoint{0.625000in}{1.407008in}}%
\pgfpathlineto{\pgfqpoint{0.633290in}{1.399123in}}%
\pgfpathlineto{\pgfqpoint{0.625000in}{1.391233in}}%
\pgfusepath{stroke}%
\end{pgfscope}%
\begin{pgfscope}%
\pgfpathrectangle{\pgfqpoint{0.625000in}{0.550000in}}{\pgfqpoint{3.875000in}{3.850000in}} %
\pgfusepath{clip}%
\pgfsetbuttcap%
\pgfsetroundjoin%
\pgfsetlinewidth{0.250937pt}%
\definecolor{currentstroke}{rgb}{0.000000,0.000000,0.000000}%
\pgfsetstrokecolor{currentstroke}%
\pgfsetdash{}{0pt}%
\pgfpathmoveto{\pgfqpoint{0.625000in}{1.561372in}}%
\pgfpathlineto{\pgfqpoint{0.633253in}{1.553509in}}%
\pgfpathlineto{\pgfqpoint{0.625000in}{1.545390in}}%
\pgfusepath{stroke}%
\end{pgfscope}%
\begin{pgfscope}%
\pgfpathrectangle{\pgfqpoint{0.625000in}{0.550000in}}{\pgfqpoint{3.875000in}{3.850000in}} %
\pgfusepath{clip}%
\pgfsetbuttcap%
\pgfsetroundjoin%
\pgfsetlinewidth{0.250937pt}%
\definecolor{currentstroke}{rgb}{0.000000,0.000000,0.000000}%
\pgfsetstrokecolor{currentstroke}%
\pgfsetdash{}{0pt}%
\pgfpathmoveto{\pgfqpoint{0.625000in}{1.715754in}}%
\pgfpathlineto{\pgfqpoint{0.634712in}{1.711131in}}%
\pgfpathlineto{\pgfqpoint{0.638171in}{1.707895in}}%
\pgfpathlineto{\pgfqpoint{0.634712in}{1.704658in}}%
\pgfpathlineto{\pgfqpoint{0.625000in}{1.699985in}}%
\pgfusepath{stroke}%
\end{pgfscope}%
\begin{pgfscope}%
\pgfpathrectangle{\pgfqpoint{0.625000in}{0.550000in}}{\pgfqpoint{3.875000in}{3.850000in}} %
\pgfusepath{clip}%
\pgfsetbuttcap%
\pgfsetroundjoin%
\pgfsetlinewidth{0.250937pt}%
\definecolor{currentstroke}{rgb}{0.000000,0.000000,0.000000}%
\pgfsetstrokecolor{currentstroke}%
\pgfsetdash{}{0pt}%
\pgfpathmoveto{\pgfqpoint{0.625000in}{1.870203in}}%
\pgfpathlineto{\pgfqpoint{0.633223in}{1.862281in}}%
\pgfpathlineto{\pgfqpoint{0.625000in}{1.854522in}}%
\pgfusepath{stroke}%
\end{pgfscope}%
\begin{pgfscope}%
\pgfpathrectangle{\pgfqpoint{0.625000in}{0.550000in}}{\pgfqpoint{3.875000in}{3.850000in}} %
\pgfusepath{clip}%
\pgfsetbuttcap%
\pgfsetroundjoin%
\pgfsetlinewidth{0.250937pt}%
\definecolor{currentstroke}{rgb}{0.000000,0.000000,0.000000}%
\pgfsetstrokecolor{currentstroke}%
\pgfsetdash{}{0pt}%
\pgfpathmoveto{\pgfqpoint{0.625000in}{2.024866in}}%
\pgfpathlineto{\pgfqpoint{0.633207in}{2.016667in}}%
\pgfpathlineto{\pgfqpoint{0.625000in}{2.008534in}}%
\pgfusepath{stroke}%
\end{pgfscope}%
\begin{pgfscope}%
\pgfpathrectangle{\pgfqpoint{0.625000in}{0.550000in}}{\pgfqpoint{3.875000in}{3.850000in}} %
\pgfusepath{clip}%
\pgfsetbuttcap%
\pgfsetroundjoin%
\pgfsetlinewidth{0.250937pt}%
\definecolor{currentstroke}{rgb}{0.000000,0.000000,0.000000}%
\pgfsetstrokecolor{currentstroke}%
\pgfsetdash{}{0pt}%
\pgfpathmoveto{\pgfqpoint{0.625000in}{2.178838in}}%
\pgfpathlineto{\pgfqpoint{0.633234in}{2.171053in}}%
\pgfpathlineto{\pgfqpoint{0.625000in}{2.163044in}}%
\pgfusepath{stroke}%
\end{pgfscope}%
\begin{pgfscope}%
\pgfpathrectangle{\pgfqpoint{0.625000in}{0.550000in}}{\pgfqpoint{3.875000in}{3.850000in}} %
\pgfusepath{clip}%
\pgfsetbuttcap%
\pgfsetroundjoin%
\pgfsetlinewidth{0.250937pt}%
\definecolor{currentstroke}{rgb}{0.000000,0.000000,0.000000}%
\pgfsetstrokecolor{currentstroke}%
\pgfsetdash{}{0pt}%
\pgfpathmoveto{\pgfqpoint{0.625000in}{2.333183in}}%
\pgfpathlineto{\pgfqpoint{0.633268in}{2.325439in}}%
\pgfpathlineto{\pgfqpoint{0.625000in}{2.317152in}}%
\pgfusepath{stroke}%
\end{pgfscope}%
\begin{pgfscope}%
\pgfpathrectangle{\pgfqpoint{0.625000in}{0.550000in}}{\pgfqpoint{3.875000in}{3.850000in}} %
\pgfusepath{clip}%
\pgfsetbuttcap%
\pgfsetroundjoin%
\pgfsetlinewidth{0.250937pt}%
\definecolor{currentstroke}{rgb}{0.000000,0.000000,0.000000}%
\pgfsetstrokecolor{currentstroke}%
\pgfsetdash{}{0pt}%
\pgfpathmoveto{\pgfqpoint{0.625000in}{2.488551in}}%
\pgfpathlineto{\pgfqpoint{0.625865in}{2.489474in}}%
\pgfpathlineto{\pgfqpoint{0.628415in}{2.508772in}}%
\pgfpathlineto{\pgfqpoint{0.634712in}{2.532385in}}%
\pgfpathlineto{\pgfqpoint{0.646611in}{2.557018in}}%
\pgfpathlineto{\pgfqpoint{0.659357in}{2.576316in}}%
\pgfpathlineto{\pgfqpoint{0.667161in}{2.585965in}}%
\pgfpathlineto{\pgfqpoint{0.676170in}{2.595614in}}%
\pgfpathlineto{\pgfqpoint{0.692982in}{2.610717in}}%
\pgfpathlineto{\pgfqpoint{0.714168in}{2.624561in}}%
\pgfpathlineto{\pgfqpoint{0.731830in}{2.633565in}}%
\pgfpathlineto{\pgfqpoint{0.762779in}{2.643860in}}%
\pgfpathlineto{\pgfqpoint{0.780388in}{2.647327in}}%
\pgfpathlineto{\pgfqpoint{0.799812in}{2.648976in}}%
\pgfpathlineto{\pgfqpoint{0.819236in}{2.648551in}}%
\pgfpathlineto{\pgfqpoint{0.838659in}{2.646018in}}%
\pgfpathlineto{\pgfqpoint{0.858083in}{2.641296in}}%
\pgfpathlineto{\pgfqpoint{0.877506in}{2.634148in}}%
\pgfpathlineto{\pgfqpoint{0.896930in}{2.624234in}}%
\pgfpathlineto{\pgfqpoint{0.916353in}{2.610875in}}%
\pgfpathlineto{\pgfqpoint{0.933278in}{2.595614in}}%
\pgfpathlineto{\pgfqpoint{0.945489in}{2.581400in}}%
\pgfpathlineto{\pgfqpoint{0.955744in}{2.566667in}}%
\pgfpathlineto{\pgfqpoint{0.965963in}{2.547368in}}%
\pgfpathlineto{\pgfqpoint{0.973217in}{2.528070in}}%
\pgfpathlineto{\pgfqpoint{0.977902in}{2.508772in}}%
\pgfpathlineto{\pgfqpoint{0.980192in}{2.489474in}}%
\pgfpathlineto{\pgfqpoint{0.980192in}{2.470175in}}%
\pgfpathlineto{\pgfqpoint{0.977902in}{2.450877in}}%
\pgfpathlineto{\pgfqpoint{0.973217in}{2.431579in}}%
\pgfpathlineto{\pgfqpoint{0.964912in}{2.410129in}}%
\pgfpathlineto{\pgfqpoint{0.955201in}{2.392155in}}%
\pgfpathlineto{\pgfqpoint{0.941937in}{2.373684in}}%
\pgfpathlineto{\pgfqpoint{0.926065in}{2.357068in}}%
\pgfpathlineto{\pgfqpoint{0.911110in}{2.344737in}}%
\pgfpathlineto{\pgfqpoint{0.896378in}{2.335088in}}%
\pgfpathlineto{\pgfqpoint{0.877360in}{2.325439in}}%
\pgfpathlineto{\pgfqpoint{0.858083in}{2.318353in}}%
\pgfpathlineto{\pgfqpoint{0.838659in}{2.313631in}}%
\pgfpathlineto{\pgfqpoint{0.819236in}{2.311098in}}%
\pgfpathlineto{\pgfqpoint{0.799812in}{2.310673in}}%
\pgfpathlineto{\pgfqpoint{0.780388in}{2.312322in}}%
\pgfpathlineto{\pgfqpoint{0.760965in}{2.316159in}}%
\pgfpathlineto{\pgfqpoint{0.741541in}{2.322085in}}%
\pgfpathlineto{\pgfqpoint{0.722118in}{2.330558in}}%
\pgfpathlineto{\pgfqpoint{0.699159in}{2.344737in}}%
\pgfpathlineto{\pgfqpoint{0.686722in}{2.354386in}}%
\pgfpathlineto{\pgfqpoint{0.683271in}{2.356999in}}%
\pgfpathlineto{\pgfqpoint{0.663847in}{2.377061in}}%
\pgfpathlineto{\pgfqpoint{0.644424in}{2.405866in}}%
\pgfpathlineto{\pgfqpoint{0.633476in}{2.431579in}}%
\pgfpathlineto{\pgfqpoint{0.628415in}{2.450877in}}%
\pgfpathlineto{\pgfqpoint{0.625000in}{2.471098in}}%
\pgfpathlineto{\pgfqpoint{0.625000in}{2.471098in}}%
\pgfusepath{stroke}%
\end{pgfscope}%
\begin{pgfscope}%
\pgfpathrectangle{\pgfqpoint{0.625000in}{0.550000in}}{\pgfqpoint{3.875000in}{3.850000in}} %
\pgfusepath{clip}%
\pgfsetbuttcap%
\pgfsetroundjoin%
\pgfsetlinewidth{0.250937pt}%
\definecolor{currentstroke}{rgb}{0.000000,0.000000,0.000000}%
\pgfsetstrokecolor{currentstroke}%
\pgfsetdash{}{0pt}%
\pgfpathmoveto{\pgfqpoint{0.625000in}{2.642495in}}%
\pgfpathlineto{\pgfqpoint{0.633265in}{2.634211in}}%
\pgfpathlineto{\pgfqpoint{0.625000in}{2.626470in}}%
\pgfusepath{stroke}%
\end{pgfscope}%
\begin{pgfscope}%
\pgfpathrectangle{\pgfqpoint{0.625000in}{0.550000in}}{\pgfqpoint{3.875000in}{3.850000in}} %
\pgfusepath{clip}%
\pgfsetbuttcap%
\pgfsetroundjoin%
\pgfsetlinewidth{0.250937pt}%
\definecolor{currentstroke}{rgb}{0.000000,0.000000,0.000000}%
\pgfsetstrokecolor{currentstroke}%
\pgfsetdash{}{0pt}%
\pgfpathmoveto{\pgfqpoint{0.625000in}{2.796595in}}%
\pgfpathlineto{\pgfqpoint{0.633224in}{2.788596in}}%
\pgfpathlineto{\pgfqpoint{0.625000in}{2.780822in}}%
\pgfusepath{stroke}%
\end{pgfscope}%
\begin{pgfscope}%
\pgfpathrectangle{\pgfqpoint{0.625000in}{0.550000in}}{\pgfqpoint{3.875000in}{3.850000in}} %
\pgfusepath{clip}%
\pgfsetbuttcap%
\pgfsetroundjoin%
\pgfsetlinewidth{0.250937pt}%
\definecolor{currentstroke}{rgb}{0.000000,0.000000,0.000000}%
\pgfsetstrokecolor{currentstroke}%
\pgfsetdash{}{0pt}%
\pgfpathmoveto{\pgfqpoint{0.625000in}{2.951089in}}%
\pgfpathlineto{\pgfqpoint{0.633181in}{2.942982in}}%
\pgfpathlineto{\pgfqpoint{0.625000in}{2.934808in}}%
\pgfusepath{stroke}%
\end{pgfscope}%
\begin{pgfscope}%
\pgfpathrectangle{\pgfqpoint{0.625000in}{0.550000in}}{\pgfqpoint{3.875000in}{3.850000in}} %
\pgfusepath{clip}%
\pgfsetbuttcap%
\pgfsetroundjoin%
\pgfsetlinewidth{0.250937pt}%
\definecolor{currentstroke}{rgb}{0.000000,0.000000,0.000000}%
\pgfsetstrokecolor{currentstroke}%
\pgfsetdash{}{0pt}%
\pgfpathmoveto{\pgfqpoint{0.625000in}{3.105103in}}%
\pgfpathlineto{\pgfqpoint{0.633202in}{3.097368in}}%
\pgfpathlineto{\pgfqpoint{0.625000in}{3.089469in}}%
\pgfusepath{stroke}%
\end{pgfscope}%
\begin{pgfscope}%
\pgfpathrectangle{\pgfqpoint{0.625000in}{0.550000in}}{\pgfqpoint{3.875000in}{3.850000in}} %
\pgfusepath{clip}%
\pgfsetbuttcap%
\pgfsetroundjoin%
\pgfsetlinewidth{0.250937pt}%
\definecolor{currentstroke}{rgb}{0.000000,0.000000,0.000000}%
\pgfsetstrokecolor{currentstroke}%
\pgfsetdash{}{0pt}%
\pgfpathmoveto{\pgfqpoint{0.625000in}{3.259580in}}%
\pgfpathlineto{\pgfqpoint{0.634712in}{3.254991in}}%
\pgfpathlineto{\pgfqpoint{0.638171in}{3.251754in}}%
\pgfpathlineto{\pgfqpoint{0.634712in}{3.248518in}}%
\pgfpathlineto{\pgfqpoint{0.625000in}{3.243980in}}%
\pgfusepath{stroke}%
\end{pgfscope}%
\begin{pgfscope}%
\pgfpathrectangle{\pgfqpoint{0.625000in}{0.550000in}}{\pgfqpoint{3.875000in}{3.850000in}} %
\pgfusepath{clip}%
\pgfsetbuttcap%
\pgfsetroundjoin%
\pgfsetlinewidth{0.250937pt}%
\definecolor{currentstroke}{rgb}{0.000000,0.000000,0.000000}%
\pgfsetstrokecolor{currentstroke}%
\pgfsetdash{}{0pt}%
\pgfpathmoveto{\pgfqpoint{0.625000in}{3.414207in}}%
\pgfpathlineto{\pgfqpoint{0.633203in}{3.406140in}}%
\pgfpathlineto{\pgfqpoint{0.625000in}{3.398336in}}%
\pgfusepath{stroke}%
\end{pgfscope}%
\begin{pgfscope}%
\pgfpathrectangle{\pgfqpoint{0.625000in}{0.550000in}}{\pgfqpoint{3.875000in}{3.850000in}} %
\pgfusepath{clip}%
\pgfsetbuttcap%
\pgfsetroundjoin%
\pgfsetlinewidth{0.250937pt}%
\definecolor{currentstroke}{rgb}{0.000000,0.000000,0.000000}%
\pgfsetstrokecolor{currentstroke}%
\pgfsetdash{}{0pt}%
\pgfpathmoveto{\pgfqpoint{0.625000in}{3.568347in}}%
\pgfpathlineto{\pgfqpoint{0.633231in}{3.560526in}}%
\pgfpathlineto{\pgfqpoint{0.625000in}{3.552711in}}%
\pgfusepath{stroke}%
\end{pgfscope}%
\begin{pgfscope}%
\pgfpathrectangle{\pgfqpoint{0.625000in}{0.550000in}}{\pgfqpoint{3.875000in}{3.850000in}} %
\pgfusepath{clip}%
\pgfsetbuttcap%
\pgfsetroundjoin%
\pgfsetlinewidth{0.250937pt}%
\definecolor{currentstroke}{rgb}{0.000000,0.000000,0.000000}%
\pgfsetstrokecolor{currentstroke}%
\pgfsetdash{}{0pt}%
\pgfpathmoveto{\pgfqpoint{0.625000in}{3.722813in}}%
\pgfpathlineto{\pgfqpoint{0.633250in}{3.714912in}}%
\pgfpathlineto{\pgfqpoint{0.625000in}{3.706774in}}%
\pgfusepath{stroke}%
\end{pgfscope}%
\begin{pgfscope}%
\pgfpathrectangle{\pgfqpoint{0.625000in}{0.550000in}}{\pgfqpoint{3.875000in}{3.850000in}} %
\pgfusepath{clip}%
\pgfsetbuttcap%
\pgfsetroundjoin%
\pgfsetlinewidth{0.250937pt}%
\definecolor{currentstroke}{rgb}{0.000000,0.000000,0.000000}%
\pgfsetstrokecolor{currentstroke}%
\pgfsetdash{}{0pt}%
\pgfpathmoveto{\pgfqpoint{0.625000in}{3.877077in}}%
\pgfpathlineto{\pgfqpoint{0.633175in}{3.869298in}}%
\pgfpathlineto{\pgfqpoint{0.625000in}{3.861487in}}%
\pgfusepath{stroke}%
\end{pgfscope}%
\begin{pgfscope}%
\pgfpathrectangle{\pgfqpoint{0.625000in}{0.550000in}}{\pgfqpoint{3.875000in}{3.850000in}} %
\pgfusepath{clip}%
\pgfsetbuttcap%
\pgfsetroundjoin%
\pgfsetlinewidth{0.250937pt}%
\definecolor{currentstroke}{rgb}{0.000000,0.000000,0.000000}%
\pgfsetstrokecolor{currentstroke}%
\pgfsetdash{}{0pt}%
\pgfpathmoveto{\pgfqpoint{0.625000in}{4.031582in}}%
\pgfpathlineto{\pgfqpoint{0.634712in}{4.026921in}}%
\pgfpathlineto{\pgfqpoint{0.638171in}{4.023684in}}%
\pgfpathlineto{\pgfqpoint{0.634712in}{4.020448in}}%
\pgfpathlineto{\pgfqpoint{0.625000in}{4.015863in}}%
\pgfusepath{stroke}%
\end{pgfscope}%
\begin{pgfscope}%
\pgfpathrectangle{\pgfqpoint{0.625000in}{0.550000in}}{\pgfqpoint{3.875000in}{3.850000in}} %
\pgfusepath{clip}%
\pgfsetbuttcap%
\pgfsetroundjoin%
\pgfsetlinewidth{0.250937pt}%
\definecolor{currentstroke}{rgb}{0.000000,0.000000,0.000000}%
\pgfsetstrokecolor{currentstroke}%
\pgfsetdash{}{0pt}%
\pgfpathmoveto{\pgfqpoint{0.625000in}{4.185936in}}%
\pgfpathlineto{\pgfqpoint{0.633171in}{4.178070in}}%
\pgfpathlineto{\pgfqpoint{0.625000in}{4.170573in}}%
\pgfusepath{stroke}%
\end{pgfscope}%
\begin{pgfscope}%
\pgfpathrectangle{\pgfqpoint{0.625000in}{0.550000in}}{\pgfqpoint{3.875000in}{3.850000in}} %
\pgfusepath{clip}%
\pgfsetbuttcap%
\pgfsetroundjoin%
\pgfsetlinewidth{0.250937pt}%
\definecolor{currentstroke}{rgb}{0.000000,0.000000,0.000000}%
\pgfsetstrokecolor{currentstroke}%
\pgfsetdash{}{0pt}%
\pgfpathmoveto{\pgfqpoint{0.625000in}{4.340283in}}%
\pgfpathlineto{\pgfqpoint{0.633130in}{4.332456in}}%
\pgfpathlineto{\pgfqpoint{0.625000in}{4.324603in}}%
\pgfusepath{stroke}%
\end{pgfscope}%
\begin{pgfscope}%
\pgfpathrectangle{\pgfqpoint{0.625000in}{0.550000in}}{\pgfqpoint{3.875000in}{3.850000in}} %
\pgfusepath{clip}%
\pgfsetbuttcap%
\pgfsetroundjoin%
\pgfsetlinewidth{0.250937pt}%
\definecolor{currentstroke}{rgb}{0.000000,0.000000,0.000000}%
\pgfsetstrokecolor{currentstroke}%
\pgfsetdash{}{0pt}%
\pgfpathmoveto{\pgfqpoint{0.644424in}{1.176755in}}%
\pgfpathlineto{\pgfqpoint{0.643195in}{1.177193in}}%
\pgfpathlineto{\pgfqpoint{0.634712in}{1.177982in}}%
\pgfpathlineto{\pgfqpoint{0.629200in}{1.186842in}}%
\pgfpathlineto{\pgfqpoint{0.627594in}{1.196491in}}%
\pgfpathlineto{\pgfqpoint{0.633200in}{1.206140in}}%
\pgfpathlineto{\pgfqpoint{0.634712in}{1.208996in}}%
\pgfpathlineto{\pgfqpoint{0.644424in}{1.210208in}}%
\pgfpathlineto{\pgfqpoint{0.653280in}{1.206140in}}%
\pgfpathlineto{\pgfqpoint{0.654135in}{1.205109in}}%
\pgfpathlineto{\pgfqpoint{0.659913in}{1.196491in}}%
\pgfpathlineto{\pgfqpoint{0.658844in}{1.186842in}}%
\pgfpathlineto{\pgfqpoint{0.654135in}{1.181640in}}%
\pgfpathlineto{\pgfqpoint{0.645891in}{1.177193in}}%
\pgfpathlineto{\pgfqpoint{0.644424in}{1.176755in}}%
\pgfusepath{stroke}%
\end{pgfscope}%
\begin{pgfscope}%
\pgfpathrectangle{\pgfqpoint{0.625000in}{0.550000in}}{\pgfqpoint{3.875000in}{3.850000in}} %
\pgfusepath{clip}%
\pgfsetbuttcap%
\pgfsetroundjoin%
\pgfsetlinewidth{0.250937pt}%
\definecolor{currentstroke}{rgb}{0.000000,0.000000,0.000000}%
\pgfsetstrokecolor{currentstroke}%
\pgfsetdash{}{0pt}%
\pgfpathmoveto{\pgfqpoint{0.634712in}{3.750653in}}%
\pgfpathlineto{\pgfqpoint{0.633200in}{3.753509in}}%
\pgfpathlineto{\pgfqpoint{0.627594in}{3.763158in}}%
\pgfpathlineto{\pgfqpoint{0.629200in}{3.772807in}}%
\pgfpathlineto{\pgfqpoint{0.634712in}{3.781667in}}%
\pgfpathlineto{\pgfqpoint{0.643195in}{3.782456in}}%
\pgfpathlineto{\pgfqpoint{0.644424in}{3.782894in}}%
\pgfpathlineto{\pgfqpoint{0.645891in}{3.782456in}}%
\pgfpathlineto{\pgfqpoint{0.654135in}{3.778009in}}%
\pgfpathlineto{\pgfqpoint{0.658844in}{3.772807in}}%
\pgfpathlineto{\pgfqpoint{0.659913in}{3.763158in}}%
\pgfpathlineto{\pgfqpoint{0.654135in}{3.754540in}}%
\pgfpathlineto{\pgfqpoint{0.653280in}{3.753509in}}%
\pgfpathlineto{\pgfqpoint{0.644424in}{3.749441in}}%
\pgfpathlineto{\pgfqpoint{0.634712in}{3.750653in}}%
\pgfusepath{stroke}%
\end{pgfscope}%
\begin{pgfscope}%
\pgfpathrectangle{\pgfqpoint{0.625000in}{0.550000in}}{\pgfqpoint{3.875000in}{3.850000in}} %
\pgfusepath{clip}%
\pgfsetbuttcap%
\pgfsetroundjoin%
\pgfsetlinewidth{0.250937pt}%
\definecolor{currentstroke}{rgb}{0.000000,0.000000,0.000000}%
\pgfsetstrokecolor{currentstroke}%
\pgfsetdash{}{0pt}%
\pgfpathmoveto{\pgfqpoint{0.625000in}{0.635360in}}%
\pgfpathlineto{\pgfqpoint{0.633423in}{0.627193in}}%
\pgfpathlineto{\pgfqpoint{0.625000in}{0.619050in}}%
\pgfusepath{stroke}%
\end{pgfscope}%
\begin{pgfscope}%
\pgfpathrectangle{\pgfqpoint{0.625000in}{0.550000in}}{\pgfqpoint{3.875000in}{3.850000in}} %
\pgfusepath{clip}%
\pgfsetbuttcap%
\pgfsetroundjoin%
\pgfsetlinewidth{0.250937pt}%
\definecolor{currentstroke}{rgb}{0.000000,0.000000,0.000000}%
\pgfsetstrokecolor{currentstroke}%
\pgfsetdash{}{0pt}%
\pgfpathmoveto{\pgfqpoint{0.625000in}{0.789346in}}%
\pgfpathlineto{\pgfqpoint{0.633402in}{0.781579in}}%
\pgfpathlineto{\pgfqpoint{0.625000in}{0.773466in}}%
\pgfusepath{stroke}%
\end{pgfscope}%
\begin{pgfscope}%
\pgfpathrectangle{\pgfqpoint{0.625000in}{0.550000in}}{\pgfqpoint{3.875000in}{3.850000in}} %
\pgfusepath{clip}%
\pgfsetbuttcap%
\pgfsetroundjoin%
\pgfsetlinewidth{0.250937pt}%
\definecolor{currentstroke}{rgb}{0.000000,0.000000,0.000000}%
\pgfsetstrokecolor{currentstroke}%
\pgfsetdash{}{0pt}%
\pgfpathmoveto{\pgfqpoint{0.625000in}{0.944018in}}%
\pgfpathlineto{\pgfqpoint{0.634712in}{0.940014in}}%
\pgfpathlineto{\pgfqpoint{0.639040in}{0.935965in}}%
\pgfpathlineto{\pgfqpoint{0.634712in}{0.931916in}}%
\pgfpathlineto{\pgfqpoint{0.625000in}{0.927840in}}%
\pgfusepath{stroke}%
\end{pgfscope}%
\begin{pgfscope}%
\pgfpathrectangle{\pgfqpoint{0.625000in}{0.550000in}}{\pgfqpoint{3.875000in}{3.850000in}} %
\pgfusepath{clip}%
\pgfsetbuttcap%
\pgfsetroundjoin%
\pgfsetlinewidth{0.250937pt}%
\definecolor{currentstroke}{rgb}{0.000000,0.000000,0.000000}%
\pgfsetstrokecolor{currentstroke}%
\pgfsetdash{}{0pt}%
\pgfpathmoveto{\pgfqpoint{0.625000in}{1.098409in}}%
\pgfpathlineto{\pgfqpoint{0.633403in}{1.090351in}}%
\pgfpathlineto{\pgfqpoint{0.625000in}{1.082322in}}%
\pgfusepath{stroke}%
\end{pgfscope}%
\begin{pgfscope}%
\pgfpathrectangle{\pgfqpoint{0.625000in}{0.550000in}}{\pgfqpoint{3.875000in}{3.850000in}} %
\pgfusepath{clip}%
\pgfsetbuttcap%
\pgfsetroundjoin%
\pgfsetlinewidth{0.250937pt}%
\definecolor{currentstroke}{rgb}{0.000000,0.000000,0.000000}%
\pgfsetstrokecolor{currentstroke}%
\pgfsetdash{}{0pt}%
\pgfpathmoveto{\pgfqpoint{0.625000in}{1.253047in}}%
\pgfpathlineto{\pgfqpoint{0.633420in}{1.244737in}}%
\pgfpathlineto{\pgfqpoint{0.625000in}{1.236656in}}%
\pgfusepath{stroke}%
\end{pgfscope}%
\begin{pgfscope}%
\pgfpathrectangle{\pgfqpoint{0.625000in}{0.550000in}}{\pgfqpoint{3.875000in}{3.850000in}} %
\pgfusepath{clip}%
\pgfsetbuttcap%
\pgfsetroundjoin%
\pgfsetlinewidth{0.250937pt}%
\definecolor{currentstroke}{rgb}{0.000000,0.000000,0.000000}%
\pgfsetstrokecolor{currentstroke}%
\pgfsetdash{}{0pt}%
\pgfpathmoveto{\pgfqpoint{0.625000in}{1.407090in}}%
\pgfpathlineto{\pgfqpoint{0.633375in}{1.399123in}}%
\pgfpathlineto{\pgfqpoint{0.625000in}{1.391151in}}%
\pgfusepath{stroke}%
\end{pgfscope}%
\begin{pgfscope}%
\pgfpathrectangle{\pgfqpoint{0.625000in}{0.550000in}}{\pgfqpoint{3.875000in}{3.850000in}} %
\pgfusepath{clip}%
\pgfsetbuttcap%
\pgfsetroundjoin%
\pgfsetlinewidth{0.250937pt}%
\definecolor{currentstroke}{rgb}{0.000000,0.000000,0.000000}%
\pgfsetstrokecolor{currentstroke}%
\pgfsetdash{}{0pt}%
\pgfpathmoveto{\pgfqpoint{0.625000in}{1.561452in}}%
\pgfpathlineto{\pgfqpoint{0.633337in}{1.553509in}}%
\pgfpathlineto{\pgfqpoint{0.625000in}{1.545307in}}%
\pgfusepath{stroke}%
\end{pgfscope}%
\begin{pgfscope}%
\pgfpathrectangle{\pgfqpoint{0.625000in}{0.550000in}}{\pgfqpoint{3.875000in}{3.850000in}} %
\pgfusepath{clip}%
\pgfsetbuttcap%
\pgfsetroundjoin%
\pgfsetlinewidth{0.250937pt}%
\definecolor{currentstroke}{rgb}{0.000000,0.000000,0.000000}%
\pgfsetstrokecolor{currentstroke}%
\pgfsetdash{}{0pt}%
\pgfpathmoveto{\pgfqpoint{0.625000in}{1.715835in}}%
\pgfpathlineto{\pgfqpoint{0.634712in}{1.711944in}}%
\pgfpathlineto{\pgfqpoint{0.639040in}{1.707895in}}%
\pgfpathlineto{\pgfqpoint{0.634712in}{1.703846in}}%
\pgfpathlineto{\pgfqpoint{0.625000in}{1.699904in}}%
\pgfusepath{stroke}%
\end{pgfscope}%
\begin{pgfscope}%
\pgfpathrectangle{\pgfqpoint{0.625000in}{0.550000in}}{\pgfqpoint{3.875000in}{3.850000in}} %
\pgfusepath{clip}%
\pgfsetbuttcap%
\pgfsetroundjoin%
\pgfsetlinewidth{0.250937pt}%
\definecolor{currentstroke}{rgb}{0.000000,0.000000,0.000000}%
\pgfsetstrokecolor{currentstroke}%
\pgfsetdash{}{0pt}%
\pgfpathmoveto{\pgfqpoint{0.625000in}{1.870286in}}%
\pgfpathlineto{\pgfqpoint{0.633309in}{1.862281in}}%
\pgfpathlineto{\pgfqpoint{0.625000in}{1.854441in}}%
\pgfusepath{stroke}%
\end{pgfscope}%
\begin{pgfscope}%
\pgfpathrectangle{\pgfqpoint{0.625000in}{0.550000in}}{\pgfqpoint{3.875000in}{3.850000in}} %
\pgfusepath{clip}%
\pgfsetbuttcap%
\pgfsetroundjoin%
\pgfsetlinewidth{0.250937pt}%
\definecolor{currentstroke}{rgb}{0.000000,0.000000,0.000000}%
\pgfsetstrokecolor{currentstroke}%
\pgfsetdash{}{0pt}%
\pgfpathmoveto{\pgfqpoint{0.625000in}{2.024953in}}%
\pgfpathlineto{\pgfqpoint{0.633295in}{2.016667in}}%
\pgfpathlineto{\pgfqpoint{0.625000in}{2.008448in}}%
\pgfusepath{stroke}%
\end{pgfscope}%
\begin{pgfscope}%
\pgfpathrectangle{\pgfqpoint{0.625000in}{0.550000in}}{\pgfqpoint{3.875000in}{3.850000in}} %
\pgfusepath{clip}%
\pgfsetbuttcap%
\pgfsetroundjoin%
\pgfsetlinewidth{0.250937pt}%
\definecolor{currentstroke}{rgb}{0.000000,0.000000,0.000000}%
\pgfsetstrokecolor{currentstroke}%
\pgfsetdash{}{0pt}%
\pgfpathmoveto{\pgfqpoint{0.625000in}{2.178921in}}%
\pgfpathlineto{\pgfqpoint{0.633321in}{2.171053in}}%
\pgfpathlineto{\pgfqpoint{0.625000in}{2.162959in}}%
\pgfusepath{stroke}%
\end{pgfscope}%
\begin{pgfscope}%
\pgfpathrectangle{\pgfqpoint{0.625000in}{0.550000in}}{\pgfqpoint{3.875000in}{3.850000in}} %
\pgfusepath{clip}%
\pgfsetbuttcap%
\pgfsetroundjoin%
\pgfsetlinewidth{0.250937pt}%
\definecolor{currentstroke}{rgb}{0.000000,0.000000,0.000000}%
\pgfsetstrokecolor{currentstroke}%
\pgfsetdash{}{0pt}%
\pgfpathmoveto{\pgfqpoint{0.625000in}{2.333266in}}%
\pgfpathlineto{\pgfqpoint{0.633356in}{2.325439in}}%
\pgfpathlineto{\pgfqpoint{0.625000in}{2.317063in}}%
\pgfusepath{stroke}%
\end{pgfscope}%
\begin{pgfscope}%
\pgfpathrectangle{\pgfqpoint{0.625000in}{0.550000in}}{\pgfqpoint{3.875000in}{3.850000in}} %
\pgfusepath{clip}%
\pgfsetbuttcap%
\pgfsetroundjoin%
\pgfsetlinewidth{0.250937pt}%
\definecolor{currentstroke}{rgb}{0.000000,0.000000,0.000000}%
\pgfsetstrokecolor{currentstroke}%
\pgfsetdash{}{0pt}%
\pgfpathmoveto{\pgfqpoint{0.625000in}{2.488635in}}%
\pgfpathlineto{\pgfqpoint{0.625787in}{2.489474in}}%
\pgfpathlineto{\pgfqpoint{0.628260in}{2.508772in}}%
\pgfpathlineto{\pgfqpoint{0.634712in}{2.533797in}}%
\pgfpathlineto{\pgfqpoint{0.654135in}{2.572453in}}%
\pgfpathlineto{\pgfqpoint{0.673559in}{2.597110in}}%
\pgfpathlineto{\pgfqpoint{0.692982in}{2.615366in}}%
\pgfpathlineto{\pgfqpoint{0.712406in}{2.629582in}}%
\pgfpathlineto{\pgfqpoint{0.731830in}{2.640422in}}%
\pgfpathlineto{\pgfqpoint{0.751253in}{2.648650in}}%
\pgfpathlineto{\pgfqpoint{0.770677in}{2.654513in}}%
\pgfpathlineto{\pgfqpoint{0.790100in}{2.658409in}}%
\pgfpathlineto{\pgfqpoint{0.809524in}{2.660259in}}%
\pgfpathlineto{\pgfqpoint{0.828947in}{2.660194in}}%
\pgfpathlineto{\pgfqpoint{0.848371in}{2.658193in}}%
\pgfpathlineto{\pgfqpoint{0.870070in}{2.653509in}}%
\pgfpathlineto{\pgfqpoint{0.887218in}{2.647960in}}%
\pgfpathlineto{\pgfqpoint{0.906642in}{2.639320in}}%
\pgfpathlineto{\pgfqpoint{0.926065in}{2.627813in}}%
\pgfpathlineto{\pgfqpoint{0.943004in}{2.614912in}}%
\pgfpathlineto{\pgfqpoint{0.955201in}{2.603368in}}%
\pgfpathlineto{\pgfqpoint{0.970125in}{2.585965in}}%
\pgfpathlineto{\pgfqpoint{0.982719in}{2.566667in}}%
\pgfpathlineto{\pgfqpoint{0.992149in}{2.547368in}}%
\pgfpathlineto{\pgfqpoint{0.998915in}{2.528070in}}%
\pgfpathlineto{\pgfqpoint{1.003759in}{2.504806in}}%
\pgfpathlineto{\pgfqpoint{1.005382in}{2.489474in}}%
\pgfpathlineto{\pgfqpoint{1.005382in}{2.470175in}}%
\pgfpathlineto{\pgfqpoint{1.003220in}{2.450877in}}%
\pgfpathlineto{\pgfqpoint{0.998915in}{2.431579in}}%
\pgfpathlineto{\pgfqpoint{0.992149in}{2.412281in}}%
\pgfpathlineto{\pgfqpoint{0.982719in}{2.392982in}}%
\pgfpathlineto{\pgfqpoint{0.970125in}{2.373684in}}%
\pgfpathlineto{\pgfqpoint{0.953394in}{2.354386in}}%
\pgfpathlineto{\pgfqpoint{0.935777in}{2.338884in}}%
\pgfpathlineto{\pgfqpoint{0.926065in}{2.331836in}}%
\pgfpathlineto{\pgfqpoint{0.906642in}{2.320329in}}%
\pgfpathlineto{\pgfqpoint{0.887218in}{2.311689in}}%
\pgfpathlineto{\pgfqpoint{0.867794in}{2.305506in}}%
\pgfpathlineto{\pgfqpoint{0.848371in}{2.301456in}}%
\pgfpathlineto{\pgfqpoint{0.828947in}{2.299455in}}%
\pgfpathlineto{\pgfqpoint{0.809524in}{2.299390in}}%
\pgfpathlineto{\pgfqpoint{0.790100in}{2.301240in}}%
\pgfpathlineto{\pgfqpoint{0.767302in}{2.306140in}}%
\pgfpathlineto{\pgfqpoint{0.751253in}{2.310999in}}%
\pgfpathlineto{\pgfqpoint{0.731830in}{2.319227in}}%
\pgfpathlineto{\pgfqpoint{0.712406in}{2.330067in}}%
\pgfpathlineto{\pgfqpoint{0.692518in}{2.344737in}}%
\pgfpathlineto{\pgfqpoint{0.673559in}{2.362539in}}%
\pgfpathlineto{\pgfqpoint{0.650895in}{2.392982in}}%
\pgfpathlineto{\pgfqpoint{0.640686in}{2.412281in}}%
\pgfpathlineto{\pgfqpoint{0.633071in}{2.431579in}}%
\pgfpathlineto{\pgfqpoint{0.628260in}{2.450877in}}%
\pgfpathlineto{\pgfqpoint{0.625000in}{2.471014in}}%
\pgfpathlineto{\pgfqpoint{0.625000in}{2.471014in}}%
\pgfusepath{stroke}%
\end{pgfscope}%
\begin{pgfscope}%
\pgfpathrectangle{\pgfqpoint{0.625000in}{0.550000in}}{\pgfqpoint{3.875000in}{3.850000in}} %
\pgfusepath{clip}%
\pgfsetbuttcap%
\pgfsetroundjoin%
\pgfsetlinewidth{0.250937pt}%
\definecolor{currentstroke}{rgb}{0.000000,0.000000,0.000000}%
\pgfsetstrokecolor{currentstroke}%
\pgfsetdash{}{0pt}%
\pgfpathmoveto{\pgfqpoint{0.625000in}{2.642584in}}%
\pgfpathlineto{\pgfqpoint{0.633353in}{2.634211in}}%
\pgfpathlineto{\pgfqpoint{0.625000in}{2.626387in}}%
\pgfusepath{stroke}%
\end{pgfscope}%
\begin{pgfscope}%
\pgfpathrectangle{\pgfqpoint{0.625000in}{0.550000in}}{\pgfqpoint{3.875000in}{3.850000in}} %
\pgfusepath{clip}%
\pgfsetbuttcap%
\pgfsetroundjoin%
\pgfsetlinewidth{0.250937pt}%
\definecolor{currentstroke}{rgb}{0.000000,0.000000,0.000000}%
\pgfsetstrokecolor{currentstroke}%
\pgfsetdash{}{0pt}%
\pgfpathmoveto{\pgfqpoint{0.625000in}{2.796680in}}%
\pgfpathlineto{\pgfqpoint{0.633312in}{2.788596in}}%
\pgfpathlineto{\pgfqpoint{0.625000in}{2.780739in}}%
\pgfusepath{stroke}%
\end{pgfscope}%
\begin{pgfscope}%
\pgfpathrectangle{\pgfqpoint{0.625000in}{0.550000in}}{\pgfqpoint{3.875000in}{3.850000in}} %
\pgfusepath{clip}%
\pgfsetbuttcap%
\pgfsetroundjoin%
\pgfsetlinewidth{0.250937pt}%
\definecolor{currentstroke}{rgb}{0.000000,0.000000,0.000000}%
\pgfsetstrokecolor{currentstroke}%
\pgfsetdash{}{0pt}%
\pgfpathmoveto{\pgfqpoint{0.625000in}{2.951177in}}%
\pgfpathlineto{\pgfqpoint{0.633270in}{2.942982in}}%
\pgfpathlineto{\pgfqpoint{0.625000in}{2.934720in}}%
\pgfusepath{stroke}%
\end{pgfscope}%
\begin{pgfscope}%
\pgfpathrectangle{\pgfqpoint{0.625000in}{0.550000in}}{\pgfqpoint{3.875000in}{3.850000in}} %
\pgfusepath{clip}%
\pgfsetbuttcap%
\pgfsetroundjoin%
\pgfsetlinewidth{0.250937pt}%
\definecolor{currentstroke}{rgb}{0.000000,0.000000,0.000000}%
\pgfsetstrokecolor{currentstroke}%
\pgfsetdash{}{0pt}%
\pgfpathmoveto{\pgfqpoint{0.625000in}{3.105185in}}%
\pgfpathlineto{\pgfqpoint{0.633289in}{3.097368in}}%
\pgfpathlineto{\pgfqpoint{0.625000in}{3.089385in}}%
\pgfusepath{stroke}%
\end{pgfscope}%
\begin{pgfscope}%
\pgfpathrectangle{\pgfqpoint{0.625000in}{0.550000in}}{\pgfqpoint{3.875000in}{3.850000in}} %
\pgfusepath{clip}%
\pgfsetbuttcap%
\pgfsetroundjoin%
\pgfsetlinewidth{0.250937pt}%
\definecolor{currentstroke}{rgb}{0.000000,0.000000,0.000000}%
\pgfsetstrokecolor{currentstroke}%
\pgfsetdash{}{0pt}%
\pgfpathmoveto{\pgfqpoint{0.625000in}{3.259665in}}%
\pgfpathlineto{\pgfqpoint{0.634712in}{3.255803in}}%
\pgfpathlineto{\pgfqpoint{0.639040in}{3.251754in}}%
\pgfpathlineto{\pgfqpoint{0.634712in}{3.247705in}}%
\pgfpathlineto{\pgfqpoint{0.625000in}{3.243896in}}%
\pgfusepath{stroke}%
\end{pgfscope}%
\begin{pgfscope}%
\pgfpathrectangle{\pgfqpoint{0.625000in}{0.550000in}}{\pgfqpoint{3.875000in}{3.850000in}} %
\pgfusepath{clip}%
\pgfsetbuttcap%
\pgfsetroundjoin%
\pgfsetlinewidth{0.250937pt}%
\definecolor{currentstroke}{rgb}{0.000000,0.000000,0.000000}%
\pgfsetstrokecolor{currentstroke}%
\pgfsetdash{}{0pt}%
\pgfpathmoveto{\pgfqpoint{0.625000in}{3.414292in}}%
\pgfpathlineto{\pgfqpoint{0.633289in}{3.406140in}}%
\pgfpathlineto{\pgfqpoint{0.625000in}{3.398254in}}%
\pgfusepath{stroke}%
\end{pgfscope}%
\begin{pgfscope}%
\pgfpathrectangle{\pgfqpoint{0.625000in}{0.550000in}}{\pgfqpoint{3.875000in}{3.850000in}} %
\pgfusepath{clip}%
\pgfsetbuttcap%
\pgfsetroundjoin%
\pgfsetlinewidth{0.250937pt}%
\definecolor{currentstroke}{rgb}{0.000000,0.000000,0.000000}%
\pgfsetstrokecolor{currentstroke}%
\pgfsetdash{}{0pt}%
\pgfpathmoveto{\pgfqpoint{0.625000in}{3.568431in}}%
\pgfpathlineto{\pgfqpoint{0.633320in}{3.560526in}}%
\pgfpathlineto{\pgfqpoint{0.625000in}{3.552626in}}%
\pgfusepath{stroke}%
\end{pgfscope}%
\begin{pgfscope}%
\pgfpathrectangle{\pgfqpoint{0.625000in}{0.550000in}}{\pgfqpoint{3.875000in}{3.850000in}} %
\pgfusepath{clip}%
\pgfsetbuttcap%
\pgfsetroundjoin%
\pgfsetlinewidth{0.250937pt}%
\definecolor{currentstroke}{rgb}{0.000000,0.000000,0.000000}%
\pgfsetstrokecolor{currentstroke}%
\pgfsetdash{}{0pt}%
\pgfpathmoveto{\pgfqpoint{0.625000in}{3.722898in}}%
\pgfpathlineto{\pgfqpoint{0.633339in}{3.714912in}}%
\pgfpathlineto{\pgfqpoint{0.625000in}{3.706686in}}%
\pgfusepath{stroke}%
\end{pgfscope}%
\begin{pgfscope}%
\pgfpathrectangle{\pgfqpoint{0.625000in}{0.550000in}}{\pgfqpoint{3.875000in}{3.850000in}} %
\pgfusepath{clip}%
\pgfsetbuttcap%
\pgfsetroundjoin%
\pgfsetlinewidth{0.250937pt}%
\definecolor{currentstroke}{rgb}{0.000000,0.000000,0.000000}%
\pgfsetstrokecolor{currentstroke}%
\pgfsetdash{}{0pt}%
\pgfpathmoveto{\pgfqpoint{0.625000in}{3.877164in}}%
\pgfpathlineto{\pgfqpoint{0.633266in}{3.869298in}}%
\pgfpathlineto{\pgfqpoint{0.625000in}{3.861400in}}%
\pgfusepath{stroke}%
\end{pgfscope}%
\begin{pgfscope}%
\pgfpathrectangle{\pgfqpoint{0.625000in}{0.550000in}}{\pgfqpoint{3.875000in}{3.850000in}} %
\pgfusepath{clip}%
\pgfsetbuttcap%
\pgfsetroundjoin%
\pgfsetlinewidth{0.250937pt}%
\definecolor{currentstroke}{rgb}{0.000000,0.000000,0.000000}%
\pgfsetstrokecolor{currentstroke}%
\pgfsetdash{}{0pt}%
\pgfpathmoveto{\pgfqpoint{0.625000in}{4.031667in}}%
\pgfpathlineto{\pgfqpoint{0.634712in}{4.027733in}}%
\pgfpathlineto{\pgfqpoint{0.639040in}{4.023684in}}%
\pgfpathlineto{\pgfqpoint{0.634712in}{4.019635in}}%
\pgfpathlineto{\pgfqpoint{0.625000in}{4.015780in}}%
\pgfusepath{stroke}%
\end{pgfscope}%
\begin{pgfscope}%
\pgfpathrectangle{\pgfqpoint{0.625000in}{0.550000in}}{\pgfqpoint{3.875000in}{3.850000in}} %
\pgfusepath{clip}%
\pgfsetbuttcap%
\pgfsetroundjoin%
\pgfsetlinewidth{0.250937pt}%
\definecolor{currentstroke}{rgb}{0.000000,0.000000,0.000000}%
\pgfsetstrokecolor{currentstroke}%
\pgfsetdash{}{0pt}%
\pgfpathmoveto{\pgfqpoint{0.625000in}{4.186024in}}%
\pgfpathlineto{\pgfqpoint{0.633263in}{4.178070in}}%
\pgfpathlineto{\pgfqpoint{0.625000in}{4.170488in}}%
\pgfusepath{stroke}%
\end{pgfscope}%
\begin{pgfscope}%
\pgfpathrectangle{\pgfqpoint{0.625000in}{0.550000in}}{\pgfqpoint{3.875000in}{3.850000in}} %
\pgfusepath{clip}%
\pgfsetbuttcap%
\pgfsetroundjoin%
\pgfsetlinewidth{0.250937pt}%
\definecolor{currentstroke}{rgb}{0.000000,0.000000,0.000000}%
\pgfsetstrokecolor{currentstroke}%
\pgfsetdash{}{0pt}%
\pgfpathmoveto{\pgfqpoint{0.625000in}{4.340370in}}%
\pgfpathlineto{\pgfqpoint{0.633221in}{4.332456in}}%
\pgfpathlineto{\pgfqpoint{0.625000in}{4.324515in}}%
\pgfusepath{stroke}%
\end{pgfscope}%
\begin{pgfscope}%
\pgfpathrectangle{\pgfqpoint{0.625000in}{0.550000in}}{\pgfqpoint{3.875000in}{3.850000in}} %
\pgfusepath{clip}%
\pgfsetbuttcap%
\pgfsetroundjoin%
\pgfsetlinewidth{0.250937pt}%
\definecolor{currentstroke}{rgb}{0.000000,0.000000,0.000000}%
\pgfsetstrokecolor{currentstroke}%
\pgfsetdash{}{0pt}%
\pgfpathmoveto{\pgfqpoint{0.644424in}{1.175475in}}%
\pgfpathlineto{\pgfqpoint{0.639605in}{1.177193in}}%
\pgfpathlineto{\pgfqpoint{0.634712in}{1.177648in}}%
\pgfpathlineto{\pgfqpoint{0.628992in}{1.186842in}}%
\pgfpathlineto{\pgfqpoint{0.627420in}{1.196491in}}%
\pgfpathlineto{\pgfqpoint{0.632820in}{1.206140in}}%
\pgfpathlineto{\pgfqpoint{0.634712in}{1.209713in}}%
\pgfpathlineto{\pgfqpoint{0.644424in}{1.211293in}}%
\pgfpathlineto{\pgfqpoint{0.654135in}{1.207467in}}%
\pgfpathlineto{\pgfqpoint{0.656057in}{1.206140in}}%
\pgfpathlineto{\pgfqpoint{0.661823in}{1.196491in}}%
\pgfpathlineto{\pgfqpoint{0.660937in}{1.186842in}}%
\pgfpathlineto{\pgfqpoint{0.654135in}{1.179328in}}%
\pgfpathlineto{\pgfqpoint{0.650178in}{1.177193in}}%
\pgfpathlineto{\pgfqpoint{0.644424in}{1.175475in}}%
\pgfusepath{stroke}%
\end{pgfscope}%
\begin{pgfscope}%
\pgfpathrectangle{\pgfqpoint{0.625000in}{0.550000in}}{\pgfqpoint{3.875000in}{3.850000in}} %
\pgfusepath{clip}%
\pgfsetbuttcap%
\pgfsetroundjoin%
\pgfsetlinewidth{0.250937pt}%
\definecolor{currentstroke}{rgb}{0.000000,0.000000,0.000000}%
\pgfsetstrokecolor{currentstroke}%
\pgfsetdash{}{0pt}%
\pgfpathmoveto{\pgfqpoint{0.634712in}{3.749937in}}%
\pgfpathlineto{\pgfqpoint{0.632820in}{3.753509in}}%
\pgfpathlineto{\pgfqpoint{0.627420in}{3.763158in}}%
\pgfpathlineto{\pgfqpoint{0.628992in}{3.772807in}}%
\pgfpathlineto{\pgfqpoint{0.634712in}{3.782001in}}%
\pgfpathlineto{\pgfqpoint{0.639605in}{3.782456in}}%
\pgfpathlineto{\pgfqpoint{0.644424in}{3.784174in}}%
\pgfpathlineto{\pgfqpoint{0.650178in}{3.782456in}}%
\pgfpathlineto{\pgfqpoint{0.654135in}{3.780321in}}%
\pgfpathlineto{\pgfqpoint{0.660937in}{3.772807in}}%
\pgfpathlineto{\pgfqpoint{0.661823in}{3.763158in}}%
\pgfpathlineto{\pgfqpoint{0.656057in}{3.753509in}}%
\pgfpathlineto{\pgfqpoint{0.654135in}{3.752182in}}%
\pgfpathlineto{\pgfqpoint{0.644424in}{3.748356in}}%
\pgfpathlineto{\pgfqpoint{0.634712in}{3.749937in}}%
\pgfusepath{stroke}%
\end{pgfscope}%
\begin{pgfscope}%
\pgfpathrectangle{\pgfqpoint{0.625000in}{0.550000in}}{\pgfqpoint{3.875000in}{3.850000in}} %
\pgfusepath{clip}%
\pgfsetbuttcap%
\pgfsetroundjoin%
\pgfsetlinewidth{0.250937pt}%
\definecolor{currentstroke}{rgb}{0.000000,0.000000,0.000000}%
\pgfsetstrokecolor{currentstroke}%
\pgfsetdash{}{0pt}%
\pgfpathmoveto{\pgfqpoint{0.625000in}{0.635436in}}%
\pgfpathlineto{\pgfqpoint{0.633501in}{0.627193in}}%
\pgfpathlineto{\pgfqpoint{0.625000in}{0.618974in}}%
\pgfusepath{stroke}%
\end{pgfscope}%
\begin{pgfscope}%
\pgfpathrectangle{\pgfqpoint{0.625000in}{0.550000in}}{\pgfqpoint{3.875000in}{3.850000in}} %
\pgfusepath{clip}%
\pgfsetbuttcap%
\pgfsetroundjoin%
\pgfsetlinewidth{0.250937pt}%
\definecolor{currentstroke}{rgb}{0.000000,0.000000,0.000000}%
\pgfsetstrokecolor{currentstroke}%
\pgfsetdash{}{0pt}%
\pgfpathmoveto{\pgfqpoint{0.625000in}{0.789423in}}%
\pgfpathlineto{\pgfqpoint{0.633485in}{0.781579in}}%
\pgfpathlineto{\pgfqpoint{0.625000in}{0.773386in}}%
\pgfusepath{stroke}%
\end{pgfscope}%
\begin{pgfscope}%
\pgfpathrectangle{\pgfqpoint{0.625000in}{0.550000in}}{\pgfqpoint{3.875000in}{3.850000in}} %
\pgfusepath{clip}%
\pgfsetbuttcap%
\pgfsetroundjoin%
\pgfsetlinewidth{0.250937pt}%
\definecolor{currentstroke}{rgb}{0.000000,0.000000,0.000000}%
\pgfsetstrokecolor{currentstroke}%
\pgfsetdash{}{0pt}%
\pgfpathmoveto{\pgfqpoint{0.625000in}{0.944094in}}%
\pgfpathlineto{\pgfqpoint{0.634712in}{0.940827in}}%
\pgfpathlineto{\pgfqpoint{0.639908in}{0.935965in}}%
\pgfpathlineto{\pgfqpoint{0.634712in}{0.931103in}}%
\pgfpathlineto{\pgfqpoint{0.625000in}{0.927762in}}%
\pgfusepath{stroke}%
\end{pgfscope}%
\begin{pgfscope}%
\pgfpathrectangle{\pgfqpoint{0.625000in}{0.550000in}}{\pgfqpoint{3.875000in}{3.850000in}} %
\pgfusepath{clip}%
\pgfsetbuttcap%
\pgfsetroundjoin%
\pgfsetlinewidth{0.250937pt}%
\definecolor{currentstroke}{rgb}{0.000000,0.000000,0.000000}%
\pgfsetstrokecolor{currentstroke}%
\pgfsetdash{}{0pt}%
\pgfpathmoveto{\pgfqpoint{0.625000in}{1.098488in}}%
\pgfpathlineto{\pgfqpoint{0.633486in}{1.090351in}}%
\pgfpathlineto{\pgfqpoint{0.625000in}{1.082244in}}%
\pgfusepath{stroke}%
\end{pgfscope}%
\begin{pgfscope}%
\pgfpathrectangle{\pgfqpoint{0.625000in}{0.550000in}}{\pgfqpoint{3.875000in}{3.850000in}} %
\pgfusepath{clip}%
\pgfsetbuttcap%
\pgfsetroundjoin%
\pgfsetlinewidth{0.250937pt}%
\definecolor{currentstroke}{rgb}{0.000000,0.000000,0.000000}%
\pgfsetstrokecolor{currentstroke}%
\pgfsetdash{}{0pt}%
\pgfpathmoveto{\pgfqpoint{0.625000in}{1.253130in}}%
\pgfpathlineto{\pgfqpoint{0.633504in}{1.244737in}}%
\pgfpathlineto{\pgfqpoint{0.625000in}{1.236575in}}%
\pgfusepath{stroke}%
\end{pgfscope}%
\begin{pgfscope}%
\pgfpathrectangle{\pgfqpoint{0.625000in}{0.550000in}}{\pgfqpoint{3.875000in}{3.850000in}} %
\pgfusepath{clip}%
\pgfsetbuttcap%
\pgfsetroundjoin%
\pgfsetlinewidth{0.250937pt}%
\definecolor{currentstroke}{rgb}{0.000000,0.000000,0.000000}%
\pgfsetstrokecolor{currentstroke}%
\pgfsetdash{}{0pt}%
\pgfpathmoveto{\pgfqpoint{0.625000in}{1.407171in}}%
\pgfpathlineto{\pgfqpoint{0.633461in}{1.399123in}}%
\pgfpathlineto{\pgfqpoint{0.625000in}{1.391070in}}%
\pgfusepath{stroke}%
\end{pgfscope}%
\begin{pgfscope}%
\pgfpathrectangle{\pgfqpoint{0.625000in}{0.550000in}}{\pgfqpoint{3.875000in}{3.850000in}} %
\pgfusepath{clip}%
\pgfsetbuttcap%
\pgfsetroundjoin%
\pgfsetlinewidth{0.250937pt}%
\definecolor{currentstroke}{rgb}{0.000000,0.000000,0.000000}%
\pgfsetstrokecolor{currentstroke}%
\pgfsetdash{}{0pt}%
\pgfpathmoveto{\pgfqpoint{0.625000in}{1.561532in}}%
\pgfpathlineto{\pgfqpoint{0.633421in}{1.553509in}}%
\pgfpathlineto{\pgfqpoint{0.625000in}{1.545225in}}%
\pgfusepath{stroke}%
\end{pgfscope}%
\begin{pgfscope}%
\pgfpathrectangle{\pgfqpoint{0.625000in}{0.550000in}}{\pgfqpoint{3.875000in}{3.850000in}} %
\pgfusepath{clip}%
\pgfsetbuttcap%
\pgfsetroundjoin%
\pgfsetlinewidth{0.250937pt}%
\definecolor{currentstroke}{rgb}{0.000000,0.000000,0.000000}%
\pgfsetstrokecolor{currentstroke}%
\pgfsetdash{}{0pt}%
\pgfpathmoveto{\pgfqpoint{0.625000in}{1.715916in}}%
\pgfpathlineto{\pgfqpoint{0.634712in}{1.712757in}}%
\pgfpathlineto{\pgfqpoint{0.639908in}{1.707895in}}%
\pgfpathlineto{\pgfqpoint{0.634712in}{1.703033in}}%
\pgfpathlineto{\pgfqpoint{0.625000in}{1.699822in}}%
\pgfusepath{stroke}%
\end{pgfscope}%
\begin{pgfscope}%
\pgfpathrectangle{\pgfqpoint{0.625000in}{0.550000in}}{\pgfqpoint{3.875000in}{3.850000in}} %
\pgfusepath{clip}%
\pgfsetbuttcap%
\pgfsetroundjoin%
\pgfsetlinewidth{0.250937pt}%
\definecolor{currentstroke}{rgb}{0.000000,0.000000,0.000000}%
\pgfsetstrokecolor{currentstroke}%
\pgfsetdash{}{0pt}%
\pgfpathmoveto{\pgfqpoint{0.625000in}{1.870369in}}%
\pgfpathlineto{\pgfqpoint{0.633395in}{1.862281in}}%
\pgfpathlineto{\pgfqpoint{0.625000in}{1.854359in}}%
\pgfusepath{stroke}%
\end{pgfscope}%
\begin{pgfscope}%
\pgfpathrectangle{\pgfqpoint{0.625000in}{0.550000in}}{\pgfqpoint{3.875000in}{3.850000in}} %
\pgfusepath{clip}%
\pgfsetbuttcap%
\pgfsetroundjoin%
\pgfsetlinewidth{0.250937pt}%
\definecolor{currentstroke}{rgb}{0.000000,0.000000,0.000000}%
\pgfsetstrokecolor{currentstroke}%
\pgfsetdash{}{0pt}%
\pgfpathmoveto{\pgfqpoint{0.625000in}{2.025041in}}%
\pgfpathlineto{\pgfqpoint{0.633382in}{2.016667in}}%
\pgfpathlineto{\pgfqpoint{0.625000in}{2.008361in}}%
\pgfusepath{stroke}%
\end{pgfscope}%
\begin{pgfscope}%
\pgfpathrectangle{\pgfqpoint{0.625000in}{0.550000in}}{\pgfqpoint{3.875000in}{3.850000in}} %
\pgfusepath{clip}%
\pgfsetbuttcap%
\pgfsetroundjoin%
\pgfsetlinewidth{0.250937pt}%
\definecolor{currentstroke}{rgb}{0.000000,0.000000,0.000000}%
\pgfsetstrokecolor{currentstroke}%
\pgfsetdash{}{0pt}%
\pgfpathmoveto{\pgfqpoint{0.625000in}{2.179003in}}%
\pgfpathlineto{\pgfqpoint{0.633408in}{2.171053in}}%
\pgfpathlineto{\pgfqpoint{0.625000in}{2.162875in}}%
\pgfusepath{stroke}%
\end{pgfscope}%
\begin{pgfscope}%
\pgfpathrectangle{\pgfqpoint{0.625000in}{0.550000in}}{\pgfqpoint{3.875000in}{3.850000in}} %
\pgfusepath{clip}%
\pgfsetbuttcap%
\pgfsetroundjoin%
\pgfsetlinewidth{0.250937pt}%
\definecolor{currentstroke}{rgb}{0.000000,0.000000,0.000000}%
\pgfsetstrokecolor{currentstroke}%
\pgfsetdash{}{0pt}%
\pgfpathmoveto{\pgfqpoint{0.625000in}{2.333348in}}%
\pgfpathlineto{\pgfqpoint{0.633444in}{2.325439in}}%
\pgfpathlineto{\pgfqpoint{0.625000in}{2.316975in}}%
\pgfusepath{stroke}%
\end{pgfscope}%
\begin{pgfscope}%
\pgfpathrectangle{\pgfqpoint{0.625000in}{0.550000in}}{\pgfqpoint{3.875000in}{3.850000in}} %
\pgfusepath{clip}%
\pgfsetbuttcap%
\pgfsetroundjoin%
\pgfsetlinewidth{0.250937pt}%
\definecolor{currentstroke}{rgb}{0.000000,0.000000,0.000000}%
\pgfsetstrokecolor{currentstroke}%
\pgfsetdash{}{0pt}%
\pgfpathmoveto{\pgfqpoint{0.625000in}{2.488719in}}%
\pgfpathlineto{\pgfqpoint{0.625708in}{2.489474in}}%
\pgfpathlineto{\pgfqpoint{0.628106in}{2.508772in}}%
\pgfpathlineto{\pgfqpoint{0.634712in}{2.535210in}}%
\pgfpathlineto{\pgfqpoint{0.654978in}{2.576316in}}%
\pgfpathlineto{\pgfqpoint{0.673559in}{2.601384in}}%
\pgfpathlineto{\pgfqpoint{0.702694in}{2.628768in}}%
\pgfpathlineto{\pgfqpoint{0.725367in}{2.643860in}}%
\pgfpathlineto{\pgfqpoint{0.741541in}{2.652642in}}%
\pgfpathlineto{\pgfqpoint{0.770677in}{2.664055in}}%
\pgfpathlineto{\pgfqpoint{0.790100in}{2.669142in}}%
\pgfpathlineto{\pgfqpoint{0.816024in}{2.672807in}}%
\pgfpathlineto{\pgfqpoint{0.828947in}{2.673581in}}%
\pgfpathlineto{\pgfqpoint{0.851957in}{2.672807in}}%
\pgfpathlineto{\pgfqpoint{0.867794in}{2.670903in}}%
\pgfpathlineto{\pgfqpoint{0.887218in}{2.666779in}}%
\pgfpathlineto{\pgfqpoint{0.906642in}{2.660621in}}%
\pgfpathlineto{\pgfqpoint{0.926065in}{2.652224in}}%
\pgfpathlineto{\pgfqpoint{0.945489in}{2.641230in}}%
\pgfpathlineto{\pgfqpoint{0.964912in}{2.627050in}}%
\pgfpathlineto{\pgfqpoint{0.978358in}{2.614912in}}%
\pgfpathlineto{\pgfqpoint{0.995352in}{2.595614in}}%
\pgfpathlineto{\pgfqpoint{1.008472in}{2.576316in}}%
\pgfpathlineto{\pgfqpoint{1.018487in}{2.557018in}}%
\pgfpathlineto{\pgfqpoint{1.025898in}{2.537719in}}%
\pgfpathlineto{\pgfqpoint{1.031018in}{2.518421in}}%
\pgfpathlineto{\pgfqpoint{1.034017in}{2.499123in}}%
\pgfpathlineto{\pgfqpoint{1.035028in}{2.479825in}}%
\pgfpathlineto{\pgfqpoint{1.034017in}{2.460526in}}%
\pgfpathlineto{\pgfqpoint{1.031018in}{2.441228in}}%
\pgfpathlineto{\pgfqpoint{1.025898in}{2.421930in}}%
\pgfpathlineto{\pgfqpoint{1.018487in}{2.402632in}}%
\pgfpathlineto{\pgfqpoint{1.008472in}{2.383333in}}%
\pgfpathlineto{\pgfqpoint{0.994048in}{2.362422in}}%
\pgfpathlineto{\pgfqpoint{0.978358in}{2.344737in}}%
\pgfpathlineto{\pgfqpoint{0.964912in}{2.332600in}}%
\pgfpathlineto{\pgfqpoint{0.945489in}{2.318420in}}%
\pgfpathlineto{\pgfqpoint{0.926065in}{2.307425in}}%
\pgfpathlineto{\pgfqpoint{0.906642in}{2.299029in}}%
\pgfpathlineto{\pgfqpoint{0.887218in}{2.292870in}}%
\pgfpathlineto{\pgfqpoint{0.867794in}{2.288746in}}%
\pgfpathlineto{\pgfqpoint{0.848371in}{2.286517in}}%
\pgfpathlineto{\pgfqpoint{0.828947in}{2.286069in}}%
\pgfpathlineto{\pgfqpoint{0.809524in}{2.287416in}}%
\pgfpathlineto{\pgfqpoint{0.790100in}{2.290507in}}%
\pgfpathlineto{\pgfqpoint{0.768192in}{2.296491in}}%
\pgfpathlineto{\pgfqpoint{0.751253in}{2.302585in}}%
\pgfpathlineto{\pgfqpoint{0.725367in}{2.315789in}}%
\pgfpathlineto{\pgfqpoint{0.712406in}{2.323851in}}%
\pgfpathlineto{\pgfqpoint{0.683271in}{2.347957in}}%
\pgfpathlineto{\pgfqpoint{0.663847in}{2.370201in}}%
\pgfpathlineto{\pgfqpoint{0.644062in}{2.402632in}}%
\pgfpathlineto{\pgfqpoint{0.634712in}{2.424439in}}%
\pgfpathlineto{\pgfqpoint{0.628106in}{2.450877in}}%
\pgfpathlineto{\pgfqpoint{0.625000in}{2.470930in}}%
\pgfpathlineto{\pgfqpoint{0.625000in}{2.470930in}}%
\pgfusepath{stroke}%
\end{pgfscope}%
\begin{pgfscope}%
\pgfpathrectangle{\pgfqpoint{0.625000in}{0.550000in}}{\pgfqpoint{3.875000in}{3.850000in}} %
\pgfusepath{clip}%
\pgfsetbuttcap%
\pgfsetroundjoin%
\pgfsetlinewidth{0.250937pt}%
\definecolor{currentstroke}{rgb}{0.000000,0.000000,0.000000}%
\pgfsetstrokecolor{currentstroke}%
\pgfsetdash{}{0pt}%
\pgfpathmoveto{\pgfqpoint{0.625000in}{2.642672in}}%
\pgfpathlineto{\pgfqpoint{0.633442in}{2.634211in}}%
\pgfpathlineto{\pgfqpoint{0.625000in}{2.626304in}}%
\pgfusepath{stroke}%
\end{pgfscope}%
\begin{pgfscope}%
\pgfpathrectangle{\pgfqpoint{0.625000in}{0.550000in}}{\pgfqpoint{3.875000in}{3.850000in}} %
\pgfusepath{clip}%
\pgfsetbuttcap%
\pgfsetroundjoin%
\pgfsetlinewidth{0.250937pt}%
\definecolor{currentstroke}{rgb}{0.000000,0.000000,0.000000}%
\pgfsetstrokecolor{currentstroke}%
\pgfsetdash{}{0pt}%
\pgfpathmoveto{\pgfqpoint{0.625000in}{2.796766in}}%
\pgfpathlineto{\pgfqpoint{0.633399in}{2.788596in}}%
\pgfpathlineto{\pgfqpoint{0.625000in}{2.780656in}}%
\pgfusepath{stroke}%
\end{pgfscope}%
\begin{pgfscope}%
\pgfpathrectangle{\pgfqpoint{0.625000in}{0.550000in}}{\pgfqpoint{3.875000in}{3.850000in}} %
\pgfusepath{clip}%
\pgfsetbuttcap%
\pgfsetroundjoin%
\pgfsetlinewidth{0.250937pt}%
\definecolor{currentstroke}{rgb}{0.000000,0.000000,0.000000}%
\pgfsetstrokecolor{currentstroke}%
\pgfsetdash{}{0pt}%
\pgfpathmoveto{\pgfqpoint{0.625000in}{2.951265in}}%
\pgfpathlineto{\pgfqpoint{0.633359in}{2.942982in}}%
\pgfpathlineto{\pgfqpoint{0.625000in}{2.934631in}}%
\pgfusepath{stroke}%
\end{pgfscope}%
\begin{pgfscope}%
\pgfpathrectangle{\pgfqpoint{0.625000in}{0.550000in}}{\pgfqpoint{3.875000in}{3.850000in}} %
\pgfusepath{clip}%
\pgfsetbuttcap%
\pgfsetroundjoin%
\pgfsetlinewidth{0.250937pt}%
\definecolor{currentstroke}{rgb}{0.000000,0.000000,0.000000}%
\pgfsetstrokecolor{currentstroke}%
\pgfsetdash{}{0pt}%
\pgfpathmoveto{\pgfqpoint{0.625000in}{3.105267in}}%
\pgfpathlineto{\pgfqpoint{0.633377in}{3.097368in}}%
\pgfpathlineto{\pgfqpoint{0.625000in}{3.089301in}}%
\pgfusepath{stroke}%
\end{pgfscope}%
\begin{pgfscope}%
\pgfpathrectangle{\pgfqpoint{0.625000in}{0.550000in}}{\pgfqpoint{3.875000in}{3.850000in}} %
\pgfusepath{clip}%
\pgfsetbuttcap%
\pgfsetroundjoin%
\pgfsetlinewidth{0.250937pt}%
\definecolor{currentstroke}{rgb}{0.000000,0.000000,0.000000}%
\pgfsetstrokecolor{currentstroke}%
\pgfsetdash{}{0pt}%
\pgfpathmoveto{\pgfqpoint{0.625000in}{3.259751in}}%
\pgfpathlineto{\pgfqpoint{0.634712in}{3.256616in}}%
\pgfpathlineto{\pgfqpoint{0.639908in}{3.251754in}}%
\pgfpathlineto{\pgfqpoint{0.634712in}{3.246893in}}%
\pgfpathlineto{\pgfqpoint{0.625000in}{3.243811in}}%
\pgfusepath{stroke}%
\end{pgfscope}%
\begin{pgfscope}%
\pgfpathrectangle{\pgfqpoint{0.625000in}{0.550000in}}{\pgfqpoint{3.875000in}{3.850000in}} %
\pgfusepath{clip}%
\pgfsetbuttcap%
\pgfsetroundjoin%
\pgfsetlinewidth{0.250937pt}%
\definecolor{currentstroke}{rgb}{0.000000,0.000000,0.000000}%
\pgfsetstrokecolor{currentstroke}%
\pgfsetdash{}{0pt}%
\pgfpathmoveto{\pgfqpoint{0.625000in}{3.414377in}}%
\pgfpathlineto{\pgfqpoint{0.633376in}{3.406140in}}%
\pgfpathlineto{\pgfqpoint{0.625000in}{3.398171in}}%
\pgfusepath{stroke}%
\end{pgfscope}%
\begin{pgfscope}%
\pgfpathrectangle{\pgfqpoint{0.625000in}{0.550000in}}{\pgfqpoint{3.875000in}{3.850000in}} %
\pgfusepath{clip}%
\pgfsetbuttcap%
\pgfsetroundjoin%
\pgfsetlinewidth{0.250937pt}%
\definecolor{currentstroke}{rgb}{0.000000,0.000000,0.000000}%
\pgfsetstrokecolor{currentstroke}%
\pgfsetdash{}{0pt}%
\pgfpathmoveto{\pgfqpoint{0.625000in}{3.568516in}}%
\pgfpathlineto{\pgfqpoint{0.633409in}{3.560526in}}%
\pgfpathlineto{\pgfqpoint{0.625000in}{3.552541in}}%
\pgfusepath{stroke}%
\end{pgfscope}%
\begin{pgfscope}%
\pgfpathrectangle{\pgfqpoint{0.625000in}{0.550000in}}{\pgfqpoint{3.875000in}{3.850000in}} %
\pgfusepath{clip}%
\pgfsetbuttcap%
\pgfsetroundjoin%
\pgfsetlinewidth{0.250937pt}%
\definecolor{currentstroke}{rgb}{0.000000,0.000000,0.000000}%
\pgfsetstrokecolor{currentstroke}%
\pgfsetdash{}{0pt}%
\pgfpathmoveto{\pgfqpoint{0.625000in}{3.722983in}}%
\pgfpathlineto{\pgfqpoint{0.633428in}{3.714912in}}%
\pgfpathlineto{\pgfqpoint{0.625000in}{3.706598in}}%
\pgfusepath{stroke}%
\end{pgfscope}%
\begin{pgfscope}%
\pgfpathrectangle{\pgfqpoint{0.625000in}{0.550000in}}{\pgfqpoint{3.875000in}{3.850000in}} %
\pgfusepath{clip}%
\pgfsetbuttcap%
\pgfsetroundjoin%
\pgfsetlinewidth{0.250937pt}%
\definecolor{currentstroke}{rgb}{0.000000,0.000000,0.000000}%
\pgfsetstrokecolor{currentstroke}%
\pgfsetdash{}{0pt}%
\pgfpathmoveto{\pgfqpoint{0.625000in}{3.877251in}}%
\pgfpathlineto{\pgfqpoint{0.633357in}{3.869298in}}%
\pgfpathlineto{\pgfqpoint{0.625000in}{3.861313in}}%
\pgfusepath{stroke}%
\end{pgfscope}%
\begin{pgfscope}%
\pgfpathrectangle{\pgfqpoint{0.625000in}{0.550000in}}{\pgfqpoint{3.875000in}{3.850000in}} %
\pgfusepath{clip}%
\pgfsetbuttcap%
\pgfsetroundjoin%
\pgfsetlinewidth{0.250937pt}%
\definecolor{currentstroke}{rgb}{0.000000,0.000000,0.000000}%
\pgfsetstrokecolor{currentstroke}%
\pgfsetdash{}{0pt}%
\pgfpathmoveto{\pgfqpoint{0.625000in}{4.031751in}}%
\pgfpathlineto{\pgfqpoint{0.634712in}{4.028546in}}%
\pgfpathlineto{\pgfqpoint{0.639908in}{4.023684in}}%
\pgfpathlineto{\pgfqpoint{0.634712in}{4.018822in}}%
\pgfpathlineto{\pgfqpoint{0.625000in}{4.015696in}}%
\pgfusepath{stroke}%
\end{pgfscope}%
\begin{pgfscope}%
\pgfpathrectangle{\pgfqpoint{0.625000in}{0.550000in}}{\pgfqpoint{3.875000in}{3.850000in}} %
\pgfusepath{clip}%
\pgfsetbuttcap%
\pgfsetroundjoin%
\pgfsetlinewidth{0.250937pt}%
\definecolor{currentstroke}{rgb}{0.000000,0.000000,0.000000}%
\pgfsetstrokecolor{currentstroke}%
\pgfsetdash{}{0pt}%
\pgfpathmoveto{\pgfqpoint{0.625000in}{4.186113in}}%
\pgfpathlineto{\pgfqpoint{0.633355in}{4.178070in}}%
\pgfpathlineto{\pgfqpoint{0.625000in}{4.170404in}}%
\pgfusepath{stroke}%
\end{pgfscope}%
\begin{pgfscope}%
\pgfpathrectangle{\pgfqpoint{0.625000in}{0.550000in}}{\pgfqpoint{3.875000in}{3.850000in}} %
\pgfusepath{clip}%
\pgfsetbuttcap%
\pgfsetroundjoin%
\pgfsetlinewidth{0.250937pt}%
\definecolor{currentstroke}{rgb}{0.000000,0.000000,0.000000}%
\pgfsetstrokecolor{currentstroke}%
\pgfsetdash{}{0pt}%
\pgfpathmoveto{\pgfqpoint{0.625000in}{4.340458in}}%
\pgfpathlineto{\pgfqpoint{0.633312in}{4.332456in}}%
\pgfpathlineto{\pgfqpoint{0.625000in}{4.324427in}}%
\pgfusepath{stroke}%
\end{pgfscope}%
\begin{pgfscope}%
\pgfpathrectangle{\pgfqpoint{0.625000in}{0.550000in}}{\pgfqpoint{3.875000in}{3.850000in}} %
\pgfusepath{clip}%
\pgfsetbuttcap%
\pgfsetroundjoin%
\pgfsetlinewidth{0.250937pt}%
\definecolor{currentstroke}{rgb}{0.000000,0.000000,0.000000}%
\pgfsetstrokecolor{currentstroke}%
\pgfsetdash{}{0pt}%
\pgfpathmoveto{\pgfqpoint{0.644424in}{1.174196in}}%
\pgfpathlineto{\pgfqpoint{0.636015in}{1.177193in}}%
\pgfpathlineto{\pgfqpoint{0.634712in}{1.177314in}}%
\pgfpathlineto{\pgfqpoint{0.628784in}{1.186842in}}%
\pgfpathlineto{\pgfqpoint{0.627245in}{1.196491in}}%
\pgfpathlineto{\pgfqpoint{0.632440in}{1.206140in}}%
\pgfpathlineto{\pgfqpoint{0.634712in}{1.210430in}}%
\pgfpathlineto{\pgfqpoint{0.644424in}{1.212379in}}%
\pgfpathlineto{\pgfqpoint{0.654135in}{1.209547in}}%
\pgfpathlineto{\pgfqpoint{0.659069in}{1.206140in}}%
\pgfpathlineto{\pgfqpoint{0.663733in}{1.196491in}}%
\pgfpathlineto{\pgfqpoint{0.663030in}{1.186842in}}%
\pgfpathlineto{\pgfqpoint{0.654440in}{1.177193in}}%
\pgfpathlineto{\pgfqpoint{0.654135in}{1.177027in}}%
\pgfpathlineto{\pgfqpoint{0.644424in}{1.174196in}}%
\pgfusepath{stroke}%
\end{pgfscope}%
\begin{pgfscope}%
\pgfpathrectangle{\pgfqpoint{0.625000in}{0.550000in}}{\pgfqpoint{3.875000in}{3.850000in}} %
\pgfusepath{clip}%
\pgfsetbuttcap%
\pgfsetroundjoin%
\pgfsetlinewidth{0.250937pt}%
\definecolor{currentstroke}{rgb}{0.000000,0.000000,0.000000}%
\pgfsetstrokecolor{currentstroke}%
\pgfsetdash{}{0pt}%
\pgfpathmoveto{\pgfqpoint{0.634712in}{3.749220in}}%
\pgfpathlineto{\pgfqpoint{0.632440in}{3.753509in}}%
\pgfpathlineto{\pgfqpoint{0.627245in}{3.763158in}}%
\pgfpathlineto{\pgfqpoint{0.628784in}{3.772807in}}%
\pgfpathlineto{\pgfqpoint{0.634712in}{3.782335in}}%
\pgfpathlineto{\pgfqpoint{0.636015in}{3.782456in}}%
\pgfpathlineto{\pgfqpoint{0.644424in}{3.785454in}}%
\pgfpathlineto{\pgfqpoint{0.654135in}{3.782622in}}%
\pgfpathlineto{\pgfqpoint{0.654440in}{3.782456in}}%
\pgfpathlineto{\pgfqpoint{0.663030in}{3.772807in}}%
\pgfpathlineto{\pgfqpoint{0.663733in}{3.763158in}}%
\pgfpathlineto{\pgfqpoint{0.659069in}{3.753509in}}%
\pgfpathlineto{\pgfqpoint{0.654135in}{3.750102in}}%
\pgfpathlineto{\pgfqpoint{0.644424in}{3.747270in}}%
\pgfpathlineto{\pgfqpoint{0.634712in}{3.749220in}}%
\pgfusepath{stroke}%
\end{pgfscope}%
\begin{pgfscope}%
\pgfpathrectangle{\pgfqpoint{0.625000in}{0.550000in}}{\pgfqpoint{3.875000in}{3.850000in}} %
\pgfusepath{clip}%
\pgfsetbuttcap%
\pgfsetroundjoin%
\pgfsetlinewidth{0.250937pt}%
\definecolor{currentstroke}{rgb}{0.000000,0.000000,0.000000}%
\pgfsetstrokecolor{currentstroke}%
\pgfsetdash{}{0pt}%
\pgfpathmoveto{\pgfqpoint{0.625000in}{0.635512in}}%
\pgfpathlineto{\pgfqpoint{0.633580in}{0.627193in}}%
\pgfpathlineto{\pgfqpoint{0.625000in}{0.618898in}}%
\pgfusepath{stroke}%
\end{pgfscope}%
\begin{pgfscope}%
\pgfpathrectangle{\pgfqpoint{0.625000in}{0.550000in}}{\pgfqpoint{3.875000in}{3.850000in}} %
\pgfusepath{clip}%
\pgfsetbuttcap%
\pgfsetroundjoin%
\pgfsetlinewidth{0.250937pt}%
\definecolor{currentstroke}{rgb}{0.000000,0.000000,0.000000}%
\pgfsetstrokecolor{currentstroke}%
\pgfsetdash{}{0pt}%
\pgfpathmoveto{\pgfqpoint{0.625000in}{0.789500in}}%
\pgfpathlineto{\pgfqpoint{0.633569in}{0.781579in}}%
\pgfpathlineto{\pgfqpoint{0.625000in}{0.773306in}}%
\pgfusepath{stroke}%
\end{pgfscope}%
\begin{pgfscope}%
\pgfpathrectangle{\pgfqpoint{0.625000in}{0.550000in}}{\pgfqpoint{3.875000in}{3.850000in}} %
\pgfusepath{clip}%
\pgfsetbuttcap%
\pgfsetroundjoin%
\pgfsetlinewidth{0.250937pt}%
\definecolor{currentstroke}{rgb}{0.000000,0.000000,0.000000}%
\pgfsetstrokecolor{currentstroke}%
\pgfsetdash{}{0pt}%
\pgfpathmoveto{\pgfqpoint{0.625000in}{0.944171in}}%
\pgfpathlineto{\pgfqpoint{0.634712in}{0.941639in}}%
\pgfpathlineto{\pgfqpoint{0.640777in}{0.935965in}}%
\pgfpathlineto{\pgfqpoint{0.634712in}{0.930290in}}%
\pgfpathlineto{\pgfqpoint{0.625000in}{0.927685in}}%
\pgfusepath{stroke}%
\end{pgfscope}%
\begin{pgfscope}%
\pgfpathrectangle{\pgfqpoint{0.625000in}{0.550000in}}{\pgfqpoint{3.875000in}{3.850000in}} %
\pgfusepath{clip}%
\pgfsetbuttcap%
\pgfsetroundjoin%
\pgfsetlinewidth{0.250937pt}%
\definecolor{currentstroke}{rgb}{0.000000,0.000000,0.000000}%
\pgfsetstrokecolor{currentstroke}%
\pgfsetdash{}{0pt}%
\pgfpathmoveto{\pgfqpoint{0.625000in}{1.098567in}}%
\pgfpathlineto{\pgfqpoint{0.633568in}{1.090351in}}%
\pgfpathlineto{\pgfqpoint{0.625000in}{1.082165in}}%
\pgfusepath{stroke}%
\end{pgfscope}%
\begin{pgfscope}%
\pgfpathrectangle{\pgfqpoint{0.625000in}{0.550000in}}{\pgfqpoint{3.875000in}{3.850000in}} %
\pgfusepath{clip}%
\pgfsetbuttcap%
\pgfsetroundjoin%
\pgfsetlinewidth{0.250937pt}%
\definecolor{currentstroke}{rgb}{0.000000,0.000000,0.000000}%
\pgfsetstrokecolor{currentstroke}%
\pgfsetdash{}{0pt}%
\pgfpathmoveto{\pgfqpoint{0.625000in}{1.253213in}}%
\pgfpathlineto{\pgfqpoint{0.633588in}{1.244737in}}%
\pgfpathlineto{\pgfqpoint{0.625000in}{1.236495in}}%
\pgfusepath{stroke}%
\end{pgfscope}%
\begin{pgfscope}%
\pgfpathrectangle{\pgfqpoint{0.625000in}{0.550000in}}{\pgfqpoint{3.875000in}{3.850000in}} %
\pgfusepath{clip}%
\pgfsetbuttcap%
\pgfsetroundjoin%
\pgfsetlinewidth{0.250937pt}%
\definecolor{currentstroke}{rgb}{0.000000,0.000000,0.000000}%
\pgfsetstrokecolor{currentstroke}%
\pgfsetdash{}{0pt}%
\pgfpathmoveto{\pgfqpoint{0.625000in}{1.407253in}}%
\pgfpathlineto{\pgfqpoint{0.633547in}{1.399123in}}%
\pgfpathlineto{\pgfqpoint{0.625000in}{1.390988in}}%
\pgfusepath{stroke}%
\end{pgfscope}%
\begin{pgfscope}%
\pgfpathrectangle{\pgfqpoint{0.625000in}{0.550000in}}{\pgfqpoint{3.875000in}{3.850000in}} %
\pgfusepath{clip}%
\pgfsetbuttcap%
\pgfsetroundjoin%
\pgfsetlinewidth{0.250937pt}%
\definecolor{currentstroke}{rgb}{0.000000,0.000000,0.000000}%
\pgfsetstrokecolor{currentstroke}%
\pgfsetdash{}{0pt}%
\pgfpathmoveto{\pgfqpoint{0.625000in}{1.561612in}}%
\pgfpathlineto{\pgfqpoint{0.633505in}{1.553509in}}%
\pgfpathlineto{\pgfqpoint{0.625000in}{1.545142in}}%
\pgfusepath{stroke}%
\end{pgfscope}%
\begin{pgfscope}%
\pgfpathrectangle{\pgfqpoint{0.625000in}{0.550000in}}{\pgfqpoint{3.875000in}{3.850000in}} %
\pgfusepath{clip}%
\pgfsetbuttcap%
\pgfsetroundjoin%
\pgfsetlinewidth{0.250937pt}%
\definecolor{currentstroke}{rgb}{0.000000,0.000000,0.000000}%
\pgfsetstrokecolor{currentstroke}%
\pgfsetdash{}{0pt}%
\pgfpathmoveto{\pgfqpoint{0.625000in}{1.715997in}}%
\pgfpathlineto{\pgfqpoint{0.634712in}{1.713569in}}%
\pgfpathlineto{\pgfqpoint{0.640777in}{1.707895in}}%
\pgfpathlineto{\pgfqpoint{0.634712in}{1.702220in}}%
\pgfpathlineto{\pgfqpoint{0.625000in}{1.699741in}}%
\pgfusepath{stroke}%
\end{pgfscope}%
\begin{pgfscope}%
\pgfpathrectangle{\pgfqpoint{0.625000in}{0.550000in}}{\pgfqpoint{3.875000in}{3.850000in}} %
\pgfusepath{clip}%
\pgfsetbuttcap%
\pgfsetroundjoin%
\pgfsetlinewidth{0.250937pt}%
\definecolor{currentstroke}{rgb}{0.000000,0.000000,0.000000}%
\pgfsetstrokecolor{currentstroke}%
\pgfsetdash{}{0pt}%
\pgfpathmoveto{\pgfqpoint{0.625000in}{1.870452in}}%
\pgfpathlineto{\pgfqpoint{0.633481in}{1.862281in}}%
\pgfpathlineto{\pgfqpoint{0.625000in}{1.854278in}}%
\pgfusepath{stroke}%
\end{pgfscope}%
\begin{pgfscope}%
\pgfpathrectangle{\pgfqpoint{0.625000in}{0.550000in}}{\pgfqpoint{3.875000in}{3.850000in}} %
\pgfusepath{clip}%
\pgfsetbuttcap%
\pgfsetroundjoin%
\pgfsetlinewidth{0.250937pt}%
\definecolor{currentstroke}{rgb}{0.000000,0.000000,0.000000}%
\pgfsetstrokecolor{currentstroke}%
\pgfsetdash{}{0pt}%
\pgfpathmoveto{\pgfqpoint{0.625000in}{2.025128in}}%
\pgfpathlineto{\pgfqpoint{0.633469in}{2.016667in}}%
\pgfpathlineto{\pgfqpoint{0.625000in}{2.008274in}}%
\pgfusepath{stroke}%
\end{pgfscope}%
\begin{pgfscope}%
\pgfpathrectangle{\pgfqpoint{0.625000in}{0.550000in}}{\pgfqpoint{3.875000in}{3.850000in}} %
\pgfusepath{clip}%
\pgfsetbuttcap%
\pgfsetroundjoin%
\pgfsetlinewidth{0.250937pt}%
\definecolor{currentstroke}{rgb}{0.000000,0.000000,0.000000}%
\pgfsetstrokecolor{currentstroke}%
\pgfsetdash{}{0pt}%
\pgfpathmoveto{\pgfqpoint{0.625000in}{2.179085in}}%
\pgfpathlineto{\pgfqpoint{0.633495in}{2.171053in}}%
\pgfpathlineto{\pgfqpoint{0.625000in}{2.162790in}}%
\pgfusepath{stroke}%
\end{pgfscope}%
\begin{pgfscope}%
\pgfpathrectangle{\pgfqpoint{0.625000in}{0.550000in}}{\pgfqpoint{3.875000in}{3.850000in}} %
\pgfusepath{clip}%
\pgfsetbuttcap%
\pgfsetroundjoin%
\pgfsetlinewidth{0.250937pt}%
\definecolor{currentstroke}{rgb}{0.000000,0.000000,0.000000}%
\pgfsetstrokecolor{currentstroke}%
\pgfsetdash{}{0pt}%
\pgfpathmoveto{\pgfqpoint{0.625000in}{2.333431in}}%
\pgfpathlineto{\pgfqpoint{0.633533in}{2.325439in}}%
\pgfpathlineto{\pgfqpoint{0.625000in}{2.316886in}}%
\pgfusepath{stroke}%
\end{pgfscope}%
\begin{pgfscope}%
\pgfpathrectangle{\pgfqpoint{0.625000in}{0.550000in}}{\pgfqpoint{3.875000in}{3.850000in}} %
\pgfusepath{clip}%
\pgfsetbuttcap%
\pgfsetroundjoin%
\pgfsetlinewidth{0.250937pt}%
\definecolor{currentstroke}{rgb}{0.000000,0.000000,0.000000}%
\pgfsetstrokecolor{currentstroke}%
\pgfsetdash{}{0pt}%
\pgfpathmoveto{\pgfqpoint{0.625000in}{2.488803in}}%
\pgfpathlineto{\pgfqpoint{0.625629in}{2.489474in}}%
\pgfpathlineto{\pgfqpoint{0.627951in}{2.508772in}}%
\pgfpathlineto{\pgfqpoint{0.632262in}{2.528070in}}%
\pgfpathlineto{\pgfqpoint{0.635232in}{2.537719in}}%
\pgfpathlineto{\pgfqpoint{0.644424in}{2.560714in}}%
\pgfpathlineto{\pgfqpoint{0.663847in}{2.593335in}}%
\pgfpathlineto{\pgfqpoint{0.692982in}{2.626434in}}%
\pgfpathlineto{\pgfqpoint{0.714227in}{2.643860in}}%
\pgfpathlineto{\pgfqpoint{0.731830in}{2.655767in}}%
\pgfpathlineto{\pgfqpoint{0.751253in}{2.666245in}}%
\pgfpathlineto{\pgfqpoint{0.770677in}{2.674460in}}%
\pgfpathlineto{\pgfqpoint{0.790100in}{2.680746in}}%
\pgfpathlineto{\pgfqpoint{0.819236in}{2.686840in}}%
\pgfpathlineto{\pgfqpoint{0.838659in}{2.688766in}}%
\pgfpathlineto{\pgfqpoint{0.858083in}{2.689054in}}%
\pgfpathlineto{\pgfqpoint{0.877506in}{2.687704in}}%
\pgfpathlineto{\pgfqpoint{0.896930in}{2.684653in}}%
\pgfpathlineto{\pgfqpoint{0.916353in}{2.679842in}}%
\pgfpathlineto{\pgfqpoint{0.936524in}{2.672807in}}%
\pgfpathlineto{\pgfqpoint{0.957321in}{2.663158in}}%
\pgfpathlineto{\pgfqpoint{0.974624in}{2.652956in}}%
\pgfpathlineto{\pgfqpoint{0.994048in}{2.638639in}}%
\pgfpathlineto{\pgfqpoint{1.009545in}{2.624561in}}%
\pgfpathlineto{\pgfqpoint{1.026517in}{2.605263in}}%
\pgfpathlineto{\pgfqpoint{1.039831in}{2.585965in}}%
\pgfpathlineto{\pgfqpoint{1.050231in}{2.566667in}}%
\pgfpathlineto{\pgfqpoint{1.058188in}{2.547368in}}%
\pgfpathlineto{\pgfqpoint{1.063928in}{2.528070in}}%
\pgfpathlineto{\pgfqpoint{1.067701in}{2.508772in}}%
\pgfpathlineto{\pgfqpoint{1.069538in}{2.489474in}}%
\pgfpathlineto{\pgfqpoint{1.069538in}{2.470175in}}%
\pgfpathlineto{\pgfqpoint{1.067701in}{2.450877in}}%
\pgfpathlineto{\pgfqpoint{1.063928in}{2.431579in}}%
\pgfpathlineto{\pgfqpoint{1.058188in}{2.412281in}}%
\pgfpathlineto{\pgfqpoint{1.050231in}{2.392982in}}%
\pgfpathlineto{\pgfqpoint{1.039831in}{2.373684in}}%
\pgfpathlineto{\pgfqpoint{1.026517in}{2.354386in}}%
\pgfpathlineto{\pgfqpoint{1.009545in}{2.335088in}}%
\pgfpathlineto{\pgfqpoint{0.994048in}{2.321010in}}%
\pgfpathlineto{\pgfqpoint{0.973783in}{2.306140in}}%
\pgfpathlineto{\pgfqpoint{0.955201in}{2.295385in}}%
\pgfpathlineto{\pgfqpoint{0.935777in}{2.286539in}}%
\pgfpathlineto{\pgfqpoint{0.916353in}{2.279808in}}%
\pgfpathlineto{\pgfqpoint{0.896930in}{2.274996in}}%
\pgfpathlineto{\pgfqpoint{0.877506in}{2.271945in}}%
\pgfpathlineto{\pgfqpoint{0.858083in}{2.270595in}}%
\pgfpathlineto{\pgfqpoint{0.838659in}{2.270883in}}%
\pgfpathlineto{\pgfqpoint{0.819236in}{2.272809in}}%
\pgfpathlineto{\pgfqpoint{0.797143in}{2.277193in}}%
\pgfpathlineto{\pgfqpoint{0.780388in}{2.281772in}}%
\pgfpathlineto{\pgfqpoint{0.760965in}{2.289038in}}%
\pgfpathlineto{\pgfqpoint{0.741541in}{2.298364in}}%
\pgfpathlineto{\pgfqpoint{0.722118in}{2.310014in}}%
\pgfpathlineto{\pgfqpoint{0.701916in}{2.325439in}}%
\pgfpathlineto{\pgfqpoint{0.683271in}{2.342867in}}%
\pgfpathlineto{\pgfqpoint{0.659124in}{2.373684in}}%
\pgfpathlineto{\pgfqpoint{0.652999in}{2.383333in}}%
\pgfpathlineto{\pgfqpoint{0.642965in}{2.402632in}}%
\pgfpathlineto{\pgfqpoint{0.634712in}{2.423027in}}%
\pgfpathlineto{\pgfqpoint{0.629879in}{2.441228in}}%
\pgfpathlineto{\pgfqpoint{0.626362in}{2.460526in}}%
\pgfpathlineto{\pgfqpoint{0.625000in}{2.470846in}}%
\pgfpathlineto{\pgfqpoint{0.625000in}{2.470846in}}%
\pgfusepath{stroke}%
\end{pgfscope}%
\begin{pgfscope}%
\pgfpathrectangle{\pgfqpoint{0.625000in}{0.550000in}}{\pgfqpoint{3.875000in}{3.850000in}} %
\pgfusepath{clip}%
\pgfsetbuttcap%
\pgfsetroundjoin%
\pgfsetlinewidth{0.250937pt}%
\definecolor{currentstroke}{rgb}{0.000000,0.000000,0.000000}%
\pgfsetstrokecolor{currentstroke}%
\pgfsetdash{}{0pt}%
\pgfpathmoveto{\pgfqpoint{0.625000in}{2.642761in}}%
\pgfpathlineto{\pgfqpoint{0.633531in}{2.634211in}}%
\pgfpathlineto{\pgfqpoint{0.625000in}{2.626221in}}%
\pgfusepath{stroke}%
\end{pgfscope}%
\begin{pgfscope}%
\pgfpathrectangle{\pgfqpoint{0.625000in}{0.550000in}}{\pgfqpoint{3.875000in}{3.850000in}} %
\pgfusepath{clip}%
\pgfsetbuttcap%
\pgfsetroundjoin%
\pgfsetlinewidth{0.250937pt}%
\definecolor{currentstroke}{rgb}{0.000000,0.000000,0.000000}%
\pgfsetstrokecolor{currentstroke}%
\pgfsetdash{}{0pt}%
\pgfpathmoveto{\pgfqpoint{0.625000in}{2.796851in}}%
\pgfpathlineto{\pgfqpoint{0.633487in}{2.788596in}}%
\pgfpathlineto{\pgfqpoint{0.625000in}{2.780573in}}%
\pgfusepath{stroke}%
\end{pgfscope}%
\begin{pgfscope}%
\pgfpathrectangle{\pgfqpoint{0.625000in}{0.550000in}}{\pgfqpoint{3.875000in}{3.850000in}} %
\pgfusepath{clip}%
\pgfsetbuttcap%
\pgfsetroundjoin%
\pgfsetlinewidth{0.250937pt}%
\definecolor{currentstroke}{rgb}{0.000000,0.000000,0.000000}%
\pgfsetstrokecolor{currentstroke}%
\pgfsetdash{}{0pt}%
\pgfpathmoveto{\pgfqpoint{0.625000in}{2.951353in}}%
\pgfpathlineto{\pgfqpoint{0.633448in}{2.942982in}}%
\pgfpathlineto{\pgfqpoint{0.625000in}{2.934542in}}%
\pgfusepath{stroke}%
\end{pgfscope}%
\begin{pgfscope}%
\pgfpathrectangle{\pgfqpoint{0.625000in}{0.550000in}}{\pgfqpoint{3.875000in}{3.850000in}} %
\pgfusepath{clip}%
\pgfsetbuttcap%
\pgfsetroundjoin%
\pgfsetlinewidth{0.250937pt}%
\definecolor{currentstroke}{rgb}{0.000000,0.000000,0.000000}%
\pgfsetstrokecolor{currentstroke}%
\pgfsetdash{}{0pt}%
\pgfpathmoveto{\pgfqpoint{0.625000in}{3.105349in}}%
\pgfpathlineto{\pgfqpoint{0.633464in}{3.097368in}}%
\pgfpathlineto{\pgfqpoint{0.625000in}{3.089217in}}%
\pgfusepath{stroke}%
\end{pgfscope}%
\begin{pgfscope}%
\pgfpathrectangle{\pgfqpoint{0.625000in}{0.550000in}}{\pgfqpoint{3.875000in}{3.850000in}} %
\pgfusepath{clip}%
\pgfsetbuttcap%
\pgfsetroundjoin%
\pgfsetlinewidth{0.250937pt}%
\definecolor{currentstroke}{rgb}{0.000000,0.000000,0.000000}%
\pgfsetstrokecolor{currentstroke}%
\pgfsetdash{}{0pt}%
\pgfpathmoveto{\pgfqpoint{0.625000in}{3.259836in}}%
\pgfpathlineto{\pgfqpoint{0.634712in}{3.257429in}}%
\pgfpathlineto{\pgfqpoint{0.640777in}{3.251754in}}%
\pgfpathlineto{\pgfqpoint{0.634712in}{3.246080in}}%
\pgfpathlineto{\pgfqpoint{0.625000in}{3.243726in}}%
\pgfusepath{stroke}%
\end{pgfscope}%
\begin{pgfscope}%
\pgfpathrectangle{\pgfqpoint{0.625000in}{0.550000in}}{\pgfqpoint{3.875000in}{3.850000in}} %
\pgfusepath{clip}%
\pgfsetbuttcap%
\pgfsetroundjoin%
\pgfsetlinewidth{0.250937pt}%
\definecolor{currentstroke}{rgb}{0.000000,0.000000,0.000000}%
\pgfsetstrokecolor{currentstroke}%
\pgfsetdash{}{0pt}%
\pgfpathmoveto{\pgfqpoint{0.625000in}{3.414463in}}%
\pgfpathlineto{\pgfqpoint{0.633463in}{3.406140in}}%
\pgfpathlineto{\pgfqpoint{0.625000in}{3.398088in}}%
\pgfusepath{stroke}%
\end{pgfscope}%
\begin{pgfscope}%
\pgfpathrectangle{\pgfqpoint{0.625000in}{0.550000in}}{\pgfqpoint{3.875000in}{3.850000in}} %
\pgfusepath{clip}%
\pgfsetbuttcap%
\pgfsetroundjoin%
\pgfsetlinewidth{0.250937pt}%
\definecolor{currentstroke}{rgb}{0.000000,0.000000,0.000000}%
\pgfsetstrokecolor{currentstroke}%
\pgfsetdash{}{0pt}%
\pgfpathmoveto{\pgfqpoint{0.625000in}{3.568601in}}%
\pgfpathlineto{\pgfqpoint{0.633498in}{3.560526in}}%
\pgfpathlineto{\pgfqpoint{0.625000in}{3.552456in}}%
\pgfusepath{stroke}%
\end{pgfscope}%
\begin{pgfscope}%
\pgfpathrectangle{\pgfqpoint{0.625000in}{0.550000in}}{\pgfqpoint{3.875000in}{3.850000in}} %
\pgfusepath{clip}%
\pgfsetbuttcap%
\pgfsetroundjoin%
\pgfsetlinewidth{0.250937pt}%
\definecolor{currentstroke}{rgb}{0.000000,0.000000,0.000000}%
\pgfsetstrokecolor{currentstroke}%
\pgfsetdash{}{0pt}%
\pgfpathmoveto{\pgfqpoint{0.625000in}{3.723069in}}%
\pgfpathlineto{\pgfqpoint{0.633517in}{3.714912in}}%
\pgfpathlineto{\pgfqpoint{0.625000in}{3.706510in}}%
\pgfusepath{stroke}%
\end{pgfscope}%
\begin{pgfscope}%
\pgfpathrectangle{\pgfqpoint{0.625000in}{0.550000in}}{\pgfqpoint{3.875000in}{3.850000in}} %
\pgfusepath{clip}%
\pgfsetbuttcap%
\pgfsetroundjoin%
\pgfsetlinewidth{0.250937pt}%
\definecolor{currentstroke}{rgb}{0.000000,0.000000,0.000000}%
\pgfsetstrokecolor{currentstroke}%
\pgfsetdash{}{0pt}%
\pgfpathmoveto{\pgfqpoint{0.625000in}{3.877337in}}%
\pgfpathlineto{\pgfqpoint{0.633448in}{3.869298in}}%
\pgfpathlineto{\pgfqpoint{0.625000in}{3.861226in}}%
\pgfusepath{stroke}%
\end{pgfscope}%
\begin{pgfscope}%
\pgfpathrectangle{\pgfqpoint{0.625000in}{0.550000in}}{\pgfqpoint{3.875000in}{3.850000in}} %
\pgfusepath{clip}%
\pgfsetbuttcap%
\pgfsetroundjoin%
\pgfsetlinewidth{0.250937pt}%
\definecolor{currentstroke}{rgb}{0.000000,0.000000,0.000000}%
\pgfsetstrokecolor{currentstroke}%
\pgfsetdash{}{0pt}%
\pgfpathmoveto{\pgfqpoint{0.625000in}{4.031836in}}%
\pgfpathlineto{\pgfqpoint{0.634712in}{4.029359in}}%
\pgfpathlineto{\pgfqpoint{0.640777in}{4.023684in}}%
\pgfpathlineto{\pgfqpoint{0.634712in}{4.018010in}}%
\pgfpathlineto{\pgfqpoint{0.625000in}{4.015612in}}%
\pgfusepath{stroke}%
\end{pgfscope}%
\begin{pgfscope}%
\pgfpathrectangle{\pgfqpoint{0.625000in}{0.550000in}}{\pgfqpoint{3.875000in}{3.850000in}} %
\pgfusepath{clip}%
\pgfsetbuttcap%
\pgfsetroundjoin%
\pgfsetlinewidth{0.250937pt}%
\definecolor{currentstroke}{rgb}{0.000000,0.000000,0.000000}%
\pgfsetstrokecolor{currentstroke}%
\pgfsetdash{}{0pt}%
\pgfpathmoveto{\pgfqpoint{0.625000in}{4.186202in}}%
\pgfpathlineto{\pgfqpoint{0.633447in}{4.178070in}}%
\pgfpathlineto{\pgfqpoint{0.625000in}{4.170319in}}%
\pgfusepath{stroke}%
\end{pgfscope}%
\begin{pgfscope}%
\pgfpathrectangle{\pgfqpoint{0.625000in}{0.550000in}}{\pgfqpoint{3.875000in}{3.850000in}} %
\pgfusepath{clip}%
\pgfsetbuttcap%
\pgfsetroundjoin%
\pgfsetlinewidth{0.250937pt}%
\definecolor{currentstroke}{rgb}{0.000000,0.000000,0.000000}%
\pgfsetstrokecolor{currentstroke}%
\pgfsetdash{}{0pt}%
\pgfpathmoveto{\pgfqpoint{0.625000in}{4.340545in}}%
\pgfpathlineto{\pgfqpoint{0.633403in}{4.332456in}}%
\pgfpathlineto{\pgfqpoint{0.625000in}{4.324340in}}%
\pgfusepath{stroke}%
\end{pgfscope}%
\begin{pgfscope}%
\pgfpathrectangle{\pgfqpoint{0.625000in}{0.550000in}}{\pgfqpoint{3.875000in}{3.850000in}} %
\pgfusepath{clip}%
\pgfsetbuttcap%
\pgfsetroundjoin%
\pgfsetlinewidth{0.250937pt}%
\definecolor{currentstroke}{rgb}{0.000000,0.000000,0.000000}%
\pgfsetstrokecolor{currentstroke}%
\pgfsetdash{}{0pt}%
\pgfpathmoveto{\pgfqpoint{0.634712in}{1.176518in}}%
\pgfpathlineto{\pgfqpoint{0.634366in}{1.177193in}}%
\pgfpathlineto{\pgfqpoint{0.628576in}{1.186842in}}%
\pgfpathlineto{\pgfqpoint{0.627071in}{1.196491in}}%
\pgfpathlineto{\pgfqpoint{0.632061in}{1.206140in}}%
\pgfpathlineto{\pgfqpoint{0.634712in}{1.211147in}}%
\pgfpathlineto{\pgfqpoint{0.644424in}{1.213464in}}%
\pgfpathlineto{\pgfqpoint{0.654135in}{1.211626in}}%
\pgfpathlineto{\pgfqpoint{0.662082in}{1.206140in}}%
\pgfpathlineto{\pgfqpoint{0.663847in}{1.202436in}}%
\pgfpathlineto{\pgfqpoint{0.666904in}{1.196491in}}%
\pgfpathlineto{\pgfqpoint{0.665942in}{1.186842in}}%
\pgfpathlineto{\pgfqpoint{0.663847in}{1.183868in}}%
\pgfpathlineto{\pgfqpoint{0.658413in}{1.177193in}}%
\pgfpathlineto{\pgfqpoint{0.654135in}{1.174868in}}%
\pgfpathlineto{\pgfqpoint{0.644424in}{1.172916in}}%
\pgfpathlineto{\pgfqpoint{0.634712in}{1.176518in}}%
\pgfusepath{stroke}%
\end{pgfscope}%
\begin{pgfscope}%
\pgfpathrectangle{\pgfqpoint{0.625000in}{0.550000in}}{\pgfqpoint{3.875000in}{3.850000in}} %
\pgfusepath{clip}%
\pgfsetbuttcap%
\pgfsetroundjoin%
\pgfsetlinewidth{0.250937pt}%
\definecolor{currentstroke}{rgb}{0.000000,0.000000,0.000000}%
\pgfsetstrokecolor{currentstroke}%
\pgfsetdash{}{0pt}%
\pgfpathmoveto{\pgfqpoint{0.634712in}{3.748503in}}%
\pgfpathlineto{\pgfqpoint{0.632061in}{3.753509in}}%
\pgfpathlineto{\pgfqpoint{0.627071in}{3.763158in}}%
\pgfpathlineto{\pgfqpoint{0.628576in}{3.772807in}}%
\pgfpathlineto{\pgfqpoint{0.634366in}{3.782456in}}%
\pgfpathlineto{\pgfqpoint{0.634712in}{3.783131in}}%
\pgfpathlineto{\pgfqpoint{0.644424in}{3.786733in}}%
\pgfpathlineto{\pgfqpoint{0.654135in}{3.784781in}}%
\pgfpathlineto{\pgfqpoint{0.658413in}{3.782456in}}%
\pgfpathlineto{\pgfqpoint{0.663847in}{3.775781in}}%
\pgfpathlineto{\pgfqpoint{0.665942in}{3.772807in}}%
\pgfpathlineto{\pgfqpoint{0.666904in}{3.763158in}}%
\pgfpathlineto{\pgfqpoint{0.663847in}{3.757213in}}%
\pgfpathlineto{\pgfqpoint{0.662082in}{3.753509in}}%
\pgfpathlineto{\pgfqpoint{0.654135in}{3.748023in}}%
\pgfpathlineto{\pgfqpoint{0.644424in}{3.746185in}}%
\pgfpathlineto{\pgfqpoint{0.634712in}{3.748503in}}%
\pgfusepath{stroke}%
\end{pgfscope}%
\begin{pgfscope}%
\pgfpathrectangle{\pgfqpoint{0.625000in}{0.550000in}}{\pgfqpoint{3.875000in}{3.850000in}} %
\pgfusepath{clip}%
\pgfsetbuttcap%
\pgfsetroundjoin%
\pgfsetlinewidth{0.250937pt}%
\definecolor{currentstroke}{rgb}{0.000000,0.000000,0.000000}%
\pgfsetstrokecolor{currentstroke}%
\pgfsetdash{}{0pt}%
\pgfpathmoveto{\pgfqpoint{0.625000in}{0.635588in}}%
\pgfpathlineto{\pgfqpoint{0.633658in}{0.627193in}}%
\pgfpathlineto{\pgfqpoint{0.625000in}{0.618823in}}%
\pgfusepath{stroke}%
\end{pgfscope}%
\begin{pgfscope}%
\pgfpathrectangle{\pgfqpoint{0.625000in}{0.550000in}}{\pgfqpoint{3.875000in}{3.850000in}} %
\pgfusepath{clip}%
\pgfsetbuttcap%
\pgfsetroundjoin%
\pgfsetlinewidth{0.250937pt}%
\definecolor{currentstroke}{rgb}{0.000000,0.000000,0.000000}%
\pgfsetstrokecolor{currentstroke}%
\pgfsetdash{}{0pt}%
\pgfpathmoveto{\pgfqpoint{0.625000in}{0.789576in}}%
\pgfpathlineto{\pgfqpoint{0.633652in}{0.781579in}}%
\pgfpathlineto{\pgfqpoint{0.625000in}{0.773225in}}%
\pgfusepath{stroke}%
\end{pgfscope}%
\begin{pgfscope}%
\pgfpathrectangle{\pgfqpoint{0.625000in}{0.550000in}}{\pgfqpoint{3.875000in}{3.850000in}} %
\pgfusepath{clip}%
\pgfsetbuttcap%
\pgfsetroundjoin%
\pgfsetlinewidth{0.250937pt}%
\definecolor{currentstroke}{rgb}{0.000000,0.000000,0.000000}%
\pgfsetstrokecolor{currentstroke}%
\pgfsetdash{}{0pt}%
\pgfpathmoveto{\pgfqpoint{0.625000in}{0.944247in}}%
\pgfpathlineto{\pgfqpoint{0.634712in}{0.942452in}}%
\pgfpathlineto{\pgfqpoint{0.641646in}{0.935965in}}%
\pgfpathlineto{\pgfqpoint{0.634712in}{0.929478in}}%
\pgfpathlineto{\pgfqpoint{0.625000in}{0.927608in}}%
\pgfusepath{stroke}%
\end{pgfscope}%
\begin{pgfscope}%
\pgfpathrectangle{\pgfqpoint{0.625000in}{0.550000in}}{\pgfqpoint{3.875000in}{3.850000in}} %
\pgfusepath{clip}%
\pgfsetbuttcap%
\pgfsetroundjoin%
\pgfsetlinewidth{0.250937pt}%
\definecolor{currentstroke}{rgb}{0.000000,0.000000,0.000000}%
\pgfsetstrokecolor{currentstroke}%
\pgfsetdash{}{0pt}%
\pgfpathmoveto{\pgfqpoint{0.625000in}{1.098647in}}%
\pgfpathlineto{\pgfqpoint{0.633651in}{1.090351in}}%
\pgfpathlineto{\pgfqpoint{0.625000in}{1.082086in}}%
\pgfusepath{stroke}%
\end{pgfscope}%
\begin{pgfscope}%
\pgfpathrectangle{\pgfqpoint{0.625000in}{0.550000in}}{\pgfqpoint{3.875000in}{3.850000in}} %
\pgfusepath{clip}%
\pgfsetbuttcap%
\pgfsetroundjoin%
\pgfsetlinewidth{0.250937pt}%
\definecolor{currentstroke}{rgb}{0.000000,0.000000,0.000000}%
\pgfsetstrokecolor{currentstroke}%
\pgfsetdash{}{0pt}%
\pgfpathmoveto{\pgfqpoint{0.625000in}{1.253296in}}%
\pgfpathlineto{\pgfqpoint{0.633672in}{1.244737in}}%
\pgfpathlineto{\pgfqpoint{0.625000in}{1.236414in}}%
\pgfusepath{stroke}%
\end{pgfscope}%
\begin{pgfscope}%
\pgfpathrectangle{\pgfqpoint{0.625000in}{0.550000in}}{\pgfqpoint{3.875000in}{3.850000in}} %
\pgfusepath{clip}%
\pgfsetbuttcap%
\pgfsetroundjoin%
\pgfsetlinewidth{0.250937pt}%
\definecolor{currentstroke}{rgb}{0.000000,0.000000,0.000000}%
\pgfsetstrokecolor{currentstroke}%
\pgfsetdash{}{0pt}%
\pgfpathmoveto{\pgfqpoint{0.625000in}{1.407334in}}%
\pgfpathlineto{\pgfqpoint{0.633632in}{1.399123in}}%
\pgfpathlineto{\pgfqpoint{0.625000in}{1.390907in}}%
\pgfusepath{stroke}%
\end{pgfscope}%
\begin{pgfscope}%
\pgfpathrectangle{\pgfqpoint{0.625000in}{0.550000in}}{\pgfqpoint{3.875000in}{3.850000in}} %
\pgfusepath{clip}%
\pgfsetbuttcap%
\pgfsetroundjoin%
\pgfsetlinewidth{0.250937pt}%
\definecolor{currentstroke}{rgb}{0.000000,0.000000,0.000000}%
\pgfsetstrokecolor{currentstroke}%
\pgfsetdash{}{0pt}%
\pgfpathmoveto{\pgfqpoint{0.625000in}{1.561692in}}%
\pgfpathlineto{\pgfqpoint{0.633589in}{1.553509in}}%
\pgfpathlineto{\pgfqpoint{0.625000in}{1.545060in}}%
\pgfusepath{stroke}%
\end{pgfscope}%
\begin{pgfscope}%
\pgfpathrectangle{\pgfqpoint{0.625000in}{0.550000in}}{\pgfqpoint{3.875000in}{3.850000in}} %
\pgfusepath{clip}%
\pgfsetbuttcap%
\pgfsetroundjoin%
\pgfsetlinewidth{0.250937pt}%
\definecolor{currentstroke}{rgb}{0.000000,0.000000,0.000000}%
\pgfsetstrokecolor{currentstroke}%
\pgfsetdash{}{0pt}%
\pgfpathmoveto{\pgfqpoint{0.625000in}{1.716078in}}%
\pgfpathlineto{\pgfqpoint{0.634712in}{1.714382in}}%
\pgfpathlineto{\pgfqpoint{0.641646in}{1.707895in}}%
\pgfpathlineto{\pgfqpoint{0.634712in}{1.701408in}}%
\pgfpathlineto{\pgfqpoint{0.625000in}{1.699660in}}%
\pgfusepath{stroke}%
\end{pgfscope}%
\begin{pgfscope}%
\pgfpathrectangle{\pgfqpoint{0.625000in}{0.550000in}}{\pgfqpoint{3.875000in}{3.850000in}} %
\pgfusepath{clip}%
\pgfsetbuttcap%
\pgfsetroundjoin%
\pgfsetlinewidth{0.250937pt}%
\definecolor{currentstroke}{rgb}{0.000000,0.000000,0.000000}%
\pgfsetstrokecolor{currentstroke}%
\pgfsetdash{}{0pt}%
\pgfpathmoveto{\pgfqpoint{0.625000in}{1.870535in}}%
\pgfpathlineto{\pgfqpoint{0.633567in}{1.862281in}}%
\pgfpathlineto{\pgfqpoint{0.625000in}{1.854197in}}%
\pgfusepath{stroke}%
\end{pgfscope}%
\begin{pgfscope}%
\pgfpathrectangle{\pgfqpoint{0.625000in}{0.550000in}}{\pgfqpoint{3.875000in}{3.850000in}} %
\pgfusepath{clip}%
\pgfsetbuttcap%
\pgfsetroundjoin%
\pgfsetlinewidth{0.250937pt}%
\definecolor{currentstroke}{rgb}{0.000000,0.000000,0.000000}%
\pgfsetstrokecolor{currentstroke}%
\pgfsetdash{}{0pt}%
\pgfpathmoveto{\pgfqpoint{0.625000in}{2.025215in}}%
\pgfpathlineto{\pgfqpoint{0.633557in}{2.016667in}}%
\pgfpathlineto{\pgfqpoint{0.625000in}{2.008188in}}%
\pgfusepath{stroke}%
\end{pgfscope}%
\begin{pgfscope}%
\pgfpathrectangle{\pgfqpoint{0.625000in}{0.550000in}}{\pgfqpoint{3.875000in}{3.850000in}} %
\pgfusepath{clip}%
\pgfsetbuttcap%
\pgfsetroundjoin%
\pgfsetlinewidth{0.250937pt}%
\definecolor{currentstroke}{rgb}{0.000000,0.000000,0.000000}%
\pgfsetstrokecolor{currentstroke}%
\pgfsetdash{}{0pt}%
\pgfpathmoveto{\pgfqpoint{0.625000in}{2.179167in}}%
\pgfpathlineto{\pgfqpoint{0.633582in}{2.171053in}}%
\pgfpathlineto{\pgfqpoint{0.625000in}{2.162705in}}%
\pgfusepath{stroke}%
\end{pgfscope}%
\begin{pgfscope}%
\pgfpathrectangle{\pgfqpoint{0.625000in}{0.550000in}}{\pgfqpoint{3.875000in}{3.850000in}} %
\pgfusepath{clip}%
\pgfsetbuttcap%
\pgfsetroundjoin%
\pgfsetlinewidth{0.250937pt}%
\definecolor{currentstroke}{rgb}{0.000000,0.000000,0.000000}%
\pgfsetstrokecolor{currentstroke}%
\pgfsetdash{}{0pt}%
\pgfpathmoveto{\pgfqpoint{0.625000in}{2.333514in}}%
\pgfpathlineto{\pgfqpoint{0.633621in}{2.325439in}}%
\pgfpathlineto{\pgfqpoint{0.625000in}{2.316798in}}%
\pgfusepath{stroke}%
\end{pgfscope}%
\begin{pgfscope}%
\pgfpathrectangle{\pgfqpoint{0.625000in}{0.550000in}}{\pgfqpoint{3.875000in}{3.850000in}} %
\pgfusepath{clip}%
\pgfsetbuttcap%
\pgfsetroundjoin%
\pgfsetlinewidth{0.250937pt}%
\definecolor{currentstroke}{rgb}{0.000000,0.000000,0.000000}%
\pgfsetstrokecolor{currentstroke}%
\pgfsetdash{}{0pt}%
\pgfpathmoveto{\pgfqpoint{0.625000in}{2.488887in}}%
\pgfpathlineto{\pgfqpoint{0.625551in}{2.489474in}}%
\pgfpathlineto{\pgfqpoint{0.627797in}{2.508772in}}%
\pgfpathlineto{\pgfqpoint{0.631857in}{2.528070in}}%
\pgfpathlineto{\pgfqpoint{0.638085in}{2.547368in}}%
\pgfpathlineto{\pgfqpoint{0.646208in}{2.566667in}}%
\pgfpathlineto{\pgfqpoint{0.656632in}{2.585965in}}%
\pgfpathlineto{\pgfqpoint{0.673559in}{2.610967in}}%
\pgfpathlineto{\pgfqpoint{0.702694in}{2.642006in}}%
\pgfpathlineto{\pgfqpoint{0.731830in}{2.664592in}}%
\pgfpathlineto{\pgfqpoint{0.751253in}{2.676363in}}%
\pgfpathlineto{\pgfqpoint{0.770677in}{2.685893in}}%
\pgfpathlineto{\pgfqpoint{0.790100in}{2.693466in}}%
\pgfpathlineto{\pgfqpoint{0.809524in}{2.699359in}}%
\pgfpathlineto{\pgfqpoint{0.838659in}{2.705152in}}%
\pgfpathlineto{\pgfqpoint{0.858083in}{2.707095in}}%
\pgfpathlineto{\pgfqpoint{0.877506in}{2.707538in}}%
\pgfpathlineto{\pgfqpoint{0.896930in}{2.706487in}}%
\pgfpathlineto{\pgfqpoint{0.916353in}{2.703901in}}%
\pgfpathlineto{\pgfqpoint{0.935777in}{2.699738in}}%
\pgfpathlineto{\pgfqpoint{0.960124in}{2.692105in}}%
\pgfpathlineto{\pgfqpoint{0.982629in}{2.682456in}}%
\pgfpathlineto{\pgfqpoint{1.000380in}{2.672807in}}%
\pgfpathlineto{\pgfqpoint{1.015185in}{2.663158in}}%
\pgfpathlineto{\pgfqpoint{1.032895in}{2.649297in}}%
\pgfpathlineto{\pgfqpoint{1.048886in}{2.634211in}}%
\pgfpathlineto{\pgfqpoint{1.065467in}{2.614912in}}%
\pgfpathlineto{\pgfqpoint{1.081454in}{2.590925in}}%
\pgfpathlineto{\pgfqpoint{1.091165in}{2.572239in}}%
\pgfpathlineto{\pgfqpoint{1.097609in}{2.557018in}}%
\pgfpathlineto{\pgfqpoint{1.103839in}{2.537719in}}%
\pgfpathlineto{\pgfqpoint{1.108184in}{2.518421in}}%
\pgfpathlineto{\pgfqpoint{1.110717in}{2.499123in}}%
\pgfpathlineto{\pgfqpoint{1.111584in}{2.479825in}}%
\pgfpathlineto{\pgfqpoint{1.110589in}{2.459387in}}%
\pgfpathlineto{\pgfqpoint{1.108184in}{2.441228in}}%
\pgfpathlineto{\pgfqpoint{1.103839in}{2.421930in}}%
\pgfpathlineto{\pgfqpoint{1.097609in}{2.402632in}}%
\pgfpathlineto{\pgfqpoint{1.089312in}{2.383333in}}%
\pgfpathlineto{\pgfqpoint{1.078736in}{2.364035in}}%
\pgfpathlineto{\pgfqpoint{1.065467in}{2.344737in}}%
\pgfpathlineto{\pgfqpoint{1.048886in}{2.325439in}}%
\pgfpathlineto{\pgfqpoint{1.027922in}{2.306140in}}%
\pgfpathlineto{\pgfqpoint{1.013471in}{2.295306in}}%
\pgfpathlineto{\pgfqpoint{0.994048in}{2.283183in}}%
\pgfpathlineto{\pgfqpoint{0.974624in}{2.273440in}}%
\pgfpathlineto{\pgfqpoint{0.955201in}{2.265759in}}%
\pgfpathlineto{\pgfqpoint{0.935777in}{2.259911in}}%
\pgfpathlineto{\pgfqpoint{0.916353in}{2.255748in}}%
\pgfpathlineto{\pgfqpoint{0.896930in}{2.253162in}}%
\pgfpathlineto{\pgfqpoint{0.877506in}{2.252112in}}%
\pgfpathlineto{\pgfqpoint{0.858083in}{2.252554in}}%
\pgfpathlineto{\pgfqpoint{0.838659in}{2.254497in}}%
\pgfpathlineto{\pgfqpoint{0.809524in}{2.260290in}}%
\pgfpathlineto{\pgfqpoint{0.786616in}{2.267544in}}%
\pgfpathlineto{\pgfqpoint{0.770677in}{2.273756in}}%
\pgfpathlineto{\pgfqpoint{0.741541in}{2.288880in}}%
\pgfpathlineto{\pgfqpoint{0.722118in}{2.301783in}}%
\pgfpathlineto{\pgfqpoint{0.694644in}{2.325439in}}%
\pgfpathlineto{\pgfqpoint{0.692982in}{2.326885in}}%
\pgfpathlineto{\pgfqpoint{0.673559in}{2.348682in}}%
\pgfpathlineto{\pgfqpoint{0.654135in}{2.377378in}}%
\pgfpathlineto{\pgfqpoint{0.641869in}{2.402632in}}%
\pgfpathlineto{\pgfqpoint{0.634580in}{2.421930in}}%
\pgfpathlineto{\pgfqpoint{0.629616in}{2.441228in}}%
\pgfpathlineto{\pgfqpoint{0.626283in}{2.460526in}}%
\pgfpathlineto{\pgfqpoint{0.625000in}{2.470763in}}%
\pgfpathlineto{\pgfqpoint{0.625000in}{2.470763in}}%
\pgfusepath{stroke}%
\end{pgfscope}%
\begin{pgfscope}%
\pgfpathrectangle{\pgfqpoint{0.625000in}{0.550000in}}{\pgfqpoint{3.875000in}{3.850000in}} %
\pgfusepath{clip}%
\pgfsetbuttcap%
\pgfsetroundjoin%
\pgfsetlinewidth{0.250937pt}%
\definecolor{currentstroke}{rgb}{0.000000,0.000000,0.000000}%
\pgfsetstrokecolor{currentstroke}%
\pgfsetdash{}{0pt}%
\pgfpathmoveto{\pgfqpoint{0.625000in}{2.642850in}}%
\pgfpathlineto{\pgfqpoint{0.633619in}{2.634211in}}%
\pgfpathlineto{\pgfqpoint{0.625000in}{2.626138in}}%
\pgfusepath{stroke}%
\end{pgfscope}%
\begin{pgfscope}%
\pgfpathrectangle{\pgfqpoint{0.625000in}{0.550000in}}{\pgfqpoint{3.875000in}{3.850000in}} %
\pgfusepath{clip}%
\pgfsetbuttcap%
\pgfsetroundjoin%
\pgfsetlinewidth{0.250937pt}%
\definecolor{currentstroke}{rgb}{0.000000,0.000000,0.000000}%
\pgfsetstrokecolor{currentstroke}%
\pgfsetdash{}{0pt}%
\pgfpathmoveto{\pgfqpoint{0.625000in}{2.796936in}}%
\pgfpathlineto{\pgfqpoint{0.633575in}{2.788596in}}%
\pgfpathlineto{\pgfqpoint{0.625000in}{2.780491in}}%
\pgfusepath{stroke}%
\end{pgfscope}%
\begin{pgfscope}%
\pgfpathrectangle{\pgfqpoint{0.625000in}{0.550000in}}{\pgfqpoint{3.875000in}{3.850000in}} %
\pgfusepath{clip}%
\pgfsetbuttcap%
\pgfsetroundjoin%
\pgfsetlinewidth{0.250937pt}%
\definecolor{currentstroke}{rgb}{0.000000,0.000000,0.000000}%
\pgfsetstrokecolor{currentstroke}%
\pgfsetdash{}{0pt}%
\pgfpathmoveto{\pgfqpoint{0.625000in}{2.951441in}}%
\pgfpathlineto{\pgfqpoint{0.633537in}{2.942982in}}%
\pgfpathlineto{\pgfqpoint{0.625000in}{2.934453in}}%
\pgfusepath{stroke}%
\end{pgfscope}%
\begin{pgfscope}%
\pgfpathrectangle{\pgfqpoint{0.625000in}{0.550000in}}{\pgfqpoint{3.875000in}{3.850000in}} %
\pgfusepath{clip}%
\pgfsetbuttcap%
\pgfsetroundjoin%
\pgfsetlinewidth{0.250937pt}%
\definecolor{currentstroke}{rgb}{0.000000,0.000000,0.000000}%
\pgfsetstrokecolor{currentstroke}%
\pgfsetdash{}{0pt}%
\pgfpathmoveto{\pgfqpoint{0.625000in}{3.105432in}}%
\pgfpathlineto{\pgfqpoint{0.633551in}{3.097368in}}%
\pgfpathlineto{\pgfqpoint{0.625000in}{3.089133in}}%
\pgfusepath{stroke}%
\end{pgfscope}%
\begin{pgfscope}%
\pgfpathrectangle{\pgfqpoint{0.625000in}{0.550000in}}{\pgfqpoint{3.875000in}{3.850000in}} %
\pgfusepath{clip}%
\pgfsetbuttcap%
\pgfsetroundjoin%
\pgfsetlinewidth{0.250937pt}%
\definecolor{currentstroke}{rgb}{0.000000,0.000000,0.000000}%
\pgfsetstrokecolor{currentstroke}%
\pgfsetdash{}{0pt}%
\pgfpathmoveto{\pgfqpoint{0.625000in}{3.259921in}}%
\pgfpathlineto{\pgfqpoint{0.634712in}{3.258242in}}%
\pgfpathlineto{\pgfqpoint{0.641646in}{3.251754in}}%
\pgfpathlineto{\pgfqpoint{0.634712in}{3.245267in}}%
\pgfpathlineto{\pgfqpoint{0.625000in}{3.243641in}}%
\pgfusepath{stroke}%
\end{pgfscope}%
\begin{pgfscope}%
\pgfpathrectangle{\pgfqpoint{0.625000in}{0.550000in}}{\pgfqpoint{3.875000in}{3.850000in}} %
\pgfusepath{clip}%
\pgfsetbuttcap%
\pgfsetroundjoin%
\pgfsetlinewidth{0.250937pt}%
\definecolor{currentstroke}{rgb}{0.000000,0.000000,0.000000}%
\pgfsetstrokecolor{currentstroke}%
\pgfsetdash{}{0pt}%
\pgfpathmoveto{\pgfqpoint{0.625000in}{3.414548in}}%
\pgfpathlineto{\pgfqpoint{0.633550in}{3.406140in}}%
\pgfpathlineto{\pgfqpoint{0.625000in}{3.398006in}}%
\pgfusepath{stroke}%
\end{pgfscope}%
\begin{pgfscope}%
\pgfpathrectangle{\pgfqpoint{0.625000in}{0.550000in}}{\pgfqpoint{3.875000in}{3.850000in}} %
\pgfusepath{clip}%
\pgfsetbuttcap%
\pgfsetroundjoin%
\pgfsetlinewidth{0.250937pt}%
\definecolor{currentstroke}{rgb}{0.000000,0.000000,0.000000}%
\pgfsetstrokecolor{currentstroke}%
\pgfsetdash{}{0pt}%
\pgfpathmoveto{\pgfqpoint{0.625000in}{3.568686in}}%
\pgfpathlineto{\pgfqpoint{0.633587in}{3.560526in}}%
\pgfpathlineto{\pgfqpoint{0.625000in}{3.552372in}}%
\pgfusepath{stroke}%
\end{pgfscope}%
\begin{pgfscope}%
\pgfpathrectangle{\pgfqpoint{0.625000in}{0.550000in}}{\pgfqpoint{3.875000in}{3.850000in}} %
\pgfusepath{clip}%
\pgfsetbuttcap%
\pgfsetroundjoin%
\pgfsetlinewidth{0.250937pt}%
\definecolor{currentstroke}{rgb}{0.000000,0.000000,0.000000}%
\pgfsetstrokecolor{currentstroke}%
\pgfsetdash{}{0pt}%
\pgfpathmoveto{\pgfqpoint{0.625000in}{3.723154in}}%
\pgfpathlineto{\pgfqpoint{0.633606in}{3.714912in}}%
\pgfpathlineto{\pgfqpoint{0.625000in}{3.706422in}}%
\pgfusepath{stroke}%
\end{pgfscope}%
\begin{pgfscope}%
\pgfpathrectangle{\pgfqpoint{0.625000in}{0.550000in}}{\pgfqpoint{3.875000in}{3.850000in}} %
\pgfusepath{clip}%
\pgfsetbuttcap%
\pgfsetroundjoin%
\pgfsetlinewidth{0.250937pt}%
\definecolor{currentstroke}{rgb}{0.000000,0.000000,0.000000}%
\pgfsetstrokecolor{currentstroke}%
\pgfsetdash{}{0pt}%
\pgfpathmoveto{\pgfqpoint{0.625000in}{3.877424in}}%
\pgfpathlineto{\pgfqpoint{0.633539in}{3.869298in}}%
\pgfpathlineto{\pgfqpoint{0.625000in}{3.861139in}}%
\pgfusepath{stroke}%
\end{pgfscope}%
\begin{pgfscope}%
\pgfpathrectangle{\pgfqpoint{0.625000in}{0.550000in}}{\pgfqpoint{3.875000in}{3.850000in}} %
\pgfusepath{clip}%
\pgfsetbuttcap%
\pgfsetroundjoin%
\pgfsetlinewidth{0.250937pt}%
\definecolor{currentstroke}{rgb}{0.000000,0.000000,0.000000}%
\pgfsetstrokecolor{currentstroke}%
\pgfsetdash{}{0pt}%
\pgfpathmoveto{\pgfqpoint{0.625000in}{4.031920in}}%
\pgfpathlineto{\pgfqpoint{0.634712in}{4.030171in}}%
\pgfpathlineto{\pgfqpoint{0.641646in}{4.023684in}}%
\pgfpathlineto{\pgfqpoint{0.634712in}{4.017197in}}%
\pgfpathlineto{\pgfqpoint{0.625000in}{4.015529in}}%
\pgfusepath{stroke}%
\end{pgfscope}%
\begin{pgfscope}%
\pgfpathrectangle{\pgfqpoint{0.625000in}{0.550000in}}{\pgfqpoint{3.875000in}{3.850000in}} %
\pgfusepath{clip}%
\pgfsetbuttcap%
\pgfsetroundjoin%
\pgfsetlinewidth{0.250937pt}%
\definecolor{currentstroke}{rgb}{0.000000,0.000000,0.000000}%
\pgfsetstrokecolor{currentstroke}%
\pgfsetdash{}{0pt}%
\pgfpathmoveto{\pgfqpoint{0.625000in}{4.186290in}}%
\pgfpathlineto{\pgfqpoint{0.633539in}{4.178070in}}%
\pgfpathlineto{\pgfqpoint{0.625000in}{4.170235in}}%
\pgfusepath{stroke}%
\end{pgfscope}%
\begin{pgfscope}%
\pgfpathrectangle{\pgfqpoint{0.625000in}{0.550000in}}{\pgfqpoint{3.875000in}{3.850000in}} %
\pgfusepath{clip}%
\pgfsetbuttcap%
\pgfsetroundjoin%
\pgfsetlinewidth{0.250937pt}%
\definecolor{currentstroke}{rgb}{0.000000,0.000000,0.000000}%
\pgfsetstrokecolor{currentstroke}%
\pgfsetdash{}{0pt}%
\pgfpathmoveto{\pgfqpoint{0.625000in}{4.340633in}}%
\pgfpathlineto{\pgfqpoint{0.633493in}{4.332456in}}%
\pgfpathlineto{\pgfqpoint{0.625000in}{4.324252in}}%
\pgfusepath{stroke}%
\end{pgfscope}%
\begin{pgfscope}%
\pgfpathrectangle{\pgfqpoint{0.625000in}{0.550000in}}{\pgfqpoint{3.875000in}{3.850000in}} %
\pgfusepath{clip}%
\pgfsetbuttcap%
\pgfsetroundjoin%
\pgfsetlinewidth{0.250937pt}%
\definecolor{currentstroke}{rgb}{0.000000,0.000000,0.000000}%
\pgfsetstrokecolor{currentstroke}%
\pgfsetdash{}{0pt}%
\pgfpathmoveto{\pgfqpoint{0.634712in}{1.175459in}}%
\pgfpathlineto{\pgfqpoint{0.633822in}{1.177193in}}%
\pgfpathlineto{\pgfqpoint{0.628368in}{1.186842in}}%
\pgfpathlineto{\pgfqpoint{0.626896in}{1.196491in}}%
\pgfpathlineto{\pgfqpoint{0.631681in}{1.206140in}}%
\pgfpathlineto{\pgfqpoint{0.634712in}{1.211864in}}%
\pgfpathlineto{\pgfqpoint{0.644424in}{1.214549in}}%
\pgfpathlineto{\pgfqpoint{0.654135in}{1.213706in}}%
\pgfpathlineto{\pgfqpoint{0.663847in}{1.207739in}}%
\pgfpathlineto{\pgfqpoint{0.665624in}{1.206140in}}%
\pgfpathlineto{\pgfqpoint{0.670153in}{1.196491in}}%
\pgfpathlineto{\pgfqpoint{0.669378in}{1.186842in}}%
\pgfpathlineto{\pgfqpoint{0.663847in}{1.178988in}}%
\pgfpathlineto{\pgfqpoint{0.662386in}{1.177193in}}%
\pgfpathlineto{\pgfqpoint{0.654135in}{1.172709in}}%
\pgfpathlineto{\pgfqpoint{0.644424in}{1.171636in}}%
\pgfpathlineto{\pgfqpoint{0.634712in}{1.175459in}}%
\pgfusepath{stroke}%
\end{pgfscope}%
\begin{pgfscope}%
\pgfpathrectangle{\pgfqpoint{0.625000in}{0.550000in}}{\pgfqpoint{3.875000in}{3.850000in}} %
\pgfusepath{clip}%
\pgfsetbuttcap%
\pgfsetroundjoin%
\pgfsetlinewidth{0.250937pt}%
\definecolor{currentstroke}{rgb}{0.000000,0.000000,0.000000}%
\pgfsetstrokecolor{currentstroke}%
\pgfsetdash{}{0pt}%
\pgfpathmoveto{\pgfqpoint{0.634712in}{3.747786in}}%
\pgfpathlineto{\pgfqpoint{0.631681in}{3.753509in}}%
\pgfpathlineto{\pgfqpoint{0.626896in}{3.763158in}}%
\pgfpathlineto{\pgfqpoint{0.628368in}{3.772807in}}%
\pgfpathlineto{\pgfqpoint{0.633822in}{3.782456in}}%
\pgfpathlineto{\pgfqpoint{0.634712in}{3.784190in}}%
\pgfpathlineto{\pgfqpoint{0.644424in}{3.788013in}}%
\pgfpathlineto{\pgfqpoint{0.654135in}{3.786940in}}%
\pgfpathlineto{\pgfqpoint{0.662386in}{3.782456in}}%
\pgfpathlineto{\pgfqpoint{0.663847in}{3.780661in}}%
\pgfpathlineto{\pgfqpoint{0.669378in}{3.772807in}}%
\pgfpathlineto{\pgfqpoint{0.670153in}{3.763158in}}%
\pgfpathlineto{\pgfqpoint{0.665624in}{3.753509in}}%
\pgfpathlineto{\pgfqpoint{0.663847in}{3.751910in}}%
\pgfpathlineto{\pgfqpoint{0.654135in}{3.745943in}}%
\pgfpathlineto{\pgfqpoint{0.644424in}{3.745100in}}%
\pgfpathlineto{\pgfqpoint{0.634712in}{3.747786in}}%
\pgfusepath{stroke}%
\end{pgfscope}%
\begin{pgfscope}%
\pgfpathrectangle{\pgfqpoint{0.625000in}{0.550000in}}{\pgfqpoint{3.875000in}{3.850000in}} %
\pgfusepath{clip}%
\pgfsetbuttcap%
\pgfsetroundjoin%
\pgfsetlinewidth{0.250937pt}%
\definecolor{currentstroke}{rgb}{0.000000,0.000000,0.000000}%
\pgfsetstrokecolor{currentstroke}%
\pgfsetdash{}{0pt}%
\pgfpathmoveto{\pgfqpoint{0.625000in}{0.635664in}}%
\pgfpathlineto{\pgfqpoint{0.633737in}{0.627193in}}%
\pgfpathlineto{\pgfqpoint{0.625000in}{0.618747in}}%
\pgfusepath{stroke}%
\end{pgfscope}%
\begin{pgfscope}%
\pgfpathrectangle{\pgfqpoint{0.625000in}{0.550000in}}{\pgfqpoint{3.875000in}{3.850000in}} %
\pgfusepath{clip}%
\pgfsetbuttcap%
\pgfsetroundjoin%
\pgfsetlinewidth{0.250937pt}%
\definecolor{currentstroke}{rgb}{0.000000,0.000000,0.000000}%
\pgfsetstrokecolor{currentstroke}%
\pgfsetdash{}{0pt}%
\pgfpathmoveto{\pgfqpoint{0.625000in}{0.789653in}}%
\pgfpathlineto{\pgfqpoint{0.633735in}{0.781579in}}%
\pgfpathlineto{\pgfqpoint{0.625000in}{0.773145in}}%
\pgfusepath{stroke}%
\end{pgfscope}%
\begin{pgfscope}%
\pgfpathrectangle{\pgfqpoint{0.625000in}{0.550000in}}{\pgfqpoint{3.875000in}{3.850000in}} %
\pgfusepath{clip}%
\pgfsetbuttcap%
\pgfsetroundjoin%
\pgfsetlinewidth{0.250937pt}%
\definecolor{currentstroke}{rgb}{0.000000,0.000000,0.000000}%
\pgfsetstrokecolor{currentstroke}%
\pgfsetdash{}{0pt}%
\pgfpathmoveto{\pgfqpoint{0.625000in}{0.944324in}}%
\pgfpathlineto{\pgfqpoint{0.634712in}{0.943265in}}%
\pgfpathlineto{\pgfqpoint{0.642515in}{0.935965in}}%
\pgfpathlineto{\pgfqpoint{0.634712in}{0.928665in}}%
\pgfpathlineto{\pgfqpoint{0.625000in}{0.927531in}}%
\pgfusepath{stroke}%
\end{pgfscope}%
\begin{pgfscope}%
\pgfpathrectangle{\pgfqpoint{0.625000in}{0.550000in}}{\pgfqpoint{3.875000in}{3.850000in}} %
\pgfusepath{clip}%
\pgfsetbuttcap%
\pgfsetroundjoin%
\pgfsetlinewidth{0.250937pt}%
\definecolor{currentstroke}{rgb}{0.000000,0.000000,0.000000}%
\pgfsetstrokecolor{currentstroke}%
\pgfsetdash{}{0pt}%
\pgfpathmoveto{\pgfqpoint{0.625000in}{1.098726in}}%
\pgfpathlineto{\pgfqpoint{0.633733in}{1.090351in}}%
\pgfpathlineto{\pgfqpoint{0.625000in}{1.082007in}}%
\pgfusepath{stroke}%
\end{pgfscope}%
\begin{pgfscope}%
\pgfpathrectangle{\pgfqpoint{0.625000in}{0.550000in}}{\pgfqpoint{3.875000in}{3.850000in}} %
\pgfusepath{clip}%
\pgfsetbuttcap%
\pgfsetroundjoin%
\pgfsetlinewidth{0.250937pt}%
\definecolor{currentstroke}{rgb}{0.000000,0.000000,0.000000}%
\pgfsetstrokecolor{currentstroke}%
\pgfsetdash{}{0pt}%
\pgfpathmoveto{\pgfqpoint{0.625000in}{1.253378in}}%
\pgfpathlineto{\pgfqpoint{0.633756in}{1.244737in}}%
\pgfpathlineto{\pgfqpoint{0.625000in}{1.236334in}}%
\pgfusepath{stroke}%
\end{pgfscope}%
\begin{pgfscope}%
\pgfpathrectangle{\pgfqpoint{0.625000in}{0.550000in}}{\pgfqpoint{3.875000in}{3.850000in}} %
\pgfusepath{clip}%
\pgfsetbuttcap%
\pgfsetroundjoin%
\pgfsetlinewidth{0.250937pt}%
\definecolor{currentstroke}{rgb}{0.000000,0.000000,0.000000}%
\pgfsetstrokecolor{currentstroke}%
\pgfsetdash{}{0pt}%
\pgfpathmoveto{\pgfqpoint{0.625000in}{1.407416in}}%
\pgfpathlineto{\pgfqpoint{0.633718in}{1.399123in}}%
\pgfpathlineto{\pgfqpoint{0.625000in}{1.390825in}}%
\pgfusepath{stroke}%
\end{pgfscope}%
\begin{pgfscope}%
\pgfpathrectangle{\pgfqpoint{0.625000in}{0.550000in}}{\pgfqpoint{3.875000in}{3.850000in}} %
\pgfusepath{clip}%
\pgfsetbuttcap%
\pgfsetroundjoin%
\pgfsetlinewidth{0.250937pt}%
\definecolor{currentstroke}{rgb}{0.000000,0.000000,0.000000}%
\pgfsetstrokecolor{currentstroke}%
\pgfsetdash{}{0pt}%
\pgfpathmoveto{\pgfqpoint{0.625000in}{1.561772in}}%
\pgfpathlineto{\pgfqpoint{0.633673in}{1.553509in}}%
\pgfpathlineto{\pgfqpoint{0.625000in}{1.544977in}}%
\pgfusepath{stroke}%
\end{pgfscope}%
\begin{pgfscope}%
\pgfpathrectangle{\pgfqpoint{0.625000in}{0.550000in}}{\pgfqpoint{3.875000in}{3.850000in}} %
\pgfusepath{clip}%
\pgfsetbuttcap%
\pgfsetroundjoin%
\pgfsetlinewidth{0.250937pt}%
\definecolor{currentstroke}{rgb}{0.000000,0.000000,0.000000}%
\pgfsetstrokecolor{currentstroke}%
\pgfsetdash{}{0pt}%
\pgfpathmoveto{\pgfqpoint{0.625000in}{1.716159in}}%
\pgfpathlineto{\pgfqpoint{0.634712in}{1.715195in}}%
\pgfpathlineto{\pgfqpoint{0.642515in}{1.707895in}}%
\pgfpathlineto{\pgfqpoint{0.634712in}{1.700595in}}%
\pgfpathlineto{\pgfqpoint{0.625000in}{1.699578in}}%
\pgfusepath{stroke}%
\end{pgfscope}%
\begin{pgfscope}%
\pgfpathrectangle{\pgfqpoint{0.625000in}{0.550000in}}{\pgfqpoint{3.875000in}{3.850000in}} %
\pgfusepath{clip}%
\pgfsetbuttcap%
\pgfsetroundjoin%
\pgfsetlinewidth{0.250937pt}%
\definecolor{currentstroke}{rgb}{0.000000,0.000000,0.000000}%
\pgfsetstrokecolor{currentstroke}%
\pgfsetdash{}{0pt}%
\pgfpathmoveto{\pgfqpoint{0.625000in}{1.870618in}}%
\pgfpathlineto{\pgfqpoint{0.633653in}{1.862281in}}%
\pgfpathlineto{\pgfqpoint{0.625000in}{1.854116in}}%
\pgfusepath{stroke}%
\end{pgfscope}%
\begin{pgfscope}%
\pgfpathrectangle{\pgfqpoint{0.625000in}{0.550000in}}{\pgfqpoint{3.875000in}{3.850000in}} %
\pgfusepath{clip}%
\pgfsetbuttcap%
\pgfsetroundjoin%
\pgfsetlinewidth{0.250937pt}%
\definecolor{currentstroke}{rgb}{0.000000,0.000000,0.000000}%
\pgfsetstrokecolor{currentstroke}%
\pgfsetdash{}{0pt}%
\pgfpathmoveto{\pgfqpoint{0.625000in}{2.025303in}}%
\pgfpathlineto{\pgfqpoint{0.633644in}{2.016667in}}%
\pgfpathlineto{\pgfqpoint{0.625000in}{2.008101in}}%
\pgfusepath{stroke}%
\end{pgfscope}%
\begin{pgfscope}%
\pgfpathrectangle{\pgfqpoint{0.625000in}{0.550000in}}{\pgfqpoint{3.875000in}{3.850000in}} %
\pgfusepath{clip}%
\pgfsetbuttcap%
\pgfsetroundjoin%
\pgfsetlinewidth{0.250937pt}%
\definecolor{currentstroke}{rgb}{0.000000,0.000000,0.000000}%
\pgfsetstrokecolor{currentstroke}%
\pgfsetdash{}{0pt}%
\pgfpathmoveto{\pgfqpoint{0.625000in}{2.179250in}}%
\pgfpathlineto{\pgfqpoint{0.633669in}{2.171053in}}%
\pgfpathlineto{\pgfqpoint{0.625000in}{2.162621in}}%
\pgfusepath{stroke}%
\end{pgfscope}%
\begin{pgfscope}%
\pgfpathrectangle{\pgfqpoint{0.625000in}{0.550000in}}{\pgfqpoint{3.875000in}{3.850000in}} %
\pgfusepath{clip}%
\pgfsetbuttcap%
\pgfsetroundjoin%
\pgfsetlinewidth{0.250937pt}%
\definecolor{currentstroke}{rgb}{0.000000,0.000000,0.000000}%
\pgfsetstrokecolor{currentstroke}%
\pgfsetdash{}{0pt}%
\pgfpathmoveto{\pgfqpoint{0.625000in}{2.333597in}}%
\pgfpathlineto{\pgfqpoint{0.633709in}{2.325439in}}%
\pgfpathlineto{\pgfqpoint{0.625000in}{2.316709in}}%
\pgfusepath{stroke}%
\end{pgfscope}%
\begin{pgfscope}%
\pgfpathrectangle{\pgfqpoint{0.625000in}{0.550000in}}{\pgfqpoint{3.875000in}{3.850000in}} %
\pgfusepath{clip}%
\pgfsetbuttcap%
\pgfsetroundjoin%
\pgfsetlinewidth{0.250937pt}%
\definecolor{currentstroke}{rgb}{0.000000,0.000000,0.000000}%
\pgfsetstrokecolor{currentstroke}%
\pgfsetdash{}{0pt}%
\pgfpathmoveto{\pgfqpoint{0.625000in}{2.488970in}}%
\pgfpathlineto{\pgfqpoint{0.625472in}{2.489474in}}%
\pgfpathlineto{\pgfqpoint{0.627642in}{2.508772in}}%
\pgfpathlineto{\pgfqpoint{0.631453in}{2.528070in}}%
\pgfpathlineto{\pgfqpoint{0.637218in}{2.547368in}}%
\pgfpathlineto{\pgfqpoint{0.644646in}{2.566667in}}%
\pgfpathlineto{\pgfqpoint{0.659847in}{2.595614in}}%
\pgfpathlineto{\pgfqpoint{0.665908in}{2.605263in}}%
\pgfpathlineto{\pgfqpoint{0.683271in}{2.628903in}}%
\pgfpathlineto{\pgfqpoint{0.706884in}{2.653509in}}%
\pgfpathlineto{\pgfqpoint{0.722118in}{2.667026in}}%
\pgfpathlineto{\pgfqpoint{0.751253in}{2.687611in}}%
\pgfpathlineto{\pgfqpoint{0.780388in}{2.703248in}}%
\pgfpathlineto{\pgfqpoint{0.799833in}{2.711404in}}%
\pgfpathlineto{\pgfqpoint{0.819236in}{2.718040in}}%
\pgfpathlineto{\pgfqpoint{0.848371in}{2.725108in}}%
\pgfpathlineto{\pgfqpoint{0.877506in}{2.728998in}}%
\pgfpathlineto{\pgfqpoint{0.906642in}{2.729862in}}%
\pgfpathlineto{\pgfqpoint{0.935777in}{2.727719in}}%
\pgfpathlineto{\pgfqpoint{0.964912in}{2.722403in}}%
\pgfpathlineto{\pgfqpoint{0.984336in}{2.717016in}}%
\pgfpathlineto{\pgfqpoint{1.003759in}{2.710007in}}%
\pgfpathlineto{\pgfqpoint{1.023183in}{2.701237in}}%
\pgfpathlineto{\pgfqpoint{1.042607in}{2.690473in}}%
\pgfpathlineto{\pgfqpoint{1.062030in}{2.677378in}}%
\pgfpathlineto{\pgfqpoint{1.081454in}{2.661464in}}%
\pgfpathlineto{\pgfqpoint{1.100877in}{2.641873in}}%
\pgfpathlineto{\pgfqpoint{1.115010in}{2.624561in}}%
\pgfpathlineto{\pgfqpoint{1.130013in}{2.601679in}}%
\pgfpathlineto{\pgfqpoint{1.139724in}{2.583261in}}%
\pgfpathlineto{\pgfqpoint{1.146923in}{2.566667in}}%
\pgfpathlineto{\pgfqpoint{1.153470in}{2.547368in}}%
\pgfpathlineto{\pgfqpoint{1.159148in}{2.523024in}}%
\pgfpathlineto{\pgfqpoint{1.161396in}{2.508772in}}%
\pgfpathlineto{\pgfqpoint{1.162967in}{2.489474in}}%
\pgfpathlineto{\pgfqpoint{1.162967in}{2.470175in}}%
\pgfpathlineto{\pgfqpoint{1.161396in}{2.450877in}}%
\pgfpathlineto{\pgfqpoint{1.158234in}{2.431579in}}%
\pgfpathlineto{\pgfqpoint{1.153470in}{2.412281in}}%
\pgfpathlineto{\pgfqpoint{1.146923in}{2.392982in}}%
\pgfpathlineto{\pgfqpoint{1.138478in}{2.373684in}}%
\pgfpathlineto{\pgfqpoint{1.127953in}{2.354386in}}%
\pgfpathlineto{\pgfqpoint{1.115010in}{2.335088in}}%
\pgfpathlineto{\pgfqpoint{1.099137in}{2.315789in}}%
\pgfpathlineto{\pgfqpoint{1.079599in}{2.296491in}}%
\pgfpathlineto{\pgfqpoint{1.062030in}{2.282272in}}%
\pgfpathlineto{\pgfqpoint{1.039912in}{2.267544in}}%
\pgfpathlineto{\pgfqpoint{1.022143in}{2.257895in}}%
\pgfpathlineto{\pgfqpoint{1.000205in}{2.248246in}}%
\pgfpathlineto{\pgfqpoint{0.974624in}{2.239751in}}%
\pgfpathlineto{\pgfqpoint{0.955201in}{2.235107in}}%
\pgfpathlineto{\pgfqpoint{0.926065in}{2.230874in}}%
\pgfpathlineto{\pgfqpoint{0.896930in}{2.229744in}}%
\pgfpathlineto{\pgfqpoint{0.867794in}{2.231603in}}%
\pgfpathlineto{\pgfqpoint{0.838659in}{2.236556in}}%
\pgfpathlineto{\pgfqpoint{0.809524in}{2.244696in}}%
\pgfpathlineto{\pgfqpoint{0.780388in}{2.256402in}}%
\pgfpathlineto{\pgfqpoint{0.751253in}{2.272039in}}%
\pgfpathlineto{\pgfqpoint{0.722118in}{2.292623in}}%
\pgfpathlineto{\pgfqpoint{0.692982in}{2.319735in}}%
\pgfpathlineto{\pgfqpoint{0.672571in}{2.344737in}}%
\pgfpathlineto{\pgfqpoint{0.659847in}{2.364035in}}%
\pgfpathlineto{\pgfqpoint{0.649304in}{2.383333in}}%
\pgfpathlineto{\pgfqpoint{0.640773in}{2.402632in}}%
\pgfpathlineto{\pgfqpoint{0.633989in}{2.421930in}}%
\pgfpathlineto{\pgfqpoint{0.629354in}{2.441228in}}%
\pgfpathlineto{\pgfqpoint{0.626203in}{2.460526in}}%
\pgfpathlineto{\pgfqpoint{0.625000in}{2.470679in}}%
\pgfpathlineto{\pgfqpoint{0.625000in}{2.470679in}}%
\pgfusepath{stroke}%
\end{pgfscope}%
\begin{pgfscope}%
\pgfpathrectangle{\pgfqpoint{0.625000in}{0.550000in}}{\pgfqpoint{3.875000in}{3.850000in}} %
\pgfusepath{clip}%
\pgfsetbuttcap%
\pgfsetroundjoin%
\pgfsetlinewidth{0.250937pt}%
\definecolor{currentstroke}{rgb}{0.000000,0.000000,0.000000}%
\pgfsetstrokecolor{currentstroke}%
\pgfsetdash{}{0pt}%
\pgfpathmoveto{\pgfqpoint{0.625000in}{2.642938in}}%
\pgfpathlineto{\pgfqpoint{0.633708in}{2.634211in}}%
\pgfpathlineto{\pgfqpoint{0.625000in}{2.626055in}}%
\pgfusepath{stroke}%
\end{pgfscope}%
\begin{pgfscope}%
\pgfpathrectangle{\pgfqpoint{0.625000in}{0.550000in}}{\pgfqpoint{3.875000in}{3.850000in}} %
\pgfusepath{clip}%
\pgfsetbuttcap%
\pgfsetroundjoin%
\pgfsetlinewidth{0.250937pt}%
\definecolor{currentstroke}{rgb}{0.000000,0.000000,0.000000}%
\pgfsetstrokecolor{currentstroke}%
\pgfsetdash{}{0pt}%
\pgfpathmoveto{\pgfqpoint{0.625000in}{2.797021in}}%
\pgfpathlineto{\pgfqpoint{0.633662in}{2.788596in}}%
\pgfpathlineto{\pgfqpoint{0.625000in}{2.780408in}}%
\pgfusepath{stroke}%
\end{pgfscope}%
\begin{pgfscope}%
\pgfpathrectangle{\pgfqpoint{0.625000in}{0.550000in}}{\pgfqpoint{3.875000in}{3.850000in}} %
\pgfusepath{clip}%
\pgfsetbuttcap%
\pgfsetroundjoin%
\pgfsetlinewidth{0.250937pt}%
\definecolor{currentstroke}{rgb}{0.000000,0.000000,0.000000}%
\pgfsetstrokecolor{currentstroke}%
\pgfsetdash{}{0pt}%
\pgfpathmoveto{\pgfqpoint{0.625000in}{2.951529in}}%
\pgfpathlineto{\pgfqpoint{0.633626in}{2.942982in}}%
\pgfpathlineto{\pgfqpoint{0.625000in}{2.934364in}}%
\pgfusepath{stroke}%
\end{pgfscope}%
\begin{pgfscope}%
\pgfpathrectangle{\pgfqpoint{0.625000in}{0.550000in}}{\pgfqpoint{3.875000in}{3.850000in}} %
\pgfusepath{clip}%
\pgfsetbuttcap%
\pgfsetroundjoin%
\pgfsetlinewidth{0.250937pt}%
\definecolor{currentstroke}{rgb}{0.000000,0.000000,0.000000}%
\pgfsetstrokecolor{currentstroke}%
\pgfsetdash{}{0pt}%
\pgfpathmoveto{\pgfqpoint{0.625000in}{3.105514in}}%
\pgfpathlineto{\pgfqpoint{0.633638in}{3.097368in}}%
\pgfpathlineto{\pgfqpoint{0.625000in}{3.089049in}}%
\pgfusepath{stroke}%
\end{pgfscope}%
\begin{pgfscope}%
\pgfpathrectangle{\pgfqpoint{0.625000in}{0.550000in}}{\pgfqpoint{3.875000in}{3.850000in}} %
\pgfusepath{clip}%
\pgfsetbuttcap%
\pgfsetroundjoin%
\pgfsetlinewidth{0.250937pt}%
\definecolor{currentstroke}{rgb}{0.000000,0.000000,0.000000}%
\pgfsetstrokecolor{currentstroke}%
\pgfsetdash{}{0pt}%
\pgfpathmoveto{\pgfqpoint{0.625000in}{3.260007in}}%
\pgfpathlineto{\pgfqpoint{0.634712in}{3.259054in}}%
\pgfpathlineto{\pgfqpoint{0.642515in}{3.251754in}}%
\pgfpathlineto{\pgfqpoint{0.634712in}{3.244455in}}%
\pgfpathlineto{\pgfqpoint{0.625000in}{3.243556in}}%
\pgfusepath{stroke}%
\end{pgfscope}%
\begin{pgfscope}%
\pgfpathrectangle{\pgfqpoint{0.625000in}{0.550000in}}{\pgfqpoint{3.875000in}{3.850000in}} %
\pgfusepath{clip}%
\pgfsetbuttcap%
\pgfsetroundjoin%
\pgfsetlinewidth{0.250937pt}%
\definecolor{currentstroke}{rgb}{0.000000,0.000000,0.000000}%
\pgfsetstrokecolor{currentstroke}%
\pgfsetdash{}{0pt}%
\pgfpathmoveto{\pgfqpoint{0.625000in}{3.414634in}}%
\pgfpathlineto{\pgfqpoint{0.633637in}{3.406140in}}%
\pgfpathlineto{\pgfqpoint{0.625000in}{3.397923in}}%
\pgfusepath{stroke}%
\end{pgfscope}%
\begin{pgfscope}%
\pgfpathrectangle{\pgfqpoint{0.625000in}{0.550000in}}{\pgfqpoint{3.875000in}{3.850000in}} %
\pgfusepath{clip}%
\pgfsetbuttcap%
\pgfsetroundjoin%
\pgfsetlinewidth{0.250937pt}%
\definecolor{currentstroke}{rgb}{0.000000,0.000000,0.000000}%
\pgfsetstrokecolor{currentstroke}%
\pgfsetdash{}{0pt}%
\pgfpathmoveto{\pgfqpoint{0.625000in}{3.568770in}}%
\pgfpathlineto{\pgfqpoint{0.633677in}{3.560526in}}%
\pgfpathlineto{\pgfqpoint{0.625000in}{3.552287in}}%
\pgfusepath{stroke}%
\end{pgfscope}%
\begin{pgfscope}%
\pgfpathrectangle{\pgfqpoint{0.625000in}{0.550000in}}{\pgfqpoint{3.875000in}{3.850000in}} %
\pgfusepath{clip}%
\pgfsetbuttcap%
\pgfsetroundjoin%
\pgfsetlinewidth{0.250937pt}%
\definecolor{currentstroke}{rgb}{0.000000,0.000000,0.000000}%
\pgfsetstrokecolor{currentstroke}%
\pgfsetdash{}{0pt}%
\pgfpathmoveto{\pgfqpoint{0.625000in}{3.723240in}}%
\pgfpathlineto{\pgfqpoint{0.633696in}{3.714912in}}%
\pgfpathlineto{\pgfqpoint{0.625000in}{3.706334in}}%
\pgfusepath{stroke}%
\end{pgfscope}%
\begin{pgfscope}%
\pgfpathrectangle{\pgfqpoint{0.625000in}{0.550000in}}{\pgfqpoint{3.875000in}{3.850000in}} %
\pgfusepath{clip}%
\pgfsetbuttcap%
\pgfsetroundjoin%
\pgfsetlinewidth{0.250937pt}%
\definecolor{currentstroke}{rgb}{0.000000,0.000000,0.000000}%
\pgfsetstrokecolor{currentstroke}%
\pgfsetdash{}{0pt}%
\pgfpathmoveto{\pgfqpoint{0.625000in}{3.877511in}}%
\pgfpathlineto{\pgfqpoint{0.633630in}{3.869298in}}%
\pgfpathlineto{\pgfqpoint{0.625000in}{3.861052in}}%
\pgfusepath{stroke}%
\end{pgfscope}%
\begin{pgfscope}%
\pgfpathrectangle{\pgfqpoint{0.625000in}{0.550000in}}{\pgfqpoint{3.875000in}{3.850000in}} %
\pgfusepath{clip}%
\pgfsetbuttcap%
\pgfsetroundjoin%
\pgfsetlinewidth{0.250937pt}%
\definecolor{currentstroke}{rgb}{0.000000,0.000000,0.000000}%
\pgfsetstrokecolor{currentstroke}%
\pgfsetdash{}{0pt}%
\pgfpathmoveto{\pgfqpoint{0.625000in}{4.032004in}}%
\pgfpathlineto{\pgfqpoint{0.634712in}{4.030984in}}%
\pgfpathlineto{\pgfqpoint{0.642515in}{4.023684in}}%
\pgfpathlineto{\pgfqpoint{0.634712in}{4.016384in}}%
\pgfpathlineto{\pgfqpoint{0.625000in}{4.015445in}}%
\pgfusepath{stroke}%
\end{pgfscope}%
\begin{pgfscope}%
\pgfpathrectangle{\pgfqpoint{0.625000in}{0.550000in}}{\pgfqpoint{3.875000in}{3.850000in}} %
\pgfusepath{clip}%
\pgfsetbuttcap%
\pgfsetroundjoin%
\pgfsetlinewidth{0.250937pt}%
\definecolor{currentstroke}{rgb}{0.000000,0.000000,0.000000}%
\pgfsetstrokecolor{currentstroke}%
\pgfsetdash{}{0pt}%
\pgfpathmoveto{\pgfqpoint{0.625000in}{4.186379in}}%
\pgfpathlineto{\pgfqpoint{0.633631in}{4.178070in}}%
\pgfpathlineto{\pgfqpoint{0.625000in}{4.170150in}}%
\pgfusepath{stroke}%
\end{pgfscope}%
\begin{pgfscope}%
\pgfpathrectangle{\pgfqpoint{0.625000in}{0.550000in}}{\pgfqpoint{3.875000in}{3.850000in}} %
\pgfusepath{clip}%
\pgfsetbuttcap%
\pgfsetroundjoin%
\pgfsetlinewidth{0.250937pt}%
\definecolor{currentstroke}{rgb}{0.000000,0.000000,0.000000}%
\pgfsetstrokecolor{currentstroke}%
\pgfsetdash{}{0pt}%
\pgfpathmoveto{\pgfqpoint{0.625000in}{4.340720in}}%
\pgfpathlineto{\pgfqpoint{0.633584in}{4.332456in}}%
\pgfpathlineto{\pgfqpoint{0.625000in}{4.324164in}}%
\pgfusepath{stroke}%
\end{pgfscope}%
\begin{pgfscope}%
\pgfpathrectangle{\pgfqpoint{0.625000in}{0.550000in}}{\pgfqpoint{3.875000in}{3.850000in}} %
\pgfusepath{clip}%
\pgfsetbuttcap%
\pgfsetroundjoin%
\pgfsetlinewidth{0.250937pt}%
\definecolor{currentstroke}{rgb}{0.000000,0.000000,0.000000}%
\pgfsetstrokecolor{currentstroke}%
\pgfsetdash{}{0pt}%
\pgfpathmoveto{\pgfqpoint{0.634712in}{1.174399in}}%
\pgfpathlineto{\pgfqpoint{0.633279in}{1.177193in}}%
\pgfpathlineto{\pgfqpoint{0.628161in}{1.186842in}}%
\pgfpathlineto{\pgfqpoint{0.626721in}{1.196491in}}%
\pgfpathlineto{\pgfqpoint{0.631301in}{1.206140in}}%
\pgfpathlineto{\pgfqpoint{0.634712in}{1.212581in}}%
\pgfpathlineto{\pgfqpoint{0.644424in}{1.215635in}}%
\pgfpathlineto{\pgfqpoint{0.654135in}{1.215785in}}%
\pgfpathlineto{\pgfqpoint{0.663847in}{1.211601in}}%
\pgfpathlineto{\pgfqpoint{0.669916in}{1.206140in}}%
\pgfpathlineto{\pgfqpoint{0.673403in}{1.196491in}}%
\pgfpathlineto{\pgfqpoint{0.672814in}{1.186842in}}%
\pgfpathlineto{\pgfqpoint{0.667055in}{1.177193in}}%
\pgfpathlineto{\pgfqpoint{0.663847in}{1.174830in}}%
\pgfpathlineto{\pgfqpoint{0.654135in}{1.170551in}}%
\pgfpathlineto{\pgfqpoint{0.644424in}{1.170356in}}%
\pgfpathlineto{\pgfqpoint{0.634712in}{1.174399in}}%
\pgfusepath{stroke}%
\end{pgfscope}%
\begin{pgfscope}%
\pgfpathrectangle{\pgfqpoint{0.625000in}{0.550000in}}{\pgfqpoint{3.875000in}{3.850000in}} %
\pgfusepath{clip}%
\pgfsetbuttcap%
\pgfsetroundjoin%
\pgfsetlinewidth{0.250937pt}%
\definecolor{currentstroke}{rgb}{0.000000,0.000000,0.000000}%
\pgfsetstrokecolor{currentstroke}%
\pgfsetdash{}{0pt}%
\pgfpathmoveto{\pgfqpoint{0.634712in}{3.747069in}}%
\pgfpathlineto{\pgfqpoint{0.631301in}{3.753509in}}%
\pgfpathlineto{\pgfqpoint{0.626721in}{3.763158in}}%
\pgfpathlineto{\pgfqpoint{0.628161in}{3.772807in}}%
\pgfpathlineto{\pgfqpoint{0.633279in}{3.782456in}}%
\pgfpathlineto{\pgfqpoint{0.634712in}{3.785250in}}%
\pgfpathlineto{\pgfqpoint{0.644424in}{3.789293in}}%
\pgfpathlineto{\pgfqpoint{0.654135in}{3.789098in}}%
\pgfpathlineto{\pgfqpoint{0.663847in}{3.784820in}}%
\pgfpathlineto{\pgfqpoint{0.667055in}{3.782456in}}%
\pgfpathlineto{\pgfqpoint{0.672814in}{3.772807in}}%
\pgfpathlineto{\pgfqpoint{0.673403in}{3.763158in}}%
\pgfpathlineto{\pgfqpoint{0.669916in}{3.753509in}}%
\pgfpathlineto{\pgfqpoint{0.663847in}{3.748048in}}%
\pgfpathlineto{\pgfqpoint{0.654135in}{3.743864in}}%
\pgfpathlineto{\pgfqpoint{0.644424in}{3.744014in}}%
\pgfpathlineto{\pgfqpoint{0.634712in}{3.747069in}}%
\pgfusepath{stroke}%
\end{pgfscope}%
\begin{pgfscope}%
\pgfpathrectangle{\pgfqpoint{0.625000in}{0.550000in}}{\pgfqpoint{3.875000in}{3.850000in}} %
\pgfusepath{clip}%
\pgfsetbuttcap%
\pgfsetroundjoin%
\pgfsetlinewidth{0.250937pt}%
\definecolor{currentstroke}{rgb}{0.000000,0.000000,0.000000}%
\pgfsetstrokecolor{currentstroke}%
\pgfsetdash{}{0pt}%
\pgfpathmoveto{\pgfqpoint{0.625000in}{0.635740in}}%
\pgfpathlineto{\pgfqpoint{0.633815in}{0.627193in}}%
\pgfpathlineto{\pgfqpoint{0.625000in}{0.618671in}}%
\pgfusepath{stroke}%
\end{pgfscope}%
\begin{pgfscope}%
\pgfpathrectangle{\pgfqpoint{0.625000in}{0.550000in}}{\pgfqpoint{3.875000in}{3.850000in}} %
\pgfusepath{clip}%
\pgfsetbuttcap%
\pgfsetroundjoin%
\pgfsetlinewidth{0.250937pt}%
\definecolor{currentstroke}{rgb}{0.000000,0.000000,0.000000}%
\pgfsetstrokecolor{currentstroke}%
\pgfsetdash{}{0pt}%
\pgfpathmoveto{\pgfqpoint{0.625000in}{0.789730in}}%
\pgfpathlineto{\pgfqpoint{0.633818in}{0.781579in}}%
\pgfpathlineto{\pgfqpoint{0.625000in}{0.773065in}}%
\pgfusepath{stroke}%
\end{pgfscope}%
\begin{pgfscope}%
\pgfpathrectangle{\pgfqpoint{0.625000in}{0.550000in}}{\pgfqpoint{3.875000in}{3.850000in}} %
\pgfusepath{clip}%
\pgfsetbuttcap%
\pgfsetroundjoin%
\pgfsetlinewidth{0.250937pt}%
\definecolor{currentstroke}{rgb}{0.000000,0.000000,0.000000}%
\pgfsetstrokecolor{currentstroke}%
\pgfsetdash{}{0pt}%
\pgfpathmoveto{\pgfqpoint{0.625000in}{0.944400in}}%
\pgfpathlineto{\pgfqpoint{0.634712in}{0.944077in}}%
\pgfpathlineto{\pgfqpoint{0.643383in}{0.935965in}}%
\pgfpathlineto{\pgfqpoint{0.634712in}{0.927852in}}%
\pgfpathlineto{\pgfqpoint{0.625000in}{0.927454in}}%
\pgfusepath{stroke}%
\end{pgfscope}%
\begin{pgfscope}%
\pgfpathrectangle{\pgfqpoint{0.625000in}{0.550000in}}{\pgfqpoint{3.875000in}{3.850000in}} %
\pgfusepath{clip}%
\pgfsetbuttcap%
\pgfsetroundjoin%
\pgfsetlinewidth{0.250937pt}%
\definecolor{currentstroke}{rgb}{0.000000,0.000000,0.000000}%
\pgfsetstrokecolor{currentstroke}%
\pgfsetdash{}{0pt}%
\pgfpathmoveto{\pgfqpoint{0.625000in}{1.098805in}}%
\pgfpathlineto{\pgfqpoint{0.633816in}{1.090351in}}%
\pgfpathlineto{\pgfqpoint{0.625000in}{1.081928in}}%
\pgfusepath{stroke}%
\end{pgfscope}%
\begin{pgfscope}%
\pgfpathrectangle{\pgfqpoint{0.625000in}{0.550000in}}{\pgfqpoint{3.875000in}{3.850000in}} %
\pgfusepath{clip}%
\pgfsetbuttcap%
\pgfsetroundjoin%
\pgfsetlinewidth{0.250937pt}%
\definecolor{currentstroke}{rgb}{0.000000,0.000000,0.000000}%
\pgfsetstrokecolor{currentstroke}%
\pgfsetdash{}{0pt}%
\pgfpathmoveto{\pgfqpoint{0.625000in}{1.253461in}}%
\pgfpathlineto{\pgfqpoint{0.633840in}{1.244737in}}%
\pgfpathlineto{\pgfqpoint{0.625000in}{1.236253in}}%
\pgfusepath{stroke}%
\end{pgfscope}%
\begin{pgfscope}%
\pgfpathrectangle{\pgfqpoint{0.625000in}{0.550000in}}{\pgfqpoint{3.875000in}{3.850000in}} %
\pgfusepath{clip}%
\pgfsetbuttcap%
\pgfsetroundjoin%
\pgfsetlinewidth{0.250937pt}%
\definecolor{currentstroke}{rgb}{0.000000,0.000000,0.000000}%
\pgfsetstrokecolor{currentstroke}%
\pgfsetdash{}{0pt}%
\pgfpathmoveto{\pgfqpoint{0.625000in}{1.407497in}}%
\pgfpathlineto{\pgfqpoint{0.633804in}{1.399123in}}%
\pgfpathlineto{\pgfqpoint{0.625000in}{1.390744in}}%
\pgfusepath{stroke}%
\end{pgfscope}%
\begin{pgfscope}%
\pgfpathrectangle{\pgfqpoint{0.625000in}{0.550000in}}{\pgfqpoint{3.875000in}{3.850000in}} %
\pgfusepath{clip}%
\pgfsetbuttcap%
\pgfsetroundjoin%
\pgfsetlinewidth{0.250937pt}%
\definecolor{currentstroke}{rgb}{0.000000,0.000000,0.000000}%
\pgfsetstrokecolor{currentstroke}%
\pgfsetdash{}{0pt}%
\pgfpathmoveto{\pgfqpoint{0.625000in}{1.561852in}}%
\pgfpathlineto{\pgfqpoint{0.633757in}{1.553509in}}%
\pgfpathlineto{\pgfqpoint{0.625000in}{1.544894in}}%
\pgfusepath{stroke}%
\end{pgfscope}%
\begin{pgfscope}%
\pgfpathrectangle{\pgfqpoint{0.625000in}{0.550000in}}{\pgfqpoint{3.875000in}{3.850000in}} %
\pgfusepath{clip}%
\pgfsetbuttcap%
\pgfsetroundjoin%
\pgfsetlinewidth{0.250937pt}%
\definecolor{currentstroke}{rgb}{0.000000,0.000000,0.000000}%
\pgfsetstrokecolor{currentstroke}%
\pgfsetdash{}{0pt}%
\pgfpathmoveto{\pgfqpoint{0.625000in}{1.716240in}}%
\pgfpathlineto{\pgfqpoint{0.634712in}{1.716007in}}%
\pgfpathlineto{\pgfqpoint{0.643383in}{1.707895in}}%
\pgfpathlineto{\pgfqpoint{0.634712in}{1.699782in}}%
\pgfpathlineto{\pgfqpoint{0.625000in}{1.699497in}}%
\pgfusepath{stroke}%
\end{pgfscope}%
\begin{pgfscope}%
\pgfpathrectangle{\pgfqpoint{0.625000in}{0.550000in}}{\pgfqpoint{3.875000in}{3.850000in}} %
\pgfusepath{clip}%
\pgfsetbuttcap%
\pgfsetroundjoin%
\pgfsetlinewidth{0.250937pt}%
\definecolor{currentstroke}{rgb}{0.000000,0.000000,0.000000}%
\pgfsetstrokecolor{currentstroke}%
\pgfsetdash{}{0pt}%
\pgfpathmoveto{\pgfqpoint{0.625000in}{1.870701in}}%
\pgfpathlineto{\pgfqpoint{0.633739in}{1.862281in}}%
\pgfpathlineto{\pgfqpoint{0.625000in}{1.854035in}}%
\pgfusepath{stroke}%
\end{pgfscope}%
\begin{pgfscope}%
\pgfpathrectangle{\pgfqpoint{0.625000in}{0.550000in}}{\pgfqpoint{3.875000in}{3.850000in}} %
\pgfusepath{clip}%
\pgfsetbuttcap%
\pgfsetroundjoin%
\pgfsetlinewidth{0.250937pt}%
\definecolor{currentstroke}{rgb}{0.000000,0.000000,0.000000}%
\pgfsetstrokecolor{currentstroke}%
\pgfsetdash{}{0pt}%
\pgfpathmoveto{\pgfqpoint{0.625000in}{2.025390in}}%
\pgfpathlineto{\pgfqpoint{0.633732in}{2.016667in}}%
\pgfpathlineto{\pgfqpoint{0.625000in}{2.008015in}}%
\pgfusepath{stroke}%
\end{pgfscope}%
\begin{pgfscope}%
\pgfpathrectangle{\pgfqpoint{0.625000in}{0.550000in}}{\pgfqpoint{3.875000in}{3.850000in}} %
\pgfusepath{clip}%
\pgfsetbuttcap%
\pgfsetroundjoin%
\pgfsetlinewidth{0.250937pt}%
\definecolor{currentstroke}{rgb}{0.000000,0.000000,0.000000}%
\pgfsetstrokecolor{currentstroke}%
\pgfsetdash{}{0pt}%
\pgfpathmoveto{\pgfqpoint{0.625000in}{2.179332in}}%
\pgfpathlineto{\pgfqpoint{0.633756in}{2.171053in}}%
\pgfpathlineto{\pgfqpoint{0.625000in}{2.162536in}}%
\pgfusepath{stroke}%
\end{pgfscope}%
\begin{pgfscope}%
\pgfpathrectangle{\pgfqpoint{0.625000in}{0.550000in}}{\pgfqpoint{3.875000in}{3.850000in}} %
\pgfusepath{clip}%
\pgfsetbuttcap%
\pgfsetroundjoin%
\pgfsetlinewidth{0.250937pt}%
\definecolor{currentstroke}{rgb}{0.000000,0.000000,0.000000}%
\pgfsetstrokecolor{currentstroke}%
\pgfsetdash{}{0pt}%
\pgfpathmoveto{\pgfqpoint{0.625000in}{2.333679in}}%
\pgfpathlineto{\pgfqpoint{0.633798in}{2.325439in}}%
\pgfpathlineto{\pgfqpoint{0.625000in}{2.316620in}}%
\pgfusepath{stroke}%
\end{pgfscope}%
\begin{pgfscope}%
\pgfpathrectangle{\pgfqpoint{0.625000in}{0.550000in}}{\pgfqpoint{3.875000in}{3.850000in}} %
\pgfusepath{clip}%
\pgfsetbuttcap%
\pgfsetroundjoin%
\pgfsetlinewidth{0.250937pt}%
\definecolor{currentstroke}{rgb}{0.000000,0.000000,0.000000}%
\pgfsetstrokecolor{currentstroke}%
\pgfsetdash{}{0pt}%
\pgfpathmoveto{\pgfqpoint{0.625000in}{2.489054in}}%
\pgfpathlineto{\pgfqpoint{0.625393in}{2.489474in}}%
\pgfpathlineto{\pgfqpoint{0.627488in}{2.508772in}}%
\pgfpathlineto{\pgfqpoint{0.633398in}{2.537719in}}%
\pgfpathlineto{\pgfqpoint{0.636351in}{2.547368in}}%
\pgfpathlineto{\pgfqpoint{0.647457in}{2.576316in}}%
\pgfpathlineto{\pgfqpoint{0.657018in}{2.595614in}}%
\pgfpathlineto{\pgfqpoint{0.673559in}{2.622782in}}%
\pgfpathlineto{\pgfqpoint{0.702694in}{2.658598in}}%
\pgfpathlineto{\pgfqpoint{0.731830in}{2.685595in}}%
\pgfpathlineto{\pgfqpoint{0.760965in}{2.706751in}}%
\pgfpathlineto{\pgfqpoint{0.790100in}{2.723366in}}%
\pgfpathlineto{\pgfqpoint{0.819236in}{2.736307in}}%
\pgfpathlineto{\pgfqpoint{0.848371in}{2.745976in}}%
\pgfpathlineto{\pgfqpoint{0.877506in}{2.752682in}}%
\pgfpathlineto{\pgfqpoint{0.906642in}{2.756622in}}%
\pgfpathlineto{\pgfqpoint{0.935777in}{2.757856in}}%
\pgfpathlineto{\pgfqpoint{0.964912in}{2.756426in}}%
\pgfpathlineto{\pgfqpoint{0.994048in}{2.752233in}}%
\pgfpathlineto{\pgfqpoint{1.023183in}{2.745150in}}%
\pgfpathlineto{\pgfqpoint{1.052318in}{2.734902in}}%
\pgfpathlineto{\pgfqpoint{1.071742in}{2.726117in}}%
\pgfpathlineto{\pgfqpoint{1.098002in}{2.711404in}}%
\pgfpathlineto{\pgfqpoint{1.120301in}{2.695840in}}%
\pgfpathlineto{\pgfqpoint{1.139724in}{2.679486in}}%
\pgfpathlineto{\pgfqpoint{1.159148in}{2.659762in}}%
\pgfpathlineto{\pgfqpoint{1.172386in}{2.643860in}}%
\pgfpathlineto{\pgfqpoint{1.188283in}{2.620725in}}%
\pgfpathlineto{\pgfqpoint{1.197995in}{2.603681in}}%
\pgfpathlineto{\pgfqpoint{1.207707in}{2.582982in}}%
\pgfpathlineto{\pgfqpoint{1.214007in}{2.566667in}}%
\pgfpathlineto{\pgfqpoint{1.222209in}{2.537719in}}%
\pgfpathlineto{\pgfqpoint{1.226975in}{2.508772in}}%
\pgfpathlineto{\pgfqpoint{1.228579in}{2.479825in}}%
\pgfpathlineto{\pgfqpoint{1.226975in}{2.450877in}}%
\pgfpathlineto{\pgfqpoint{1.222209in}{2.421930in}}%
\pgfpathlineto{\pgfqpoint{1.214007in}{2.392982in}}%
\pgfpathlineto{\pgfqpoint{1.206488in}{2.373684in}}%
\pgfpathlineto{\pgfqpoint{1.197196in}{2.354386in}}%
\pgfpathlineto{\pgfqpoint{1.185936in}{2.335088in}}%
\pgfpathlineto{\pgfqpoint{1.168860in}{2.311377in}}%
\pgfpathlineto{\pgfqpoint{1.156123in}{2.296491in}}%
\pgfpathlineto{\pgfqpoint{1.136496in}{2.277193in}}%
\pgfpathlineto{\pgfqpoint{1.112364in}{2.257895in}}%
\pgfpathlineto{\pgfqpoint{1.091165in}{2.244083in}}%
\pgfpathlineto{\pgfqpoint{1.071742in}{2.233532in}}%
\pgfpathlineto{\pgfqpoint{1.042607in}{2.220962in}}%
\pgfpathlineto{\pgfqpoint{1.023183in}{2.214499in}}%
\pgfpathlineto{\pgfqpoint{0.994048in}{2.207416in}}%
\pgfpathlineto{\pgfqpoint{0.964912in}{2.203223in}}%
\pgfpathlineto{\pgfqpoint{0.935777in}{2.201794in}}%
\pgfpathlineto{\pgfqpoint{0.906642in}{2.203027in}}%
\pgfpathlineto{\pgfqpoint{0.877506in}{2.206967in}}%
\pgfpathlineto{\pgfqpoint{0.848371in}{2.213673in}}%
\pgfpathlineto{\pgfqpoint{0.819236in}{2.223342in}}%
\pgfpathlineto{\pgfqpoint{0.790100in}{2.236283in}}%
\pgfpathlineto{\pgfqpoint{0.760965in}{2.252898in}}%
\pgfpathlineto{\pgfqpoint{0.731830in}{2.274054in}}%
\pgfpathlineto{\pgfqpoint{0.702694in}{2.301051in}}%
\pgfpathlineto{\pgfqpoint{0.682073in}{2.325439in}}%
\pgfpathlineto{\pgfqpoint{0.663847in}{2.351744in}}%
\pgfpathlineto{\pgfqpoint{0.643260in}{2.392982in}}%
\pgfpathlineto{\pgfqpoint{0.633398in}{2.421930in}}%
\pgfpathlineto{\pgfqpoint{0.627488in}{2.450877in}}%
\pgfpathlineto{\pgfqpoint{0.625000in}{2.470595in}}%
\pgfpathlineto{\pgfqpoint{0.625000in}{2.470595in}}%
\pgfusepath{stroke}%
\end{pgfscope}%
\begin{pgfscope}%
\pgfpathrectangle{\pgfqpoint{0.625000in}{0.550000in}}{\pgfqpoint{3.875000in}{3.850000in}} %
\pgfusepath{clip}%
\pgfsetbuttcap%
\pgfsetroundjoin%
\pgfsetlinewidth{0.250937pt}%
\definecolor{currentstroke}{rgb}{0.000000,0.000000,0.000000}%
\pgfsetstrokecolor{currentstroke}%
\pgfsetdash{}{0pt}%
\pgfpathmoveto{\pgfqpoint{0.625000in}{2.643027in}}%
\pgfpathlineto{\pgfqpoint{0.633796in}{2.634211in}}%
\pgfpathlineto{\pgfqpoint{0.625000in}{2.625972in}}%
\pgfusepath{stroke}%
\end{pgfscope}%
\begin{pgfscope}%
\pgfpathrectangle{\pgfqpoint{0.625000in}{0.550000in}}{\pgfqpoint{3.875000in}{3.850000in}} %
\pgfusepath{clip}%
\pgfsetbuttcap%
\pgfsetroundjoin%
\pgfsetlinewidth{0.250937pt}%
\definecolor{currentstroke}{rgb}{0.000000,0.000000,0.000000}%
\pgfsetstrokecolor{currentstroke}%
\pgfsetdash{}{0pt}%
\pgfpathmoveto{\pgfqpoint{0.625000in}{2.797106in}}%
\pgfpathlineto{\pgfqpoint{0.633750in}{2.788596in}}%
\pgfpathlineto{\pgfqpoint{0.625000in}{2.780325in}}%
\pgfusepath{stroke}%
\end{pgfscope}%
\begin{pgfscope}%
\pgfpathrectangle{\pgfqpoint{0.625000in}{0.550000in}}{\pgfqpoint{3.875000in}{3.850000in}} %
\pgfusepath{clip}%
\pgfsetbuttcap%
\pgfsetroundjoin%
\pgfsetlinewidth{0.250937pt}%
\definecolor{currentstroke}{rgb}{0.000000,0.000000,0.000000}%
\pgfsetstrokecolor{currentstroke}%
\pgfsetdash{}{0pt}%
\pgfpathmoveto{\pgfqpoint{0.625000in}{2.951617in}}%
\pgfpathlineto{\pgfqpoint{0.633715in}{2.942982in}}%
\pgfpathlineto{\pgfqpoint{0.625000in}{2.934275in}}%
\pgfusepath{stroke}%
\end{pgfscope}%
\begin{pgfscope}%
\pgfpathrectangle{\pgfqpoint{0.625000in}{0.550000in}}{\pgfqpoint{3.875000in}{3.850000in}} %
\pgfusepath{clip}%
\pgfsetbuttcap%
\pgfsetroundjoin%
\pgfsetlinewidth{0.250937pt}%
\definecolor{currentstroke}{rgb}{0.000000,0.000000,0.000000}%
\pgfsetstrokecolor{currentstroke}%
\pgfsetdash{}{0pt}%
\pgfpathmoveto{\pgfqpoint{0.625000in}{3.105596in}}%
\pgfpathlineto{\pgfqpoint{0.633725in}{3.097368in}}%
\pgfpathlineto{\pgfqpoint{0.625000in}{3.088965in}}%
\pgfusepath{stroke}%
\end{pgfscope}%
\begin{pgfscope}%
\pgfpathrectangle{\pgfqpoint{0.625000in}{0.550000in}}{\pgfqpoint{3.875000in}{3.850000in}} %
\pgfusepath{clip}%
\pgfsetbuttcap%
\pgfsetroundjoin%
\pgfsetlinewidth{0.250937pt}%
\definecolor{currentstroke}{rgb}{0.000000,0.000000,0.000000}%
\pgfsetstrokecolor{currentstroke}%
\pgfsetdash{}{0pt}%
\pgfpathmoveto{\pgfqpoint{0.625000in}{3.260092in}}%
\pgfpathlineto{\pgfqpoint{0.634712in}{3.259867in}}%
\pgfpathlineto{\pgfqpoint{0.643383in}{3.251754in}}%
\pgfpathlineto{\pgfqpoint{0.634712in}{3.243642in}}%
\pgfpathlineto{\pgfqpoint{0.625000in}{3.243472in}}%
\pgfusepath{stroke}%
\end{pgfscope}%
\begin{pgfscope}%
\pgfpathrectangle{\pgfqpoint{0.625000in}{0.550000in}}{\pgfqpoint{3.875000in}{3.850000in}} %
\pgfusepath{clip}%
\pgfsetbuttcap%
\pgfsetroundjoin%
\pgfsetlinewidth{0.250937pt}%
\definecolor{currentstroke}{rgb}{0.000000,0.000000,0.000000}%
\pgfsetstrokecolor{currentstroke}%
\pgfsetdash{}{0pt}%
\pgfpathmoveto{\pgfqpoint{0.625000in}{3.414719in}}%
\pgfpathlineto{\pgfqpoint{0.633724in}{3.406140in}}%
\pgfpathlineto{\pgfqpoint{0.625000in}{3.397841in}}%
\pgfusepath{stroke}%
\end{pgfscope}%
\begin{pgfscope}%
\pgfpathrectangle{\pgfqpoint{0.625000in}{0.550000in}}{\pgfqpoint{3.875000in}{3.850000in}} %
\pgfusepath{clip}%
\pgfsetbuttcap%
\pgfsetroundjoin%
\pgfsetlinewidth{0.250937pt}%
\definecolor{currentstroke}{rgb}{0.000000,0.000000,0.000000}%
\pgfsetstrokecolor{currentstroke}%
\pgfsetdash{}{0pt}%
\pgfpathmoveto{\pgfqpoint{0.625000in}{3.568855in}}%
\pgfpathlineto{\pgfqpoint{0.633766in}{3.560526in}}%
\pgfpathlineto{\pgfqpoint{0.625000in}{3.552202in}}%
\pgfusepath{stroke}%
\end{pgfscope}%
\begin{pgfscope}%
\pgfpathrectangle{\pgfqpoint{0.625000in}{0.550000in}}{\pgfqpoint{3.875000in}{3.850000in}} %
\pgfusepath{clip}%
\pgfsetbuttcap%
\pgfsetroundjoin%
\pgfsetlinewidth{0.250937pt}%
\definecolor{currentstroke}{rgb}{0.000000,0.000000,0.000000}%
\pgfsetstrokecolor{currentstroke}%
\pgfsetdash{}{0pt}%
\pgfpathmoveto{\pgfqpoint{0.625000in}{3.723325in}}%
\pgfpathlineto{\pgfqpoint{0.633785in}{3.714912in}}%
\pgfpathlineto{\pgfqpoint{0.625000in}{3.706246in}}%
\pgfusepath{stroke}%
\end{pgfscope}%
\begin{pgfscope}%
\pgfpathrectangle{\pgfqpoint{0.625000in}{0.550000in}}{\pgfqpoint{3.875000in}{3.850000in}} %
\pgfusepath{clip}%
\pgfsetbuttcap%
\pgfsetroundjoin%
\pgfsetlinewidth{0.250937pt}%
\definecolor{currentstroke}{rgb}{0.000000,0.000000,0.000000}%
\pgfsetstrokecolor{currentstroke}%
\pgfsetdash{}{0pt}%
\pgfpathmoveto{\pgfqpoint{0.625000in}{3.877598in}}%
\pgfpathlineto{\pgfqpoint{0.633722in}{3.869298in}}%
\pgfpathlineto{\pgfqpoint{0.625000in}{3.860965in}}%
\pgfusepath{stroke}%
\end{pgfscope}%
\begin{pgfscope}%
\pgfpathrectangle{\pgfqpoint{0.625000in}{0.550000in}}{\pgfqpoint{3.875000in}{3.850000in}} %
\pgfusepath{clip}%
\pgfsetbuttcap%
\pgfsetroundjoin%
\pgfsetlinewidth{0.250937pt}%
\definecolor{currentstroke}{rgb}{0.000000,0.000000,0.000000}%
\pgfsetstrokecolor{currentstroke}%
\pgfsetdash{}{0pt}%
\pgfpathmoveto{\pgfqpoint{0.625000in}{4.032089in}}%
\pgfpathlineto{\pgfqpoint{0.634712in}{4.031797in}}%
\pgfpathlineto{\pgfqpoint{0.643383in}{4.023684in}}%
\pgfpathlineto{\pgfqpoint{0.634712in}{4.015572in}}%
\pgfpathlineto{\pgfqpoint{0.625000in}{4.015362in}}%
\pgfusepath{stroke}%
\end{pgfscope}%
\begin{pgfscope}%
\pgfpathrectangle{\pgfqpoint{0.625000in}{0.550000in}}{\pgfqpoint{3.875000in}{3.850000in}} %
\pgfusepath{clip}%
\pgfsetbuttcap%
\pgfsetroundjoin%
\pgfsetlinewidth{0.250937pt}%
\definecolor{currentstroke}{rgb}{0.000000,0.000000,0.000000}%
\pgfsetstrokecolor{currentstroke}%
\pgfsetdash{}{0pt}%
\pgfpathmoveto{\pgfqpoint{0.625000in}{4.186468in}}%
\pgfpathlineto{\pgfqpoint{0.633723in}{4.178070in}}%
\pgfpathlineto{\pgfqpoint{0.625000in}{4.170066in}}%
\pgfusepath{stroke}%
\end{pgfscope}%
\begin{pgfscope}%
\pgfpathrectangle{\pgfqpoint{0.625000in}{0.550000in}}{\pgfqpoint{3.875000in}{3.850000in}} %
\pgfusepath{clip}%
\pgfsetbuttcap%
\pgfsetroundjoin%
\pgfsetlinewidth{0.250937pt}%
\definecolor{currentstroke}{rgb}{0.000000,0.000000,0.000000}%
\pgfsetstrokecolor{currentstroke}%
\pgfsetdash{}{0pt}%
\pgfpathmoveto{\pgfqpoint{0.625000in}{4.340807in}}%
\pgfpathlineto{\pgfqpoint{0.633675in}{4.332456in}}%
\pgfpathlineto{\pgfqpoint{0.625000in}{4.324077in}}%
\pgfusepath{stroke}%
\end{pgfscope}%
\begin{pgfscope}%
\pgfpathrectangle{\pgfqpoint{0.625000in}{0.550000in}}{\pgfqpoint{3.875000in}{3.850000in}} %
\pgfusepath{clip}%
\pgfsetbuttcap%
\pgfsetroundjoin%
\pgfsetlinewidth{0.250937pt}%
\definecolor{currentstroke}{rgb}{0.000000,0.000000,0.000000}%
\pgfsetstrokecolor{currentstroke}%
\pgfsetdash{}{0pt}%
\pgfpathmoveto{\pgfqpoint{0.634712in}{1.173340in}}%
\pgfpathlineto{\pgfqpoint{0.632736in}{1.177193in}}%
\pgfpathlineto{\pgfqpoint{0.627953in}{1.186842in}}%
\pgfpathlineto{\pgfqpoint{0.626547in}{1.196491in}}%
\pgfpathlineto{\pgfqpoint{0.630922in}{1.206140in}}%
\pgfpathlineto{\pgfqpoint{0.634712in}{1.213297in}}%
\pgfpathlineto{\pgfqpoint{0.642502in}{1.215789in}}%
\pgfpathlineto{\pgfqpoint{0.644424in}{1.217420in}}%
\pgfpathlineto{\pgfqpoint{0.654135in}{1.218354in}}%
\pgfpathlineto{\pgfqpoint{0.663090in}{1.215789in}}%
\pgfpathlineto{\pgfqpoint{0.663847in}{1.215464in}}%
\pgfpathlineto{\pgfqpoint{0.673559in}{1.207153in}}%
\pgfpathlineto{\pgfqpoint{0.674475in}{1.206140in}}%
\pgfpathlineto{\pgfqpoint{0.678331in}{1.196491in}}%
\pgfpathlineto{\pgfqpoint{0.677636in}{1.186842in}}%
\pgfpathlineto{\pgfqpoint{0.673559in}{1.179743in}}%
\pgfpathlineto{\pgfqpoint{0.672131in}{1.177193in}}%
\pgfpathlineto{\pgfqpoint{0.663847in}{1.171091in}}%
\pgfpathlineto{\pgfqpoint{0.654135in}{1.168392in}}%
\pgfpathlineto{\pgfqpoint{0.644424in}{1.169077in}}%
\pgfpathlineto{\pgfqpoint{0.634712in}{1.173340in}}%
\pgfusepath{stroke}%
\end{pgfscope}%
\begin{pgfscope}%
\pgfpathrectangle{\pgfqpoint{0.625000in}{0.550000in}}{\pgfqpoint{3.875000in}{3.850000in}} %
\pgfusepath{clip}%
\pgfsetbuttcap%
\pgfsetroundjoin%
\pgfsetlinewidth{0.250937pt}%
\definecolor{currentstroke}{rgb}{0.000000,0.000000,0.000000}%
\pgfsetstrokecolor{currentstroke}%
\pgfsetdash{}{0pt}%
\pgfpathmoveto{\pgfqpoint{0.644424in}{3.742229in}}%
\pgfpathlineto{\pgfqpoint{0.642502in}{3.743860in}}%
\pgfpathlineto{\pgfqpoint{0.634712in}{3.746352in}}%
\pgfpathlineto{\pgfqpoint{0.630922in}{3.753509in}}%
\pgfpathlineto{\pgfqpoint{0.626547in}{3.763158in}}%
\pgfpathlineto{\pgfqpoint{0.627953in}{3.772807in}}%
\pgfpathlineto{\pgfqpoint{0.632736in}{3.782456in}}%
\pgfpathlineto{\pgfqpoint{0.634712in}{3.786309in}}%
\pgfpathlineto{\pgfqpoint{0.644424in}{3.790573in}}%
\pgfpathlineto{\pgfqpoint{0.654135in}{3.791257in}}%
\pgfpathlineto{\pgfqpoint{0.663847in}{3.788558in}}%
\pgfpathlineto{\pgfqpoint{0.672131in}{3.782456in}}%
\pgfpathlineto{\pgfqpoint{0.673559in}{3.779906in}}%
\pgfpathlineto{\pgfqpoint{0.677636in}{3.772807in}}%
\pgfpathlineto{\pgfqpoint{0.678331in}{3.763158in}}%
\pgfpathlineto{\pgfqpoint{0.674475in}{3.753509in}}%
\pgfpathlineto{\pgfqpoint{0.673559in}{3.752497in}}%
\pgfpathlineto{\pgfqpoint{0.663847in}{3.744186in}}%
\pgfpathlineto{\pgfqpoint{0.663090in}{3.743860in}}%
\pgfpathlineto{\pgfqpoint{0.654135in}{3.741295in}}%
\pgfpathlineto{\pgfqpoint{0.644424in}{3.742229in}}%
\pgfusepath{stroke}%
\end{pgfscope}%
\begin{pgfscope}%
\pgfpathrectangle{\pgfqpoint{0.625000in}{0.550000in}}{\pgfqpoint{3.875000in}{3.850000in}} %
\pgfusepath{clip}%
\pgfsetbuttcap%
\pgfsetroundjoin%
\pgfsetlinewidth{0.250937pt}%
\definecolor{currentstroke}{rgb}{0.000000,0.000000,0.000000}%
\pgfsetstrokecolor{currentstroke}%
\pgfsetdash{}{0pt}%
\pgfpathmoveto{\pgfqpoint{0.625000in}{0.635816in}}%
\pgfpathlineto{\pgfqpoint{0.633894in}{0.627193in}}%
\pgfpathlineto{\pgfqpoint{0.625000in}{0.618595in}}%
\pgfusepath{stroke}%
\end{pgfscope}%
\begin{pgfscope}%
\pgfpathrectangle{\pgfqpoint{0.625000in}{0.550000in}}{\pgfqpoint{3.875000in}{3.850000in}} %
\pgfusepath{clip}%
\pgfsetbuttcap%
\pgfsetroundjoin%
\pgfsetlinewidth{0.250937pt}%
\definecolor{currentstroke}{rgb}{0.000000,0.000000,0.000000}%
\pgfsetstrokecolor{currentstroke}%
\pgfsetdash{}{0pt}%
\pgfpathmoveto{\pgfqpoint{0.625000in}{0.789807in}}%
\pgfpathlineto{\pgfqpoint{0.633902in}{0.781579in}}%
\pgfpathlineto{\pgfqpoint{0.625000in}{0.772984in}}%
\pgfusepath{stroke}%
\end{pgfscope}%
\begin{pgfscope}%
\pgfpathrectangle{\pgfqpoint{0.625000in}{0.550000in}}{\pgfqpoint{3.875000in}{3.850000in}} %
\pgfusepath{clip}%
\pgfsetbuttcap%
\pgfsetroundjoin%
\pgfsetlinewidth{0.250937pt}%
\definecolor{currentstroke}{rgb}{0.000000,0.000000,0.000000}%
\pgfsetstrokecolor{currentstroke}%
\pgfsetdash{}{0pt}%
\pgfpathmoveto{\pgfqpoint{0.625000in}{0.944477in}}%
\pgfpathlineto{\pgfqpoint{0.634712in}{0.944890in}}%
\pgfpathlineto{\pgfqpoint{0.644252in}{0.935965in}}%
\pgfpathlineto{\pgfqpoint{0.634712in}{0.927040in}}%
\pgfpathlineto{\pgfqpoint{0.625000in}{0.927377in}}%
\pgfusepath{stroke}%
\end{pgfscope}%
\begin{pgfscope}%
\pgfpathrectangle{\pgfqpoint{0.625000in}{0.550000in}}{\pgfqpoint{3.875000in}{3.850000in}} %
\pgfusepath{clip}%
\pgfsetbuttcap%
\pgfsetroundjoin%
\pgfsetlinewidth{0.250937pt}%
\definecolor{currentstroke}{rgb}{0.000000,0.000000,0.000000}%
\pgfsetstrokecolor{currentstroke}%
\pgfsetdash{}{0pt}%
\pgfpathmoveto{\pgfqpoint{0.625000in}{1.098884in}}%
\pgfpathlineto{\pgfqpoint{0.633898in}{1.090351in}}%
\pgfpathlineto{\pgfqpoint{0.625000in}{1.081850in}}%
\pgfusepath{stroke}%
\end{pgfscope}%
\begin{pgfscope}%
\pgfpathrectangle{\pgfqpoint{0.625000in}{0.550000in}}{\pgfqpoint{3.875000in}{3.850000in}} %
\pgfusepath{clip}%
\pgfsetbuttcap%
\pgfsetroundjoin%
\pgfsetlinewidth{0.250937pt}%
\definecolor{currentstroke}{rgb}{0.000000,0.000000,0.000000}%
\pgfsetstrokecolor{currentstroke}%
\pgfsetdash{}{0pt}%
\pgfpathmoveto{\pgfqpoint{0.625000in}{1.253544in}}%
\pgfpathlineto{\pgfqpoint{0.633924in}{1.244737in}}%
\pgfpathlineto{\pgfqpoint{0.625000in}{1.236173in}}%
\pgfusepath{stroke}%
\end{pgfscope}%
\begin{pgfscope}%
\pgfpathrectangle{\pgfqpoint{0.625000in}{0.550000in}}{\pgfqpoint{3.875000in}{3.850000in}} %
\pgfusepath{clip}%
\pgfsetbuttcap%
\pgfsetroundjoin%
\pgfsetlinewidth{0.250937pt}%
\definecolor{currentstroke}{rgb}{0.000000,0.000000,0.000000}%
\pgfsetstrokecolor{currentstroke}%
\pgfsetdash{}{0pt}%
\pgfpathmoveto{\pgfqpoint{0.625000in}{1.407579in}}%
\pgfpathlineto{\pgfqpoint{0.633889in}{1.399123in}}%
\pgfpathlineto{\pgfqpoint{0.625000in}{1.390662in}}%
\pgfusepath{stroke}%
\end{pgfscope}%
\begin{pgfscope}%
\pgfpathrectangle{\pgfqpoint{0.625000in}{0.550000in}}{\pgfqpoint{3.875000in}{3.850000in}} %
\pgfusepath{clip}%
\pgfsetbuttcap%
\pgfsetroundjoin%
\pgfsetlinewidth{0.250937pt}%
\definecolor{currentstroke}{rgb}{0.000000,0.000000,0.000000}%
\pgfsetstrokecolor{currentstroke}%
\pgfsetdash{}{0pt}%
\pgfpathmoveto{\pgfqpoint{0.625000in}{1.561932in}}%
\pgfpathlineto{\pgfqpoint{0.633841in}{1.553509in}}%
\pgfpathlineto{\pgfqpoint{0.625000in}{1.544812in}}%
\pgfusepath{stroke}%
\end{pgfscope}%
\begin{pgfscope}%
\pgfpathrectangle{\pgfqpoint{0.625000in}{0.550000in}}{\pgfqpoint{3.875000in}{3.850000in}} %
\pgfusepath{clip}%
\pgfsetbuttcap%
\pgfsetroundjoin%
\pgfsetlinewidth{0.250937pt}%
\definecolor{currentstroke}{rgb}{0.000000,0.000000,0.000000}%
\pgfsetstrokecolor{currentstroke}%
\pgfsetdash{}{0pt}%
\pgfpathmoveto{\pgfqpoint{0.625000in}{1.716321in}}%
\pgfpathlineto{\pgfqpoint{0.634712in}{1.716820in}}%
\pgfpathlineto{\pgfqpoint{0.644252in}{1.707895in}}%
\pgfpathlineto{\pgfqpoint{0.634712in}{1.698970in}}%
\pgfpathlineto{\pgfqpoint{0.625000in}{1.699415in}}%
\pgfusepath{stroke}%
\end{pgfscope}%
\begin{pgfscope}%
\pgfpathrectangle{\pgfqpoint{0.625000in}{0.550000in}}{\pgfqpoint{3.875000in}{3.850000in}} %
\pgfusepath{clip}%
\pgfsetbuttcap%
\pgfsetroundjoin%
\pgfsetlinewidth{0.250937pt}%
\definecolor{currentstroke}{rgb}{0.000000,0.000000,0.000000}%
\pgfsetstrokecolor{currentstroke}%
\pgfsetdash{}{0pt}%
\pgfpathmoveto{\pgfqpoint{0.625000in}{1.870784in}}%
\pgfpathlineto{\pgfqpoint{0.633825in}{1.862281in}}%
\pgfpathlineto{\pgfqpoint{0.625000in}{1.853953in}}%
\pgfusepath{stroke}%
\end{pgfscope}%
\begin{pgfscope}%
\pgfpathrectangle{\pgfqpoint{0.625000in}{0.550000in}}{\pgfqpoint{3.875000in}{3.850000in}} %
\pgfusepath{clip}%
\pgfsetbuttcap%
\pgfsetroundjoin%
\pgfsetlinewidth{0.250937pt}%
\definecolor{currentstroke}{rgb}{0.000000,0.000000,0.000000}%
\pgfsetstrokecolor{currentstroke}%
\pgfsetdash{}{0pt}%
\pgfpathmoveto{\pgfqpoint{0.625000in}{2.025477in}}%
\pgfpathlineto{\pgfqpoint{0.633819in}{2.016667in}}%
\pgfpathlineto{\pgfqpoint{0.625000in}{2.007928in}}%
\pgfusepath{stroke}%
\end{pgfscope}%
\begin{pgfscope}%
\pgfpathrectangle{\pgfqpoint{0.625000in}{0.550000in}}{\pgfqpoint{3.875000in}{3.850000in}} %
\pgfusepath{clip}%
\pgfsetbuttcap%
\pgfsetroundjoin%
\pgfsetlinewidth{0.250937pt}%
\definecolor{currentstroke}{rgb}{0.000000,0.000000,0.000000}%
\pgfsetstrokecolor{currentstroke}%
\pgfsetdash{}{0pt}%
\pgfpathmoveto{\pgfqpoint{0.625000in}{2.179414in}}%
\pgfpathlineto{\pgfqpoint{0.633843in}{2.171053in}}%
\pgfpathlineto{\pgfqpoint{0.625000in}{2.162452in}}%
\pgfusepath{stroke}%
\end{pgfscope}%
\begin{pgfscope}%
\pgfpathrectangle{\pgfqpoint{0.625000in}{0.550000in}}{\pgfqpoint{3.875000in}{3.850000in}} %
\pgfusepath{clip}%
\pgfsetbuttcap%
\pgfsetroundjoin%
\pgfsetlinewidth{0.250937pt}%
\definecolor{currentstroke}{rgb}{0.000000,0.000000,0.000000}%
\pgfsetstrokecolor{currentstroke}%
\pgfsetdash{}{0pt}%
\pgfpathmoveto{\pgfqpoint{0.625000in}{2.333762in}}%
\pgfpathlineto{\pgfqpoint{0.633886in}{2.325439in}}%
\pgfpathlineto{\pgfqpoint{0.625000in}{2.316532in}}%
\pgfusepath{stroke}%
\end{pgfscope}%
\begin{pgfscope}%
\pgfpathrectangle{\pgfqpoint{0.625000in}{0.550000in}}{\pgfqpoint{3.875000in}{3.850000in}} %
\pgfusepath{clip}%
\pgfsetbuttcap%
\pgfsetroundjoin%
\pgfsetlinewidth{0.250937pt}%
\definecolor{currentstroke}{rgb}{0.000000,0.000000,0.000000}%
\pgfsetstrokecolor{currentstroke}%
\pgfsetdash{}{0pt}%
\pgfpathmoveto{\pgfqpoint{0.625000in}{2.489138in}}%
\pgfpathlineto{\pgfqpoint{0.625315in}{2.489474in}}%
\pgfpathlineto{\pgfqpoint{0.627333in}{2.508772in}}%
\pgfpathlineto{\pgfqpoint{0.632807in}{2.537719in}}%
\pgfpathlineto{\pgfqpoint{0.638580in}{2.557018in}}%
\pgfpathlineto{\pgfqpoint{0.645609in}{2.576316in}}%
\pgfpathlineto{\pgfqpoint{0.659250in}{2.605263in}}%
\pgfpathlineto{\pgfqpoint{0.673559in}{2.630024in}}%
\pgfpathlineto{\pgfqpoint{0.702694in}{2.668588in}}%
\pgfpathlineto{\pgfqpoint{0.731830in}{2.698277in}}%
\pgfpathlineto{\pgfqpoint{0.760965in}{2.722000in}}%
\pgfpathlineto{\pgfqpoint{0.788612in}{2.740351in}}%
\pgfpathlineto{\pgfqpoint{0.799812in}{2.746967in}}%
\pgfpathlineto{\pgfqpoint{0.828947in}{2.761523in}}%
\pgfpathlineto{\pgfqpoint{0.858083in}{2.773117in}}%
\pgfpathlineto{\pgfqpoint{0.887218in}{2.781968in}}%
\pgfpathlineto{\pgfqpoint{0.918433in}{2.788596in}}%
\pgfpathlineto{\pgfqpoint{0.945489in}{2.792252in}}%
\pgfpathlineto{\pgfqpoint{0.974624in}{2.793860in}}%
\pgfpathlineto{\pgfqpoint{1.003759in}{2.793144in}}%
\pgfpathlineto{\pgfqpoint{1.032895in}{2.790059in}}%
\pgfpathlineto{\pgfqpoint{1.062030in}{2.784547in}}%
\pgfpathlineto{\pgfqpoint{1.091165in}{2.776424in}}%
\pgfpathlineto{\pgfqpoint{1.120301in}{2.765468in}}%
\pgfpathlineto{\pgfqpoint{1.149436in}{2.751318in}}%
\pgfpathlineto{\pgfqpoint{1.168860in}{2.739855in}}%
\pgfpathlineto{\pgfqpoint{1.195446in}{2.721053in}}%
\pgfpathlineto{\pgfqpoint{1.217856in}{2.701754in}}%
\pgfpathlineto{\pgfqpoint{1.236842in}{2.682248in}}%
\pgfpathlineto{\pgfqpoint{1.256266in}{2.658217in}}%
\pgfpathlineto{\pgfqpoint{1.266195in}{2.643860in}}%
\pgfpathlineto{\pgfqpoint{1.277758in}{2.624561in}}%
\pgfpathlineto{\pgfqpoint{1.287496in}{2.605263in}}%
\pgfpathlineto{\pgfqpoint{1.299088in}{2.576316in}}%
\pgfpathlineto{\pgfqpoint{1.307372in}{2.547368in}}%
\pgfpathlineto{\pgfqpoint{1.312619in}{2.518421in}}%
\pgfpathlineto{\pgfqpoint{1.314960in}{2.489474in}}%
\pgfpathlineto{\pgfqpoint{1.314484in}{2.460526in}}%
\pgfpathlineto{\pgfqpoint{1.311205in}{2.431579in}}%
\pgfpathlineto{\pgfqpoint{1.304825in}{2.402273in}}%
\pgfpathlineto{\pgfqpoint{1.299088in}{2.383333in}}%
\pgfpathlineto{\pgfqpoint{1.287496in}{2.354386in}}%
\pgfpathlineto{\pgfqpoint{1.275689in}{2.331483in}}%
\pgfpathlineto{\pgfqpoint{1.265977in}{2.315465in}}%
\pgfpathlineto{\pgfqpoint{1.252640in}{2.296491in}}%
\pgfpathlineto{\pgfqpoint{1.236661in}{2.277193in}}%
\pgfpathlineto{\pgfqpoint{1.217419in}{2.257491in}}%
\pgfpathlineto{\pgfqpoint{1.195446in}{2.238596in}}%
\pgfpathlineto{\pgfqpoint{1.168860in}{2.219794in}}%
\pgfpathlineto{\pgfqpoint{1.149436in}{2.208331in}}%
\pgfpathlineto{\pgfqpoint{1.120301in}{2.194181in}}%
\pgfpathlineto{\pgfqpoint{1.091165in}{2.183225in}}%
\pgfpathlineto{\pgfqpoint{1.062030in}{2.175102in}}%
\pgfpathlineto{\pgfqpoint{1.032895in}{2.169590in}}%
\pgfpathlineto{\pgfqpoint{1.003759in}{2.166505in}}%
\pgfpathlineto{\pgfqpoint{0.974624in}{2.165789in}}%
\pgfpathlineto{\pgfqpoint{0.945489in}{2.167397in}}%
\pgfpathlineto{\pgfqpoint{0.916353in}{2.171376in}}%
\pgfpathlineto{\pgfqpoint{0.887218in}{2.177681in}}%
\pgfpathlineto{\pgfqpoint{0.858083in}{2.186532in}}%
\pgfpathlineto{\pgfqpoint{0.828947in}{2.198126in}}%
\pgfpathlineto{\pgfqpoint{0.799812in}{2.212682in}}%
\pgfpathlineto{\pgfqpoint{0.770677in}{2.230733in}}%
\pgfpathlineto{\pgfqpoint{0.747412in}{2.248246in}}%
\pgfpathlineto{\pgfqpoint{0.731830in}{2.261372in}}%
\pgfpathlineto{\pgfqpoint{0.702694in}{2.291061in}}%
\pgfpathlineto{\pgfqpoint{0.683185in}{2.315789in}}%
\pgfpathlineto{\pgfqpoint{0.663847in}{2.345691in}}%
\pgfpathlineto{\pgfqpoint{0.649810in}{2.373684in}}%
\pgfpathlineto{\pgfqpoint{0.641903in}{2.392982in}}%
\pgfpathlineto{\pgfqpoint{0.632807in}{2.421930in}}%
\pgfpathlineto{\pgfqpoint{0.627333in}{2.450877in}}%
\pgfpathlineto{\pgfqpoint{0.625000in}{2.470511in}}%
\pgfpathlineto{\pgfqpoint{0.625000in}{2.470511in}}%
\pgfusepath{stroke}%
\end{pgfscope}%
\begin{pgfscope}%
\pgfpathrectangle{\pgfqpoint{0.625000in}{0.550000in}}{\pgfqpoint{3.875000in}{3.850000in}} %
\pgfusepath{clip}%
\pgfsetbuttcap%
\pgfsetroundjoin%
\pgfsetlinewidth{0.250937pt}%
\definecolor{currentstroke}{rgb}{0.000000,0.000000,0.000000}%
\pgfsetstrokecolor{currentstroke}%
\pgfsetdash{}{0pt}%
\pgfpathmoveto{\pgfqpoint{0.625000in}{2.643116in}}%
\pgfpathlineto{\pgfqpoint{0.633885in}{2.634211in}}%
\pgfpathlineto{\pgfqpoint{0.625000in}{2.625889in}}%
\pgfusepath{stroke}%
\end{pgfscope}%
\begin{pgfscope}%
\pgfpathrectangle{\pgfqpoint{0.625000in}{0.550000in}}{\pgfqpoint{3.875000in}{3.850000in}} %
\pgfusepath{clip}%
\pgfsetbuttcap%
\pgfsetroundjoin%
\pgfsetlinewidth{0.250937pt}%
\definecolor{currentstroke}{rgb}{0.000000,0.000000,0.000000}%
\pgfsetstrokecolor{currentstroke}%
\pgfsetdash{}{0pt}%
\pgfpathmoveto{\pgfqpoint{0.625000in}{2.797191in}}%
\pgfpathlineto{\pgfqpoint{0.633837in}{2.788596in}}%
\pgfpathlineto{\pgfqpoint{0.625000in}{2.780242in}}%
\pgfusepath{stroke}%
\end{pgfscope}%
\begin{pgfscope}%
\pgfpathrectangle{\pgfqpoint{0.625000in}{0.550000in}}{\pgfqpoint{3.875000in}{3.850000in}} %
\pgfusepath{clip}%
\pgfsetbuttcap%
\pgfsetroundjoin%
\pgfsetlinewidth{0.250937pt}%
\definecolor{currentstroke}{rgb}{0.000000,0.000000,0.000000}%
\pgfsetstrokecolor{currentstroke}%
\pgfsetdash{}{0pt}%
\pgfpathmoveto{\pgfqpoint{0.625000in}{2.951705in}}%
\pgfpathlineto{\pgfqpoint{0.633803in}{2.942982in}}%
\pgfpathlineto{\pgfqpoint{0.625000in}{2.934187in}}%
\pgfusepath{stroke}%
\end{pgfscope}%
\begin{pgfscope}%
\pgfpathrectangle{\pgfqpoint{0.625000in}{0.550000in}}{\pgfqpoint{3.875000in}{3.850000in}} %
\pgfusepath{clip}%
\pgfsetbuttcap%
\pgfsetroundjoin%
\pgfsetlinewidth{0.250937pt}%
\definecolor{currentstroke}{rgb}{0.000000,0.000000,0.000000}%
\pgfsetstrokecolor{currentstroke}%
\pgfsetdash{}{0pt}%
\pgfpathmoveto{\pgfqpoint{0.625000in}{3.105678in}}%
\pgfpathlineto{\pgfqpoint{0.633813in}{3.097368in}}%
\pgfpathlineto{\pgfqpoint{0.625000in}{3.088881in}}%
\pgfusepath{stroke}%
\end{pgfscope}%
\begin{pgfscope}%
\pgfpathrectangle{\pgfqpoint{0.625000in}{0.550000in}}{\pgfqpoint{3.875000in}{3.850000in}} %
\pgfusepath{clip}%
\pgfsetbuttcap%
\pgfsetroundjoin%
\pgfsetlinewidth{0.250937pt}%
\definecolor{currentstroke}{rgb}{0.000000,0.000000,0.000000}%
\pgfsetstrokecolor{currentstroke}%
\pgfsetdash{}{0pt}%
\pgfpathmoveto{\pgfqpoint{0.625000in}{3.260177in}}%
\pgfpathlineto{\pgfqpoint{0.634712in}{3.260680in}}%
\pgfpathlineto{\pgfqpoint{0.644252in}{3.251754in}}%
\pgfpathlineto{\pgfqpoint{0.634712in}{3.242829in}}%
\pgfpathlineto{\pgfqpoint{0.625000in}{3.243387in}}%
\pgfusepath{stroke}%
\end{pgfscope}%
\begin{pgfscope}%
\pgfpathrectangle{\pgfqpoint{0.625000in}{0.550000in}}{\pgfqpoint{3.875000in}{3.850000in}} %
\pgfusepath{clip}%
\pgfsetbuttcap%
\pgfsetroundjoin%
\pgfsetlinewidth{0.250937pt}%
\definecolor{currentstroke}{rgb}{0.000000,0.000000,0.000000}%
\pgfsetstrokecolor{currentstroke}%
\pgfsetdash{}{0pt}%
\pgfpathmoveto{\pgfqpoint{0.625000in}{3.414804in}}%
\pgfpathlineto{\pgfqpoint{0.633811in}{3.406140in}}%
\pgfpathlineto{\pgfqpoint{0.625000in}{3.397758in}}%
\pgfusepath{stroke}%
\end{pgfscope}%
\begin{pgfscope}%
\pgfpathrectangle{\pgfqpoint{0.625000in}{0.550000in}}{\pgfqpoint{3.875000in}{3.850000in}} %
\pgfusepath{clip}%
\pgfsetbuttcap%
\pgfsetroundjoin%
\pgfsetlinewidth{0.250937pt}%
\definecolor{currentstroke}{rgb}{0.000000,0.000000,0.000000}%
\pgfsetstrokecolor{currentstroke}%
\pgfsetdash{}{0pt}%
\pgfpathmoveto{\pgfqpoint{0.625000in}{3.568940in}}%
\pgfpathlineto{\pgfqpoint{0.633855in}{3.560526in}}%
\pgfpathlineto{\pgfqpoint{0.625000in}{3.552118in}}%
\pgfusepath{stroke}%
\end{pgfscope}%
\begin{pgfscope}%
\pgfpathrectangle{\pgfqpoint{0.625000in}{0.550000in}}{\pgfqpoint{3.875000in}{3.850000in}} %
\pgfusepath{clip}%
\pgfsetbuttcap%
\pgfsetroundjoin%
\pgfsetlinewidth{0.250937pt}%
\definecolor{currentstroke}{rgb}{0.000000,0.000000,0.000000}%
\pgfsetstrokecolor{currentstroke}%
\pgfsetdash{}{0pt}%
\pgfpathmoveto{\pgfqpoint{0.625000in}{3.723410in}}%
\pgfpathlineto{\pgfqpoint{0.633874in}{3.714912in}}%
\pgfpathlineto{\pgfqpoint{0.625000in}{3.706158in}}%
\pgfusepath{stroke}%
\end{pgfscope}%
\begin{pgfscope}%
\pgfpathrectangle{\pgfqpoint{0.625000in}{0.550000in}}{\pgfqpoint{3.875000in}{3.850000in}} %
\pgfusepath{clip}%
\pgfsetbuttcap%
\pgfsetroundjoin%
\pgfsetlinewidth{0.250937pt}%
\definecolor{currentstroke}{rgb}{0.000000,0.000000,0.000000}%
\pgfsetstrokecolor{currentstroke}%
\pgfsetdash{}{0pt}%
\pgfpathmoveto{\pgfqpoint{0.625000in}{3.877684in}}%
\pgfpathlineto{\pgfqpoint{0.633813in}{3.869298in}}%
\pgfpathlineto{\pgfqpoint{0.625000in}{3.860878in}}%
\pgfusepath{stroke}%
\end{pgfscope}%
\begin{pgfscope}%
\pgfpathrectangle{\pgfqpoint{0.625000in}{0.550000in}}{\pgfqpoint{3.875000in}{3.850000in}} %
\pgfusepath{clip}%
\pgfsetbuttcap%
\pgfsetroundjoin%
\pgfsetlinewidth{0.250937pt}%
\definecolor{currentstroke}{rgb}{0.000000,0.000000,0.000000}%
\pgfsetstrokecolor{currentstroke}%
\pgfsetdash{}{0pt}%
\pgfpathmoveto{\pgfqpoint{0.625000in}{4.032173in}}%
\pgfpathlineto{\pgfqpoint{0.634712in}{4.032609in}}%
\pgfpathlineto{\pgfqpoint{0.644252in}{4.023684in}}%
\pgfpathlineto{\pgfqpoint{0.634712in}{4.014759in}}%
\pgfpathlineto{\pgfqpoint{0.625000in}{4.015278in}}%
\pgfusepath{stroke}%
\end{pgfscope}%
\begin{pgfscope}%
\pgfpathrectangle{\pgfqpoint{0.625000in}{0.550000in}}{\pgfqpoint{3.875000in}{3.850000in}} %
\pgfusepath{clip}%
\pgfsetbuttcap%
\pgfsetroundjoin%
\pgfsetlinewidth{0.250937pt}%
\definecolor{currentstroke}{rgb}{0.000000,0.000000,0.000000}%
\pgfsetstrokecolor{currentstroke}%
\pgfsetdash{}{0pt}%
\pgfpathmoveto{\pgfqpoint{0.625000in}{4.186556in}}%
\pgfpathlineto{\pgfqpoint{0.633816in}{4.178070in}}%
\pgfpathlineto{\pgfqpoint{0.625000in}{4.169981in}}%
\pgfusepath{stroke}%
\end{pgfscope}%
\begin{pgfscope}%
\pgfpathrectangle{\pgfqpoint{0.625000in}{0.550000in}}{\pgfqpoint{3.875000in}{3.850000in}} %
\pgfusepath{clip}%
\pgfsetbuttcap%
\pgfsetroundjoin%
\pgfsetlinewidth{0.250937pt}%
\definecolor{currentstroke}{rgb}{0.000000,0.000000,0.000000}%
\pgfsetstrokecolor{currentstroke}%
\pgfsetdash{}{0pt}%
\pgfpathmoveto{\pgfqpoint{0.625000in}{4.340895in}}%
\pgfpathlineto{\pgfqpoint{0.633766in}{4.332456in}}%
\pgfpathlineto{\pgfqpoint{0.625000in}{4.323989in}}%
\pgfusepath{stroke}%
\end{pgfscope}%
\begin{pgfscope}%
\pgfpathrectangle{\pgfqpoint{0.625000in}{0.550000in}}{\pgfqpoint{3.875000in}{3.850000in}} %
\pgfusepath{clip}%
\pgfsetbuttcap%
\pgfsetroundjoin%
\pgfsetlinewidth{0.250937pt}%
\definecolor{currentstroke}{rgb}{0.000000,0.000000,0.000000}%
\pgfsetstrokecolor{currentstroke}%
\pgfsetdash{}{0pt}%
\pgfpathmoveto{\pgfqpoint{0.654135in}{1.165788in}}%
\pgfpathlineto{\pgfqpoint{0.646809in}{1.167544in}}%
\pgfpathlineto{\pgfqpoint{0.644424in}{1.167797in}}%
\pgfpathlineto{\pgfqpoint{0.634712in}{1.172281in}}%
\pgfpathlineto{\pgfqpoint{0.632193in}{1.177193in}}%
\pgfpathlineto{\pgfqpoint{0.627745in}{1.186842in}}%
\pgfpathlineto{\pgfqpoint{0.626372in}{1.196491in}}%
\pgfpathlineto{\pgfqpoint{0.630542in}{1.206140in}}%
\pgfpathlineto{\pgfqpoint{0.634712in}{1.214014in}}%
\pgfpathlineto{\pgfqpoint{0.640261in}{1.215789in}}%
\pgfpathlineto{\pgfqpoint{0.644424in}{1.219322in}}%
\pgfpathlineto{\pgfqpoint{0.654135in}{1.220924in}}%
\pgfpathlineto{\pgfqpoint{0.663847in}{1.219341in}}%
\pgfpathlineto{\pgfqpoint{0.671220in}{1.215789in}}%
\pgfpathlineto{\pgfqpoint{0.673559in}{1.213846in}}%
\pgfpathlineto{\pgfqpoint{0.680531in}{1.206140in}}%
\pgfpathlineto{\pgfqpoint{0.683271in}{1.196788in}}%
\pgfpathlineto{\pgfqpoint{0.683374in}{1.196491in}}%
\pgfpathlineto{\pgfqpoint{0.683271in}{1.195065in}}%
\pgfpathlineto{\pgfqpoint{0.682841in}{1.186842in}}%
\pgfpathlineto{\pgfqpoint{0.678437in}{1.177193in}}%
\pgfpathlineto{\pgfqpoint{0.673559in}{1.172722in}}%
\pgfpathlineto{\pgfqpoint{0.664412in}{1.167544in}}%
\pgfpathlineto{\pgfqpoint{0.663847in}{1.167334in}}%
\pgfpathlineto{\pgfqpoint{0.654135in}{1.165788in}}%
\pgfusepath{stroke}%
\end{pgfscope}%
\begin{pgfscope}%
\pgfpathrectangle{\pgfqpoint{0.625000in}{0.550000in}}{\pgfqpoint{3.875000in}{3.850000in}} %
\pgfusepath{clip}%
\pgfsetbuttcap%
\pgfsetroundjoin%
\pgfsetlinewidth{0.250937pt}%
\definecolor{currentstroke}{rgb}{0.000000,0.000000,0.000000}%
\pgfsetstrokecolor{currentstroke}%
\pgfsetdash{}{0pt}%
\pgfpathmoveto{\pgfqpoint{0.644424in}{3.740327in}}%
\pgfpathlineto{\pgfqpoint{0.640261in}{3.743860in}}%
\pgfpathlineto{\pgfqpoint{0.634712in}{3.745635in}}%
\pgfpathlineto{\pgfqpoint{0.630542in}{3.753509in}}%
\pgfpathlineto{\pgfqpoint{0.626372in}{3.763158in}}%
\pgfpathlineto{\pgfqpoint{0.627745in}{3.772807in}}%
\pgfpathlineto{\pgfqpoint{0.632193in}{3.782456in}}%
\pgfpathlineto{\pgfqpoint{0.634712in}{3.787368in}}%
\pgfpathlineto{\pgfqpoint{0.644424in}{3.791852in}}%
\pgfpathlineto{\pgfqpoint{0.646809in}{3.792105in}}%
\pgfpathlineto{\pgfqpoint{0.654135in}{3.793861in}}%
\pgfpathlineto{\pgfqpoint{0.663847in}{3.792316in}}%
\pgfpathlineto{\pgfqpoint{0.664412in}{3.792105in}}%
\pgfpathlineto{\pgfqpoint{0.673559in}{3.786927in}}%
\pgfpathlineto{\pgfqpoint{0.678437in}{3.782456in}}%
\pgfpathlineto{\pgfqpoint{0.682841in}{3.772807in}}%
\pgfpathlineto{\pgfqpoint{0.683271in}{3.764585in}}%
\pgfpathlineto{\pgfqpoint{0.683374in}{3.763158in}}%
\pgfpathlineto{\pgfqpoint{0.683271in}{3.762862in}}%
\pgfpathlineto{\pgfqpoint{0.680531in}{3.753509in}}%
\pgfpathlineto{\pgfqpoint{0.673559in}{3.745804in}}%
\pgfpathlineto{\pgfqpoint{0.671220in}{3.743860in}}%
\pgfpathlineto{\pgfqpoint{0.663847in}{3.740308in}}%
\pgfpathlineto{\pgfqpoint{0.654135in}{3.738725in}}%
\pgfpathlineto{\pgfqpoint{0.644424in}{3.740327in}}%
\pgfusepath{stroke}%
\end{pgfscope}%
\begin{pgfscope}%
\pgfpathrectangle{\pgfqpoint{0.625000in}{0.550000in}}{\pgfqpoint{3.875000in}{3.850000in}} %
\pgfusepath{clip}%
\pgfsetbuttcap%
\pgfsetroundjoin%
\pgfsetlinewidth{0.250937pt}%
\definecolor{currentstroke}{rgb}{0.000000,0.000000,0.000000}%
\pgfsetstrokecolor{currentstroke}%
\pgfsetdash{}{0pt}%
\pgfpathmoveto{\pgfqpoint{0.625000in}{0.635892in}}%
\pgfpathlineto{\pgfqpoint{0.633972in}{0.627193in}}%
\pgfpathlineto{\pgfqpoint{0.625000in}{0.618519in}}%
\pgfusepath{stroke}%
\end{pgfscope}%
\begin{pgfscope}%
\pgfpathrectangle{\pgfqpoint{0.625000in}{0.550000in}}{\pgfqpoint{3.875000in}{3.850000in}} %
\pgfusepath{clip}%
\pgfsetbuttcap%
\pgfsetroundjoin%
\pgfsetlinewidth{0.250937pt}%
\definecolor{currentstroke}{rgb}{0.000000,0.000000,0.000000}%
\pgfsetstrokecolor{currentstroke}%
\pgfsetdash{}{0pt}%
\pgfpathmoveto{\pgfqpoint{0.625000in}{0.789884in}}%
\pgfpathlineto{\pgfqpoint{0.633985in}{0.781579in}}%
\pgfpathlineto{\pgfqpoint{0.625000in}{0.772904in}}%
\pgfusepath{stroke}%
\end{pgfscope}%
\begin{pgfscope}%
\pgfpathrectangle{\pgfqpoint{0.625000in}{0.550000in}}{\pgfqpoint{3.875000in}{3.850000in}} %
\pgfusepath{clip}%
\pgfsetbuttcap%
\pgfsetroundjoin%
\pgfsetlinewidth{0.250937pt}%
\definecolor{currentstroke}{rgb}{0.000000,0.000000,0.000000}%
\pgfsetstrokecolor{currentstroke}%
\pgfsetdash{}{0pt}%
\pgfpathmoveto{\pgfqpoint{0.625000in}{0.944553in}}%
\pgfpathlineto{\pgfqpoint{0.634636in}{0.945614in}}%
\pgfpathlineto{\pgfqpoint{0.634712in}{0.945773in}}%
\pgfpathlineto{\pgfqpoint{0.635376in}{0.945614in}}%
\pgfpathlineto{\pgfqpoint{0.644424in}{0.939345in}}%
\pgfpathlineto{\pgfqpoint{0.646730in}{0.935965in}}%
\pgfpathlineto{\pgfqpoint{0.644424in}{0.932584in}}%
\pgfpathlineto{\pgfqpoint{0.635376in}{0.926316in}}%
\pgfpathlineto{\pgfqpoint{0.634712in}{0.926157in}}%
\pgfpathlineto{\pgfqpoint{0.634629in}{0.926316in}}%
\pgfpathlineto{\pgfqpoint{0.625000in}{0.927299in}}%
\pgfusepath{stroke}%
\end{pgfscope}%
\begin{pgfscope}%
\pgfpathrectangle{\pgfqpoint{0.625000in}{0.550000in}}{\pgfqpoint{3.875000in}{3.850000in}} %
\pgfusepath{clip}%
\pgfsetbuttcap%
\pgfsetroundjoin%
\pgfsetlinewidth{0.250937pt}%
\definecolor{currentstroke}{rgb}{0.000000,0.000000,0.000000}%
\pgfsetstrokecolor{currentstroke}%
\pgfsetdash{}{0pt}%
\pgfpathmoveto{\pgfqpoint{0.625000in}{1.098963in}}%
\pgfpathlineto{\pgfqpoint{0.633981in}{1.090351in}}%
\pgfpathlineto{\pgfqpoint{0.625000in}{1.081771in}}%
\pgfusepath{stroke}%
\end{pgfscope}%
\begin{pgfscope}%
\pgfpathrectangle{\pgfqpoint{0.625000in}{0.550000in}}{\pgfqpoint{3.875000in}{3.850000in}} %
\pgfusepath{clip}%
\pgfsetbuttcap%
\pgfsetroundjoin%
\pgfsetlinewidth{0.250937pt}%
\definecolor{currentstroke}{rgb}{0.000000,0.000000,0.000000}%
\pgfsetstrokecolor{currentstroke}%
\pgfsetdash{}{0pt}%
\pgfpathmoveto{\pgfqpoint{0.625000in}{1.253627in}}%
\pgfpathlineto{\pgfqpoint{0.634008in}{1.244737in}}%
\pgfpathlineto{\pgfqpoint{0.625000in}{1.236092in}}%
\pgfusepath{stroke}%
\end{pgfscope}%
\begin{pgfscope}%
\pgfpathrectangle{\pgfqpoint{0.625000in}{0.550000in}}{\pgfqpoint{3.875000in}{3.850000in}} %
\pgfusepath{clip}%
\pgfsetbuttcap%
\pgfsetroundjoin%
\pgfsetlinewidth{0.250937pt}%
\definecolor{currentstroke}{rgb}{0.000000,0.000000,0.000000}%
\pgfsetstrokecolor{currentstroke}%
\pgfsetdash{}{0pt}%
\pgfpathmoveto{\pgfqpoint{0.625000in}{1.407660in}}%
\pgfpathlineto{\pgfqpoint{0.633975in}{1.399123in}}%
\pgfpathlineto{\pgfqpoint{0.625000in}{1.390581in}}%
\pgfusepath{stroke}%
\end{pgfscope}%
\begin{pgfscope}%
\pgfpathrectangle{\pgfqpoint{0.625000in}{0.550000in}}{\pgfqpoint{3.875000in}{3.850000in}} %
\pgfusepath{clip}%
\pgfsetbuttcap%
\pgfsetroundjoin%
\pgfsetlinewidth{0.250937pt}%
\definecolor{currentstroke}{rgb}{0.000000,0.000000,0.000000}%
\pgfsetstrokecolor{currentstroke}%
\pgfsetdash{}{0pt}%
\pgfpathmoveto{\pgfqpoint{0.625000in}{1.562012in}}%
\pgfpathlineto{\pgfqpoint{0.633925in}{1.553509in}}%
\pgfpathlineto{\pgfqpoint{0.625000in}{1.544729in}}%
\pgfusepath{stroke}%
\end{pgfscope}%
\begin{pgfscope}%
\pgfpathrectangle{\pgfqpoint{0.625000in}{0.550000in}}{\pgfqpoint{3.875000in}{3.850000in}} %
\pgfusepath{clip}%
\pgfsetbuttcap%
\pgfsetroundjoin%
\pgfsetlinewidth{0.250937pt}%
\definecolor{currentstroke}{rgb}{0.000000,0.000000,0.000000}%
\pgfsetstrokecolor{currentstroke}%
\pgfsetdash{}{0pt}%
\pgfpathmoveto{\pgfqpoint{0.625000in}{1.716402in}}%
\pgfpathlineto{\pgfqpoint{0.634637in}{1.717544in}}%
\pgfpathlineto{\pgfqpoint{0.634712in}{1.717703in}}%
\pgfpathlineto{\pgfqpoint{0.635376in}{1.717544in}}%
\pgfpathlineto{\pgfqpoint{0.644424in}{1.711275in}}%
\pgfpathlineto{\pgfqpoint{0.646730in}{1.707895in}}%
\pgfpathlineto{\pgfqpoint{0.644424in}{1.704514in}}%
\pgfpathlineto{\pgfqpoint{0.635376in}{1.698246in}}%
\pgfpathlineto{\pgfqpoint{0.634712in}{1.698087in}}%
\pgfpathlineto{\pgfqpoint{0.634633in}{1.698246in}}%
\pgfpathlineto{\pgfqpoint{0.625000in}{1.699334in}}%
\pgfusepath{stroke}%
\end{pgfscope}%
\begin{pgfscope}%
\pgfpathrectangle{\pgfqpoint{0.625000in}{0.550000in}}{\pgfqpoint{3.875000in}{3.850000in}} %
\pgfusepath{clip}%
\pgfsetbuttcap%
\pgfsetroundjoin%
\pgfsetlinewidth{0.250937pt}%
\definecolor{currentstroke}{rgb}{0.000000,0.000000,0.000000}%
\pgfsetstrokecolor{currentstroke}%
\pgfsetdash{}{0pt}%
\pgfpathmoveto{\pgfqpoint{0.625000in}{1.870867in}}%
\pgfpathlineto{\pgfqpoint{0.633911in}{1.862281in}}%
\pgfpathlineto{\pgfqpoint{0.625000in}{1.853872in}}%
\pgfusepath{stroke}%
\end{pgfscope}%
\begin{pgfscope}%
\pgfpathrectangle{\pgfqpoint{0.625000in}{0.550000in}}{\pgfqpoint{3.875000in}{3.850000in}} %
\pgfusepath{clip}%
\pgfsetbuttcap%
\pgfsetroundjoin%
\pgfsetlinewidth{0.250937pt}%
\definecolor{currentstroke}{rgb}{0.000000,0.000000,0.000000}%
\pgfsetstrokecolor{currentstroke}%
\pgfsetdash{}{0pt}%
\pgfpathmoveto{\pgfqpoint{0.625000in}{2.025564in}}%
\pgfpathlineto{\pgfqpoint{0.633906in}{2.016667in}}%
\pgfpathlineto{\pgfqpoint{0.625000in}{2.007841in}}%
\pgfusepath{stroke}%
\end{pgfscope}%
\begin{pgfscope}%
\pgfpathrectangle{\pgfqpoint{0.625000in}{0.550000in}}{\pgfqpoint{3.875000in}{3.850000in}} %
\pgfusepath{clip}%
\pgfsetbuttcap%
\pgfsetroundjoin%
\pgfsetlinewidth{0.250937pt}%
\definecolor{currentstroke}{rgb}{0.000000,0.000000,0.000000}%
\pgfsetstrokecolor{currentstroke}%
\pgfsetdash{}{0pt}%
\pgfpathmoveto{\pgfqpoint{0.625000in}{2.179496in}}%
\pgfpathlineto{\pgfqpoint{0.633930in}{2.171053in}}%
\pgfpathlineto{\pgfqpoint{0.625000in}{2.162367in}}%
\pgfusepath{stroke}%
\end{pgfscope}%
\begin{pgfscope}%
\pgfpathrectangle{\pgfqpoint{0.625000in}{0.550000in}}{\pgfqpoint{3.875000in}{3.850000in}} %
\pgfusepath{clip}%
\pgfsetbuttcap%
\pgfsetroundjoin%
\pgfsetlinewidth{0.250937pt}%
\definecolor{currentstroke}{rgb}{0.000000,0.000000,0.000000}%
\pgfsetstrokecolor{currentstroke}%
\pgfsetdash{}{0pt}%
\pgfpathmoveto{\pgfqpoint{0.625000in}{2.333845in}}%
\pgfpathlineto{\pgfqpoint{0.633974in}{2.325439in}}%
\pgfpathlineto{\pgfqpoint{0.625000in}{2.316443in}}%
\pgfusepath{stroke}%
\end{pgfscope}%
\begin{pgfscope}%
\pgfpathrectangle{\pgfqpoint{0.625000in}{0.550000in}}{\pgfqpoint{3.875000in}{3.850000in}} %
\pgfusepath{clip}%
\pgfsetbuttcap%
\pgfsetroundjoin%
\pgfsetlinewidth{0.250937pt}%
\definecolor{currentstroke}{rgb}{0.000000,0.000000,0.000000}%
\pgfsetstrokecolor{currentstroke}%
\pgfsetdash{}{0pt}%
\pgfpathmoveto{\pgfqpoint{0.625000in}{2.489222in}}%
\pgfpathlineto{\pgfqpoint{0.625236in}{2.489474in}}%
\pgfpathlineto{\pgfqpoint{0.627179in}{2.508772in}}%
\pgfpathlineto{\pgfqpoint{0.632216in}{2.537719in}}%
\pgfpathlineto{\pgfqpoint{0.637484in}{2.557018in}}%
\pgfpathlineto{\pgfqpoint{0.647645in}{2.585965in}}%
\pgfpathlineto{\pgfqpoint{0.660810in}{2.614912in}}%
\pgfpathlineto{\pgfqpoint{0.671347in}{2.634211in}}%
\pgfpathlineto{\pgfqpoint{0.677224in}{2.643860in}}%
\pgfpathlineto{\pgfqpoint{0.692982in}{2.667453in}}%
\pgfpathlineto{\pgfqpoint{0.722118in}{2.702748in}}%
\pgfpathlineto{\pgfqpoint{0.741541in}{2.722478in}}%
\pgfpathlineto{\pgfqpoint{0.770677in}{2.747764in}}%
\pgfpathlineto{\pgfqpoint{0.809524in}{2.775182in}}%
\pgfpathlineto{\pgfqpoint{0.838659in}{2.791915in}}%
\pgfpathlineto{\pgfqpoint{0.867794in}{2.805862in}}%
\pgfpathlineto{\pgfqpoint{0.906642in}{2.820660in}}%
\pgfpathlineto{\pgfqpoint{0.935777in}{2.829118in}}%
\pgfpathlineto{\pgfqpoint{0.964912in}{2.835482in}}%
\pgfpathlineto{\pgfqpoint{0.994048in}{2.839852in}}%
\pgfpathlineto{\pgfqpoint{1.023183in}{2.842246in}}%
\pgfpathlineto{\pgfqpoint{1.052318in}{2.842689in}}%
\pgfpathlineto{\pgfqpoint{1.081454in}{2.841178in}}%
\pgfpathlineto{\pgfqpoint{1.115516in}{2.836842in}}%
\pgfpathlineto{\pgfqpoint{1.139724in}{2.832092in}}%
\pgfpathlineto{\pgfqpoint{1.168860in}{2.824329in}}%
\pgfpathlineto{\pgfqpoint{1.197995in}{2.814221in}}%
\pgfpathlineto{\pgfqpoint{1.227130in}{2.801532in}}%
\pgfpathlineto{\pgfqpoint{1.256266in}{2.785932in}}%
\pgfpathlineto{\pgfqpoint{1.285401in}{2.766936in}}%
\pgfpathlineto{\pgfqpoint{1.307317in}{2.750000in}}%
\pgfpathlineto{\pgfqpoint{1.328760in}{2.730702in}}%
\pgfpathlineto{\pgfqpoint{1.347230in}{2.711404in}}%
\pgfpathlineto{\pgfqpoint{1.363235in}{2.692105in}}%
\pgfpathlineto{\pgfqpoint{1.382519in}{2.664617in}}%
\pgfpathlineto{\pgfqpoint{1.394762in}{2.643860in}}%
\pgfpathlineto{\pgfqpoint{1.408811in}{2.614912in}}%
\pgfpathlineto{\pgfqpoint{1.419738in}{2.585965in}}%
\pgfpathlineto{\pgfqpoint{1.427846in}{2.557018in}}%
\pgfpathlineto{\pgfqpoint{1.433281in}{2.528070in}}%
\pgfpathlineto{\pgfqpoint{1.436189in}{2.499123in}}%
\pgfpathlineto{\pgfqpoint{1.436600in}{2.470175in}}%
\pgfpathlineto{\pgfqpoint{1.434533in}{2.441228in}}%
\pgfpathlineto{\pgfqpoint{1.429932in}{2.412281in}}%
\pgfpathlineto{\pgfqpoint{1.422739in}{2.383333in}}%
\pgfpathlineto{\pgfqpoint{1.411654in}{2.351685in}}%
\pgfpathlineto{\pgfqpoint{1.399809in}{2.325439in}}%
\pgfpathlineto{\pgfqpoint{1.383465in}{2.296491in}}%
\pgfpathlineto{\pgfqpoint{1.370470in}{2.277193in}}%
\pgfpathlineto{\pgfqpoint{1.353383in}{2.255393in}}%
\pgfpathlineto{\pgfqpoint{1.333960in}{2.234154in}}%
\pgfpathlineto{\pgfqpoint{1.314536in}{2.215853in}}%
\pgfpathlineto{\pgfqpoint{1.295113in}{2.199908in}}%
\pgfpathlineto{\pgfqpoint{1.265977in}{2.179638in}}%
\pgfpathlineto{\pgfqpoint{1.236842in}{2.162973in}}%
\pgfpathlineto{\pgfqpoint{1.213297in}{2.151754in}}%
\pgfpathlineto{\pgfqpoint{1.188283in}{2.141789in}}%
\pgfpathlineto{\pgfqpoint{1.159025in}{2.132456in}}%
\pgfpathlineto{\pgfqpoint{1.130013in}{2.125465in}}%
\pgfpathlineto{\pgfqpoint{1.100877in}{2.120590in}}%
\pgfpathlineto{\pgfqpoint{1.071742in}{2.117748in}}%
\pgfpathlineto{\pgfqpoint{1.042607in}{2.116892in}}%
\pgfpathlineto{\pgfqpoint{1.013471in}{2.117982in}}%
\pgfpathlineto{\pgfqpoint{0.984336in}{2.121038in}}%
\pgfpathlineto{\pgfqpoint{0.955201in}{2.126046in}}%
\pgfpathlineto{\pgfqpoint{0.926065in}{2.133134in}}%
\pgfpathlineto{\pgfqpoint{0.896930in}{2.142342in}}%
\pgfpathlineto{\pgfqpoint{0.867794in}{2.153787in}}%
\pgfpathlineto{\pgfqpoint{0.838659in}{2.167734in}}%
\pgfpathlineto{\pgfqpoint{0.809524in}{2.184467in}}%
\pgfpathlineto{\pgfqpoint{0.780388in}{2.204400in}}%
\pgfpathlineto{\pgfqpoint{0.750545in}{2.228947in}}%
\pgfpathlineto{\pgfqpoint{0.730435in}{2.248246in}}%
\pgfpathlineto{\pgfqpoint{0.712406in}{2.267852in}}%
\pgfpathlineto{\pgfqpoint{0.690106in}{2.296491in}}%
\pgfpathlineto{\pgfqpoint{0.673559in}{2.321365in}}%
\pgfpathlineto{\pgfqpoint{0.654135in}{2.358039in}}%
\pgfpathlineto{\pgfqpoint{0.640546in}{2.392982in}}%
\pgfpathlineto{\pgfqpoint{0.632216in}{2.421930in}}%
\pgfpathlineto{\pgfqpoint{0.627179in}{2.450877in}}%
\pgfpathlineto{\pgfqpoint{0.625000in}{2.470427in}}%
\pgfpathlineto{\pgfqpoint{0.625000in}{2.470427in}}%
\pgfusepath{stroke}%
\end{pgfscope}%
\begin{pgfscope}%
\pgfpathrectangle{\pgfqpoint{0.625000in}{0.550000in}}{\pgfqpoint{3.875000in}{3.850000in}} %
\pgfusepath{clip}%
\pgfsetbuttcap%
\pgfsetroundjoin%
\pgfsetlinewidth{0.250937pt}%
\definecolor{currentstroke}{rgb}{0.000000,0.000000,0.000000}%
\pgfsetstrokecolor{currentstroke}%
\pgfsetdash{}{0pt}%
\pgfpathmoveto{\pgfqpoint{0.625000in}{2.643205in}}%
\pgfpathlineto{\pgfqpoint{0.633973in}{2.634211in}}%
\pgfpathlineto{\pgfqpoint{0.625000in}{2.625806in}}%
\pgfusepath{stroke}%
\end{pgfscope}%
\begin{pgfscope}%
\pgfpathrectangle{\pgfqpoint{0.625000in}{0.550000in}}{\pgfqpoint{3.875000in}{3.850000in}} %
\pgfusepath{clip}%
\pgfsetbuttcap%
\pgfsetroundjoin%
\pgfsetlinewidth{0.250937pt}%
\definecolor{currentstroke}{rgb}{0.000000,0.000000,0.000000}%
\pgfsetstrokecolor{currentstroke}%
\pgfsetdash{}{0pt}%
\pgfpathmoveto{\pgfqpoint{0.625000in}{2.797276in}}%
\pgfpathlineto{\pgfqpoint{0.633925in}{2.788596in}}%
\pgfpathlineto{\pgfqpoint{0.625000in}{2.780160in}}%
\pgfusepath{stroke}%
\end{pgfscope}%
\begin{pgfscope}%
\pgfpathrectangle{\pgfqpoint{0.625000in}{0.550000in}}{\pgfqpoint{3.875000in}{3.850000in}} %
\pgfusepath{clip}%
\pgfsetbuttcap%
\pgfsetroundjoin%
\pgfsetlinewidth{0.250937pt}%
\definecolor{currentstroke}{rgb}{0.000000,0.000000,0.000000}%
\pgfsetstrokecolor{currentstroke}%
\pgfsetdash{}{0pt}%
\pgfpathmoveto{\pgfqpoint{0.625000in}{2.951793in}}%
\pgfpathlineto{\pgfqpoint{0.633892in}{2.942982in}}%
\pgfpathlineto{\pgfqpoint{0.625000in}{2.934098in}}%
\pgfusepath{stroke}%
\end{pgfscope}%
\begin{pgfscope}%
\pgfpathrectangle{\pgfqpoint{0.625000in}{0.550000in}}{\pgfqpoint{3.875000in}{3.850000in}} %
\pgfusepath{clip}%
\pgfsetbuttcap%
\pgfsetroundjoin%
\pgfsetlinewidth{0.250937pt}%
\definecolor{currentstroke}{rgb}{0.000000,0.000000,0.000000}%
\pgfsetstrokecolor{currentstroke}%
\pgfsetdash{}{0pt}%
\pgfpathmoveto{\pgfqpoint{0.625000in}{3.105761in}}%
\pgfpathlineto{\pgfqpoint{0.633900in}{3.097368in}}%
\pgfpathlineto{\pgfqpoint{0.625000in}{3.088797in}}%
\pgfusepath{stroke}%
\end{pgfscope}%
\begin{pgfscope}%
\pgfpathrectangle{\pgfqpoint{0.625000in}{0.550000in}}{\pgfqpoint{3.875000in}{3.850000in}} %
\pgfusepath{clip}%
\pgfsetbuttcap%
\pgfsetroundjoin%
\pgfsetlinewidth{0.250937pt}%
\definecolor{currentstroke}{rgb}{0.000000,0.000000,0.000000}%
\pgfsetstrokecolor{currentstroke}%
\pgfsetdash{}{0pt}%
\pgfpathmoveto{\pgfqpoint{0.625000in}{3.260263in}}%
\pgfpathlineto{\pgfqpoint{0.634633in}{3.261404in}}%
\pgfpathlineto{\pgfqpoint{0.634712in}{3.261563in}}%
\pgfpathlineto{\pgfqpoint{0.635376in}{3.261404in}}%
\pgfpathlineto{\pgfqpoint{0.644424in}{3.255135in}}%
\pgfpathlineto{\pgfqpoint{0.646730in}{3.251754in}}%
\pgfpathlineto{\pgfqpoint{0.644424in}{3.248374in}}%
\pgfpathlineto{\pgfqpoint{0.635376in}{3.242105in}}%
\pgfpathlineto{\pgfqpoint{0.634712in}{3.241946in}}%
\pgfpathlineto{\pgfqpoint{0.634637in}{3.242105in}}%
\pgfpathlineto{\pgfqpoint{0.625000in}{3.243302in}}%
\pgfusepath{stroke}%
\end{pgfscope}%
\begin{pgfscope}%
\pgfpathrectangle{\pgfqpoint{0.625000in}{0.550000in}}{\pgfqpoint{3.875000in}{3.850000in}} %
\pgfusepath{clip}%
\pgfsetbuttcap%
\pgfsetroundjoin%
\pgfsetlinewidth{0.250937pt}%
\definecolor{currentstroke}{rgb}{0.000000,0.000000,0.000000}%
\pgfsetstrokecolor{currentstroke}%
\pgfsetdash{}{0pt}%
\pgfpathmoveto{\pgfqpoint{0.625000in}{3.414890in}}%
\pgfpathlineto{\pgfqpoint{0.633897in}{3.406140in}}%
\pgfpathlineto{\pgfqpoint{0.625000in}{3.397675in}}%
\pgfusepath{stroke}%
\end{pgfscope}%
\begin{pgfscope}%
\pgfpathrectangle{\pgfqpoint{0.625000in}{0.550000in}}{\pgfqpoint{3.875000in}{3.850000in}} %
\pgfusepath{clip}%
\pgfsetbuttcap%
\pgfsetroundjoin%
\pgfsetlinewidth{0.250937pt}%
\definecolor{currentstroke}{rgb}{0.000000,0.000000,0.000000}%
\pgfsetstrokecolor{currentstroke}%
\pgfsetdash{}{0pt}%
\pgfpathmoveto{\pgfqpoint{0.625000in}{3.569025in}}%
\pgfpathlineto{\pgfqpoint{0.633944in}{3.560526in}}%
\pgfpathlineto{\pgfqpoint{0.625000in}{3.552033in}}%
\pgfusepath{stroke}%
\end{pgfscope}%
\begin{pgfscope}%
\pgfpathrectangle{\pgfqpoint{0.625000in}{0.550000in}}{\pgfqpoint{3.875000in}{3.850000in}} %
\pgfusepath{clip}%
\pgfsetbuttcap%
\pgfsetroundjoin%
\pgfsetlinewidth{0.250937pt}%
\definecolor{currentstroke}{rgb}{0.000000,0.000000,0.000000}%
\pgfsetstrokecolor{currentstroke}%
\pgfsetdash{}{0pt}%
\pgfpathmoveto{\pgfqpoint{0.625000in}{3.723496in}}%
\pgfpathlineto{\pgfqpoint{0.633963in}{3.714912in}}%
\pgfpathlineto{\pgfqpoint{0.625000in}{3.706070in}}%
\pgfusepath{stroke}%
\end{pgfscope}%
\begin{pgfscope}%
\pgfpathrectangle{\pgfqpoint{0.625000in}{0.550000in}}{\pgfqpoint{3.875000in}{3.850000in}} %
\pgfusepath{clip}%
\pgfsetbuttcap%
\pgfsetroundjoin%
\pgfsetlinewidth{0.250937pt}%
\definecolor{currentstroke}{rgb}{0.000000,0.000000,0.000000}%
\pgfsetstrokecolor{currentstroke}%
\pgfsetdash{}{0pt}%
\pgfpathmoveto{\pgfqpoint{0.625000in}{3.877771in}}%
\pgfpathlineto{\pgfqpoint{0.633904in}{3.869298in}}%
\pgfpathlineto{\pgfqpoint{0.625000in}{3.860791in}}%
\pgfusepath{stroke}%
\end{pgfscope}%
\begin{pgfscope}%
\pgfpathrectangle{\pgfqpoint{0.625000in}{0.550000in}}{\pgfqpoint{3.875000in}{3.850000in}} %
\pgfusepath{clip}%
\pgfsetbuttcap%
\pgfsetroundjoin%
\pgfsetlinewidth{0.250937pt}%
\definecolor{currentstroke}{rgb}{0.000000,0.000000,0.000000}%
\pgfsetstrokecolor{currentstroke}%
\pgfsetdash{}{0pt}%
\pgfpathmoveto{\pgfqpoint{0.625000in}{4.032258in}}%
\pgfpathlineto{\pgfqpoint{0.634629in}{4.033333in}}%
\pgfpathlineto{\pgfqpoint{0.634712in}{4.033492in}}%
\pgfpathlineto{\pgfqpoint{0.635376in}{4.033333in}}%
\pgfpathlineto{\pgfqpoint{0.644424in}{4.027065in}}%
\pgfpathlineto{\pgfqpoint{0.646730in}{4.023684in}}%
\pgfpathlineto{\pgfqpoint{0.644424in}{4.020304in}}%
\pgfpathlineto{\pgfqpoint{0.635376in}{4.014035in}}%
\pgfpathlineto{\pgfqpoint{0.634712in}{4.013876in}}%
\pgfpathlineto{\pgfqpoint{0.634636in}{4.014035in}}%
\pgfpathlineto{\pgfqpoint{0.625000in}{4.015194in}}%
\pgfusepath{stroke}%
\end{pgfscope}%
\begin{pgfscope}%
\pgfpathrectangle{\pgfqpoint{0.625000in}{0.550000in}}{\pgfqpoint{3.875000in}{3.850000in}} %
\pgfusepath{clip}%
\pgfsetbuttcap%
\pgfsetroundjoin%
\pgfsetlinewidth{0.250937pt}%
\definecolor{currentstroke}{rgb}{0.000000,0.000000,0.000000}%
\pgfsetstrokecolor{currentstroke}%
\pgfsetdash{}{0pt}%
\pgfpathmoveto{\pgfqpoint{0.625000in}{4.186645in}}%
\pgfpathlineto{\pgfqpoint{0.633908in}{4.178070in}}%
\pgfpathlineto{\pgfqpoint{0.625000in}{4.169897in}}%
\pgfusepath{stroke}%
\end{pgfscope}%
\begin{pgfscope}%
\pgfpathrectangle{\pgfqpoint{0.625000in}{0.550000in}}{\pgfqpoint{3.875000in}{3.850000in}} %
\pgfusepath{clip}%
\pgfsetbuttcap%
\pgfsetroundjoin%
\pgfsetlinewidth{0.250937pt}%
\definecolor{currentstroke}{rgb}{0.000000,0.000000,0.000000}%
\pgfsetstrokecolor{currentstroke}%
\pgfsetdash{}{0pt}%
\pgfpathmoveto{\pgfqpoint{0.625000in}{4.340982in}}%
\pgfpathlineto{\pgfqpoint{0.633857in}{4.332456in}}%
\pgfpathlineto{\pgfqpoint{0.625000in}{4.323901in}}%
\pgfusepath{stroke}%
\end{pgfscope}%
\begin{pgfscope}%
\pgfpathrectangle{\pgfqpoint{0.625000in}{0.550000in}}{\pgfqpoint{3.875000in}{3.850000in}} %
\pgfusepath{clip}%
\pgfsetbuttcap%
\pgfsetroundjoin%
\pgfsetlinewidth{0.250937pt}%
\definecolor{currentstroke}{rgb}{0.000000,0.000000,0.000000}%
\pgfsetstrokecolor{currentstroke}%
\pgfsetdash{}{0pt}%
\pgfpathmoveto{\pgfqpoint{0.644424in}{1.165661in}}%
\pgfpathlineto{\pgfqpoint{0.642600in}{1.167544in}}%
\pgfpathlineto{\pgfqpoint{0.634712in}{1.171221in}}%
\pgfpathlineto{\pgfqpoint{0.631649in}{1.177193in}}%
\pgfpathlineto{\pgfqpoint{0.627537in}{1.186842in}}%
\pgfpathlineto{\pgfqpoint{0.626198in}{1.196491in}}%
\pgfpathlineto{\pgfqpoint{0.630162in}{1.206140in}}%
\pgfpathlineto{\pgfqpoint{0.634712in}{1.214731in}}%
\pgfpathlineto{\pgfqpoint{0.638019in}{1.215789in}}%
\pgfpathlineto{\pgfqpoint{0.644424in}{1.221223in}}%
\pgfpathlineto{\pgfqpoint{0.654135in}{1.223494in}}%
\pgfpathlineto{\pgfqpoint{0.663847in}{1.223221in}}%
\pgfpathlineto{\pgfqpoint{0.673559in}{1.219994in}}%
\pgfpathlineto{\pgfqpoint{0.680121in}{1.215789in}}%
\pgfpathlineto{\pgfqpoint{0.683271in}{1.212089in}}%
\pgfpathlineto{\pgfqpoint{0.687824in}{1.206140in}}%
\pgfpathlineto{\pgfqpoint{0.690614in}{1.196491in}}%
\pgfpathlineto{\pgfqpoint{0.690099in}{1.186842in}}%
\pgfpathlineto{\pgfqpoint{0.685872in}{1.177193in}}%
\pgfpathlineto{\pgfqpoint{0.683271in}{1.174358in}}%
\pgfpathlineto{\pgfqpoint{0.675454in}{1.167544in}}%
\pgfpathlineto{\pgfqpoint{0.673559in}{1.166539in}}%
\pgfpathlineto{\pgfqpoint{0.663847in}{1.163239in}}%
\pgfpathlineto{\pgfqpoint{0.654135in}{1.162896in}}%
\pgfpathlineto{\pgfqpoint{0.644424in}{1.165661in}}%
\pgfusepath{stroke}%
\end{pgfscope}%
\begin{pgfscope}%
\pgfpathrectangle{\pgfqpoint{0.625000in}{0.550000in}}{\pgfqpoint{3.875000in}{3.850000in}} %
\pgfusepath{clip}%
\pgfsetbuttcap%
\pgfsetroundjoin%
\pgfsetlinewidth{0.250937pt}%
\definecolor{currentstroke}{rgb}{0.000000,0.000000,0.000000}%
\pgfsetstrokecolor{currentstroke}%
\pgfsetdash{}{0pt}%
\pgfpathmoveto{\pgfqpoint{0.634712in}{1.929598in}}%
\pgfpathlineto{\pgfqpoint{0.634506in}{1.929825in}}%
\pgfpathlineto{\pgfqpoint{0.634712in}{1.929995in}}%
\pgfpathlineto{\pgfqpoint{0.634932in}{1.929825in}}%
\pgfpathlineto{\pgfqpoint{0.634712in}{1.929598in}}%
\pgfusepath{stroke}%
\end{pgfscope}%
\begin{pgfscope}%
\pgfpathrectangle{\pgfqpoint{0.625000in}{0.550000in}}{\pgfqpoint{3.875000in}{3.850000in}} %
\pgfusepath{clip}%
\pgfsetbuttcap%
\pgfsetroundjoin%
\pgfsetlinewidth{0.250937pt}%
\definecolor{currentstroke}{rgb}{0.000000,0.000000,0.000000}%
\pgfsetstrokecolor{currentstroke}%
\pgfsetdash{}{0pt}%
\pgfpathmoveto{\pgfqpoint{0.634712in}{3.029654in}}%
\pgfpathlineto{\pgfqpoint{0.634506in}{3.029825in}}%
\pgfpathlineto{\pgfqpoint{0.634712in}{3.030051in}}%
\pgfpathlineto{\pgfqpoint{0.634932in}{3.029825in}}%
\pgfpathlineto{\pgfqpoint{0.634712in}{3.029654in}}%
\pgfusepath{stroke}%
\end{pgfscope}%
\begin{pgfscope}%
\pgfpathrectangle{\pgfqpoint{0.625000in}{0.550000in}}{\pgfqpoint{3.875000in}{3.850000in}} %
\pgfusepath{clip}%
\pgfsetbuttcap%
\pgfsetroundjoin%
\pgfsetlinewidth{0.250937pt}%
\definecolor{currentstroke}{rgb}{0.000000,0.000000,0.000000}%
\pgfsetstrokecolor{currentstroke}%
\pgfsetdash{}{0pt}%
\pgfpathmoveto{\pgfqpoint{0.644424in}{3.738426in}}%
\pgfpathlineto{\pgfqpoint{0.638019in}{3.743860in}}%
\pgfpathlineto{\pgfqpoint{0.634712in}{3.744918in}}%
\pgfpathlineto{\pgfqpoint{0.630162in}{3.753509in}}%
\pgfpathlineto{\pgfqpoint{0.626198in}{3.763158in}}%
\pgfpathlineto{\pgfqpoint{0.627537in}{3.772807in}}%
\pgfpathlineto{\pgfqpoint{0.631649in}{3.782456in}}%
\pgfpathlineto{\pgfqpoint{0.634712in}{3.788428in}}%
\pgfpathlineto{\pgfqpoint{0.642600in}{3.792105in}}%
\pgfpathlineto{\pgfqpoint{0.644424in}{3.793988in}}%
\pgfpathlineto{\pgfqpoint{0.654135in}{3.796753in}}%
\pgfpathlineto{\pgfqpoint{0.663847in}{3.796410in}}%
\pgfpathlineto{\pgfqpoint{0.673559in}{3.793110in}}%
\pgfpathlineto{\pgfqpoint{0.675454in}{3.792105in}}%
\pgfpathlineto{\pgfqpoint{0.683271in}{3.785291in}}%
\pgfpathlineto{\pgfqpoint{0.685872in}{3.782456in}}%
\pgfpathlineto{\pgfqpoint{0.690099in}{3.772807in}}%
\pgfpathlineto{\pgfqpoint{0.690614in}{3.763158in}}%
\pgfpathlineto{\pgfqpoint{0.687824in}{3.753509in}}%
\pgfpathlineto{\pgfqpoint{0.683271in}{3.747560in}}%
\pgfpathlineto{\pgfqpoint{0.680121in}{3.743860in}}%
\pgfpathlineto{\pgfqpoint{0.673559in}{3.739655in}}%
\pgfpathlineto{\pgfqpoint{0.663847in}{3.736428in}}%
\pgfpathlineto{\pgfqpoint{0.654135in}{3.736155in}}%
\pgfpathlineto{\pgfqpoint{0.644424in}{3.738426in}}%
\pgfusepath{stroke}%
\end{pgfscope}%
\begin{pgfscope}%
\pgfpathrectangle{\pgfqpoint{0.625000in}{0.550000in}}{\pgfqpoint{3.875000in}{3.850000in}} %
\pgfusepath{clip}%
\pgfsetbuttcap%
\pgfsetroundjoin%
\pgfsetlinewidth{0.250937pt}%
\definecolor{currentstroke}{rgb}{0.000000,0.000000,0.000000}%
\pgfsetstrokecolor{currentstroke}%
\pgfsetdash{}{0pt}%
\pgfpathmoveto{\pgfqpoint{0.625000in}{0.635968in}}%
\pgfpathlineto{\pgfqpoint{0.634051in}{0.627193in}}%
\pgfpathlineto{\pgfqpoint{0.625000in}{0.618443in}}%
\pgfusepath{stroke}%
\end{pgfscope}%
\begin{pgfscope}%
\pgfpathrectangle{\pgfqpoint{0.625000in}{0.550000in}}{\pgfqpoint{3.875000in}{3.850000in}} %
\pgfusepath{clip}%
\pgfsetbuttcap%
\pgfsetroundjoin%
\pgfsetlinewidth{0.250937pt}%
\definecolor{currentstroke}{rgb}{0.000000,0.000000,0.000000}%
\pgfsetstrokecolor{currentstroke}%
\pgfsetdash{}{0pt}%
\pgfpathmoveto{\pgfqpoint{0.625000in}{0.789961in}}%
\pgfpathlineto{\pgfqpoint{0.634068in}{0.781579in}}%
\pgfpathlineto{\pgfqpoint{0.625000in}{0.772824in}}%
\pgfusepath{stroke}%
\end{pgfscope}%
\begin{pgfscope}%
\pgfpathrectangle{\pgfqpoint{0.625000in}{0.550000in}}{\pgfqpoint{3.875000in}{3.850000in}} %
\pgfusepath{clip}%
\pgfsetbuttcap%
\pgfsetroundjoin%
\pgfsetlinewidth{0.250937pt}%
\definecolor{currentstroke}{rgb}{0.000000,0.000000,0.000000}%
\pgfsetstrokecolor{currentstroke}%
\pgfsetdash{}{0pt}%
\pgfpathmoveto{\pgfqpoint{0.625000in}{0.944630in}}%
\pgfpathlineto{\pgfqpoint{0.633941in}{0.945614in}}%
\pgfpathlineto{\pgfqpoint{0.634712in}{0.947230in}}%
\pgfpathlineto{\pgfqpoint{0.641457in}{0.945614in}}%
\pgfpathlineto{\pgfqpoint{0.644424in}{0.943559in}}%
\pgfpathlineto{\pgfqpoint{0.649605in}{0.935965in}}%
\pgfpathlineto{\pgfqpoint{0.644424in}{0.928371in}}%
\pgfpathlineto{\pgfqpoint{0.641457in}{0.926316in}}%
\pgfpathlineto{\pgfqpoint{0.634712in}{0.924700in}}%
\pgfpathlineto{\pgfqpoint{0.633874in}{0.926316in}}%
\pgfpathlineto{\pgfqpoint{0.625000in}{0.927222in}}%
\pgfusepath{stroke}%
\end{pgfscope}%
\begin{pgfscope}%
\pgfpathrectangle{\pgfqpoint{0.625000in}{0.550000in}}{\pgfqpoint{3.875000in}{3.850000in}} %
\pgfusepath{clip}%
\pgfsetbuttcap%
\pgfsetroundjoin%
\pgfsetlinewidth{0.250937pt}%
\definecolor{currentstroke}{rgb}{0.000000,0.000000,0.000000}%
\pgfsetstrokecolor{currentstroke}%
\pgfsetdash{}{0pt}%
\pgfpathmoveto{\pgfqpoint{0.625000in}{1.099042in}}%
\pgfpathlineto{\pgfqpoint{0.634063in}{1.090351in}}%
\pgfpathlineto{\pgfqpoint{0.625000in}{1.081692in}}%
\pgfusepath{stroke}%
\end{pgfscope}%
\begin{pgfscope}%
\pgfpathrectangle{\pgfqpoint{0.625000in}{0.550000in}}{\pgfqpoint{3.875000in}{3.850000in}} %
\pgfusepath{clip}%
\pgfsetbuttcap%
\pgfsetroundjoin%
\pgfsetlinewidth{0.250937pt}%
\definecolor{currentstroke}{rgb}{0.000000,0.000000,0.000000}%
\pgfsetstrokecolor{currentstroke}%
\pgfsetdash{}{0pt}%
\pgfpathmoveto{\pgfqpoint{0.625000in}{1.253710in}}%
\pgfpathlineto{\pgfqpoint{0.634091in}{1.244737in}}%
\pgfpathlineto{\pgfqpoint{0.625000in}{1.236012in}}%
\pgfusepath{stroke}%
\end{pgfscope}%
\begin{pgfscope}%
\pgfpathrectangle{\pgfqpoint{0.625000in}{0.550000in}}{\pgfqpoint{3.875000in}{3.850000in}} %
\pgfusepath{clip}%
\pgfsetbuttcap%
\pgfsetroundjoin%
\pgfsetlinewidth{0.250937pt}%
\definecolor{currentstroke}{rgb}{0.000000,0.000000,0.000000}%
\pgfsetstrokecolor{currentstroke}%
\pgfsetdash{}{0pt}%
\pgfpathmoveto{\pgfqpoint{0.625000in}{1.407742in}}%
\pgfpathlineto{\pgfqpoint{0.634060in}{1.399123in}}%
\pgfpathlineto{\pgfqpoint{0.625000in}{1.390499in}}%
\pgfusepath{stroke}%
\end{pgfscope}%
\begin{pgfscope}%
\pgfpathrectangle{\pgfqpoint{0.625000in}{0.550000in}}{\pgfqpoint{3.875000in}{3.850000in}} %
\pgfusepath{clip}%
\pgfsetbuttcap%
\pgfsetroundjoin%
\pgfsetlinewidth{0.250937pt}%
\definecolor{currentstroke}{rgb}{0.000000,0.000000,0.000000}%
\pgfsetstrokecolor{currentstroke}%
\pgfsetdash{}{0pt}%
\pgfpathmoveto{\pgfqpoint{0.625000in}{1.562092in}}%
\pgfpathlineto{\pgfqpoint{0.634009in}{1.553509in}}%
\pgfpathlineto{\pgfqpoint{0.625000in}{1.544647in}}%
\pgfusepath{stroke}%
\end{pgfscope}%
\begin{pgfscope}%
\pgfpathrectangle{\pgfqpoint{0.625000in}{0.550000in}}{\pgfqpoint{3.875000in}{3.850000in}} %
\pgfusepath{clip}%
\pgfsetbuttcap%
\pgfsetroundjoin%
\pgfsetlinewidth{0.250937pt}%
\definecolor{currentstroke}{rgb}{0.000000,0.000000,0.000000}%
\pgfsetstrokecolor{currentstroke}%
\pgfsetdash{}{0pt}%
\pgfpathmoveto{\pgfqpoint{0.625000in}{1.716483in}}%
\pgfpathlineto{\pgfqpoint{0.633955in}{1.717544in}}%
\pgfpathlineto{\pgfqpoint{0.634712in}{1.719160in}}%
\pgfpathlineto{\pgfqpoint{0.641457in}{1.717544in}}%
\pgfpathlineto{\pgfqpoint{0.644424in}{1.715489in}}%
\pgfpathlineto{\pgfqpoint{0.649605in}{1.707895in}}%
\pgfpathlineto{\pgfqpoint{0.644424in}{1.700301in}}%
\pgfpathlineto{\pgfqpoint{0.641457in}{1.698246in}}%
\pgfpathlineto{\pgfqpoint{0.634712in}{1.696630in}}%
\pgfpathlineto{\pgfqpoint{0.633912in}{1.698246in}}%
\pgfpathlineto{\pgfqpoint{0.625000in}{1.699252in}}%
\pgfusepath{stroke}%
\end{pgfscope}%
\begin{pgfscope}%
\pgfpathrectangle{\pgfqpoint{0.625000in}{0.550000in}}{\pgfqpoint{3.875000in}{3.850000in}} %
\pgfusepath{clip}%
\pgfsetbuttcap%
\pgfsetroundjoin%
\pgfsetlinewidth{0.250937pt}%
\definecolor{currentstroke}{rgb}{0.000000,0.000000,0.000000}%
\pgfsetstrokecolor{currentstroke}%
\pgfsetdash{}{0pt}%
\pgfpathmoveto{\pgfqpoint{0.625000in}{1.870950in}}%
\pgfpathlineto{\pgfqpoint{0.633997in}{1.862281in}}%
\pgfpathlineto{\pgfqpoint{0.625000in}{1.853791in}}%
\pgfusepath{stroke}%
\end{pgfscope}%
\begin{pgfscope}%
\pgfpathrectangle{\pgfqpoint{0.625000in}{0.550000in}}{\pgfqpoint{3.875000in}{3.850000in}} %
\pgfusepath{clip}%
\pgfsetbuttcap%
\pgfsetroundjoin%
\pgfsetlinewidth{0.250937pt}%
\definecolor{currentstroke}{rgb}{0.000000,0.000000,0.000000}%
\pgfsetstrokecolor{currentstroke}%
\pgfsetdash{}{0pt}%
\pgfpathmoveto{\pgfqpoint{0.625000in}{2.025652in}}%
\pgfpathlineto{\pgfqpoint{0.633994in}{2.016667in}}%
\pgfpathlineto{\pgfqpoint{0.625000in}{2.007755in}}%
\pgfusepath{stroke}%
\end{pgfscope}%
\begin{pgfscope}%
\pgfpathrectangle{\pgfqpoint{0.625000in}{0.550000in}}{\pgfqpoint{3.875000in}{3.850000in}} %
\pgfusepath{clip}%
\pgfsetbuttcap%
\pgfsetroundjoin%
\pgfsetlinewidth{0.250937pt}%
\definecolor{currentstroke}{rgb}{0.000000,0.000000,0.000000}%
\pgfsetstrokecolor{currentstroke}%
\pgfsetdash{}{0pt}%
\pgfpathmoveto{\pgfqpoint{0.625000in}{2.179579in}}%
\pgfpathlineto{\pgfqpoint{0.634017in}{2.171053in}}%
\pgfpathlineto{\pgfqpoint{0.625000in}{2.162282in}}%
\pgfusepath{stroke}%
\end{pgfscope}%
\begin{pgfscope}%
\pgfpathrectangle{\pgfqpoint{0.625000in}{0.550000in}}{\pgfqpoint{3.875000in}{3.850000in}} %
\pgfusepath{clip}%
\pgfsetbuttcap%
\pgfsetroundjoin%
\pgfsetlinewidth{0.250937pt}%
\definecolor{currentstroke}{rgb}{0.000000,0.000000,0.000000}%
\pgfsetstrokecolor{currentstroke}%
\pgfsetdash{}{0pt}%
\pgfpathmoveto{\pgfqpoint{0.625000in}{2.333928in}}%
\pgfpathlineto{\pgfqpoint{0.634063in}{2.325439in}}%
\pgfpathlineto{\pgfqpoint{0.625000in}{2.316355in}}%
\pgfusepath{stroke}%
\end{pgfscope}%
\begin{pgfscope}%
\pgfpathrectangle{\pgfqpoint{0.625000in}{0.550000in}}{\pgfqpoint{3.875000in}{3.850000in}} %
\pgfusepath{clip}%
\pgfsetbuttcap%
\pgfsetroundjoin%
\pgfsetlinewidth{0.250937pt}%
\definecolor{currentstroke}{rgb}{0.000000,0.000000,0.000000}%
\pgfsetstrokecolor{currentstroke}%
\pgfsetdash{}{0pt}%
\pgfpathmoveto{\pgfqpoint{0.625000in}{2.489306in}}%
\pgfpathlineto{\pgfqpoint{0.625157in}{2.489474in}}%
\pgfpathlineto{\pgfqpoint{0.628305in}{2.518421in}}%
\pgfpathlineto{\pgfqpoint{0.633548in}{2.547368in}}%
\pgfpathlineto{\pgfqpoint{0.645480in}{2.585965in}}%
\pgfpathlineto{\pgfqpoint{0.657205in}{2.614912in}}%
\pgfpathlineto{\pgfqpoint{0.673559in}{2.648001in}}%
\pgfpathlineto{\pgfqpoint{0.694899in}{2.682456in}}%
\pgfpathlineto{\pgfqpoint{0.716128in}{2.711404in}}%
\pgfpathlineto{\pgfqpoint{0.732120in}{2.730702in}}%
\pgfpathlineto{\pgfqpoint{0.760965in}{2.761031in}}%
\pgfpathlineto{\pgfqpoint{0.791966in}{2.788596in}}%
\pgfpathlineto{\pgfqpoint{0.819236in}{2.809722in}}%
\pgfpathlineto{\pgfqpoint{0.848371in}{2.829326in}}%
\pgfpathlineto{\pgfqpoint{0.877835in}{2.846491in}}%
\pgfpathlineto{\pgfqpoint{0.906642in}{2.861142in}}%
\pgfpathlineto{\pgfqpoint{0.945489in}{2.877630in}}%
\pgfpathlineto{\pgfqpoint{0.974624in}{2.887827in}}%
\pgfpathlineto{\pgfqpoint{1.013471in}{2.898711in}}%
\pgfpathlineto{\pgfqpoint{1.052318in}{2.906648in}}%
\pgfpathlineto{\pgfqpoint{1.091165in}{2.911779in}}%
\pgfpathlineto{\pgfqpoint{1.130013in}{2.914141in}}%
\pgfpathlineto{\pgfqpoint{1.168860in}{2.913780in}}%
\pgfpathlineto{\pgfqpoint{1.207707in}{2.910653in}}%
\pgfpathlineto{\pgfqpoint{1.246554in}{2.904635in}}%
\pgfpathlineto{\pgfqpoint{1.275689in}{2.898171in}}%
\pgfpathlineto{\pgfqpoint{1.304825in}{2.889914in}}%
\pgfpathlineto{\pgfqpoint{1.333960in}{2.879755in}}%
\pgfpathlineto{\pgfqpoint{1.366882in}{2.865789in}}%
\pgfpathlineto{\pgfqpoint{1.392231in}{2.853047in}}%
\pgfpathlineto{\pgfqpoint{1.421366in}{2.836009in}}%
\pgfpathlineto{\pgfqpoint{1.450501in}{2.816015in}}%
\pgfpathlineto{\pgfqpoint{1.479637in}{2.792493in}}%
\pgfpathlineto{\pgfqpoint{1.499060in}{2.774442in}}%
\pgfpathlineto{\pgfqpoint{1.522115in}{2.750000in}}%
\pgfpathlineto{\pgfqpoint{1.545287in}{2.721053in}}%
\pgfpathlineto{\pgfqpoint{1.564744in}{2.692105in}}%
\pgfpathlineto{\pgfqpoint{1.580998in}{2.663158in}}%
\pgfpathlineto{\pgfqpoint{1.594380in}{2.634211in}}%
\pgfpathlineto{\pgfqpoint{1.605890in}{2.602956in}}%
\pgfpathlineto{\pgfqpoint{1.613553in}{2.576316in}}%
\pgfpathlineto{\pgfqpoint{1.619657in}{2.547368in}}%
\pgfpathlineto{\pgfqpoint{1.623547in}{2.518421in}}%
\pgfpathlineto{\pgfqpoint{1.625313in}{2.487742in}}%
\pgfpathlineto{\pgfqpoint{1.624945in}{2.460526in}}%
\pgfpathlineto{\pgfqpoint{1.622493in}{2.431579in}}%
\pgfpathlineto{\pgfqpoint{1.617865in}{2.402632in}}%
\pgfpathlineto{\pgfqpoint{1.611029in}{2.373684in}}%
\pgfpathlineto{\pgfqpoint{1.601862in}{2.344737in}}%
\pgfpathlineto{\pgfqpoint{1.590225in}{2.315789in}}%
\pgfpathlineto{\pgfqpoint{1.575897in}{2.286842in}}%
\pgfpathlineto{\pgfqpoint{1.558630in}{2.257895in}}%
\pgfpathlineto{\pgfqpoint{1.537907in}{2.228845in}}%
\pgfpathlineto{\pgfqpoint{1.518484in}{2.205589in}}%
\pgfpathlineto{\pgfqpoint{1.499060in}{2.185207in}}%
\pgfpathlineto{\pgfqpoint{1.479637in}{2.167156in}}%
\pgfpathlineto{\pgfqpoint{1.460213in}{2.151041in}}%
\pgfpathlineto{\pgfqpoint{1.431078in}{2.129956in}}%
\pgfpathlineto{\pgfqpoint{1.401942in}{2.111980in}}%
\pgfpathlineto{\pgfqpoint{1.372807in}{2.096679in}}%
\pgfpathlineto{\pgfqpoint{1.343672in}{2.083730in}}%
\pgfpathlineto{\pgfqpoint{1.314536in}{2.072905in}}%
\pgfpathlineto{\pgfqpoint{1.285401in}{2.064032in}}%
\pgfpathlineto{\pgfqpoint{1.247810in}{2.055263in}}%
\pgfpathlineto{\pgfqpoint{1.217419in}{2.050224in}}%
\pgfpathlineto{\pgfqpoint{1.178571in}{2.046387in}}%
\pgfpathlineto{\pgfqpoint{1.139724in}{2.045340in}}%
\pgfpathlineto{\pgfqpoint{1.100877in}{2.047024in}}%
\pgfpathlineto{\pgfqpoint{1.062030in}{2.051450in}}%
\pgfpathlineto{\pgfqpoint{1.023183in}{2.058684in}}%
\pgfpathlineto{\pgfqpoint{0.984336in}{2.068813in}}%
\pgfpathlineto{\pgfqpoint{0.955201in}{2.078409in}}%
\pgfpathlineto{\pgfqpoint{0.926065in}{2.089824in}}%
\pgfpathlineto{\pgfqpoint{0.887218in}{2.108107in}}%
\pgfpathlineto{\pgfqpoint{0.858083in}{2.124385in}}%
\pgfpathlineto{\pgfqpoint{0.828947in}{2.143105in}}%
\pgfpathlineto{\pgfqpoint{0.799812in}{2.164584in}}%
\pgfpathlineto{\pgfqpoint{0.769805in}{2.190351in}}%
\pgfpathlineto{\pgfqpoint{0.749992in}{2.209649in}}%
\pgfpathlineto{\pgfqpoint{0.731830in}{2.229249in}}%
\pgfpathlineto{\pgfqpoint{0.708690in}{2.257895in}}%
\pgfpathlineto{\pgfqpoint{0.692982in}{2.279813in}}%
\pgfpathlineto{\pgfqpoint{0.671473in}{2.315789in}}%
\pgfpathlineto{\pgfqpoint{0.654135in}{2.351290in}}%
\pgfpathlineto{\pgfqpoint{0.639190in}{2.392982in}}%
\pgfpathlineto{\pgfqpoint{0.631625in}{2.421930in}}%
\pgfpathlineto{\pgfqpoint{0.627024in}{2.450877in}}%
\pgfpathlineto{\pgfqpoint{0.625000in}{2.470343in}}%
\pgfpathlineto{\pgfqpoint{0.625000in}{2.470343in}}%
\pgfusepath{stroke}%
\end{pgfscope}%
\begin{pgfscope}%
\pgfpathrectangle{\pgfqpoint{0.625000in}{0.550000in}}{\pgfqpoint{3.875000in}{3.850000in}} %
\pgfusepath{clip}%
\pgfsetbuttcap%
\pgfsetroundjoin%
\pgfsetlinewidth{0.250937pt}%
\definecolor{currentstroke}{rgb}{0.000000,0.000000,0.000000}%
\pgfsetstrokecolor{currentstroke}%
\pgfsetdash{}{0pt}%
\pgfpathmoveto{\pgfqpoint{0.625000in}{2.643293in}}%
\pgfpathlineto{\pgfqpoint{0.634062in}{2.634211in}}%
\pgfpathlineto{\pgfqpoint{0.625000in}{2.625724in}}%
\pgfusepath{stroke}%
\end{pgfscope}%
\begin{pgfscope}%
\pgfpathrectangle{\pgfqpoint{0.625000in}{0.550000in}}{\pgfqpoint{3.875000in}{3.850000in}} %
\pgfusepath{clip}%
\pgfsetbuttcap%
\pgfsetroundjoin%
\pgfsetlinewidth{0.250937pt}%
\definecolor{currentstroke}{rgb}{0.000000,0.000000,0.000000}%
\pgfsetstrokecolor{currentstroke}%
\pgfsetdash{}{0pt}%
\pgfpathmoveto{\pgfqpoint{0.625000in}{2.797362in}}%
\pgfpathlineto{\pgfqpoint{0.634012in}{2.788596in}}%
\pgfpathlineto{\pgfqpoint{0.625000in}{2.780077in}}%
\pgfusepath{stroke}%
\end{pgfscope}%
\begin{pgfscope}%
\pgfpathrectangle{\pgfqpoint{0.625000in}{0.550000in}}{\pgfqpoint{3.875000in}{3.850000in}} %
\pgfusepath{clip}%
\pgfsetbuttcap%
\pgfsetroundjoin%
\pgfsetlinewidth{0.250937pt}%
\definecolor{currentstroke}{rgb}{0.000000,0.000000,0.000000}%
\pgfsetstrokecolor{currentstroke}%
\pgfsetdash{}{0pt}%
\pgfpathmoveto{\pgfqpoint{0.625000in}{2.951881in}}%
\pgfpathlineto{\pgfqpoint{0.633981in}{2.942982in}}%
\pgfpathlineto{\pgfqpoint{0.625000in}{2.934009in}}%
\pgfusepath{stroke}%
\end{pgfscope}%
\begin{pgfscope}%
\pgfpathrectangle{\pgfqpoint{0.625000in}{0.550000in}}{\pgfqpoint{3.875000in}{3.850000in}} %
\pgfusepath{clip}%
\pgfsetbuttcap%
\pgfsetroundjoin%
\pgfsetlinewidth{0.250937pt}%
\definecolor{currentstroke}{rgb}{0.000000,0.000000,0.000000}%
\pgfsetstrokecolor{currentstroke}%
\pgfsetdash{}{0pt}%
\pgfpathmoveto{\pgfqpoint{0.625000in}{3.105843in}}%
\pgfpathlineto{\pgfqpoint{0.633987in}{3.097368in}}%
\pgfpathlineto{\pgfqpoint{0.625000in}{3.088713in}}%
\pgfusepath{stroke}%
\end{pgfscope}%
\begin{pgfscope}%
\pgfpathrectangle{\pgfqpoint{0.625000in}{0.550000in}}{\pgfqpoint{3.875000in}{3.850000in}} %
\pgfusepath{clip}%
\pgfsetbuttcap%
\pgfsetroundjoin%
\pgfsetlinewidth{0.250937pt}%
\definecolor{currentstroke}{rgb}{0.000000,0.000000,0.000000}%
\pgfsetstrokecolor{currentstroke}%
\pgfsetdash{}{0pt}%
\pgfpathmoveto{\pgfqpoint{0.625000in}{3.260348in}}%
\pgfpathlineto{\pgfqpoint{0.633912in}{3.261404in}}%
\pgfpathlineto{\pgfqpoint{0.634712in}{3.263019in}}%
\pgfpathlineto{\pgfqpoint{0.641457in}{3.261404in}}%
\pgfpathlineto{\pgfqpoint{0.644424in}{3.259348in}}%
\pgfpathlineto{\pgfqpoint{0.649605in}{3.251754in}}%
\pgfpathlineto{\pgfqpoint{0.644424in}{3.244161in}}%
\pgfpathlineto{\pgfqpoint{0.641457in}{3.242105in}}%
\pgfpathlineto{\pgfqpoint{0.634712in}{3.240489in}}%
\pgfpathlineto{\pgfqpoint{0.633955in}{3.242105in}}%
\pgfpathlineto{\pgfqpoint{0.625000in}{3.243217in}}%
\pgfusepath{stroke}%
\end{pgfscope}%
\begin{pgfscope}%
\pgfpathrectangle{\pgfqpoint{0.625000in}{0.550000in}}{\pgfqpoint{3.875000in}{3.850000in}} %
\pgfusepath{clip}%
\pgfsetbuttcap%
\pgfsetroundjoin%
\pgfsetlinewidth{0.250937pt}%
\definecolor{currentstroke}{rgb}{0.000000,0.000000,0.000000}%
\pgfsetstrokecolor{currentstroke}%
\pgfsetdash{}{0pt}%
\pgfpathmoveto{\pgfqpoint{0.625000in}{3.414975in}}%
\pgfpathlineto{\pgfqpoint{0.633984in}{3.406140in}}%
\pgfpathlineto{\pgfqpoint{0.625000in}{3.397593in}}%
\pgfusepath{stroke}%
\end{pgfscope}%
\begin{pgfscope}%
\pgfpathrectangle{\pgfqpoint{0.625000in}{0.550000in}}{\pgfqpoint{3.875000in}{3.850000in}} %
\pgfusepath{clip}%
\pgfsetbuttcap%
\pgfsetroundjoin%
\pgfsetlinewidth{0.250937pt}%
\definecolor{currentstroke}{rgb}{0.000000,0.000000,0.000000}%
\pgfsetstrokecolor{currentstroke}%
\pgfsetdash{}{0pt}%
\pgfpathmoveto{\pgfqpoint{0.625000in}{3.569109in}}%
\pgfpathlineto{\pgfqpoint{0.634033in}{3.560526in}}%
\pgfpathlineto{\pgfqpoint{0.625000in}{3.551948in}}%
\pgfusepath{stroke}%
\end{pgfscope}%
\begin{pgfscope}%
\pgfpathrectangle{\pgfqpoint{0.625000in}{0.550000in}}{\pgfqpoint{3.875000in}{3.850000in}} %
\pgfusepath{clip}%
\pgfsetbuttcap%
\pgfsetroundjoin%
\pgfsetlinewidth{0.250937pt}%
\definecolor{currentstroke}{rgb}{0.000000,0.000000,0.000000}%
\pgfsetstrokecolor{currentstroke}%
\pgfsetdash{}{0pt}%
\pgfpathmoveto{\pgfqpoint{0.625000in}{3.723581in}}%
\pgfpathlineto{\pgfqpoint{0.634052in}{3.714912in}}%
\pgfpathlineto{\pgfqpoint{0.625000in}{3.705982in}}%
\pgfusepath{stroke}%
\end{pgfscope}%
\begin{pgfscope}%
\pgfpathrectangle{\pgfqpoint{0.625000in}{0.550000in}}{\pgfqpoint{3.875000in}{3.850000in}} %
\pgfusepath{clip}%
\pgfsetbuttcap%
\pgfsetroundjoin%
\pgfsetlinewidth{0.250937pt}%
\definecolor{currentstroke}{rgb}{0.000000,0.000000,0.000000}%
\pgfsetstrokecolor{currentstroke}%
\pgfsetdash{}{0pt}%
\pgfpathmoveto{\pgfqpoint{0.625000in}{3.877858in}}%
\pgfpathlineto{\pgfqpoint{0.633995in}{3.869298in}}%
\pgfpathlineto{\pgfqpoint{0.625000in}{3.860704in}}%
\pgfusepath{stroke}%
\end{pgfscope}%
\begin{pgfscope}%
\pgfpathrectangle{\pgfqpoint{0.625000in}{0.550000in}}{\pgfqpoint{3.875000in}{3.850000in}} %
\pgfusepath{clip}%
\pgfsetbuttcap%
\pgfsetroundjoin%
\pgfsetlinewidth{0.250937pt}%
\definecolor{currentstroke}{rgb}{0.000000,0.000000,0.000000}%
\pgfsetstrokecolor{currentstroke}%
\pgfsetdash{}{0pt}%
\pgfpathmoveto{\pgfqpoint{0.625000in}{4.032342in}}%
\pgfpathlineto{\pgfqpoint{0.633874in}{4.033333in}}%
\pgfpathlineto{\pgfqpoint{0.634712in}{4.034949in}}%
\pgfpathlineto{\pgfqpoint{0.641457in}{4.033333in}}%
\pgfpathlineto{\pgfqpoint{0.644424in}{4.031278in}}%
\pgfpathlineto{\pgfqpoint{0.649605in}{4.023684in}}%
\pgfpathlineto{\pgfqpoint{0.644424in}{4.016090in}}%
\pgfpathlineto{\pgfqpoint{0.641457in}{4.014035in}}%
\pgfpathlineto{\pgfqpoint{0.634712in}{4.012419in}}%
\pgfpathlineto{\pgfqpoint{0.633941in}{4.014035in}}%
\pgfpathlineto{\pgfqpoint{0.625000in}{4.015111in}}%
\pgfusepath{stroke}%
\end{pgfscope}%
\begin{pgfscope}%
\pgfpathrectangle{\pgfqpoint{0.625000in}{0.550000in}}{\pgfqpoint{3.875000in}{3.850000in}} %
\pgfusepath{clip}%
\pgfsetbuttcap%
\pgfsetroundjoin%
\pgfsetlinewidth{0.250937pt}%
\definecolor{currentstroke}{rgb}{0.000000,0.000000,0.000000}%
\pgfsetstrokecolor{currentstroke}%
\pgfsetdash{}{0pt}%
\pgfpathmoveto{\pgfqpoint{0.625000in}{4.186734in}}%
\pgfpathlineto{\pgfqpoint{0.634000in}{4.178070in}}%
\pgfpathlineto{\pgfqpoint{0.625000in}{4.169812in}}%
\pgfusepath{stroke}%
\end{pgfscope}%
\begin{pgfscope}%
\pgfpathrectangle{\pgfqpoint{0.625000in}{0.550000in}}{\pgfqpoint{3.875000in}{3.850000in}} %
\pgfusepath{clip}%
\pgfsetbuttcap%
\pgfsetroundjoin%
\pgfsetlinewidth{0.250937pt}%
\definecolor{currentstroke}{rgb}{0.000000,0.000000,0.000000}%
\pgfsetstrokecolor{currentstroke}%
\pgfsetdash{}{0pt}%
\pgfpathmoveto{\pgfqpoint{0.625000in}{4.341069in}}%
\pgfpathlineto{\pgfqpoint{0.633947in}{4.332456in}}%
\pgfpathlineto{\pgfqpoint{0.625000in}{4.323814in}}%
\pgfusepath{stroke}%
\end{pgfscope}%
\begin{pgfscope}%
\pgfpathrectangle{\pgfqpoint{0.625000in}{0.550000in}}{\pgfqpoint{3.875000in}{3.850000in}} %
\pgfusepath{clip}%
\pgfsetbuttcap%
\pgfsetroundjoin%
\pgfsetlinewidth{0.250937pt}%
\definecolor{currentstroke}{rgb}{0.000000,0.000000,0.000000}%
\pgfsetstrokecolor{currentstroke}%
\pgfsetdash{}{0pt}%
\pgfpathmoveto{\pgfqpoint{0.644424in}{1.163314in}}%
\pgfpathlineto{\pgfqpoint{0.640328in}{1.167544in}}%
\pgfpathlineto{\pgfqpoint{0.634712in}{1.170162in}}%
\pgfpathlineto{\pgfqpoint{0.631106in}{1.177193in}}%
\pgfpathlineto{\pgfqpoint{0.627329in}{1.186842in}}%
\pgfpathlineto{\pgfqpoint{0.626023in}{1.196491in}}%
\pgfpathlineto{\pgfqpoint{0.629783in}{1.206140in}}%
\pgfpathlineto{\pgfqpoint{0.634712in}{1.215448in}}%
\pgfpathlineto{\pgfqpoint{0.635778in}{1.215789in}}%
\pgfpathlineto{\pgfqpoint{0.644424in}{1.223125in}}%
\pgfpathlineto{\pgfqpoint{0.652516in}{1.225439in}}%
\pgfpathlineto{\pgfqpoint{0.654135in}{1.226359in}}%
\pgfpathlineto{\pgfqpoint{0.663847in}{1.227483in}}%
\pgfpathlineto{\pgfqpoint{0.673559in}{1.225960in}}%
\pgfpathlineto{\pgfqpoint{0.675104in}{1.225439in}}%
\pgfpathlineto{\pgfqpoint{0.683271in}{1.221640in}}%
\pgfpathlineto{\pgfqpoint{0.690732in}{1.215789in}}%
\pgfpathlineto{\pgfqpoint{0.692982in}{1.212473in}}%
\pgfpathlineto{\pgfqpoint{0.697198in}{1.206140in}}%
\pgfpathlineto{\pgfqpoint{0.699700in}{1.196491in}}%
\pgfpathlineto{\pgfqpoint{0.699229in}{1.186842in}}%
\pgfpathlineto{\pgfqpoint{0.695507in}{1.177193in}}%
\pgfpathlineto{\pgfqpoint{0.692982in}{1.174021in}}%
\pgfpathlineto{\pgfqpoint{0.687295in}{1.167544in}}%
\pgfpathlineto{\pgfqpoint{0.683271in}{1.164860in}}%
\pgfpathlineto{\pgfqpoint{0.673559in}{1.160549in}}%
\pgfpathlineto{\pgfqpoint{0.663847in}{1.159145in}}%
\pgfpathlineto{\pgfqpoint{0.654135in}{1.160004in}}%
\pgfpathlineto{\pgfqpoint{0.644424in}{1.163314in}}%
\pgfusepath{stroke}%
\end{pgfscope}%
\begin{pgfscope}%
\pgfpathrectangle{\pgfqpoint{0.625000in}{0.550000in}}{\pgfqpoint{3.875000in}{3.850000in}} %
\pgfusepath{clip}%
\pgfsetbuttcap%
\pgfsetroundjoin%
\pgfsetlinewidth{0.250937pt}%
\definecolor{currentstroke}{rgb}{0.000000,0.000000,0.000000}%
\pgfsetstrokecolor{currentstroke}%
\pgfsetdash{}{0pt}%
\pgfpathmoveto{\pgfqpoint{0.634712in}{1.378372in}}%
\pgfpathlineto{\pgfqpoint{0.634332in}{1.379825in}}%
\pgfpathlineto{\pgfqpoint{0.634712in}{1.380543in}}%
\pgfpathlineto{\pgfqpoint{0.635756in}{1.379825in}}%
\pgfpathlineto{\pgfqpoint{0.634712in}{1.378372in}}%
\pgfusepath{stroke}%
\end{pgfscope}%
\begin{pgfscope}%
\pgfpathrectangle{\pgfqpoint{0.625000in}{0.550000in}}{\pgfqpoint{3.875000in}{3.850000in}} %
\pgfusepath{clip}%
\pgfsetbuttcap%
\pgfsetroundjoin%
\pgfsetlinewidth{0.250937pt}%
\definecolor{currentstroke}{rgb}{0.000000,0.000000,0.000000}%
\pgfsetstrokecolor{currentstroke}%
\pgfsetdash{}{0pt}%
\pgfpathmoveto{\pgfqpoint{0.634712in}{1.927520in}}%
\pgfpathlineto{\pgfqpoint{0.632625in}{1.929825in}}%
\pgfpathlineto{\pgfqpoint{0.634712in}{1.931553in}}%
\pgfpathlineto{\pgfqpoint{0.636948in}{1.929825in}}%
\pgfpathlineto{\pgfqpoint{0.634712in}{1.927520in}}%
\pgfusepath{stroke}%
\end{pgfscope}%
\begin{pgfscope}%
\pgfpathrectangle{\pgfqpoint{0.625000in}{0.550000in}}{\pgfqpoint{3.875000in}{3.850000in}} %
\pgfusepath{clip}%
\pgfsetbuttcap%
\pgfsetroundjoin%
\pgfsetlinewidth{0.250937pt}%
\definecolor{currentstroke}{rgb}{0.000000,0.000000,0.000000}%
\pgfsetstrokecolor{currentstroke}%
\pgfsetdash{}{0pt}%
\pgfpathmoveto{\pgfqpoint{0.634712in}{3.028096in}}%
\pgfpathlineto{\pgfqpoint{0.632625in}{3.029825in}}%
\pgfpathlineto{\pgfqpoint{0.634712in}{3.032129in}}%
\pgfpathlineto{\pgfqpoint{0.636948in}{3.029825in}}%
\pgfpathlineto{\pgfqpoint{0.634712in}{3.028096in}}%
\pgfusepath{stroke}%
\end{pgfscope}%
\begin{pgfscope}%
\pgfpathrectangle{\pgfqpoint{0.625000in}{0.550000in}}{\pgfqpoint{3.875000in}{3.850000in}} %
\pgfusepath{clip}%
\pgfsetbuttcap%
\pgfsetroundjoin%
\pgfsetlinewidth{0.250937pt}%
\definecolor{currentstroke}{rgb}{0.000000,0.000000,0.000000}%
\pgfsetstrokecolor{currentstroke}%
\pgfsetdash{}{0pt}%
\pgfpathmoveto{\pgfqpoint{0.634712in}{3.579106in}}%
\pgfpathlineto{\pgfqpoint{0.634332in}{3.579825in}}%
\pgfpathlineto{\pgfqpoint{0.634712in}{3.581277in}}%
\pgfpathlineto{\pgfqpoint{0.635756in}{3.579825in}}%
\pgfpathlineto{\pgfqpoint{0.634712in}{3.579106in}}%
\pgfusepath{stroke}%
\end{pgfscope}%
\begin{pgfscope}%
\pgfpathrectangle{\pgfqpoint{0.625000in}{0.550000in}}{\pgfqpoint{3.875000in}{3.850000in}} %
\pgfusepath{clip}%
\pgfsetbuttcap%
\pgfsetroundjoin%
\pgfsetlinewidth{0.250937pt}%
\definecolor{currentstroke}{rgb}{0.000000,0.000000,0.000000}%
\pgfsetstrokecolor{currentstroke}%
\pgfsetdash{}{0pt}%
\pgfpathmoveto{\pgfqpoint{0.654135in}{3.733290in}}%
\pgfpathlineto{\pgfqpoint{0.652516in}{3.734211in}}%
\pgfpathlineto{\pgfqpoint{0.644424in}{3.736524in}}%
\pgfpathlineto{\pgfqpoint{0.635778in}{3.743860in}}%
\pgfpathlineto{\pgfqpoint{0.634712in}{3.744201in}}%
\pgfpathlineto{\pgfqpoint{0.629783in}{3.753509in}}%
\pgfpathlineto{\pgfqpoint{0.626023in}{3.763158in}}%
\pgfpathlineto{\pgfqpoint{0.627329in}{3.772807in}}%
\pgfpathlineto{\pgfqpoint{0.631106in}{3.782456in}}%
\pgfpathlineto{\pgfqpoint{0.634712in}{3.789487in}}%
\pgfpathlineto{\pgfqpoint{0.640328in}{3.792105in}}%
\pgfpathlineto{\pgfqpoint{0.644424in}{3.796335in}}%
\pgfpathlineto{\pgfqpoint{0.654135in}{3.799645in}}%
\pgfpathlineto{\pgfqpoint{0.663847in}{3.800504in}}%
\pgfpathlineto{\pgfqpoint{0.673559in}{3.799100in}}%
\pgfpathlineto{\pgfqpoint{0.683271in}{3.794789in}}%
\pgfpathlineto{\pgfqpoint{0.687295in}{3.792105in}}%
\pgfpathlineto{\pgfqpoint{0.692982in}{3.785629in}}%
\pgfpathlineto{\pgfqpoint{0.695507in}{3.782456in}}%
\pgfpathlineto{\pgfqpoint{0.699229in}{3.772807in}}%
\pgfpathlineto{\pgfqpoint{0.699700in}{3.763158in}}%
\pgfpathlineto{\pgfqpoint{0.697198in}{3.753509in}}%
\pgfpathlineto{\pgfqpoint{0.692982in}{3.747177in}}%
\pgfpathlineto{\pgfqpoint{0.690732in}{3.743860in}}%
\pgfpathlineto{\pgfqpoint{0.683271in}{3.738010in}}%
\pgfpathlineto{\pgfqpoint{0.675104in}{3.734211in}}%
\pgfpathlineto{\pgfqpoint{0.673559in}{3.733689in}}%
\pgfpathlineto{\pgfqpoint{0.663847in}{3.732167in}}%
\pgfpathlineto{\pgfqpoint{0.654135in}{3.733290in}}%
\pgfusepath{stroke}%
\end{pgfscope}%
\begin{pgfscope}%
\pgfpathrectangle{\pgfqpoint{0.625000in}{0.550000in}}{\pgfqpoint{3.875000in}{3.850000in}} %
\pgfusepath{clip}%
\pgfsetbuttcap%
\pgfsetroundjoin%
\pgfsetlinewidth{0.250937pt}%
\definecolor{currentstroke}{rgb}{0.000000,0.000000,0.000000}%
\pgfsetstrokecolor{currentstroke}%
\pgfsetdash{}{0pt}%
\pgfpathmoveto{\pgfqpoint{0.625000in}{0.636045in}}%
\pgfpathlineto{\pgfqpoint{0.634129in}{0.627193in}}%
\pgfpathlineto{\pgfqpoint{0.625000in}{0.618367in}}%
\pgfusepath{stroke}%
\end{pgfscope}%
\begin{pgfscope}%
\pgfpathrectangle{\pgfqpoint{0.625000in}{0.550000in}}{\pgfqpoint{3.875000in}{3.850000in}} %
\pgfusepath{clip}%
\pgfsetbuttcap%
\pgfsetroundjoin%
\pgfsetlinewidth{0.250937pt}%
\definecolor{currentstroke}{rgb}{0.000000,0.000000,0.000000}%
\pgfsetstrokecolor{currentstroke}%
\pgfsetdash{}{0pt}%
\pgfpathmoveto{\pgfqpoint{0.625000in}{0.790038in}}%
\pgfpathlineto{\pgfqpoint{0.634151in}{0.781579in}}%
\pgfpathlineto{\pgfqpoint{0.625000in}{0.772743in}}%
\pgfusepath{stroke}%
\end{pgfscope}%
\begin{pgfscope}%
\pgfpathrectangle{\pgfqpoint{0.625000in}{0.550000in}}{\pgfqpoint{3.875000in}{3.850000in}} %
\pgfusepath{clip}%
\pgfsetbuttcap%
\pgfsetroundjoin%
\pgfsetlinewidth{0.250937pt}%
\definecolor{currentstroke}{rgb}{0.000000,0.000000,0.000000}%
\pgfsetstrokecolor{currentstroke}%
\pgfsetdash{}{0pt}%
\pgfpathmoveto{\pgfqpoint{0.625000in}{0.944706in}}%
\pgfpathlineto{\pgfqpoint{0.633246in}{0.945614in}}%
\pgfpathlineto{\pgfqpoint{0.634712in}{0.948687in}}%
\pgfpathlineto{\pgfqpoint{0.644424in}{0.947112in}}%
\pgfpathlineto{\pgfqpoint{0.647156in}{0.945614in}}%
\pgfpathlineto{\pgfqpoint{0.652481in}{0.935965in}}%
\pgfpathlineto{\pgfqpoint{0.647156in}{0.926316in}}%
\pgfpathlineto{\pgfqpoint{0.644424in}{0.924818in}}%
\pgfpathlineto{\pgfqpoint{0.634712in}{0.923243in}}%
\pgfpathlineto{\pgfqpoint{0.633118in}{0.926316in}}%
\pgfpathlineto{\pgfqpoint{0.625000in}{0.927145in}}%
\pgfusepath{stroke}%
\end{pgfscope}%
\begin{pgfscope}%
\pgfpathrectangle{\pgfqpoint{0.625000in}{0.550000in}}{\pgfqpoint{3.875000in}{3.850000in}} %
\pgfusepath{clip}%
\pgfsetbuttcap%
\pgfsetroundjoin%
\pgfsetlinewidth{0.250937pt}%
\definecolor{currentstroke}{rgb}{0.000000,0.000000,0.000000}%
\pgfsetstrokecolor{currentstroke}%
\pgfsetdash{}{0pt}%
\pgfpathmoveto{\pgfqpoint{0.625000in}{1.099121in}}%
\pgfpathlineto{\pgfqpoint{0.634146in}{1.090351in}}%
\pgfpathlineto{\pgfqpoint{0.625000in}{1.081613in}}%
\pgfusepath{stroke}%
\end{pgfscope}%
\begin{pgfscope}%
\pgfpathrectangle{\pgfqpoint{0.625000in}{0.550000in}}{\pgfqpoint{3.875000in}{3.850000in}} %
\pgfusepath{clip}%
\pgfsetbuttcap%
\pgfsetroundjoin%
\pgfsetlinewidth{0.250937pt}%
\definecolor{currentstroke}{rgb}{0.000000,0.000000,0.000000}%
\pgfsetstrokecolor{currentstroke}%
\pgfsetdash{}{0pt}%
\pgfpathmoveto{\pgfqpoint{0.625000in}{1.253792in}}%
\pgfpathlineto{\pgfqpoint{0.634175in}{1.244737in}}%
\pgfpathlineto{\pgfqpoint{0.625000in}{1.235931in}}%
\pgfusepath{stroke}%
\end{pgfscope}%
\begin{pgfscope}%
\pgfpathrectangle{\pgfqpoint{0.625000in}{0.550000in}}{\pgfqpoint{3.875000in}{3.850000in}} %
\pgfusepath{clip}%
\pgfsetbuttcap%
\pgfsetroundjoin%
\pgfsetlinewidth{0.250937pt}%
\definecolor{currentstroke}{rgb}{0.000000,0.000000,0.000000}%
\pgfsetstrokecolor{currentstroke}%
\pgfsetdash{}{0pt}%
\pgfpathmoveto{\pgfqpoint{0.625000in}{1.407823in}}%
\pgfpathlineto{\pgfqpoint{0.634146in}{1.399123in}}%
\pgfpathlineto{\pgfqpoint{0.625000in}{1.390417in}}%
\pgfusepath{stroke}%
\end{pgfscope}%
\begin{pgfscope}%
\pgfpathrectangle{\pgfqpoint{0.625000in}{0.550000in}}{\pgfqpoint{3.875000in}{3.850000in}} %
\pgfusepath{clip}%
\pgfsetbuttcap%
\pgfsetroundjoin%
\pgfsetlinewidth{0.250937pt}%
\definecolor{currentstroke}{rgb}{0.000000,0.000000,0.000000}%
\pgfsetstrokecolor{currentstroke}%
\pgfsetdash{}{0pt}%
\pgfpathmoveto{\pgfqpoint{0.625000in}{1.562172in}}%
\pgfpathlineto{\pgfqpoint{0.634093in}{1.553509in}}%
\pgfpathlineto{\pgfqpoint{0.625000in}{1.544564in}}%
\pgfusepath{stroke}%
\end{pgfscope}%
\begin{pgfscope}%
\pgfpathrectangle{\pgfqpoint{0.625000in}{0.550000in}}{\pgfqpoint{3.875000in}{3.850000in}} %
\pgfusepath{clip}%
\pgfsetbuttcap%
\pgfsetroundjoin%
\pgfsetlinewidth{0.250937pt}%
\definecolor{currentstroke}{rgb}{0.000000,0.000000,0.000000}%
\pgfsetstrokecolor{currentstroke}%
\pgfsetdash{}{0pt}%
\pgfpathmoveto{\pgfqpoint{0.625000in}{1.716564in}}%
\pgfpathlineto{\pgfqpoint{0.633272in}{1.717544in}}%
\pgfpathlineto{\pgfqpoint{0.634712in}{1.720617in}}%
\pgfpathlineto{\pgfqpoint{0.644424in}{1.719042in}}%
\pgfpathlineto{\pgfqpoint{0.647156in}{1.717544in}}%
\pgfpathlineto{\pgfqpoint{0.652481in}{1.707895in}}%
\pgfpathlineto{\pgfqpoint{0.647156in}{1.698246in}}%
\pgfpathlineto{\pgfqpoint{0.644424in}{1.696748in}}%
\pgfpathlineto{\pgfqpoint{0.634712in}{1.695173in}}%
\pgfpathlineto{\pgfqpoint{0.633192in}{1.698246in}}%
\pgfpathlineto{\pgfqpoint{0.625000in}{1.699171in}}%
\pgfusepath{stroke}%
\end{pgfscope}%
\begin{pgfscope}%
\pgfpathrectangle{\pgfqpoint{0.625000in}{0.550000in}}{\pgfqpoint{3.875000in}{3.850000in}} %
\pgfusepath{clip}%
\pgfsetbuttcap%
\pgfsetroundjoin%
\pgfsetlinewidth{0.250937pt}%
\definecolor{currentstroke}{rgb}{0.000000,0.000000,0.000000}%
\pgfsetstrokecolor{currentstroke}%
\pgfsetdash{}{0pt}%
\pgfpathmoveto{\pgfqpoint{0.625000in}{1.871032in}}%
\pgfpathlineto{\pgfqpoint{0.634083in}{1.862281in}}%
\pgfpathlineto{\pgfqpoint{0.625000in}{1.853710in}}%
\pgfusepath{stroke}%
\end{pgfscope}%
\begin{pgfscope}%
\pgfpathrectangle{\pgfqpoint{0.625000in}{0.550000in}}{\pgfqpoint{3.875000in}{3.850000in}} %
\pgfusepath{clip}%
\pgfsetbuttcap%
\pgfsetroundjoin%
\pgfsetlinewidth{0.250937pt}%
\definecolor{currentstroke}{rgb}{0.000000,0.000000,0.000000}%
\pgfsetstrokecolor{currentstroke}%
\pgfsetdash{}{0pt}%
\pgfpathmoveto{\pgfqpoint{0.625000in}{2.025739in}}%
\pgfpathlineto{\pgfqpoint{0.634081in}{2.016667in}}%
\pgfpathlineto{\pgfqpoint{0.625000in}{2.007668in}}%
\pgfusepath{stroke}%
\end{pgfscope}%
\begin{pgfscope}%
\pgfpathrectangle{\pgfqpoint{0.625000in}{0.550000in}}{\pgfqpoint{3.875000in}{3.850000in}} %
\pgfusepath{clip}%
\pgfsetbuttcap%
\pgfsetroundjoin%
\pgfsetlinewidth{0.250937pt}%
\definecolor{currentstroke}{rgb}{0.000000,0.000000,0.000000}%
\pgfsetstrokecolor{currentstroke}%
\pgfsetdash{}{0pt}%
\pgfpathmoveto{\pgfqpoint{0.625000in}{2.179661in}}%
\pgfpathlineto{\pgfqpoint{0.634104in}{2.171053in}}%
\pgfpathlineto{\pgfqpoint{0.625000in}{2.162198in}}%
\pgfusepath{stroke}%
\end{pgfscope}%
\begin{pgfscope}%
\pgfpathrectangle{\pgfqpoint{0.625000in}{0.550000in}}{\pgfqpoint{3.875000in}{3.850000in}} %
\pgfusepath{clip}%
\pgfsetbuttcap%
\pgfsetroundjoin%
\pgfsetlinewidth{0.250937pt}%
\definecolor{currentstroke}{rgb}{0.000000,0.000000,0.000000}%
\pgfsetstrokecolor{currentstroke}%
\pgfsetdash{}{0pt}%
\pgfpathmoveto{\pgfqpoint{0.625000in}{2.334010in}}%
\pgfpathlineto{\pgfqpoint{0.634151in}{2.325439in}}%
\pgfpathlineto{\pgfqpoint{0.625000in}{2.316266in}}%
\pgfusepath{stroke}%
\end{pgfscope}%
\begin{pgfscope}%
\pgfpathrectangle{\pgfqpoint{0.625000in}{0.550000in}}{\pgfqpoint{3.875000in}{3.850000in}} %
\pgfusepath{clip}%
\pgfsetbuttcap%
\pgfsetroundjoin%
\pgfsetlinewidth{0.250937pt}%
\definecolor{currentstroke}{rgb}{0.000000,0.000000,0.000000}%
\pgfsetstrokecolor{currentstroke}%
\pgfsetdash{}{0pt}%
\pgfpathmoveto{\pgfqpoint{0.625000in}{2.489390in}}%
\pgfpathlineto{\pgfqpoint{0.625079in}{2.489474in}}%
\pgfpathlineto{\pgfqpoint{0.628042in}{2.518421in}}%
\pgfpathlineto{\pgfqpoint{0.632499in}{2.547368in}}%
\pgfpathlineto{\pgfqpoint{0.650076in}{2.605263in}}%
\pgfpathlineto{\pgfqpoint{0.663847in}{2.639405in}}%
\pgfpathlineto{\pgfqpoint{0.685906in}{2.682456in}}%
\pgfpathlineto{\pgfqpoint{0.703364in}{2.711404in}}%
\pgfpathlineto{\pgfqpoint{0.723292in}{2.740351in}}%
\pgfpathlineto{\pgfqpoint{0.751253in}{2.775952in}}%
\pgfpathlineto{\pgfqpoint{0.790100in}{2.817843in}}%
\pgfpathlineto{\pgfqpoint{0.820868in}{2.846491in}}%
\pgfpathlineto{\pgfqpoint{0.848371in}{2.869533in}}%
\pgfpathlineto{\pgfqpoint{0.887218in}{2.898247in}}%
\pgfpathlineto{\pgfqpoint{0.926832in}{2.923684in}}%
\pgfpathlineto{\pgfqpoint{0.960993in}{2.942982in}}%
\pgfpathlineto{\pgfqpoint{0.984336in}{2.954984in}}%
\pgfpathlineto{\pgfqpoint{1.023183in}{2.972677in}}%
\pgfpathlineto{\pgfqpoint{1.062030in}{2.987942in}}%
\pgfpathlineto{\pgfqpoint{1.100940in}{3.000877in}}%
\pgfpathlineto{\pgfqpoint{1.139724in}{3.011656in}}%
\pgfpathlineto{\pgfqpoint{1.178571in}{3.020367in}}%
\pgfpathlineto{\pgfqpoint{1.217419in}{3.027110in}}%
\pgfpathlineto{\pgfqpoint{1.256266in}{3.031883in}}%
\pgfpathlineto{\pgfqpoint{1.295113in}{3.034743in}}%
\pgfpathlineto{\pgfqpoint{1.333960in}{3.035683in}}%
\pgfpathlineto{\pgfqpoint{1.372807in}{3.034705in}}%
\pgfpathlineto{\pgfqpoint{1.411654in}{3.031780in}}%
\pgfpathlineto{\pgfqpoint{1.450501in}{3.026875in}}%
\pgfpathlineto{\pgfqpoint{1.489348in}{3.019917in}}%
\pgfpathlineto{\pgfqpoint{1.529334in}{3.010526in}}%
\pgfpathlineto{\pgfqpoint{1.567043in}{2.999483in}}%
\pgfpathlineto{\pgfqpoint{1.605890in}{2.985719in}}%
\pgfpathlineto{\pgfqpoint{1.644737in}{2.969326in}}%
\pgfpathlineto{\pgfqpoint{1.683584in}{2.950016in}}%
\pgfpathlineto{\pgfqpoint{1.712824in}{2.933333in}}%
\pgfpathlineto{\pgfqpoint{1.742829in}{2.914035in}}%
\pgfpathlineto{\pgfqpoint{1.770990in}{2.893624in}}%
\pgfpathlineto{\pgfqpoint{1.804711in}{2.865789in}}%
\pgfpathlineto{\pgfqpoint{1.835051in}{2.836842in}}%
\pgfpathlineto{\pgfqpoint{1.861426in}{2.807895in}}%
\pgfpathlineto{\pgfqpoint{1.884424in}{2.778947in}}%
\pgfpathlineto{\pgfqpoint{1.906955in}{2.746033in}}%
\pgfpathlineto{\pgfqpoint{1.921857in}{2.721053in}}%
\pgfpathlineto{\pgfqpoint{1.936822in}{2.692105in}}%
\pgfpathlineto{\pgfqpoint{1.949594in}{2.663158in}}%
\pgfpathlineto{\pgfqpoint{1.960268in}{2.634211in}}%
\pgfpathlineto{\pgfqpoint{1.968967in}{2.605263in}}%
\pgfpathlineto{\pgfqpoint{1.975763in}{2.576316in}}%
\pgfpathlineto{\pgfqpoint{1.980764in}{2.547368in}}%
\pgfpathlineto{\pgfqpoint{1.983952in}{2.518421in}}%
\pgfpathlineto{\pgfqpoint{1.985403in}{2.489474in}}%
\pgfpathlineto{\pgfqpoint{1.985111in}{2.460526in}}%
\pgfpathlineto{\pgfqpoint{1.983086in}{2.431579in}}%
\pgfpathlineto{\pgfqpoint{1.979303in}{2.402632in}}%
\pgfpathlineto{\pgfqpoint{1.973701in}{2.373684in}}%
\pgfpathlineto{\pgfqpoint{1.965226in}{2.341291in}}%
\pgfpathlineto{\pgfqpoint{1.955514in}{2.311986in}}%
\pgfpathlineto{\pgfqpoint{1.945560in}{2.286842in}}%
\pgfpathlineto{\pgfqpoint{1.932099in}{2.257895in}}%
\pgfpathlineto{\pgfqpoint{1.916318in}{2.228947in}}%
\pgfpathlineto{\pgfqpoint{1.897243in}{2.198792in}}%
\pgfpathlineto{\pgfqpoint{1.877093in}{2.171053in}}%
\pgfpathlineto{\pgfqpoint{1.853037in}{2.142105in}}%
\pgfpathlineto{\pgfqpoint{1.825413in}{2.113158in}}%
\pgfpathlineto{\pgfqpoint{1.800125in}{2.089859in}}%
\pgfpathlineto{\pgfqpoint{1.769552in}{2.064912in}}%
\pgfpathlineto{\pgfqpoint{1.741855in}{2.044956in}}%
\pgfpathlineto{\pgfqpoint{1.712719in}{2.026253in}}%
\pgfpathlineto{\pgfqpoint{1.673872in}{2.004513in}}%
\pgfpathlineto{\pgfqpoint{1.635025in}{1.985964in}}%
\pgfpathlineto{\pgfqpoint{1.596178in}{1.970252in}}%
\pgfpathlineto{\pgfqpoint{1.557331in}{1.957110in}}%
\pgfpathlineto{\pgfqpoint{1.518484in}{1.946340in}}%
\pgfpathlineto{\pgfqpoint{1.479637in}{1.937795in}}%
\pgfpathlineto{\pgfqpoint{1.440789in}{1.931361in}}%
\pgfpathlineto{\pgfqpoint{1.401942in}{1.926953in}}%
\pgfpathlineto{\pgfqpoint{1.363095in}{1.924519in}}%
\pgfpathlineto{\pgfqpoint{1.324248in}{1.924022in}}%
\pgfpathlineto{\pgfqpoint{1.285401in}{1.925440in}}%
\pgfpathlineto{\pgfqpoint{1.246554in}{1.928785in}}%
\pgfpathlineto{\pgfqpoint{1.207707in}{1.934032in}}%
\pgfpathlineto{\pgfqpoint{1.168860in}{1.941259in}}%
\pgfpathlineto{\pgfqpoint{1.130013in}{1.950490in}}%
\pgfpathlineto{\pgfqpoint{1.091165in}{1.961792in}}%
\pgfpathlineto{\pgfqpoint{1.052318in}{1.975299in}}%
\pgfpathlineto{\pgfqpoint{1.013471in}{1.991136in}}%
\pgfpathlineto{\pgfqpoint{0.974624in}{2.009502in}}%
\pgfpathlineto{\pgfqpoint{0.935777in}{2.030612in}}%
\pgfpathlineto{\pgfqpoint{0.896364in}{2.055263in}}%
\pgfpathlineto{\pgfqpoint{0.867794in}{2.075303in}}%
\pgfpathlineto{\pgfqpoint{0.838659in}{2.097944in}}%
\pgfpathlineto{\pgfqpoint{0.799728in}{2.132456in}}%
\pgfpathlineto{\pgfqpoint{0.770677in}{2.161876in}}%
\pgfpathlineto{\pgfqpoint{0.741541in}{2.195411in}}%
\pgfpathlineto{\pgfqpoint{0.709775in}{2.238596in}}%
\pgfpathlineto{\pgfqpoint{0.691458in}{2.267544in}}%
\pgfpathlineto{\pgfqpoint{0.673559in}{2.299967in}}%
\pgfpathlineto{\pgfqpoint{0.653644in}{2.344737in}}%
\pgfpathlineto{\pgfqpoint{0.640536in}{2.383333in}}%
\pgfpathlineto{\pgfqpoint{0.632499in}{2.412281in}}%
\pgfpathlineto{\pgfqpoint{0.625808in}{2.460526in}}%
\pgfpathlineto{\pgfqpoint{0.625000in}{2.470259in}}%
\pgfpathlineto{\pgfqpoint{0.625000in}{2.470259in}}%
\pgfusepath{stroke}%
\end{pgfscope}%
\begin{pgfscope}%
\pgfpathrectangle{\pgfqpoint{0.625000in}{0.550000in}}{\pgfqpoint{3.875000in}{3.850000in}} %
\pgfusepath{clip}%
\pgfsetbuttcap%
\pgfsetroundjoin%
\pgfsetlinewidth{0.250937pt}%
\definecolor{currentstroke}{rgb}{0.000000,0.000000,0.000000}%
\pgfsetstrokecolor{currentstroke}%
\pgfsetdash{}{0pt}%
\pgfpathmoveto{\pgfqpoint{0.625000in}{2.643382in}}%
\pgfpathlineto{\pgfqpoint{0.634150in}{2.634211in}}%
\pgfpathlineto{\pgfqpoint{0.625000in}{2.625641in}}%
\pgfusepath{stroke}%
\end{pgfscope}%
\begin{pgfscope}%
\pgfpathrectangle{\pgfqpoint{0.625000in}{0.550000in}}{\pgfqpoint{3.875000in}{3.850000in}} %
\pgfusepath{clip}%
\pgfsetbuttcap%
\pgfsetroundjoin%
\pgfsetlinewidth{0.250937pt}%
\definecolor{currentstroke}{rgb}{0.000000,0.000000,0.000000}%
\pgfsetstrokecolor{currentstroke}%
\pgfsetdash{}{0pt}%
\pgfpathmoveto{\pgfqpoint{0.625000in}{2.797447in}}%
\pgfpathlineto{\pgfqpoint{0.634100in}{2.788596in}}%
\pgfpathlineto{\pgfqpoint{0.625000in}{2.779994in}}%
\pgfusepath{stroke}%
\end{pgfscope}%
\begin{pgfscope}%
\pgfpathrectangle{\pgfqpoint{0.625000in}{0.550000in}}{\pgfqpoint{3.875000in}{3.850000in}} %
\pgfusepath{clip}%
\pgfsetbuttcap%
\pgfsetroundjoin%
\pgfsetlinewidth{0.250937pt}%
\definecolor{currentstroke}{rgb}{0.000000,0.000000,0.000000}%
\pgfsetstrokecolor{currentstroke}%
\pgfsetdash{}{0pt}%
\pgfpathmoveto{\pgfqpoint{0.625000in}{2.951970in}}%
\pgfpathlineto{\pgfqpoint{0.634070in}{2.942982in}}%
\pgfpathlineto{\pgfqpoint{0.625000in}{2.933920in}}%
\pgfusepath{stroke}%
\end{pgfscope}%
\begin{pgfscope}%
\pgfpathrectangle{\pgfqpoint{0.625000in}{0.550000in}}{\pgfqpoint{3.875000in}{3.850000in}} %
\pgfusepath{clip}%
\pgfsetbuttcap%
\pgfsetroundjoin%
\pgfsetlinewidth{0.250937pt}%
\definecolor{currentstroke}{rgb}{0.000000,0.000000,0.000000}%
\pgfsetstrokecolor{currentstroke}%
\pgfsetdash{}{0pt}%
\pgfpathmoveto{\pgfqpoint{0.625000in}{3.105925in}}%
\pgfpathlineto{\pgfqpoint{0.634074in}{3.097368in}}%
\pgfpathlineto{\pgfqpoint{0.625000in}{3.088629in}}%
\pgfusepath{stroke}%
\end{pgfscope}%
\begin{pgfscope}%
\pgfpathrectangle{\pgfqpoint{0.625000in}{0.550000in}}{\pgfqpoint{3.875000in}{3.850000in}} %
\pgfusepath{clip}%
\pgfsetbuttcap%
\pgfsetroundjoin%
\pgfsetlinewidth{0.250937pt}%
\definecolor{currentstroke}{rgb}{0.000000,0.000000,0.000000}%
\pgfsetstrokecolor{currentstroke}%
\pgfsetdash{}{0pt}%
\pgfpathmoveto{\pgfqpoint{0.625000in}{3.260433in}}%
\pgfpathlineto{\pgfqpoint{0.633192in}{3.261404in}}%
\pgfpathlineto{\pgfqpoint{0.634712in}{3.264476in}}%
\pgfpathlineto{\pgfqpoint{0.644424in}{3.262901in}}%
\pgfpathlineto{\pgfqpoint{0.647156in}{3.261404in}}%
\pgfpathlineto{\pgfqpoint{0.652481in}{3.251754in}}%
\pgfpathlineto{\pgfqpoint{0.647156in}{3.242105in}}%
\pgfpathlineto{\pgfqpoint{0.644424in}{3.240608in}}%
\pgfpathlineto{\pgfqpoint{0.634712in}{3.239033in}}%
\pgfpathlineto{\pgfqpoint{0.633272in}{3.242105in}}%
\pgfpathlineto{\pgfqpoint{0.625000in}{3.243132in}}%
\pgfusepath{stroke}%
\end{pgfscope}%
\begin{pgfscope}%
\pgfpathrectangle{\pgfqpoint{0.625000in}{0.550000in}}{\pgfqpoint{3.875000in}{3.850000in}} %
\pgfusepath{clip}%
\pgfsetbuttcap%
\pgfsetroundjoin%
\pgfsetlinewidth{0.250937pt}%
\definecolor{currentstroke}{rgb}{0.000000,0.000000,0.000000}%
\pgfsetstrokecolor{currentstroke}%
\pgfsetdash{}{0pt}%
\pgfpathmoveto{\pgfqpoint{0.625000in}{3.415061in}}%
\pgfpathlineto{\pgfqpoint{0.634071in}{3.406140in}}%
\pgfpathlineto{\pgfqpoint{0.625000in}{3.397510in}}%
\pgfusepath{stroke}%
\end{pgfscope}%
\begin{pgfscope}%
\pgfpathrectangle{\pgfqpoint{0.625000in}{0.550000in}}{\pgfqpoint{3.875000in}{3.850000in}} %
\pgfusepath{clip}%
\pgfsetbuttcap%
\pgfsetroundjoin%
\pgfsetlinewidth{0.250937pt}%
\definecolor{currentstroke}{rgb}{0.000000,0.000000,0.000000}%
\pgfsetstrokecolor{currentstroke}%
\pgfsetdash{}{0pt}%
\pgfpathmoveto{\pgfqpoint{0.625000in}{3.569194in}}%
\pgfpathlineto{\pgfqpoint{0.634123in}{3.560526in}}%
\pgfpathlineto{\pgfqpoint{0.625000in}{3.551863in}}%
\pgfusepath{stroke}%
\end{pgfscope}%
\begin{pgfscope}%
\pgfpathrectangle{\pgfqpoint{0.625000in}{0.550000in}}{\pgfqpoint{3.875000in}{3.850000in}} %
\pgfusepath{clip}%
\pgfsetbuttcap%
\pgfsetroundjoin%
\pgfsetlinewidth{0.250937pt}%
\definecolor{currentstroke}{rgb}{0.000000,0.000000,0.000000}%
\pgfsetstrokecolor{currentstroke}%
\pgfsetdash{}{0pt}%
\pgfpathmoveto{\pgfqpoint{0.625000in}{3.723667in}}%
\pgfpathlineto{\pgfqpoint{0.634141in}{3.714912in}}%
\pgfpathlineto{\pgfqpoint{0.625000in}{3.705894in}}%
\pgfusepath{stroke}%
\end{pgfscope}%
\begin{pgfscope}%
\pgfpathrectangle{\pgfqpoint{0.625000in}{0.550000in}}{\pgfqpoint{3.875000in}{3.850000in}} %
\pgfusepath{clip}%
\pgfsetbuttcap%
\pgfsetroundjoin%
\pgfsetlinewidth{0.250937pt}%
\definecolor{currentstroke}{rgb}{0.000000,0.000000,0.000000}%
\pgfsetstrokecolor{currentstroke}%
\pgfsetdash{}{0pt}%
\pgfpathmoveto{\pgfqpoint{0.625000in}{3.877944in}}%
\pgfpathlineto{\pgfqpoint{0.634086in}{3.869298in}}%
\pgfpathlineto{\pgfqpoint{0.625000in}{3.860617in}}%
\pgfusepath{stroke}%
\end{pgfscope}%
\begin{pgfscope}%
\pgfpathrectangle{\pgfqpoint{0.625000in}{0.550000in}}{\pgfqpoint{3.875000in}{3.850000in}} %
\pgfusepath{clip}%
\pgfsetbuttcap%
\pgfsetroundjoin%
\pgfsetlinewidth{0.250937pt}%
\definecolor{currentstroke}{rgb}{0.000000,0.000000,0.000000}%
\pgfsetstrokecolor{currentstroke}%
\pgfsetdash{}{0pt}%
\pgfpathmoveto{\pgfqpoint{0.625000in}{4.032426in}}%
\pgfpathlineto{\pgfqpoint{0.633118in}{4.033333in}}%
\pgfpathlineto{\pgfqpoint{0.634712in}{4.036406in}}%
\pgfpathlineto{\pgfqpoint{0.644424in}{4.034831in}}%
\pgfpathlineto{\pgfqpoint{0.647156in}{4.033333in}}%
\pgfpathlineto{\pgfqpoint{0.652481in}{4.023684in}}%
\pgfpathlineto{\pgfqpoint{0.647156in}{4.014035in}}%
\pgfpathlineto{\pgfqpoint{0.644424in}{4.012537in}}%
\pgfpathlineto{\pgfqpoint{0.634712in}{4.010962in}}%
\pgfpathlineto{\pgfqpoint{0.633246in}{4.014035in}}%
\pgfpathlineto{\pgfqpoint{0.625000in}{4.015027in}}%
\pgfusepath{stroke}%
\end{pgfscope}%
\begin{pgfscope}%
\pgfpathrectangle{\pgfqpoint{0.625000in}{0.550000in}}{\pgfqpoint{3.875000in}{3.850000in}} %
\pgfusepath{clip}%
\pgfsetbuttcap%
\pgfsetroundjoin%
\pgfsetlinewidth{0.250937pt}%
\definecolor{currentstroke}{rgb}{0.000000,0.000000,0.000000}%
\pgfsetstrokecolor{currentstroke}%
\pgfsetdash{}{0pt}%
\pgfpathmoveto{\pgfqpoint{0.625000in}{4.186822in}}%
\pgfpathlineto{\pgfqpoint{0.634092in}{4.178070in}}%
\pgfpathlineto{\pgfqpoint{0.625000in}{4.169728in}}%
\pgfusepath{stroke}%
\end{pgfscope}%
\begin{pgfscope}%
\pgfpathrectangle{\pgfqpoint{0.625000in}{0.550000in}}{\pgfqpoint{3.875000in}{3.850000in}} %
\pgfusepath{clip}%
\pgfsetbuttcap%
\pgfsetroundjoin%
\pgfsetlinewidth{0.250937pt}%
\definecolor{currentstroke}{rgb}{0.000000,0.000000,0.000000}%
\pgfsetstrokecolor{currentstroke}%
\pgfsetdash{}{0pt}%
\pgfpathmoveto{\pgfqpoint{0.625000in}{4.341157in}}%
\pgfpathlineto{\pgfqpoint{0.634038in}{4.332456in}}%
\pgfpathlineto{\pgfqpoint{0.625000in}{4.323726in}}%
\pgfusepath{stroke}%
\end{pgfscope}%
\begin{pgfscope}%
\pgfpathrectangle{\pgfqpoint{0.625000in}{0.550000in}}{\pgfqpoint{3.875000in}{3.850000in}} %
\pgfusepath{clip}%
\pgfsetbuttcap%
\pgfsetroundjoin%
\pgfsetlinewidth{0.250937pt}%
\definecolor{currentstroke}{rgb}{0.000000,0.000000,0.000000}%
\pgfsetstrokecolor{currentstroke}%
\pgfsetdash{}{0pt}%
\pgfpathmoveto{\pgfqpoint{0.634712in}{0.825019in}}%
\pgfpathlineto{\pgfqpoint{0.634336in}{0.829825in}}%
\pgfpathlineto{\pgfqpoint{0.634712in}{0.830716in}}%
\pgfpathlineto{\pgfqpoint{0.636192in}{0.829825in}}%
\pgfpathlineto{\pgfqpoint{0.634712in}{0.825019in}}%
\pgfusepath{stroke}%
\end{pgfscope}%
\begin{pgfscope}%
\pgfpathrectangle{\pgfqpoint{0.625000in}{0.550000in}}{\pgfqpoint{3.875000in}{3.850000in}} %
\pgfusepath{clip}%
\pgfsetbuttcap%
\pgfsetroundjoin%
\pgfsetlinewidth{0.250937pt}%
\definecolor{currentstroke}{rgb}{0.000000,0.000000,0.000000}%
\pgfsetstrokecolor{currentstroke}%
\pgfsetdash{}{0pt}%
\pgfpathmoveto{\pgfqpoint{0.654135in}{1.156714in}}%
\pgfpathlineto{\pgfqpoint{0.652472in}{1.157895in}}%
\pgfpathlineto{\pgfqpoint{0.644424in}{1.160967in}}%
\pgfpathlineto{\pgfqpoint{0.638056in}{1.167544in}}%
\pgfpathlineto{\pgfqpoint{0.634712in}{1.169103in}}%
\pgfpathlineto{\pgfqpoint{0.630563in}{1.177193in}}%
\pgfpathlineto{\pgfqpoint{0.627122in}{1.186842in}}%
\pgfpathlineto{\pgfqpoint{0.625848in}{1.196491in}}%
\pgfpathlineto{\pgfqpoint{0.629403in}{1.206140in}}%
\pgfpathlineto{\pgfqpoint{0.634344in}{1.215789in}}%
\pgfpathlineto{\pgfqpoint{0.634712in}{1.216929in}}%
\pgfpathlineto{\pgfqpoint{0.644424in}{1.225027in}}%
\pgfpathlineto{\pgfqpoint{0.645864in}{1.225439in}}%
\pgfpathlineto{\pgfqpoint{0.654135in}{1.230141in}}%
\pgfpathlineto{\pgfqpoint{0.663847in}{1.232256in}}%
\pgfpathlineto{\pgfqpoint{0.673559in}{1.232391in}}%
\pgfpathlineto{\pgfqpoint{0.683271in}{1.230484in}}%
\pgfpathlineto{\pgfqpoint{0.692982in}{1.226090in}}%
\pgfpathlineto{\pgfqpoint{0.694095in}{1.225439in}}%
\pgfpathlineto{\pgfqpoint{0.702694in}{1.217817in}}%
\pgfpathlineto{\pgfqpoint{0.704637in}{1.215789in}}%
\pgfpathlineto{\pgfqpoint{0.710010in}{1.206140in}}%
\pgfpathlineto{\pgfqpoint{0.712010in}{1.196491in}}%
\pgfpathlineto{\pgfqpoint{0.711629in}{1.186842in}}%
\pgfpathlineto{\pgfqpoint{0.708686in}{1.177193in}}%
\pgfpathlineto{\pgfqpoint{0.702694in}{1.168705in}}%
\pgfpathlineto{\pgfqpoint{0.701850in}{1.167544in}}%
\pgfpathlineto{\pgfqpoint{0.692982in}{1.160526in}}%
\pgfpathlineto{\pgfqpoint{0.687995in}{1.157895in}}%
\pgfpathlineto{\pgfqpoint{0.683271in}{1.156080in}}%
\pgfpathlineto{\pgfqpoint{0.673559in}{1.154114in}}%
\pgfpathlineto{\pgfqpoint{0.663847in}{1.154258in}}%
\pgfpathlineto{\pgfqpoint{0.654135in}{1.156714in}}%
\pgfusepath{stroke}%
\end{pgfscope}%
\begin{pgfscope}%
\pgfpathrectangle{\pgfqpoint{0.625000in}{0.550000in}}{\pgfqpoint{3.875000in}{3.850000in}} %
\pgfusepath{clip}%
\pgfsetbuttcap%
\pgfsetroundjoin%
\pgfsetlinewidth{0.250937pt}%
\definecolor{currentstroke}{rgb}{0.000000,0.000000,0.000000}%
\pgfsetstrokecolor{currentstroke}%
\pgfsetdash{}{0pt}%
\pgfpathmoveto{\pgfqpoint{0.634712in}{1.375188in}}%
\pgfpathlineto{\pgfqpoint{0.633500in}{1.379825in}}%
\pgfpathlineto{\pgfqpoint{0.634712in}{1.382119in}}%
\pgfpathlineto{\pgfqpoint{0.638044in}{1.379825in}}%
\pgfpathlineto{\pgfqpoint{0.634712in}{1.375188in}}%
\pgfusepath{stroke}%
\end{pgfscope}%
\begin{pgfscope}%
\pgfpathrectangle{\pgfqpoint{0.625000in}{0.550000in}}{\pgfqpoint{3.875000in}{3.850000in}} %
\pgfusepath{clip}%
\pgfsetbuttcap%
\pgfsetroundjoin%
\pgfsetlinewidth{0.250937pt}%
\definecolor{currentstroke}{rgb}{0.000000,0.000000,0.000000}%
\pgfsetstrokecolor{currentstroke}%
\pgfsetdash{}{0pt}%
\pgfpathmoveto{\pgfqpoint{0.634712in}{1.925442in}}%
\pgfpathlineto{\pgfqpoint{0.630744in}{1.929825in}}%
\pgfpathlineto{\pgfqpoint{0.634712in}{1.933112in}}%
\pgfpathlineto{\pgfqpoint{0.638965in}{1.929825in}}%
\pgfpathlineto{\pgfqpoint{0.634712in}{1.925442in}}%
\pgfusepath{stroke}%
\end{pgfscope}%
\begin{pgfscope}%
\pgfpathrectangle{\pgfqpoint{0.625000in}{0.550000in}}{\pgfqpoint{3.875000in}{3.850000in}} %
\pgfusepath{clip}%
\pgfsetbuttcap%
\pgfsetroundjoin%
\pgfsetlinewidth{0.250937pt}%
\definecolor{currentstroke}{rgb}{0.000000,0.000000,0.000000}%
\pgfsetstrokecolor{currentstroke}%
\pgfsetdash{}{0pt}%
\pgfpathmoveto{\pgfqpoint{0.634712in}{3.026537in}}%
\pgfpathlineto{\pgfqpoint{0.630744in}{3.029825in}}%
\pgfpathlineto{\pgfqpoint{0.634712in}{3.034207in}}%
\pgfpathlineto{\pgfqpoint{0.638965in}{3.029825in}}%
\pgfpathlineto{\pgfqpoint{0.634712in}{3.026537in}}%
\pgfusepath{stroke}%
\end{pgfscope}%
\begin{pgfscope}%
\pgfpathrectangle{\pgfqpoint{0.625000in}{0.550000in}}{\pgfqpoint{3.875000in}{3.850000in}} %
\pgfusepath{clip}%
\pgfsetbuttcap%
\pgfsetroundjoin%
\pgfsetlinewidth{0.250937pt}%
\definecolor{currentstroke}{rgb}{0.000000,0.000000,0.000000}%
\pgfsetstrokecolor{currentstroke}%
\pgfsetdash{}{0pt}%
\pgfpathmoveto{\pgfqpoint{0.634712in}{3.577530in}}%
\pgfpathlineto{\pgfqpoint{0.633500in}{3.579825in}}%
\pgfpathlineto{\pgfqpoint{0.634712in}{3.584461in}}%
\pgfpathlineto{\pgfqpoint{0.638044in}{3.579825in}}%
\pgfpathlineto{\pgfqpoint{0.634712in}{3.577530in}}%
\pgfusepath{stroke}%
\end{pgfscope}%
\begin{pgfscope}%
\pgfpathrectangle{\pgfqpoint{0.625000in}{0.550000in}}{\pgfqpoint{3.875000in}{3.850000in}} %
\pgfusepath{clip}%
\pgfsetbuttcap%
\pgfsetroundjoin%
\pgfsetlinewidth{0.250937pt}%
\definecolor{currentstroke}{rgb}{0.000000,0.000000,0.000000}%
\pgfsetstrokecolor{currentstroke}%
\pgfsetdash{}{0pt}%
\pgfpathmoveto{\pgfqpoint{0.654135in}{3.729508in}}%
\pgfpathlineto{\pgfqpoint{0.645864in}{3.734211in}}%
\pgfpathlineto{\pgfqpoint{0.644424in}{3.734622in}}%
\pgfpathlineto{\pgfqpoint{0.634712in}{3.742720in}}%
\pgfpathlineto{\pgfqpoint{0.634344in}{3.743860in}}%
\pgfpathlineto{\pgfqpoint{0.629403in}{3.753509in}}%
\pgfpathlineto{\pgfqpoint{0.625848in}{3.763158in}}%
\pgfpathlineto{\pgfqpoint{0.627122in}{3.772807in}}%
\pgfpathlineto{\pgfqpoint{0.630563in}{3.782456in}}%
\pgfpathlineto{\pgfqpoint{0.634712in}{3.790546in}}%
\pgfpathlineto{\pgfqpoint{0.638056in}{3.792105in}}%
\pgfpathlineto{\pgfqpoint{0.644424in}{3.798682in}}%
\pgfpathlineto{\pgfqpoint{0.652472in}{3.801754in}}%
\pgfpathlineto{\pgfqpoint{0.654135in}{3.802935in}}%
\pgfpathlineto{\pgfqpoint{0.663847in}{3.805391in}}%
\pgfpathlineto{\pgfqpoint{0.673559in}{3.805535in}}%
\pgfpathlineto{\pgfqpoint{0.683271in}{3.803570in}}%
\pgfpathlineto{\pgfqpoint{0.687995in}{3.801754in}}%
\pgfpathlineto{\pgfqpoint{0.692982in}{3.799123in}}%
\pgfpathlineto{\pgfqpoint{0.701850in}{3.792105in}}%
\pgfpathlineto{\pgfqpoint{0.702694in}{3.790944in}}%
\pgfpathlineto{\pgfqpoint{0.708686in}{3.782456in}}%
\pgfpathlineto{\pgfqpoint{0.711629in}{3.772807in}}%
\pgfpathlineto{\pgfqpoint{0.712010in}{3.763158in}}%
\pgfpathlineto{\pgfqpoint{0.710010in}{3.753509in}}%
\pgfpathlineto{\pgfqpoint{0.704637in}{3.743860in}}%
\pgfpathlineto{\pgfqpoint{0.702694in}{3.741832in}}%
\pgfpathlineto{\pgfqpoint{0.694095in}{3.734211in}}%
\pgfpathlineto{\pgfqpoint{0.692982in}{3.733559in}}%
\pgfpathlineto{\pgfqpoint{0.683271in}{3.729166in}}%
\pgfpathlineto{\pgfqpoint{0.673559in}{3.727258in}}%
\pgfpathlineto{\pgfqpoint{0.663847in}{3.727394in}}%
\pgfpathlineto{\pgfqpoint{0.654135in}{3.729508in}}%
\pgfusepath{stroke}%
\end{pgfscope}%
\begin{pgfscope}%
\pgfpathrectangle{\pgfqpoint{0.625000in}{0.550000in}}{\pgfqpoint{3.875000in}{3.850000in}} %
\pgfusepath{clip}%
\pgfsetbuttcap%
\pgfsetroundjoin%
\pgfsetlinewidth{0.250937pt}%
\definecolor{currentstroke}{rgb}{0.000000,0.000000,0.000000}%
\pgfsetstrokecolor{currentstroke}%
\pgfsetdash{}{0pt}%
\pgfpathmoveto{\pgfqpoint{0.634712in}{4.128933in}}%
\pgfpathlineto{\pgfqpoint{0.634336in}{4.129825in}}%
\pgfpathlineto{\pgfqpoint{0.634712in}{4.134630in}}%
\pgfpathlineto{\pgfqpoint{0.636192in}{4.129825in}}%
\pgfpathlineto{\pgfqpoint{0.634712in}{4.128933in}}%
\pgfusepath{stroke}%
\end{pgfscope}%
\begin{pgfscope}%
\pgfpathrectangle{\pgfqpoint{0.625000in}{0.550000in}}{\pgfqpoint{3.875000in}{3.850000in}} %
\pgfusepath{clip}%
\pgfsetbuttcap%
\pgfsetroundjoin%
\pgfsetlinewidth{0.250937pt}%
\definecolor{currentstroke}{rgb}{0.000000,0.000000,0.000000}%
\pgfsetstrokecolor{currentstroke}%
\pgfsetdash{}{0pt}%
\pgfpathmoveto{\pgfqpoint{0.625000in}{0.636121in}}%
\pgfpathlineto{\pgfqpoint{0.634208in}{0.627193in}}%
\pgfpathlineto{\pgfqpoint{0.625000in}{0.618291in}}%
\pgfusepath{stroke}%
\end{pgfscope}%
\begin{pgfscope}%
\pgfpathrectangle{\pgfqpoint{0.625000in}{0.550000in}}{\pgfqpoint{3.875000in}{3.850000in}} %
\pgfusepath{clip}%
\pgfsetbuttcap%
\pgfsetroundjoin%
\pgfsetlinewidth{0.250937pt}%
\definecolor{currentstroke}{rgb}{0.000000,0.000000,0.000000}%
\pgfsetstrokecolor{currentstroke}%
\pgfsetdash{}{0pt}%
\pgfpathmoveto{\pgfqpoint{0.625000in}{0.790115in}}%
\pgfpathlineto{\pgfqpoint{0.634235in}{0.781579in}}%
\pgfpathlineto{\pgfqpoint{0.625000in}{0.772663in}}%
\pgfusepath{stroke}%
\end{pgfscope}%
\begin{pgfscope}%
\pgfpathrectangle{\pgfqpoint{0.625000in}{0.550000in}}{\pgfqpoint{3.875000in}{3.850000in}} %
\pgfusepath{clip}%
\pgfsetbuttcap%
\pgfsetroundjoin%
\pgfsetlinewidth{0.250937pt}%
\definecolor{currentstroke}{rgb}{0.000000,0.000000,0.000000}%
\pgfsetstrokecolor{currentstroke}%
\pgfsetdash{}{0pt}%
\pgfpathmoveto{\pgfqpoint{0.625000in}{0.944783in}}%
\pgfpathlineto{\pgfqpoint{0.632552in}{0.945614in}}%
\pgfpathlineto{\pgfqpoint{0.634712in}{0.950144in}}%
\pgfpathlineto{\pgfqpoint{0.644424in}{0.950036in}}%
\pgfpathlineto{\pgfqpoint{0.652492in}{0.945614in}}%
\pgfpathlineto{\pgfqpoint{0.654135in}{0.941557in}}%
\pgfpathlineto{\pgfqpoint{0.656849in}{0.935965in}}%
\pgfpathlineto{\pgfqpoint{0.654135in}{0.930373in}}%
\pgfpathlineto{\pgfqpoint{0.652492in}{0.926316in}}%
\pgfpathlineto{\pgfqpoint{0.644424in}{0.921894in}}%
\pgfpathlineto{\pgfqpoint{0.634712in}{0.921786in}}%
\pgfpathlineto{\pgfqpoint{0.632362in}{0.926316in}}%
\pgfpathlineto{\pgfqpoint{0.625000in}{0.927068in}}%
\pgfusepath{stroke}%
\end{pgfscope}%
\begin{pgfscope}%
\pgfpathrectangle{\pgfqpoint{0.625000in}{0.550000in}}{\pgfqpoint{3.875000in}{3.850000in}} %
\pgfusepath{clip}%
\pgfsetbuttcap%
\pgfsetroundjoin%
\pgfsetlinewidth{0.250937pt}%
\definecolor{currentstroke}{rgb}{0.000000,0.000000,0.000000}%
\pgfsetstrokecolor{currentstroke}%
\pgfsetdash{}{0pt}%
\pgfpathmoveto{\pgfqpoint{0.625000in}{1.099200in}}%
\pgfpathlineto{\pgfqpoint{0.634228in}{1.090351in}}%
\pgfpathlineto{\pgfqpoint{0.625000in}{1.081534in}}%
\pgfusepath{stroke}%
\end{pgfscope}%
\begin{pgfscope}%
\pgfpathrectangle{\pgfqpoint{0.625000in}{0.550000in}}{\pgfqpoint{3.875000in}{3.850000in}} %
\pgfusepath{clip}%
\pgfsetbuttcap%
\pgfsetroundjoin%
\pgfsetlinewidth{0.250937pt}%
\definecolor{currentstroke}{rgb}{0.000000,0.000000,0.000000}%
\pgfsetstrokecolor{currentstroke}%
\pgfsetdash{}{0pt}%
\pgfpathmoveto{\pgfqpoint{0.625000in}{1.253875in}}%
\pgfpathlineto{\pgfqpoint{0.634259in}{1.244737in}}%
\pgfpathlineto{\pgfqpoint{0.625000in}{1.235851in}}%
\pgfusepath{stroke}%
\end{pgfscope}%
\begin{pgfscope}%
\pgfpathrectangle{\pgfqpoint{0.625000in}{0.550000in}}{\pgfqpoint{3.875000in}{3.850000in}} %
\pgfusepath{clip}%
\pgfsetbuttcap%
\pgfsetroundjoin%
\pgfsetlinewidth{0.250937pt}%
\definecolor{currentstroke}{rgb}{0.000000,0.000000,0.000000}%
\pgfsetstrokecolor{currentstroke}%
\pgfsetdash{}{0pt}%
\pgfpathmoveto{\pgfqpoint{0.625000in}{1.407905in}}%
\pgfpathlineto{\pgfqpoint{0.634232in}{1.399123in}}%
\pgfpathlineto{\pgfqpoint{0.625000in}{1.390336in}}%
\pgfusepath{stroke}%
\end{pgfscope}%
\begin{pgfscope}%
\pgfpathrectangle{\pgfqpoint{0.625000in}{0.550000in}}{\pgfqpoint{3.875000in}{3.850000in}} %
\pgfusepath{clip}%
\pgfsetbuttcap%
\pgfsetroundjoin%
\pgfsetlinewidth{0.250937pt}%
\definecolor{currentstroke}{rgb}{0.000000,0.000000,0.000000}%
\pgfsetstrokecolor{currentstroke}%
\pgfsetdash{}{0pt}%
\pgfpathmoveto{\pgfqpoint{4.500000in}{1.516078in}}%
\pgfpathlineto{\pgfqpoint{3.810464in}{1.518467in}}%
\pgfpathlineto{\pgfqpoint{3.383145in}{1.522008in}}%
\pgfpathlineto{\pgfqpoint{3.072368in}{1.526645in}}%
\pgfpathlineto{\pgfqpoint{2.819862in}{1.532573in}}%
\pgfpathlineto{\pgfqpoint{2.615915in}{1.539496in}}%
\pgfpathlineto{\pgfqpoint{2.441103in}{1.547575in}}%
\pgfpathlineto{\pgfqpoint{2.285714in}{1.556997in}}%
\pgfpathlineto{\pgfqpoint{2.149749in}{1.567501in}}%
\pgfpathlineto{\pgfqpoint{2.033208in}{1.578647in}}%
\pgfpathlineto{\pgfqpoint{1.926378in}{1.591038in}}%
\pgfpathlineto{\pgfqpoint{1.829261in}{1.604515in}}%
\pgfpathlineto{\pgfqpoint{1.741855in}{1.618849in}}%
\pgfpathlineto{\pgfqpoint{1.664160in}{1.633694in}}%
\pgfpathlineto{\pgfqpoint{1.586466in}{1.650910in}}%
\pgfpathlineto{\pgfqpoint{1.518484in}{1.668247in}}%
\pgfpathlineto{\pgfqpoint{1.448838in}{1.688596in}}%
\pgfpathlineto{\pgfqpoint{1.390542in}{1.707895in}}%
\pgfpathlineto{\pgfqpoint{1.333960in}{1.728815in}}%
\pgfpathlineto{\pgfqpoint{1.275689in}{1.752867in}}%
\pgfpathlineto{\pgfqpoint{1.217419in}{1.779647in}}%
\pgfpathlineto{\pgfqpoint{1.159148in}{1.809351in}}%
\pgfpathlineto{\pgfqpoint{1.099713in}{1.842982in}}%
\pgfpathlineto{\pgfqpoint{1.053111in}{1.871930in}}%
\pgfpathlineto{\pgfqpoint{1.023183in}{1.891883in}}%
\pgfpathlineto{\pgfqpoint{0.983655in}{1.920175in}}%
\pgfpathlineto{\pgfqpoint{0.945489in}{1.949736in}}%
\pgfpathlineto{\pgfqpoint{0.906642in}{1.982421in}}%
\pgfpathlineto{\pgfqpoint{0.858083in}{2.027738in}}%
\pgfpathlineto{\pgfqpoint{0.819236in}{2.068297in}}%
\pgfpathlineto{\pgfqpoint{0.780388in}{2.113888in}}%
\pgfpathlineto{\pgfqpoint{0.751253in}{2.152293in}}%
\pgfpathlineto{\pgfqpoint{0.722118in}{2.195515in}}%
\pgfpathlineto{\pgfqpoint{0.691840in}{2.248246in}}%
\pgfpathlineto{\pgfqpoint{0.673029in}{2.286842in}}%
\pgfpathlineto{\pgfqpoint{0.657187in}{2.325439in}}%
\pgfpathlineto{\pgfqpoint{0.647180in}{2.354386in}}%
\pgfpathlineto{\pgfqpoint{0.638888in}{2.383333in}}%
\pgfpathlineto{\pgfqpoint{0.631449in}{2.412281in}}%
\pgfpathlineto{\pgfqpoint{0.625000in}{2.470173in}}%
\pgfpathlineto{\pgfqpoint{0.625000in}{2.470173in}}%
\pgfusepath{stroke}%
\end{pgfscope}%
\begin{pgfscope}%
\pgfpathrectangle{\pgfqpoint{0.625000in}{0.550000in}}{\pgfqpoint{3.875000in}{3.850000in}} %
\pgfusepath{clip}%
\pgfsetbuttcap%
\pgfsetroundjoin%
\pgfsetlinewidth{0.250937pt}%
\definecolor{currentstroke}{rgb}{0.000000,0.000000,0.000000}%
\pgfsetstrokecolor{currentstroke}%
\pgfsetdash{}{0pt}%
\pgfpathmoveto{\pgfqpoint{0.625000in}{1.562252in}}%
\pgfpathlineto{\pgfqpoint{0.634177in}{1.553509in}}%
\pgfpathlineto{\pgfqpoint{0.625000in}{1.544482in}}%
\pgfusepath{stroke}%
\end{pgfscope}%
\begin{pgfscope}%
\pgfpathrectangle{\pgfqpoint{0.625000in}{0.550000in}}{\pgfqpoint{3.875000in}{3.850000in}} %
\pgfusepath{clip}%
\pgfsetbuttcap%
\pgfsetroundjoin%
\pgfsetlinewidth{0.250937pt}%
\definecolor{currentstroke}{rgb}{0.000000,0.000000,0.000000}%
\pgfsetstrokecolor{currentstroke}%
\pgfsetdash{}{0pt}%
\pgfpathmoveto{\pgfqpoint{0.625000in}{1.716644in}}%
\pgfpathlineto{\pgfqpoint{0.632589in}{1.717544in}}%
\pgfpathlineto{\pgfqpoint{0.634712in}{1.722073in}}%
\pgfpathlineto{\pgfqpoint{0.644424in}{1.721966in}}%
\pgfpathlineto{\pgfqpoint{0.652492in}{1.717544in}}%
\pgfpathlineto{\pgfqpoint{0.654135in}{1.713487in}}%
\pgfpathlineto{\pgfqpoint{0.656849in}{1.707895in}}%
\pgfpathlineto{\pgfqpoint{0.654135in}{1.702303in}}%
\pgfpathlineto{\pgfqpoint{0.652492in}{1.698246in}}%
\pgfpathlineto{\pgfqpoint{0.644424in}{1.693824in}}%
\pgfpathlineto{\pgfqpoint{0.634712in}{1.693716in}}%
\pgfpathlineto{\pgfqpoint{0.632471in}{1.698246in}}%
\pgfpathlineto{\pgfqpoint{0.625000in}{1.699090in}}%
\pgfusepath{stroke}%
\end{pgfscope}%
\begin{pgfscope}%
\pgfpathrectangle{\pgfqpoint{0.625000in}{0.550000in}}{\pgfqpoint{3.875000in}{3.850000in}} %
\pgfusepath{clip}%
\pgfsetbuttcap%
\pgfsetroundjoin%
\pgfsetlinewidth{0.250937pt}%
\definecolor{currentstroke}{rgb}{0.000000,0.000000,0.000000}%
\pgfsetstrokecolor{currentstroke}%
\pgfsetdash{}{0pt}%
\pgfpathmoveto{\pgfqpoint{0.625000in}{1.871115in}}%
\pgfpathlineto{\pgfqpoint{0.634169in}{1.862281in}}%
\pgfpathlineto{\pgfqpoint{0.625000in}{1.853629in}}%
\pgfusepath{stroke}%
\end{pgfscope}%
\begin{pgfscope}%
\pgfpathrectangle{\pgfqpoint{0.625000in}{0.550000in}}{\pgfqpoint{3.875000in}{3.850000in}} %
\pgfusepath{clip}%
\pgfsetbuttcap%
\pgfsetroundjoin%
\pgfsetlinewidth{0.250937pt}%
\definecolor{currentstroke}{rgb}{0.000000,0.000000,0.000000}%
\pgfsetstrokecolor{currentstroke}%
\pgfsetdash{}{0pt}%
\pgfpathmoveto{\pgfqpoint{0.625000in}{2.025826in}}%
\pgfpathlineto{\pgfqpoint{0.634169in}{2.016667in}}%
\pgfpathlineto{\pgfqpoint{0.625000in}{2.007582in}}%
\pgfusepath{stroke}%
\end{pgfscope}%
\begin{pgfscope}%
\pgfpathrectangle{\pgfqpoint{0.625000in}{0.550000in}}{\pgfqpoint{3.875000in}{3.850000in}} %
\pgfusepath{clip}%
\pgfsetbuttcap%
\pgfsetroundjoin%
\pgfsetlinewidth{0.250937pt}%
\definecolor{currentstroke}{rgb}{0.000000,0.000000,0.000000}%
\pgfsetstrokecolor{currentstroke}%
\pgfsetdash{}{0pt}%
\pgfpathmoveto{\pgfqpoint{0.625000in}{2.179743in}}%
\pgfpathlineto{\pgfqpoint{0.634191in}{2.171053in}}%
\pgfpathlineto{\pgfqpoint{0.625000in}{2.162113in}}%
\pgfusepath{stroke}%
\end{pgfscope}%
\begin{pgfscope}%
\pgfpathrectangle{\pgfqpoint{0.625000in}{0.550000in}}{\pgfqpoint{3.875000in}{3.850000in}} %
\pgfusepath{clip}%
\pgfsetbuttcap%
\pgfsetroundjoin%
\pgfsetlinewidth{0.250937pt}%
\definecolor{currentstroke}{rgb}{0.000000,0.000000,0.000000}%
\pgfsetstrokecolor{currentstroke}%
\pgfsetdash{}{0pt}%
\pgfpathmoveto{\pgfqpoint{0.625000in}{2.334093in}}%
\pgfpathlineto{\pgfqpoint{0.634239in}{2.325439in}}%
\pgfpathlineto{\pgfqpoint{0.625000in}{2.316178in}}%
\pgfusepath{stroke}%
\end{pgfscope}%
\begin{pgfscope}%
\pgfpathrectangle{\pgfqpoint{0.625000in}{0.550000in}}{\pgfqpoint{3.875000in}{3.850000in}} %
\pgfusepath{clip}%
\pgfsetbuttcap%
\pgfsetroundjoin%
\pgfsetlinewidth{0.250937pt}%
\definecolor{currentstroke}{rgb}{0.000000,0.000000,0.000000}%
\pgfsetstrokecolor{currentstroke}%
\pgfsetdash{}{0pt}%
\pgfpathmoveto{\pgfqpoint{0.625000in}{2.489476in}}%
\pgfpathlineto{\pgfqpoint{0.627780in}{2.518421in}}%
\pgfpathlineto{\pgfqpoint{0.631449in}{2.547368in}}%
\pgfpathlineto{\pgfqpoint{0.636476in}{2.566667in}}%
\pgfpathlineto{\pgfqpoint{0.647180in}{2.605263in}}%
\pgfpathlineto{\pgfqpoint{0.660884in}{2.643860in}}%
\pgfpathlineto{\pgfqpoint{0.673559in}{2.674101in}}%
\pgfpathlineto{\pgfqpoint{0.697035in}{2.721053in}}%
\pgfpathlineto{\pgfqpoint{0.713539in}{2.750000in}}%
\pgfpathlineto{\pgfqpoint{0.741541in}{2.793626in}}%
\pgfpathlineto{\pgfqpoint{0.781010in}{2.846491in}}%
\pgfpathlineto{\pgfqpoint{0.813732in}{2.885088in}}%
\pgfpathlineto{\pgfqpoint{0.840533in}{2.914035in}}%
\pgfpathlineto{\pgfqpoint{0.869496in}{2.942982in}}%
\pgfpathlineto{\pgfqpoint{0.906642in}{2.977228in}}%
\pgfpathlineto{\pgfqpoint{0.958452in}{3.020175in}}%
\pgfpathlineto{\pgfqpoint{0.994048in}{3.047169in}}%
\pgfpathlineto{\pgfqpoint{1.053111in}{3.087719in}}%
\pgfpathlineto{\pgfqpoint{1.100877in}{3.117383in}}%
\pgfpathlineto{\pgfqpoint{1.150559in}{3.145614in}}%
\pgfpathlineto{\pgfqpoint{1.197995in}{3.170436in}}%
\pgfpathlineto{\pgfqpoint{1.256266in}{3.198169in}}%
\pgfpathlineto{\pgfqpoint{1.314536in}{3.223094in}}%
\pgfpathlineto{\pgfqpoint{1.372807in}{3.245465in}}%
\pgfpathlineto{\pgfqpoint{1.431078in}{3.265435in}}%
\pgfpathlineto{\pgfqpoint{1.489348in}{3.283241in}}%
\pgfpathlineto{\pgfqpoint{1.557331in}{3.301593in}}%
\pgfpathlineto{\pgfqpoint{1.625313in}{3.317668in}}%
\pgfpathlineto{\pgfqpoint{1.703008in}{3.333651in}}%
\pgfpathlineto{\pgfqpoint{1.790414in}{3.349042in}}%
\pgfpathlineto{\pgfqpoint{1.877820in}{3.362168in}}%
\pgfpathlineto{\pgfqpoint{1.974937in}{3.374531in}}%
\pgfpathlineto{\pgfqpoint{2.081767in}{3.385914in}}%
\pgfpathlineto{\pgfqpoint{2.208020in}{3.396941in}}%
\pgfpathlineto{\pgfqpoint{2.343985in}{3.406470in}}%
\pgfpathlineto{\pgfqpoint{2.499373in}{3.415027in}}%
\pgfpathlineto{\pgfqpoint{2.674185in}{3.422371in}}%
\pgfpathlineto{\pgfqpoint{2.878133in}{3.428667in}}%
\pgfpathlineto{\pgfqpoint{3.120927in}{3.433890in}}%
\pgfpathlineto{\pgfqpoint{3.421992in}{3.438074in}}%
\pgfpathlineto{\pgfqpoint{3.820175in}{3.441238in}}%
\pgfpathlineto{\pgfqpoint{4.393170in}{3.443352in}}%
\pgfpathlineto{\pgfqpoint{4.500000in}{3.443572in}}%
\pgfpathlineto{\pgfqpoint{4.500000in}{3.443572in}}%
\pgfusepath{stroke}%
\end{pgfscope}%
\begin{pgfscope}%
\pgfpathrectangle{\pgfqpoint{0.625000in}{0.550000in}}{\pgfqpoint{3.875000in}{3.850000in}} %
\pgfusepath{clip}%
\pgfsetbuttcap%
\pgfsetroundjoin%
\pgfsetlinewidth{0.250937pt}%
\definecolor{currentstroke}{rgb}{0.000000,0.000000,0.000000}%
\pgfsetstrokecolor{currentstroke}%
\pgfsetdash{}{0pt}%
\pgfpathmoveto{\pgfqpoint{0.625000in}{2.643471in}}%
\pgfpathlineto{\pgfqpoint{0.634239in}{2.634211in}}%
\pgfpathlineto{\pgfqpoint{0.625000in}{2.625558in}}%
\pgfusepath{stroke}%
\end{pgfscope}%
\begin{pgfscope}%
\pgfpathrectangle{\pgfqpoint{0.625000in}{0.550000in}}{\pgfqpoint{3.875000in}{3.850000in}} %
\pgfusepath{clip}%
\pgfsetbuttcap%
\pgfsetroundjoin%
\pgfsetlinewidth{0.250937pt}%
\definecolor{currentstroke}{rgb}{0.000000,0.000000,0.000000}%
\pgfsetstrokecolor{currentstroke}%
\pgfsetdash{}{0pt}%
\pgfpathmoveto{\pgfqpoint{0.625000in}{2.797532in}}%
\pgfpathlineto{\pgfqpoint{0.634187in}{2.788596in}}%
\pgfpathlineto{\pgfqpoint{0.625000in}{2.779911in}}%
\pgfusepath{stroke}%
\end{pgfscope}%
\begin{pgfscope}%
\pgfpathrectangle{\pgfqpoint{0.625000in}{0.550000in}}{\pgfqpoint{3.875000in}{3.850000in}} %
\pgfusepath{clip}%
\pgfsetbuttcap%
\pgfsetroundjoin%
\pgfsetlinewidth{0.250937pt}%
\definecolor{currentstroke}{rgb}{0.000000,0.000000,0.000000}%
\pgfsetstrokecolor{currentstroke}%
\pgfsetdash{}{0pt}%
\pgfpathmoveto{\pgfqpoint{0.625000in}{2.952058in}}%
\pgfpathlineto{\pgfqpoint{0.634159in}{2.942982in}}%
\pgfpathlineto{\pgfqpoint{0.625000in}{2.933831in}}%
\pgfusepath{stroke}%
\end{pgfscope}%
\begin{pgfscope}%
\pgfpathrectangle{\pgfqpoint{0.625000in}{0.550000in}}{\pgfqpoint{3.875000in}{3.850000in}} %
\pgfusepath{clip}%
\pgfsetbuttcap%
\pgfsetroundjoin%
\pgfsetlinewidth{0.250937pt}%
\definecolor{currentstroke}{rgb}{0.000000,0.000000,0.000000}%
\pgfsetstrokecolor{currentstroke}%
\pgfsetdash{}{0pt}%
\pgfpathmoveto{\pgfqpoint{0.625000in}{3.106007in}}%
\pgfpathlineto{\pgfqpoint{0.634162in}{3.097368in}}%
\pgfpathlineto{\pgfqpoint{0.625000in}{3.088545in}}%
\pgfusepath{stroke}%
\end{pgfscope}%
\begin{pgfscope}%
\pgfpathrectangle{\pgfqpoint{0.625000in}{0.550000in}}{\pgfqpoint{3.875000in}{3.850000in}} %
\pgfusepath{clip}%
\pgfsetbuttcap%
\pgfsetroundjoin%
\pgfsetlinewidth{0.250937pt}%
\definecolor{currentstroke}{rgb}{0.000000,0.000000,0.000000}%
\pgfsetstrokecolor{currentstroke}%
\pgfsetdash{}{0pt}%
\pgfpathmoveto{\pgfqpoint{0.625000in}{3.260519in}}%
\pgfpathlineto{\pgfqpoint{0.632471in}{3.261404in}}%
\pgfpathlineto{\pgfqpoint{0.634712in}{3.265933in}}%
\pgfpathlineto{\pgfqpoint{0.644424in}{3.265825in}}%
\pgfpathlineto{\pgfqpoint{0.652492in}{3.261404in}}%
\pgfpathlineto{\pgfqpoint{0.654135in}{3.257346in}}%
\pgfpathlineto{\pgfqpoint{0.656849in}{3.251754in}}%
\pgfpathlineto{\pgfqpoint{0.654135in}{3.246162in}}%
\pgfpathlineto{\pgfqpoint{0.652492in}{3.242105in}}%
\pgfpathlineto{\pgfqpoint{0.644424in}{3.237684in}}%
\pgfpathlineto{\pgfqpoint{0.634712in}{3.237576in}}%
\pgfpathlineto{\pgfqpoint{0.632589in}{3.242105in}}%
\pgfpathlineto{\pgfqpoint{0.625000in}{3.243048in}}%
\pgfusepath{stroke}%
\end{pgfscope}%
\begin{pgfscope}%
\pgfpathrectangle{\pgfqpoint{0.625000in}{0.550000in}}{\pgfqpoint{3.875000in}{3.850000in}} %
\pgfusepath{clip}%
\pgfsetbuttcap%
\pgfsetroundjoin%
\pgfsetlinewidth{0.250937pt}%
\definecolor{currentstroke}{rgb}{0.000000,0.000000,0.000000}%
\pgfsetstrokecolor{currentstroke}%
\pgfsetdash{}{0pt}%
\pgfpathmoveto{\pgfqpoint{0.625000in}{3.415146in}}%
\pgfpathlineto{\pgfqpoint{0.634158in}{3.406140in}}%
\pgfpathlineto{\pgfqpoint{0.625000in}{3.397427in}}%
\pgfusepath{stroke}%
\end{pgfscope}%
\begin{pgfscope}%
\pgfpathrectangle{\pgfqpoint{0.625000in}{0.550000in}}{\pgfqpoint{3.875000in}{3.850000in}} %
\pgfusepath{clip}%
\pgfsetbuttcap%
\pgfsetroundjoin%
\pgfsetlinewidth{0.250937pt}%
\definecolor{currentstroke}{rgb}{0.000000,0.000000,0.000000}%
\pgfsetstrokecolor{currentstroke}%
\pgfsetdash{}{0pt}%
\pgfpathmoveto{\pgfqpoint{0.625000in}{3.569279in}}%
\pgfpathlineto{\pgfqpoint{0.634212in}{3.560526in}}%
\pgfpathlineto{\pgfqpoint{0.625000in}{3.551779in}}%
\pgfusepath{stroke}%
\end{pgfscope}%
\begin{pgfscope}%
\pgfpathrectangle{\pgfqpoint{0.625000in}{0.550000in}}{\pgfqpoint{3.875000in}{3.850000in}} %
\pgfusepath{clip}%
\pgfsetbuttcap%
\pgfsetroundjoin%
\pgfsetlinewidth{0.250937pt}%
\definecolor{currentstroke}{rgb}{0.000000,0.000000,0.000000}%
\pgfsetstrokecolor{currentstroke}%
\pgfsetdash{}{0pt}%
\pgfpathmoveto{\pgfqpoint{0.625000in}{3.723752in}}%
\pgfpathlineto{\pgfqpoint{0.634231in}{3.714912in}}%
\pgfpathlineto{\pgfqpoint{0.625000in}{3.705806in}}%
\pgfusepath{stroke}%
\end{pgfscope}%
\begin{pgfscope}%
\pgfpathrectangle{\pgfqpoint{0.625000in}{0.550000in}}{\pgfqpoint{3.875000in}{3.850000in}} %
\pgfusepath{clip}%
\pgfsetbuttcap%
\pgfsetroundjoin%
\pgfsetlinewidth{0.250937pt}%
\definecolor{currentstroke}{rgb}{0.000000,0.000000,0.000000}%
\pgfsetstrokecolor{currentstroke}%
\pgfsetdash{}{0pt}%
\pgfpathmoveto{\pgfqpoint{0.625000in}{3.878031in}}%
\pgfpathlineto{\pgfqpoint{0.634177in}{3.869298in}}%
\pgfpathlineto{\pgfqpoint{0.625000in}{3.860530in}}%
\pgfusepath{stroke}%
\end{pgfscope}%
\begin{pgfscope}%
\pgfpathrectangle{\pgfqpoint{0.625000in}{0.550000in}}{\pgfqpoint{3.875000in}{3.850000in}} %
\pgfusepath{clip}%
\pgfsetbuttcap%
\pgfsetroundjoin%
\pgfsetlinewidth{0.250937pt}%
\definecolor{currentstroke}{rgb}{0.000000,0.000000,0.000000}%
\pgfsetstrokecolor{currentstroke}%
\pgfsetdash{}{0pt}%
\pgfpathmoveto{\pgfqpoint{0.625000in}{4.032511in}}%
\pgfpathlineto{\pgfqpoint{0.632362in}{4.033333in}}%
\pgfpathlineto{\pgfqpoint{0.634712in}{4.037863in}}%
\pgfpathlineto{\pgfqpoint{0.644424in}{4.037755in}}%
\pgfpathlineto{\pgfqpoint{0.652492in}{4.033333in}}%
\pgfpathlineto{\pgfqpoint{0.654135in}{4.029276in}}%
\pgfpathlineto{\pgfqpoint{0.656849in}{4.023684in}}%
\pgfpathlineto{\pgfqpoint{0.654135in}{4.018092in}}%
\pgfpathlineto{\pgfqpoint{0.652492in}{4.014035in}}%
\pgfpathlineto{\pgfqpoint{0.644424in}{4.009613in}}%
\pgfpathlineto{\pgfqpoint{0.634712in}{4.009506in}}%
\pgfpathlineto{\pgfqpoint{0.632552in}{4.014035in}}%
\pgfpathlineto{\pgfqpoint{0.625000in}{4.014944in}}%
\pgfusepath{stroke}%
\end{pgfscope}%
\begin{pgfscope}%
\pgfpathrectangle{\pgfqpoint{0.625000in}{0.550000in}}{\pgfqpoint{3.875000in}{3.850000in}} %
\pgfusepath{clip}%
\pgfsetbuttcap%
\pgfsetroundjoin%
\pgfsetlinewidth{0.250937pt}%
\definecolor{currentstroke}{rgb}{0.000000,0.000000,0.000000}%
\pgfsetstrokecolor{currentstroke}%
\pgfsetdash{}{0pt}%
\pgfpathmoveto{\pgfqpoint{0.625000in}{4.186911in}}%
\pgfpathlineto{\pgfqpoint{0.634184in}{4.178070in}}%
\pgfpathlineto{\pgfqpoint{0.625000in}{4.169643in}}%
\pgfusepath{stroke}%
\end{pgfscope}%
\begin{pgfscope}%
\pgfpathrectangle{\pgfqpoint{0.625000in}{0.550000in}}{\pgfqpoint{3.875000in}{3.850000in}} %
\pgfusepath{clip}%
\pgfsetbuttcap%
\pgfsetroundjoin%
\pgfsetlinewidth{0.250937pt}%
\definecolor{currentstroke}{rgb}{0.000000,0.000000,0.000000}%
\pgfsetstrokecolor{currentstroke}%
\pgfsetdash{}{0pt}%
\pgfpathmoveto{\pgfqpoint{0.625000in}{4.341244in}}%
\pgfpathlineto{\pgfqpoint{0.634129in}{4.332456in}}%
\pgfpathlineto{\pgfqpoint{0.625000in}{4.323638in}}%
\pgfusepath{stroke}%
\end{pgfscope}%
\begin{pgfscope}%
\pgfpathrectangle{\pgfqpoint{0.625000in}{0.550000in}}{\pgfqpoint{3.875000in}{3.850000in}} %
\pgfusepath{clip}%
\pgfsetbuttcap%
\pgfsetroundjoin%
\pgfsetlinewidth{0.250937pt}%
\definecolor{currentstroke}{rgb}{0.000000,0.000000,0.000000}%
\pgfsetstrokecolor{currentstroke}%
\pgfsetdash{}{0pt}%
\pgfpathmoveto{\pgfqpoint{0.634712in}{0.819267in}}%
\pgfpathlineto{\pgfqpoint{0.634298in}{0.820175in}}%
\pgfpathlineto{\pgfqpoint{0.633620in}{0.829825in}}%
\pgfpathlineto{\pgfqpoint{0.634712in}{0.832417in}}%
\pgfpathlineto{\pgfqpoint{0.639014in}{0.829825in}}%
\pgfpathlineto{\pgfqpoint{0.636503in}{0.820175in}}%
\pgfpathlineto{\pgfqpoint{0.634712in}{0.819267in}}%
\pgfusepath{stroke}%
\end{pgfscope}%
\begin{pgfscope}%
\pgfpathrectangle{\pgfqpoint{0.625000in}{0.550000in}}{\pgfqpoint{3.875000in}{3.850000in}} %
\pgfusepath{clip}%
\pgfsetbuttcap%
\pgfsetroundjoin%
\pgfsetlinewidth{0.250937pt}%
\definecolor{currentstroke}{rgb}{0.000000,0.000000,0.000000}%
\pgfsetstrokecolor{currentstroke}%
\pgfsetdash{}{0pt}%
\pgfpathmoveto{\pgfqpoint{0.673559in}{1.147108in}}%
\pgfpathlineto{\pgfqpoint{0.668920in}{1.148246in}}%
\pgfpathlineto{\pgfqpoint{0.663847in}{1.149022in}}%
\pgfpathlineto{\pgfqpoint{0.654135in}{1.152350in}}%
\pgfpathlineto{\pgfqpoint{0.646323in}{1.157895in}}%
\pgfpathlineto{\pgfqpoint{0.644424in}{1.158620in}}%
\pgfpathlineto{\pgfqpoint{0.635784in}{1.167544in}}%
\pgfpathlineto{\pgfqpoint{0.634712in}{1.168044in}}%
\pgfpathlineto{\pgfqpoint{0.630020in}{1.177193in}}%
\pgfpathlineto{\pgfqpoint{0.626914in}{1.186842in}}%
\pgfpathlineto{\pgfqpoint{0.625674in}{1.196491in}}%
\pgfpathlineto{\pgfqpoint{0.629023in}{1.206140in}}%
\pgfpathlineto{\pgfqpoint{0.633641in}{1.215789in}}%
\pgfpathlineto{\pgfqpoint{0.634712in}{1.219104in}}%
\pgfpathlineto{\pgfqpoint{0.642366in}{1.225439in}}%
\pgfpathlineto{\pgfqpoint{0.644424in}{1.228239in}}%
\pgfpathlineto{\pgfqpoint{0.654135in}{1.233923in}}%
\pgfpathlineto{\pgfqpoint{0.658321in}{1.235088in}}%
\pgfpathlineto{\pgfqpoint{0.663847in}{1.237701in}}%
\pgfpathlineto{\pgfqpoint{0.673559in}{1.239597in}}%
\pgfpathlineto{\pgfqpoint{0.683271in}{1.239774in}}%
\pgfpathlineto{\pgfqpoint{0.692982in}{1.238290in}}%
\pgfpathlineto{\pgfqpoint{0.702370in}{1.235088in}}%
\pgfpathlineto{\pgfqpoint{0.702694in}{1.234944in}}%
\pgfpathlineto{\pgfqpoint{0.712406in}{1.229182in}}%
\pgfpathlineto{\pgfqpoint{0.717179in}{1.225439in}}%
\pgfpathlineto{\pgfqpoint{0.722118in}{1.219409in}}%
\pgfpathlineto{\pgfqpoint{0.724959in}{1.215789in}}%
\pgfpathlineto{\pgfqpoint{0.729341in}{1.206140in}}%
\pgfpathlineto{\pgfqpoint{0.731073in}{1.196491in}}%
\pgfpathlineto{\pgfqpoint{0.730738in}{1.186842in}}%
\pgfpathlineto{\pgfqpoint{0.728227in}{1.177193in}}%
\pgfpathlineto{\pgfqpoint{0.722673in}{1.167544in}}%
\pgfpathlineto{\pgfqpoint{0.722118in}{1.166924in}}%
\pgfpathlineto{\pgfqpoint{0.713293in}{1.157895in}}%
\pgfpathlineto{\pgfqpoint{0.712406in}{1.157284in}}%
\pgfpathlineto{\pgfqpoint{0.702694in}{1.151618in}}%
\pgfpathlineto{\pgfqpoint{0.692982in}{1.148341in}}%
\pgfpathlineto{\pgfqpoint{0.692446in}{1.148246in}}%
\pgfpathlineto{\pgfqpoint{0.683271in}{1.146859in}}%
\pgfpathlineto{\pgfqpoint{0.673559in}{1.147108in}}%
\pgfusepath{stroke}%
\end{pgfscope}%
\begin{pgfscope}%
\pgfpathrectangle{\pgfqpoint{0.625000in}{0.550000in}}{\pgfqpoint{3.875000in}{3.850000in}} %
\pgfusepath{clip}%
\pgfsetbuttcap%
\pgfsetroundjoin%
\pgfsetlinewidth{0.250937pt}%
\definecolor{currentstroke}{rgb}{0.000000,0.000000,0.000000}%
\pgfsetstrokecolor{currentstroke}%
\pgfsetdash{}{0pt}%
\pgfpathmoveto{\pgfqpoint{0.634712in}{1.372004in}}%
\pgfpathlineto{\pgfqpoint{0.632667in}{1.379825in}}%
\pgfpathlineto{\pgfqpoint{0.634712in}{1.383694in}}%
\pgfpathlineto{\pgfqpoint{0.640332in}{1.379825in}}%
\pgfpathlineto{\pgfqpoint{0.634712in}{1.372004in}}%
\pgfusepath{stroke}%
\end{pgfscope}%
\begin{pgfscope}%
\pgfpathrectangle{\pgfqpoint{0.625000in}{0.550000in}}{\pgfqpoint{3.875000in}{3.850000in}} %
\pgfusepath{clip}%
\pgfsetbuttcap%
\pgfsetroundjoin%
\pgfsetlinewidth{0.250937pt}%
\definecolor{currentstroke}{rgb}{0.000000,0.000000,0.000000}%
\pgfsetstrokecolor{currentstroke}%
\pgfsetdash{}{0pt}%
\pgfpathmoveto{\pgfqpoint{0.634712in}{1.923364in}}%
\pgfpathlineto{\pgfqpoint{0.628862in}{1.929825in}}%
\pgfpathlineto{\pgfqpoint{0.634712in}{1.934671in}}%
\pgfpathlineto{\pgfqpoint{0.640981in}{1.929825in}}%
\pgfpathlineto{\pgfqpoint{0.634712in}{1.923364in}}%
\pgfusepath{stroke}%
\end{pgfscope}%
\begin{pgfscope}%
\pgfpathrectangle{\pgfqpoint{0.625000in}{0.550000in}}{\pgfqpoint{3.875000in}{3.850000in}} %
\pgfusepath{clip}%
\pgfsetbuttcap%
\pgfsetroundjoin%
\pgfsetlinewidth{0.250937pt}%
\definecolor{currentstroke}{rgb}{0.000000,0.000000,0.000000}%
\pgfsetstrokecolor{currentstroke}%
\pgfsetdash{}{0pt}%
\pgfpathmoveto{\pgfqpoint{0.634712in}{3.024978in}}%
\pgfpathlineto{\pgfqpoint{0.628862in}{3.029825in}}%
\pgfpathlineto{\pgfqpoint{0.634712in}{3.036285in}}%
\pgfpathlineto{\pgfqpoint{0.640981in}{3.029825in}}%
\pgfpathlineto{\pgfqpoint{0.634712in}{3.024978in}}%
\pgfusepath{stroke}%
\end{pgfscope}%
\begin{pgfscope}%
\pgfpathrectangle{\pgfqpoint{0.625000in}{0.550000in}}{\pgfqpoint{3.875000in}{3.850000in}} %
\pgfusepath{clip}%
\pgfsetbuttcap%
\pgfsetroundjoin%
\pgfsetlinewidth{0.250937pt}%
\definecolor{currentstroke}{rgb}{0.000000,0.000000,0.000000}%
\pgfsetstrokecolor{currentstroke}%
\pgfsetdash{}{0pt}%
\pgfpathmoveto{\pgfqpoint{0.634712in}{3.575955in}}%
\pgfpathlineto{\pgfqpoint{0.632667in}{3.579825in}}%
\pgfpathlineto{\pgfqpoint{0.634712in}{3.587645in}}%
\pgfpathlineto{\pgfqpoint{0.640332in}{3.579825in}}%
\pgfpathlineto{\pgfqpoint{0.634712in}{3.575955in}}%
\pgfusepath{stroke}%
\end{pgfscope}%
\begin{pgfscope}%
\pgfpathrectangle{\pgfqpoint{0.625000in}{0.550000in}}{\pgfqpoint{3.875000in}{3.850000in}} %
\pgfusepath{clip}%
\pgfsetbuttcap%
\pgfsetroundjoin%
\pgfsetlinewidth{0.250937pt}%
\definecolor{currentstroke}{rgb}{0.000000,0.000000,0.000000}%
\pgfsetstrokecolor{currentstroke}%
\pgfsetdash{}{0pt}%
\pgfpathmoveto{\pgfqpoint{0.663847in}{3.721948in}}%
\pgfpathlineto{\pgfqpoint{0.658321in}{3.724561in}}%
\pgfpathlineto{\pgfqpoint{0.654135in}{3.725726in}}%
\pgfpathlineto{\pgfqpoint{0.644424in}{3.731410in}}%
\pgfpathlineto{\pgfqpoint{0.642366in}{3.734211in}}%
\pgfpathlineto{\pgfqpoint{0.634712in}{3.740545in}}%
\pgfpathlineto{\pgfqpoint{0.633641in}{3.743860in}}%
\pgfpathlineto{\pgfqpoint{0.629023in}{3.753509in}}%
\pgfpathlineto{\pgfqpoint{0.625674in}{3.763158in}}%
\pgfpathlineto{\pgfqpoint{0.626914in}{3.772807in}}%
\pgfpathlineto{\pgfqpoint{0.630020in}{3.782456in}}%
\pgfpathlineto{\pgfqpoint{0.634712in}{3.791606in}}%
\pgfpathlineto{\pgfqpoint{0.635784in}{3.792105in}}%
\pgfpathlineto{\pgfqpoint{0.644424in}{3.801029in}}%
\pgfpathlineto{\pgfqpoint{0.646323in}{3.801754in}}%
\pgfpathlineto{\pgfqpoint{0.654135in}{3.807300in}}%
\pgfpathlineto{\pgfqpoint{0.663847in}{3.810627in}}%
\pgfpathlineto{\pgfqpoint{0.668920in}{3.811404in}}%
\pgfpathlineto{\pgfqpoint{0.673559in}{3.812541in}}%
\pgfpathlineto{\pgfqpoint{0.683271in}{3.812790in}}%
\pgfpathlineto{\pgfqpoint{0.692446in}{3.811404in}}%
\pgfpathlineto{\pgfqpoint{0.692982in}{3.811308in}}%
\pgfpathlineto{\pgfqpoint{0.702694in}{3.808031in}}%
\pgfpathlineto{\pgfqpoint{0.712406in}{3.802365in}}%
\pgfpathlineto{\pgfqpoint{0.713293in}{3.801754in}}%
\pgfpathlineto{\pgfqpoint{0.722118in}{3.792725in}}%
\pgfpathlineto{\pgfqpoint{0.722673in}{3.792105in}}%
\pgfpathlineto{\pgfqpoint{0.728227in}{3.782456in}}%
\pgfpathlineto{\pgfqpoint{0.730738in}{3.772807in}}%
\pgfpathlineto{\pgfqpoint{0.731073in}{3.763158in}}%
\pgfpathlineto{\pgfqpoint{0.729341in}{3.753509in}}%
\pgfpathlineto{\pgfqpoint{0.724959in}{3.743860in}}%
\pgfpathlineto{\pgfqpoint{0.722118in}{3.740240in}}%
\pgfpathlineto{\pgfqpoint{0.717179in}{3.734211in}}%
\pgfpathlineto{\pgfqpoint{0.712406in}{3.730467in}}%
\pgfpathlineto{\pgfqpoint{0.702694in}{3.724705in}}%
\pgfpathlineto{\pgfqpoint{0.702370in}{3.724561in}}%
\pgfpathlineto{\pgfqpoint{0.692982in}{3.721359in}}%
\pgfpathlineto{\pgfqpoint{0.683271in}{3.719875in}}%
\pgfpathlineto{\pgfqpoint{0.673559in}{3.720052in}}%
\pgfpathlineto{\pgfqpoint{0.663847in}{3.721948in}}%
\pgfusepath{stroke}%
\end{pgfscope}%
\begin{pgfscope}%
\pgfpathrectangle{\pgfqpoint{0.625000in}{0.550000in}}{\pgfqpoint{3.875000in}{3.850000in}} %
\pgfusepath{clip}%
\pgfsetbuttcap%
\pgfsetroundjoin%
\pgfsetlinewidth{0.250937pt}%
\definecolor{currentstroke}{rgb}{0.000000,0.000000,0.000000}%
\pgfsetstrokecolor{currentstroke}%
\pgfsetdash{}{0pt}%
\pgfpathmoveto{\pgfqpoint{0.634712in}{4.127232in}}%
\pgfpathlineto{\pgfqpoint{0.633620in}{4.129825in}}%
\pgfpathlineto{\pgfqpoint{0.634298in}{4.139474in}}%
\pgfpathlineto{\pgfqpoint{0.634712in}{4.140382in}}%
\pgfpathlineto{\pgfqpoint{0.636503in}{4.139474in}}%
\pgfpathlineto{\pgfqpoint{0.639014in}{4.129825in}}%
\pgfpathlineto{\pgfqpoint{0.634712in}{4.127232in}}%
\pgfusepath{stroke}%
\end{pgfscope}%
\begin{pgfscope}%
\pgfpathrectangle{\pgfqpoint{0.625000in}{0.550000in}}{\pgfqpoint{3.875000in}{3.850000in}} %
\pgfusepath{clip}%
\pgfsetbuttcap%
\pgfsetroundjoin%
\pgfsetlinewidth{0.250937pt}%
\definecolor{currentstroke}{rgb}{0.000000,0.000000,0.000000}%
\pgfsetstrokecolor{currentstroke}%
\pgfsetdash{}{0pt}%
\pgfpathmoveto{\pgfqpoint{1.833534in}{0.550000in}}%
\pgfpathlineto{\pgfqpoint{1.832543in}{0.588596in}}%
\pgfpathlineto{\pgfqpoint{1.829261in}{0.629908in}}%
\pgfpathlineto{\pgfqpoint{1.824608in}{0.665789in}}%
\pgfpathlineto{\pgfqpoint{1.817628in}{0.704386in}}%
\pgfpathlineto{\pgfqpoint{1.808618in}{0.742982in}}%
\pgfpathlineto{\pgfqpoint{1.797555in}{0.781579in}}%
\pgfpathlineto{\pgfqpoint{1.784392in}{0.820175in}}%
\pgfpathlineto{\pgfqpoint{1.769088in}{0.858772in}}%
\pgfpathlineto{\pgfqpoint{1.751566in}{0.897440in}}%
\pgfpathlineto{\pgfqpoint{1.731907in}{0.935965in}}%
\pgfpathlineto{\pgfqpoint{1.709963in}{0.974561in}}%
\pgfpathlineto{\pgfqpoint{1.683584in}{1.016352in}}%
\pgfpathlineto{\pgfqpoint{1.654449in}{1.058251in}}%
\pgfpathlineto{\pgfqpoint{1.625313in}{1.096729in}}%
\pgfpathlineto{\pgfqpoint{1.596178in}{1.132569in}}%
\pgfpathlineto{\pgfqpoint{1.557331in}{1.177239in}}%
\pgfpathlineto{\pgfqpoint{1.508772in}{1.229401in}}%
\pgfpathlineto{\pgfqpoint{1.455884in}{1.283333in}}%
\pgfpathlineto{\pgfqpoint{1.295113in}{1.444381in}}%
\pgfpathlineto{\pgfqpoint{1.217419in}{1.527950in}}%
\pgfpathlineto{\pgfqpoint{0.981130in}{1.794737in}}%
\pgfpathlineto{\pgfqpoint{0.935777in}{1.848547in}}%
\pgfpathlineto{\pgfqpoint{0.867794in}{1.935420in}}%
\pgfpathlineto{\pgfqpoint{0.809524in}{2.016763in}}%
\pgfpathlineto{\pgfqpoint{0.770677in}{2.076648in}}%
\pgfpathlineto{\pgfqpoint{0.737937in}{2.132456in}}%
\pgfpathlineto{\pgfqpoint{0.712288in}{2.180702in}}%
\pgfpathlineto{\pgfqpoint{0.689497in}{2.228947in}}%
\pgfpathlineto{\pgfqpoint{0.673264in}{2.267544in}}%
\pgfpathlineto{\pgfqpoint{0.654135in}{2.320781in}}%
\pgfpathlineto{\pgfqpoint{0.639472in}{2.373684in}}%
\pgfpathlineto{\pgfqpoint{0.630400in}{2.412281in}}%
\pgfpathlineto{\pgfqpoint{0.628621in}{2.431579in}}%
\pgfpathlineto{\pgfqpoint{0.625000in}{2.469126in}}%
\pgfpathlineto{\pgfqpoint{0.625000in}{2.469126in}}%
\pgfusepath{stroke}%
\end{pgfscope}%
\begin{pgfscope}%
\pgfpathrectangle{\pgfqpoint{0.625000in}{0.550000in}}{\pgfqpoint{3.875000in}{3.850000in}} %
\pgfusepath{clip}%
\pgfsetbuttcap%
\pgfsetroundjoin%
\pgfsetlinewidth{0.250937pt}%
\definecolor{currentstroke}{rgb}{0.000000,0.000000,0.000000}%
\pgfsetstrokecolor{currentstroke}%
\pgfsetdash{}{0pt}%
\pgfpathmoveto{\pgfqpoint{0.625000in}{0.636197in}}%
\pgfpathlineto{\pgfqpoint{0.634286in}{0.627193in}}%
\pgfpathlineto{\pgfqpoint{0.625000in}{0.618216in}}%
\pgfusepath{stroke}%
\end{pgfscope}%
\begin{pgfscope}%
\pgfpathrectangle{\pgfqpoint{0.625000in}{0.550000in}}{\pgfqpoint{3.875000in}{3.850000in}} %
\pgfusepath{clip}%
\pgfsetbuttcap%
\pgfsetroundjoin%
\pgfsetlinewidth{0.250937pt}%
\definecolor{currentstroke}{rgb}{0.000000,0.000000,0.000000}%
\pgfsetstrokecolor{currentstroke}%
\pgfsetdash{}{0pt}%
\pgfpathmoveto{\pgfqpoint{0.625000in}{0.790192in}}%
\pgfpathlineto{\pgfqpoint{0.634318in}{0.781579in}}%
\pgfpathlineto{\pgfqpoint{0.625000in}{0.772582in}}%
\pgfusepath{stroke}%
\end{pgfscope}%
\begin{pgfscope}%
\pgfpathrectangle{\pgfqpoint{0.625000in}{0.550000in}}{\pgfqpoint{3.875000in}{3.850000in}} %
\pgfusepath{clip}%
\pgfsetbuttcap%
\pgfsetroundjoin%
\pgfsetlinewidth{0.250937pt}%
\definecolor{currentstroke}{rgb}{0.000000,0.000000,0.000000}%
\pgfsetstrokecolor{currentstroke}%
\pgfsetdash{}{0pt}%
\pgfpathmoveto{\pgfqpoint{0.625000in}{0.944859in}}%
\pgfpathlineto{\pgfqpoint{0.631857in}{0.945614in}}%
\pgfpathlineto{\pgfqpoint{0.634712in}{0.951600in}}%
\pgfpathlineto{\pgfqpoint{0.644424in}{0.952960in}}%
\pgfpathlineto{\pgfqpoint{0.654135in}{0.950133in}}%
\pgfpathlineto{\pgfqpoint{0.660270in}{0.945614in}}%
\pgfpathlineto{\pgfqpoint{0.663242in}{0.935965in}}%
\pgfpathlineto{\pgfqpoint{0.660270in}{0.926316in}}%
\pgfpathlineto{\pgfqpoint{0.654135in}{0.921797in}}%
\pgfpathlineto{\pgfqpoint{0.644424in}{0.918970in}}%
\pgfpathlineto{\pgfqpoint{0.634712in}{0.920330in}}%
\pgfpathlineto{\pgfqpoint{0.631607in}{0.926316in}}%
\pgfpathlineto{\pgfqpoint{0.625000in}{0.926991in}}%
\pgfusepath{stroke}%
\end{pgfscope}%
\begin{pgfscope}%
\pgfpathrectangle{\pgfqpoint{0.625000in}{0.550000in}}{\pgfqpoint{3.875000in}{3.850000in}} %
\pgfusepath{clip}%
\pgfsetbuttcap%
\pgfsetroundjoin%
\pgfsetlinewidth{0.250937pt}%
\definecolor{currentstroke}{rgb}{0.000000,0.000000,0.000000}%
\pgfsetstrokecolor{currentstroke}%
\pgfsetdash{}{0pt}%
\pgfpathmoveto{\pgfqpoint{0.625000in}{1.099279in}}%
\pgfpathlineto{\pgfqpoint{0.634310in}{1.090351in}}%
\pgfpathlineto{\pgfqpoint{0.625000in}{1.081455in}}%
\pgfusepath{stroke}%
\end{pgfscope}%
\begin{pgfscope}%
\pgfpathrectangle{\pgfqpoint{0.625000in}{0.550000in}}{\pgfqpoint{3.875000in}{3.850000in}} %
\pgfusepath{clip}%
\pgfsetbuttcap%
\pgfsetroundjoin%
\pgfsetlinewidth{0.250937pt}%
\definecolor{currentstroke}{rgb}{0.000000,0.000000,0.000000}%
\pgfsetstrokecolor{currentstroke}%
\pgfsetdash{}{0pt}%
\pgfpathmoveto{\pgfqpoint{0.625000in}{1.253958in}}%
\pgfpathlineto{\pgfqpoint{0.634343in}{1.244737in}}%
\pgfpathlineto{\pgfqpoint{0.625000in}{1.235770in}}%
\pgfusepath{stroke}%
\end{pgfscope}%
\begin{pgfscope}%
\pgfpathrectangle{\pgfqpoint{0.625000in}{0.550000in}}{\pgfqpoint{3.875000in}{3.850000in}} %
\pgfusepath{clip}%
\pgfsetbuttcap%
\pgfsetroundjoin%
\pgfsetlinewidth{0.250937pt}%
\definecolor{currentstroke}{rgb}{0.000000,0.000000,0.000000}%
\pgfsetstrokecolor{currentstroke}%
\pgfsetdash{}{0pt}%
\pgfpathmoveto{\pgfqpoint{0.625000in}{1.407986in}}%
\pgfpathlineto{\pgfqpoint{0.634317in}{1.399123in}}%
\pgfpathlineto{\pgfqpoint{0.625000in}{1.390254in}}%
\pgfusepath{stroke}%
\end{pgfscope}%
\begin{pgfscope}%
\pgfpathrectangle{\pgfqpoint{0.625000in}{0.550000in}}{\pgfqpoint{3.875000in}{3.850000in}} %
\pgfusepath{clip}%
\pgfsetbuttcap%
\pgfsetroundjoin%
\pgfsetlinewidth{0.250937pt}%
\definecolor{currentstroke}{rgb}{0.000000,0.000000,0.000000}%
\pgfsetstrokecolor{currentstroke}%
\pgfsetdash{}{0pt}%
\pgfpathmoveto{\pgfqpoint{0.625000in}{1.562332in}}%
\pgfpathlineto{\pgfqpoint{0.634260in}{1.553509in}}%
\pgfpathlineto{\pgfqpoint{0.625000in}{1.544399in}}%
\pgfusepath{stroke}%
\end{pgfscope}%
\begin{pgfscope}%
\pgfpathrectangle{\pgfqpoint{0.625000in}{0.550000in}}{\pgfqpoint{3.875000in}{3.850000in}} %
\pgfusepath{clip}%
\pgfsetbuttcap%
\pgfsetroundjoin%
\pgfsetlinewidth{0.250937pt}%
\definecolor{currentstroke}{rgb}{0.000000,0.000000,0.000000}%
\pgfsetstrokecolor{currentstroke}%
\pgfsetdash{}{0pt}%
\pgfpathmoveto{\pgfqpoint{0.625000in}{1.716725in}}%
\pgfpathlineto{\pgfqpoint{0.631906in}{1.717544in}}%
\pgfpathlineto{\pgfqpoint{0.634712in}{1.723530in}}%
\pgfpathlineto{\pgfqpoint{0.644424in}{1.724890in}}%
\pgfpathlineto{\pgfqpoint{0.654135in}{1.722063in}}%
\pgfpathlineto{\pgfqpoint{0.660270in}{1.717544in}}%
\pgfpathlineto{\pgfqpoint{0.663243in}{1.707895in}}%
\pgfpathlineto{\pgfqpoint{0.660270in}{1.698246in}}%
\pgfpathlineto{\pgfqpoint{0.654135in}{1.693726in}}%
\pgfpathlineto{\pgfqpoint{0.644424in}{1.690900in}}%
\pgfpathlineto{\pgfqpoint{0.634712in}{1.692259in}}%
\pgfpathlineto{\pgfqpoint{0.631750in}{1.698246in}}%
\pgfpathlineto{\pgfqpoint{0.625000in}{1.699008in}}%
\pgfusepath{stroke}%
\end{pgfscope}%
\begin{pgfscope}%
\pgfpathrectangle{\pgfqpoint{0.625000in}{0.550000in}}{\pgfqpoint{3.875000in}{3.850000in}} %
\pgfusepath{clip}%
\pgfsetbuttcap%
\pgfsetroundjoin%
\pgfsetlinewidth{0.250937pt}%
\definecolor{currentstroke}{rgb}{0.000000,0.000000,0.000000}%
\pgfsetstrokecolor{currentstroke}%
\pgfsetdash{}{0pt}%
\pgfpathmoveto{\pgfqpoint{0.625000in}{1.871198in}}%
\pgfpathlineto{\pgfqpoint{0.634255in}{1.862281in}}%
\pgfpathlineto{\pgfqpoint{0.625000in}{1.853548in}}%
\pgfusepath{stroke}%
\end{pgfscope}%
\begin{pgfscope}%
\pgfpathrectangle{\pgfqpoint{0.625000in}{0.550000in}}{\pgfqpoint{3.875000in}{3.850000in}} %
\pgfusepath{clip}%
\pgfsetbuttcap%
\pgfsetroundjoin%
\pgfsetlinewidth{0.250937pt}%
\definecolor{currentstroke}{rgb}{0.000000,0.000000,0.000000}%
\pgfsetstrokecolor{currentstroke}%
\pgfsetdash{}{0pt}%
\pgfpathmoveto{\pgfqpoint{0.625000in}{2.025914in}}%
\pgfpathlineto{\pgfqpoint{0.634256in}{2.016667in}}%
\pgfpathlineto{\pgfqpoint{0.625000in}{2.007495in}}%
\pgfusepath{stroke}%
\end{pgfscope}%
\begin{pgfscope}%
\pgfpathrectangle{\pgfqpoint{0.625000in}{0.550000in}}{\pgfqpoint{3.875000in}{3.850000in}} %
\pgfusepath{clip}%
\pgfsetbuttcap%
\pgfsetroundjoin%
\pgfsetlinewidth{0.250937pt}%
\definecolor{currentstroke}{rgb}{0.000000,0.000000,0.000000}%
\pgfsetstrokecolor{currentstroke}%
\pgfsetdash{}{0pt}%
\pgfpathmoveto{\pgfqpoint{0.625000in}{2.179826in}}%
\pgfpathlineto{\pgfqpoint{0.634278in}{2.171053in}}%
\pgfpathlineto{\pgfqpoint{0.625000in}{2.162028in}}%
\pgfusepath{stroke}%
\end{pgfscope}%
\begin{pgfscope}%
\pgfpathrectangle{\pgfqpoint{0.625000in}{0.550000in}}{\pgfqpoint{3.875000in}{3.850000in}} %
\pgfusepath{clip}%
\pgfsetbuttcap%
\pgfsetroundjoin%
\pgfsetlinewidth{0.250937pt}%
\definecolor{currentstroke}{rgb}{0.000000,0.000000,0.000000}%
\pgfsetstrokecolor{currentstroke}%
\pgfsetdash{}{0pt}%
\pgfpathmoveto{\pgfqpoint{0.625000in}{2.334176in}}%
\pgfpathlineto{\pgfqpoint{0.634328in}{2.325439in}}%
\pgfpathlineto{\pgfqpoint{0.625000in}{2.316089in}}%
\pgfusepath{stroke}%
\end{pgfscope}%
\begin{pgfscope}%
\pgfpathrectangle{\pgfqpoint{0.625000in}{0.550000in}}{\pgfqpoint{3.875000in}{3.850000in}} %
\pgfusepath{clip}%
\pgfsetbuttcap%
\pgfsetroundjoin%
\pgfsetlinewidth{0.250937pt}%
\definecolor{currentstroke}{rgb}{0.000000,0.000000,0.000000}%
\pgfsetstrokecolor{currentstroke}%
\pgfsetdash{}{0pt}%
\pgfpathmoveto{\pgfqpoint{0.625000in}{2.490523in}}%
\pgfpathlineto{\pgfqpoint{0.628621in}{2.528070in}}%
\pgfpathlineto{\pgfqpoint{0.630400in}{2.547368in}}%
\pgfpathlineto{\pgfqpoint{0.633564in}{2.557018in}}%
\pgfpathlineto{\pgfqpoint{0.637239in}{2.576316in}}%
\pgfpathlineto{\pgfqpoint{0.647029in}{2.614912in}}%
\pgfpathlineto{\pgfqpoint{0.659127in}{2.653509in}}%
\pgfpathlineto{\pgfqpoint{0.673559in}{2.692923in}}%
\pgfpathlineto{\pgfqpoint{0.693800in}{2.740351in}}%
\pgfpathlineto{\pgfqpoint{0.717249in}{2.788596in}}%
\pgfpathlineto{\pgfqpoint{0.741541in}{2.833796in}}%
\pgfpathlineto{\pgfqpoint{0.790409in}{2.914035in}}%
\pgfpathlineto{\pgfqpoint{0.829630in}{2.971930in}}%
\pgfpathlineto{\pgfqpoint{0.872103in}{3.029825in}}%
\pgfpathlineto{\pgfqpoint{0.909337in}{3.078070in}}%
\pgfpathlineto{\pgfqpoint{0.948392in}{3.126316in}}%
\pgfpathlineto{\pgfqpoint{0.994048in}{3.179858in}}%
\pgfpathlineto{\pgfqpoint{1.285401in}{3.505112in}}%
\pgfpathlineto{\pgfqpoint{1.363095in}{3.584357in}}%
\pgfpathlineto{\pgfqpoint{1.499060in}{3.720170in}}%
\pgfpathlineto{\pgfqpoint{1.539788in}{3.763158in}}%
\pgfpathlineto{\pgfqpoint{1.586466in}{3.815617in}}%
\pgfpathlineto{\pgfqpoint{1.622758in}{3.859649in}}%
\pgfpathlineto{\pgfqpoint{1.659172in}{3.907895in}}%
\pgfpathlineto{\pgfqpoint{1.685717in}{3.946491in}}%
\pgfpathlineto{\pgfqpoint{1.712719in}{3.989767in}}%
\pgfpathlineto{\pgfqpoint{1.732143in}{4.024129in}}%
\pgfpathlineto{\pgfqpoint{1.751602in}{4.062281in}}%
\pgfpathlineto{\pgfqpoint{1.770990in}{4.105453in}}%
\pgfpathlineto{\pgfqpoint{1.784392in}{4.139474in}}%
\pgfpathlineto{\pgfqpoint{1.797555in}{4.178070in}}%
\pgfpathlineto{\pgfqpoint{1.809837in}{4.221504in}}%
\pgfpathlineto{\pgfqpoint{1.817628in}{4.255263in}}%
\pgfpathlineto{\pgfqpoint{1.824608in}{4.293860in}}%
\pgfpathlineto{\pgfqpoint{1.829559in}{4.332456in}}%
\pgfpathlineto{\pgfqpoint{1.832543in}{4.371053in}}%
\pgfpathlineto{\pgfqpoint{1.833472in}{4.400000in}}%
\pgfpathlineto{\pgfqpoint{1.833472in}{4.400000in}}%
\pgfusepath{stroke}%
\end{pgfscope}%
\begin{pgfscope}%
\pgfpathrectangle{\pgfqpoint{0.625000in}{0.550000in}}{\pgfqpoint{3.875000in}{3.850000in}} %
\pgfusepath{clip}%
\pgfsetbuttcap%
\pgfsetroundjoin%
\pgfsetlinewidth{0.250937pt}%
\definecolor{currentstroke}{rgb}{0.000000,0.000000,0.000000}%
\pgfsetstrokecolor{currentstroke}%
\pgfsetdash{}{0pt}%
\pgfpathmoveto{\pgfqpoint{0.625000in}{2.643559in}}%
\pgfpathlineto{\pgfqpoint{0.634327in}{2.634211in}}%
\pgfpathlineto{\pgfqpoint{0.625000in}{2.625475in}}%
\pgfusepath{stroke}%
\end{pgfscope}%
\begin{pgfscope}%
\pgfpathrectangle{\pgfqpoint{0.625000in}{0.550000in}}{\pgfqpoint{3.875000in}{3.850000in}} %
\pgfusepath{clip}%
\pgfsetbuttcap%
\pgfsetroundjoin%
\pgfsetlinewidth{0.250937pt}%
\definecolor{currentstroke}{rgb}{0.000000,0.000000,0.000000}%
\pgfsetstrokecolor{currentstroke}%
\pgfsetdash{}{0pt}%
\pgfpathmoveto{\pgfqpoint{0.625000in}{2.797617in}}%
\pgfpathlineto{\pgfqpoint{0.634275in}{2.788596in}}%
\pgfpathlineto{\pgfqpoint{0.625000in}{2.779829in}}%
\pgfusepath{stroke}%
\end{pgfscope}%
\begin{pgfscope}%
\pgfpathrectangle{\pgfqpoint{0.625000in}{0.550000in}}{\pgfqpoint{3.875000in}{3.850000in}} %
\pgfusepath{clip}%
\pgfsetbuttcap%
\pgfsetroundjoin%
\pgfsetlinewidth{0.250937pt}%
\definecolor{currentstroke}{rgb}{0.000000,0.000000,0.000000}%
\pgfsetstrokecolor{currentstroke}%
\pgfsetdash{}{0pt}%
\pgfpathmoveto{\pgfqpoint{0.625000in}{2.952146in}}%
\pgfpathlineto{\pgfqpoint{0.634248in}{2.942982in}}%
\pgfpathlineto{\pgfqpoint{0.625000in}{2.933742in}}%
\pgfusepath{stroke}%
\end{pgfscope}%
\begin{pgfscope}%
\pgfpathrectangle{\pgfqpoint{0.625000in}{0.550000in}}{\pgfqpoint{3.875000in}{3.850000in}} %
\pgfusepath{clip}%
\pgfsetbuttcap%
\pgfsetroundjoin%
\pgfsetlinewidth{0.250937pt}%
\definecolor{currentstroke}{rgb}{0.000000,0.000000,0.000000}%
\pgfsetstrokecolor{currentstroke}%
\pgfsetdash{}{0pt}%
\pgfpathmoveto{\pgfqpoint{0.625000in}{3.106090in}}%
\pgfpathlineto{\pgfqpoint{0.634249in}{3.097368in}}%
\pgfpathlineto{\pgfqpoint{0.625000in}{3.088461in}}%
\pgfusepath{stroke}%
\end{pgfscope}%
\begin{pgfscope}%
\pgfpathrectangle{\pgfqpoint{0.625000in}{0.550000in}}{\pgfqpoint{3.875000in}{3.850000in}} %
\pgfusepath{clip}%
\pgfsetbuttcap%
\pgfsetroundjoin%
\pgfsetlinewidth{0.250937pt}%
\definecolor{currentstroke}{rgb}{0.000000,0.000000,0.000000}%
\pgfsetstrokecolor{currentstroke}%
\pgfsetdash{}{0pt}%
\pgfpathmoveto{\pgfqpoint{0.625000in}{3.260604in}}%
\pgfpathlineto{\pgfqpoint{0.631750in}{3.261404in}}%
\pgfpathlineto{\pgfqpoint{0.634712in}{3.267390in}}%
\pgfpathlineto{\pgfqpoint{0.644424in}{3.268749in}}%
\pgfpathlineto{\pgfqpoint{0.654135in}{3.265923in}}%
\pgfpathlineto{\pgfqpoint{0.660270in}{3.261404in}}%
\pgfpathlineto{\pgfqpoint{0.663243in}{3.251754in}}%
\pgfpathlineto{\pgfqpoint{0.660270in}{3.242105in}}%
\pgfpathlineto{\pgfqpoint{0.654135in}{3.237586in}}%
\pgfpathlineto{\pgfqpoint{0.644424in}{3.234759in}}%
\pgfpathlineto{\pgfqpoint{0.634712in}{3.236119in}}%
\pgfpathlineto{\pgfqpoint{0.631906in}{3.242105in}}%
\pgfpathlineto{\pgfqpoint{0.625000in}{3.242963in}}%
\pgfusepath{stroke}%
\end{pgfscope}%
\begin{pgfscope}%
\pgfpathrectangle{\pgfqpoint{0.625000in}{0.550000in}}{\pgfqpoint{3.875000in}{3.850000in}} %
\pgfusepath{clip}%
\pgfsetbuttcap%
\pgfsetroundjoin%
\pgfsetlinewidth{0.250937pt}%
\definecolor{currentstroke}{rgb}{0.000000,0.000000,0.000000}%
\pgfsetstrokecolor{currentstroke}%
\pgfsetdash{}{0pt}%
\pgfpathmoveto{\pgfqpoint{0.625000in}{3.415231in}}%
\pgfpathlineto{\pgfqpoint{0.634245in}{3.406140in}}%
\pgfpathlineto{\pgfqpoint{0.625000in}{3.397345in}}%
\pgfusepath{stroke}%
\end{pgfscope}%
\begin{pgfscope}%
\pgfpathrectangle{\pgfqpoint{0.625000in}{0.550000in}}{\pgfqpoint{3.875000in}{3.850000in}} %
\pgfusepath{clip}%
\pgfsetbuttcap%
\pgfsetroundjoin%
\pgfsetlinewidth{0.250937pt}%
\definecolor{currentstroke}{rgb}{0.000000,0.000000,0.000000}%
\pgfsetstrokecolor{currentstroke}%
\pgfsetdash{}{0pt}%
\pgfpathmoveto{\pgfqpoint{0.625000in}{3.569364in}}%
\pgfpathlineto{\pgfqpoint{0.634301in}{3.560526in}}%
\pgfpathlineto{\pgfqpoint{0.625000in}{3.551694in}}%
\pgfusepath{stroke}%
\end{pgfscope}%
\begin{pgfscope}%
\pgfpathrectangle{\pgfqpoint{0.625000in}{0.550000in}}{\pgfqpoint{3.875000in}{3.850000in}} %
\pgfusepath{clip}%
\pgfsetbuttcap%
\pgfsetroundjoin%
\pgfsetlinewidth{0.250937pt}%
\definecolor{currentstroke}{rgb}{0.000000,0.000000,0.000000}%
\pgfsetstrokecolor{currentstroke}%
\pgfsetdash{}{0pt}%
\pgfpathmoveto{\pgfqpoint{0.625000in}{3.723837in}}%
\pgfpathlineto{\pgfqpoint{0.634320in}{3.714912in}}%
\pgfpathlineto{\pgfqpoint{0.625000in}{3.705718in}}%
\pgfusepath{stroke}%
\end{pgfscope}%
\begin{pgfscope}%
\pgfpathrectangle{\pgfqpoint{0.625000in}{0.550000in}}{\pgfqpoint{3.875000in}{3.850000in}} %
\pgfusepath{clip}%
\pgfsetbuttcap%
\pgfsetroundjoin%
\pgfsetlinewidth{0.250937pt}%
\definecolor{currentstroke}{rgb}{0.000000,0.000000,0.000000}%
\pgfsetstrokecolor{currentstroke}%
\pgfsetdash{}{0pt}%
\pgfpathmoveto{\pgfqpoint{0.625000in}{3.878118in}}%
\pgfpathlineto{\pgfqpoint{0.634268in}{3.869298in}}%
\pgfpathlineto{\pgfqpoint{0.625000in}{3.860442in}}%
\pgfusepath{stroke}%
\end{pgfscope}%
\begin{pgfscope}%
\pgfpathrectangle{\pgfqpoint{0.625000in}{0.550000in}}{\pgfqpoint{3.875000in}{3.850000in}} %
\pgfusepath{clip}%
\pgfsetbuttcap%
\pgfsetroundjoin%
\pgfsetlinewidth{0.250937pt}%
\definecolor{currentstroke}{rgb}{0.000000,0.000000,0.000000}%
\pgfsetstrokecolor{currentstroke}%
\pgfsetdash{}{0pt}%
\pgfpathmoveto{\pgfqpoint{0.625000in}{4.032595in}}%
\pgfpathlineto{\pgfqpoint{0.631607in}{4.033333in}}%
\pgfpathlineto{\pgfqpoint{0.634712in}{4.039319in}}%
\pgfpathlineto{\pgfqpoint{0.644424in}{4.040679in}}%
\pgfpathlineto{\pgfqpoint{0.654135in}{4.037853in}}%
\pgfpathlineto{\pgfqpoint{0.660270in}{4.033333in}}%
\pgfpathlineto{\pgfqpoint{0.663242in}{4.023684in}}%
\pgfpathlineto{\pgfqpoint{0.660270in}{4.014035in}}%
\pgfpathlineto{\pgfqpoint{0.654135in}{4.009516in}}%
\pgfpathlineto{\pgfqpoint{0.644424in}{4.006689in}}%
\pgfpathlineto{\pgfqpoint{0.634712in}{4.008049in}}%
\pgfpathlineto{\pgfqpoint{0.631857in}{4.014035in}}%
\pgfpathlineto{\pgfqpoint{0.625000in}{4.014860in}}%
\pgfusepath{stroke}%
\end{pgfscope}%
\begin{pgfscope}%
\pgfpathrectangle{\pgfqpoint{0.625000in}{0.550000in}}{\pgfqpoint{3.875000in}{3.850000in}} %
\pgfusepath{clip}%
\pgfsetbuttcap%
\pgfsetroundjoin%
\pgfsetlinewidth{0.250937pt}%
\definecolor{currentstroke}{rgb}{0.000000,0.000000,0.000000}%
\pgfsetstrokecolor{currentstroke}%
\pgfsetdash{}{0pt}%
\pgfpathmoveto{\pgfqpoint{0.625000in}{4.186999in}}%
\pgfpathlineto{\pgfqpoint{0.634276in}{4.178070in}}%
\pgfpathlineto{\pgfqpoint{0.625000in}{4.169559in}}%
\pgfusepath{stroke}%
\end{pgfscope}%
\begin{pgfscope}%
\pgfpathrectangle{\pgfqpoint{0.625000in}{0.550000in}}{\pgfqpoint{3.875000in}{3.850000in}} %
\pgfusepath{clip}%
\pgfsetbuttcap%
\pgfsetroundjoin%
\pgfsetlinewidth{0.250937pt}%
\definecolor{currentstroke}{rgb}{0.000000,0.000000,0.000000}%
\pgfsetstrokecolor{currentstroke}%
\pgfsetdash{}{0pt}%
\pgfpathmoveto{\pgfqpoint{0.625000in}{4.341332in}}%
\pgfpathlineto{\pgfqpoint{0.634220in}{4.332456in}}%
\pgfpathlineto{\pgfqpoint{0.625000in}{4.323551in}}%
\pgfusepath{stroke}%
\end{pgfscope}%
\begin{pgfscope}%
\pgfpathrectangle{\pgfqpoint{0.625000in}{0.550000in}}{\pgfqpoint{3.875000in}{3.850000in}} %
\pgfusepath{clip}%
\pgfsetbuttcap%
\pgfsetroundjoin%
\pgfsetlinewidth{0.250937pt}%
\definecolor{currentstroke}{rgb}{0.000000,0.000000,0.000000}%
\pgfsetstrokecolor{currentstroke}%
\pgfsetdash{}{0pt}%
\pgfpathmoveto{\pgfqpoint{0.634712in}{0.817340in}}%
\pgfpathlineto{\pgfqpoint{0.633421in}{0.820175in}}%
\pgfpathlineto{\pgfqpoint{0.632904in}{0.829825in}}%
\pgfpathlineto{\pgfqpoint{0.634712in}{0.834118in}}%
\pgfpathlineto{\pgfqpoint{0.641837in}{0.829825in}}%
\pgfpathlineto{\pgfqpoint{0.640302in}{0.820175in}}%
\pgfpathlineto{\pgfqpoint{0.634712in}{0.817340in}}%
\pgfusepath{stroke}%
\end{pgfscope}%
\begin{pgfscope}%
\pgfpathrectangle{\pgfqpoint{0.625000in}{0.550000in}}{\pgfqpoint{3.875000in}{3.850000in}} %
\pgfusepath{clip}%
\pgfsetbuttcap%
\pgfsetroundjoin%
\pgfsetlinewidth{0.250937pt}%
\definecolor{currentstroke}{rgb}{0.000000,0.000000,0.000000}%
\pgfsetstrokecolor{currentstroke}%
\pgfsetdash{}{0pt}%
\pgfpathmoveto{\pgfqpoint{0.683271in}{1.136122in}}%
\pgfpathlineto{\pgfqpoint{0.674238in}{1.138596in}}%
\pgfpathlineto{\pgfqpoint{0.673559in}{1.138722in}}%
\pgfpathlineto{\pgfqpoint{0.663847in}{1.142139in}}%
\pgfpathlineto{\pgfqpoint{0.654135in}{1.147840in}}%
\pgfpathlineto{\pgfqpoint{0.653765in}{1.148246in}}%
\pgfpathlineto{\pgfqpoint{0.644424in}{1.154860in}}%
\pgfpathlineto{\pgfqpoint{0.642426in}{1.157895in}}%
\pgfpathlineto{\pgfqpoint{0.634712in}{1.165949in}}%
\pgfpathlineto{\pgfqpoint{0.634266in}{1.167544in}}%
\pgfpathlineto{\pgfqpoint{0.629477in}{1.177193in}}%
\pgfpathlineto{\pgfqpoint{0.626706in}{1.186842in}}%
\pgfpathlineto{\pgfqpoint{0.625499in}{1.196491in}}%
\pgfpathlineto{\pgfqpoint{0.628644in}{1.206140in}}%
\pgfpathlineto{\pgfqpoint{0.632939in}{1.215789in}}%
\pgfpathlineto{\pgfqpoint{0.634712in}{1.221278in}}%
\pgfpathlineto{\pgfqpoint{0.639738in}{1.225439in}}%
\pgfpathlineto{\pgfqpoint{0.644424in}{1.231813in}}%
\pgfpathlineto{\pgfqpoint{0.649956in}{1.235088in}}%
\pgfpathlineto{\pgfqpoint{0.654135in}{1.239140in}}%
\pgfpathlineto{\pgfqpoint{0.663847in}{1.244129in}}%
\pgfpathlineto{\pgfqpoint{0.665971in}{1.244737in}}%
\pgfpathlineto{\pgfqpoint{0.673559in}{1.248090in}}%
\pgfpathlineto{\pgfqpoint{0.683271in}{1.250476in}}%
\pgfpathlineto{\pgfqpoint{0.692982in}{1.251599in}}%
\pgfpathlineto{\pgfqpoint{0.702694in}{1.251585in}}%
\pgfpathlineto{\pgfqpoint{0.712406in}{1.250386in}}%
\pgfpathlineto{\pgfqpoint{0.722118in}{1.247872in}}%
\pgfpathlineto{\pgfqpoint{0.730058in}{1.244737in}}%
\pgfpathlineto{\pgfqpoint{0.731830in}{1.243850in}}%
\pgfpathlineto{\pgfqpoint{0.741541in}{1.237883in}}%
\pgfpathlineto{\pgfqpoint{0.745351in}{1.235088in}}%
\pgfpathlineto{\pgfqpoint{0.751253in}{1.228996in}}%
\pgfpathlineto{\pgfqpoint{0.754519in}{1.225439in}}%
\pgfpathlineto{\pgfqpoint{0.760288in}{1.215789in}}%
\pgfpathlineto{\pgfqpoint{0.760965in}{1.213848in}}%
\pgfpathlineto{\pgfqpoint{0.763933in}{1.206140in}}%
\pgfpathlineto{\pgfqpoint{0.765505in}{1.196491in}}%
\pgfpathlineto{\pgfqpoint{0.765188in}{1.186842in}}%
\pgfpathlineto{\pgfqpoint{0.762926in}{1.177193in}}%
\pgfpathlineto{\pgfqpoint{0.760965in}{1.172912in}}%
\pgfpathlineto{\pgfqpoint{0.758697in}{1.167544in}}%
\pgfpathlineto{\pgfqpoint{0.751756in}{1.157895in}}%
\pgfpathlineto{\pgfqpoint{0.751253in}{1.157402in}}%
\pgfpathlineto{\pgfqpoint{0.741541in}{1.148595in}}%
\pgfpathlineto{\pgfqpoint{0.741110in}{1.148246in}}%
\pgfpathlineto{\pgfqpoint{0.731830in}{1.142715in}}%
\pgfpathlineto{\pgfqpoint{0.722118in}{1.138698in}}%
\pgfpathlineto{\pgfqpoint{0.721793in}{1.138596in}}%
\pgfpathlineto{\pgfqpoint{0.712406in}{1.136172in}}%
\pgfpathlineto{\pgfqpoint{0.702694in}{1.134944in}}%
\pgfpathlineto{\pgfqpoint{0.692982in}{1.134924in}}%
\pgfpathlineto{\pgfqpoint{0.683271in}{1.136122in}}%
\pgfusepath{stroke}%
\end{pgfscope}%
\begin{pgfscope}%
\pgfpathrectangle{\pgfqpoint{0.625000in}{0.550000in}}{\pgfqpoint{3.875000in}{3.850000in}} %
\pgfusepath{clip}%
\pgfsetbuttcap%
\pgfsetroundjoin%
\pgfsetlinewidth{0.250937pt}%
\definecolor{currentstroke}{rgb}{0.000000,0.000000,0.000000}%
\pgfsetstrokecolor{currentstroke}%
\pgfsetdash{}{0pt}%
\pgfpathmoveto{\pgfqpoint{0.634712in}{1.369216in}}%
\pgfpathlineto{\pgfqpoint{0.634275in}{1.370175in}}%
\pgfpathlineto{\pgfqpoint{0.631835in}{1.379825in}}%
\pgfpathlineto{\pgfqpoint{0.634712in}{1.385270in}}%
\pgfpathlineto{\pgfqpoint{0.642621in}{1.379825in}}%
\pgfpathlineto{\pgfqpoint{0.637128in}{1.370175in}}%
\pgfpathlineto{\pgfqpoint{0.634712in}{1.369216in}}%
\pgfusepath{stroke}%
\end{pgfscope}%
\begin{pgfscope}%
\pgfpathrectangle{\pgfqpoint{0.625000in}{0.550000in}}{\pgfqpoint{3.875000in}{3.850000in}} %
\pgfusepath{clip}%
\pgfsetbuttcap%
\pgfsetroundjoin%
\pgfsetlinewidth{0.250937pt}%
\definecolor{currentstroke}{rgb}{0.000000,0.000000,0.000000}%
\pgfsetstrokecolor{currentstroke}%
\pgfsetdash{}{0pt}%
\pgfpathmoveto{\pgfqpoint{0.634712in}{1.921286in}}%
\pgfpathlineto{\pgfqpoint{0.626981in}{1.929825in}}%
\pgfpathlineto{\pgfqpoint{0.634712in}{1.936229in}}%
\pgfpathlineto{\pgfqpoint{0.642998in}{1.929825in}}%
\pgfpathlineto{\pgfqpoint{0.634712in}{1.921286in}}%
\pgfusepath{stroke}%
\end{pgfscope}%
\begin{pgfscope}%
\pgfpathrectangle{\pgfqpoint{0.625000in}{0.550000in}}{\pgfqpoint{3.875000in}{3.850000in}} %
\pgfusepath{clip}%
\pgfsetbuttcap%
\pgfsetroundjoin%
\pgfsetlinewidth{0.250937pt}%
\definecolor{currentstroke}{rgb}{0.000000,0.000000,0.000000}%
\pgfsetstrokecolor{currentstroke}%
\pgfsetdash{}{0pt}%
\pgfpathmoveto{\pgfqpoint{0.634712in}{3.023420in}}%
\pgfpathlineto{\pgfqpoint{0.626981in}{3.029825in}}%
\pgfpathlineto{\pgfqpoint{0.634712in}{3.038363in}}%
\pgfpathlineto{\pgfqpoint{0.642998in}{3.029825in}}%
\pgfpathlineto{\pgfqpoint{0.634712in}{3.023420in}}%
\pgfusepath{stroke}%
\end{pgfscope}%
\begin{pgfscope}%
\pgfpathrectangle{\pgfqpoint{0.625000in}{0.550000in}}{\pgfqpoint{3.875000in}{3.850000in}} %
\pgfusepath{clip}%
\pgfsetbuttcap%
\pgfsetroundjoin%
\pgfsetlinewidth{0.250937pt}%
\definecolor{currentstroke}{rgb}{0.000000,0.000000,0.000000}%
\pgfsetstrokecolor{currentstroke}%
\pgfsetdash{}{0pt}%
\pgfpathmoveto{\pgfqpoint{0.634712in}{3.574379in}}%
\pgfpathlineto{\pgfqpoint{0.631835in}{3.579825in}}%
\pgfpathlineto{\pgfqpoint{0.634275in}{3.589474in}}%
\pgfpathlineto{\pgfqpoint{0.634712in}{3.590433in}}%
\pgfpathlineto{\pgfqpoint{0.637128in}{3.589474in}}%
\pgfpathlineto{\pgfqpoint{0.642621in}{3.579825in}}%
\pgfpathlineto{\pgfqpoint{0.634712in}{3.574379in}}%
\pgfusepath{stroke}%
\end{pgfscope}%
\begin{pgfscope}%
\pgfpathrectangle{\pgfqpoint{0.625000in}{0.550000in}}{\pgfqpoint{3.875000in}{3.850000in}} %
\pgfusepath{clip}%
\pgfsetbuttcap%
\pgfsetroundjoin%
\pgfsetlinewidth{0.250937pt}%
\definecolor{currentstroke}{rgb}{0.000000,0.000000,0.000000}%
\pgfsetstrokecolor{currentstroke}%
\pgfsetdash{}{0pt}%
\pgfpathmoveto{\pgfqpoint{0.673559in}{3.711559in}}%
\pgfpathlineto{\pgfqpoint{0.665971in}{3.714912in}}%
\pgfpathlineto{\pgfqpoint{0.663847in}{3.715520in}}%
\pgfpathlineto{\pgfqpoint{0.654135in}{3.720509in}}%
\pgfpathlineto{\pgfqpoint{0.649956in}{3.724561in}}%
\pgfpathlineto{\pgfqpoint{0.644424in}{3.727836in}}%
\pgfpathlineto{\pgfqpoint{0.639738in}{3.734211in}}%
\pgfpathlineto{\pgfqpoint{0.634712in}{3.738371in}}%
\pgfpathlineto{\pgfqpoint{0.632939in}{3.743860in}}%
\pgfpathlineto{\pgfqpoint{0.628644in}{3.753509in}}%
\pgfpathlineto{\pgfqpoint{0.625499in}{3.763158in}}%
\pgfpathlineto{\pgfqpoint{0.626706in}{3.772807in}}%
\pgfpathlineto{\pgfqpoint{0.629477in}{3.782456in}}%
\pgfpathlineto{\pgfqpoint{0.634266in}{3.792105in}}%
\pgfpathlineto{\pgfqpoint{0.634712in}{3.793700in}}%
\pgfpathlineto{\pgfqpoint{0.642426in}{3.801754in}}%
\pgfpathlineto{\pgfqpoint{0.644424in}{3.804790in}}%
\pgfpathlineto{\pgfqpoint{0.653765in}{3.811404in}}%
\pgfpathlineto{\pgfqpoint{0.654135in}{3.811809in}}%
\pgfpathlineto{\pgfqpoint{0.663847in}{3.817510in}}%
\pgfpathlineto{\pgfqpoint{0.673559in}{3.820927in}}%
\pgfpathlineto{\pgfqpoint{0.674238in}{3.821053in}}%
\pgfpathlineto{\pgfqpoint{0.683271in}{3.823527in}}%
\pgfpathlineto{\pgfqpoint{0.692982in}{3.824725in}}%
\pgfpathlineto{\pgfqpoint{0.702694in}{3.824705in}}%
\pgfpathlineto{\pgfqpoint{0.712406in}{3.823477in}}%
\pgfpathlineto{\pgfqpoint{0.721793in}{3.821053in}}%
\pgfpathlineto{\pgfqpoint{0.722118in}{3.820951in}}%
\pgfpathlineto{\pgfqpoint{0.731830in}{3.816935in}}%
\pgfpathlineto{\pgfqpoint{0.741110in}{3.811404in}}%
\pgfpathlineto{\pgfqpoint{0.741541in}{3.811054in}}%
\pgfpathlineto{\pgfqpoint{0.751253in}{3.802247in}}%
\pgfpathlineto{\pgfqpoint{0.751756in}{3.801754in}}%
\pgfpathlineto{\pgfqpoint{0.758697in}{3.792105in}}%
\pgfpathlineto{\pgfqpoint{0.760965in}{3.786737in}}%
\pgfpathlineto{\pgfqpoint{0.762926in}{3.782456in}}%
\pgfpathlineto{\pgfqpoint{0.765188in}{3.772807in}}%
\pgfpathlineto{\pgfqpoint{0.765505in}{3.763158in}}%
\pgfpathlineto{\pgfqpoint{0.763933in}{3.753509in}}%
\pgfpathlineto{\pgfqpoint{0.760965in}{3.745801in}}%
\pgfpathlineto{\pgfqpoint{0.760288in}{3.743860in}}%
\pgfpathlineto{\pgfqpoint{0.754519in}{3.734211in}}%
\pgfpathlineto{\pgfqpoint{0.751253in}{3.730653in}}%
\pgfpathlineto{\pgfqpoint{0.745351in}{3.724561in}}%
\pgfpathlineto{\pgfqpoint{0.741541in}{3.721766in}}%
\pgfpathlineto{\pgfqpoint{0.731830in}{3.715799in}}%
\pgfpathlineto{\pgfqpoint{0.730058in}{3.714912in}}%
\pgfpathlineto{\pgfqpoint{0.722118in}{3.711777in}}%
\pgfpathlineto{\pgfqpoint{0.712406in}{3.709263in}}%
\pgfpathlineto{\pgfqpoint{0.702694in}{3.708064in}}%
\pgfpathlineto{\pgfqpoint{0.692982in}{3.708050in}}%
\pgfpathlineto{\pgfqpoint{0.683271in}{3.709173in}}%
\pgfpathlineto{\pgfqpoint{0.673559in}{3.711559in}}%
\pgfusepath{stroke}%
\end{pgfscope}%
\begin{pgfscope}%
\pgfpathrectangle{\pgfqpoint{0.625000in}{0.550000in}}{\pgfqpoint{3.875000in}{3.850000in}} %
\pgfusepath{clip}%
\pgfsetbuttcap%
\pgfsetroundjoin%
\pgfsetlinewidth{0.250937pt}%
\definecolor{currentstroke}{rgb}{0.000000,0.000000,0.000000}%
\pgfsetstrokecolor{currentstroke}%
\pgfsetdash{}{0pt}%
\pgfpathmoveto{\pgfqpoint{0.634712in}{4.125531in}}%
\pgfpathlineto{\pgfqpoint{0.632904in}{4.129825in}}%
\pgfpathlineto{\pgfqpoint{0.633421in}{4.139474in}}%
\pgfpathlineto{\pgfqpoint{0.634712in}{4.142309in}}%
\pgfpathlineto{\pgfqpoint{0.640302in}{4.139474in}}%
\pgfpathlineto{\pgfqpoint{0.641837in}{4.129825in}}%
\pgfpathlineto{\pgfqpoint{0.634712in}{4.125531in}}%
\pgfusepath{stroke}%
\end{pgfscope}%
\begin{pgfscope}%
\pgfpathrectangle{\pgfqpoint{0.625000in}{0.550000in}}{\pgfqpoint{3.875000in}{3.850000in}} %
\pgfusepath{clip}%
\pgfsetbuttcap%
\pgfsetroundjoin%
\pgfsetlinewidth{0.250937pt}%
\definecolor{currentstroke}{rgb}{0.000000,0.000000,0.000000}%
\pgfsetstrokecolor{currentstroke}%
\pgfsetdash{}{0pt}%
\pgfpathmoveto{\pgfqpoint{1.340132in}{0.550000in}}%
\pgfpathlineto{\pgfqpoint{1.339023in}{0.588596in}}%
\pgfpathlineto{\pgfqpoint{1.335683in}{0.627193in}}%
\pgfpathlineto{\pgfqpoint{1.330114in}{0.665789in}}%
\pgfpathlineto{\pgfqpoint{1.322265in}{0.704386in}}%
\pgfpathlineto{\pgfqpoint{1.312113in}{0.742982in}}%
\pgfpathlineto{\pgfqpoint{1.299604in}{0.781579in}}%
\pgfpathlineto{\pgfqpoint{1.284661in}{0.820175in}}%
\pgfpathlineto{\pgfqpoint{1.265977in}{0.861330in}}%
\pgfpathlineto{\pgfqpoint{1.246554in}{0.898577in}}%
\pgfpathlineto{\pgfqpoint{1.224559in}{0.935965in}}%
\pgfpathlineto{\pgfqpoint{1.197995in}{0.976164in}}%
\pgfpathlineto{\pgfqpoint{1.168860in}{1.015670in}}%
\pgfpathlineto{\pgfqpoint{1.139520in}{1.051754in}}%
\pgfpathlineto{\pgfqpoint{1.100877in}{1.095011in}}%
\pgfpathlineto{\pgfqpoint{1.052318in}{1.144460in}}%
\pgfpathlineto{\pgfqpoint{1.018846in}{1.177193in}}%
\pgfpathlineto{\pgfqpoint{0.994048in}{1.205148in}}%
\pgfpathlineto{\pgfqpoint{0.993207in}{1.206140in}}%
\pgfpathlineto{\pgfqpoint{0.984336in}{1.221913in}}%
\pgfpathlineto{\pgfqpoint{0.982720in}{1.225439in}}%
\pgfpathlineto{\pgfqpoint{0.980210in}{1.235088in}}%
\pgfpathlineto{\pgfqpoint{0.979342in}{1.244737in}}%
\pgfpathlineto{\pgfqpoint{0.980632in}{1.264035in}}%
\pgfpathlineto{\pgfqpoint{0.995698in}{1.370175in}}%
\pgfpathlineto{\pgfqpoint{0.996618in}{1.399123in}}%
\pgfpathlineto{\pgfqpoint{0.995758in}{1.428070in}}%
\pgfpathlineto{\pgfqpoint{0.993119in}{1.457018in}}%
\pgfpathlineto{\pgfqpoint{0.988742in}{1.485965in}}%
\pgfpathlineto{\pgfqpoint{0.980157in}{1.524561in}}%
\pgfpathlineto{\pgfqpoint{0.968522in}{1.563158in}}%
\pgfpathlineto{\pgfqpoint{0.953967in}{1.601754in}}%
\pgfpathlineto{\pgfqpoint{0.935777in}{1.642532in}}%
\pgfpathlineto{\pgfqpoint{0.912847in}{1.688596in}}%
\pgfpathlineto{\pgfqpoint{0.877506in}{1.759051in}}%
\pgfpathlineto{\pgfqpoint{0.853222in}{1.814035in}}%
\pgfpathlineto{\pgfqpoint{0.815131in}{1.900877in}}%
\pgfpathlineto{\pgfqpoint{0.770604in}{1.997368in}}%
\pgfpathlineto{\pgfqpoint{0.712406in}{2.134435in}}%
\pgfpathlineto{\pgfqpoint{0.683271in}{2.211965in}}%
\pgfpathlineto{\pgfqpoint{0.661979in}{2.277193in}}%
\pgfpathlineto{\pgfqpoint{0.648544in}{2.325439in}}%
\pgfpathlineto{\pgfqpoint{0.639498in}{2.364035in}}%
\pgfpathlineto{\pgfqpoint{0.633989in}{2.392982in}}%
\pgfpathlineto{\pgfqpoint{0.632784in}{2.402632in}}%
\pgfpathlineto{\pgfqpoint{0.629351in}{2.412281in}}%
\pgfpathlineto{\pgfqpoint{0.628216in}{2.431579in}}%
\pgfpathlineto{\pgfqpoint{0.625000in}{2.468079in}}%
\pgfpathlineto{\pgfqpoint{0.625000in}{2.468079in}}%
\pgfusepath{stroke}%
\end{pgfscope}%
\begin{pgfscope}%
\pgfpathrectangle{\pgfqpoint{0.625000in}{0.550000in}}{\pgfqpoint{3.875000in}{3.850000in}} %
\pgfusepath{clip}%
\pgfsetbuttcap%
\pgfsetroundjoin%
\pgfsetlinewidth{0.250937pt}%
\definecolor{currentstroke}{rgb}{0.000000,0.000000,0.000000}%
\pgfsetstrokecolor{currentstroke}%
\pgfsetdash{}{0pt}%
\pgfpathmoveto{\pgfqpoint{0.625000in}{0.636273in}}%
\pgfpathlineto{\pgfqpoint{0.634365in}{0.627193in}}%
\pgfpathlineto{\pgfqpoint{0.625000in}{0.618140in}}%
\pgfusepath{stroke}%
\end{pgfscope}%
\begin{pgfscope}%
\pgfpathrectangle{\pgfqpoint{0.625000in}{0.550000in}}{\pgfqpoint{3.875000in}{3.850000in}} %
\pgfusepath{clip}%
\pgfsetbuttcap%
\pgfsetroundjoin%
\pgfsetlinewidth{0.250937pt}%
\definecolor{currentstroke}{rgb}{0.000000,0.000000,0.000000}%
\pgfsetstrokecolor{currentstroke}%
\pgfsetdash{}{0pt}%
\pgfpathmoveto{\pgfqpoint{0.625000in}{0.790269in}}%
\pgfpathlineto{\pgfqpoint{0.634401in}{0.781579in}}%
\pgfpathlineto{\pgfqpoint{0.625000in}{0.772502in}}%
\pgfusepath{stroke}%
\end{pgfscope}%
\begin{pgfscope}%
\pgfpathrectangle{\pgfqpoint{0.625000in}{0.550000in}}{\pgfqpoint{3.875000in}{3.850000in}} %
\pgfusepath{clip}%
\pgfsetbuttcap%
\pgfsetroundjoin%
\pgfsetlinewidth{0.250937pt}%
\definecolor{currentstroke}{rgb}{0.000000,0.000000,0.000000}%
\pgfsetstrokecolor{currentstroke}%
\pgfsetdash{}{0pt}%
\pgfpathmoveto{\pgfqpoint{0.625000in}{0.944936in}}%
\pgfpathlineto{\pgfqpoint{0.631162in}{0.945614in}}%
\pgfpathlineto{\pgfqpoint{0.634712in}{0.953057in}}%
\pgfpathlineto{\pgfqpoint{0.643229in}{0.955263in}}%
\pgfpathlineto{\pgfqpoint{0.644424in}{0.956323in}}%
\pgfpathlineto{\pgfqpoint{0.654135in}{0.956874in}}%
\pgfpathlineto{\pgfqpoint{0.660213in}{0.955263in}}%
\pgfpathlineto{\pgfqpoint{0.663847in}{0.953556in}}%
\pgfpathlineto{\pgfqpoint{0.672688in}{0.945614in}}%
\pgfpathlineto{\pgfqpoint{0.673559in}{0.941926in}}%
\pgfpathlineto{\pgfqpoint{0.675509in}{0.935965in}}%
\pgfpathlineto{\pgfqpoint{0.673559in}{0.930021in}}%
\pgfpathlineto{\pgfqpoint{0.672682in}{0.926316in}}%
\pgfpathlineto{\pgfqpoint{0.663847in}{0.918371in}}%
\pgfpathlineto{\pgfqpoint{0.660218in}{0.916667in}}%
\pgfpathlineto{\pgfqpoint{0.654135in}{0.915058in}}%
\pgfpathlineto{\pgfqpoint{0.644424in}{0.915605in}}%
\pgfpathlineto{\pgfqpoint{0.643230in}{0.916667in}}%
\pgfpathlineto{\pgfqpoint{0.634712in}{0.918873in}}%
\pgfpathlineto{\pgfqpoint{0.630851in}{0.926316in}}%
\pgfpathlineto{\pgfqpoint{0.625000in}{0.926913in}}%
\pgfusepath{stroke}%
\end{pgfscope}%
\begin{pgfscope}%
\pgfpathrectangle{\pgfqpoint{0.625000in}{0.550000in}}{\pgfqpoint{3.875000in}{3.850000in}} %
\pgfusepath{clip}%
\pgfsetbuttcap%
\pgfsetroundjoin%
\pgfsetlinewidth{0.250937pt}%
\definecolor{currentstroke}{rgb}{0.000000,0.000000,0.000000}%
\pgfsetstrokecolor{currentstroke}%
\pgfsetdash{}{0pt}%
\pgfpathmoveto{\pgfqpoint{0.625000in}{1.099359in}}%
\pgfpathlineto{\pgfqpoint{0.634393in}{1.090351in}}%
\pgfpathlineto{\pgfqpoint{0.625000in}{1.081377in}}%
\pgfusepath{stroke}%
\end{pgfscope}%
\begin{pgfscope}%
\pgfpathrectangle{\pgfqpoint{0.625000in}{0.550000in}}{\pgfqpoint{3.875000in}{3.850000in}} %
\pgfusepath{clip}%
\pgfsetbuttcap%
\pgfsetroundjoin%
\pgfsetlinewidth{0.250937pt}%
\definecolor{currentstroke}{rgb}{0.000000,0.000000,0.000000}%
\pgfsetstrokecolor{currentstroke}%
\pgfsetdash{}{0pt}%
\pgfpathmoveto{\pgfqpoint{0.625000in}{1.254041in}}%
\pgfpathlineto{\pgfqpoint{0.634427in}{1.244737in}}%
\pgfpathlineto{\pgfqpoint{0.625000in}{1.235690in}}%
\pgfusepath{stroke}%
\end{pgfscope}%
\begin{pgfscope}%
\pgfpathrectangle{\pgfqpoint{0.625000in}{0.550000in}}{\pgfqpoint{3.875000in}{3.850000in}} %
\pgfusepath{clip}%
\pgfsetbuttcap%
\pgfsetroundjoin%
\pgfsetlinewidth{0.250937pt}%
\definecolor{currentstroke}{rgb}{0.000000,0.000000,0.000000}%
\pgfsetstrokecolor{currentstroke}%
\pgfsetdash{}{0pt}%
\pgfpathmoveto{\pgfqpoint{0.625000in}{1.293105in}}%
\pgfpathlineto{\pgfqpoint{0.625343in}{1.292982in}}%
\pgfpathlineto{\pgfqpoint{0.625000in}{1.292879in}}%
\pgfusepath{stroke}%
\end{pgfscope}%
\begin{pgfscope}%
\pgfpathrectangle{\pgfqpoint{0.625000in}{0.550000in}}{\pgfqpoint{3.875000in}{3.850000in}} %
\pgfusepath{clip}%
\pgfsetbuttcap%
\pgfsetroundjoin%
\pgfsetlinewidth{0.250937pt}%
\definecolor{currentstroke}{rgb}{0.000000,0.000000,0.000000}%
\pgfsetstrokecolor{currentstroke}%
\pgfsetdash{}{0pt}%
\pgfpathmoveto{\pgfqpoint{0.625000in}{1.408068in}}%
\pgfpathlineto{\pgfqpoint{0.634403in}{1.399123in}}%
\pgfpathlineto{\pgfqpoint{0.625000in}{1.390173in}}%
\pgfusepath{stroke}%
\end{pgfscope}%
\begin{pgfscope}%
\pgfpathrectangle{\pgfqpoint{0.625000in}{0.550000in}}{\pgfqpoint{3.875000in}{3.850000in}} %
\pgfusepath{clip}%
\pgfsetbuttcap%
\pgfsetroundjoin%
\pgfsetlinewidth{0.250937pt}%
\definecolor{currentstroke}{rgb}{0.000000,0.000000,0.000000}%
\pgfsetstrokecolor{currentstroke}%
\pgfsetdash{}{0pt}%
\pgfpathmoveto{\pgfqpoint{0.625000in}{1.562412in}}%
\pgfpathlineto{\pgfqpoint{0.634344in}{1.553509in}}%
\pgfpathlineto{\pgfqpoint{0.625000in}{1.544317in}}%
\pgfusepath{stroke}%
\end{pgfscope}%
\begin{pgfscope}%
\pgfpathrectangle{\pgfqpoint{0.625000in}{0.550000in}}{\pgfqpoint{3.875000in}{3.850000in}} %
\pgfusepath{clip}%
\pgfsetbuttcap%
\pgfsetroundjoin%
\pgfsetlinewidth{0.250937pt}%
\definecolor{currentstroke}{rgb}{0.000000,0.000000,0.000000}%
\pgfsetstrokecolor{currentstroke}%
\pgfsetdash{}{0pt}%
\pgfpathmoveto{\pgfqpoint{0.625000in}{1.716806in}}%
\pgfpathlineto{\pgfqpoint{0.631224in}{1.717544in}}%
\pgfpathlineto{\pgfqpoint{0.634712in}{1.724987in}}%
\pgfpathlineto{\pgfqpoint{0.643229in}{1.727193in}}%
\pgfpathlineto{\pgfqpoint{0.644424in}{1.728254in}}%
\pgfpathlineto{\pgfqpoint{0.654135in}{1.728803in}}%
\pgfpathlineto{\pgfqpoint{0.660215in}{1.727193in}}%
\pgfpathlineto{\pgfqpoint{0.663847in}{1.725487in}}%
\pgfpathlineto{\pgfqpoint{0.672685in}{1.717544in}}%
\pgfpathlineto{\pgfqpoint{0.673559in}{1.713854in}}%
\pgfpathlineto{\pgfqpoint{0.675516in}{1.707895in}}%
\pgfpathlineto{\pgfqpoint{0.673559in}{1.701947in}}%
\pgfpathlineto{\pgfqpoint{0.672681in}{1.698246in}}%
\pgfpathlineto{\pgfqpoint{0.663847in}{1.690301in}}%
\pgfpathlineto{\pgfqpoint{0.660218in}{1.688596in}}%
\pgfpathlineto{\pgfqpoint{0.654135in}{1.686988in}}%
\pgfpathlineto{\pgfqpoint{0.644424in}{1.687535in}}%
\pgfpathlineto{\pgfqpoint{0.643230in}{1.688596in}}%
\pgfpathlineto{\pgfqpoint{0.634712in}{1.690803in}}%
\pgfpathlineto{\pgfqpoint{0.631030in}{1.698246in}}%
\pgfpathlineto{\pgfqpoint{0.625000in}{1.698927in}}%
\pgfusepath{stroke}%
\end{pgfscope}%
\begin{pgfscope}%
\pgfpathrectangle{\pgfqpoint{0.625000in}{0.550000in}}{\pgfqpoint{3.875000in}{3.850000in}} %
\pgfusepath{clip}%
\pgfsetbuttcap%
\pgfsetroundjoin%
\pgfsetlinewidth{0.250937pt}%
\definecolor{currentstroke}{rgb}{0.000000,0.000000,0.000000}%
\pgfsetstrokecolor{currentstroke}%
\pgfsetdash{}{0pt}%
\pgfpathmoveto{\pgfqpoint{0.625000in}{1.871281in}}%
\pgfpathlineto{\pgfqpoint{0.634341in}{1.862281in}}%
\pgfpathlineto{\pgfqpoint{0.625000in}{1.853466in}}%
\pgfusepath{stroke}%
\end{pgfscope}%
\begin{pgfscope}%
\pgfpathrectangle{\pgfqpoint{0.625000in}{0.550000in}}{\pgfqpoint{3.875000in}{3.850000in}} %
\pgfusepath{clip}%
\pgfsetbuttcap%
\pgfsetroundjoin%
\pgfsetlinewidth{0.250937pt}%
\definecolor{currentstroke}{rgb}{0.000000,0.000000,0.000000}%
\pgfsetstrokecolor{currentstroke}%
\pgfsetdash{}{0pt}%
\pgfpathmoveto{\pgfqpoint{0.625000in}{1.891644in}}%
\pgfpathlineto{\pgfqpoint{0.626422in}{1.891228in}}%
\pgfpathlineto{\pgfqpoint{0.625000in}{1.890565in}}%
\pgfusepath{stroke}%
\end{pgfscope}%
\begin{pgfscope}%
\pgfpathrectangle{\pgfqpoint{0.625000in}{0.550000in}}{\pgfqpoint{3.875000in}{3.850000in}} %
\pgfusepath{clip}%
\pgfsetbuttcap%
\pgfsetroundjoin%
\pgfsetlinewidth{0.250937pt}%
\definecolor{currentstroke}{rgb}{0.000000,0.000000,0.000000}%
\pgfsetstrokecolor{currentstroke}%
\pgfsetdash{}{0pt}%
\pgfpathmoveto{\pgfqpoint{0.625000in}{2.026001in}}%
\pgfpathlineto{\pgfqpoint{0.634343in}{2.016667in}}%
\pgfpathlineto{\pgfqpoint{0.625000in}{2.007409in}}%
\pgfusepath{stroke}%
\end{pgfscope}%
\begin{pgfscope}%
\pgfpathrectangle{\pgfqpoint{0.625000in}{0.550000in}}{\pgfqpoint{3.875000in}{3.850000in}} %
\pgfusepath{clip}%
\pgfsetbuttcap%
\pgfsetroundjoin%
\pgfsetlinewidth{0.250937pt}%
\definecolor{currentstroke}{rgb}{0.000000,0.000000,0.000000}%
\pgfsetstrokecolor{currentstroke}%
\pgfsetdash{}{0pt}%
\pgfpathmoveto{\pgfqpoint{0.625000in}{2.179908in}}%
\pgfpathlineto{\pgfqpoint{0.634365in}{2.171053in}}%
\pgfpathlineto{\pgfqpoint{0.625000in}{2.161944in}}%
\pgfusepath{stroke}%
\end{pgfscope}%
\begin{pgfscope}%
\pgfpathrectangle{\pgfqpoint{0.625000in}{0.550000in}}{\pgfqpoint{3.875000in}{3.850000in}} %
\pgfusepath{clip}%
\pgfsetbuttcap%
\pgfsetroundjoin%
\pgfsetlinewidth{0.250937pt}%
\definecolor{currentstroke}{rgb}{0.000000,0.000000,0.000000}%
\pgfsetstrokecolor{currentstroke}%
\pgfsetdash{}{0pt}%
\pgfpathmoveto{\pgfqpoint{0.625000in}{2.334259in}}%
\pgfpathlineto{\pgfqpoint{0.634416in}{2.325439in}}%
\pgfpathlineto{\pgfqpoint{0.625000in}{2.316001in}}%
\pgfusepath{stroke}%
\end{pgfscope}%
\begin{pgfscope}%
\pgfpathrectangle{\pgfqpoint{0.625000in}{0.550000in}}{\pgfqpoint{3.875000in}{3.850000in}} %
\pgfusepath{clip}%
\pgfsetbuttcap%
\pgfsetroundjoin%
\pgfsetlinewidth{0.250937pt}%
\definecolor{currentstroke}{rgb}{0.000000,0.000000,0.000000}%
\pgfsetstrokecolor{currentstroke}%
\pgfsetdash{}{0pt}%
\pgfpathmoveto{\pgfqpoint{0.625000in}{2.491571in}}%
\pgfpathlineto{\pgfqpoint{0.629351in}{2.547368in}}%
\pgfpathlineto{\pgfqpoint{0.632784in}{2.557018in}}%
\pgfpathlineto{\pgfqpoint{0.637501in}{2.585965in}}%
\pgfpathlineto{\pgfqpoint{0.648544in}{2.634211in}}%
\pgfpathlineto{\pgfqpoint{0.663847in}{2.688878in}}%
\pgfpathlineto{\pgfqpoint{0.694657in}{2.778947in}}%
\pgfpathlineto{\pgfqpoint{0.722118in}{2.849254in}}%
\pgfpathlineto{\pgfqpoint{0.770677in}{2.962456in}}%
\pgfpathlineto{\pgfqpoint{0.801435in}{3.029825in}}%
\pgfpathlineto{\pgfqpoint{0.832625in}{3.097368in}}%
\pgfpathlineto{\pgfqpoint{0.887218in}{3.220738in}}%
\pgfpathlineto{\pgfqpoint{0.949925in}{3.348246in}}%
\pgfpathlineto{\pgfqpoint{0.965124in}{3.386842in}}%
\pgfpathlineto{\pgfqpoint{0.974682in}{3.415789in}}%
\pgfpathlineto{\pgfqpoint{0.984336in}{3.452415in}}%
\pgfpathlineto{\pgfqpoint{0.988742in}{3.473684in}}%
\pgfpathlineto{\pgfqpoint{0.993119in}{3.502632in}}%
\pgfpathlineto{\pgfqpoint{0.996244in}{3.541228in}}%
\pgfpathlineto{\pgfqpoint{0.996508in}{3.570175in}}%
\pgfpathlineto{\pgfqpoint{0.995003in}{3.599123in}}%
\pgfpathlineto{\pgfqpoint{0.990424in}{3.637719in}}%
\pgfpathlineto{\pgfqpoint{0.979616in}{3.705263in}}%
\pgfpathlineto{\pgfqpoint{0.979342in}{3.714912in}}%
\pgfpathlineto{\pgfqpoint{0.980210in}{3.724561in}}%
\pgfpathlineto{\pgfqpoint{0.984336in}{3.737736in}}%
\pgfpathlineto{\pgfqpoint{0.986985in}{3.743860in}}%
\pgfpathlineto{\pgfqpoint{0.994048in}{3.754501in}}%
\pgfpathlineto{\pgfqpoint{1.009477in}{3.772807in}}%
\pgfpathlineto{\pgfqpoint{1.091165in}{3.854388in}}%
\pgfpathlineto{\pgfqpoint{1.131221in}{3.898246in}}%
\pgfpathlineto{\pgfqpoint{1.163259in}{3.936842in}}%
\pgfpathlineto{\pgfqpoint{1.192310in}{3.975439in}}%
\pgfpathlineto{\pgfqpoint{1.218459in}{4.014035in}}%
\pgfpathlineto{\pgfqpoint{1.241821in}{4.052632in}}%
\pgfpathlineto{\pgfqpoint{1.262484in}{4.091228in}}%
\pgfpathlineto{\pgfqpoint{1.280550in}{4.129825in}}%
\pgfpathlineto{\pgfqpoint{1.296092in}{4.168421in}}%
\pgfpathlineto{\pgfqpoint{1.309212in}{4.207018in}}%
\pgfpathlineto{\pgfqpoint{1.319950in}{4.245614in}}%
\pgfpathlineto{\pgfqpoint{1.328367in}{4.284211in}}%
\pgfpathlineto{\pgfqpoint{1.334493in}{4.322807in}}%
\pgfpathlineto{\pgfqpoint{1.338399in}{4.361404in}}%
\pgfpathlineto{\pgfqpoint{1.340062in}{4.400000in}}%
\pgfpathlineto{\pgfqpoint{1.340062in}{4.400000in}}%
\pgfusepath{stroke}%
\end{pgfscope}%
\begin{pgfscope}%
\pgfpathrectangle{\pgfqpoint{0.625000in}{0.550000in}}{\pgfqpoint{3.875000in}{3.850000in}} %
\pgfusepath{clip}%
\pgfsetbuttcap%
\pgfsetroundjoin%
\pgfsetlinewidth{0.250937pt}%
\definecolor{currentstroke}{rgb}{0.000000,0.000000,0.000000}%
\pgfsetstrokecolor{currentstroke}%
\pgfsetdash{}{0pt}%
\pgfpathmoveto{\pgfqpoint{0.625000in}{2.643648in}}%
\pgfpathlineto{\pgfqpoint{0.634416in}{2.634211in}}%
\pgfpathlineto{\pgfqpoint{0.625000in}{2.625392in}}%
\pgfusepath{stroke}%
\end{pgfscope}%
\begin{pgfscope}%
\pgfpathrectangle{\pgfqpoint{0.625000in}{0.550000in}}{\pgfqpoint{3.875000in}{3.850000in}} %
\pgfusepath{clip}%
\pgfsetbuttcap%
\pgfsetroundjoin%
\pgfsetlinewidth{0.250937pt}%
\definecolor{currentstroke}{rgb}{0.000000,0.000000,0.000000}%
\pgfsetstrokecolor{currentstroke}%
\pgfsetdash{}{0pt}%
\pgfpathmoveto{\pgfqpoint{0.625000in}{2.797702in}}%
\pgfpathlineto{\pgfqpoint{0.634362in}{2.788596in}}%
\pgfpathlineto{\pgfqpoint{0.625000in}{2.779746in}}%
\pgfusepath{stroke}%
\end{pgfscope}%
\begin{pgfscope}%
\pgfpathrectangle{\pgfqpoint{0.625000in}{0.550000in}}{\pgfqpoint{3.875000in}{3.850000in}} %
\pgfusepath{clip}%
\pgfsetbuttcap%
\pgfsetroundjoin%
\pgfsetlinewidth{0.250937pt}%
\definecolor{currentstroke}{rgb}{0.000000,0.000000,0.000000}%
\pgfsetstrokecolor{currentstroke}%
\pgfsetdash{}{0pt}%
\pgfpathmoveto{\pgfqpoint{0.625000in}{2.952234in}}%
\pgfpathlineto{\pgfqpoint{0.634337in}{2.942982in}}%
\pgfpathlineto{\pgfqpoint{0.625000in}{2.933654in}}%
\pgfusepath{stroke}%
\end{pgfscope}%
\begin{pgfscope}%
\pgfpathrectangle{\pgfqpoint{0.625000in}{0.550000in}}{\pgfqpoint{3.875000in}{3.850000in}} %
\pgfusepath{clip}%
\pgfsetbuttcap%
\pgfsetroundjoin%
\pgfsetlinewidth{0.250937pt}%
\definecolor{currentstroke}{rgb}{0.000000,0.000000,0.000000}%
\pgfsetstrokecolor{currentstroke}%
\pgfsetdash{}{0pt}%
\pgfpathmoveto{\pgfqpoint{0.625000in}{3.069084in}}%
\pgfpathlineto{\pgfqpoint{0.626422in}{3.068421in}}%
\pgfpathlineto{\pgfqpoint{0.625000in}{3.068005in}}%
\pgfusepath{stroke}%
\end{pgfscope}%
\begin{pgfscope}%
\pgfpathrectangle{\pgfqpoint{0.625000in}{0.550000in}}{\pgfqpoint{3.875000in}{3.850000in}} %
\pgfusepath{clip}%
\pgfsetbuttcap%
\pgfsetroundjoin%
\pgfsetlinewidth{0.250937pt}%
\definecolor{currentstroke}{rgb}{0.000000,0.000000,0.000000}%
\pgfsetstrokecolor{currentstroke}%
\pgfsetdash{}{0pt}%
\pgfpathmoveto{\pgfqpoint{0.625000in}{3.106172in}}%
\pgfpathlineto{\pgfqpoint{0.634336in}{3.097368in}}%
\pgfpathlineto{\pgfqpoint{0.625000in}{3.088377in}}%
\pgfusepath{stroke}%
\end{pgfscope}%
\begin{pgfscope}%
\pgfpathrectangle{\pgfqpoint{0.625000in}{0.550000in}}{\pgfqpoint{3.875000in}{3.850000in}} %
\pgfusepath{clip}%
\pgfsetbuttcap%
\pgfsetroundjoin%
\pgfsetlinewidth{0.250937pt}%
\definecolor{currentstroke}{rgb}{0.000000,0.000000,0.000000}%
\pgfsetstrokecolor{currentstroke}%
\pgfsetdash{}{0pt}%
\pgfpathmoveto{\pgfqpoint{0.625000in}{3.260689in}}%
\pgfpathlineto{\pgfqpoint{0.631030in}{3.261404in}}%
\pgfpathlineto{\pgfqpoint{0.634712in}{3.268846in}}%
\pgfpathlineto{\pgfqpoint{0.643230in}{3.271053in}}%
\pgfpathlineto{\pgfqpoint{0.644424in}{3.272114in}}%
\pgfpathlineto{\pgfqpoint{0.654135in}{3.272661in}}%
\pgfpathlineto{\pgfqpoint{0.660218in}{3.271053in}}%
\pgfpathlineto{\pgfqpoint{0.663847in}{3.269349in}}%
\pgfpathlineto{\pgfqpoint{0.672681in}{3.261404in}}%
\pgfpathlineto{\pgfqpoint{0.673559in}{3.257703in}}%
\pgfpathlineto{\pgfqpoint{0.675516in}{3.251754in}}%
\pgfpathlineto{\pgfqpoint{0.673559in}{3.245795in}}%
\pgfpathlineto{\pgfqpoint{0.672685in}{3.242105in}}%
\pgfpathlineto{\pgfqpoint{0.663847in}{3.234162in}}%
\pgfpathlineto{\pgfqpoint{0.660215in}{3.232456in}}%
\pgfpathlineto{\pgfqpoint{0.654135in}{3.230846in}}%
\pgfpathlineto{\pgfqpoint{0.644424in}{3.231395in}}%
\pgfpathlineto{\pgfqpoint{0.643229in}{3.232456in}}%
\pgfpathlineto{\pgfqpoint{0.634712in}{3.234662in}}%
\pgfpathlineto{\pgfqpoint{0.631224in}{3.242105in}}%
\pgfpathlineto{\pgfqpoint{0.625000in}{3.242878in}}%
\pgfusepath{stroke}%
\end{pgfscope}%
\begin{pgfscope}%
\pgfpathrectangle{\pgfqpoint{0.625000in}{0.550000in}}{\pgfqpoint{3.875000in}{3.850000in}} %
\pgfusepath{clip}%
\pgfsetbuttcap%
\pgfsetroundjoin%
\pgfsetlinewidth{0.250937pt}%
\definecolor{currentstroke}{rgb}{0.000000,0.000000,0.000000}%
\pgfsetstrokecolor{currentstroke}%
\pgfsetdash{}{0pt}%
\pgfpathmoveto{\pgfqpoint{0.625000in}{3.415317in}}%
\pgfpathlineto{\pgfqpoint{0.634332in}{3.406140in}}%
\pgfpathlineto{\pgfqpoint{0.625000in}{3.397262in}}%
\pgfusepath{stroke}%
\end{pgfscope}%
\begin{pgfscope}%
\pgfpathrectangle{\pgfqpoint{0.625000in}{0.550000in}}{\pgfqpoint{3.875000in}{3.850000in}} %
\pgfusepath{clip}%
\pgfsetbuttcap%
\pgfsetroundjoin%
\pgfsetlinewidth{0.250937pt}%
\definecolor{currentstroke}{rgb}{0.000000,0.000000,0.000000}%
\pgfsetstrokecolor{currentstroke}%
\pgfsetdash{}{0pt}%
\pgfpathmoveto{\pgfqpoint{0.625000in}{3.569449in}}%
\pgfpathlineto{\pgfqpoint{0.634390in}{3.560526in}}%
\pgfpathlineto{\pgfqpoint{0.625000in}{3.551609in}}%
\pgfusepath{stroke}%
\end{pgfscope}%
\begin{pgfscope}%
\pgfpathrectangle{\pgfqpoint{0.625000in}{0.550000in}}{\pgfqpoint{3.875000in}{3.850000in}} %
\pgfusepath{clip}%
\pgfsetbuttcap%
\pgfsetroundjoin%
\pgfsetlinewidth{0.250937pt}%
\definecolor{currentstroke}{rgb}{0.000000,0.000000,0.000000}%
\pgfsetstrokecolor{currentstroke}%
\pgfsetdash{}{0pt}%
\pgfpathmoveto{\pgfqpoint{0.625000in}{3.666770in}}%
\pgfpathlineto{\pgfqpoint{0.625343in}{3.666667in}}%
\pgfpathlineto{\pgfqpoint{0.625000in}{3.666544in}}%
\pgfusepath{stroke}%
\end{pgfscope}%
\begin{pgfscope}%
\pgfpathrectangle{\pgfqpoint{0.625000in}{0.550000in}}{\pgfqpoint{3.875000in}{3.850000in}} %
\pgfusepath{clip}%
\pgfsetbuttcap%
\pgfsetroundjoin%
\pgfsetlinewidth{0.250937pt}%
\definecolor{currentstroke}{rgb}{0.000000,0.000000,0.000000}%
\pgfsetstrokecolor{currentstroke}%
\pgfsetdash{}{0pt}%
\pgfpathmoveto{\pgfqpoint{0.625000in}{3.723923in}}%
\pgfpathlineto{\pgfqpoint{0.634409in}{3.714912in}}%
\pgfpathlineto{\pgfqpoint{0.625000in}{3.705630in}}%
\pgfusepath{stroke}%
\end{pgfscope}%
\begin{pgfscope}%
\pgfpathrectangle{\pgfqpoint{0.625000in}{0.550000in}}{\pgfqpoint{3.875000in}{3.850000in}} %
\pgfusepath{clip}%
\pgfsetbuttcap%
\pgfsetroundjoin%
\pgfsetlinewidth{0.250937pt}%
\definecolor{currentstroke}{rgb}{0.000000,0.000000,0.000000}%
\pgfsetstrokecolor{currentstroke}%
\pgfsetdash{}{0pt}%
\pgfpathmoveto{\pgfqpoint{0.625000in}{3.878205in}}%
\pgfpathlineto{\pgfqpoint{0.634360in}{3.869298in}}%
\pgfpathlineto{\pgfqpoint{0.625000in}{3.860355in}}%
\pgfusepath{stroke}%
\end{pgfscope}%
\begin{pgfscope}%
\pgfpathrectangle{\pgfqpoint{0.625000in}{0.550000in}}{\pgfqpoint{3.875000in}{3.850000in}} %
\pgfusepath{clip}%
\pgfsetbuttcap%
\pgfsetroundjoin%
\pgfsetlinewidth{0.250937pt}%
\definecolor{currentstroke}{rgb}{0.000000,0.000000,0.000000}%
\pgfsetstrokecolor{currentstroke}%
\pgfsetdash{}{0pt}%
\pgfpathmoveto{\pgfqpoint{0.625000in}{4.032680in}}%
\pgfpathlineto{\pgfqpoint{0.630851in}{4.033333in}}%
\pgfpathlineto{\pgfqpoint{0.634712in}{4.040776in}}%
\pgfpathlineto{\pgfqpoint{0.643230in}{4.042982in}}%
\pgfpathlineto{\pgfqpoint{0.644424in}{4.044044in}}%
\pgfpathlineto{\pgfqpoint{0.654135in}{4.044591in}}%
\pgfpathlineto{\pgfqpoint{0.660218in}{4.042982in}}%
\pgfpathlineto{\pgfqpoint{0.663847in}{4.041278in}}%
\pgfpathlineto{\pgfqpoint{0.672682in}{4.033333in}}%
\pgfpathlineto{\pgfqpoint{0.673559in}{4.029628in}}%
\pgfpathlineto{\pgfqpoint{0.675509in}{4.023684in}}%
\pgfpathlineto{\pgfqpoint{0.673559in}{4.017723in}}%
\pgfpathlineto{\pgfqpoint{0.672688in}{4.014035in}}%
\pgfpathlineto{\pgfqpoint{0.663847in}{4.006093in}}%
\pgfpathlineto{\pgfqpoint{0.660213in}{4.004386in}}%
\pgfpathlineto{\pgfqpoint{0.654135in}{4.002775in}}%
\pgfpathlineto{\pgfqpoint{0.644424in}{4.003326in}}%
\pgfpathlineto{\pgfqpoint{0.643229in}{4.004386in}}%
\pgfpathlineto{\pgfqpoint{0.634712in}{4.006592in}}%
\pgfpathlineto{\pgfqpoint{0.631162in}{4.014035in}}%
\pgfpathlineto{\pgfqpoint{0.625000in}{4.014776in}}%
\pgfusepath{stroke}%
\end{pgfscope}%
\begin{pgfscope}%
\pgfpathrectangle{\pgfqpoint{0.625000in}{0.550000in}}{\pgfqpoint{3.875000in}{3.850000in}} %
\pgfusepath{clip}%
\pgfsetbuttcap%
\pgfsetroundjoin%
\pgfsetlinewidth{0.250937pt}%
\definecolor{currentstroke}{rgb}{0.000000,0.000000,0.000000}%
\pgfsetstrokecolor{currentstroke}%
\pgfsetdash{}{0pt}%
\pgfpathmoveto{\pgfqpoint{0.625000in}{4.187088in}}%
\pgfpathlineto{\pgfqpoint{0.634368in}{4.178070in}}%
\pgfpathlineto{\pgfqpoint{0.625000in}{4.169474in}}%
\pgfusepath{stroke}%
\end{pgfscope}%
\begin{pgfscope}%
\pgfpathrectangle{\pgfqpoint{0.625000in}{0.550000in}}{\pgfqpoint{3.875000in}{3.850000in}} %
\pgfusepath{clip}%
\pgfsetbuttcap%
\pgfsetroundjoin%
\pgfsetlinewidth{0.250937pt}%
\definecolor{currentstroke}{rgb}{0.000000,0.000000,0.000000}%
\pgfsetstrokecolor{currentstroke}%
\pgfsetdash{}{0pt}%
\pgfpathmoveto{\pgfqpoint{0.625000in}{4.341419in}}%
\pgfpathlineto{\pgfqpoint{0.634310in}{4.332456in}}%
\pgfpathlineto{\pgfqpoint{0.625000in}{4.323463in}}%
\pgfusepath{stroke}%
\end{pgfscope}%
\begin{pgfscope}%
\pgfpathrectangle{\pgfqpoint{0.625000in}{0.550000in}}{\pgfqpoint{3.875000in}{3.850000in}} %
\pgfusepath{clip}%
\pgfsetbuttcap%
\pgfsetroundjoin%
\pgfsetlinewidth{0.250937pt}%
\definecolor{currentstroke}{rgb}{0.000000,0.000000,0.000000}%
\pgfsetstrokecolor{currentstroke}%
\pgfsetdash{}{0pt}%
\pgfpathmoveto{\pgfqpoint{0.634712in}{0.760968in}}%
\pgfpathlineto{\pgfqpoint{0.634123in}{0.762281in}}%
\pgfpathlineto{\pgfqpoint{0.634712in}{0.764385in}}%
\pgfpathlineto{\pgfqpoint{0.636755in}{0.762281in}}%
\pgfpathlineto{\pgfqpoint{0.634712in}{0.760968in}}%
\pgfusepath{stroke}%
\end{pgfscope}%
\begin{pgfscope}%
\pgfpathrectangle{\pgfqpoint{0.625000in}{0.550000in}}{\pgfqpoint{3.875000in}{3.850000in}} %
\pgfusepath{clip}%
\pgfsetbuttcap%
\pgfsetroundjoin%
\pgfsetlinewidth{0.250937pt}%
\definecolor{currentstroke}{rgb}{0.000000,0.000000,0.000000}%
\pgfsetstrokecolor{currentstroke}%
\pgfsetdash{}{0pt}%
\pgfpathmoveto{\pgfqpoint{0.634712in}{0.815414in}}%
\pgfpathlineto{\pgfqpoint{0.632543in}{0.820175in}}%
\pgfpathlineto{\pgfqpoint{0.632188in}{0.829825in}}%
\pgfpathlineto{\pgfqpoint{0.634712in}{0.835818in}}%
\pgfpathlineto{\pgfqpoint{0.644424in}{0.830357in}}%
\pgfpathlineto{\pgfqpoint{0.645128in}{0.829825in}}%
\pgfpathlineto{\pgfqpoint{0.644424in}{0.825038in}}%
\pgfpathlineto{\pgfqpoint{0.644100in}{0.820175in}}%
\pgfpathlineto{\pgfqpoint{0.634712in}{0.815414in}}%
\pgfusepath{stroke}%
\end{pgfscope}%
\begin{pgfscope}%
\pgfpathrectangle{\pgfqpoint{0.625000in}{0.550000in}}{\pgfqpoint{3.875000in}{3.850000in}} %
\pgfusepath{clip}%
\pgfsetbuttcap%
\pgfsetroundjoin%
\pgfsetlinewidth{0.250937pt}%
\definecolor{currentstroke}{rgb}{0.000000,0.000000,0.000000}%
\pgfsetstrokecolor{currentstroke}%
\pgfsetdash{}{0pt}%
\pgfpathmoveto{\pgfqpoint{0.731830in}{1.108723in}}%
\pgfpathlineto{\pgfqpoint{0.726575in}{1.109649in}}%
\pgfpathlineto{\pgfqpoint{0.722118in}{1.110265in}}%
\pgfpathlineto{\pgfqpoint{0.712406in}{1.112263in}}%
\pgfpathlineto{\pgfqpoint{0.702694in}{1.114901in}}%
\pgfpathlineto{\pgfqpoint{0.692982in}{1.118403in}}%
\pgfpathlineto{\pgfqpoint{0.691288in}{1.119298in}}%
\pgfpathlineto{\pgfqpoint{0.683271in}{1.122400in}}%
\pgfpathlineto{\pgfqpoint{0.673559in}{1.127583in}}%
\pgfpathlineto{\pgfqpoint{0.671867in}{1.128947in}}%
\pgfpathlineto{\pgfqpoint{0.663847in}{1.133493in}}%
\pgfpathlineto{\pgfqpoint{0.658181in}{1.138596in}}%
\pgfpathlineto{\pgfqpoint{0.654135in}{1.141050in}}%
\pgfpathlineto{\pgfqpoint{0.647562in}{1.148246in}}%
\pgfpathlineto{\pgfqpoint{0.644424in}{1.150468in}}%
\pgfpathlineto{\pgfqpoint{0.639535in}{1.157895in}}%
\pgfpathlineto{\pgfqpoint{0.634712in}{1.162931in}}%
\pgfpathlineto{\pgfqpoint{0.633422in}{1.167544in}}%
\pgfpathlineto{\pgfqpoint{0.628933in}{1.177193in}}%
\pgfpathlineto{\pgfqpoint{0.626498in}{1.186842in}}%
\pgfpathlineto{\pgfqpoint{0.625325in}{1.196491in}}%
\pgfpathlineto{\pgfqpoint{0.628264in}{1.206140in}}%
\pgfpathlineto{\pgfqpoint{0.632236in}{1.215789in}}%
\pgfpathlineto{\pgfqpoint{0.634712in}{1.223452in}}%
\pgfpathlineto{\pgfqpoint{0.637111in}{1.225439in}}%
\pgfpathlineto{\pgfqpoint{0.644127in}{1.235088in}}%
\pgfpathlineto{\pgfqpoint{0.644424in}{1.235636in}}%
\pgfpathlineto{\pgfqpoint{0.653836in}{1.244737in}}%
\pgfpathlineto{\pgfqpoint{0.654135in}{1.245141in}}%
\pgfpathlineto{\pgfqpoint{0.663847in}{1.252876in}}%
\pgfpathlineto{\pgfqpoint{0.666907in}{1.254386in}}%
\pgfpathlineto{\pgfqpoint{0.673559in}{1.259154in}}%
\pgfpathlineto{\pgfqpoint{0.683271in}{1.263904in}}%
\pgfpathlineto{\pgfqpoint{0.683666in}{1.264035in}}%
\pgfpathlineto{\pgfqpoint{0.692982in}{1.268293in}}%
\pgfpathlineto{\pgfqpoint{0.702694in}{1.271609in}}%
\pgfpathlineto{\pgfqpoint{0.710765in}{1.273684in}}%
\pgfpathlineto{\pgfqpoint{0.712406in}{1.274242in}}%
\pgfpathlineto{\pgfqpoint{0.722118in}{1.276412in}}%
\pgfpathlineto{\pgfqpoint{0.731830in}{1.277889in}}%
\pgfpathlineto{\pgfqpoint{0.741541in}{1.278757in}}%
\pgfpathlineto{\pgfqpoint{0.751253in}{1.279063in}}%
\pgfpathlineto{\pgfqpoint{0.760965in}{1.278841in}}%
\pgfpathlineto{\pgfqpoint{0.770677in}{1.278114in}}%
\pgfpathlineto{\pgfqpoint{0.780388in}{1.276905in}}%
\pgfpathlineto{\pgfqpoint{0.790100in}{1.275239in}}%
\pgfpathlineto{\pgfqpoint{0.797552in}{1.273684in}}%
\pgfpathlineto{\pgfqpoint{0.799812in}{1.273131in}}%
\pgfpathlineto{\pgfqpoint{0.809524in}{1.270550in}}%
\pgfpathlineto{\pgfqpoint{0.819236in}{1.267546in}}%
\pgfpathlineto{\pgfqpoint{0.828947in}{1.264157in}}%
\pgfpathlineto{\pgfqpoint{0.829309in}{1.264035in}}%
\pgfpathlineto{\pgfqpoint{0.838659in}{1.260133in}}%
\pgfpathlineto{\pgfqpoint{0.848371in}{1.255724in}}%
\pgfpathlineto{\pgfqpoint{0.851411in}{1.254386in}}%
\pgfpathlineto{\pgfqpoint{0.858083in}{1.250559in}}%
\pgfpathlineto{\pgfqpoint{0.867794in}{1.244853in}}%
\pgfpathlineto{\pgfqpoint{0.868012in}{1.244737in}}%
\pgfpathlineto{\pgfqpoint{0.877506in}{1.237674in}}%
\pgfpathlineto{\pgfqpoint{0.881155in}{1.235088in}}%
\pgfpathlineto{\pgfqpoint{0.887218in}{1.228487in}}%
\pgfpathlineto{\pgfqpoint{0.890333in}{1.225439in}}%
\pgfpathlineto{\pgfqpoint{0.895788in}{1.215789in}}%
\pgfpathlineto{\pgfqpoint{0.896930in}{1.210435in}}%
\pgfpathlineto{\pgfqpoint{0.898039in}{1.206140in}}%
\pgfpathlineto{\pgfqpoint{0.897327in}{1.196491in}}%
\pgfpathlineto{\pgfqpoint{0.896930in}{1.195443in}}%
\pgfpathlineto{\pgfqpoint{0.894498in}{1.186842in}}%
\pgfpathlineto{\pgfqpoint{0.889459in}{1.177193in}}%
\pgfpathlineto{\pgfqpoint{0.887218in}{1.174348in}}%
\pgfpathlineto{\pgfqpoint{0.882777in}{1.167544in}}%
\pgfpathlineto{\pgfqpoint{0.877506in}{1.161850in}}%
\pgfpathlineto{\pgfqpoint{0.874311in}{1.157895in}}%
\pgfpathlineto{\pgfqpoint{0.867794in}{1.151933in}}%
\pgfpathlineto{\pgfqpoint{0.864103in}{1.148246in}}%
\pgfpathlineto{\pgfqpoint{0.858083in}{1.143632in}}%
\pgfpathlineto{\pgfqpoint{0.851748in}{1.138596in}}%
\pgfpathlineto{\pgfqpoint{0.848371in}{1.136470in}}%
\pgfpathlineto{\pgfqpoint{0.838659in}{1.130327in}}%
\pgfpathlineto{\pgfqpoint{0.836472in}{1.128947in}}%
\pgfpathlineto{\pgfqpoint{0.828947in}{1.125089in}}%
\pgfpathlineto{\pgfqpoint{0.819236in}{1.120553in}}%
\pgfpathlineto{\pgfqpoint{0.816379in}{1.119298in}}%
\pgfpathlineto{\pgfqpoint{0.809524in}{1.116772in}}%
\pgfpathlineto{\pgfqpoint{0.799812in}{1.113663in}}%
\pgfpathlineto{\pgfqpoint{0.790100in}{1.111187in}}%
\pgfpathlineto{\pgfqpoint{0.782209in}{1.109649in}}%
\pgfpathlineto{\pgfqpoint{0.780388in}{1.109338in}}%
\pgfpathlineto{\pgfqpoint{0.770677in}{1.108073in}}%
\pgfpathlineto{\pgfqpoint{0.760965in}{1.107388in}}%
\pgfpathlineto{\pgfqpoint{0.751253in}{1.107271in}}%
\pgfpathlineto{\pgfqpoint{0.741541in}{1.107713in}}%
\pgfpathlineto{\pgfqpoint{0.731830in}{1.108723in}}%
\pgfusepath{stroke}%
\end{pgfscope}%
\begin{pgfscope}%
\pgfpathrectangle{\pgfqpoint{0.625000in}{0.550000in}}{\pgfqpoint{3.875000in}{3.850000in}} %
\pgfusepath{clip}%
\pgfsetbuttcap%
\pgfsetroundjoin%
\pgfsetlinewidth{0.250937pt}%
\definecolor{currentstroke}{rgb}{0.000000,0.000000,0.000000}%
\pgfsetstrokecolor{currentstroke}%
\pgfsetdash{}{0pt}%
\pgfpathmoveto{\pgfqpoint{0.634712in}{1.366963in}}%
\pgfpathlineto{\pgfqpoint{0.633248in}{1.370175in}}%
\pgfpathlineto{\pgfqpoint{0.631002in}{1.379825in}}%
\pgfpathlineto{\pgfqpoint{0.634712in}{1.386845in}}%
\pgfpathlineto{\pgfqpoint{0.644424in}{1.381266in}}%
\pgfpathlineto{\pgfqpoint{0.646075in}{1.379825in}}%
\pgfpathlineto{\pgfqpoint{0.644424in}{1.375705in}}%
\pgfpathlineto{\pgfqpoint{0.642806in}{1.370175in}}%
\pgfpathlineto{\pgfqpoint{0.634712in}{1.366963in}}%
\pgfusepath{stroke}%
\end{pgfscope}%
\begin{pgfscope}%
\pgfpathrectangle{\pgfqpoint{0.625000in}{0.550000in}}{\pgfqpoint{3.875000in}{3.850000in}} %
\pgfusepath{clip}%
\pgfsetbuttcap%
\pgfsetroundjoin%
\pgfsetlinewidth{0.250937pt}%
\definecolor{currentstroke}{rgb}{0.000000,0.000000,0.000000}%
\pgfsetstrokecolor{currentstroke}%
\pgfsetdash{}{0pt}%
\pgfpathmoveto{\pgfqpoint{0.634712in}{1.619068in}}%
\pgfpathlineto{\pgfqpoint{0.633345in}{1.621053in}}%
\pgfpathlineto{\pgfqpoint{0.634712in}{1.623501in}}%
\pgfpathlineto{\pgfqpoint{0.637521in}{1.621053in}}%
\pgfpathlineto{\pgfqpoint{0.634712in}{1.619068in}}%
\pgfusepath{stroke}%
\end{pgfscope}%
\begin{pgfscope}%
\pgfpathrectangle{\pgfqpoint{0.625000in}{0.550000in}}{\pgfqpoint{3.875000in}{3.850000in}} %
\pgfusepath{clip}%
\pgfsetbuttcap%
\pgfsetroundjoin%
\pgfsetlinewidth{0.250937pt}%
\definecolor{currentstroke}{rgb}{0.000000,0.000000,0.000000}%
\pgfsetstrokecolor{currentstroke}%
\pgfsetdash{}{0pt}%
\pgfpathmoveto{\pgfqpoint{0.634712in}{1.918916in}}%
\pgfpathlineto{\pgfqpoint{0.634123in}{1.920175in}}%
\pgfpathlineto{\pgfqpoint{0.625100in}{1.929825in}}%
\pgfpathlineto{\pgfqpoint{0.634712in}{1.937788in}}%
\pgfpathlineto{\pgfqpoint{0.644424in}{1.932053in}}%
\pgfpathlineto{\pgfqpoint{0.646602in}{1.929825in}}%
\pgfpathlineto{\pgfqpoint{0.644424in}{1.926287in}}%
\pgfpathlineto{\pgfqpoint{0.639365in}{1.920175in}}%
\pgfpathlineto{\pgfqpoint{0.634712in}{1.918916in}}%
\pgfusepath{stroke}%
\end{pgfscope}%
\begin{pgfscope}%
\pgfpathrectangle{\pgfqpoint{0.625000in}{0.550000in}}{\pgfqpoint{3.875000in}{3.850000in}} %
\pgfusepath{clip}%
\pgfsetbuttcap%
\pgfsetroundjoin%
\pgfsetlinewidth{0.250937pt}%
\definecolor{currentstroke}{rgb}{0.000000,0.000000,0.000000}%
\pgfsetstrokecolor{currentstroke}%
\pgfsetdash{}{0pt}%
\pgfpathmoveto{\pgfqpoint{0.634712in}{3.021861in}}%
\pgfpathlineto{\pgfqpoint{0.625100in}{3.029825in}}%
\pgfpathlineto{\pgfqpoint{0.634123in}{3.039474in}}%
\pgfpathlineto{\pgfqpoint{0.634712in}{3.040733in}}%
\pgfpathlineto{\pgfqpoint{0.639365in}{3.039474in}}%
\pgfpathlineto{\pgfqpoint{0.644424in}{3.033362in}}%
\pgfpathlineto{\pgfqpoint{0.646602in}{3.029825in}}%
\pgfpathlineto{\pgfqpoint{0.644424in}{3.027596in}}%
\pgfpathlineto{\pgfqpoint{0.634712in}{3.021861in}}%
\pgfusepath{stroke}%
\end{pgfscope}%
\begin{pgfscope}%
\pgfpathrectangle{\pgfqpoint{0.625000in}{0.550000in}}{\pgfqpoint{3.875000in}{3.850000in}} %
\pgfusepath{clip}%
\pgfsetbuttcap%
\pgfsetroundjoin%
\pgfsetlinewidth{0.250937pt}%
\definecolor{currentstroke}{rgb}{0.000000,0.000000,0.000000}%
\pgfsetstrokecolor{currentstroke}%
\pgfsetdash{}{0pt}%
\pgfpathmoveto{\pgfqpoint{0.634712in}{3.336148in}}%
\pgfpathlineto{\pgfqpoint{0.633345in}{3.338596in}}%
\pgfpathlineto{\pgfqpoint{0.634712in}{3.340581in}}%
\pgfpathlineto{\pgfqpoint{0.637521in}{3.338596in}}%
\pgfpathlineto{\pgfqpoint{0.634712in}{3.336148in}}%
\pgfusepath{stroke}%
\end{pgfscope}%
\begin{pgfscope}%
\pgfpathrectangle{\pgfqpoint{0.625000in}{0.550000in}}{\pgfqpoint{3.875000in}{3.850000in}} %
\pgfusepath{clip}%
\pgfsetbuttcap%
\pgfsetroundjoin%
\pgfsetlinewidth{0.250937pt}%
\definecolor{currentstroke}{rgb}{0.000000,0.000000,0.000000}%
\pgfsetstrokecolor{currentstroke}%
\pgfsetdash{}{0pt}%
\pgfpathmoveto{\pgfqpoint{0.634712in}{3.572804in}}%
\pgfpathlineto{\pgfqpoint{0.631002in}{3.579825in}}%
\pgfpathlineto{\pgfqpoint{0.633248in}{3.589474in}}%
\pgfpathlineto{\pgfqpoint{0.634712in}{3.592686in}}%
\pgfpathlineto{\pgfqpoint{0.642806in}{3.589474in}}%
\pgfpathlineto{\pgfqpoint{0.644424in}{3.583944in}}%
\pgfpathlineto{\pgfqpoint{0.646075in}{3.579825in}}%
\pgfpathlineto{\pgfqpoint{0.644424in}{3.578383in}}%
\pgfpathlineto{\pgfqpoint{0.634712in}{3.572804in}}%
\pgfusepath{stroke}%
\end{pgfscope}%
\begin{pgfscope}%
\pgfpathrectangle{\pgfqpoint{0.625000in}{0.550000in}}{\pgfqpoint{3.875000in}{3.850000in}} %
\pgfusepath{clip}%
\pgfsetbuttcap%
\pgfsetroundjoin%
\pgfsetlinewidth{0.250937pt}%
\definecolor{currentstroke}{rgb}{0.000000,0.000000,0.000000}%
\pgfsetstrokecolor{currentstroke}%
\pgfsetdash{}{0pt}%
\pgfpathmoveto{\pgfqpoint{0.712406in}{3.685407in}}%
\pgfpathlineto{\pgfqpoint{0.710765in}{3.685965in}}%
\pgfpathlineto{\pgfqpoint{0.702694in}{3.688040in}}%
\pgfpathlineto{\pgfqpoint{0.692982in}{3.691357in}}%
\pgfpathlineto{\pgfqpoint{0.683666in}{3.695614in}}%
\pgfpathlineto{\pgfqpoint{0.683271in}{3.695745in}}%
\pgfpathlineto{\pgfqpoint{0.673559in}{3.700495in}}%
\pgfpathlineto{\pgfqpoint{0.666907in}{3.705263in}}%
\pgfpathlineto{\pgfqpoint{0.663847in}{3.706773in}}%
\pgfpathlineto{\pgfqpoint{0.654135in}{3.714509in}}%
\pgfpathlineto{\pgfqpoint{0.653836in}{3.714912in}}%
\pgfpathlineto{\pgfqpoint{0.644424in}{3.724013in}}%
\pgfpathlineto{\pgfqpoint{0.644127in}{3.724561in}}%
\pgfpathlineto{\pgfqpoint{0.637111in}{3.734211in}}%
\pgfpathlineto{\pgfqpoint{0.634712in}{3.736197in}}%
\pgfpathlineto{\pgfqpoint{0.632236in}{3.743860in}}%
\pgfpathlineto{\pgfqpoint{0.628264in}{3.753509in}}%
\pgfpathlineto{\pgfqpoint{0.625325in}{3.763158in}}%
\pgfpathlineto{\pgfqpoint{0.626498in}{3.772807in}}%
\pgfpathlineto{\pgfqpoint{0.628933in}{3.782456in}}%
\pgfpathlineto{\pgfqpoint{0.633422in}{3.792105in}}%
\pgfpathlineto{\pgfqpoint{0.634712in}{3.796718in}}%
\pgfpathlineto{\pgfqpoint{0.639535in}{3.801754in}}%
\pgfpathlineto{\pgfqpoint{0.644424in}{3.809182in}}%
\pgfpathlineto{\pgfqpoint{0.647562in}{3.811404in}}%
\pgfpathlineto{\pgfqpoint{0.654135in}{3.818600in}}%
\pgfpathlineto{\pgfqpoint{0.658181in}{3.821053in}}%
\pgfpathlineto{\pgfqpoint{0.663847in}{3.826156in}}%
\pgfpathlineto{\pgfqpoint{0.671867in}{3.830702in}}%
\pgfpathlineto{\pgfqpoint{0.673559in}{3.832066in}}%
\pgfpathlineto{\pgfqpoint{0.683271in}{3.837249in}}%
\pgfpathlineto{\pgfqpoint{0.691288in}{3.840351in}}%
\pgfpathlineto{\pgfqpoint{0.692982in}{3.841247in}}%
\pgfpathlineto{\pgfqpoint{0.702694in}{3.844749in}}%
\pgfpathlineto{\pgfqpoint{0.712406in}{3.847386in}}%
\pgfpathlineto{\pgfqpoint{0.722118in}{3.849384in}}%
\pgfpathlineto{\pgfqpoint{0.726575in}{3.850000in}}%
\pgfpathlineto{\pgfqpoint{0.731830in}{3.850926in}}%
\pgfpathlineto{\pgfqpoint{0.741541in}{3.851936in}}%
\pgfpathlineto{\pgfqpoint{0.751253in}{3.852378in}}%
\pgfpathlineto{\pgfqpoint{0.760965in}{3.852261in}}%
\pgfpathlineto{\pgfqpoint{0.770677in}{3.851576in}}%
\pgfpathlineto{\pgfqpoint{0.780388in}{3.850311in}}%
\pgfpathlineto{\pgfqpoint{0.782209in}{3.850000in}}%
\pgfpathlineto{\pgfqpoint{0.790100in}{3.848463in}}%
\pgfpathlineto{\pgfqpoint{0.799812in}{3.845986in}}%
\pgfpathlineto{\pgfqpoint{0.809524in}{3.842877in}}%
\pgfpathlineto{\pgfqpoint{0.816379in}{3.840351in}}%
\pgfpathlineto{\pgfqpoint{0.819236in}{3.839096in}}%
\pgfpathlineto{\pgfqpoint{0.828947in}{3.834560in}}%
\pgfpathlineto{\pgfqpoint{0.836472in}{3.830702in}}%
\pgfpathlineto{\pgfqpoint{0.838659in}{3.829322in}}%
\pgfpathlineto{\pgfqpoint{0.848371in}{3.823179in}}%
\pgfpathlineto{\pgfqpoint{0.851748in}{3.821053in}}%
\pgfpathlineto{\pgfqpoint{0.858083in}{3.816017in}}%
\pgfpathlineto{\pgfqpoint{0.864103in}{3.811404in}}%
\pgfpathlineto{\pgfqpoint{0.867794in}{3.807716in}}%
\pgfpathlineto{\pgfqpoint{0.874311in}{3.801754in}}%
\pgfpathlineto{\pgfqpoint{0.877506in}{3.797799in}}%
\pgfpathlineto{\pgfqpoint{0.882777in}{3.792105in}}%
\pgfpathlineto{\pgfqpoint{0.887218in}{3.785301in}}%
\pgfpathlineto{\pgfqpoint{0.889459in}{3.782456in}}%
\pgfpathlineto{\pgfqpoint{0.894498in}{3.772807in}}%
\pgfpathlineto{\pgfqpoint{0.896930in}{3.764206in}}%
\pgfpathlineto{\pgfqpoint{0.897327in}{3.763158in}}%
\pgfpathlineto{\pgfqpoint{0.898039in}{3.753509in}}%
\pgfpathlineto{\pgfqpoint{0.896930in}{3.749214in}}%
\pgfpathlineto{\pgfqpoint{0.895788in}{3.743860in}}%
\pgfpathlineto{\pgfqpoint{0.890333in}{3.734211in}}%
\pgfpathlineto{\pgfqpoint{0.887218in}{3.731162in}}%
\pgfpathlineto{\pgfqpoint{0.881155in}{3.724561in}}%
\pgfpathlineto{\pgfqpoint{0.877506in}{3.721975in}}%
\pgfpathlineto{\pgfqpoint{0.868012in}{3.714912in}}%
\pgfpathlineto{\pgfqpoint{0.867794in}{3.714796in}}%
\pgfpathlineto{\pgfqpoint{0.858083in}{3.709091in}}%
\pgfpathlineto{\pgfqpoint{0.851411in}{3.705263in}}%
\pgfpathlineto{\pgfqpoint{0.848371in}{3.703925in}}%
\pgfpathlineto{\pgfqpoint{0.838659in}{3.699516in}}%
\pgfpathlineto{\pgfqpoint{0.829309in}{3.695614in}}%
\pgfpathlineto{\pgfqpoint{0.828947in}{3.695492in}}%
\pgfpathlineto{\pgfqpoint{0.819236in}{3.692103in}}%
\pgfpathlineto{\pgfqpoint{0.809524in}{3.689099in}}%
\pgfpathlineto{\pgfqpoint{0.799812in}{3.686518in}}%
\pgfpathlineto{\pgfqpoint{0.797552in}{3.685965in}}%
\pgfpathlineto{\pgfqpoint{0.790100in}{3.684410in}}%
\pgfpathlineto{\pgfqpoint{0.780388in}{3.682744in}}%
\pgfpathlineto{\pgfqpoint{0.770677in}{3.681535in}}%
\pgfpathlineto{\pgfqpoint{0.760965in}{3.680808in}}%
\pgfpathlineto{\pgfqpoint{0.751253in}{3.680586in}}%
\pgfpathlineto{\pgfqpoint{0.741541in}{3.680892in}}%
\pgfpathlineto{\pgfqpoint{0.731830in}{3.681760in}}%
\pgfpathlineto{\pgfqpoint{0.722118in}{3.683237in}}%
\pgfpathlineto{\pgfqpoint{0.712406in}{3.685407in}}%
\pgfusepath{stroke}%
\end{pgfscope}%
\begin{pgfscope}%
\pgfpathrectangle{\pgfqpoint{0.625000in}{0.550000in}}{\pgfqpoint{3.875000in}{3.850000in}} %
\pgfusepath{clip}%
\pgfsetbuttcap%
\pgfsetroundjoin%
\pgfsetlinewidth{0.250937pt}%
\definecolor{currentstroke}{rgb}{0.000000,0.000000,0.000000}%
\pgfsetstrokecolor{currentstroke}%
\pgfsetdash{}{0pt}%
\pgfpathmoveto{\pgfqpoint{0.634712in}{4.123831in}}%
\pgfpathlineto{\pgfqpoint{0.632188in}{4.129825in}}%
\pgfpathlineto{\pgfqpoint{0.632543in}{4.139474in}}%
\pgfpathlineto{\pgfqpoint{0.634712in}{4.144235in}}%
\pgfpathlineto{\pgfqpoint{0.644100in}{4.139474in}}%
\pgfpathlineto{\pgfqpoint{0.644424in}{4.134611in}}%
\pgfpathlineto{\pgfqpoint{0.645128in}{4.129825in}}%
\pgfpathlineto{\pgfqpoint{0.644424in}{4.129292in}}%
\pgfpathlineto{\pgfqpoint{0.634712in}{4.123831in}}%
\pgfusepath{stroke}%
\end{pgfscope}%
\begin{pgfscope}%
\pgfpathrectangle{\pgfqpoint{0.625000in}{0.550000in}}{\pgfqpoint{3.875000in}{3.850000in}} %
\pgfusepath{clip}%
\pgfsetbuttcap%
\pgfsetroundjoin%
\pgfsetlinewidth{0.250937pt}%
\definecolor{currentstroke}{rgb}{0.000000,0.000000,0.000000}%
\pgfsetstrokecolor{currentstroke}%
\pgfsetdash{}{0pt}%
\pgfpathmoveto{\pgfqpoint{0.634712in}{4.195264in}}%
\pgfpathlineto{\pgfqpoint{0.634123in}{4.197368in}}%
\pgfpathlineto{\pgfqpoint{0.634712in}{4.198681in}}%
\pgfpathlineto{\pgfqpoint{0.636755in}{4.197368in}}%
\pgfpathlineto{\pgfqpoint{0.634712in}{4.195264in}}%
\pgfusepath{stroke}%
\end{pgfscope}%
\begin{pgfscope}%
\pgfpathrectangle{\pgfqpoint{0.625000in}{0.550000in}}{\pgfqpoint{3.875000in}{3.850000in}} %
\pgfusepath{clip}%
\pgfsetbuttcap%
\pgfsetroundjoin%
\pgfsetlinewidth{0.250937pt}%
\definecolor{currentstroke}{rgb}{0.000000,0.000000,0.000000}%
\pgfsetstrokecolor{currentstroke}%
\pgfsetdash{}{0pt}%
\pgfpathmoveto{\pgfqpoint{1.058425in}{0.550000in}}%
\pgfpathlineto{\pgfqpoint{1.057446in}{0.578947in}}%
\pgfpathlineto{\pgfqpoint{1.054481in}{0.607895in}}%
\pgfpathlineto{\pgfqpoint{1.049512in}{0.636842in}}%
\pgfpathlineto{\pgfqpoint{1.042435in}{0.665789in}}%
\pgfpathlineto{\pgfqpoint{1.032895in}{0.695516in}}%
\pgfpathlineto{\pgfqpoint{1.021594in}{0.723684in}}%
\pgfpathlineto{\pgfqpoint{1.007436in}{0.752632in}}%
\pgfpathlineto{\pgfqpoint{0.990402in}{0.781579in}}%
\pgfpathlineto{\pgfqpoint{0.970065in}{0.810526in}}%
\pgfpathlineto{\pgfqpoint{0.954364in}{0.829825in}}%
\pgfpathlineto{\pgfqpoint{0.935777in}{0.850035in}}%
\pgfpathlineto{\pgfqpoint{0.916353in}{0.868687in}}%
\pgfpathlineto{\pgfqpoint{0.887218in}{0.892946in}}%
\pgfpathlineto{\pgfqpoint{0.858083in}{0.913857in}}%
\pgfpathlineto{\pgfqpoint{0.790100in}{0.959122in}}%
\pgfpathlineto{\pgfqpoint{0.780388in}{0.968512in}}%
\pgfpathlineto{\pgfqpoint{0.770677in}{0.982934in}}%
\pgfpathlineto{\pgfqpoint{0.760965in}{1.005176in}}%
\pgfpathlineto{\pgfqpoint{0.753338in}{1.022807in}}%
\pgfpathlineto{\pgfqpoint{0.735256in}{1.051754in}}%
\pgfpathlineto{\pgfqpoint{0.707607in}{1.080702in}}%
\pgfpathlineto{\pgfqpoint{0.692982in}{1.093066in}}%
\pgfpathlineto{\pgfqpoint{0.663847in}{1.121089in}}%
\pgfpathlineto{\pgfqpoint{0.642486in}{1.148246in}}%
\pgfpathlineto{\pgfqpoint{0.634712in}{1.159912in}}%
\pgfpathlineto{\pgfqpoint{0.632579in}{1.167544in}}%
\pgfpathlineto{\pgfqpoint{0.628390in}{1.177193in}}%
\pgfpathlineto{\pgfqpoint{0.625150in}{1.196491in}}%
\pgfpathlineto{\pgfqpoint{0.631534in}{1.215789in}}%
\pgfpathlineto{\pgfqpoint{0.634712in}{1.225917in}}%
\pgfpathlineto{\pgfqpoint{0.654135in}{1.254262in}}%
\pgfpathlineto{\pgfqpoint{0.654291in}{1.254386in}}%
\pgfpathlineto{\pgfqpoint{0.673559in}{1.275375in}}%
\pgfpathlineto{\pgfqpoint{0.712858in}{1.312281in}}%
\pgfpathlineto{\pgfqpoint{0.741541in}{1.338606in}}%
\pgfpathlineto{\pgfqpoint{0.768337in}{1.360526in}}%
\pgfpathlineto{\pgfqpoint{0.790100in}{1.380683in}}%
\pgfpathlineto{\pgfqpoint{0.806046in}{1.399123in}}%
\pgfpathlineto{\pgfqpoint{0.819236in}{1.419417in}}%
\pgfpathlineto{\pgfqpoint{0.828947in}{1.439975in}}%
\pgfpathlineto{\pgfqpoint{0.834775in}{1.457018in}}%
\pgfpathlineto{\pgfqpoint{0.839041in}{1.476316in}}%
\pgfpathlineto{\pgfqpoint{0.841340in}{1.495614in}}%
\pgfpathlineto{\pgfqpoint{0.841746in}{1.514912in}}%
\pgfpathlineto{\pgfqpoint{0.840400in}{1.534211in}}%
\pgfpathlineto{\pgfqpoint{0.837404in}{1.553509in}}%
\pgfpathlineto{\pgfqpoint{0.832818in}{1.572807in}}%
\pgfpathlineto{\pgfqpoint{0.822887in}{1.601754in}}%
\pgfpathlineto{\pgfqpoint{0.809524in}{1.630213in}}%
\pgfpathlineto{\pgfqpoint{0.798063in}{1.650000in}}%
\pgfpathlineto{\pgfqpoint{0.777997in}{1.678947in}}%
\pgfpathlineto{\pgfqpoint{0.757101in}{1.707895in}}%
\pgfpathlineto{\pgfqpoint{0.754225in}{1.717544in}}%
\pgfpathlineto{\pgfqpoint{0.754878in}{1.727193in}}%
\pgfpathlineto{\pgfqpoint{0.766558in}{1.775439in}}%
\pgfpathlineto{\pgfqpoint{0.768653in}{1.794737in}}%
\pgfpathlineto{\pgfqpoint{0.768946in}{1.814035in}}%
\pgfpathlineto{\pgfqpoint{0.767522in}{1.833333in}}%
\pgfpathlineto{\pgfqpoint{0.764422in}{1.852632in}}%
\pgfpathlineto{\pgfqpoint{0.756966in}{1.881579in}}%
\pgfpathlineto{\pgfqpoint{0.735669in}{1.949123in}}%
\pgfpathlineto{\pgfqpoint{0.719176in}{2.026316in}}%
\pgfpathlineto{\pgfqpoint{0.705584in}{2.074561in}}%
\pgfpathlineto{\pgfqpoint{0.690596in}{2.132456in}}%
\pgfpathlineto{\pgfqpoint{0.663705in}{2.238596in}}%
\pgfpathlineto{\pgfqpoint{0.646314in}{2.315789in}}%
\pgfpathlineto{\pgfqpoint{0.637173in}{2.364035in}}%
\pgfpathlineto{\pgfqpoint{0.632956in}{2.392982in}}%
\pgfpathlineto{\pgfqpoint{0.632005in}{2.402632in}}%
\pgfpathlineto{\pgfqpoint{0.628301in}{2.412281in}}%
\pgfpathlineto{\pgfqpoint{0.628670in}{2.421930in}}%
\pgfpathlineto{\pgfqpoint{0.625000in}{2.467031in}}%
\pgfpathlineto{\pgfqpoint{0.625000in}{2.467031in}}%
\pgfusepath{stroke}%
\end{pgfscope}%
\begin{pgfscope}%
\pgfpathrectangle{\pgfqpoint{0.625000in}{0.550000in}}{\pgfqpoint{3.875000in}{3.850000in}} %
\pgfusepath{clip}%
\pgfsetbuttcap%
\pgfsetroundjoin%
\pgfsetlinewidth{0.250937pt}%
\definecolor{currentstroke}{rgb}{0.000000,0.000000,0.000000}%
\pgfsetstrokecolor{currentstroke}%
\pgfsetdash{}{0pt}%
\pgfpathmoveto{\pgfqpoint{0.625000in}{0.636349in}}%
\pgfpathlineto{\pgfqpoint{0.634443in}{0.627193in}}%
\pgfpathlineto{\pgfqpoint{0.625000in}{0.618064in}}%
\pgfusepath{stroke}%
\end{pgfscope}%
\begin{pgfscope}%
\pgfpathrectangle{\pgfqpoint{0.625000in}{0.550000in}}{\pgfqpoint{3.875000in}{3.850000in}} %
\pgfusepath{clip}%
\pgfsetbuttcap%
\pgfsetroundjoin%
\pgfsetlinewidth{0.250937pt}%
\definecolor{currentstroke}{rgb}{0.000000,0.000000,0.000000}%
\pgfsetstrokecolor{currentstroke}%
\pgfsetdash{}{0pt}%
\pgfpathmoveto{\pgfqpoint{0.625000in}{0.790346in}}%
\pgfpathlineto{\pgfqpoint{0.634484in}{0.781579in}}%
\pgfpathlineto{\pgfqpoint{0.625000in}{0.772422in}}%
\pgfusepath{stroke}%
\end{pgfscope}%
\begin{pgfscope}%
\pgfpathrectangle{\pgfqpoint{0.625000in}{0.550000in}}{\pgfqpoint{3.875000in}{3.850000in}} %
\pgfusepath{clip}%
\pgfsetbuttcap%
\pgfsetroundjoin%
\pgfsetlinewidth{0.250937pt}%
\definecolor{currentstroke}{rgb}{0.000000,0.000000,0.000000}%
\pgfsetstrokecolor{currentstroke}%
\pgfsetdash{}{0pt}%
\pgfpathmoveto{\pgfqpoint{0.625000in}{0.945012in}}%
\pgfpathlineto{\pgfqpoint{0.630467in}{0.945614in}}%
\pgfpathlineto{\pgfqpoint{0.634712in}{0.954514in}}%
\pgfpathlineto{\pgfqpoint{0.637604in}{0.955263in}}%
\pgfpathlineto{\pgfqpoint{0.644424in}{0.961315in}}%
\pgfpathlineto{\pgfqpoint{0.654135in}{0.964384in}}%
\pgfpathlineto{\pgfqpoint{0.658419in}{0.964912in}}%
\pgfpathlineto{\pgfqpoint{0.663847in}{0.966267in}}%
\pgfpathlineto{\pgfqpoint{0.673559in}{0.966411in}}%
\pgfpathlineto{\pgfqpoint{0.683271in}{0.965155in}}%
\pgfpathlineto{\pgfqpoint{0.684598in}{0.964912in}}%
\pgfpathlineto{\pgfqpoint{0.692982in}{0.962446in}}%
\pgfpathlineto{\pgfqpoint{0.702694in}{0.958022in}}%
\pgfpathlineto{\pgfqpoint{0.707854in}{0.955263in}}%
\pgfpathlineto{\pgfqpoint{0.712406in}{0.949297in}}%
\pgfpathlineto{\pgfqpoint{0.715496in}{0.945614in}}%
\pgfpathlineto{\pgfqpoint{0.716715in}{0.935965in}}%
\pgfpathlineto{\pgfqpoint{0.712688in}{0.926316in}}%
\pgfpathlineto{\pgfqpoint{0.712406in}{0.926033in}}%
\pgfpathlineto{\pgfqpoint{0.704533in}{0.916667in}}%
\pgfpathlineto{\pgfqpoint{0.702694in}{0.915541in}}%
\pgfpathlineto{\pgfqpoint{0.692982in}{0.909941in}}%
\pgfpathlineto{\pgfqpoint{0.684659in}{0.907018in}}%
\pgfpathlineto{\pgfqpoint{0.683271in}{0.906701in}}%
\pgfpathlineto{\pgfqpoint{0.673559in}{0.905379in}}%
\pgfpathlineto{\pgfqpoint{0.663847in}{0.905655in}}%
\pgfpathlineto{\pgfqpoint{0.658453in}{0.907018in}}%
\pgfpathlineto{\pgfqpoint{0.654135in}{0.907556in}}%
\pgfpathlineto{\pgfqpoint{0.644424in}{0.910606in}}%
\pgfpathlineto{\pgfqpoint{0.637606in}{0.916667in}}%
\pgfpathlineto{\pgfqpoint{0.634712in}{0.917416in}}%
\pgfpathlineto{\pgfqpoint{0.630095in}{0.926316in}}%
\pgfpathlineto{\pgfqpoint{0.625000in}{0.926836in}}%
\pgfusepath{stroke}%
\end{pgfscope}%
\begin{pgfscope}%
\pgfpathrectangle{\pgfqpoint{0.625000in}{0.550000in}}{\pgfqpoint{3.875000in}{3.850000in}} %
\pgfusepath{clip}%
\pgfsetbuttcap%
\pgfsetroundjoin%
\pgfsetlinewidth{0.250937pt}%
\definecolor{currentstroke}{rgb}{0.000000,0.000000,0.000000}%
\pgfsetstrokecolor{currentstroke}%
\pgfsetdash{}{0pt}%
\pgfpathmoveto{\pgfqpoint{0.625000in}{1.099438in}}%
\pgfpathlineto{\pgfqpoint{0.634475in}{1.090351in}}%
\pgfpathlineto{\pgfqpoint{0.625000in}{1.081298in}}%
\pgfusepath{stroke}%
\end{pgfscope}%
\begin{pgfscope}%
\pgfpathrectangle{\pgfqpoint{0.625000in}{0.550000in}}{\pgfqpoint{3.875000in}{3.850000in}} %
\pgfusepath{clip}%
\pgfsetbuttcap%
\pgfsetroundjoin%
\pgfsetlinewidth{0.250937pt}%
\definecolor{currentstroke}{rgb}{0.000000,0.000000,0.000000}%
\pgfsetstrokecolor{currentstroke}%
\pgfsetdash{}{0pt}%
\pgfpathmoveto{\pgfqpoint{0.625000in}{1.254123in}}%
\pgfpathlineto{\pgfqpoint{0.634511in}{1.244737in}}%
\pgfpathlineto{\pgfqpoint{0.625000in}{1.235609in}}%
\pgfusepath{stroke}%
\end{pgfscope}%
\begin{pgfscope}%
\pgfpathrectangle{\pgfqpoint{0.625000in}{0.550000in}}{\pgfqpoint{3.875000in}{3.850000in}} %
\pgfusepath{clip}%
\pgfsetbuttcap%
\pgfsetroundjoin%
\pgfsetlinewidth{0.250937pt}%
\definecolor{currentstroke}{rgb}{0.000000,0.000000,0.000000}%
\pgfsetstrokecolor{currentstroke}%
\pgfsetdash{}{0pt}%
\pgfpathmoveto{\pgfqpoint{0.625000in}{1.294426in}}%
\pgfpathlineto{\pgfqpoint{0.629036in}{1.292982in}}%
\pgfpathlineto{\pgfqpoint{0.625000in}{1.291769in}}%
\pgfusepath{stroke}%
\end{pgfscope}%
\begin{pgfscope}%
\pgfpathrectangle{\pgfqpoint{0.625000in}{0.550000in}}{\pgfqpoint{3.875000in}{3.850000in}} %
\pgfusepath{clip}%
\pgfsetbuttcap%
\pgfsetroundjoin%
\pgfsetlinewidth{0.250937pt}%
\definecolor{currentstroke}{rgb}{0.000000,0.000000,0.000000}%
\pgfsetstrokecolor{currentstroke}%
\pgfsetdash{}{0pt}%
\pgfpathmoveto{\pgfqpoint{0.625000in}{1.408149in}}%
\pgfpathlineto{\pgfqpoint{0.634489in}{1.399123in}}%
\pgfpathlineto{\pgfqpoint{0.625000in}{1.390091in}}%
\pgfusepath{stroke}%
\end{pgfscope}%
\begin{pgfscope}%
\pgfpathrectangle{\pgfqpoint{0.625000in}{0.550000in}}{\pgfqpoint{3.875000in}{3.850000in}} %
\pgfusepath{clip}%
\pgfsetbuttcap%
\pgfsetroundjoin%
\pgfsetlinewidth{0.250937pt}%
\definecolor{currentstroke}{rgb}{0.000000,0.000000,0.000000}%
\pgfsetstrokecolor{currentstroke}%
\pgfsetdash{}{0pt}%
\pgfpathmoveto{\pgfqpoint{0.625000in}{1.562492in}}%
\pgfpathlineto{\pgfqpoint{0.634428in}{1.553509in}}%
\pgfpathlineto{\pgfqpoint{0.625000in}{1.544234in}}%
\pgfusepath{stroke}%
\end{pgfscope}%
\begin{pgfscope}%
\pgfpathrectangle{\pgfqpoint{0.625000in}{0.550000in}}{\pgfqpoint{3.875000in}{3.850000in}} %
\pgfusepath{clip}%
\pgfsetbuttcap%
\pgfsetroundjoin%
\pgfsetlinewidth{0.250937pt}%
\definecolor{currentstroke}{rgb}{0.000000,0.000000,0.000000}%
\pgfsetstrokecolor{currentstroke}%
\pgfsetdash{}{0pt}%
\pgfpathmoveto{\pgfqpoint{0.625000in}{1.716887in}}%
\pgfpathlineto{\pgfqpoint{0.630541in}{1.717544in}}%
\pgfpathlineto{\pgfqpoint{0.634712in}{1.726444in}}%
\pgfpathlineto{\pgfqpoint{0.637604in}{1.727193in}}%
\pgfpathlineto{\pgfqpoint{0.644424in}{1.733250in}}%
\pgfpathlineto{\pgfqpoint{0.654135in}{1.736309in}}%
\pgfpathlineto{\pgfqpoint{0.658406in}{1.736842in}}%
\pgfpathlineto{\pgfqpoint{0.663847in}{1.738235in}}%
\pgfpathlineto{\pgfqpoint{0.673559in}{1.738359in}}%
\pgfpathlineto{\pgfqpoint{0.683271in}{1.736930in}}%
\pgfpathlineto{\pgfqpoint{0.683711in}{1.736842in}}%
\pgfpathlineto{\pgfqpoint{0.692982in}{1.733952in}}%
\pgfpathlineto{\pgfqpoint{0.702694in}{1.729530in}}%
\pgfpathlineto{\pgfqpoint{0.707791in}{1.727193in}}%
\pgfpathlineto{\pgfqpoint{0.712406in}{1.722447in}}%
\pgfpathlineto{\pgfqpoint{0.718376in}{1.717544in}}%
\pgfpathlineto{\pgfqpoint{0.720378in}{1.707895in}}%
\pgfpathlineto{\pgfqpoint{0.715938in}{1.698246in}}%
\pgfpathlineto{\pgfqpoint{0.712406in}{1.695536in}}%
\pgfpathlineto{\pgfqpoint{0.705461in}{1.688596in}}%
\pgfpathlineto{\pgfqpoint{0.702694in}{1.687195in}}%
\pgfpathlineto{\pgfqpoint{0.692982in}{1.682110in}}%
\pgfpathlineto{\pgfqpoint{0.683834in}{1.678947in}}%
\pgfpathlineto{\pgfqpoint{0.683271in}{1.678819in}}%
\pgfpathlineto{\pgfqpoint{0.673559in}{1.677344in}}%
\pgfpathlineto{\pgfqpoint{0.663847in}{1.677550in}}%
\pgfpathlineto{\pgfqpoint{0.658427in}{1.678947in}}%
\pgfpathlineto{\pgfqpoint{0.654135in}{1.679488in}}%
\pgfpathlineto{\pgfqpoint{0.644424in}{1.682534in}}%
\pgfpathlineto{\pgfqpoint{0.637605in}{1.688596in}}%
\pgfpathlineto{\pgfqpoint{0.634712in}{1.689346in}}%
\pgfpathlineto{\pgfqpoint{0.630309in}{1.698246in}}%
\pgfpathlineto{\pgfqpoint{0.625000in}{1.698845in}}%
\pgfusepath{stroke}%
\end{pgfscope}%
\begin{pgfscope}%
\pgfpathrectangle{\pgfqpoint{0.625000in}{0.550000in}}{\pgfqpoint{3.875000in}{3.850000in}} %
\pgfusepath{clip}%
\pgfsetbuttcap%
\pgfsetroundjoin%
\pgfsetlinewidth{0.250937pt}%
\definecolor{currentstroke}{rgb}{0.000000,0.000000,0.000000}%
\pgfsetstrokecolor{currentstroke}%
\pgfsetdash{}{0pt}%
\pgfpathmoveto{\pgfqpoint{0.625000in}{1.871364in}}%
\pgfpathlineto{\pgfqpoint{0.634427in}{1.862281in}}%
\pgfpathlineto{\pgfqpoint{0.625000in}{1.853385in}}%
\pgfusepath{stroke}%
\end{pgfscope}%
\begin{pgfscope}%
\pgfpathrectangle{\pgfqpoint{0.625000in}{0.550000in}}{\pgfqpoint{3.875000in}{3.850000in}} %
\pgfusepath{clip}%
\pgfsetbuttcap%
\pgfsetroundjoin%
\pgfsetlinewidth{0.250937pt}%
\definecolor{currentstroke}{rgb}{0.000000,0.000000,0.000000}%
\pgfsetstrokecolor{currentstroke}%
\pgfsetdash{}{0pt}%
\pgfpathmoveto{\pgfqpoint{0.625000in}{1.892425in}}%
\pgfpathlineto{\pgfqpoint{0.629096in}{1.891228in}}%
\pgfpathlineto{\pgfqpoint{0.625000in}{1.889319in}}%
\pgfusepath{stroke}%
\end{pgfscope}%
\begin{pgfscope}%
\pgfpathrectangle{\pgfqpoint{0.625000in}{0.550000in}}{\pgfqpoint{3.875000in}{3.850000in}} %
\pgfusepath{clip}%
\pgfsetbuttcap%
\pgfsetroundjoin%
\pgfsetlinewidth{0.250937pt}%
\definecolor{currentstroke}{rgb}{0.000000,0.000000,0.000000}%
\pgfsetstrokecolor{currentstroke}%
\pgfsetdash{}{0pt}%
\pgfpathmoveto{\pgfqpoint{0.625000in}{1.930745in}}%
\pgfpathlineto{\pgfqpoint{0.634712in}{1.939347in}}%
\pgfpathlineto{\pgfqpoint{0.642159in}{1.939474in}}%
\pgfpathlineto{\pgfqpoint{0.644424in}{1.939650in}}%
\pgfpathlineto{\pgfqpoint{0.644972in}{1.939474in}}%
\pgfpathlineto{\pgfqpoint{0.654042in}{1.929825in}}%
\pgfpathlineto{\pgfqpoint{0.650861in}{1.920175in}}%
\pgfpathlineto{\pgfqpoint{0.644424in}{1.917033in}}%
\pgfpathlineto{\pgfqpoint{0.634712in}{1.916210in}}%
\pgfpathlineto{\pgfqpoint{0.632858in}{1.920175in}}%
\pgfpathlineto{\pgfqpoint{0.625000in}{1.928548in}}%
\pgfusepath{stroke}%
\end{pgfscope}%
\begin{pgfscope}%
\pgfpathrectangle{\pgfqpoint{0.625000in}{0.550000in}}{\pgfqpoint{3.875000in}{3.850000in}} %
\pgfusepath{clip}%
\pgfsetbuttcap%
\pgfsetroundjoin%
\pgfsetlinewidth{0.250937pt}%
\definecolor{currentstroke}{rgb}{0.000000,0.000000,0.000000}%
\pgfsetstrokecolor{currentstroke}%
\pgfsetdash{}{0pt}%
\pgfpathmoveto{\pgfqpoint{0.625000in}{2.026088in}}%
\pgfpathlineto{\pgfqpoint{0.634431in}{2.016667in}}%
\pgfpathlineto{\pgfqpoint{0.625000in}{2.007322in}}%
\pgfusepath{stroke}%
\end{pgfscope}%
\begin{pgfscope}%
\pgfpathrectangle{\pgfqpoint{0.625000in}{0.550000in}}{\pgfqpoint{3.875000in}{3.850000in}} %
\pgfusepath{clip}%
\pgfsetbuttcap%
\pgfsetroundjoin%
\pgfsetlinewidth{0.250937pt}%
\definecolor{currentstroke}{rgb}{0.000000,0.000000,0.000000}%
\pgfsetstrokecolor{currentstroke}%
\pgfsetdash{}{0pt}%
\pgfpathmoveto{\pgfqpoint{0.625000in}{2.179990in}}%
\pgfpathlineto{\pgfqpoint{0.634452in}{2.171053in}}%
\pgfpathlineto{\pgfqpoint{0.625000in}{2.161859in}}%
\pgfusepath{stroke}%
\end{pgfscope}%
\begin{pgfscope}%
\pgfpathrectangle{\pgfqpoint{0.625000in}{0.550000in}}{\pgfqpoint{3.875000in}{3.850000in}} %
\pgfusepath{clip}%
\pgfsetbuttcap%
\pgfsetroundjoin%
\pgfsetlinewidth{0.250937pt}%
\definecolor{currentstroke}{rgb}{0.000000,0.000000,0.000000}%
\pgfsetstrokecolor{currentstroke}%
\pgfsetdash{}{0pt}%
\pgfpathmoveto{\pgfqpoint{0.625000in}{2.334341in}}%
\pgfpathlineto{\pgfqpoint{0.634504in}{2.325439in}}%
\pgfpathlineto{\pgfqpoint{0.625000in}{2.315912in}}%
\pgfusepath{stroke}%
\end{pgfscope}%
\begin{pgfscope}%
\pgfpathrectangle{\pgfqpoint{0.625000in}{0.550000in}}{\pgfqpoint{3.875000in}{3.850000in}} %
\pgfusepath{clip}%
\pgfsetbuttcap%
\pgfsetroundjoin%
\pgfsetlinewidth{0.250937pt}%
\definecolor{currentstroke}{rgb}{0.000000,0.000000,0.000000}%
\pgfsetstrokecolor{currentstroke}%
\pgfsetdash{}{0pt}%
\pgfpathmoveto{\pgfqpoint{0.625000in}{2.492618in}}%
\pgfpathlineto{\pgfqpoint{0.628301in}{2.547368in}}%
\pgfpathlineto{\pgfqpoint{0.632005in}{2.557018in}}%
\pgfpathlineto{\pgfqpoint{0.635530in}{2.585965in}}%
\pgfpathlineto{\pgfqpoint{0.646314in}{2.643860in}}%
\pgfpathlineto{\pgfqpoint{0.663847in}{2.721707in}}%
\pgfpathlineto{\pgfqpoint{0.690596in}{2.827193in}}%
\pgfpathlineto{\pgfqpoint{0.705584in}{2.885088in}}%
\pgfpathlineto{\pgfqpoint{0.716327in}{2.923684in}}%
\pgfpathlineto{\pgfqpoint{0.724296in}{2.952632in}}%
\pgfpathlineto{\pgfqpoint{0.730197in}{2.981579in}}%
\pgfpathlineto{\pgfqpoint{0.738018in}{3.020175in}}%
\pgfpathlineto{\pgfqpoint{0.747201in}{3.049123in}}%
\pgfpathlineto{\pgfqpoint{0.759767in}{3.087719in}}%
\pgfpathlineto{\pgfqpoint{0.766179in}{3.116667in}}%
\pgfpathlineto{\pgfqpoint{0.768445in}{3.135965in}}%
\pgfpathlineto{\pgfqpoint{0.769018in}{3.155263in}}%
\pgfpathlineto{\pgfqpoint{0.767838in}{3.174561in}}%
\pgfpathlineto{\pgfqpoint{0.764814in}{3.193860in}}%
\pgfpathlineto{\pgfqpoint{0.757293in}{3.222807in}}%
\pgfpathlineto{\pgfqpoint{0.754878in}{3.232456in}}%
\pgfpathlineto{\pgfqpoint{0.754225in}{3.242105in}}%
\pgfpathlineto{\pgfqpoint{0.757101in}{3.251754in}}%
\pgfpathlineto{\pgfqpoint{0.799812in}{3.312427in}}%
\pgfpathlineto{\pgfqpoint{0.814226in}{3.338596in}}%
\pgfpathlineto{\pgfqpoint{0.826595in}{3.367544in}}%
\pgfpathlineto{\pgfqpoint{0.835322in}{3.396491in}}%
\pgfpathlineto{\pgfqpoint{0.839088in}{3.415789in}}%
\pgfpathlineto{\pgfqpoint{0.841287in}{3.435088in}}%
\pgfpathlineto{\pgfqpoint{0.841769in}{3.454386in}}%
\pgfpathlineto{\pgfqpoint{0.840440in}{3.473684in}}%
\pgfpathlineto{\pgfqpoint{0.837177in}{3.492982in}}%
\pgfpathlineto{\pgfqpoint{0.831758in}{3.512281in}}%
\pgfpathlineto{\pgfqpoint{0.823816in}{3.531579in}}%
\pgfpathlineto{\pgfqpoint{0.812852in}{3.550877in}}%
\pgfpathlineto{\pgfqpoint{0.798213in}{3.570175in}}%
\pgfpathlineto{\pgfqpoint{0.779306in}{3.589474in}}%
\pgfpathlineto{\pgfqpoint{0.744955in}{3.618421in}}%
\pgfpathlineto{\pgfqpoint{0.731830in}{3.629461in}}%
\pgfpathlineto{\pgfqpoint{0.663847in}{3.694220in}}%
\pgfpathlineto{\pgfqpoint{0.662897in}{3.695614in}}%
\pgfpathlineto{\pgfqpoint{0.644424in}{3.717464in}}%
\pgfpathlineto{\pgfqpoint{0.634623in}{3.734211in}}%
\pgfpathlineto{\pgfqpoint{0.625150in}{3.763158in}}%
\pgfpathlineto{\pgfqpoint{0.628390in}{3.782456in}}%
\pgfpathlineto{\pgfqpoint{0.636644in}{3.801754in}}%
\pgfpathlineto{\pgfqpoint{0.649569in}{3.821053in}}%
\pgfpathlineto{\pgfqpoint{0.657360in}{3.830702in}}%
\pgfpathlineto{\pgfqpoint{0.666065in}{3.840351in}}%
\pgfpathlineto{\pgfqpoint{0.683271in}{3.857874in}}%
\pgfpathlineto{\pgfqpoint{0.685647in}{3.859649in}}%
\pgfpathlineto{\pgfqpoint{0.712406in}{3.883787in}}%
\pgfpathlineto{\pgfqpoint{0.722118in}{3.893317in}}%
\pgfpathlineto{\pgfqpoint{0.731830in}{3.904175in}}%
\pgfpathlineto{\pgfqpoint{0.735256in}{3.907895in}}%
\pgfpathlineto{\pgfqpoint{0.748232in}{3.927193in}}%
\pgfpathlineto{\pgfqpoint{0.765827in}{3.965789in}}%
\pgfpathlineto{\pgfqpoint{0.770677in}{3.976715in}}%
\pgfpathlineto{\pgfqpoint{0.780388in}{3.991137in}}%
\pgfpathlineto{\pgfqpoint{0.790100in}{4.000527in}}%
\pgfpathlineto{\pgfqpoint{0.799812in}{4.007963in}}%
\pgfpathlineto{\pgfqpoint{0.828947in}{4.026948in}}%
\pgfpathlineto{\pgfqpoint{0.868019in}{4.052632in}}%
\pgfpathlineto{\pgfqpoint{0.896930in}{4.074359in}}%
\pgfpathlineto{\pgfqpoint{0.916647in}{4.091228in}}%
\pgfpathlineto{\pgfqpoint{0.936670in}{4.110526in}}%
\pgfpathlineto{\pgfqpoint{0.955201in}{4.130802in}}%
\pgfpathlineto{\pgfqpoint{0.977243in}{4.158772in}}%
\pgfpathlineto{\pgfqpoint{0.996420in}{4.187719in}}%
\pgfpathlineto{\pgfqpoint{1.013471in}{4.218745in}}%
\pgfpathlineto{\pgfqpoint{1.025738in}{4.245614in}}%
\pgfpathlineto{\pgfqpoint{1.036539in}{4.274561in}}%
\pgfpathlineto{\pgfqpoint{1.045042in}{4.303509in}}%
\pgfpathlineto{\pgfqpoint{1.052318in}{4.337933in}}%
\pgfpathlineto{\pgfqpoint{1.055693in}{4.361404in}}%
\pgfpathlineto{\pgfqpoint{1.057990in}{4.390351in}}%
\pgfpathlineto{\pgfqpoint{1.058316in}{4.400000in}}%
\pgfpathlineto{\pgfqpoint{1.058316in}{4.400000in}}%
\pgfusepath{stroke}%
\end{pgfscope}%
\begin{pgfscope}%
\pgfpathrectangle{\pgfqpoint{0.625000in}{0.550000in}}{\pgfqpoint{3.875000in}{3.850000in}} %
\pgfusepath{clip}%
\pgfsetbuttcap%
\pgfsetroundjoin%
\pgfsetlinewidth{0.250937pt}%
\definecolor{currentstroke}{rgb}{0.000000,0.000000,0.000000}%
\pgfsetstrokecolor{currentstroke}%
\pgfsetdash{}{0pt}%
\pgfpathmoveto{\pgfqpoint{0.625000in}{2.643737in}}%
\pgfpathlineto{\pgfqpoint{0.634504in}{2.634211in}}%
\pgfpathlineto{\pgfqpoint{0.625000in}{2.625309in}}%
\pgfusepath{stroke}%
\end{pgfscope}%
\begin{pgfscope}%
\pgfpathrectangle{\pgfqpoint{0.625000in}{0.550000in}}{\pgfqpoint{3.875000in}{3.850000in}} %
\pgfusepath{clip}%
\pgfsetbuttcap%
\pgfsetroundjoin%
\pgfsetlinewidth{0.250937pt}%
\definecolor{currentstroke}{rgb}{0.000000,0.000000,0.000000}%
\pgfsetstrokecolor{currentstroke}%
\pgfsetdash{}{0pt}%
\pgfpathmoveto{\pgfqpoint{0.625000in}{2.797787in}}%
\pgfpathlineto{\pgfqpoint{0.634450in}{2.788596in}}%
\pgfpathlineto{\pgfqpoint{0.625000in}{2.779663in}}%
\pgfusepath{stroke}%
\end{pgfscope}%
\begin{pgfscope}%
\pgfpathrectangle{\pgfqpoint{0.625000in}{0.550000in}}{\pgfqpoint{3.875000in}{3.850000in}} %
\pgfusepath{clip}%
\pgfsetbuttcap%
\pgfsetroundjoin%
\pgfsetlinewidth{0.250937pt}%
\definecolor{currentstroke}{rgb}{0.000000,0.000000,0.000000}%
\pgfsetstrokecolor{currentstroke}%
\pgfsetdash{}{0pt}%
\pgfpathmoveto{\pgfqpoint{0.625000in}{2.952322in}}%
\pgfpathlineto{\pgfqpoint{0.634426in}{2.942982in}}%
\pgfpathlineto{\pgfqpoint{0.625000in}{2.933565in}}%
\pgfusepath{stroke}%
\end{pgfscope}%
\begin{pgfscope}%
\pgfpathrectangle{\pgfqpoint{0.625000in}{0.550000in}}{\pgfqpoint{3.875000in}{3.850000in}} %
\pgfusepath{clip}%
\pgfsetbuttcap%
\pgfsetroundjoin%
\pgfsetlinewidth{0.250937pt}%
\definecolor{currentstroke}{rgb}{0.000000,0.000000,0.000000}%
\pgfsetstrokecolor{currentstroke}%
\pgfsetdash{}{0pt}%
\pgfpathmoveto{\pgfqpoint{0.625000in}{3.031101in}}%
\pgfpathlineto{\pgfqpoint{0.632858in}{3.039474in}}%
\pgfpathlineto{\pgfqpoint{0.634712in}{3.043439in}}%
\pgfpathlineto{\pgfqpoint{0.644424in}{3.042616in}}%
\pgfpathlineto{\pgfqpoint{0.650861in}{3.039474in}}%
\pgfpathlineto{\pgfqpoint{0.654042in}{3.029825in}}%
\pgfpathlineto{\pgfqpoint{0.644972in}{3.020175in}}%
\pgfpathlineto{\pgfqpoint{0.644424in}{3.019999in}}%
\pgfpathlineto{\pgfqpoint{0.642159in}{3.020175in}}%
\pgfpathlineto{\pgfqpoint{0.634712in}{3.020302in}}%
\pgfpathlineto{\pgfqpoint{0.625000in}{3.028904in}}%
\pgfusepath{stroke}%
\end{pgfscope}%
\begin{pgfscope}%
\pgfpathrectangle{\pgfqpoint{0.625000in}{0.550000in}}{\pgfqpoint{3.875000in}{3.850000in}} %
\pgfusepath{clip}%
\pgfsetbuttcap%
\pgfsetroundjoin%
\pgfsetlinewidth{0.250937pt}%
\definecolor{currentstroke}{rgb}{0.000000,0.000000,0.000000}%
\pgfsetstrokecolor{currentstroke}%
\pgfsetdash{}{0pt}%
\pgfpathmoveto{\pgfqpoint{0.625000in}{3.070330in}}%
\pgfpathlineto{\pgfqpoint{0.629096in}{3.068421in}}%
\pgfpathlineto{\pgfqpoint{0.625000in}{3.067224in}}%
\pgfusepath{stroke}%
\end{pgfscope}%
\begin{pgfscope}%
\pgfpathrectangle{\pgfqpoint{0.625000in}{0.550000in}}{\pgfqpoint{3.875000in}{3.850000in}} %
\pgfusepath{clip}%
\pgfsetbuttcap%
\pgfsetroundjoin%
\pgfsetlinewidth{0.250937pt}%
\definecolor{currentstroke}{rgb}{0.000000,0.000000,0.000000}%
\pgfsetstrokecolor{currentstroke}%
\pgfsetdash{}{0pt}%
\pgfpathmoveto{\pgfqpoint{0.625000in}{3.106254in}}%
\pgfpathlineto{\pgfqpoint{0.634423in}{3.097368in}}%
\pgfpathlineto{\pgfqpoint{0.625000in}{3.088293in}}%
\pgfusepath{stroke}%
\end{pgfscope}%
\begin{pgfscope}%
\pgfpathrectangle{\pgfqpoint{0.625000in}{0.550000in}}{\pgfqpoint{3.875000in}{3.850000in}} %
\pgfusepath{clip}%
\pgfsetbuttcap%
\pgfsetroundjoin%
\pgfsetlinewidth{0.250937pt}%
\definecolor{currentstroke}{rgb}{0.000000,0.000000,0.000000}%
\pgfsetstrokecolor{currentstroke}%
\pgfsetdash{}{0pt}%
\pgfpathmoveto{\pgfqpoint{0.625000in}{3.260775in}}%
\pgfpathlineto{\pgfqpoint{0.630309in}{3.261404in}}%
\pgfpathlineto{\pgfqpoint{0.634712in}{3.270303in}}%
\pgfpathlineto{\pgfqpoint{0.637605in}{3.271053in}}%
\pgfpathlineto{\pgfqpoint{0.644424in}{3.277115in}}%
\pgfpathlineto{\pgfqpoint{0.654135in}{3.280162in}}%
\pgfpathlineto{\pgfqpoint{0.658427in}{3.280702in}}%
\pgfpathlineto{\pgfqpoint{0.663847in}{3.282100in}}%
\pgfpathlineto{\pgfqpoint{0.673559in}{3.282306in}}%
\pgfpathlineto{\pgfqpoint{0.683271in}{3.280830in}}%
\pgfpathlineto{\pgfqpoint{0.683834in}{3.280702in}}%
\pgfpathlineto{\pgfqpoint{0.692982in}{3.277539in}}%
\pgfpathlineto{\pgfqpoint{0.702694in}{3.272454in}}%
\pgfpathlineto{\pgfqpoint{0.705461in}{3.271053in}}%
\pgfpathlineto{\pgfqpoint{0.712406in}{3.264113in}}%
\pgfpathlineto{\pgfqpoint{0.715938in}{3.261404in}}%
\pgfpathlineto{\pgfqpoint{0.720378in}{3.251754in}}%
\pgfpathlineto{\pgfqpoint{0.718376in}{3.242105in}}%
\pgfpathlineto{\pgfqpoint{0.712406in}{3.237202in}}%
\pgfpathlineto{\pgfqpoint{0.707791in}{3.232456in}}%
\pgfpathlineto{\pgfqpoint{0.702694in}{3.230119in}}%
\pgfpathlineto{\pgfqpoint{0.692982in}{3.225698in}}%
\pgfpathlineto{\pgfqpoint{0.683711in}{3.222807in}}%
\pgfpathlineto{\pgfqpoint{0.683271in}{3.222719in}}%
\pgfpathlineto{\pgfqpoint{0.673559in}{3.221290in}}%
\pgfpathlineto{\pgfqpoint{0.663847in}{3.221414in}}%
\pgfpathlineto{\pgfqpoint{0.658406in}{3.222807in}}%
\pgfpathlineto{\pgfqpoint{0.654135in}{3.223341in}}%
\pgfpathlineto{\pgfqpoint{0.644424in}{3.226399in}}%
\pgfpathlineto{\pgfqpoint{0.637604in}{3.232456in}}%
\pgfpathlineto{\pgfqpoint{0.634712in}{3.233205in}}%
\pgfpathlineto{\pgfqpoint{0.630541in}{3.242105in}}%
\pgfpathlineto{\pgfqpoint{0.625000in}{3.242793in}}%
\pgfusepath{stroke}%
\end{pgfscope}%
\begin{pgfscope}%
\pgfpathrectangle{\pgfqpoint{0.625000in}{0.550000in}}{\pgfqpoint{3.875000in}{3.850000in}} %
\pgfusepath{clip}%
\pgfsetbuttcap%
\pgfsetroundjoin%
\pgfsetlinewidth{0.250937pt}%
\definecolor{currentstroke}{rgb}{0.000000,0.000000,0.000000}%
\pgfsetstrokecolor{currentstroke}%
\pgfsetdash{}{0pt}%
\pgfpathmoveto{\pgfqpoint{0.625000in}{3.415402in}}%
\pgfpathlineto{\pgfqpoint{0.634418in}{3.406140in}}%
\pgfpathlineto{\pgfqpoint{0.625000in}{3.397180in}}%
\pgfusepath{stroke}%
\end{pgfscope}%
\begin{pgfscope}%
\pgfpathrectangle{\pgfqpoint{0.625000in}{0.550000in}}{\pgfqpoint{3.875000in}{3.850000in}} %
\pgfusepath{clip}%
\pgfsetbuttcap%
\pgfsetroundjoin%
\pgfsetlinewidth{0.250937pt}%
\definecolor{currentstroke}{rgb}{0.000000,0.000000,0.000000}%
\pgfsetstrokecolor{currentstroke}%
\pgfsetdash{}{0pt}%
\pgfpathmoveto{\pgfqpoint{0.625000in}{3.569533in}}%
\pgfpathlineto{\pgfqpoint{0.634479in}{3.560526in}}%
\pgfpathlineto{\pgfqpoint{0.625000in}{3.551525in}}%
\pgfusepath{stroke}%
\end{pgfscope}%
\begin{pgfscope}%
\pgfpathrectangle{\pgfqpoint{0.625000in}{0.550000in}}{\pgfqpoint{3.875000in}{3.850000in}} %
\pgfusepath{clip}%
\pgfsetbuttcap%
\pgfsetroundjoin%
\pgfsetlinewidth{0.250937pt}%
\definecolor{currentstroke}{rgb}{0.000000,0.000000,0.000000}%
\pgfsetstrokecolor{currentstroke}%
\pgfsetdash{}{0pt}%
\pgfpathmoveto{\pgfqpoint{0.625000in}{3.667880in}}%
\pgfpathlineto{\pgfqpoint{0.629036in}{3.666667in}}%
\pgfpathlineto{\pgfqpoint{0.625000in}{3.665223in}}%
\pgfusepath{stroke}%
\end{pgfscope}%
\begin{pgfscope}%
\pgfpathrectangle{\pgfqpoint{0.625000in}{0.550000in}}{\pgfqpoint{3.875000in}{3.850000in}} %
\pgfusepath{clip}%
\pgfsetbuttcap%
\pgfsetroundjoin%
\pgfsetlinewidth{0.250937pt}%
\definecolor{currentstroke}{rgb}{0.000000,0.000000,0.000000}%
\pgfsetstrokecolor{currentstroke}%
\pgfsetdash{}{0pt}%
\pgfpathmoveto{\pgfqpoint{0.625000in}{3.724008in}}%
\pgfpathlineto{\pgfqpoint{0.634498in}{3.714912in}}%
\pgfpathlineto{\pgfqpoint{0.625000in}{3.705542in}}%
\pgfusepath{stroke}%
\end{pgfscope}%
\begin{pgfscope}%
\pgfpathrectangle{\pgfqpoint{0.625000in}{0.550000in}}{\pgfqpoint{3.875000in}{3.850000in}} %
\pgfusepath{clip}%
\pgfsetbuttcap%
\pgfsetroundjoin%
\pgfsetlinewidth{0.250937pt}%
\definecolor{currentstroke}{rgb}{0.000000,0.000000,0.000000}%
\pgfsetstrokecolor{currentstroke}%
\pgfsetdash{}{0pt}%
\pgfpathmoveto{\pgfqpoint{0.625000in}{3.878291in}}%
\pgfpathlineto{\pgfqpoint{0.634451in}{3.869298in}}%
\pgfpathlineto{\pgfqpoint{0.625000in}{3.860268in}}%
\pgfusepath{stroke}%
\end{pgfscope}%
\begin{pgfscope}%
\pgfpathrectangle{\pgfqpoint{0.625000in}{0.550000in}}{\pgfqpoint{3.875000in}{3.850000in}} %
\pgfusepath{clip}%
\pgfsetbuttcap%
\pgfsetroundjoin%
\pgfsetlinewidth{0.250937pt}%
\definecolor{currentstroke}{rgb}{0.000000,0.000000,0.000000}%
\pgfsetstrokecolor{currentstroke}%
\pgfsetdash{}{0pt}%
\pgfpathmoveto{\pgfqpoint{0.625000in}{4.032764in}}%
\pgfpathlineto{\pgfqpoint{0.630095in}{4.033333in}}%
\pgfpathlineto{\pgfqpoint{0.634712in}{4.042233in}}%
\pgfpathlineto{\pgfqpoint{0.637606in}{4.042982in}}%
\pgfpathlineto{\pgfqpoint{0.644424in}{4.049043in}}%
\pgfpathlineto{\pgfqpoint{0.654135in}{4.052093in}}%
\pgfpathlineto{\pgfqpoint{0.658453in}{4.052632in}}%
\pgfpathlineto{\pgfqpoint{0.663847in}{4.053994in}}%
\pgfpathlineto{\pgfqpoint{0.673559in}{4.054270in}}%
\pgfpathlineto{\pgfqpoint{0.683271in}{4.052948in}}%
\pgfpathlineto{\pgfqpoint{0.684659in}{4.052632in}}%
\pgfpathlineto{\pgfqpoint{0.692982in}{4.049708in}}%
\pgfpathlineto{\pgfqpoint{0.702694in}{4.044108in}}%
\pgfpathlineto{\pgfqpoint{0.704533in}{4.042982in}}%
\pgfpathlineto{\pgfqpoint{0.712406in}{4.033617in}}%
\pgfpathlineto{\pgfqpoint{0.712688in}{4.033333in}}%
\pgfpathlineto{\pgfqpoint{0.716715in}{4.023684in}}%
\pgfpathlineto{\pgfqpoint{0.715496in}{4.014035in}}%
\pgfpathlineto{\pgfqpoint{0.712406in}{4.010352in}}%
\pgfpathlineto{\pgfqpoint{0.707854in}{4.004386in}}%
\pgfpathlineto{\pgfqpoint{0.702694in}{4.001627in}}%
\pgfpathlineto{\pgfqpoint{0.692982in}{3.997203in}}%
\pgfpathlineto{\pgfqpoint{0.684598in}{3.994737in}}%
\pgfpathlineto{\pgfqpoint{0.683271in}{3.994494in}}%
\pgfpathlineto{\pgfqpoint{0.673559in}{3.993238in}}%
\pgfpathlineto{\pgfqpoint{0.663847in}{3.993382in}}%
\pgfpathlineto{\pgfqpoint{0.658419in}{3.994737in}}%
\pgfpathlineto{\pgfqpoint{0.654135in}{3.995265in}}%
\pgfpathlineto{\pgfqpoint{0.644424in}{3.998334in}}%
\pgfpathlineto{\pgfqpoint{0.637604in}{4.004386in}}%
\pgfpathlineto{\pgfqpoint{0.634712in}{4.005135in}}%
\pgfpathlineto{\pgfqpoint{0.630467in}{4.014035in}}%
\pgfpathlineto{\pgfqpoint{0.625000in}{4.014693in}}%
\pgfusepath{stroke}%
\end{pgfscope}%
\begin{pgfscope}%
\pgfpathrectangle{\pgfqpoint{0.625000in}{0.550000in}}{\pgfqpoint{3.875000in}{3.850000in}} %
\pgfusepath{clip}%
\pgfsetbuttcap%
\pgfsetroundjoin%
\pgfsetlinewidth{0.250937pt}%
\definecolor{currentstroke}{rgb}{0.000000,0.000000,0.000000}%
\pgfsetstrokecolor{currentstroke}%
\pgfsetdash{}{0pt}%
\pgfpathmoveto{\pgfqpoint{0.625000in}{4.187177in}}%
\pgfpathlineto{\pgfqpoint{0.634460in}{4.178070in}}%
\pgfpathlineto{\pgfqpoint{0.625000in}{4.169390in}}%
\pgfusepath{stroke}%
\end{pgfscope}%
\begin{pgfscope}%
\pgfpathrectangle{\pgfqpoint{0.625000in}{0.550000in}}{\pgfqpoint{3.875000in}{3.850000in}} %
\pgfusepath{clip}%
\pgfsetbuttcap%
\pgfsetroundjoin%
\pgfsetlinewidth{0.250937pt}%
\definecolor{currentstroke}{rgb}{0.000000,0.000000,0.000000}%
\pgfsetstrokecolor{currentstroke}%
\pgfsetdash{}{0pt}%
\pgfpathmoveto{\pgfqpoint{0.625000in}{4.341506in}}%
\pgfpathlineto{\pgfqpoint{0.634401in}{4.332456in}}%
\pgfpathlineto{\pgfqpoint{0.625000in}{4.323375in}}%
\pgfusepath{stroke}%
\end{pgfscope}%
\begin{pgfscope}%
\pgfpathrectangle{\pgfqpoint{0.625000in}{0.550000in}}{\pgfqpoint{3.875000in}{3.850000in}} %
\pgfusepath{clip}%
\pgfsetbuttcap%
\pgfsetroundjoin%
\pgfsetlinewidth{0.250937pt}%
\definecolor{currentstroke}{rgb}{0.000000,0.000000,0.000000}%
\pgfsetstrokecolor{currentstroke}%
\pgfsetdash{}{0pt}%
\pgfpathmoveto{\pgfqpoint{0.634712in}{0.758149in}}%
\pgfpathlineto{\pgfqpoint{0.632858in}{0.762281in}}%
\pgfpathlineto{\pgfqpoint{0.634712in}{0.768907in}}%
\pgfpathlineto{\pgfqpoint{0.641145in}{0.762281in}}%
\pgfpathlineto{\pgfqpoint{0.634712in}{0.758149in}}%
\pgfusepath{stroke}%
\end{pgfscope}%
\begin{pgfscope}%
\pgfpathrectangle{\pgfqpoint{0.625000in}{0.550000in}}{\pgfqpoint{3.875000in}{3.850000in}} %
\pgfusepath{clip}%
\pgfsetbuttcap%
\pgfsetroundjoin%
\pgfsetlinewidth{0.250937pt}%
\definecolor{currentstroke}{rgb}{0.000000,0.000000,0.000000}%
\pgfsetstrokecolor{currentstroke}%
\pgfsetdash{}{0pt}%
\pgfpathmoveto{\pgfqpoint{0.634712in}{0.813487in}}%
\pgfpathlineto{\pgfqpoint{0.631666in}{0.820175in}}%
\pgfpathlineto{\pgfqpoint{0.631472in}{0.829825in}}%
\pgfpathlineto{\pgfqpoint{0.634712in}{0.837519in}}%
\pgfpathlineto{\pgfqpoint{0.644424in}{0.836714in}}%
\pgfpathlineto{\pgfqpoint{0.653540in}{0.829825in}}%
\pgfpathlineto{\pgfqpoint{0.653003in}{0.820175in}}%
\pgfpathlineto{\pgfqpoint{0.644424in}{0.814523in}}%
\pgfpathlineto{\pgfqpoint{0.634712in}{0.813487in}}%
\pgfusepath{stroke}%
\end{pgfscope}%
\begin{pgfscope}%
\pgfpathrectangle{\pgfqpoint{0.625000in}{0.550000in}}{\pgfqpoint{3.875000in}{3.850000in}} %
\pgfusepath{clip}%
\pgfsetbuttcap%
\pgfsetroundjoin%
\pgfsetlinewidth{0.250937pt}%
\definecolor{currentstroke}{rgb}{0.000000,0.000000,0.000000}%
\pgfsetstrokecolor{currentstroke}%
\pgfsetdash{}{0pt}%
\pgfpathmoveto{\pgfqpoint{0.634712in}{1.364710in}}%
\pgfpathlineto{\pgfqpoint{0.632221in}{1.370175in}}%
\pgfpathlineto{\pgfqpoint{0.630170in}{1.379825in}}%
\pgfpathlineto{\pgfqpoint{0.634712in}{1.388421in}}%
\pgfpathlineto{\pgfqpoint{0.644424in}{1.388059in}}%
\pgfpathlineto{\pgfqpoint{0.653857in}{1.379825in}}%
\pgfpathlineto{\pgfqpoint{0.652191in}{1.370175in}}%
\pgfpathlineto{\pgfqpoint{0.644424in}{1.365749in}}%
\pgfpathlineto{\pgfqpoint{0.634712in}{1.364710in}}%
\pgfusepath{stroke}%
\end{pgfscope}%
\begin{pgfscope}%
\pgfpathrectangle{\pgfqpoint{0.625000in}{0.550000in}}{\pgfqpoint{3.875000in}{3.850000in}} %
\pgfusepath{clip}%
\pgfsetbuttcap%
\pgfsetroundjoin%
\pgfsetlinewidth{0.250937pt}%
\definecolor{currentstroke}{rgb}{0.000000,0.000000,0.000000}%
\pgfsetstrokecolor{currentstroke}%
\pgfsetdash{}{0pt}%
\pgfpathmoveto{\pgfqpoint{0.634712in}{1.616233in}}%
\pgfpathlineto{\pgfqpoint{0.631393in}{1.621053in}}%
\pgfpathlineto{\pgfqpoint{0.634712in}{1.626999in}}%
\pgfpathlineto{\pgfqpoint{0.641535in}{1.621053in}}%
\pgfpathlineto{\pgfqpoint{0.634712in}{1.616233in}}%
\pgfusepath{stroke}%
\end{pgfscope}%
\begin{pgfscope}%
\pgfpathrectangle{\pgfqpoint{0.625000in}{0.550000in}}{\pgfqpoint{3.875000in}{3.850000in}} %
\pgfusepath{clip}%
\pgfsetbuttcap%
\pgfsetroundjoin%
\pgfsetlinewidth{0.250937pt}%
\definecolor{currentstroke}{rgb}{0.000000,0.000000,0.000000}%
\pgfsetstrokecolor{currentstroke}%
\pgfsetdash{}{0pt}%
\pgfpathmoveto{\pgfqpoint{0.634712in}{2.044816in}}%
\pgfpathlineto{\pgfqpoint{0.634421in}{2.045614in}}%
\pgfpathlineto{\pgfqpoint{0.633911in}{2.055263in}}%
\pgfpathlineto{\pgfqpoint{0.634712in}{2.057463in}}%
\pgfpathlineto{\pgfqpoint{0.639057in}{2.055263in}}%
\pgfpathlineto{\pgfqpoint{0.636605in}{2.045614in}}%
\pgfpathlineto{\pgfqpoint{0.634712in}{2.044816in}}%
\pgfusepath{stroke}%
\end{pgfscope}%
\begin{pgfscope}%
\pgfpathrectangle{\pgfqpoint{0.625000in}{0.550000in}}{\pgfqpoint{3.875000in}{3.850000in}} %
\pgfusepath{clip}%
\pgfsetbuttcap%
\pgfsetroundjoin%
\pgfsetlinewidth{0.250937pt}%
\definecolor{currentstroke}{rgb}{0.000000,0.000000,0.000000}%
\pgfsetstrokecolor{currentstroke}%
\pgfsetdash{}{0pt}%
\pgfpathmoveto{\pgfqpoint{0.634712in}{2.902186in}}%
\pgfpathlineto{\pgfqpoint{0.633911in}{2.904386in}}%
\pgfpathlineto{\pgfqpoint{0.634421in}{2.914035in}}%
\pgfpathlineto{\pgfqpoint{0.634712in}{2.914833in}}%
\pgfpathlineto{\pgfqpoint{0.636605in}{2.914035in}}%
\pgfpathlineto{\pgfqpoint{0.639057in}{2.904386in}}%
\pgfpathlineto{\pgfqpoint{0.634712in}{2.902186in}}%
\pgfusepath{stroke}%
\end{pgfscope}%
\begin{pgfscope}%
\pgfpathrectangle{\pgfqpoint{0.625000in}{0.550000in}}{\pgfqpoint{3.875000in}{3.850000in}} %
\pgfusepath{clip}%
\pgfsetbuttcap%
\pgfsetroundjoin%
\pgfsetlinewidth{0.250937pt}%
\definecolor{currentstroke}{rgb}{0.000000,0.000000,0.000000}%
\pgfsetstrokecolor{currentstroke}%
\pgfsetdash{}{0pt}%
\pgfpathmoveto{\pgfqpoint{0.634712in}{3.332650in}}%
\pgfpathlineto{\pgfqpoint{0.631393in}{3.338596in}}%
\pgfpathlineto{\pgfqpoint{0.634712in}{3.343416in}}%
\pgfpathlineto{\pgfqpoint{0.641535in}{3.338596in}}%
\pgfpathlineto{\pgfqpoint{0.634712in}{3.332650in}}%
\pgfusepath{stroke}%
\end{pgfscope}%
\begin{pgfscope}%
\pgfpathrectangle{\pgfqpoint{0.625000in}{0.550000in}}{\pgfqpoint{3.875000in}{3.850000in}} %
\pgfusepath{clip}%
\pgfsetbuttcap%
\pgfsetroundjoin%
\pgfsetlinewidth{0.250937pt}%
\definecolor{currentstroke}{rgb}{0.000000,0.000000,0.000000}%
\pgfsetstrokecolor{currentstroke}%
\pgfsetdash{}{0pt}%
\pgfpathmoveto{\pgfqpoint{0.634712in}{3.571228in}}%
\pgfpathlineto{\pgfqpoint{0.630170in}{3.579825in}}%
\pgfpathlineto{\pgfqpoint{0.632221in}{3.589474in}}%
\pgfpathlineto{\pgfqpoint{0.634712in}{3.594939in}}%
\pgfpathlineto{\pgfqpoint{0.644424in}{3.593900in}}%
\pgfpathlineto{\pgfqpoint{0.652191in}{3.589474in}}%
\pgfpathlineto{\pgfqpoint{0.653857in}{3.579825in}}%
\pgfpathlineto{\pgfqpoint{0.644424in}{3.571590in}}%
\pgfpathlineto{\pgfqpoint{0.634712in}{3.571228in}}%
\pgfusepath{stroke}%
\end{pgfscope}%
\begin{pgfscope}%
\pgfpathrectangle{\pgfqpoint{0.625000in}{0.550000in}}{\pgfqpoint{3.875000in}{3.850000in}} %
\pgfusepath{clip}%
\pgfsetbuttcap%
\pgfsetroundjoin%
\pgfsetlinewidth{0.250937pt}%
\definecolor{currentstroke}{rgb}{0.000000,0.000000,0.000000}%
\pgfsetstrokecolor{currentstroke}%
\pgfsetdash{}{0pt}%
\pgfpathmoveto{\pgfqpoint{0.634712in}{4.122130in}}%
\pgfpathlineto{\pgfqpoint{0.631472in}{4.129825in}}%
\pgfpathlineto{\pgfqpoint{0.631666in}{4.139474in}}%
\pgfpathlineto{\pgfqpoint{0.634712in}{4.146162in}}%
\pgfpathlineto{\pgfqpoint{0.644424in}{4.145126in}}%
\pgfpathlineto{\pgfqpoint{0.653003in}{4.139474in}}%
\pgfpathlineto{\pgfqpoint{0.653540in}{4.129825in}}%
\pgfpathlineto{\pgfqpoint{0.644424in}{4.122935in}}%
\pgfpathlineto{\pgfqpoint{0.634712in}{4.122130in}}%
\pgfusepath{stroke}%
\end{pgfscope}%
\begin{pgfscope}%
\pgfpathrectangle{\pgfqpoint{0.625000in}{0.550000in}}{\pgfqpoint{3.875000in}{3.850000in}} %
\pgfusepath{clip}%
\pgfsetbuttcap%
\pgfsetroundjoin%
\pgfsetlinewidth{0.250937pt}%
\definecolor{currentstroke}{rgb}{0.000000,0.000000,0.000000}%
\pgfsetstrokecolor{currentstroke}%
\pgfsetdash{}{0pt}%
\pgfpathmoveto{\pgfqpoint{0.634712in}{4.190742in}}%
\pgfpathlineto{\pgfqpoint{0.632858in}{4.197368in}}%
\pgfpathlineto{\pgfqpoint{0.634712in}{4.201500in}}%
\pgfpathlineto{\pgfqpoint{0.641145in}{4.197368in}}%
\pgfpathlineto{\pgfqpoint{0.634712in}{4.190742in}}%
\pgfusepath{stroke}%
\end{pgfscope}%
\begin{pgfscope}%
\pgfpathrectangle{\pgfqpoint{0.625000in}{0.550000in}}{\pgfqpoint{3.875000in}{3.850000in}} %
\pgfusepath{clip}%
\pgfsetbuttcap%
\pgfsetroundjoin%
\pgfsetlinewidth{0.250937pt}%
\definecolor{currentstroke}{rgb}{0.000000,0.000000,0.000000}%
\pgfsetstrokecolor{currentstroke}%
\pgfsetdash{}{0pt}%
\pgfpathmoveto{\pgfqpoint{0.887880in}{0.550000in}}%
\pgfpathlineto{\pgfqpoint{0.887698in}{0.559649in}}%
\pgfpathlineto{\pgfqpoint{0.887218in}{0.568096in}}%
\pgfpathlineto{\pgfqpoint{0.887152in}{0.569298in}}%
\pgfpathlineto{\pgfqpoint{0.886266in}{0.578947in}}%
\pgfpathlineto{\pgfqpoint{0.885014in}{0.588596in}}%
\pgfpathlineto{\pgfqpoint{0.883386in}{0.598246in}}%
\pgfpathlineto{\pgfqpoint{0.881369in}{0.607895in}}%
\pgfpathlineto{\pgfqpoint{0.878946in}{0.617544in}}%
\pgfpathlineto{\pgfqpoint{0.877506in}{0.622474in}}%
\pgfpathlineto{\pgfqpoint{0.876144in}{0.627193in}}%
\pgfpathlineto{\pgfqpoint{0.872949in}{0.636842in}}%
\pgfpathlineto{\pgfqpoint{0.869289in}{0.646491in}}%
\pgfpathlineto{\pgfqpoint{0.867794in}{0.650027in}}%
\pgfpathlineto{\pgfqpoint{0.865208in}{0.656140in}}%
\pgfpathlineto{\pgfqpoint{0.860647in}{0.665789in}}%
\pgfpathlineto{\pgfqpoint{0.858083in}{0.670704in}}%
\pgfpathlineto{\pgfqpoint{0.855586in}{0.675439in}}%
\pgfpathlineto{\pgfqpoint{0.849989in}{0.685088in}}%
\pgfpathlineto{\pgfqpoint{0.848371in}{0.687662in}}%
\pgfpathlineto{\pgfqpoint{0.843832in}{0.694737in}}%
\pgfpathlineto{\pgfqpoint{0.838659in}{0.702128in}}%
\pgfpathlineto{\pgfqpoint{0.837034in}{0.704386in}}%
\pgfpathlineto{\pgfqpoint{0.829557in}{0.714035in}}%
\pgfpathlineto{\pgfqpoint{0.828947in}{0.714775in}}%
\pgfpathlineto{\pgfqpoint{0.821324in}{0.723684in}}%
\pgfpathlineto{\pgfqpoint{0.819236in}{0.725977in}}%
\pgfpathlineto{\pgfqpoint{0.812221in}{0.733333in}}%
\pgfpathlineto{\pgfqpoint{0.809524in}{0.736013in}}%
\pgfpathlineto{\pgfqpoint{0.802120in}{0.742982in}}%
\pgfpathlineto{\pgfqpoint{0.799812in}{0.745057in}}%
\pgfpathlineto{\pgfqpoint{0.790845in}{0.752632in}}%
\pgfpathlineto{\pgfqpoint{0.790100in}{0.753237in}}%
\pgfpathlineto{\pgfqpoint{0.780388in}{0.760668in}}%
\pgfpathlineto{\pgfqpoint{0.778118in}{0.762281in}}%
\pgfpathlineto{\pgfqpoint{0.770677in}{0.767424in}}%
\pgfpathlineto{\pgfqpoint{0.763561in}{0.771930in}}%
\pgfpathlineto{\pgfqpoint{0.760965in}{0.773542in}}%
\pgfpathlineto{\pgfqpoint{0.751253in}{0.779101in}}%
\pgfpathlineto{\pgfqpoint{0.746475in}{0.781579in}}%
\pgfpathlineto{\pgfqpoint{0.741541in}{0.784114in}}%
\pgfpathlineto{\pgfqpoint{0.731830in}{0.788598in}}%
\pgfpathlineto{\pgfqpoint{0.725329in}{0.791228in}}%
\pgfpathlineto{\pgfqpoint{0.722118in}{0.792525in}}%
\pgfpathlineto{\pgfqpoint{0.712406in}{0.795886in}}%
\pgfpathlineto{\pgfqpoint{0.702694in}{0.798488in}}%
\pgfpathlineto{\pgfqpoint{0.692982in}{0.800264in}}%
\pgfpathlineto{\pgfqpoint{0.688091in}{0.800877in}}%
\pgfpathlineto{\pgfqpoint{0.683271in}{0.801367in}}%
\pgfpathlineto{\pgfqpoint{0.673559in}{0.802201in}}%
\pgfpathlineto{\pgfqpoint{0.663847in}{0.803334in}}%
\pgfpathlineto{\pgfqpoint{0.654135in}{0.804977in}}%
\pgfpathlineto{\pgfqpoint{0.644424in}{0.807343in}}%
\pgfpathlineto{\pgfqpoint{0.640571in}{0.810526in}}%
\pgfpathlineto{\pgfqpoint{0.634712in}{0.811560in}}%
\pgfpathlineto{\pgfqpoint{0.630788in}{0.820175in}}%
\pgfpathlineto{\pgfqpoint{0.630756in}{0.829825in}}%
\pgfpathlineto{\pgfqpoint{0.634712in}{0.839220in}}%
\pgfpathlineto{\pgfqpoint{0.636738in}{0.839474in}}%
\pgfpathlineto{\pgfqpoint{0.644424in}{0.844125in}}%
\pgfpathlineto{\pgfqpoint{0.654135in}{0.845936in}}%
\pgfpathlineto{\pgfqpoint{0.663847in}{0.848518in}}%
\pgfpathlineto{\pgfqpoint{0.665476in}{0.849123in}}%
\pgfpathlineto{\pgfqpoint{0.673559in}{0.858168in}}%
\pgfpathlineto{\pgfqpoint{0.673937in}{0.858772in}}%
\pgfpathlineto{\pgfqpoint{0.674568in}{0.868421in}}%
\pgfpathlineto{\pgfqpoint{0.673559in}{0.870485in}}%
\pgfpathlineto{\pgfqpoint{0.671304in}{0.878070in}}%
\pgfpathlineto{\pgfqpoint{0.663847in}{0.887340in}}%
\pgfpathlineto{\pgfqpoint{0.663630in}{0.887719in}}%
\pgfpathlineto{\pgfqpoint{0.654135in}{0.895981in}}%
\pgfpathlineto{\pgfqpoint{0.653074in}{0.897368in}}%
\pgfpathlineto{\pgfqpoint{0.644424in}{0.903999in}}%
\pgfpathlineto{\pgfqpoint{0.642737in}{0.907018in}}%
\pgfpathlineto{\pgfqpoint{0.634712in}{0.914104in}}%
\pgfpathlineto{\pgfqpoint{0.633884in}{0.916667in}}%
\pgfpathlineto{\pgfqpoint{0.629340in}{0.926316in}}%
\pgfpathlineto{\pgfqpoint{0.625000in}{0.926759in}}%
\pgfusepath{stroke}%
\end{pgfscope}%
\begin{pgfscope}%
\pgfpathrectangle{\pgfqpoint{0.625000in}{0.550000in}}{\pgfqpoint{3.875000in}{3.850000in}} %
\pgfusepath{clip}%
\pgfsetbuttcap%
\pgfsetroundjoin%
\pgfsetlinewidth{0.250937pt}%
\definecolor{currentstroke}{rgb}{0.000000,0.000000,0.000000}%
\pgfsetstrokecolor{currentstroke}%
\pgfsetdash{}{0pt}%
\pgfpathmoveto{\pgfqpoint{0.625000in}{0.636425in}}%
\pgfpathlineto{\pgfqpoint{0.634522in}{0.627193in}}%
\pgfpathlineto{\pgfqpoint{0.625000in}{0.617988in}}%
\pgfusepath{stroke}%
\end{pgfscope}%
\begin{pgfscope}%
\pgfpathrectangle{\pgfqpoint{0.625000in}{0.550000in}}{\pgfqpoint{3.875000in}{3.850000in}} %
\pgfusepath{clip}%
\pgfsetbuttcap%
\pgfsetroundjoin%
\pgfsetlinewidth{0.250937pt}%
\definecolor{currentstroke}{rgb}{0.000000,0.000000,0.000000}%
\pgfsetstrokecolor{currentstroke}%
\pgfsetdash{}{0pt}%
\pgfpathmoveto{\pgfqpoint{0.625000in}{0.790423in}}%
\pgfpathlineto{\pgfqpoint{0.634567in}{0.781579in}}%
\pgfpathlineto{\pgfqpoint{0.634712in}{0.773479in}}%
\pgfpathlineto{\pgfqpoint{0.642652in}{0.771930in}}%
\pgfpathlineto{\pgfqpoint{0.644424in}{0.769748in}}%
\pgfpathlineto{\pgfqpoint{0.649591in}{0.762281in}}%
\pgfpathlineto{\pgfqpoint{0.644424in}{0.758727in}}%
\pgfpathlineto{\pgfqpoint{0.634712in}{0.755330in}}%
\pgfpathlineto{\pgfqpoint{0.631592in}{0.762281in}}%
\pgfpathlineto{\pgfqpoint{0.634121in}{0.771930in}}%
\pgfpathlineto{\pgfqpoint{0.625000in}{0.772341in}}%
\pgfusepath{stroke}%
\end{pgfscope}%
\begin{pgfscope}%
\pgfpathrectangle{\pgfqpoint{0.625000in}{0.550000in}}{\pgfqpoint{3.875000in}{3.850000in}} %
\pgfusepath{clip}%
\pgfsetbuttcap%
\pgfsetroundjoin%
\pgfsetlinewidth{0.250937pt}%
\definecolor{currentstroke}{rgb}{0.000000,0.000000,0.000000}%
\pgfsetstrokecolor{currentstroke}%
\pgfsetdash{}{0pt}%
\pgfpathmoveto{\pgfqpoint{0.625000in}{0.908442in}}%
\pgfpathlineto{\pgfqpoint{0.628904in}{0.907018in}}%
\pgfpathlineto{\pgfqpoint{0.625000in}{0.905856in}}%
\pgfusepath{stroke}%
\end{pgfscope}%
\begin{pgfscope}%
\pgfpathrectangle{\pgfqpoint{0.625000in}{0.550000in}}{\pgfqpoint{3.875000in}{3.850000in}} %
\pgfusepath{clip}%
\pgfsetbuttcap%
\pgfsetroundjoin%
\pgfsetlinewidth{0.250937pt}%
\definecolor{currentstroke}{rgb}{0.000000,0.000000,0.000000}%
\pgfsetstrokecolor{currentstroke}%
\pgfsetdash{}{0pt}%
\pgfpathmoveto{\pgfqpoint{0.625000in}{0.945089in}}%
\pgfpathlineto{\pgfqpoint{0.629772in}{0.945614in}}%
\pgfpathlineto{\pgfqpoint{0.633884in}{0.955263in}}%
\pgfpathlineto{\pgfqpoint{0.634712in}{0.957869in}}%
\pgfpathlineto{\pgfqpoint{0.642719in}{0.964912in}}%
\pgfpathlineto{\pgfqpoint{0.644424in}{0.967815in}}%
\pgfpathlineto{\pgfqpoint{0.652908in}{0.974561in}}%
\pgfpathlineto{\pgfqpoint{0.654135in}{0.976505in}}%
\pgfpathlineto{\pgfqpoint{0.663801in}{0.984211in}}%
\pgfpathlineto{\pgfqpoint{0.663847in}{0.984263in}}%
\pgfpathlineto{\pgfqpoint{0.673559in}{0.992419in}}%
\pgfpathlineto{\pgfqpoint{0.675454in}{0.993860in}}%
\pgfpathlineto{\pgfqpoint{0.683271in}{1.003170in}}%
\pgfpathlineto{\pgfqpoint{0.683560in}{1.003509in}}%
\pgfpathlineto{\pgfqpoint{0.688729in}{1.013158in}}%
\pgfpathlineto{\pgfqpoint{0.691074in}{1.022807in}}%
\pgfpathlineto{\pgfqpoint{0.691437in}{1.032456in}}%
\pgfpathlineto{\pgfqpoint{0.690112in}{1.042105in}}%
\pgfpathlineto{\pgfqpoint{0.687117in}{1.051754in}}%
\pgfpathlineto{\pgfqpoint{0.683271in}{1.059737in}}%
\pgfpathlineto{\pgfqpoint{0.682522in}{1.061404in}}%
\pgfpathlineto{\pgfqpoint{0.676948in}{1.071053in}}%
\pgfpathlineto{\pgfqpoint{0.673559in}{1.076381in}}%
\pgfpathlineto{\pgfqpoint{0.671034in}{1.080702in}}%
\pgfpathlineto{\pgfqpoint{0.666308in}{1.090351in}}%
\pgfpathlineto{\pgfqpoint{0.663847in}{1.095385in}}%
\pgfpathlineto{\pgfqpoint{0.662354in}{1.100000in}}%
\pgfpathlineto{\pgfqpoint{0.658083in}{1.109649in}}%
\pgfpathlineto{\pgfqpoint{0.654135in}{1.116197in}}%
\pgfpathlineto{\pgfqpoint{0.652731in}{1.119298in}}%
\pgfpathlineto{\pgfqpoint{0.647912in}{1.128947in}}%
\pgfpathlineto{\pgfqpoint{0.644424in}{1.134665in}}%
\pgfpathlineto{\pgfqpoint{0.642900in}{1.138596in}}%
\pgfpathlineto{\pgfqpoint{0.638566in}{1.148246in}}%
\pgfpathlineto{\pgfqpoint{0.634712in}{1.155461in}}%
\pgfpathlineto{\pgfqpoint{0.634062in}{1.157895in}}%
\pgfpathlineto{\pgfqpoint{0.631735in}{1.167544in}}%
\pgfpathlineto{\pgfqpoint{0.627847in}{1.177193in}}%
\pgfpathlineto{\pgfqpoint{0.626083in}{1.186842in}}%
\pgfpathlineto{\pgfqpoint{0.625000in}{1.196238in}}%
\pgfusepath{stroke}%
\end{pgfscope}%
\begin{pgfscope}%
\pgfpathrectangle{\pgfqpoint{0.625000in}{0.550000in}}{\pgfqpoint{3.875000in}{3.850000in}} %
\pgfusepath{clip}%
\pgfsetbuttcap%
\pgfsetroundjoin%
\pgfsetlinewidth{0.250937pt}%
\definecolor{currentstroke}{rgb}{0.000000,0.000000,0.000000}%
\pgfsetstrokecolor{currentstroke}%
\pgfsetdash{}{0pt}%
\pgfpathmoveto{\pgfqpoint{0.625000in}{1.099517in}}%
\pgfpathlineto{\pgfqpoint{0.634558in}{1.090351in}}%
\pgfpathlineto{\pgfqpoint{0.625000in}{1.081219in}}%
\pgfusepath{stroke}%
\end{pgfscope}%
\begin{pgfscope}%
\pgfpathrectangle{\pgfqpoint{0.625000in}{0.550000in}}{\pgfqpoint{3.875000in}{3.850000in}} %
\pgfusepath{clip}%
\pgfsetbuttcap%
\pgfsetroundjoin%
\pgfsetlinewidth{0.250937pt}%
\definecolor{currentstroke}{rgb}{0.000000,0.000000,0.000000}%
\pgfsetstrokecolor{currentstroke}%
\pgfsetdash{}{0pt}%
\pgfpathmoveto{\pgfqpoint{0.625000in}{1.196692in}}%
\pgfpathlineto{\pgfqpoint{0.627505in}{1.206140in}}%
\pgfpathlineto{\pgfqpoint{0.630832in}{1.215789in}}%
\pgfpathlineto{\pgfqpoint{0.633603in}{1.225439in}}%
\pgfpathlineto{\pgfqpoint{0.634712in}{1.231446in}}%
\pgfpathlineto{\pgfqpoint{0.637044in}{1.235088in}}%
\pgfpathlineto{\pgfqpoint{0.641470in}{1.244737in}}%
\pgfpathlineto{\pgfqpoint{0.644424in}{1.251891in}}%
\pgfpathlineto{\pgfqpoint{0.646148in}{1.254386in}}%
\pgfpathlineto{\pgfqpoint{0.651275in}{1.264035in}}%
\pgfpathlineto{\pgfqpoint{0.654135in}{1.270713in}}%
\pgfpathlineto{\pgfqpoint{0.656028in}{1.273684in}}%
\pgfpathlineto{\pgfqpoint{0.660678in}{1.283333in}}%
\pgfpathlineto{\pgfqpoint{0.663847in}{1.289572in}}%
\pgfpathlineto{\pgfqpoint{0.666045in}{1.292982in}}%
\pgfpathlineto{\pgfqpoint{0.671441in}{1.302632in}}%
\pgfpathlineto{\pgfqpoint{0.673559in}{1.307828in}}%
\pgfpathlineto{\pgfqpoint{0.675741in}{1.312281in}}%
\pgfpathlineto{\pgfqpoint{0.678536in}{1.321930in}}%
\pgfpathlineto{\pgfqpoint{0.679384in}{1.331579in}}%
\pgfpathlineto{\pgfqpoint{0.677836in}{1.341228in}}%
\pgfpathlineto{\pgfqpoint{0.673559in}{1.348518in}}%
\pgfpathlineto{\pgfqpoint{0.671630in}{1.350877in}}%
\pgfpathlineto{\pgfqpoint{0.663847in}{1.354976in}}%
\pgfpathlineto{\pgfqpoint{0.654135in}{1.356909in}}%
\pgfpathlineto{\pgfqpoint{0.644424in}{1.358928in}}%
\pgfpathlineto{\pgfqpoint{0.642924in}{1.360526in}}%
\pgfpathlineto{\pgfqpoint{0.634712in}{1.362456in}}%
\pgfpathlineto{\pgfqpoint{0.631195in}{1.370175in}}%
\pgfpathlineto{\pgfqpoint{0.629337in}{1.379825in}}%
\pgfpathlineto{\pgfqpoint{0.634246in}{1.389474in}}%
\pgfpathlineto{\pgfqpoint{0.625000in}{1.390010in}}%
\pgfusepath{stroke}%
\end{pgfscope}%
\begin{pgfscope}%
\pgfpathrectangle{\pgfqpoint{0.625000in}{0.550000in}}{\pgfqpoint{3.875000in}{3.850000in}} %
\pgfusepath{clip}%
\pgfsetbuttcap%
\pgfsetroundjoin%
\pgfsetlinewidth{0.250937pt}%
\definecolor{currentstroke}{rgb}{0.000000,0.000000,0.000000}%
\pgfsetstrokecolor{currentstroke}%
\pgfsetdash{}{0pt}%
\pgfpathmoveto{\pgfqpoint{0.625000in}{1.254206in}}%
\pgfpathlineto{\pgfqpoint{0.634595in}{1.244737in}}%
\pgfpathlineto{\pgfqpoint{0.625000in}{1.235529in}}%
\pgfusepath{stroke}%
\end{pgfscope}%
\begin{pgfscope}%
\pgfpathrectangle{\pgfqpoint{0.625000in}{0.550000in}}{\pgfqpoint{3.875000in}{3.850000in}} %
\pgfusepath{clip}%
\pgfsetbuttcap%
\pgfsetroundjoin%
\pgfsetlinewidth{0.250937pt}%
\definecolor{currentstroke}{rgb}{0.000000,0.000000,0.000000}%
\pgfsetstrokecolor{currentstroke}%
\pgfsetdash{}{0pt}%
\pgfpathmoveto{\pgfqpoint{0.625000in}{1.295747in}}%
\pgfpathlineto{\pgfqpoint{0.632729in}{1.292982in}}%
\pgfpathlineto{\pgfqpoint{0.625000in}{1.290659in}}%
\pgfusepath{stroke}%
\end{pgfscope}%
\begin{pgfscope}%
\pgfpathrectangle{\pgfqpoint{0.625000in}{0.550000in}}{\pgfqpoint{3.875000in}{3.850000in}} %
\pgfusepath{clip}%
\pgfsetbuttcap%
\pgfsetroundjoin%
\pgfsetlinewidth{0.250937pt}%
\definecolor{currentstroke}{rgb}{0.000000,0.000000,0.000000}%
\pgfsetstrokecolor{currentstroke}%
\pgfsetdash{}{0pt}%
\pgfpathmoveto{\pgfqpoint{0.625000in}{1.408231in}}%
\pgfpathlineto{\pgfqpoint{0.634574in}{1.399123in}}%
\pgfpathlineto{\pgfqpoint{0.634712in}{1.391127in}}%
\pgfpathlineto{\pgfqpoint{0.644424in}{1.395514in}}%
\pgfpathlineto{\pgfqpoint{0.654135in}{1.397434in}}%
\pgfpathlineto{\pgfqpoint{0.663847in}{1.398385in}}%
\pgfpathlineto{\pgfqpoint{0.673559in}{1.398917in}}%
\pgfpathlineto{\pgfqpoint{0.675202in}{1.399123in}}%
\pgfpathlineto{\pgfqpoint{0.683271in}{1.400352in}}%
\pgfpathlineto{\pgfqpoint{0.692982in}{1.403762in}}%
\pgfpathlineto{\pgfqpoint{0.701504in}{1.408772in}}%
\pgfpathlineto{\pgfqpoint{0.702694in}{1.409521in}}%
\pgfpathlineto{\pgfqpoint{0.712406in}{1.417147in}}%
\pgfpathlineto{\pgfqpoint{0.713785in}{1.418421in}}%
\pgfpathlineto{\pgfqpoint{0.722118in}{1.426771in}}%
\pgfpathlineto{\pgfqpoint{0.723282in}{1.428070in}}%
\pgfpathlineto{\pgfqpoint{0.731051in}{1.437719in}}%
\pgfpathlineto{\pgfqpoint{0.731830in}{1.438827in}}%
\pgfpathlineto{\pgfqpoint{0.737525in}{1.447368in}}%
\pgfpathlineto{\pgfqpoint{0.741541in}{1.454641in}}%
\pgfpathlineto{\pgfqpoint{0.742823in}{1.457018in}}%
\pgfpathlineto{\pgfqpoint{0.747194in}{1.466667in}}%
\pgfpathlineto{\pgfqpoint{0.750581in}{1.476316in}}%
\pgfpathlineto{\pgfqpoint{0.751253in}{1.478825in}}%
\pgfpathlineto{\pgfqpoint{0.753221in}{1.485965in}}%
\pgfpathlineto{\pgfqpoint{0.755079in}{1.495614in}}%
\pgfpathlineto{\pgfqpoint{0.756168in}{1.505263in}}%
\pgfpathlineto{\pgfqpoint{0.756524in}{1.514912in}}%
\pgfpathlineto{\pgfqpoint{0.756160in}{1.524561in}}%
\pgfpathlineto{\pgfqpoint{0.755068in}{1.534211in}}%
\pgfpathlineto{\pgfqpoint{0.753213in}{1.543860in}}%
\pgfpathlineto{\pgfqpoint{0.751253in}{1.551007in}}%
\pgfpathlineto{\pgfqpoint{0.750585in}{1.553509in}}%
\pgfpathlineto{\pgfqpoint{0.747214in}{1.563158in}}%
\pgfpathlineto{\pgfqpoint{0.742856in}{1.572807in}}%
\pgfpathlineto{\pgfqpoint{0.741541in}{1.575242in}}%
\pgfpathlineto{\pgfqpoint{0.737544in}{1.582456in}}%
\pgfpathlineto{\pgfqpoint{0.731830in}{1.590927in}}%
\pgfpathlineto{\pgfqpoint{0.730993in}{1.592105in}}%
\pgfpathlineto{\pgfqpoint{0.723048in}{1.601754in}}%
\pgfpathlineto{\pgfqpoint{0.722118in}{1.602750in}}%
\pgfpathlineto{\pgfqpoint{0.713336in}{1.611404in}}%
\pgfpathlineto{\pgfqpoint{0.712406in}{1.612226in}}%
\pgfpathlineto{\pgfqpoint{0.702694in}{1.620200in}}%
\pgfpathlineto{\pgfqpoint{0.701578in}{1.621053in}}%
\pgfpathlineto{\pgfqpoint{0.692982in}{1.627392in}}%
\pgfpathlineto{\pgfqpoint{0.688668in}{1.630702in}}%
\pgfpathlineto{\pgfqpoint{0.683271in}{1.634873in}}%
\pgfpathlineto{\pgfqpoint{0.678217in}{1.640351in}}%
\pgfpathlineto{\pgfqpoint{0.673559in}{1.645020in}}%
\pgfpathlineto{\pgfqpoint{0.670865in}{1.650000in}}%
\pgfpathlineto{\pgfqpoint{0.663847in}{1.657853in}}%
\pgfpathlineto{\pgfqpoint{0.662960in}{1.659649in}}%
\pgfpathlineto{\pgfqpoint{0.654135in}{1.667779in}}%
\pgfpathlineto{\pgfqpoint{0.653101in}{1.669298in}}%
\pgfpathlineto{\pgfqpoint{0.644424in}{1.676020in}}%
\pgfpathlineto{\pgfqpoint{0.642752in}{1.678947in}}%
\pgfpathlineto{\pgfqpoint{0.634712in}{1.686059in}}%
\pgfpathlineto{\pgfqpoint{0.633884in}{1.688596in}}%
\pgfpathlineto{\pgfqpoint{0.629588in}{1.698246in}}%
\pgfpathlineto{\pgfqpoint{0.625000in}{1.698764in}}%
\pgfusepath{stroke}%
\end{pgfscope}%
\begin{pgfscope}%
\pgfpathrectangle{\pgfqpoint{0.625000in}{0.550000in}}{\pgfqpoint{3.875000in}{3.850000in}} %
\pgfusepath{clip}%
\pgfsetbuttcap%
\pgfsetroundjoin%
\pgfsetlinewidth{0.250937pt}%
\definecolor{currentstroke}{rgb}{0.000000,0.000000,0.000000}%
\pgfsetstrokecolor{currentstroke}%
\pgfsetdash{}{0pt}%
\pgfpathmoveto{\pgfqpoint{0.625000in}{1.562572in}}%
\pgfpathlineto{\pgfqpoint{0.634512in}{1.553509in}}%
\pgfpathlineto{\pgfqpoint{0.625000in}{1.544151in}}%
\pgfusepath{stroke}%
\end{pgfscope}%
\begin{pgfscope}%
\pgfpathrectangle{\pgfqpoint{0.625000in}{0.550000in}}{\pgfqpoint{3.875000in}{3.850000in}} %
\pgfusepath{clip}%
\pgfsetbuttcap%
\pgfsetroundjoin%
\pgfsetlinewidth{0.250937pt}%
\definecolor{currentstroke}{rgb}{0.000000,0.000000,0.000000}%
\pgfsetstrokecolor{currentstroke}%
\pgfsetdash{}{0pt}%
\pgfpathmoveto{\pgfqpoint{0.625000in}{1.716968in}}%
\pgfpathlineto{\pgfqpoint{0.629858in}{1.717544in}}%
\pgfpathlineto{\pgfqpoint{0.633884in}{1.727193in}}%
\pgfpathlineto{\pgfqpoint{0.634712in}{1.729757in}}%
\pgfpathlineto{\pgfqpoint{0.642742in}{1.736842in}}%
\pgfpathlineto{\pgfqpoint{0.644424in}{1.739702in}}%
\pgfpathlineto{\pgfqpoint{0.652998in}{1.746491in}}%
\pgfpathlineto{\pgfqpoint{0.654135in}{1.748378in}}%
\pgfpathlineto{\pgfqpoint{0.662968in}{1.756140in}}%
\pgfpathlineto{\pgfqpoint{0.663847in}{1.757458in}}%
\pgfpathlineto{\pgfqpoint{0.673559in}{1.765617in}}%
\pgfpathlineto{\pgfqpoint{0.673836in}{1.765789in}}%
\pgfpathlineto{\pgfqpoint{0.683271in}{1.773569in}}%
\pgfpathlineto{\pgfqpoint{0.685651in}{1.775439in}}%
\pgfpathlineto{\pgfqpoint{0.692982in}{1.782869in}}%
\pgfpathlineto{\pgfqpoint{0.695067in}{1.785088in}}%
\pgfpathlineto{\pgfqpoint{0.701868in}{1.794737in}}%
\pgfpathlineto{\pgfqpoint{0.702694in}{1.796349in}}%
\pgfpathlineto{\pgfqpoint{0.706751in}{1.804386in}}%
\pgfpathlineto{\pgfqpoint{0.709949in}{1.814035in}}%
\pgfpathlineto{\pgfqpoint{0.711792in}{1.823684in}}%
\pgfpathlineto{\pgfqpoint{0.712406in}{1.831777in}}%
\pgfpathlineto{\pgfqpoint{0.712534in}{1.833333in}}%
\pgfpathlineto{\pgfqpoint{0.712406in}{1.837483in}}%
\pgfpathlineto{\pgfqpoint{0.712254in}{1.842982in}}%
\pgfpathlineto{\pgfqpoint{0.711013in}{1.852632in}}%
\pgfpathlineto{\pgfqpoint{0.708731in}{1.862281in}}%
\pgfpathlineto{\pgfqpoint{0.705281in}{1.871930in}}%
\pgfpathlineto{\pgfqpoint{0.702694in}{1.877376in}}%
\pgfpathlineto{\pgfqpoint{0.700564in}{1.881579in}}%
\pgfpathlineto{\pgfqpoint{0.694101in}{1.891228in}}%
\pgfpathlineto{\pgfqpoint{0.692982in}{1.892642in}}%
\pgfpathlineto{\pgfqpoint{0.684165in}{1.900877in}}%
\pgfpathlineto{\pgfqpoint{0.683271in}{1.901576in}}%
\pgfpathlineto{\pgfqpoint{0.673559in}{1.906039in}}%
\pgfpathlineto{\pgfqpoint{0.663847in}{1.907388in}}%
\pgfpathlineto{\pgfqpoint{0.654135in}{1.908118in}}%
\pgfpathlineto{\pgfqpoint{0.645877in}{1.910526in}}%
\pgfpathlineto{\pgfqpoint{0.644424in}{1.910673in}}%
\pgfpathlineto{\pgfqpoint{0.634712in}{1.913504in}}%
\pgfpathlineto{\pgfqpoint{0.631592in}{1.920175in}}%
\pgfpathlineto{\pgfqpoint{0.625000in}{1.927200in}}%
\pgfusepath{stroke}%
\end{pgfscope}%
\begin{pgfscope}%
\pgfpathrectangle{\pgfqpoint{0.625000in}{0.550000in}}{\pgfqpoint{3.875000in}{3.850000in}} %
\pgfusepath{clip}%
\pgfsetbuttcap%
\pgfsetroundjoin%
\pgfsetlinewidth{0.250937pt}%
\definecolor{currentstroke}{rgb}{0.000000,0.000000,0.000000}%
\pgfsetstrokecolor{currentstroke}%
\pgfsetdash{}{0pt}%
\pgfpathmoveto{\pgfqpoint{0.625000in}{1.871447in}}%
\pgfpathlineto{\pgfqpoint{0.634513in}{1.862281in}}%
\pgfpathlineto{\pgfqpoint{0.625000in}{1.853304in}}%
\pgfusepath{stroke}%
\end{pgfscope}%
\begin{pgfscope}%
\pgfpathrectangle{\pgfqpoint{0.625000in}{0.550000in}}{\pgfqpoint{3.875000in}{3.850000in}} %
\pgfusepath{clip}%
\pgfsetbuttcap%
\pgfsetroundjoin%
\pgfsetlinewidth{0.250937pt}%
\definecolor{currentstroke}{rgb}{0.000000,0.000000,0.000000}%
\pgfsetstrokecolor{currentstroke}%
\pgfsetdash{}{0pt}%
\pgfpathmoveto{\pgfqpoint{0.625000in}{1.893207in}}%
\pgfpathlineto{\pgfqpoint{0.631769in}{1.891228in}}%
\pgfpathlineto{\pgfqpoint{0.625000in}{1.888073in}}%
\pgfusepath{stroke}%
\end{pgfscope}%
\begin{pgfscope}%
\pgfpathrectangle{\pgfqpoint{0.625000in}{0.550000in}}{\pgfqpoint{3.875000in}{3.850000in}} %
\pgfusepath{clip}%
\pgfsetbuttcap%
\pgfsetroundjoin%
\pgfsetlinewidth{0.250937pt}%
\definecolor{currentstroke}{rgb}{0.000000,0.000000,0.000000}%
\pgfsetstrokecolor{currentstroke}%
\pgfsetdash{}{0pt}%
\pgfpathmoveto{\pgfqpoint{0.625000in}{1.931717in}}%
\pgfpathlineto{\pgfqpoint{0.633710in}{1.939474in}}%
\pgfpathlineto{\pgfqpoint{0.634712in}{1.943133in}}%
\pgfpathlineto{\pgfqpoint{0.644424in}{1.946761in}}%
\pgfpathlineto{\pgfqpoint{0.654135in}{1.948362in}}%
\pgfpathlineto{\pgfqpoint{0.660912in}{1.949123in}}%
\pgfpathlineto{\pgfqpoint{0.663847in}{1.949757in}}%
\pgfpathlineto{\pgfqpoint{0.673559in}{1.953172in}}%
\pgfpathlineto{\pgfqpoint{0.680000in}{1.958772in}}%
\pgfpathlineto{\pgfqpoint{0.683271in}{1.963650in}}%
\pgfpathlineto{\pgfqpoint{0.685745in}{1.968421in}}%
\pgfpathlineto{\pgfqpoint{0.688803in}{1.978070in}}%
\pgfpathlineto{\pgfqpoint{0.690183in}{1.987719in}}%
\pgfpathlineto{\pgfqpoint{0.690304in}{1.997368in}}%
\pgfpathlineto{\pgfqpoint{0.689323in}{2.007018in}}%
\pgfpathlineto{\pgfqpoint{0.687276in}{2.016667in}}%
\pgfpathlineto{\pgfqpoint{0.684239in}{2.026316in}}%
\pgfpathlineto{\pgfqpoint{0.683271in}{2.029009in}}%
\pgfpathlineto{\pgfqpoint{0.680402in}{2.035965in}}%
\pgfpathlineto{\pgfqpoint{0.676417in}{2.045614in}}%
\pgfpathlineto{\pgfqpoint{0.674403in}{2.055263in}}%
\pgfpathlineto{\pgfqpoint{0.674865in}{2.064912in}}%
\pgfpathlineto{\pgfqpoint{0.676019in}{2.074561in}}%
\pgfpathlineto{\pgfqpoint{0.676604in}{2.084211in}}%
\pgfpathlineto{\pgfqpoint{0.676243in}{2.093860in}}%
\pgfpathlineto{\pgfqpoint{0.674895in}{2.103509in}}%
\pgfpathlineto{\pgfqpoint{0.673559in}{2.109518in}}%
\pgfpathlineto{\pgfqpoint{0.672757in}{2.113158in}}%
\pgfpathlineto{\pgfqpoint{0.670187in}{2.122807in}}%
\pgfpathlineto{\pgfqpoint{0.668222in}{2.132456in}}%
\pgfpathlineto{\pgfqpoint{0.667471in}{2.142105in}}%
\pgfpathlineto{\pgfqpoint{0.667015in}{2.151754in}}%
\pgfpathlineto{\pgfqpoint{0.665992in}{2.161404in}}%
\pgfpathlineto{\pgfqpoint{0.664234in}{2.171053in}}%
\pgfpathlineto{\pgfqpoint{0.663847in}{2.172892in}}%
\pgfpathlineto{\pgfqpoint{0.662174in}{2.180702in}}%
\pgfpathlineto{\pgfqpoint{0.660668in}{2.190351in}}%
\pgfpathlineto{\pgfqpoint{0.659724in}{2.200000in}}%
\pgfpathlineto{\pgfqpoint{0.658375in}{2.209649in}}%
\pgfpathlineto{\pgfqpoint{0.656523in}{2.219298in}}%
\pgfpathlineto{\pgfqpoint{0.654956in}{2.228947in}}%
\pgfpathlineto{\pgfqpoint{0.654135in}{2.234809in}}%
\pgfpathlineto{\pgfqpoint{0.653689in}{2.238596in}}%
\pgfpathlineto{\pgfqpoint{0.652083in}{2.248246in}}%
\pgfpathlineto{\pgfqpoint{0.650490in}{2.257895in}}%
\pgfpathlineto{\pgfqpoint{0.649180in}{2.267544in}}%
\pgfpathlineto{\pgfqpoint{0.647516in}{2.277193in}}%
\pgfpathlineto{\pgfqpoint{0.646112in}{2.286842in}}%
\pgfpathlineto{\pgfqpoint{0.644558in}{2.296491in}}%
\pgfpathlineto{\pgfqpoint{0.644424in}{2.297295in}}%
\pgfpathlineto{\pgfqpoint{0.643181in}{2.306140in}}%
\pgfpathlineto{\pgfqpoint{0.641680in}{2.315789in}}%
\pgfpathlineto{\pgfqpoint{0.640227in}{2.325439in}}%
\pgfpathlineto{\pgfqpoint{0.638879in}{2.335088in}}%
\pgfpathlineto{\pgfqpoint{0.637516in}{2.344737in}}%
\pgfpathlineto{\pgfqpoint{0.636166in}{2.354386in}}%
\pgfpathlineto{\pgfqpoint{0.634848in}{2.364035in}}%
\pgfpathlineto{\pgfqpoint{0.634712in}{2.364914in}}%
\pgfpathlineto{\pgfqpoint{0.633381in}{2.373684in}}%
\pgfpathlineto{\pgfqpoint{0.633272in}{2.383333in}}%
\pgfpathlineto{\pgfqpoint{0.631923in}{2.392982in}}%
\pgfpathlineto{\pgfqpoint{0.631225in}{2.402632in}}%
\pgfpathlineto{\pgfqpoint{0.627252in}{2.412281in}}%
\pgfpathlineto{\pgfqpoint{0.628079in}{2.421930in}}%
\pgfpathlineto{\pgfqpoint{0.627407in}{2.431579in}}%
\pgfpathlineto{\pgfqpoint{0.626731in}{2.441228in}}%
\pgfpathlineto{\pgfqpoint{0.626097in}{2.450877in}}%
\pgfpathlineto{\pgfqpoint{0.625412in}{2.460526in}}%
\pgfpathlineto{\pgfqpoint{0.625000in}{2.465984in}}%
\pgfusepath{stroke}%
\end{pgfscope}%
\begin{pgfscope}%
\pgfpathrectangle{\pgfqpoint{0.625000in}{0.550000in}}{\pgfqpoint{3.875000in}{3.850000in}} %
\pgfusepath{clip}%
\pgfsetbuttcap%
\pgfsetroundjoin%
\pgfsetlinewidth{0.250937pt}%
\definecolor{currentstroke}{rgb}{0.000000,0.000000,0.000000}%
\pgfsetstrokecolor{currentstroke}%
\pgfsetdash{}{0pt}%
\pgfpathmoveto{\pgfqpoint{0.625000in}{2.026176in}}%
\pgfpathlineto{\pgfqpoint{0.634518in}{2.016667in}}%
\pgfpathlineto{\pgfqpoint{0.625000in}{2.007235in}}%
\pgfusepath{stroke}%
\end{pgfscope}%
\begin{pgfscope}%
\pgfpathrectangle{\pgfqpoint{0.625000in}{0.550000in}}{\pgfqpoint{3.875000in}{3.850000in}} %
\pgfusepath{clip}%
\pgfsetbuttcap%
\pgfsetroundjoin%
\pgfsetlinewidth{0.250937pt}%
\definecolor{currentstroke}{rgb}{0.000000,0.000000,0.000000}%
\pgfsetstrokecolor{currentstroke}%
\pgfsetdash{}{0pt}%
\pgfpathmoveto{\pgfqpoint{0.625000in}{2.180072in}}%
\pgfpathlineto{\pgfqpoint{0.634539in}{2.171053in}}%
\pgfpathlineto{\pgfqpoint{0.625000in}{2.161775in}}%
\pgfusepath{stroke}%
\end{pgfscope}%
\begin{pgfscope}%
\pgfpathrectangle{\pgfqpoint{0.625000in}{0.550000in}}{\pgfqpoint{3.875000in}{3.850000in}} %
\pgfusepath{clip}%
\pgfsetbuttcap%
\pgfsetroundjoin%
\pgfsetlinewidth{0.250937pt}%
\definecolor{currentstroke}{rgb}{0.000000,0.000000,0.000000}%
\pgfsetstrokecolor{currentstroke}%
\pgfsetdash{}{0pt}%
\pgfpathmoveto{\pgfqpoint{0.625000in}{2.334424in}}%
\pgfpathlineto{\pgfqpoint{0.634593in}{2.325439in}}%
\pgfpathlineto{\pgfqpoint{0.625000in}{2.315824in}}%
\pgfusepath{stroke}%
\end{pgfscope}%
\begin{pgfscope}%
\pgfpathrectangle{\pgfqpoint{0.625000in}{0.550000in}}{\pgfqpoint{3.875000in}{3.850000in}} %
\pgfusepath{clip}%
\pgfsetbuttcap%
\pgfsetroundjoin%
\pgfsetlinewidth{0.250937pt}%
\definecolor{currentstroke}{rgb}{0.000000,0.000000,0.000000}%
\pgfsetstrokecolor{currentstroke}%
\pgfsetdash{}{0pt}%
\pgfpathmoveto{\pgfqpoint{0.625000in}{2.493665in}}%
\pgfpathlineto{\pgfqpoint{0.625411in}{2.499123in}}%
\pgfpathlineto{\pgfqpoint{0.626097in}{2.508772in}}%
\pgfpathlineto{\pgfqpoint{0.626731in}{2.518421in}}%
\pgfpathlineto{\pgfqpoint{0.627407in}{2.528070in}}%
\pgfpathlineto{\pgfqpoint{0.628079in}{2.537719in}}%
\pgfpathlineto{\pgfqpoint{0.627252in}{2.547368in}}%
\pgfpathlineto{\pgfqpoint{0.631225in}{2.557018in}}%
\pgfpathlineto{\pgfqpoint{0.631923in}{2.566667in}}%
\pgfpathlineto{\pgfqpoint{0.633272in}{2.576316in}}%
\pgfpathlineto{\pgfqpoint{0.633381in}{2.585965in}}%
\pgfpathlineto{\pgfqpoint{0.634712in}{2.594735in}}%
\pgfpathlineto{\pgfqpoint{0.634848in}{2.595614in}}%
\pgfpathlineto{\pgfqpoint{0.636166in}{2.605263in}}%
\pgfpathlineto{\pgfqpoint{0.637516in}{2.614912in}}%
\pgfpathlineto{\pgfqpoint{0.638879in}{2.624561in}}%
\pgfpathlineto{\pgfqpoint{0.640227in}{2.634211in}}%
\pgfpathlineto{\pgfqpoint{0.641680in}{2.643860in}}%
\pgfpathlineto{\pgfqpoint{0.643181in}{2.653509in}}%
\pgfpathlineto{\pgfqpoint{0.644424in}{2.662354in}}%
\pgfpathlineto{\pgfqpoint{0.644558in}{2.663158in}}%
\pgfpathlineto{\pgfqpoint{0.646112in}{2.672807in}}%
\pgfpathlineto{\pgfqpoint{0.647516in}{2.682456in}}%
\pgfpathlineto{\pgfqpoint{0.649180in}{2.692105in}}%
\pgfpathlineto{\pgfqpoint{0.650490in}{2.701754in}}%
\pgfpathlineto{\pgfqpoint{0.652083in}{2.711404in}}%
\pgfpathlineto{\pgfqpoint{0.653689in}{2.721053in}}%
\pgfpathlineto{\pgfqpoint{0.654135in}{2.724840in}}%
\pgfpathlineto{\pgfqpoint{0.654956in}{2.730702in}}%
\pgfpathlineto{\pgfqpoint{0.656523in}{2.740351in}}%
\pgfpathlineto{\pgfqpoint{0.658375in}{2.750000in}}%
\pgfpathlineto{\pgfqpoint{0.659724in}{2.759649in}}%
\pgfpathlineto{\pgfqpoint{0.660668in}{2.769298in}}%
\pgfpathlineto{\pgfqpoint{0.662174in}{2.778947in}}%
\pgfpathlineto{\pgfqpoint{0.663847in}{2.786757in}}%
\pgfpathlineto{\pgfqpoint{0.664234in}{2.788596in}}%
\pgfpathlineto{\pgfqpoint{0.665992in}{2.798246in}}%
\pgfpathlineto{\pgfqpoint{0.667015in}{2.807895in}}%
\pgfpathlineto{\pgfqpoint{0.667471in}{2.817544in}}%
\pgfpathlineto{\pgfqpoint{0.668222in}{2.827193in}}%
\pgfpathlineto{\pgfqpoint{0.670187in}{2.836842in}}%
\pgfpathlineto{\pgfqpoint{0.672757in}{2.846491in}}%
\pgfpathlineto{\pgfqpoint{0.673559in}{2.850131in}}%
\pgfpathlineto{\pgfqpoint{0.674895in}{2.856140in}}%
\pgfpathlineto{\pgfqpoint{0.676243in}{2.865789in}}%
\pgfpathlineto{\pgfqpoint{0.676604in}{2.875439in}}%
\pgfpathlineto{\pgfqpoint{0.676019in}{2.885088in}}%
\pgfpathlineto{\pgfqpoint{0.674865in}{2.894737in}}%
\pgfpathlineto{\pgfqpoint{0.674403in}{2.904386in}}%
\pgfpathlineto{\pgfqpoint{0.676417in}{2.914035in}}%
\pgfpathlineto{\pgfqpoint{0.680402in}{2.923684in}}%
\pgfpathlineto{\pgfqpoint{0.683271in}{2.930640in}}%
\pgfpathlineto{\pgfqpoint{0.684239in}{2.933333in}}%
\pgfpathlineto{\pgfqpoint{0.687276in}{2.942982in}}%
\pgfpathlineto{\pgfqpoint{0.689323in}{2.952632in}}%
\pgfpathlineto{\pgfqpoint{0.690304in}{2.962281in}}%
\pgfpathlineto{\pgfqpoint{0.690183in}{2.971930in}}%
\pgfpathlineto{\pgfqpoint{0.688803in}{2.981579in}}%
\pgfpathlineto{\pgfqpoint{0.685745in}{2.991228in}}%
\pgfpathlineto{\pgfqpoint{0.683271in}{2.995999in}}%
\pgfpathlineto{\pgfqpoint{0.680000in}{3.000877in}}%
\pgfpathlineto{\pgfqpoint{0.673559in}{3.006477in}}%
\pgfpathlineto{\pgfqpoint{0.663847in}{3.009892in}}%
\pgfpathlineto{\pgfqpoint{0.660912in}{3.010526in}}%
\pgfpathlineto{\pgfqpoint{0.654135in}{3.011287in}}%
\pgfpathlineto{\pgfqpoint{0.644424in}{3.012888in}}%
\pgfpathlineto{\pgfqpoint{0.634712in}{3.016516in}}%
\pgfpathlineto{\pgfqpoint{0.633710in}{3.020175in}}%
\pgfpathlineto{\pgfqpoint{0.625000in}{3.027933in}}%
\pgfusepath{stroke}%
\end{pgfscope}%
\begin{pgfscope}%
\pgfpathrectangle{\pgfqpoint{0.625000in}{0.550000in}}{\pgfqpoint{3.875000in}{3.850000in}} %
\pgfusepath{clip}%
\pgfsetbuttcap%
\pgfsetroundjoin%
\pgfsetlinewidth{0.250937pt}%
\definecolor{currentstroke}{rgb}{0.000000,0.000000,0.000000}%
\pgfsetstrokecolor{currentstroke}%
\pgfsetdash{}{0pt}%
\pgfpathmoveto{\pgfqpoint{0.625000in}{2.643826in}}%
\pgfpathlineto{\pgfqpoint{0.634593in}{2.634211in}}%
\pgfpathlineto{\pgfqpoint{0.625000in}{2.625226in}}%
\pgfusepath{stroke}%
\end{pgfscope}%
\begin{pgfscope}%
\pgfpathrectangle{\pgfqpoint{0.625000in}{0.550000in}}{\pgfqpoint{3.875000in}{3.850000in}} %
\pgfusepath{clip}%
\pgfsetbuttcap%
\pgfsetroundjoin%
\pgfsetlinewidth{0.250937pt}%
\definecolor{currentstroke}{rgb}{0.000000,0.000000,0.000000}%
\pgfsetstrokecolor{currentstroke}%
\pgfsetdash{}{0pt}%
\pgfpathmoveto{\pgfqpoint{0.625000in}{2.797872in}}%
\pgfpathlineto{\pgfqpoint{0.634537in}{2.788596in}}%
\pgfpathlineto{\pgfqpoint{0.625000in}{2.779580in}}%
\pgfusepath{stroke}%
\end{pgfscope}%
\begin{pgfscope}%
\pgfpathrectangle{\pgfqpoint{0.625000in}{0.550000in}}{\pgfqpoint{3.875000in}{3.850000in}} %
\pgfusepath{clip}%
\pgfsetbuttcap%
\pgfsetroundjoin%
\pgfsetlinewidth{0.250937pt}%
\definecolor{currentstroke}{rgb}{0.000000,0.000000,0.000000}%
\pgfsetstrokecolor{currentstroke}%
\pgfsetdash{}{0pt}%
\pgfpathmoveto{\pgfqpoint{0.625000in}{2.952410in}}%
\pgfpathlineto{\pgfqpoint{0.634515in}{2.942982in}}%
\pgfpathlineto{\pgfqpoint{0.625000in}{2.933476in}}%
\pgfusepath{stroke}%
\end{pgfscope}%
\begin{pgfscope}%
\pgfpathrectangle{\pgfqpoint{0.625000in}{0.550000in}}{\pgfqpoint{3.875000in}{3.850000in}} %
\pgfusepath{clip}%
\pgfsetbuttcap%
\pgfsetroundjoin%
\pgfsetlinewidth{0.250937pt}%
\definecolor{currentstroke}{rgb}{0.000000,0.000000,0.000000}%
\pgfsetstrokecolor{currentstroke}%
\pgfsetdash{}{0pt}%
\pgfpathmoveto{\pgfqpoint{0.625000in}{3.032449in}}%
\pgfpathlineto{\pgfqpoint{0.631592in}{3.039474in}}%
\pgfpathlineto{\pgfqpoint{0.634712in}{3.046145in}}%
\pgfpathlineto{\pgfqpoint{0.644424in}{3.048976in}}%
\pgfpathlineto{\pgfqpoint{0.645877in}{3.049123in}}%
\pgfpathlineto{\pgfqpoint{0.654135in}{3.051531in}}%
\pgfpathlineto{\pgfqpoint{0.663847in}{3.052261in}}%
\pgfpathlineto{\pgfqpoint{0.673559in}{3.053610in}}%
\pgfpathlineto{\pgfqpoint{0.683271in}{3.058073in}}%
\pgfpathlineto{\pgfqpoint{0.684165in}{3.058772in}}%
\pgfpathlineto{\pgfqpoint{0.692982in}{3.067007in}}%
\pgfpathlineto{\pgfqpoint{0.694101in}{3.068421in}}%
\pgfpathlineto{\pgfqpoint{0.700564in}{3.078070in}}%
\pgfpathlineto{\pgfqpoint{0.702694in}{3.082273in}}%
\pgfpathlineto{\pgfqpoint{0.705281in}{3.087719in}}%
\pgfpathlineto{\pgfqpoint{0.708731in}{3.097368in}}%
\pgfpathlineto{\pgfqpoint{0.711013in}{3.107018in}}%
\pgfpathlineto{\pgfqpoint{0.712254in}{3.116667in}}%
\pgfpathlineto{\pgfqpoint{0.712406in}{3.122166in}}%
\pgfpathlineto{\pgfqpoint{0.712534in}{3.126316in}}%
\pgfpathlineto{\pgfqpoint{0.712406in}{3.127872in}}%
\pgfpathlineto{\pgfqpoint{0.711792in}{3.135965in}}%
\pgfpathlineto{\pgfqpoint{0.709949in}{3.145614in}}%
\pgfpathlineto{\pgfqpoint{0.706751in}{3.155263in}}%
\pgfpathlineto{\pgfqpoint{0.702694in}{3.163301in}}%
\pgfpathlineto{\pgfqpoint{0.701868in}{3.164912in}}%
\pgfpathlineto{\pgfqpoint{0.695067in}{3.174561in}}%
\pgfpathlineto{\pgfqpoint{0.692982in}{3.176780in}}%
\pgfpathlineto{\pgfqpoint{0.685651in}{3.184211in}}%
\pgfpathlineto{\pgfqpoint{0.683271in}{3.186080in}}%
\pgfpathlineto{\pgfqpoint{0.673836in}{3.193860in}}%
\pgfpathlineto{\pgfqpoint{0.673559in}{3.194032in}}%
\pgfpathlineto{\pgfqpoint{0.663847in}{3.202191in}}%
\pgfpathlineto{\pgfqpoint{0.662968in}{3.203509in}}%
\pgfpathlineto{\pgfqpoint{0.654135in}{3.211271in}}%
\pgfpathlineto{\pgfqpoint{0.652998in}{3.213158in}}%
\pgfpathlineto{\pgfqpoint{0.644424in}{3.219947in}}%
\pgfpathlineto{\pgfqpoint{0.642742in}{3.222807in}}%
\pgfpathlineto{\pgfqpoint{0.634712in}{3.229893in}}%
\pgfpathlineto{\pgfqpoint{0.633884in}{3.232456in}}%
\pgfpathlineto{\pgfqpoint{0.629858in}{3.242105in}}%
\pgfpathlineto{\pgfqpoint{0.625000in}{3.242709in}}%
\pgfusepath{stroke}%
\end{pgfscope}%
\begin{pgfscope}%
\pgfpathrectangle{\pgfqpoint{0.625000in}{0.550000in}}{\pgfqpoint{3.875000in}{3.850000in}} %
\pgfusepath{clip}%
\pgfsetbuttcap%
\pgfsetroundjoin%
\pgfsetlinewidth{0.250937pt}%
\definecolor{currentstroke}{rgb}{0.000000,0.000000,0.000000}%
\pgfsetstrokecolor{currentstroke}%
\pgfsetdash{}{0pt}%
\pgfpathmoveto{\pgfqpoint{0.625000in}{3.071577in}}%
\pgfpathlineto{\pgfqpoint{0.631769in}{3.068421in}}%
\pgfpathlineto{\pgfqpoint{0.625000in}{3.066442in}}%
\pgfusepath{stroke}%
\end{pgfscope}%
\begin{pgfscope}%
\pgfpathrectangle{\pgfqpoint{0.625000in}{0.550000in}}{\pgfqpoint{3.875000in}{3.850000in}} %
\pgfusepath{clip}%
\pgfsetbuttcap%
\pgfsetroundjoin%
\pgfsetlinewidth{0.250937pt}%
\definecolor{currentstroke}{rgb}{0.000000,0.000000,0.000000}%
\pgfsetstrokecolor{currentstroke}%
\pgfsetdash{}{0pt}%
\pgfpathmoveto{\pgfqpoint{0.625000in}{3.106336in}}%
\pgfpathlineto{\pgfqpoint{0.634510in}{3.097368in}}%
\pgfpathlineto{\pgfqpoint{0.625000in}{3.088209in}}%
\pgfusepath{stroke}%
\end{pgfscope}%
\begin{pgfscope}%
\pgfpathrectangle{\pgfqpoint{0.625000in}{0.550000in}}{\pgfqpoint{3.875000in}{3.850000in}} %
\pgfusepath{clip}%
\pgfsetbuttcap%
\pgfsetroundjoin%
\pgfsetlinewidth{0.250937pt}%
\definecolor{currentstroke}{rgb}{0.000000,0.000000,0.000000}%
\pgfsetstrokecolor{currentstroke}%
\pgfsetdash{}{0pt}%
\pgfpathmoveto{\pgfqpoint{0.625000in}{3.260860in}}%
\pgfpathlineto{\pgfqpoint{0.629588in}{3.261404in}}%
\pgfpathlineto{\pgfqpoint{0.633884in}{3.271053in}}%
\pgfpathlineto{\pgfqpoint{0.634712in}{3.273590in}}%
\pgfpathlineto{\pgfqpoint{0.642752in}{3.280702in}}%
\pgfpathlineto{\pgfqpoint{0.644424in}{3.283629in}}%
\pgfpathlineto{\pgfqpoint{0.653101in}{3.290351in}}%
\pgfpathlineto{\pgfqpoint{0.654135in}{3.291870in}}%
\pgfpathlineto{\pgfqpoint{0.662960in}{3.300000in}}%
\pgfpathlineto{\pgfqpoint{0.663847in}{3.301796in}}%
\pgfpathlineto{\pgfqpoint{0.670865in}{3.309649in}}%
\pgfpathlineto{\pgfqpoint{0.673559in}{3.314629in}}%
\pgfpathlineto{\pgfqpoint{0.678217in}{3.319298in}}%
\pgfpathlineto{\pgfqpoint{0.683271in}{3.324776in}}%
\pgfpathlineto{\pgfqpoint{0.688668in}{3.328947in}}%
\pgfpathlineto{\pgfqpoint{0.692982in}{3.332257in}}%
\pgfpathlineto{\pgfqpoint{0.701578in}{3.338596in}}%
\pgfpathlineto{\pgfqpoint{0.702694in}{3.339449in}}%
\pgfpathlineto{\pgfqpoint{0.712406in}{3.347423in}}%
\pgfpathlineto{\pgfqpoint{0.713336in}{3.348246in}}%
\pgfpathlineto{\pgfqpoint{0.722118in}{3.356899in}}%
\pgfpathlineto{\pgfqpoint{0.723048in}{3.357895in}}%
\pgfpathlineto{\pgfqpoint{0.730993in}{3.367544in}}%
\pgfpathlineto{\pgfqpoint{0.731830in}{3.368722in}}%
\pgfpathlineto{\pgfqpoint{0.737544in}{3.377193in}}%
\pgfpathlineto{\pgfqpoint{0.741541in}{3.384408in}}%
\pgfpathlineto{\pgfqpoint{0.742856in}{3.386842in}}%
\pgfpathlineto{\pgfqpoint{0.747214in}{3.396491in}}%
\pgfpathlineto{\pgfqpoint{0.750585in}{3.406140in}}%
\pgfpathlineto{\pgfqpoint{0.751253in}{3.408642in}}%
\pgfpathlineto{\pgfqpoint{0.753213in}{3.415789in}}%
\pgfpathlineto{\pgfqpoint{0.755068in}{3.425439in}}%
\pgfpathlineto{\pgfqpoint{0.756160in}{3.435088in}}%
\pgfpathlineto{\pgfqpoint{0.756524in}{3.444737in}}%
\pgfpathlineto{\pgfqpoint{0.756168in}{3.454386in}}%
\pgfpathlineto{\pgfqpoint{0.755079in}{3.464035in}}%
\pgfpathlineto{\pgfqpoint{0.753221in}{3.473684in}}%
\pgfpathlineto{\pgfqpoint{0.751253in}{3.480824in}}%
\pgfpathlineto{\pgfqpoint{0.750581in}{3.483333in}}%
\pgfpathlineto{\pgfqpoint{0.747194in}{3.492982in}}%
\pgfpathlineto{\pgfqpoint{0.742823in}{3.502632in}}%
\pgfpathlineto{\pgfqpoint{0.741541in}{3.505008in}}%
\pgfpathlineto{\pgfqpoint{0.737525in}{3.512281in}}%
\pgfpathlineto{\pgfqpoint{0.731830in}{3.520822in}}%
\pgfpathlineto{\pgfqpoint{0.731051in}{3.521930in}}%
\pgfpathlineto{\pgfqpoint{0.723282in}{3.531579in}}%
\pgfpathlineto{\pgfqpoint{0.722118in}{3.532878in}}%
\pgfpathlineto{\pgfqpoint{0.713785in}{3.541228in}}%
\pgfpathlineto{\pgfqpoint{0.712406in}{3.542502in}}%
\pgfpathlineto{\pgfqpoint{0.702694in}{3.550129in}}%
\pgfpathlineto{\pgfqpoint{0.701504in}{3.550877in}}%
\pgfpathlineto{\pgfqpoint{0.692982in}{3.555887in}}%
\pgfpathlineto{\pgfqpoint{0.683271in}{3.559297in}}%
\pgfpathlineto{\pgfqpoint{0.675202in}{3.560526in}}%
\pgfpathlineto{\pgfqpoint{0.673559in}{3.560732in}}%
\pgfpathlineto{\pgfqpoint{0.663847in}{3.561264in}}%
\pgfpathlineto{\pgfqpoint{0.654135in}{3.562215in}}%
\pgfpathlineto{\pgfqpoint{0.644424in}{3.564135in}}%
\pgfpathlineto{\pgfqpoint{0.634712in}{3.568522in}}%
\pgfpathlineto{\pgfqpoint{0.634569in}{3.560526in}}%
\pgfpathlineto{\pgfqpoint{0.625000in}{3.551440in}}%
\pgfusepath{stroke}%
\end{pgfscope}%
\begin{pgfscope}%
\pgfpathrectangle{\pgfqpoint{0.625000in}{0.550000in}}{\pgfqpoint{3.875000in}{3.850000in}} %
\pgfusepath{clip}%
\pgfsetbuttcap%
\pgfsetroundjoin%
\pgfsetlinewidth{0.250937pt}%
\definecolor{currentstroke}{rgb}{0.000000,0.000000,0.000000}%
\pgfsetstrokecolor{currentstroke}%
\pgfsetdash{}{0pt}%
\pgfpathmoveto{\pgfqpoint{0.625000in}{3.415488in}}%
\pgfpathlineto{\pgfqpoint{0.634505in}{3.406140in}}%
\pgfpathlineto{\pgfqpoint{0.625000in}{3.397097in}}%
\pgfusepath{stroke}%
\end{pgfscope}%
\begin{pgfscope}%
\pgfpathrectangle{\pgfqpoint{0.625000in}{0.550000in}}{\pgfqpoint{3.875000in}{3.850000in}} %
\pgfusepath{clip}%
\pgfsetbuttcap%
\pgfsetroundjoin%
\pgfsetlinewidth{0.250937pt}%
\definecolor{currentstroke}{rgb}{0.000000,0.000000,0.000000}%
\pgfsetstrokecolor{currentstroke}%
\pgfsetdash{}{0pt}%
\pgfpathmoveto{\pgfqpoint{0.625000in}{3.569618in}}%
\pgfpathlineto{\pgfqpoint{0.634246in}{3.570175in}}%
\pgfpathlineto{\pgfqpoint{0.629337in}{3.579825in}}%
\pgfpathlineto{\pgfqpoint{0.631195in}{3.589474in}}%
\pgfpathlineto{\pgfqpoint{0.634712in}{3.597193in}}%
\pgfpathlineto{\pgfqpoint{0.642924in}{3.599123in}}%
\pgfpathlineto{\pgfqpoint{0.644424in}{3.600721in}}%
\pgfpathlineto{\pgfqpoint{0.654135in}{3.602740in}}%
\pgfpathlineto{\pgfqpoint{0.663847in}{3.604673in}}%
\pgfpathlineto{\pgfqpoint{0.671630in}{3.608772in}}%
\pgfpathlineto{\pgfqpoint{0.673559in}{3.611131in}}%
\pgfpathlineto{\pgfqpoint{0.677836in}{3.618421in}}%
\pgfpathlineto{\pgfqpoint{0.679384in}{3.628070in}}%
\pgfpathlineto{\pgfqpoint{0.678536in}{3.637719in}}%
\pgfpathlineto{\pgfqpoint{0.675741in}{3.647368in}}%
\pgfpathlineto{\pgfqpoint{0.673559in}{3.651821in}}%
\pgfpathlineto{\pgfqpoint{0.671441in}{3.657018in}}%
\pgfpathlineto{\pgfqpoint{0.666045in}{3.666667in}}%
\pgfpathlineto{\pgfqpoint{0.663847in}{3.670077in}}%
\pgfpathlineto{\pgfqpoint{0.660678in}{3.676316in}}%
\pgfpathlineto{\pgfqpoint{0.656028in}{3.685965in}}%
\pgfpathlineto{\pgfqpoint{0.654135in}{3.688936in}}%
\pgfpathlineto{\pgfqpoint{0.651275in}{3.695614in}}%
\pgfpathlineto{\pgfqpoint{0.646148in}{3.705263in}}%
\pgfpathlineto{\pgfqpoint{0.644424in}{3.707759in}}%
\pgfpathlineto{\pgfqpoint{0.641470in}{3.714912in}}%
\pgfpathlineto{\pgfqpoint{0.637044in}{3.724561in}}%
\pgfpathlineto{\pgfqpoint{0.634712in}{3.728203in}}%
\pgfpathlineto{\pgfqpoint{0.633603in}{3.734211in}}%
\pgfpathlineto{\pgfqpoint{0.630832in}{3.743860in}}%
\pgfpathlineto{\pgfqpoint{0.627505in}{3.753509in}}%
\pgfpathlineto{\pgfqpoint{0.625000in}{3.762957in}}%
\pgfusepath{stroke}%
\end{pgfscope}%
\begin{pgfscope}%
\pgfpathrectangle{\pgfqpoint{0.625000in}{0.550000in}}{\pgfqpoint{3.875000in}{3.850000in}} %
\pgfusepath{clip}%
\pgfsetbuttcap%
\pgfsetroundjoin%
\pgfsetlinewidth{0.250937pt}%
\definecolor{currentstroke}{rgb}{0.000000,0.000000,0.000000}%
\pgfsetstrokecolor{currentstroke}%
\pgfsetdash{}{0pt}%
\pgfpathmoveto{\pgfqpoint{0.625000in}{3.668991in}}%
\pgfpathlineto{\pgfqpoint{0.632729in}{3.666667in}}%
\pgfpathlineto{\pgfqpoint{0.625000in}{3.663902in}}%
\pgfusepath{stroke}%
\end{pgfscope}%
\begin{pgfscope}%
\pgfpathrectangle{\pgfqpoint{0.625000in}{0.550000in}}{\pgfqpoint{3.875000in}{3.850000in}} %
\pgfusepath{clip}%
\pgfsetbuttcap%
\pgfsetroundjoin%
\pgfsetlinewidth{0.250937pt}%
\definecolor{currentstroke}{rgb}{0.000000,0.000000,0.000000}%
\pgfsetstrokecolor{currentstroke}%
\pgfsetdash{}{0pt}%
\pgfpathmoveto{\pgfqpoint{0.625000in}{3.724094in}}%
\pgfpathlineto{\pgfqpoint{0.634587in}{3.714912in}}%
\pgfpathlineto{\pgfqpoint{0.625000in}{3.705454in}}%
\pgfusepath{stroke}%
\end{pgfscope}%
\begin{pgfscope}%
\pgfpathrectangle{\pgfqpoint{0.625000in}{0.550000in}}{\pgfqpoint{3.875000in}{3.850000in}} %
\pgfusepath{clip}%
\pgfsetbuttcap%
\pgfsetroundjoin%
\pgfsetlinewidth{0.250937pt}%
\definecolor{currentstroke}{rgb}{0.000000,0.000000,0.000000}%
\pgfsetstrokecolor{currentstroke}%
\pgfsetdash{}{0pt}%
\pgfpathmoveto{\pgfqpoint{0.625000in}{3.763411in}}%
\pgfpathlineto{\pgfqpoint{0.626083in}{3.772807in}}%
\pgfpathlineto{\pgfqpoint{0.627847in}{3.782456in}}%
\pgfpathlineto{\pgfqpoint{0.631735in}{3.792105in}}%
\pgfpathlineto{\pgfqpoint{0.634062in}{3.801754in}}%
\pgfpathlineto{\pgfqpoint{0.634712in}{3.804189in}}%
\pgfpathlineto{\pgfqpoint{0.638566in}{3.811404in}}%
\pgfpathlineto{\pgfqpoint{0.642900in}{3.821053in}}%
\pgfpathlineto{\pgfqpoint{0.644424in}{3.824984in}}%
\pgfpathlineto{\pgfqpoint{0.647912in}{3.830702in}}%
\pgfpathlineto{\pgfqpoint{0.652731in}{3.840351in}}%
\pgfpathlineto{\pgfqpoint{0.654135in}{3.843452in}}%
\pgfpathlineto{\pgfqpoint{0.658083in}{3.850000in}}%
\pgfpathlineto{\pgfqpoint{0.662354in}{3.859649in}}%
\pgfpathlineto{\pgfqpoint{0.663847in}{3.864264in}}%
\pgfpathlineto{\pgfqpoint{0.666308in}{3.869298in}}%
\pgfpathlineto{\pgfqpoint{0.671034in}{3.878947in}}%
\pgfpathlineto{\pgfqpoint{0.673559in}{3.883268in}}%
\pgfpathlineto{\pgfqpoint{0.676948in}{3.888596in}}%
\pgfpathlineto{\pgfqpoint{0.682522in}{3.898246in}}%
\pgfpathlineto{\pgfqpoint{0.683271in}{3.899912in}}%
\pgfpathlineto{\pgfqpoint{0.687117in}{3.907895in}}%
\pgfpathlineto{\pgfqpoint{0.690112in}{3.917544in}}%
\pgfpathlineto{\pgfqpoint{0.691437in}{3.927193in}}%
\pgfpathlineto{\pgfqpoint{0.691074in}{3.936842in}}%
\pgfpathlineto{\pgfqpoint{0.688729in}{3.946491in}}%
\pgfpathlineto{\pgfqpoint{0.683560in}{3.956140in}}%
\pgfpathlineto{\pgfqpoint{0.683271in}{3.956480in}}%
\pgfpathlineto{\pgfqpoint{0.675454in}{3.965789in}}%
\pgfpathlineto{\pgfqpoint{0.673559in}{3.967230in}}%
\pgfpathlineto{\pgfqpoint{0.663847in}{3.975386in}}%
\pgfpathlineto{\pgfqpoint{0.663801in}{3.975439in}}%
\pgfpathlineto{\pgfqpoint{0.654135in}{3.983144in}}%
\pgfpathlineto{\pgfqpoint{0.652908in}{3.985088in}}%
\pgfpathlineto{\pgfqpoint{0.644424in}{3.991834in}}%
\pgfpathlineto{\pgfqpoint{0.642719in}{3.994737in}}%
\pgfpathlineto{\pgfqpoint{0.634712in}{4.001780in}}%
\pgfpathlineto{\pgfqpoint{0.633884in}{4.004386in}}%
\pgfpathlineto{\pgfqpoint{0.629772in}{4.014035in}}%
\pgfpathlineto{\pgfqpoint{0.625000in}{4.014609in}}%
\pgfusepath{stroke}%
\end{pgfscope}%
\begin{pgfscope}%
\pgfpathrectangle{\pgfqpoint{0.625000in}{0.550000in}}{\pgfqpoint{3.875000in}{3.850000in}} %
\pgfusepath{clip}%
\pgfsetbuttcap%
\pgfsetroundjoin%
\pgfsetlinewidth{0.250937pt}%
\definecolor{currentstroke}{rgb}{0.000000,0.000000,0.000000}%
\pgfsetstrokecolor{currentstroke}%
\pgfsetdash{}{0pt}%
\pgfpathmoveto{\pgfqpoint{0.625000in}{3.878378in}}%
\pgfpathlineto{\pgfqpoint{0.634542in}{3.869298in}}%
\pgfpathlineto{\pgfqpoint{0.625000in}{3.860181in}}%
\pgfusepath{stroke}%
\end{pgfscope}%
\begin{pgfscope}%
\pgfpathrectangle{\pgfqpoint{0.625000in}{0.550000in}}{\pgfqpoint{3.875000in}{3.850000in}} %
\pgfusepath{clip}%
\pgfsetbuttcap%
\pgfsetroundjoin%
\pgfsetlinewidth{0.250937pt}%
\definecolor{currentstroke}{rgb}{0.000000,0.000000,0.000000}%
\pgfsetstrokecolor{currentstroke}%
\pgfsetdash{}{0pt}%
\pgfpathmoveto{\pgfqpoint{0.625000in}{4.032849in}}%
\pgfpathlineto{\pgfqpoint{0.629340in}{4.033333in}}%
\pgfpathlineto{\pgfqpoint{0.633884in}{4.042982in}}%
\pgfpathlineto{\pgfqpoint{0.634712in}{4.045545in}}%
\pgfpathlineto{\pgfqpoint{0.642737in}{4.052632in}}%
\pgfpathlineto{\pgfqpoint{0.644424in}{4.055650in}}%
\pgfpathlineto{\pgfqpoint{0.653074in}{4.062281in}}%
\pgfpathlineto{\pgfqpoint{0.654135in}{4.063668in}}%
\pgfpathlineto{\pgfqpoint{0.663630in}{4.071930in}}%
\pgfpathlineto{\pgfqpoint{0.663847in}{4.072309in}}%
\pgfpathlineto{\pgfqpoint{0.671304in}{4.081579in}}%
\pgfpathlineto{\pgfqpoint{0.673559in}{4.089164in}}%
\pgfpathlineto{\pgfqpoint{0.674568in}{4.091228in}}%
\pgfpathlineto{\pgfqpoint{0.673937in}{4.100877in}}%
\pgfpathlineto{\pgfqpoint{0.673559in}{4.101481in}}%
\pgfpathlineto{\pgfqpoint{0.665476in}{4.110526in}}%
\pgfpathlineto{\pgfqpoint{0.663847in}{4.111131in}}%
\pgfpathlineto{\pgfqpoint{0.654135in}{4.113713in}}%
\pgfpathlineto{\pgfqpoint{0.644424in}{4.115524in}}%
\pgfpathlineto{\pgfqpoint{0.636738in}{4.120175in}}%
\pgfpathlineto{\pgfqpoint{0.634712in}{4.120429in}}%
\pgfpathlineto{\pgfqpoint{0.630756in}{4.129825in}}%
\pgfpathlineto{\pgfqpoint{0.630788in}{4.139474in}}%
\pgfpathlineto{\pgfqpoint{0.634712in}{4.148089in}}%
\pgfpathlineto{\pgfqpoint{0.640571in}{4.149123in}}%
\pgfpathlineto{\pgfqpoint{0.644424in}{4.152306in}}%
\pgfpathlineto{\pgfqpoint{0.654135in}{4.154672in}}%
\pgfpathlineto{\pgfqpoint{0.663847in}{4.156315in}}%
\pgfpathlineto{\pgfqpoint{0.673559in}{4.157448in}}%
\pgfpathlineto{\pgfqpoint{0.683271in}{4.158282in}}%
\pgfpathlineto{\pgfqpoint{0.688091in}{4.158772in}}%
\pgfpathlineto{\pgfqpoint{0.692982in}{4.159385in}}%
\pgfpathlineto{\pgfqpoint{0.702694in}{4.161161in}}%
\pgfpathlineto{\pgfqpoint{0.712406in}{4.163763in}}%
\pgfpathlineto{\pgfqpoint{0.722118in}{4.167124in}}%
\pgfpathlineto{\pgfqpoint{0.725329in}{4.168421in}}%
\pgfpathlineto{\pgfqpoint{0.731830in}{4.171051in}}%
\pgfpathlineto{\pgfqpoint{0.741541in}{4.175535in}}%
\pgfpathlineto{\pgfqpoint{0.746475in}{4.178070in}}%
\pgfpathlineto{\pgfqpoint{0.751253in}{4.180548in}}%
\pgfpathlineto{\pgfqpoint{0.760965in}{4.186108in}}%
\pgfpathlineto{\pgfqpoint{0.763561in}{4.187719in}}%
\pgfpathlineto{\pgfqpoint{0.770677in}{4.192225in}}%
\pgfpathlineto{\pgfqpoint{0.778118in}{4.197368in}}%
\pgfpathlineto{\pgfqpoint{0.780388in}{4.198981in}}%
\pgfpathlineto{\pgfqpoint{0.790100in}{4.206412in}}%
\pgfpathlineto{\pgfqpoint{0.790845in}{4.207018in}}%
\pgfpathlineto{\pgfqpoint{0.799812in}{4.214592in}}%
\pgfpathlineto{\pgfqpoint{0.802120in}{4.216667in}}%
\pgfpathlineto{\pgfqpoint{0.809524in}{4.223636in}}%
\pgfpathlineto{\pgfqpoint{0.812221in}{4.226316in}}%
\pgfpathlineto{\pgfqpoint{0.819236in}{4.233672in}}%
\pgfpathlineto{\pgfqpoint{0.821324in}{4.235965in}}%
\pgfpathlineto{\pgfqpoint{0.828947in}{4.244874in}}%
\pgfpathlineto{\pgfqpoint{0.829557in}{4.245614in}}%
\pgfpathlineto{\pgfqpoint{0.837034in}{4.255263in}}%
\pgfpathlineto{\pgfqpoint{0.838659in}{4.257521in}}%
\pgfpathlineto{\pgfqpoint{0.843832in}{4.264912in}}%
\pgfpathlineto{\pgfqpoint{0.848371in}{4.271987in}}%
\pgfpathlineto{\pgfqpoint{0.849989in}{4.274561in}}%
\pgfpathlineto{\pgfqpoint{0.855586in}{4.284211in}}%
\pgfpathlineto{\pgfqpoint{0.858083in}{4.288945in}}%
\pgfpathlineto{\pgfqpoint{0.860647in}{4.293860in}}%
\pgfpathlineto{\pgfqpoint{0.865208in}{4.303509in}}%
\pgfpathlineto{\pgfqpoint{0.867794in}{4.309622in}}%
\pgfpathlineto{\pgfqpoint{0.869289in}{4.313158in}}%
\pgfpathlineto{\pgfqpoint{0.872949in}{4.322807in}}%
\pgfpathlineto{\pgfqpoint{0.876144in}{4.332456in}}%
\pgfpathlineto{\pgfqpoint{0.877506in}{4.337175in}}%
\pgfpathlineto{\pgfqpoint{0.878946in}{4.342105in}}%
\pgfpathlineto{\pgfqpoint{0.881369in}{4.351754in}}%
\pgfpathlineto{\pgfqpoint{0.883386in}{4.361404in}}%
\pgfpathlineto{\pgfqpoint{0.885014in}{4.371053in}}%
\pgfpathlineto{\pgfqpoint{0.886266in}{4.380702in}}%
\pgfpathlineto{\pgfqpoint{0.887152in}{4.390351in}}%
\pgfpathlineto{\pgfqpoint{0.887218in}{4.391553in}}%
\pgfpathlineto{\pgfqpoint{0.887698in}{4.400000in}}%
\pgfusepath{stroke}%
\end{pgfscope}%
\begin{pgfscope}%
\pgfpathrectangle{\pgfqpoint{0.625000in}{0.550000in}}{\pgfqpoint{3.875000in}{3.850000in}} %
\pgfusepath{clip}%
\pgfsetbuttcap%
\pgfsetroundjoin%
\pgfsetlinewidth{0.250937pt}%
\definecolor{currentstroke}{rgb}{0.000000,0.000000,0.000000}%
\pgfsetstrokecolor{currentstroke}%
\pgfsetdash{}{0pt}%
\pgfpathmoveto{\pgfqpoint{0.625000in}{4.053793in}}%
\pgfpathlineto{\pgfqpoint{0.628904in}{4.052632in}}%
\pgfpathlineto{\pgfqpoint{0.625000in}{4.051207in}}%
\pgfusepath{stroke}%
\end{pgfscope}%
\begin{pgfscope}%
\pgfpathrectangle{\pgfqpoint{0.625000in}{0.550000in}}{\pgfqpoint{3.875000in}{3.850000in}} %
\pgfusepath{clip}%
\pgfsetbuttcap%
\pgfsetroundjoin%
\pgfsetlinewidth{0.250937pt}%
\definecolor{currentstroke}{rgb}{0.000000,0.000000,0.000000}%
\pgfsetstrokecolor{currentstroke}%
\pgfsetdash{}{0pt}%
\pgfpathmoveto{\pgfqpoint{0.625000in}{4.187265in}}%
\pgfpathlineto{\pgfqpoint{0.634121in}{4.187719in}}%
\pgfpathlineto{\pgfqpoint{0.631592in}{4.197368in}}%
\pgfpathlineto{\pgfqpoint{0.634712in}{4.204320in}}%
\pgfpathlineto{\pgfqpoint{0.644424in}{4.200922in}}%
\pgfpathlineto{\pgfqpoint{0.649591in}{4.197368in}}%
\pgfpathlineto{\pgfqpoint{0.644424in}{4.189901in}}%
\pgfpathlineto{\pgfqpoint{0.642652in}{4.187719in}}%
\pgfpathlineto{\pgfqpoint{0.634712in}{4.186170in}}%
\pgfpathlineto{\pgfqpoint{0.634552in}{4.178070in}}%
\pgfpathlineto{\pgfqpoint{0.625000in}{4.169305in}}%
\pgfusepath{stroke}%
\end{pgfscope}%
\begin{pgfscope}%
\pgfpathrectangle{\pgfqpoint{0.625000in}{0.550000in}}{\pgfqpoint{3.875000in}{3.850000in}} %
\pgfusepath{clip}%
\pgfsetbuttcap%
\pgfsetroundjoin%
\pgfsetlinewidth{0.250937pt}%
\definecolor{currentstroke}{rgb}{0.000000,0.000000,0.000000}%
\pgfsetstrokecolor{currentstroke}%
\pgfsetdash{}{0pt}%
\pgfpathmoveto{\pgfqpoint{0.625000in}{4.341594in}}%
\pgfpathlineto{\pgfqpoint{0.634492in}{4.332456in}}%
\pgfpathlineto{\pgfqpoint{0.625000in}{4.323288in}}%
\pgfusepath{stroke}%
\end{pgfscope}%
\begin{pgfscope}%
\pgfpathrectangle{\pgfqpoint{0.625000in}{0.550000in}}{\pgfqpoint{3.875000in}{3.850000in}} %
\pgfusepath{clip}%
\pgfsetbuttcap%
\pgfsetroundjoin%
\pgfsetlinewidth{0.250937pt}%
\definecolor{currentstroke}{rgb}{0.000000,0.000000,0.000000}%
\pgfsetstrokecolor{currentstroke}%
\pgfsetdash{}{0pt}%
\pgfpathmoveto{\pgfqpoint{0.634712in}{0.721353in}}%
\pgfpathlineto{\pgfqpoint{0.633884in}{0.723684in}}%
\pgfpathlineto{\pgfqpoint{0.634712in}{0.726932in}}%
\pgfpathlineto{\pgfqpoint{0.638951in}{0.723684in}}%
\pgfpathlineto{\pgfqpoint{0.634712in}{0.721353in}}%
\pgfusepath{stroke}%
\end{pgfscope}%
\begin{pgfscope}%
\pgfpathrectangle{\pgfqpoint{0.625000in}{0.550000in}}{\pgfqpoint{3.875000in}{3.850000in}} %
\pgfusepath{clip}%
\pgfsetbuttcap%
\pgfsetroundjoin%
\pgfsetlinewidth{0.250937pt}%
\definecolor{currentstroke}{rgb}{0.000000,0.000000,0.000000}%
\pgfsetstrokecolor{currentstroke}%
\pgfsetdash{}{0pt}%
\pgfpathmoveto{\pgfqpoint{0.634712in}{1.424801in}}%
\pgfpathlineto{\pgfqpoint{0.632545in}{1.428070in}}%
\pgfpathlineto{\pgfqpoint{0.634712in}{1.430918in}}%
\pgfpathlineto{\pgfqpoint{0.639474in}{1.428070in}}%
\pgfpathlineto{\pgfqpoint{0.634712in}{1.424801in}}%
\pgfusepath{stroke}%
\end{pgfscope}%
\begin{pgfscope}%
\pgfpathrectangle{\pgfqpoint{0.625000in}{0.550000in}}{\pgfqpoint{3.875000in}{3.850000in}} %
\pgfusepath{clip}%
\pgfsetbuttcap%
\pgfsetroundjoin%
\pgfsetlinewidth{0.250937pt}%
\definecolor{currentstroke}{rgb}{0.000000,0.000000,0.000000}%
\pgfsetstrokecolor{currentstroke}%
\pgfsetdash{}{0pt}%
\pgfpathmoveto{\pgfqpoint{0.634712in}{1.613397in}}%
\pgfpathlineto{\pgfqpoint{0.629441in}{1.621053in}}%
\pgfpathlineto{\pgfqpoint{0.634712in}{1.630497in}}%
\pgfpathlineto{\pgfqpoint{0.644424in}{1.627100in}}%
\pgfpathlineto{\pgfqpoint{0.650064in}{1.621053in}}%
\pgfpathlineto{\pgfqpoint{0.644424in}{1.616778in}}%
\pgfpathlineto{\pgfqpoint{0.634712in}{1.613397in}}%
\pgfusepath{stroke}%
\end{pgfscope}%
\begin{pgfscope}%
\pgfpathrectangle{\pgfqpoint{0.625000in}{0.550000in}}{\pgfqpoint{3.875000in}{3.850000in}} %
\pgfusepath{clip}%
\pgfsetbuttcap%
\pgfsetroundjoin%
\pgfsetlinewidth{0.250937pt}%
\definecolor{currentstroke}{rgb}{0.000000,0.000000,0.000000}%
\pgfsetstrokecolor{currentstroke}%
\pgfsetdash{}{0pt}%
\pgfpathmoveto{\pgfqpoint{0.634712in}{1.773860in}}%
\pgfpathlineto{\pgfqpoint{0.634252in}{1.775439in}}%
\pgfpathlineto{\pgfqpoint{0.634712in}{1.778402in}}%
\pgfpathlineto{\pgfqpoint{0.638163in}{1.775439in}}%
\pgfpathlineto{\pgfqpoint{0.634712in}{1.773860in}}%
\pgfusepath{stroke}%
\end{pgfscope}%
\begin{pgfscope}%
\pgfpathrectangle{\pgfqpoint{0.625000in}{0.550000in}}{\pgfqpoint{3.875000in}{3.850000in}} %
\pgfusepath{clip}%
\pgfsetbuttcap%
\pgfsetroundjoin%
\pgfsetlinewidth{0.250937pt}%
\definecolor{currentstroke}{rgb}{0.000000,0.000000,0.000000}%
\pgfsetstrokecolor{currentstroke}%
\pgfsetdash{}{0pt}%
\pgfpathmoveto{\pgfqpoint{0.634712in}{2.041216in}}%
\pgfpathlineto{\pgfqpoint{0.633111in}{2.045614in}}%
\pgfpathlineto{\pgfqpoint{0.632738in}{2.055263in}}%
\pgfpathlineto{\pgfqpoint{0.634712in}{2.060683in}}%
\pgfpathlineto{\pgfqpoint{0.644424in}{2.057149in}}%
\pgfpathlineto{\pgfqpoint{0.648658in}{2.055263in}}%
\pgfpathlineto{\pgfqpoint{0.646982in}{2.045614in}}%
\pgfpathlineto{\pgfqpoint{0.644424in}{2.044597in}}%
\pgfpathlineto{\pgfqpoint{0.634712in}{2.041216in}}%
\pgfusepath{stroke}%
\end{pgfscope}%
\begin{pgfscope}%
\pgfpathrectangle{\pgfqpoint{0.625000in}{0.550000in}}{\pgfqpoint{3.875000in}{3.850000in}} %
\pgfusepath{clip}%
\pgfsetbuttcap%
\pgfsetroundjoin%
\pgfsetlinewidth{0.250937pt}%
\definecolor{currentstroke}{rgb}{0.000000,0.000000,0.000000}%
\pgfsetstrokecolor{currentstroke}%
\pgfsetdash{}{0pt}%
\pgfpathmoveto{\pgfqpoint{0.634712in}{2.130499in}}%
\pgfpathlineto{\pgfqpoint{0.634515in}{2.132456in}}%
\pgfpathlineto{\pgfqpoint{0.634712in}{2.133169in}}%
\pgfpathlineto{\pgfqpoint{0.636465in}{2.132456in}}%
\pgfpathlineto{\pgfqpoint{0.634712in}{2.130499in}}%
\pgfusepath{stroke}%
\end{pgfscope}%
\begin{pgfscope}%
\pgfpathrectangle{\pgfqpoint{0.625000in}{0.550000in}}{\pgfqpoint{3.875000in}{3.850000in}} %
\pgfusepath{clip}%
\pgfsetbuttcap%
\pgfsetroundjoin%
\pgfsetlinewidth{0.250937pt}%
\definecolor{currentstroke}{rgb}{0.000000,0.000000,0.000000}%
\pgfsetstrokecolor{currentstroke}%
\pgfsetdash{}{0pt}%
\pgfpathmoveto{\pgfqpoint{0.634712in}{2.826480in}}%
\pgfpathlineto{\pgfqpoint{0.634515in}{2.827193in}}%
\pgfpathlineto{\pgfqpoint{0.634712in}{2.829151in}}%
\pgfpathlineto{\pgfqpoint{0.636465in}{2.827193in}}%
\pgfpathlineto{\pgfqpoint{0.634712in}{2.826480in}}%
\pgfusepath{stroke}%
\end{pgfscope}%
\begin{pgfscope}%
\pgfpathrectangle{\pgfqpoint{0.625000in}{0.550000in}}{\pgfqpoint{3.875000in}{3.850000in}} %
\pgfusepath{clip}%
\pgfsetbuttcap%
\pgfsetroundjoin%
\pgfsetlinewidth{0.250937pt}%
\definecolor{currentstroke}{rgb}{0.000000,0.000000,0.000000}%
\pgfsetstrokecolor{currentstroke}%
\pgfsetdash{}{0pt}%
\pgfpathmoveto{\pgfqpoint{0.634712in}{2.898966in}}%
\pgfpathlineto{\pgfqpoint{0.632738in}{2.904386in}}%
\pgfpathlineto{\pgfqpoint{0.633111in}{2.914035in}}%
\pgfpathlineto{\pgfqpoint{0.634712in}{2.918434in}}%
\pgfpathlineto{\pgfqpoint{0.644424in}{2.915052in}}%
\pgfpathlineto{\pgfqpoint{0.646982in}{2.914035in}}%
\pgfpathlineto{\pgfqpoint{0.648658in}{2.904386in}}%
\pgfpathlineto{\pgfqpoint{0.644424in}{2.902500in}}%
\pgfpathlineto{\pgfqpoint{0.634712in}{2.898966in}}%
\pgfusepath{stroke}%
\end{pgfscope}%
\begin{pgfscope}%
\pgfpathrectangle{\pgfqpoint{0.625000in}{0.550000in}}{\pgfqpoint{3.875000in}{3.850000in}} %
\pgfusepath{clip}%
\pgfsetbuttcap%
\pgfsetroundjoin%
\pgfsetlinewidth{0.250937pt}%
\definecolor{currentstroke}{rgb}{0.000000,0.000000,0.000000}%
\pgfsetstrokecolor{currentstroke}%
\pgfsetdash{}{0pt}%
\pgfpathmoveto{\pgfqpoint{0.634712in}{3.181247in}}%
\pgfpathlineto{\pgfqpoint{0.634252in}{3.184211in}}%
\pgfpathlineto{\pgfqpoint{0.634712in}{3.185789in}}%
\pgfpathlineto{\pgfqpoint{0.638163in}{3.184211in}}%
\pgfpathlineto{\pgfqpoint{0.634712in}{3.181247in}}%
\pgfusepath{stroke}%
\end{pgfscope}%
\begin{pgfscope}%
\pgfpathrectangle{\pgfqpoint{0.625000in}{0.550000in}}{\pgfqpoint{3.875000in}{3.850000in}} %
\pgfusepath{clip}%
\pgfsetbuttcap%
\pgfsetroundjoin%
\pgfsetlinewidth{0.250937pt}%
\definecolor{currentstroke}{rgb}{0.000000,0.000000,0.000000}%
\pgfsetstrokecolor{currentstroke}%
\pgfsetdash{}{0pt}%
\pgfpathmoveto{\pgfqpoint{0.634712in}{3.329152in}}%
\pgfpathlineto{\pgfqpoint{0.629441in}{3.338596in}}%
\pgfpathlineto{\pgfqpoint{0.634712in}{3.346252in}}%
\pgfpathlineto{\pgfqpoint{0.644424in}{3.342871in}}%
\pgfpathlineto{\pgfqpoint{0.650064in}{3.338596in}}%
\pgfpathlineto{\pgfqpoint{0.644424in}{3.332549in}}%
\pgfpathlineto{\pgfqpoint{0.634712in}{3.329152in}}%
\pgfusepath{stroke}%
\end{pgfscope}%
\begin{pgfscope}%
\pgfpathrectangle{\pgfqpoint{0.625000in}{0.550000in}}{\pgfqpoint{3.875000in}{3.850000in}} %
\pgfusepath{clip}%
\pgfsetbuttcap%
\pgfsetroundjoin%
\pgfsetlinewidth{0.250937pt}%
\definecolor{currentstroke}{rgb}{0.000000,0.000000,0.000000}%
\pgfsetstrokecolor{currentstroke}%
\pgfsetdash{}{0pt}%
\pgfpathmoveto{\pgfqpoint{0.634712in}{3.528731in}}%
\pgfpathlineto{\pgfqpoint{0.632545in}{3.531579in}}%
\pgfpathlineto{\pgfqpoint{0.634712in}{3.534848in}}%
\pgfpathlineto{\pgfqpoint{0.639474in}{3.531579in}}%
\pgfpathlineto{\pgfqpoint{0.634712in}{3.528731in}}%
\pgfusepath{stroke}%
\end{pgfscope}%
\begin{pgfscope}%
\pgfpathrectangle{\pgfqpoint{0.625000in}{0.550000in}}{\pgfqpoint{3.875000in}{3.850000in}} %
\pgfusepath{clip}%
\pgfsetbuttcap%
\pgfsetroundjoin%
\pgfsetlinewidth{0.250937pt}%
\definecolor{currentstroke}{rgb}{0.000000,0.000000,0.000000}%
\pgfsetstrokecolor{currentstroke}%
\pgfsetdash{}{0pt}%
\pgfpathmoveto{\pgfqpoint{0.634712in}{4.232717in}}%
\pgfpathlineto{\pgfqpoint{0.633884in}{4.235965in}}%
\pgfpathlineto{\pgfqpoint{0.634712in}{4.238296in}}%
\pgfpathlineto{\pgfqpoint{0.638951in}{4.235965in}}%
\pgfpathlineto{\pgfqpoint{0.634712in}{4.232717in}}%
\pgfusepath{stroke}%
\end{pgfscope}%
\begin{pgfscope}%
\pgfpathrectangle{\pgfqpoint{0.625000in}{0.550000in}}{\pgfqpoint{3.875000in}{3.850000in}} %
\pgfusepath{clip}%
\pgfsetbuttcap%
\pgfsetroundjoin%
\pgfsetlinewidth{0.250937pt}%
\definecolor{currentstroke}{rgb}{0.000000,0.000000,0.000000}%
\pgfsetstrokecolor{currentstroke}%
\pgfsetdash{}{0pt}%
\pgfpathmoveto{\pgfqpoint{0.784510in}{0.550000in}}%
\pgfpathlineto{\pgfqpoint{0.784212in}{0.559649in}}%
\pgfpathlineto{\pgfqpoint{0.783311in}{0.569298in}}%
\pgfpathlineto{\pgfqpoint{0.781787in}{0.578947in}}%
\pgfpathlineto{\pgfqpoint{0.780388in}{0.585204in}}%
\pgfpathlineto{\pgfqpoint{0.779653in}{0.588596in}}%
\pgfpathlineto{\pgfqpoint{0.776922in}{0.598246in}}%
\pgfpathlineto{\pgfqpoint{0.773456in}{0.607895in}}%
\pgfpathlineto{\pgfqpoint{0.770677in}{0.614270in}}%
\pgfpathlineto{\pgfqpoint{0.769245in}{0.617544in}}%
\pgfpathlineto{\pgfqpoint{0.764267in}{0.627193in}}%
\pgfpathlineto{\pgfqpoint{0.760965in}{0.632670in}}%
\pgfpathlineto{\pgfqpoint{0.758372in}{0.636842in}}%
\pgfpathlineto{\pgfqpoint{0.751448in}{0.646491in}}%
\pgfpathlineto{\pgfqpoint{0.751253in}{0.646738in}}%
\pgfpathlineto{\pgfqpoint{0.743362in}{0.656140in}}%
\pgfpathlineto{\pgfqpoint{0.741541in}{0.658100in}}%
\pgfpathlineto{\pgfqpoint{0.733802in}{0.665789in}}%
\pgfpathlineto{\pgfqpoint{0.731830in}{0.667599in}}%
\pgfpathlineto{\pgfqpoint{0.722366in}{0.675439in}}%
\pgfpathlineto{\pgfqpoint{0.722118in}{0.675633in}}%
\pgfpathlineto{\pgfqpoint{0.712406in}{0.682511in}}%
\pgfpathlineto{\pgfqpoint{0.708207in}{0.685088in}}%
\pgfpathlineto{\pgfqpoint{0.702694in}{0.688368in}}%
\pgfpathlineto{\pgfqpoint{0.692982in}{0.693314in}}%
\pgfpathlineto{\pgfqpoint{0.689687in}{0.694737in}}%
\pgfpathlineto{\pgfqpoint{0.683271in}{0.697497in}}%
\pgfpathlineto{\pgfqpoint{0.673559in}{0.700941in}}%
\pgfpathlineto{\pgfqpoint{0.663847in}{0.703698in}}%
\pgfpathlineto{\pgfqpoint{0.661225in}{0.704386in}}%
\pgfpathlineto{\pgfqpoint{0.654135in}{0.706142in}}%
\pgfpathlineto{\pgfqpoint{0.644424in}{0.710229in}}%
\pgfpathlineto{\pgfqpoint{0.643623in}{0.714035in}}%
\pgfpathlineto{\pgfqpoint{0.634712in}{0.716553in}}%
\pgfpathlineto{\pgfqpoint{0.632179in}{0.723684in}}%
\pgfpathlineto{\pgfqpoint{0.634654in}{0.733333in}}%
\pgfpathlineto{\pgfqpoint{0.634712in}{0.733730in}}%
\pgfpathlineto{\pgfqpoint{0.643728in}{0.742982in}}%
\pgfpathlineto{\pgfqpoint{0.634712in}{0.752235in}}%
\pgfpathlineto{\pgfqpoint{0.634655in}{0.752632in}}%
\pgfpathlineto{\pgfqpoint{0.630327in}{0.762281in}}%
\pgfpathlineto{\pgfqpoint{0.632340in}{0.771930in}}%
\pgfpathlineto{\pgfqpoint{0.625000in}{0.772261in}}%
\pgfusepath{stroke}%
\end{pgfscope}%
\begin{pgfscope}%
\pgfpathrectangle{\pgfqpoint{0.625000in}{0.550000in}}{\pgfqpoint{3.875000in}{3.850000in}} %
\pgfusepath{clip}%
\pgfsetbuttcap%
\pgfsetroundjoin%
\pgfsetlinewidth{0.250937pt}%
\definecolor{currentstroke}{rgb}{0.000000,0.000000,0.000000}%
\pgfsetstrokecolor{currentstroke}%
\pgfsetdash{}{0pt}%
\pgfpathmoveto{\pgfqpoint{0.625000in}{0.636501in}}%
\pgfpathlineto{\pgfqpoint{0.634600in}{0.627193in}}%
\pgfpathlineto{\pgfqpoint{0.625000in}{0.617912in}}%
\pgfusepath{stroke}%
\end{pgfscope}%
\begin{pgfscope}%
\pgfpathrectangle{\pgfqpoint{0.625000in}{0.550000in}}{\pgfqpoint{3.875000in}{3.850000in}} %
\pgfusepath{clip}%
\pgfsetbuttcap%
\pgfsetroundjoin%
\pgfsetlinewidth{0.250937pt}%
\definecolor{currentstroke}{rgb}{0.000000,0.000000,0.000000}%
\pgfsetstrokecolor{currentstroke}%
\pgfsetdash{}{0pt}%
\pgfpathmoveto{\pgfqpoint{0.625000in}{0.790500in}}%
\pgfpathlineto{\pgfqpoint{0.634651in}{0.781579in}}%
\pgfpathlineto{\pgfqpoint{0.634712in}{0.778152in}}%
\pgfpathlineto{\pgfqpoint{0.642609in}{0.781579in}}%
\pgfpathlineto{\pgfqpoint{0.644424in}{0.787509in}}%
\pgfpathlineto{\pgfqpoint{0.646155in}{0.791228in}}%
\pgfpathlineto{\pgfqpoint{0.644424in}{0.793852in}}%
\pgfpathlineto{\pgfqpoint{0.641832in}{0.800877in}}%
\pgfpathlineto{\pgfqpoint{0.634712in}{0.806924in}}%
\pgfpathlineto{\pgfqpoint{0.633327in}{0.810526in}}%
\pgfpathlineto{\pgfqpoint{0.629911in}{0.820175in}}%
\pgfpathlineto{\pgfqpoint{0.630040in}{0.829825in}}%
\pgfpathlineto{\pgfqpoint{0.633714in}{0.839474in}}%
\pgfpathlineto{\pgfqpoint{0.634712in}{0.845127in}}%
\pgfpathlineto{\pgfqpoint{0.640591in}{0.849123in}}%
\pgfpathlineto{\pgfqpoint{0.644424in}{0.855096in}}%
\pgfpathlineto{\pgfqpoint{0.648270in}{0.858772in}}%
\pgfpathlineto{\pgfqpoint{0.651982in}{0.868421in}}%
\pgfpathlineto{\pgfqpoint{0.651080in}{0.878070in}}%
\pgfpathlineto{\pgfqpoint{0.645981in}{0.887719in}}%
\pgfpathlineto{\pgfqpoint{0.644424in}{0.889801in}}%
\pgfpathlineto{\pgfqpoint{0.641179in}{0.897368in}}%
\pgfpathlineto{\pgfqpoint{0.636760in}{0.907018in}}%
\pgfpathlineto{\pgfqpoint{0.634712in}{0.908826in}}%
\pgfpathlineto{\pgfqpoint{0.632179in}{0.916667in}}%
\pgfpathlineto{\pgfqpoint{0.628584in}{0.926316in}}%
\pgfpathlineto{\pgfqpoint{0.625000in}{0.926682in}}%
\pgfusepath{stroke}%
\end{pgfscope}%
\begin{pgfscope}%
\pgfpathrectangle{\pgfqpoint{0.625000in}{0.550000in}}{\pgfqpoint{3.875000in}{3.850000in}} %
\pgfusepath{clip}%
\pgfsetbuttcap%
\pgfsetroundjoin%
\pgfsetlinewidth{0.250937pt}%
\definecolor{currentstroke}{rgb}{0.000000,0.000000,0.000000}%
\pgfsetstrokecolor{currentstroke}%
\pgfsetdash{}{0pt}%
\pgfpathmoveto{\pgfqpoint{0.625000in}{0.910021in}}%
\pgfpathlineto{\pgfqpoint{0.633230in}{0.907018in}}%
\pgfpathlineto{\pgfqpoint{0.625000in}{0.904569in}}%
\pgfusepath{stroke}%
\end{pgfscope}%
\begin{pgfscope}%
\pgfpathrectangle{\pgfqpoint{0.625000in}{0.550000in}}{\pgfqpoint{3.875000in}{3.850000in}} %
\pgfusepath{clip}%
\pgfsetbuttcap%
\pgfsetroundjoin%
\pgfsetlinewidth{0.250937pt}%
\definecolor{currentstroke}{rgb}{0.000000,0.000000,0.000000}%
\pgfsetstrokecolor{currentstroke}%
\pgfsetdash{}{0pt}%
\pgfpathmoveto{\pgfqpoint{0.625000in}{0.945165in}}%
\pgfpathlineto{\pgfqpoint{0.629078in}{0.945614in}}%
\pgfpathlineto{\pgfqpoint{0.632179in}{0.955263in}}%
\pgfpathlineto{\pgfqpoint{0.634712in}{0.963232in}}%
\pgfpathlineto{\pgfqpoint{0.636622in}{0.964912in}}%
\pgfpathlineto{\pgfqpoint{0.641934in}{0.974561in}}%
\pgfpathlineto{\pgfqpoint{0.643823in}{0.984211in}}%
\pgfpathlineto{\pgfqpoint{0.634712in}{0.990856in}}%
\pgfpathlineto{\pgfqpoint{0.633833in}{0.993860in}}%
\pgfpathlineto{\pgfqpoint{0.634712in}{0.998143in}}%
\pgfpathlineto{\pgfqpoint{0.644424in}{0.995726in}}%
\pgfpathlineto{\pgfqpoint{0.652356in}{1.003509in}}%
\pgfpathlineto{\pgfqpoint{0.654135in}{1.005230in}}%
\pgfpathlineto{\pgfqpoint{0.659939in}{1.013158in}}%
\pgfpathlineto{\pgfqpoint{0.663680in}{1.022807in}}%
\pgfpathlineto{\pgfqpoint{0.663847in}{1.024254in}}%
\pgfpathlineto{\pgfqpoint{0.664978in}{1.032456in}}%
\pgfpathlineto{\pgfqpoint{0.663847in}{1.040550in}}%
\pgfpathlineto{\pgfqpoint{0.663665in}{1.042105in}}%
\pgfpathlineto{\pgfqpoint{0.659984in}{1.051754in}}%
\pgfpathlineto{\pgfqpoint{0.654135in}{1.059580in}}%
\pgfpathlineto{\pgfqpoint{0.652189in}{1.061404in}}%
\pgfpathlineto{\pgfqpoint{0.644424in}{1.065848in}}%
\pgfpathlineto{\pgfqpoint{0.634712in}{1.066180in}}%
\pgfpathlineto{\pgfqpoint{0.633345in}{1.071053in}}%
\pgfpathlineto{\pgfqpoint{0.633337in}{1.080702in}}%
\pgfpathlineto{\pgfqpoint{0.625000in}{1.081140in}}%
\pgfusepath{stroke}%
\end{pgfscope}%
\begin{pgfscope}%
\pgfpathrectangle{\pgfqpoint{0.625000in}{0.550000in}}{\pgfqpoint{3.875000in}{3.850000in}} %
\pgfusepath{clip}%
\pgfsetbuttcap%
\pgfsetroundjoin%
\pgfsetlinewidth{0.250937pt}%
\definecolor{currentstroke}{rgb}{0.000000,0.000000,0.000000}%
\pgfsetstrokecolor{currentstroke}%
\pgfsetdash{}{0pt}%
\pgfpathmoveto{\pgfqpoint{0.625000in}{1.099596in}}%
\pgfpathlineto{\pgfqpoint{0.634640in}{1.090351in}}%
\pgfpathlineto{\pgfqpoint{0.634712in}{1.085669in}}%
\pgfpathlineto{\pgfqpoint{0.644424in}{1.088736in}}%
\pgfpathlineto{\pgfqpoint{0.645760in}{1.090351in}}%
\pgfpathlineto{\pgfqpoint{0.648209in}{1.100000in}}%
\pgfpathlineto{\pgfqpoint{0.645902in}{1.109649in}}%
\pgfpathlineto{\pgfqpoint{0.644424in}{1.113173in}}%
\pgfpathlineto{\pgfqpoint{0.641423in}{1.119298in}}%
\pgfpathlineto{\pgfqpoint{0.640770in}{1.128947in}}%
\pgfpathlineto{\pgfqpoint{0.637460in}{1.138596in}}%
\pgfpathlineto{\pgfqpoint{0.634712in}{1.147930in}}%
\pgfpathlineto{\pgfqpoint{0.634673in}{1.148246in}}%
\pgfpathlineto{\pgfqpoint{0.632102in}{1.157895in}}%
\pgfpathlineto{\pgfqpoint{0.630891in}{1.167544in}}%
\pgfpathlineto{\pgfqpoint{0.627304in}{1.177193in}}%
\pgfpathlineto{\pgfqpoint{0.625875in}{1.186842in}}%
\pgfpathlineto{\pgfqpoint{0.625000in}{1.194435in}}%
\pgfusepath{stroke}%
\end{pgfscope}%
\begin{pgfscope}%
\pgfpathrectangle{\pgfqpoint{0.625000in}{0.550000in}}{\pgfqpoint{3.875000in}{3.850000in}} %
\pgfusepath{clip}%
\pgfsetbuttcap%
\pgfsetroundjoin%
\pgfsetlinewidth{0.250937pt}%
\definecolor{currentstroke}{rgb}{0.000000,0.000000,0.000000}%
\pgfsetstrokecolor{currentstroke}%
\pgfsetdash{}{0pt}%
\pgfpathmoveto{\pgfqpoint{0.625000in}{1.198125in}}%
\pgfpathlineto{\pgfqpoint{0.627125in}{1.206140in}}%
\pgfpathlineto{\pgfqpoint{0.630129in}{1.215789in}}%
\pgfpathlineto{\pgfqpoint{0.632582in}{1.225439in}}%
\pgfpathlineto{\pgfqpoint{0.634024in}{1.235088in}}%
\pgfpathlineto{\pgfqpoint{0.625000in}{1.235448in}}%
\pgfusepath{stroke}%
\end{pgfscope}%
\begin{pgfscope}%
\pgfpathrectangle{\pgfqpoint{0.625000in}{0.550000in}}{\pgfqpoint{3.875000in}{3.850000in}} %
\pgfusepath{clip}%
\pgfsetbuttcap%
\pgfsetroundjoin%
\pgfsetlinewidth{0.250937pt}%
\definecolor{currentstroke}{rgb}{0.000000,0.000000,0.000000}%
\pgfsetstrokecolor{currentstroke}%
\pgfsetdash{}{0pt}%
\pgfpathmoveto{\pgfqpoint{0.625000in}{1.254289in}}%
\pgfpathlineto{\pgfqpoint{0.634679in}{1.244737in}}%
\pgfpathlineto{\pgfqpoint{0.634712in}{1.239552in}}%
\pgfpathlineto{\pgfqpoint{0.636631in}{1.244737in}}%
\pgfpathlineto{\pgfqpoint{0.639500in}{1.254386in}}%
\pgfpathlineto{\pgfqpoint{0.641870in}{1.264035in}}%
\pgfpathlineto{\pgfqpoint{0.644424in}{1.272783in}}%
\pgfpathlineto{\pgfqpoint{0.644740in}{1.273684in}}%
\pgfpathlineto{\pgfqpoint{0.644424in}{1.279769in}}%
\pgfpathlineto{\pgfqpoint{0.644188in}{1.283333in}}%
\pgfpathlineto{\pgfqpoint{0.634712in}{1.289094in}}%
\pgfpathlineto{\pgfqpoint{0.625000in}{1.289548in}}%
\pgfusepath{stroke}%
\end{pgfscope}%
\begin{pgfscope}%
\pgfpathrectangle{\pgfqpoint{0.625000in}{0.550000in}}{\pgfqpoint{3.875000in}{3.850000in}} %
\pgfusepath{clip}%
\pgfsetbuttcap%
\pgfsetroundjoin%
\pgfsetlinewidth{0.250937pt}%
\definecolor{currentstroke}{rgb}{0.000000,0.000000,0.000000}%
\pgfsetstrokecolor{currentstroke}%
\pgfsetdash{}{0pt}%
\pgfpathmoveto{\pgfqpoint{0.625000in}{1.297069in}}%
\pgfpathlineto{\pgfqpoint{0.634712in}{1.296564in}}%
\pgfpathlineto{\pgfqpoint{0.644424in}{1.294741in}}%
\pgfpathlineto{\pgfqpoint{0.650548in}{1.302632in}}%
\pgfpathlineto{\pgfqpoint{0.654135in}{1.309503in}}%
\pgfpathlineto{\pgfqpoint{0.655409in}{1.312281in}}%
\pgfpathlineto{\pgfqpoint{0.657214in}{1.321930in}}%
\pgfpathlineto{\pgfqpoint{0.655589in}{1.331579in}}%
\pgfpathlineto{\pgfqpoint{0.654135in}{1.334172in}}%
\pgfpathlineto{\pgfqpoint{0.650181in}{1.341228in}}%
\pgfpathlineto{\pgfqpoint{0.644424in}{1.346904in}}%
\pgfpathlineto{\pgfqpoint{0.642105in}{1.350877in}}%
\pgfpathlineto{\pgfqpoint{0.634712in}{1.358837in}}%
\pgfpathlineto{\pgfqpoint{0.634469in}{1.360526in}}%
\pgfpathlineto{\pgfqpoint{0.630168in}{1.370175in}}%
\pgfpathlineto{\pgfqpoint{0.628505in}{1.379825in}}%
\pgfpathlineto{\pgfqpoint{0.632840in}{1.389474in}}%
\pgfpathlineto{\pgfqpoint{0.625000in}{1.389928in}}%
\pgfusepath{stroke}%
\end{pgfscope}%
\begin{pgfscope}%
\pgfpathrectangle{\pgfqpoint{0.625000in}{0.550000in}}{\pgfqpoint{3.875000in}{3.850000in}} %
\pgfusepath{clip}%
\pgfsetbuttcap%
\pgfsetroundjoin%
\pgfsetlinewidth{0.250937pt}%
\definecolor{currentstroke}{rgb}{0.000000,0.000000,0.000000}%
\pgfsetstrokecolor{currentstroke}%
\pgfsetdash{}{0pt}%
\pgfpathmoveto{\pgfqpoint{0.625000in}{1.408312in}}%
\pgfpathlineto{\pgfqpoint{0.634660in}{1.399123in}}%
\pgfpathlineto{\pgfqpoint{0.634712in}{1.396111in}}%
\pgfpathlineto{\pgfqpoint{0.639899in}{1.399123in}}%
\pgfpathlineto{\pgfqpoint{0.644102in}{1.408772in}}%
\pgfpathlineto{\pgfqpoint{0.638657in}{1.418421in}}%
\pgfpathlineto{\pgfqpoint{0.634712in}{1.419212in}}%
\pgfpathlineto{\pgfqpoint{0.628840in}{1.428070in}}%
\pgfpathlineto{\pgfqpoint{0.634712in}{1.435788in}}%
\pgfpathlineto{\pgfqpoint{0.643594in}{1.437719in}}%
\pgfpathlineto{\pgfqpoint{0.644424in}{1.438922in}}%
\pgfpathlineto{\pgfqpoint{0.654135in}{1.441402in}}%
\pgfpathlineto{\pgfqpoint{0.663847in}{1.445716in}}%
\pgfpathlineto{\pgfqpoint{0.666517in}{1.447368in}}%
\pgfpathlineto{\pgfqpoint{0.673559in}{1.452046in}}%
\pgfpathlineto{\pgfqpoint{0.679386in}{1.457018in}}%
\pgfpathlineto{\pgfqpoint{0.683271in}{1.460877in}}%
\pgfpathlineto{\pgfqpoint{0.688271in}{1.466667in}}%
\pgfpathlineto{\pgfqpoint{0.692982in}{1.473641in}}%
\pgfpathlineto{\pgfqpoint{0.694664in}{1.476316in}}%
\pgfpathlineto{\pgfqpoint{0.699311in}{1.485965in}}%
\pgfpathlineto{\pgfqpoint{0.702337in}{1.495614in}}%
\pgfpathlineto{\pgfqpoint{0.702694in}{1.497575in}}%
\pgfpathlineto{\pgfqpoint{0.704197in}{1.505263in}}%
\pgfpathlineto{\pgfqpoint{0.704816in}{1.514912in}}%
\pgfpathlineto{\pgfqpoint{0.704197in}{1.524561in}}%
\pgfpathlineto{\pgfqpoint{0.702694in}{1.532250in}}%
\pgfpathlineto{\pgfqpoint{0.702337in}{1.534211in}}%
\pgfpathlineto{\pgfqpoint{0.699311in}{1.543860in}}%
\pgfpathlineto{\pgfqpoint{0.694664in}{1.553509in}}%
\pgfpathlineto{\pgfqpoint{0.692982in}{1.556184in}}%
\pgfpathlineto{\pgfqpoint{0.688271in}{1.563158in}}%
\pgfpathlineto{\pgfqpoint{0.683271in}{1.568947in}}%
\pgfpathlineto{\pgfqpoint{0.679385in}{1.572807in}}%
\pgfpathlineto{\pgfqpoint{0.673559in}{1.577771in}}%
\pgfpathlineto{\pgfqpoint{0.666562in}{1.582456in}}%
\pgfpathlineto{\pgfqpoint{0.663847in}{1.584145in}}%
\pgfpathlineto{\pgfqpoint{0.654135in}{1.589103in}}%
\pgfpathlineto{\pgfqpoint{0.646630in}{1.592105in}}%
\pgfpathlineto{\pgfqpoint{0.644424in}{1.594121in}}%
\pgfpathlineto{\pgfqpoint{0.643624in}{1.601754in}}%
\pgfpathlineto{\pgfqpoint{0.634712in}{1.608881in}}%
\pgfpathlineto{\pgfqpoint{0.634339in}{1.611404in}}%
\pgfpathlineto{\pgfqpoint{0.627489in}{1.621053in}}%
\pgfpathlineto{\pgfqpoint{0.633558in}{1.630702in}}%
\pgfpathlineto{\pgfqpoint{0.634712in}{1.636024in}}%
\pgfpathlineto{\pgfqpoint{0.643675in}{1.640351in}}%
\pgfpathlineto{\pgfqpoint{0.644424in}{1.641986in}}%
\pgfpathlineto{\pgfqpoint{0.648686in}{1.650000in}}%
\pgfpathlineto{\pgfqpoint{0.646627in}{1.659649in}}%
\pgfpathlineto{\pgfqpoint{0.644424in}{1.662824in}}%
\pgfpathlineto{\pgfqpoint{0.640938in}{1.669298in}}%
\pgfpathlineto{\pgfqpoint{0.636843in}{1.678947in}}%
\pgfpathlineto{\pgfqpoint{0.634712in}{1.680832in}}%
\pgfpathlineto{\pgfqpoint{0.632179in}{1.688596in}}%
\pgfpathlineto{\pgfqpoint{0.628868in}{1.698246in}}%
\pgfpathlineto{\pgfqpoint{0.625000in}{1.698683in}}%
\pgfusepath{stroke}%
\end{pgfscope}%
\begin{pgfscope}%
\pgfpathrectangle{\pgfqpoint{0.625000in}{0.550000in}}{\pgfqpoint{3.875000in}{3.850000in}} %
\pgfusepath{clip}%
\pgfsetbuttcap%
\pgfsetroundjoin%
\pgfsetlinewidth{0.250937pt}%
\definecolor{currentstroke}{rgb}{0.000000,0.000000,0.000000}%
\pgfsetstrokecolor{currentstroke}%
\pgfsetdash{}{0pt}%
\pgfpathmoveto{\pgfqpoint{0.625000in}{1.562652in}}%
\pgfpathlineto{\pgfqpoint{0.634596in}{1.553509in}}%
\pgfpathlineto{\pgfqpoint{0.625000in}{1.544069in}}%
\pgfusepath{stroke}%
\end{pgfscope}%
\begin{pgfscope}%
\pgfpathrectangle{\pgfqpoint{0.625000in}{0.550000in}}{\pgfqpoint{3.875000in}{3.850000in}} %
\pgfusepath{clip}%
\pgfsetbuttcap%
\pgfsetroundjoin%
\pgfsetlinewidth{0.250937pt}%
\definecolor{currentstroke}{rgb}{0.000000,0.000000,0.000000}%
\pgfsetstrokecolor{currentstroke}%
\pgfsetdash{}{0pt}%
\pgfpathmoveto{\pgfqpoint{0.625000in}{1.717049in}}%
\pgfpathlineto{\pgfqpoint{0.629175in}{1.717544in}}%
\pgfpathlineto{\pgfqpoint{0.632179in}{1.727193in}}%
\pgfpathlineto{\pgfqpoint{0.634712in}{1.735034in}}%
\pgfpathlineto{\pgfqpoint{0.636761in}{1.736842in}}%
\pgfpathlineto{\pgfqpoint{0.641578in}{1.746491in}}%
\pgfpathlineto{\pgfqpoint{0.644424in}{1.754377in}}%
\pgfpathlineto{\pgfqpoint{0.645440in}{1.756140in}}%
\pgfpathlineto{\pgfqpoint{0.644424in}{1.761469in}}%
\pgfpathlineto{\pgfqpoint{0.643649in}{1.765789in}}%
\pgfpathlineto{\pgfqpoint{0.634712in}{1.769100in}}%
\pgfpathlineto{\pgfqpoint{0.632866in}{1.775439in}}%
\pgfpathlineto{\pgfqpoint{0.634374in}{1.785088in}}%
\pgfpathlineto{\pgfqpoint{0.634712in}{1.787936in}}%
\pgfpathlineto{\pgfqpoint{0.644424in}{1.789010in}}%
\pgfpathlineto{\pgfqpoint{0.654135in}{1.792390in}}%
\pgfpathlineto{\pgfqpoint{0.657319in}{1.794737in}}%
\pgfpathlineto{\pgfqpoint{0.663847in}{1.800487in}}%
\pgfpathlineto{\pgfqpoint{0.667221in}{1.804386in}}%
\pgfpathlineto{\pgfqpoint{0.673100in}{1.814035in}}%
\pgfpathlineto{\pgfqpoint{0.673559in}{1.815281in}}%
\pgfpathlineto{\pgfqpoint{0.676716in}{1.823684in}}%
\pgfpathlineto{\pgfqpoint{0.678267in}{1.833333in}}%
\pgfpathlineto{\pgfqpoint{0.677961in}{1.842982in}}%
\pgfpathlineto{\pgfqpoint{0.675748in}{1.852632in}}%
\pgfpathlineto{\pgfqpoint{0.673559in}{1.857636in}}%
\pgfpathlineto{\pgfqpoint{0.671498in}{1.862281in}}%
\pgfpathlineto{\pgfqpoint{0.664457in}{1.871930in}}%
\pgfpathlineto{\pgfqpoint{0.663847in}{1.872576in}}%
\pgfpathlineto{\pgfqpoint{0.654135in}{1.880996in}}%
\pgfpathlineto{\pgfqpoint{0.652921in}{1.881579in}}%
\pgfpathlineto{\pgfqpoint{0.644424in}{1.888919in}}%
\pgfpathlineto{\pgfqpoint{0.641615in}{1.891228in}}%
\pgfpathlineto{\pgfqpoint{0.642888in}{1.900877in}}%
\pgfpathlineto{\pgfqpoint{0.635618in}{1.910526in}}%
\pgfpathlineto{\pgfqpoint{0.634712in}{1.910798in}}%
\pgfpathlineto{\pgfqpoint{0.630327in}{1.920175in}}%
\pgfpathlineto{\pgfqpoint{0.625000in}{1.925852in}}%
\pgfusepath{stroke}%
\end{pgfscope}%
\begin{pgfscope}%
\pgfpathrectangle{\pgfqpoint{0.625000in}{0.550000in}}{\pgfqpoint{3.875000in}{3.850000in}} %
\pgfusepath{clip}%
\pgfsetbuttcap%
\pgfsetroundjoin%
\pgfsetlinewidth{0.250937pt}%
\definecolor{currentstroke}{rgb}{0.000000,0.000000,0.000000}%
\pgfsetstrokecolor{currentstroke}%
\pgfsetdash{}{0pt}%
\pgfpathmoveto{\pgfqpoint{0.625000in}{1.871530in}}%
\pgfpathlineto{\pgfqpoint{0.634599in}{1.862281in}}%
\pgfpathlineto{\pgfqpoint{0.625000in}{1.853223in}}%
\pgfusepath{stroke}%
\end{pgfscope}%
\begin{pgfscope}%
\pgfpathrectangle{\pgfqpoint{0.625000in}{0.550000in}}{\pgfqpoint{3.875000in}{3.850000in}} %
\pgfusepath{clip}%
\pgfsetbuttcap%
\pgfsetroundjoin%
\pgfsetlinewidth{0.250937pt}%
\definecolor{currentstroke}{rgb}{0.000000,0.000000,0.000000}%
\pgfsetstrokecolor{currentstroke}%
\pgfsetdash{}{0pt}%
\pgfpathmoveto{\pgfqpoint{0.625000in}{1.893988in}}%
\pgfpathlineto{\pgfqpoint{0.634443in}{1.891228in}}%
\pgfpathlineto{\pgfqpoint{0.625000in}{1.886826in}}%
\pgfusepath{stroke}%
\end{pgfscope}%
\begin{pgfscope}%
\pgfpathrectangle{\pgfqpoint{0.625000in}{0.550000in}}{\pgfqpoint{3.875000in}{3.850000in}} %
\pgfusepath{clip}%
\pgfsetbuttcap%
\pgfsetroundjoin%
\pgfsetlinewidth{0.250937pt}%
\definecolor{currentstroke}{rgb}{0.000000,0.000000,0.000000}%
\pgfsetstrokecolor{currentstroke}%
\pgfsetdash{}{0pt}%
\pgfpathmoveto{\pgfqpoint{0.625000in}{1.932688in}}%
\pgfpathlineto{\pgfqpoint{0.632619in}{1.939474in}}%
\pgfpathlineto{\pgfqpoint{0.634712in}{1.947117in}}%
\pgfpathlineto{\pgfqpoint{0.638886in}{1.949123in}}%
\pgfpathlineto{\pgfqpoint{0.643557in}{1.958772in}}%
\pgfpathlineto{\pgfqpoint{0.644424in}{1.960500in}}%
\pgfpathlineto{\pgfqpoint{0.652127in}{1.968421in}}%
\pgfpathlineto{\pgfqpoint{0.654135in}{1.970526in}}%
\pgfpathlineto{\pgfqpoint{0.659969in}{1.978070in}}%
\pgfpathlineto{\pgfqpoint{0.663693in}{1.987719in}}%
\pgfpathlineto{\pgfqpoint{0.663847in}{1.989079in}}%
\pgfpathlineto{\pgfqpoint{0.664955in}{1.997368in}}%
\pgfpathlineto{\pgfqpoint{0.663847in}{2.005546in}}%
\pgfpathlineto{\pgfqpoint{0.663679in}{2.007018in}}%
\pgfpathlineto{\pgfqpoint{0.660015in}{2.016667in}}%
\pgfpathlineto{\pgfqpoint{0.654135in}{2.024118in}}%
\pgfpathlineto{\pgfqpoint{0.651977in}{2.026316in}}%
\pgfpathlineto{\pgfqpoint{0.644424in}{2.030999in}}%
\pgfpathlineto{\pgfqpoint{0.640716in}{2.035965in}}%
\pgfpathlineto{\pgfqpoint{0.634712in}{2.037616in}}%
\pgfpathlineto{\pgfqpoint{0.631800in}{2.045614in}}%
\pgfpathlineto{\pgfqpoint{0.631565in}{2.055263in}}%
\pgfpathlineto{\pgfqpoint{0.634712in}{2.063904in}}%
\pgfpathlineto{\pgfqpoint{0.639255in}{2.064912in}}%
\pgfpathlineto{\pgfqpoint{0.644424in}{2.070263in}}%
\pgfpathlineto{\pgfqpoint{0.650362in}{2.074561in}}%
\pgfpathlineto{\pgfqpoint{0.654135in}{2.081278in}}%
\pgfpathlineto{\pgfqpoint{0.655608in}{2.084211in}}%
\pgfpathlineto{\pgfqpoint{0.657103in}{2.093860in}}%
\pgfpathlineto{\pgfqpoint{0.655495in}{2.103509in}}%
\pgfpathlineto{\pgfqpoint{0.654135in}{2.106589in}}%
\pgfpathlineto{\pgfqpoint{0.650609in}{2.113158in}}%
\pgfpathlineto{\pgfqpoint{0.644424in}{2.118222in}}%
\pgfpathlineto{\pgfqpoint{0.634712in}{2.119325in}}%
\pgfpathlineto{\pgfqpoint{0.633976in}{2.122807in}}%
\pgfpathlineto{\pgfqpoint{0.633072in}{2.132456in}}%
\pgfpathlineto{\pgfqpoint{0.634712in}{2.138386in}}%
\pgfpathlineto{\pgfqpoint{0.644424in}{2.140176in}}%
\pgfpathlineto{\pgfqpoint{0.646445in}{2.142105in}}%
\pgfpathlineto{\pgfqpoint{0.651207in}{2.151754in}}%
\pgfpathlineto{\pgfqpoint{0.651509in}{2.161404in}}%
\pgfpathlineto{\pgfqpoint{0.648646in}{2.171053in}}%
\pgfpathlineto{\pgfqpoint{0.644424in}{2.178640in}}%
\pgfpathlineto{\pgfqpoint{0.634712in}{2.178230in}}%
\pgfpathlineto{\pgfqpoint{0.634626in}{2.171053in}}%
\pgfpathlineto{\pgfqpoint{0.625000in}{2.161690in}}%
\pgfusepath{stroke}%
\end{pgfscope}%
\begin{pgfscope}%
\pgfpathrectangle{\pgfqpoint{0.625000in}{0.550000in}}{\pgfqpoint{3.875000in}{3.850000in}} %
\pgfusepath{clip}%
\pgfsetbuttcap%
\pgfsetroundjoin%
\pgfsetlinewidth{0.250937pt}%
\definecolor{currentstroke}{rgb}{0.000000,0.000000,0.000000}%
\pgfsetstrokecolor{currentstroke}%
\pgfsetdash{}{0pt}%
\pgfpathmoveto{\pgfqpoint{0.625000in}{2.026263in}}%
\pgfpathlineto{\pgfqpoint{0.634605in}{2.016667in}}%
\pgfpathlineto{\pgfqpoint{0.625000in}{2.007149in}}%
\pgfusepath{stroke}%
\end{pgfscope}%
\begin{pgfscope}%
\pgfpathrectangle{\pgfqpoint{0.625000in}{0.550000in}}{\pgfqpoint{3.875000in}{3.850000in}} %
\pgfusepath{clip}%
\pgfsetbuttcap%
\pgfsetroundjoin%
\pgfsetlinewidth{0.250937pt}%
\definecolor{currentstroke}{rgb}{0.000000,0.000000,0.000000}%
\pgfsetstrokecolor{currentstroke}%
\pgfsetdash{}{0pt}%
\pgfpathmoveto{\pgfqpoint{0.625000in}{2.180155in}}%
\pgfpathlineto{\pgfqpoint{0.634238in}{2.180702in}}%
\pgfpathlineto{\pgfqpoint{0.634712in}{2.184736in}}%
\pgfpathlineto{\pgfqpoint{0.644424in}{2.188920in}}%
\pgfpathlineto{\pgfqpoint{0.644643in}{2.190351in}}%
\pgfpathlineto{\pgfqpoint{0.647591in}{2.200000in}}%
\pgfpathlineto{\pgfqpoint{0.646983in}{2.209649in}}%
\pgfpathlineto{\pgfqpoint{0.644424in}{2.218234in}}%
\pgfpathlineto{\pgfqpoint{0.643668in}{2.219298in}}%
\pgfpathlineto{\pgfqpoint{0.643665in}{2.228947in}}%
\pgfpathlineto{\pgfqpoint{0.644401in}{2.238596in}}%
\pgfpathlineto{\pgfqpoint{0.642654in}{2.248246in}}%
\pgfpathlineto{\pgfqpoint{0.641746in}{2.257895in}}%
\pgfpathlineto{\pgfqpoint{0.642048in}{2.267544in}}%
\pgfpathlineto{\pgfqpoint{0.639772in}{2.277193in}}%
\pgfpathlineto{\pgfqpoint{0.640251in}{2.286842in}}%
\pgfpathlineto{\pgfqpoint{0.638465in}{2.296491in}}%
\pgfpathlineto{\pgfqpoint{0.638274in}{2.306140in}}%
\pgfpathlineto{\pgfqpoint{0.637109in}{2.315789in}}%
\pgfpathlineto{\pgfqpoint{0.636131in}{2.325439in}}%
\pgfpathlineto{\pgfqpoint{0.635306in}{2.335088in}}%
\pgfpathlineto{\pgfqpoint{0.634712in}{2.341070in}}%
\pgfpathlineto{\pgfqpoint{0.634482in}{2.344737in}}%
\pgfpathlineto{\pgfqpoint{0.633901in}{2.354386in}}%
\pgfpathlineto{\pgfqpoint{0.633313in}{2.364035in}}%
\pgfpathlineto{\pgfqpoint{0.631106in}{2.373684in}}%
\pgfpathlineto{\pgfqpoint{0.632291in}{2.383333in}}%
\pgfpathlineto{\pgfqpoint{0.630889in}{2.392982in}}%
\pgfpathlineto{\pgfqpoint{0.630445in}{2.402632in}}%
\pgfpathlineto{\pgfqpoint{0.626203in}{2.412281in}}%
\pgfpathlineto{\pgfqpoint{0.627488in}{2.421930in}}%
\pgfpathlineto{\pgfqpoint{0.627002in}{2.431579in}}%
\pgfpathlineto{\pgfqpoint{0.626468in}{2.441228in}}%
\pgfpathlineto{\pgfqpoint{0.625943in}{2.450877in}}%
\pgfpathlineto{\pgfqpoint{0.625333in}{2.460526in}}%
\pgfpathlineto{\pgfqpoint{0.625000in}{2.464936in}}%
\pgfusepath{stroke}%
\end{pgfscope}%
\begin{pgfscope}%
\pgfpathrectangle{\pgfqpoint{0.625000in}{0.550000in}}{\pgfqpoint{3.875000in}{3.850000in}} %
\pgfusepath{clip}%
\pgfsetbuttcap%
\pgfsetroundjoin%
\pgfsetlinewidth{0.250937pt}%
\definecolor{currentstroke}{rgb}{0.000000,0.000000,0.000000}%
\pgfsetstrokecolor{currentstroke}%
\pgfsetdash{}{0pt}%
\pgfpathmoveto{\pgfqpoint{0.625000in}{2.334507in}}%
\pgfpathlineto{\pgfqpoint{0.634681in}{2.325439in}}%
\pgfpathlineto{\pgfqpoint{0.630244in}{2.315789in}}%
\pgfpathlineto{\pgfqpoint{0.625000in}{2.314559in}}%
\pgfusepath{stroke}%
\end{pgfscope}%
\begin{pgfscope}%
\pgfpathrectangle{\pgfqpoint{0.625000in}{0.550000in}}{\pgfqpoint{3.875000in}{3.850000in}} %
\pgfusepath{clip}%
\pgfsetbuttcap%
\pgfsetroundjoin%
\pgfsetlinewidth{0.250937pt}%
\definecolor{currentstroke}{rgb}{0.000000,0.000000,0.000000}%
\pgfsetstrokecolor{currentstroke}%
\pgfsetdash{}{0pt}%
\pgfpathmoveto{\pgfqpoint{0.625000in}{2.494713in}}%
\pgfpathlineto{\pgfqpoint{0.625332in}{2.499123in}}%
\pgfpathlineto{\pgfqpoint{0.625943in}{2.508772in}}%
\pgfpathlineto{\pgfqpoint{0.626468in}{2.518421in}}%
\pgfpathlineto{\pgfqpoint{0.627002in}{2.528070in}}%
\pgfpathlineto{\pgfqpoint{0.627488in}{2.537719in}}%
\pgfpathlineto{\pgfqpoint{0.626203in}{2.547368in}}%
\pgfpathlineto{\pgfqpoint{0.630445in}{2.557018in}}%
\pgfpathlineto{\pgfqpoint{0.630889in}{2.566667in}}%
\pgfpathlineto{\pgfqpoint{0.632291in}{2.576316in}}%
\pgfpathlineto{\pgfqpoint{0.631106in}{2.585965in}}%
\pgfpathlineto{\pgfqpoint{0.633313in}{2.595614in}}%
\pgfpathlineto{\pgfqpoint{0.633901in}{2.605263in}}%
\pgfpathlineto{\pgfqpoint{0.634482in}{2.614912in}}%
\pgfpathlineto{\pgfqpoint{0.634712in}{2.618579in}}%
\pgfpathlineto{\pgfqpoint{0.635306in}{2.624561in}}%
\pgfpathlineto{\pgfqpoint{0.636131in}{2.634211in}}%
\pgfpathlineto{\pgfqpoint{0.637109in}{2.643860in}}%
\pgfpathlineto{\pgfqpoint{0.638274in}{2.653509in}}%
\pgfpathlineto{\pgfqpoint{0.638465in}{2.663158in}}%
\pgfpathlineto{\pgfqpoint{0.640251in}{2.672807in}}%
\pgfpathlineto{\pgfqpoint{0.639772in}{2.682456in}}%
\pgfpathlineto{\pgfqpoint{0.642048in}{2.692105in}}%
\pgfpathlineto{\pgfqpoint{0.641746in}{2.701754in}}%
\pgfpathlineto{\pgfqpoint{0.642654in}{2.711404in}}%
\pgfpathlineto{\pgfqpoint{0.644401in}{2.721053in}}%
\pgfpathlineto{\pgfqpoint{0.643665in}{2.730702in}}%
\pgfpathlineto{\pgfqpoint{0.643668in}{2.740351in}}%
\pgfpathlineto{\pgfqpoint{0.644424in}{2.741415in}}%
\pgfpathlineto{\pgfqpoint{0.646983in}{2.750000in}}%
\pgfpathlineto{\pgfqpoint{0.647591in}{2.759649in}}%
\pgfpathlineto{\pgfqpoint{0.644643in}{2.769298in}}%
\pgfpathlineto{\pgfqpoint{0.644424in}{2.770730in}}%
\pgfpathlineto{\pgfqpoint{0.634712in}{2.774913in}}%
\pgfpathlineto{\pgfqpoint{0.634238in}{2.778947in}}%
\pgfpathlineto{\pgfqpoint{0.625000in}{2.779498in}}%
\pgfusepath{stroke}%
\end{pgfscope}%
\begin{pgfscope}%
\pgfpathrectangle{\pgfqpoint{0.625000in}{0.550000in}}{\pgfqpoint{3.875000in}{3.850000in}} %
\pgfusepath{clip}%
\pgfsetbuttcap%
\pgfsetroundjoin%
\pgfsetlinewidth{0.250937pt}%
\definecolor{currentstroke}{rgb}{0.000000,0.000000,0.000000}%
\pgfsetstrokecolor{currentstroke}%
\pgfsetdash{}{0pt}%
\pgfpathmoveto{\pgfqpoint{0.625000in}{2.645090in}}%
\pgfpathlineto{\pgfqpoint{0.630244in}{2.643860in}}%
\pgfpathlineto{\pgfqpoint{0.634681in}{2.634211in}}%
\pgfpathlineto{\pgfqpoint{0.625000in}{2.625143in}}%
\pgfusepath{stroke}%
\end{pgfscope}%
\begin{pgfscope}%
\pgfpathrectangle{\pgfqpoint{0.625000in}{0.550000in}}{\pgfqpoint{3.875000in}{3.850000in}} %
\pgfusepath{clip}%
\pgfsetbuttcap%
\pgfsetroundjoin%
\pgfsetlinewidth{0.250937pt}%
\definecolor{currentstroke}{rgb}{0.000000,0.000000,0.000000}%
\pgfsetstrokecolor{currentstroke}%
\pgfsetdash{}{0pt}%
\pgfpathmoveto{\pgfqpoint{0.625000in}{2.797957in}}%
\pgfpathlineto{\pgfqpoint{0.634625in}{2.788596in}}%
\pgfpathlineto{\pgfqpoint{0.634712in}{2.781419in}}%
\pgfpathlineto{\pgfqpoint{0.644424in}{2.781009in}}%
\pgfpathlineto{\pgfqpoint{0.648646in}{2.788596in}}%
\pgfpathlineto{\pgfqpoint{0.651509in}{2.798246in}}%
\pgfpathlineto{\pgfqpoint{0.651207in}{2.807895in}}%
\pgfpathlineto{\pgfqpoint{0.646445in}{2.817544in}}%
\pgfpathlineto{\pgfqpoint{0.644424in}{2.819473in}}%
\pgfpathlineto{\pgfqpoint{0.634712in}{2.821264in}}%
\pgfpathlineto{\pgfqpoint{0.633072in}{2.827193in}}%
\pgfpathlineto{\pgfqpoint{0.633976in}{2.836842in}}%
\pgfpathlineto{\pgfqpoint{0.634712in}{2.840324in}}%
\pgfpathlineto{\pgfqpoint{0.644424in}{2.841427in}}%
\pgfpathlineto{\pgfqpoint{0.650609in}{2.846491in}}%
\pgfpathlineto{\pgfqpoint{0.654135in}{2.853061in}}%
\pgfpathlineto{\pgfqpoint{0.655495in}{2.856140in}}%
\pgfpathlineto{\pgfqpoint{0.657103in}{2.865789in}}%
\pgfpathlineto{\pgfqpoint{0.655608in}{2.875439in}}%
\pgfpathlineto{\pgfqpoint{0.654135in}{2.878371in}}%
\pgfpathlineto{\pgfqpoint{0.650362in}{2.885088in}}%
\pgfpathlineto{\pgfqpoint{0.644424in}{2.889386in}}%
\pgfpathlineto{\pgfqpoint{0.639255in}{2.894737in}}%
\pgfpathlineto{\pgfqpoint{0.634712in}{2.895746in}}%
\pgfpathlineto{\pgfqpoint{0.631565in}{2.904386in}}%
\pgfpathlineto{\pgfqpoint{0.631800in}{2.914035in}}%
\pgfpathlineto{\pgfqpoint{0.634712in}{2.922034in}}%
\pgfpathlineto{\pgfqpoint{0.640716in}{2.923684in}}%
\pgfpathlineto{\pgfqpoint{0.644424in}{2.928650in}}%
\pgfpathlineto{\pgfqpoint{0.651977in}{2.933333in}}%
\pgfpathlineto{\pgfqpoint{0.654135in}{2.935531in}}%
\pgfpathlineto{\pgfqpoint{0.660015in}{2.942982in}}%
\pgfpathlineto{\pgfqpoint{0.663679in}{2.952632in}}%
\pgfpathlineto{\pgfqpoint{0.663847in}{2.954103in}}%
\pgfpathlineto{\pgfqpoint{0.664955in}{2.962281in}}%
\pgfpathlineto{\pgfqpoint{0.663847in}{2.970570in}}%
\pgfpathlineto{\pgfqpoint{0.663693in}{2.971930in}}%
\pgfpathlineto{\pgfqpoint{0.659969in}{2.981579in}}%
\pgfpathlineto{\pgfqpoint{0.654135in}{2.989123in}}%
\pgfpathlineto{\pgfqpoint{0.652127in}{2.991228in}}%
\pgfpathlineto{\pgfqpoint{0.644424in}{2.999150in}}%
\pgfpathlineto{\pgfqpoint{0.643557in}{3.000877in}}%
\pgfpathlineto{\pgfqpoint{0.638886in}{3.010526in}}%
\pgfpathlineto{\pgfqpoint{0.634712in}{3.012532in}}%
\pgfpathlineto{\pgfqpoint{0.632619in}{3.020175in}}%
\pgfpathlineto{\pgfqpoint{0.625000in}{3.026961in}}%
\pgfusepath{stroke}%
\end{pgfscope}%
\begin{pgfscope}%
\pgfpathrectangle{\pgfqpoint{0.625000in}{0.550000in}}{\pgfqpoint{3.875000in}{3.850000in}} %
\pgfusepath{clip}%
\pgfsetbuttcap%
\pgfsetroundjoin%
\pgfsetlinewidth{0.250937pt}%
\definecolor{currentstroke}{rgb}{0.000000,0.000000,0.000000}%
\pgfsetstrokecolor{currentstroke}%
\pgfsetdash{}{0pt}%
\pgfpathmoveto{\pgfqpoint{0.625000in}{2.952498in}}%
\pgfpathlineto{\pgfqpoint{0.634604in}{2.942982in}}%
\pgfpathlineto{\pgfqpoint{0.625000in}{2.933387in}}%
\pgfusepath{stroke}%
\end{pgfscope}%
\begin{pgfscope}%
\pgfpathrectangle{\pgfqpoint{0.625000in}{0.550000in}}{\pgfqpoint{3.875000in}{3.850000in}} %
\pgfusepath{clip}%
\pgfsetbuttcap%
\pgfsetroundjoin%
\pgfsetlinewidth{0.250937pt}%
\definecolor{currentstroke}{rgb}{0.000000,0.000000,0.000000}%
\pgfsetstrokecolor{currentstroke}%
\pgfsetdash{}{0pt}%
\pgfpathmoveto{\pgfqpoint{0.625000in}{3.033798in}}%
\pgfpathlineto{\pgfqpoint{0.630327in}{3.039474in}}%
\pgfpathlineto{\pgfqpoint{0.634712in}{3.048851in}}%
\pgfpathlineto{\pgfqpoint{0.635618in}{3.049123in}}%
\pgfpathlineto{\pgfqpoint{0.642888in}{3.058772in}}%
\pgfpathlineto{\pgfqpoint{0.641615in}{3.068421in}}%
\pgfpathlineto{\pgfqpoint{0.644424in}{3.070731in}}%
\pgfpathlineto{\pgfqpoint{0.652921in}{3.078070in}}%
\pgfpathlineto{\pgfqpoint{0.654135in}{3.078654in}}%
\pgfpathlineto{\pgfqpoint{0.663847in}{3.087073in}}%
\pgfpathlineto{\pgfqpoint{0.664457in}{3.087719in}}%
\pgfpathlineto{\pgfqpoint{0.671498in}{3.097368in}}%
\pgfpathlineto{\pgfqpoint{0.673559in}{3.102013in}}%
\pgfpathlineto{\pgfqpoint{0.675748in}{3.107018in}}%
\pgfpathlineto{\pgfqpoint{0.677961in}{3.116667in}}%
\pgfpathlineto{\pgfqpoint{0.678267in}{3.126316in}}%
\pgfpathlineto{\pgfqpoint{0.676716in}{3.135965in}}%
\pgfpathlineto{\pgfqpoint{0.673559in}{3.144368in}}%
\pgfpathlineto{\pgfqpoint{0.673100in}{3.145614in}}%
\pgfpathlineto{\pgfqpoint{0.667221in}{3.155263in}}%
\pgfpathlineto{\pgfqpoint{0.663847in}{3.159162in}}%
\pgfpathlineto{\pgfqpoint{0.657319in}{3.164912in}}%
\pgfpathlineto{\pgfqpoint{0.654135in}{3.167259in}}%
\pgfpathlineto{\pgfqpoint{0.644424in}{3.170640in}}%
\pgfpathlineto{\pgfqpoint{0.634712in}{3.171713in}}%
\pgfpathlineto{\pgfqpoint{0.634374in}{3.174561in}}%
\pgfpathlineto{\pgfqpoint{0.632866in}{3.184211in}}%
\pgfpathlineto{\pgfqpoint{0.634712in}{3.190550in}}%
\pgfpathlineto{\pgfqpoint{0.643649in}{3.193860in}}%
\pgfpathlineto{\pgfqpoint{0.644424in}{3.198180in}}%
\pgfpathlineto{\pgfqpoint{0.645440in}{3.203509in}}%
\pgfpathlineto{\pgfqpoint{0.644424in}{3.205272in}}%
\pgfpathlineto{\pgfqpoint{0.641578in}{3.213158in}}%
\pgfpathlineto{\pgfqpoint{0.636761in}{3.222807in}}%
\pgfpathlineto{\pgfqpoint{0.634712in}{3.224615in}}%
\pgfpathlineto{\pgfqpoint{0.632179in}{3.232456in}}%
\pgfpathlineto{\pgfqpoint{0.629175in}{3.242105in}}%
\pgfpathlineto{\pgfqpoint{0.625000in}{3.242624in}}%
\pgfusepath{stroke}%
\end{pgfscope}%
\begin{pgfscope}%
\pgfpathrectangle{\pgfqpoint{0.625000in}{0.550000in}}{\pgfqpoint{3.875000in}{3.850000in}} %
\pgfusepath{clip}%
\pgfsetbuttcap%
\pgfsetroundjoin%
\pgfsetlinewidth{0.250937pt}%
\definecolor{currentstroke}{rgb}{0.000000,0.000000,0.000000}%
\pgfsetstrokecolor{currentstroke}%
\pgfsetdash{}{0pt}%
\pgfpathmoveto{\pgfqpoint{0.625000in}{3.072823in}}%
\pgfpathlineto{\pgfqpoint{0.634443in}{3.068421in}}%
\pgfpathlineto{\pgfqpoint{0.625000in}{3.065661in}}%
\pgfusepath{stroke}%
\end{pgfscope}%
\begin{pgfscope}%
\pgfpathrectangle{\pgfqpoint{0.625000in}{0.550000in}}{\pgfqpoint{3.875000in}{3.850000in}} %
\pgfusepath{clip}%
\pgfsetbuttcap%
\pgfsetroundjoin%
\pgfsetlinewidth{0.250937pt}%
\definecolor{currentstroke}{rgb}{0.000000,0.000000,0.000000}%
\pgfsetstrokecolor{currentstroke}%
\pgfsetdash{}{0pt}%
\pgfpathmoveto{\pgfqpoint{0.625000in}{3.106419in}}%
\pgfpathlineto{\pgfqpoint{0.634598in}{3.097368in}}%
\pgfpathlineto{\pgfqpoint{0.625000in}{3.088125in}}%
\pgfusepath{stroke}%
\end{pgfscope}%
\begin{pgfscope}%
\pgfpathrectangle{\pgfqpoint{0.625000in}{0.550000in}}{\pgfqpoint{3.875000in}{3.850000in}} %
\pgfusepath{clip}%
\pgfsetbuttcap%
\pgfsetroundjoin%
\pgfsetlinewidth{0.250937pt}%
\definecolor{currentstroke}{rgb}{0.000000,0.000000,0.000000}%
\pgfsetstrokecolor{currentstroke}%
\pgfsetdash{}{0pt}%
\pgfpathmoveto{\pgfqpoint{0.625000in}{3.260946in}}%
\pgfpathlineto{\pgfqpoint{0.628868in}{3.261404in}}%
\pgfpathlineto{\pgfqpoint{0.632179in}{3.271053in}}%
\pgfpathlineto{\pgfqpoint{0.634712in}{3.278817in}}%
\pgfpathlineto{\pgfqpoint{0.636843in}{3.280702in}}%
\pgfpathlineto{\pgfqpoint{0.640938in}{3.290351in}}%
\pgfpathlineto{\pgfqpoint{0.644424in}{3.296825in}}%
\pgfpathlineto{\pgfqpoint{0.646627in}{3.300000in}}%
\pgfpathlineto{\pgfqpoint{0.648686in}{3.309649in}}%
\pgfpathlineto{\pgfqpoint{0.644424in}{3.317663in}}%
\pgfpathlineto{\pgfqpoint{0.643675in}{3.319298in}}%
\pgfpathlineto{\pgfqpoint{0.634712in}{3.323625in}}%
\pgfpathlineto{\pgfqpoint{0.633558in}{3.328947in}}%
\pgfpathlineto{\pgfqpoint{0.627489in}{3.338596in}}%
\pgfpathlineto{\pgfqpoint{0.634339in}{3.348246in}}%
\pgfpathlineto{\pgfqpoint{0.634712in}{3.350768in}}%
\pgfpathlineto{\pgfqpoint{0.643624in}{3.357895in}}%
\pgfpathlineto{\pgfqpoint{0.644424in}{3.365528in}}%
\pgfpathlineto{\pgfqpoint{0.646630in}{3.367544in}}%
\pgfpathlineto{\pgfqpoint{0.654135in}{3.370546in}}%
\pgfpathlineto{\pgfqpoint{0.663847in}{3.375504in}}%
\pgfpathlineto{\pgfqpoint{0.666562in}{3.377193in}}%
\pgfpathlineto{\pgfqpoint{0.673559in}{3.381878in}}%
\pgfpathlineto{\pgfqpoint{0.679385in}{3.386842in}}%
\pgfpathlineto{\pgfqpoint{0.683271in}{3.390703in}}%
\pgfpathlineto{\pgfqpoint{0.688271in}{3.396491in}}%
\pgfpathlineto{\pgfqpoint{0.692982in}{3.403465in}}%
\pgfpathlineto{\pgfqpoint{0.694664in}{3.406140in}}%
\pgfpathlineto{\pgfqpoint{0.699311in}{3.415789in}}%
\pgfpathlineto{\pgfqpoint{0.702337in}{3.425439in}}%
\pgfpathlineto{\pgfqpoint{0.702694in}{3.427399in}}%
\pgfpathlineto{\pgfqpoint{0.704197in}{3.435088in}}%
\pgfpathlineto{\pgfqpoint{0.704816in}{3.444737in}}%
\pgfpathlineto{\pgfqpoint{0.704197in}{3.454386in}}%
\pgfpathlineto{\pgfqpoint{0.702694in}{3.462075in}}%
\pgfpathlineto{\pgfqpoint{0.702337in}{3.464035in}}%
\pgfpathlineto{\pgfqpoint{0.699311in}{3.473684in}}%
\pgfpathlineto{\pgfqpoint{0.694664in}{3.483333in}}%
\pgfpathlineto{\pgfqpoint{0.692982in}{3.486008in}}%
\pgfpathlineto{\pgfqpoint{0.688271in}{3.492982in}}%
\pgfpathlineto{\pgfqpoint{0.683271in}{3.498772in}}%
\pgfpathlineto{\pgfqpoint{0.679386in}{3.502632in}}%
\pgfpathlineto{\pgfqpoint{0.673559in}{3.507603in}}%
\pgfpathlineto{\pgfqpoint{0.666517in}{3.512281in}}%
\pgfpathlineto{\pgfqpoint{0.663847in}{3.513933in}}%
\pgfpathlineto{\pgfqpoint{0.654135in}{3.518247in}}%
\pgfpathlineto{\pgfqpoint{0.644424in}{3.520727in}}%
\pgfpathlineto{\pgfqpoint{0.643594in}{3.521930in}}%
\pgfpathlineto{\pgfqpoint{0.634712in}{3.523861in}}%
\pgfpathlineto{\pgfqpoint{0.628840in}{3.531579in}}%
\pgfpathlineto{\pgfqpoint{0.634712in}{3.540437in}}%
\pgfpathlineto{\pgfqpoint{0.638657in}{3.541228in}}%
\pgfpathlineto{\pgfqpoint{0.644102in}{3.550877in}}%
\pgfpathlineto{\pgfqpoint{0.639899in}{3.560526in}}%
\pgfpathlineto{\pgfqpoint{0.634712in}{3.563538in}}%
\pgfpathlineto{\pgfqpoint{0.634658in}{3.560526in}}%
\pgfpathlineto{\pgfqpoint{0.625000in}{3.551355in}}%
\pgfusepath{stroke}%
\end{pgfscope}%
\begin{pgfscope}%
\pgfpathrectangle{\pgfqpoint{0.625000in}{0.550000in}}{\pgfqpoint{3.875000in}{3.850000in}} %
\pgfusepath{clip}%
\pgfsetbuttcap%
\pgfsetroundjoin%
\pgfsetlinewidth{0.250937pt}%
\definecolor{currentstroke}{rgb}{0.000000,0.000000,0.000000}%
\pgfsetstrokecolor{currentstroke}%
\pgfsetdash{}{0pt}%
\pgfpathmoveto{\pgfqpoint{0.625000in}{3.415573in}}%
\pgfpathlineto{\pgfqpoint{0.634592in}{3.406140in}}%
\pgfpathlineto{\pgfqpoint{0.625000in}{3.397014in}}%
\pgfusepath{stroke}%
\end{pgfscope}%
\begin{pgfscope}%
\pgfpathrectangle{\pgfqpoint{0.625000in}{0.550000in}}{\pgfqpoint{3.875000in}{3.850000in}} %
\pgfusepath{clip}%
\pgfsetbuttcap%
\pgfsetroundjoin%
\pgfsetlinewidth{0.250937pt}%
\definecolor{currentstroke}{rgb}{0.000000,0.000000,0.000000}%
\pgfsetstrokecolor{currentstroke}%
\pgfsetdash{}{0pt}%
\pgfpathmoveto{\pgfqpoint{0.625000in}{3.569703in}}%
\pgfpathlineto{\pgfqpoint{0.632840in}{3.570175in}}%
\pgfpathlineto{\pgfqpoint{0.628505in}{3.579825in}}%
\pgfpathlineto{\pgfqpoint{0.630168in}{3.589474in}}%
\pgfpathlineto{\pgfqpoint{0.634469in}{3.599123in}}%
\pgfpathlineto{\pgfqpoint{0.634712in}{3.600812in}}%
\pgfpathlineto{\pgfqpoint{0.642105in}{3.608772in}}%
\pgfpathlineto{\pgfqpoint{0.644424in}{3.612745in}}%
\pgfpathlineto{\pgfqpoint{0.650181in}{3.618421in}}%
\pgfpathlineto{\pgfqpoint{0.654135in}{3.625477in}}%
\pgfpathlineto{\pgfqpoint{0.655589in}{3.628070in}}%
\pgfpathlineto{\pgfqpoint{0.657214in}{3.637719in}}%
\pgfpathlineto{\pgfqpoint{0.655409in}{3.647368in}}%
\pgfpathlineto{\pgfqpoint{0.654135in}{3.650146in}}%
\pgfpathlineto{\pgfqpoint{0.650548in}{3.657018in}}%
\pgfpathlineto{\pgfqpoint{0.644424in}{3.664908in}}%
\pgfpathlineto{\pgfqpoint{0.634712in}{3.663085in}}%
\pgfpathlineto{\pgfqpoint{0.625000in}{3.662580in}}%
\pgfusepath{stroke}%
\end{pgfscope}%
\begin{pgfscope}%
\pgfpathrectangle{\pgfqpoint{0.625000in}{0.550000in}}{\pgfqpoint{3.875000in}{3.850000in}} %
\pgfusepath{clip}%
\pgfsetbuttcap%
\pgfsetroundjoin%
\pgfsetlinewidth{0.250937pt}%
\definecolor{currentstroke}{rgb}{0.000000,0.000000,0.000000}%
\pgfsetstrokecolor{currentstroke}%
\pgfsetdash{}{0pt}%
\pgfpathmoveto{\pgfqpoint{0.625000in}{3.670101in}}%
\pgfpathlineto{\pgfqpoint{0.634712in}{3.670555in}}%
\pgfpathlineto{\pgfqpoint{0.644188in}{3.676316in}}%
\pgfpathlineto{\pgfqpoint{0.644424in}{3.679880in}}%
\pgfpathlineto{\pgfqpoint{0.644740in}{3.685965in}}%
\pgfpathlineto{\pgfqpoint{0.644424in}{3.686866in}}%
\pgfpathlineto{\pgfqpoint{0.641870in}{3.695614in}}%
\pgfpathlineto{\pgfqpoint{0.639500in}{3.705263in}}%
\pgfpathlineto{\pgfqpoint{0.636631in}{3.714912in}}%
\pgfpathlineto{\pgfqpoint{0.634712in}{3.720097in}}%
\pgfpathlineto{\pgfqpoint{0.634676in}{3.714912in}}%
\pgfpathlineto{\pgfqpoint{0.625000in}{3.705366in}}%
\pgfusepath{stroke}%
\end{pgfscope}%
\begin{pgfscope}%
\pgfpathrectangle{\pgfqpoint{0.625000in}{0.550000in}}{\pgfqpoint{3.875000in}{3.850000in}} %
\pgfusepath{clip}%
\pgfsetbuttcap%
\pgfsetroundjoin%
\pgfsetlinewidth{0.250937pt}%
\definecolor{currentstroke}{rgb}{0.000000,0.000000,0.000000}%
\pgfsetstrokecolor{currentstroke}%
\pgfsetdash{}{0pt}%
\pgfpathmoveto{\pgfqpoint{0.625000in}{3.724179in}}%
\pgfpathlineto{\pgfqpoint{0.634024in}{3.724561in}}%
\pgfpathlineto{\pgfqpoint{0.632582in}{3.734211in}}%
\pgfpathlineto{\pgfqpoint{0.630129in}{3.743860in}}%
\pgfpathlineto{\pgfqpoint{0.627125in}{3.753509in}}%
\pgfpathlineto{\pgfqpoint{0.625000in}{3.761525in}}%
\pgfusepath{stroke}%
\end{pgfscope}%
\begin{pgfscope}%
\pgfpathrectangle{\pgfqpoint{0.625000in}{0.550000in}}{\pgfqpoint{3.875000in}{3.850000in}} %
\pgfusepath{clip}%
\pgfsetbuttcap%
\pgfsetroundjoin%
\pgfsetlinewidth{0.250937pt}%
\definecolor{currentstroke}{rgb}{0.000000,0.000000,0.000000}%
\pgfsetstrokecolor{currentstroke}%
\pgfsetdash{}{0pt}%
\pgfpathmoveto{\pgfqpoint{0.625000in}{3.765214in}}%
\pgfpathlineto{\pgfqpoint{0.625875in}{3.772807in}}%
\pgfpathlineto{\pgfqpoint{0.627304in}{3.782456in}}%
\pgfpathlineto{\pgfqpoint{0.630891in}{3.792105in}}%
\pgfpathlineto{\pgfqpoint{0.632102in}{3.801754in}}%
\pgfpathlineto{\pgfqpoint{0.634673in}{3.811404in}}%
\pgfpathlineto{\pgfqpoint{0.634712in}{3.811719in}}%
\pgfpathlineto{\pgfqpoint{0.637460in}{3.821053in}}%
\pgfpathlineto{\pgfqpoint{0.640770in}{3.830702in}}%
\pgfpathlineto{\pgfqpoint{0.641423in}{3.840351in}}%
\pgfpathlineto{\pgfqpoint{0.644424in}{3.846476in}}%
\pgfpathlineto{\pgfqpoint{0.645902in}{3.850000in}}%
\pgfpathlineto{\pgfqpoint{0.648209in}{3.859649in}}%
\pgfpathlineto{\pgfqpoint{0.645760in}{3.869298in}}%
\pgfpathlineto{\pgfqpoint{0.644424in}{3.870913in}}%
\pgfpathlineto{\pgfqpoint{0.634712in}{3.873981in}}%
\pgfpathlineto{\pgfqpoint{0.634633in}{3.869298in}}%
\pgfpathlineto{\pgfqpoint{0.625000in}{3.860094in}}%
\pgfusepath{stroke}%
\end{pgfscope}%
\begin{pgfscope}%
\pgfpathrectangle{\pgfqpoint{0.625000in}{0.550000in}}{\pgfqpoint{3.875000in}{3.850000in}} %
\pgfusepath{clip}%
\pgfsetbuttcap%
\pgfsetroundjoin%
\pgfsetlinewidth{0.250937pt}%
\definecolor{currentstroke}{rgb}{0.000000,0.000000,0.000000}%
\pgfsetstrokecolor{currentstroke}%
\pgfsetdash{}{0pt}%
\pgfpathmoveto{\pgfqpoint{0.625000in}{3.878465in}}%
\pgfpathlineto{\pgfqpoint{0.633337in}{3.878947in}}%
\pgfpathlineto{\pgfqpoint{0.633345in}{3.888596in}}%
\pgfpathlineto{\pgfqpoint{0.634712in}{3.893469in}}%
\pgfpathlineto{\pgfqpoint{0.644424in}{3.893801in}}%
\pgfpathlineto{\pgfqpoint{0.652189in}{3.898246in}}%
\pgfpathlineto{\pgfqpoint{0.654135in}{3.900069in}}%
\pgfpathlineto{\pgfqpoint{0.659984in}{3.907895in}}%
\pgfpathlineto{\pgfqpoint{0.663665in}{3.917544in}}%
\pgfpathlineto{\pgfqpoint{0.663847in}{3.919099in}}%
\pgfpathlineto{\pgfqpoint{0.664978in}{3.927193in}}%
\pgfpathlineto{\pgfqpoint{0.663847in}{3.935395in}}%
\pgfpathlineto{\pgfqpoint{0.663680in}{3.936842in}}%
\pgfpathlineto{\pgfqpoint{0.659939in}{3.946491in}}%
\pgfpathlineto{\pgfqpoint{0.654135in}{3.954419in}}%
\pgfpathlineto{\pgfqpoint{0.652356in}{3.956140in}}%
\pgfpathlineto{\pgfqpoint{0.644424in}{3.963923in}}%
\pgfpathlineto{\pgfqpoint{0.634712in}{3.961506in}}%
\pgfpathlineto{\pgfqpoint{0.633833in}{3.965789in}}%
\pgfpathlineto{\pgfqpoint{0.634712in}{3.968793in}}%
\pgfpathlineto{\pgfqpoint{0.643823in}{3.975439in}}%
\pgfpathlineto{\pgfqpoint{0.641934in}{3.985088in}}%
\pgfpathlineto{\pgfqpoint{0.636622in}{3.994737in}}%
\pgfpathlineto{\pgfqpoint{0.634712in}{3.996417in}}%
\pgfpathlineto{\pgfqpoint{0.632179in}{4.004386in}}%
\pgfpathlineto{\pgfqpoint{0.629078in}{4.014035in}}%
\pgfpathlineto{\pgfqpoint{0.625000in}{4.014526in}}%
\pgfusepath{stroke}%
\end{pgfscope}%
\begin{pgfscope}%
\pgfpathrectangle{\pgfqpoint{0.625000in}{0.550000in}}{\pgfqpoint{3.875000in}{3.850000in}} %
\pgfusepath{clip}%
\pgfsetbuttcap%
\pgfsetroundjoin%
\pgfsetlinewidth{0.250937pt}%
\definecolor{currentstroke}{rgb}{0.000000,0.000000,0.000000}%
\pgfsetstrokecolor{currentstroke}%
\pgfsetdash{}{0pt}%
\pgfpathmoveto{\pgfqpoint{0.625000in}{4.032933in}}%
\pgfpathlineto{\pgfqpoint{0.628584in}{4.033333in}}%
\pgfpathlineto{\pgfqpoint{0.632179in}{4.042982in}}%
\pgfpathlineto{\pgfqpoint{0.634712in}{4.050823in}}%
\pgfpathlineto{\pgfqpoint{0.636760in}{4.052632in}}%
\pgfpathlineto{\pgfqpoint{0.641179in}{4.062281in}}%
\pgfpathlineto{\pgfqpoint{0.644424in}{4.069848in}}%
\pgfpathlineto{\pgfqpoint{0.645981in}{4.071930in}}%
\pgfpathlineto{\pgfqpoint{0.651080in}{4.081579in}}%
\pgfpathlineto{\pgfqpoint{0.651982in}{4.091228in}}%
\pgfpathlineto{\pgfqpoint{0.648270in}{4.100877in}}%
\pgfpathlineto{\pgfqpoint{0.644424in}{4.104553in}}%
\pgfpathlineto{\pgfqpoint{0.640591in}{4.110526in}}%
\pgfpathlineto{\pgfqpoint{0.634712in}{4.114522in}}%
\pgfpathlineto{\pgfqpoint{0.633714in}{4.120175in}}%
\pgfpathlineto{\pgfqpoint{0.630040in}{4.129825in}}%
\pgfpathlineto{\pgfqpoint{0.629911in}{4.139474in}}%
\pgfpathlineto{\pgfqpoint{0.633327in}{4.149123in}}%
\pgfpathlineto{\pgfqpoint{0.634712in}{4.152725in}}%
\pgfpathlineto{\pgfqpoint{0.641832in}{4.158772in}}%
\pgfpathlineto{\pgfqpoint{0.644424in}{4.165798in}}%
\pgfpathlineto{\pgfqpoint{0.646155in}{4.168421in}}%
\pgfpathlineto{\pgfqpoint{0.644424in}{4.172140in}}%
\pgfpathlineto{\pgfqpoint{0.642609in}{4.178070in}}%
\pgfpathlineto{\pgfqpoint{0.634712in}{4.181497in}}%
\pgfpathlineto{\pgfqpoint{0.634644in}{4.178070in}}%
\pgfpathlineto{\pgfqpoint{0.625000in}{4.169221in}}%
\pgfusepath{stroke}%
\end{pgfscope}%
\begin{pgfscope}%
\pgfpathrectangle{\pgfqpoint{0.625000in}{0.550000in}}{\pgfqpoint{3.875000in}{3.850000in}} %
\pgfusepath{clip}%
\pgfsetbuttcap%
\pgfsetroundjoin%
\pgfsetlinewidth{0.250937pt}%
\definecolor{currentstroke}{rgb}{0.000000,0.000000,0.000000}%
\pgfsetstrokecolor{currentstroke}%
\pgfsetdash{}{0pt}%
\pgfpathmoveto{\pgfqpoint{0.625000in}{4.055080in}}%
\pgfpathlineto{\pgfqpoint{0.633230in}{4.052632in}}%
\pgfpathlineto{\pgfqpoint{0.625000in}{4.049628in}}%
\pgfusepath{stroke}%
\end{pgfscope}%
\begin{pgfscope}%
\pgfpathrectangle{\pgfqpoint{0.625000in}{0.550000in}}{\pgfqpoint{3.875000in}{3.850000in}} %
\pgfusepath{clip}%
\pgfsetbuttcap%
\pgfsetroundjoin%
\pgfsetlinewidth{0.250937pt}%
\definecolor{currentstroke}{rgb}{0.000000,0.000000,0.000000}%
\pgfsetstrokecolor{currentstroke}%
\pgfsetdash{}{0pt}%
\pgfpathmoveto{\pgfqpoint{0.625000in}{4.187354in}}%
\pgfpathlineto{\pgfqpoint{0.632340in}{4.187719in}}%
\pgfpathlineto{\pgfqpoint{0.630327in}{4.197368in}}%
\pgfpathlineto{\pgfqpoint{0.634655in}{4.207018in}}%
\pgfpathlineto{\pgfqpoint{0.634712in}{4.207414in}}%
\pgfpathlineto{\pgfqpoint{0.643728in}{4.216667in}}%
\pgfpathlineto{\pgfqpoint{0.634712in}{4.225919in}}%
\pgfpathlineto{\pgfqpoint{0.634654in}{4.226316in}}%
\pgfpathlineto{\pgfqpoint{0.632179in}{4.235965in}}%
\pgfpathlineto{\pgfqpoint{0.634712in}{4.243096in}}%
\pgfpathlineto{\pgfqpoint{0.643623in}{4.245614in}}%
\pgfpathlineto{\pgfqpoint{0.644424in}{4.249420in}}%
\pgfpathlineto{\pgfqpoint{0.654135in}{4.253508in}}%
\pgfpathlineto{\pgfqpoint{0.661225in}{4.255263in}}%
\pgfpathlineto{\pgfqpoint{0.663847in}{4.255951in}}%
\pgfpathlineto{\pgfqpoint{0.673559in}{4.258708in}}%
\pgfpathlineto{\pgfqpoint{0.683271in}{4.262152in}}%
\pgfpathlineto{\pgfqpoint{0.689687in}{4.264912in}}%
\pgfpathlineto{\pgfqpoint{0.692982in}{4.266335in}}%
\pgfpathlineto{\pgfqpoint{0.702694in}{4.271281in}}%
\pgfpathlineto{\pgfqpoint{0.708207in}{4.274561in}}%
\pgfpathlineto{\pgfqpoint{0.712406in}{4.277138in}}%
\pgfpathlineto{\pgfqpoint{0.722118in}{4.284016in}}%
\pgfpathlineto{\pgfqpoint{0.722366in}{4.284211in}}%
\pgfpathlineto{\pgfqpoint{0.731830in}{4.292050in}}%
\pgfpathlineto{\pgfqpoint{0.733802in}{4.293860in}}%
\pgfpathlineto{\pgfqpoint{0.741541in}{4.301549in}}%
\pgfpathlineto{\pgfqpoint{0.743362in}{4.303509in}}%
\pgfpathlineto{\pgfqpoint{0.751253in}{4.312911in}}%
\pgfpathlineto{\pgfqpoint{0.751448in}{4.313158in}}%
\pgfpathlineto{\pgfqpoint{0.758372in}{4.322807in}}%
\pgfpathlineto{\pgfqpoint{0.760965in}{4.326979in}}%
\pgfpathlineto{\pgfqpoint{0.764267in}{4.332456in}}%
\pgfpathlineto{\pgfqpoint{0.769245in}{4.342105in}}%
\pgfpathlineto{\pgfqpoint{0.770677in}{4.345379in}}%
\pgfpathlineto{\pgfqpoint{0.773456in}{4.351754in}}%
\pgfpathlineto{\pgfqpoint{0.776922in}{4.361404in}}%
\pgfpathlineto{\pgfqpoint{0.779653in}{4.371053in}}%
\pgfpathlineto{\pgfqpoint{0.780388in}{4.374445in}}%
\pgfpathlineto{\pgfqpoint{0.781787in}{4.380702in}}%
\pgfpathlineto{\pgfqpoint{0.783311in}{4.390351in}}%
\pgfpathlineto{\pgfqpoint{0.784212in}{4.400000in}}%
\pgfusepath{stroke}%
\end{pgfscope}%
\begin{pgfscope}%
\pgfpathrectangle{\pgfqpoint{0.625000in}{0.550000in}}{\pgfqpoint{3.875000in}{3.850000in}} %
\pgfusepath{clip}%
\pgfsetbuttcap%
\pgfsetroundjoin%
\pgfsetlinewidth{0.250937pt}%
\definecolor{currentstroke}{rgb}{0.000000,0.000000,0.000000}%
\pgfsetstrokecolor{currentstroke}%
\pgfsetdash{}{0pt}%
\pgfpathmoveto{\pgfqpoint{0.625000in}{4.341681in}}%
\pgfpathlineto{\pgfqpoint{0.634583in}{4.332456in}}%
\pgfpathlineto{\pgfqpoint{0.625000in}{4.323200in}}%
\pgfusepath{stroke}%
\end{pgfscope}%
\begin{pgfscope}%
\pgfpathrectangle{\pgfqpoint{0.625000in}{0.550000in}}{\pgfqpoint{3.875000in}{3.850000in}} %
\pgfusepath{clip}%
\pgfsetbuttcap%
\pgfsetroundjoin%
\pgfsetlinewidth{0.250937pt}%
\definecolor{currentstroke}{rgb}{0.000000,0.000000,0.000000}%
\pgfsetstrokecolor{currentstroke}%
\pgfsetdash{}{0pt}%
\pgfpathmoveto{\pgfqpoint{0.634712in}{0.692830in}}%
\pgfpathlineto{\pgfqpoint{0.634454in}{0.694737in}}%
\pgfpathlineto{\pgfqpoint{0.634712in}{0.698628in}}%
\pgfpathlineto{\pgfqpoint{0.639723in}{0.694737in}}%
\pgfpathlineto{\pgfqpoint{0.634712in}{0.692830in}}%
\pgfusepath{stroke}%
\end{pgfscope}%
\begin{pgfscope}%
\pgfpathrectangle{\pgfqpoint{0.625000in}{0.550000in}}{\pgfqpoint{3.875000in}{3.850000in}} %
\pgfusepath{clip}%
\pgfsetbuttcap%
\pgfsetroundjoin%
\pgfsetlinewidth{0.250937pt}%
\definecolor{currentstroke}{rgb}{0.000000,0.000000,0.000000}%
\pgfsetstrokecolor{currentstroke}%
\pgfsetdash{}{0pt}%
\pgfpathmoveto{\pgfqpoint{0.634712in}{1.587804in}}%
\pgfpathlineto{\pgfqpoint{0.634293in}{1.592105in}}%
\pgfpathlineto{\pgfqpoint{0.634712in}{1.594335in}}%
\pgfpathlineto{\pgfqpoint{0.643714in}{1.592105in}}%
\pgfpathlineto{\pgfqpoint{0.634712in}{1.587804in}}%
\pgfusepath{stroke}%
\end{pgfscope}%
\begin{pgfscope}%
\pgfpathrectangle{\pgfqpoint{0.625000in}{0.550000in}}{\pgfqpoint{3.875000in}{3.850000in}} %
\pgfusepath{clip}%
\pgfsetbuttcap%
\pgfsetroundjoin%
\pgfsetlinewidth{0.250937pt}%
\definecolor{currentstroke}{rgb}{0.000000,0.000000,0.000000}%
\pgfsetstrokecolor{currentstroke}%
\pgfsetdash{}{0pt}%
\pgfpathmoveto{\pgfqpoint{0.634712in}{1.881368in}}%
\pgfpathlineto{\pgfqpoint{0.634673in}{1.881579in}}%
\pgfpathlineto{\pgfqpoint{0.634712in}{1.882980in}}%
\pgfpathlineto{\pgfqpoint{0.635840in}{1.881579in}}%
\pgfpathlineto{\pgfqpoint{0.634712in}{1.881368in}}%
\pgfusepath{stroke}%
\end{pgfscope}%
\begin{pgfscope}%
\pgfpathrectangle{\pgfqpoint{0.625000in}{0.550000in}}{\pgfqpoint{3.875000in}{3.850000in}} %
\pgfusepath{clip}%
\pgfsetbuttcap%
\pgfsetroundjoin%
\pgfsetlinewidth{0.250937pt}%
\definecolor{currentstroke}{rgb}{0.000000,0.000000,0.000000}%
\pgfsetstrokecolor{currentstroke}%
\pgfsetdash{}{0pt}%
\pgfpathmoveto{\pgfqpoint{0.634712in}{3.076669in}}%
\pgfpathlineto{\pgfqpoint{0.634673in}{3.078070in}}%
\pgfpathlineto{\pgfqpoint{0.634712in}{3.078281in}}%
\pgfpathlineto{\pgfqpoint{0.635840in}{3.078070in}}%
\pgfpathlineto{\pgfqpoint{0.634712in}{3.076669in}}%
\pgfusepath{stroke}%
\end{pgfscope}%
\begin{pgfscope}%
\pgfpathrectangle{\pgfqpoint{0.625000in}{0.550000in}}{\pgfqpoint{3.875000in}{3.850000in}} %
\pgfusepath{clip}%
\pgfsetbuttcap%
\pgfsetroundjoin%
\pgfsetlinewidth{0.250937pt}%
\definecolor{currentstroke}{rgb}{0.000000,0.000000,0.000000}%
\pgfsetstrokecolor{currentstroke}%
\pgfsetdash{}{0pt}%
\pgfpathmoveto{\pgfqpoint{0.634712in}{3.365315in}}%
\pgfpathlineto{\pgfqpoint{0.634293in}{3.367544in}}%
\pgfpathlineto{\pgfqpoint{0.634712in}{3.371845in}}%
\pgfpathlineto{\pgfqpoint{0.643714in}{3.367544in}}%
\pgfpathlineto{\pgfqpoint{0.634712in}{3.365315in}}%
\pgfusepath{stroke}%
\end{pgfscope}%
\begin{pgfscope}%
\pgfpathrectangle{\pgfqpoint{0.625000in}{0.550000in}}{\pgfqpoint{3.875000in}{3.850000in}} %
\pgfusepath{clip}%
\pgfsetbuttcap%
\pgfsetroundjoin%
\pgfsetlinewidth{0.250937pt}%
\definecolor{currentstroke}{rgb}{0.000000,0.000000,0.000000}%
\pgfsetstrokecolor{currentstroke}%
\pgfsetdash{}{0pt}%
\pgfpathmoveto{\pgfqpoint{0.634712in}{4.261021in}}%
\pgfpathlineto{\pgfqpoint{0.634454in}{4.264912in}}%
\pgfpathlineto{\pgfqpoint{0.634712in}{4.266819in}}%
\pgfpathlineto{\pgfqpoint{0.639723in}{4.264912in}}%
\pgfpathlineto{\pgfqpoint{0.634712in}{4.261021in}}%
\pgfusepath{stroke}%
\end{pgfscope}%
\begin{pgfscope}%
\pgfpathrectangle{\pgfqpoint{0.625000in}{0.550000in}}{\pgfqpoint{3.875000in}{3.850000in}} %
\pgfusepath{clip}%
\pgfsetbuttcap%
\pgfsetroundjoin%
\pgfsetlinewidth{0.250937pt}%
\definecolor{currentstroke}{rgb}{0.000000,0.000000,0.000000}%
\pgfsetstrokecolor{currentstroke}%
\pgfsetdash{}{0pt}%
\pgfpathmoveto{\pgfqpoint{0.721724in}{0.550000in}}%
\pgfpathlineto{\pgfqpoint{0.721256in}{0.559649in}}%
\pgfpathlineto{\pgfqpoint{0.719819in}{0.569298in}}%
\pgfpathlineto{\pgfqpoint{0.717315in}{0.578947in}}%
\pgfpathlineto{\pgfqpoint{0.713596in}{0.588596in}}%
\pgfpathlineto{\pgfqpoint{0.712406in}{0.590981in}}%
\pgfpathlineto{\pgfqpoint{0.708694in}{0.598246in}}%
\pgfpathlineto{\pgfqpoint{0.702694in}{0.607188in}}%
\pgfpathlineto{\pgfqpoint{0.702184in}{0.607895in}}%
\pgfpathlineto{\pgfqpoint{0.693778in}{0.617544in}}%
\pgfpathlineto{\pgfqpoint{0.692982in}{0.618334in}}%
\pgfpathlineto{\pgfqpoint{0.683271in}{0.626686in}}%
\pgfpathlineto{\pgfqpoint{0.682559in}{0.627193in}}%
\pgfpathlineto{\pgfqpoint{0.673559in}{0.633154in}}%
\pgfpathlineto{\pgfqpoint{0.666247in}{0.636842in}}%
\pgfpathlineto{\pgfqpoint{0.663847in}{0.638025in}}%
\pgfpathlineto{\pgfqpoint{0.654135in}{0.641720in}}%
\pgfpathlineto{\pgfqpoint{0.644424in}{0.644206in}}%
\pgfpathlineto{\pgfqpoint{0.634712in}{0.646221in}}%
\pgfpathlineto{\pgfqpoint{0.634703in}{0.646491in}}%
\pgfpathlineto{\pgfqpoint{0.634368in}{0.656140in}}%
\pgfpathlineto{\pgfqpoint{0.633942in}{0.665789in}}%
\pgfpathlineto{\pgfqpoint{0.632349in}{0.675439in}}%
\pgfpathlineto{\pgfqpoint{0.633561in}{0.685088in}}%
\pgfpathlineto{\pgfqpoint{0.632611in}{0.694737in}}%
\pgfpathlineto{\pgfqpoint{0.633575in}{0.704386in}}%
\pgfpathlineto{\pgfqpoint{0.633796in}{0.714035in}}%
\pgfpathlineto{\pgfqpoint{0.630474in}{0.723684in}}%
\pgfpathlineto{\pgfqpoint{0.633325in}{0.733333in}}%
\pgfpathlineto{\pgfqpoint{0.634712in}{0.742930in}}%
\pgfpathlineto{\pgfqpoint{0.634762in}{0.742982in}}%
\pgfpathlineto{\pgfqpoint{0.634712in}{0.743034in}}%
\pgfpathlineto{\pgfqpoint{0.633342in}{0.752632in}}%
\pgfpathlineto{\pgfqpoint{0.629062in}{0.762281in}}%
\pgfpathlineto{\pgfqpoint{0.630559in}{0.771930in}}%
\pgfpathlineto{\pgfqpoint{0.625000in}{0.772181in}}%
\pgfusepath{stroke}%
\end{pgfscope}%
\begin{pgfscope}%
\pgfpathrectangle{\pgfqpoint{0.625000in}{0.550000in}}{\pgfqpoint{3.875000in}{3.850000in}} %
\pgfusepath{clip}%
\pgfsetbuttcap%
\pgfsetroundjoin%
\pgfsetlinewidth{0.250937pt}%
\definecolor{currentstroke}{rgb}{0.000000,0.000000,0.000000}%
\pgfsetstrokecolor{currentstroke}%
\pgfsetdash{}{0pt}%
\pgfpathmoveto{\pgfqpoint{0.625000in}{0.636577in}}%
\pgfpathlineto{\pgfqpoint{0.634679in}{0.627193in}}%
\pgfpathlineto{\pgfqpoint{0.625000in}{0.617836in}}%
\pgfusepath{stroke}%
\end{pgfscope}%
\begin{pgfscope}%
\pgfpathrectangle{\pgfqpoint{0.625000in}{0.550000in}}{\pgfqpoint{3.875000in}{3.850000in}} %
\pgfusepath{clip}%
\pgfsetbuttcap%
\pgfsetroundjoin%
\pgfsetlinewidth{0.250937pt}%
\definecolor{currentstroke}{rgb}{0.000000,0.000000,0.000000}%
\pgfsetstrokecolor{currentstroke}%
\pgfsetdash{}{0pt}%
\pgfpathmoveto{\pgfqpoint{0.625000in}{0.790577in}}%
\pgfpathlineto{\pgfqpoint{0.634712in}{0.785228in}}%
\pgfpathlineto{\pgfqpoint{0.637905in}{0.791228in}}%
\pgfpathlineto{\pgfqpoint{0.634712in}{0.797628in}}%
\pgfpathlineto{\pgfqpoint{0.634083in}{0.800877in}}%
\pgfpathlineto{\pgfqpoint{0.630338in}{0.810526in}}%
\pgfpathlineto{\pgfqpoint{0.629033in}{0.820175in}}%
\pgfpathlineto{\pgfqpoint{0.629324in}{0.829825in}}%
\pgfpathlineto{\pgfqpoint{0.632541in}{0.839474in}}%
\pgfpathlineto{\pgfqpoint{0.633941in}{0.849123in}}%
\pgfpathlineto{\pgfqpoint{0.634698in}{0.858772in}}%
\pgfpathlineto{\pgfqpoint{0.634712in}{0.858862in}}%
\pgfpathlineto{\pgfqpoint{0.641455in}{0.868421in}}%
\pgfpathlineto{\pgfqpoint{0.640273in}{0.878070in}}%
\pgfpathlineto{\pgfqpoint{0.634712in}{0.884365in}}%
\pgfpathlineto{\pgfqpoint{0.634115in}{0.887719in}}%
\pgfpathlineto{\pgfqpoint{0.634225in}{0.897368in}}%
\pgfpathlineto{\pgfqpoint{0.625000in}{0.903283in}}%
\pgfusepath{stroke}%
\end{pgfscope}%
\begin{pgfscope}%
\pgfpathrectangle{\pgfqpoint{0.625000in}{0.550000in}}{\pgfqpoint{3.875000in}{3.850000in}} %
\pgfusepath{clip}%
\pgfsetbuttcap%
\pgfsetroundjoin%
\pgfsetlinewidth{0.250937pt}%
\definecolor{currentstroke}{rgb}{0.000000,0.000000,0.000000}%
\pgfsetstrokecolor{currentstroke}%
\pgfsetdash{}{0pt}%
\pgfpathmoveto{\pgfqpoint{0.625000in}{0.911599in}}%
\pgfpathlineto{\pgfqpoint{0.630474in}{0.916667in}}%
\pgfpathlineto{\pgfqpoint{0.627829in}{0.926316in}}%
\pgfpathlineto{\pgfqpoint{0.625000in}{0.926605in}}%
\pgfusepath{stroke}%
\end{pgfscope}%
\begin{pgfscope}%
\pgfpathrectangle{\pgfqpoint{0.625000in}{0.550000in}}{\pgfqpoint{3.875000in}{3.850000in}} %
\pgfusepath{clip}%
\pgfsetbuttcap%
\pgfsetroundjoin%
\pgfsetlinewidth{0.250937pt}%
\definecolor{currentstroke}{rgb}{0.000000,0.000000,0.000000}%
\pgfsetstrokecolor{currentstroke}%
\pgfsetdash{}{0pt}%
\pgfpathmoveto{\pgfqpoint{0.625000in}{0.945242in}}%
\pgfpathlineto{\pgfqpoint{0.628383in}{0.945614in}}%
\pgfpathlineto{\pgfqpoint{0.630474in}{0.955263in}}%
\pgfpathlineto{\pgfqpoint{0.632696in}{0.964912in}}%
\pgfpathlineto{\pgfqpoint{0.634712in}{0.974332in}}%
\pgfpathlineto{\pgfqpoint{0.634831in}{0.974561in}}%
\pgfpathlineto{\pgfqpoint{0.634712in}{0.976349in}}%
\pgfpathlineto{\pgfqpoint{0.634590in}{0.984211in}}%
\pgfpathlineto{\pgfqpoint{0.631735in}{0.993860in}}%
\pgfpathlineto{\pgfqpoint{0.633944in}{1.003509in}}%
\pgfpathlineto{\pgfqpoint{0.634712in}{1.010178in}}%
\pgfpathlineto{\pgfqpoint{0.639118in}{1.013158in}}%
\pgfpathlineto{\pgfqpoint{0.644424in}{1.018405in}}%
\pgfpathlineto{\pgfqpoint{0.647430in}{1.022807in}}%
\pgfpathlineto{\pgfqpoint{0.649666in}{1.032456in}}%
\pgfpathlineto{\pgfqpoint{0.647430in}{1.042105in}}%
\pgfpathlineto{\pgfqpoint{0.644424in}{1.046510in}}%
\pgfpathlineto{\pgfqpoint{0.639120in}{1.051754in}}%
\pgfpathlineto{\pgfqpoint{0.634712in}{1.055576in}}%
\pgfpathlineto{\pgfqpoint{0.634063in}{1.061404in}}%
\pgfpathlineto{\pgfqpoint{0.631363in}{1.071053in}}%
\pgfpathlineto{\pgfqpoint{0.631839in}{1.080702in}}%
\pgfpathlineto{\pgfqpoint{0.625000in}{1.081061in}}%
\pgfusepath{stroke}%
\end{pgfscope}%
\begin{pgfscope}%
\pgfpathrectangle{\pgfqpoint{0.625000in}{0.550000in}}{\pgfqpoint{3.875000in}{3.850000in}} %
\pgfusepath{clip}%
\pgfsetbuttcap%
\pgfsetroundjoin%
\pgfsetlinewidth{0.250937pt}%
\definecolor{currentstroke}{rgb}{0.000000,0.000000,0.000000}%
\pgfsetstrokecolor{currentstroke}%
\pgfsetdash{}{0pt}%
\pgfpathmoveto{\pgfqpoint{0.625000in}{1.099675in}}%
\pgfpathlineto{\pgfqpoint{0.634712in}{1.091943in}}%
\pgfpathlineto{\pgfqpoint{0.639572in}{1.100000in}}%
\pgfpathlineto{\pgfqpoint{0.636161in}{1.109649in}}%
\pgfpathlineto{\pgfqpoint{0.634712in}{1.111924in}}%
\pgfpathlineto{\pgfqpoint{0.633895in}{1.119298in}}%
\pgfpathlineto{\pgfqpoint{0.634690in}{1.128947in}}%
\pgfpathlineto{\pgfqpoint{0.633811in}{1.138596in}}%
\pgfpathlineto{\pgfqpoint{0.632375in}{1.148246in}}%
\pgfpathlineto{\pgfqpoint{0.630142in}{1.157895in}}%
\pgfpathlineto{\pgfqpoint{0.630048in}{1.167544in}}%
\pgfpathlineto{\pgfqpoint{0.626760in}{1.177193in}}%
\pgfpathlineto{\pgfqpoint{0.625667in}{1.186842in}}%
\pgfpathlineto{\pgfqpoint{0.625000in}{1.192631in}}%
\pgfusepath{stroke}%
\end{pgfscope}%
\begin{pgfscope}%
\pgfpathrectangle{\pgfqpoint{0.625000in}{0.550000in}}{\pgfqpoint{3.875000in}{3.850000in}} %
\pgfusepath{clip}%
\pgfsetbuttcap%
\pgfsetroundjoin%
\pgfsetlinewidth{0.250937pt}%
\definecolor{currentstroke}{rgb}{0.000000,0.000000,0.000000}%
\pgfsetstrokecolor{currentstroke}%
\pgfsetdash{}{0pt}%
\pgfpathmoveto{\pgfqpoint{0.625000in}{1.199557in}}%
\pgfpathlineto{\pgfqpoint{0.626745in}{1.206140in}}%
\pgfpathlineto{\pgfqpoint{0.629427in}{1.215789in}}%
\pgfpathlineto{\pgfqpoint{0.631562in}{1.225439in}}%
\pgfpathlineto{\pgfqpoint{0.632008in}{1.235088in}}%
\pgfpathlineto{\pgfqpoint{0.625000in}{1.235368in}}%
\pgfusepath{stroke}%
\end{pgfscope}%
\begin{pgfscope}%
\pgfpathrectangle{\pgfqpoint{0.625000in}{0.550000in}}{\pgfqpoint{3.875000in}{3.850000in}} %
\pgfusepath{clip}%
\pgfsetbuttcap%
\pgfsetroundjoin%
\pgfsetlinewidth{0.250937pt}%
\definecolor{currentstroke}{rgb}{0.000000,0.000000,0.000000}%
\pgfsetstrokecolor{currentstroke}%
\pgfsetdash{}{0pt}%
\pgfpathmoveto{\pgfqpoint{0.625000in}{1.254372in}}%
\pgfpathlineto{\pgfqpoint{0.629099in}{1.254386in}}%
\pgfpathlineto{\pgfqpoint{0.634125in}{1.264035in}}%
\pgfpathlineto{\pgfqpoint{0.634712in}{1.267128in}}%
\pgfpathlineto{\pgfqpoint{0.637729in}{1.273684in}}%
\pgfpathlineto{\pgfqpoint{0.634712in}{1.279304in}}%
\pgfpathlineto{\pgfqpoint{0.634091in}{1.283333in}}%
\pgfpathlineto{\pgfqpoint{0.625000in}{1.288438in}}%
\pgfusepath{stroke}%
\end{pgfscope}%
\begin{pgfscope}%
\pgfpathrectangle{\pgfqpoint{0.625000in}{0.550000in}}{\pgfqpoint{3.875000in}{3.850000in}} %
\pgfusepath{clip}%
\pgfsetbuttcap%
\pgfsetroundjoin%
\pgfsetlinewidth{0.250937pt}%
\definecolor{currentstroke}{rgb}{0.000000,0.000000,0.000000}%
\pgfsetstrokecolor{currentstroke}%
\pgfsetdash{}{0pt}%
\pgfpathmoveto{\pgfqpoint{0.625000in}{1.298390in}}%
\pgfpathlineto{\pgfqpoint{0.634102in}{1.302632in}}%
\pgfpathlineto{\pgfqpoint{0.634712in}{1.304939in}}%
\pgfpathlineto{\pgfqpoint{0.641971in}{1.312281in}}%
\pgfpathlineto{\pgfqpoint{0.644364in}{1.321930in}}%
\pgfpathlineto{\pgfqpoint{0.641966in}{1.331579in}}%
\pgfpathlineto{\pgfqpoint{0.634712in}{1.338468in}}%
\pgfpathlineto{\pgfqpoint{0.633947in}{1.341228in}}%
\pgfpathlineto{\pgfqpoint{0.633976in}{1.350877in}}%
\pgfpathlineto{\pgfqpoint{0.632777in}{1.360526in}}%
\pgfpathlineto{\pgfqpoint{0.629141in}{1.370175in}}%
\pgfpathlineto{\pgfqpoint{0.627672in}{1.379825in}}%
\pgfpathlineto{\pgfqpoint{0.631434in}{1.389474in}}%
\pgfpathlineto{\pgfqpoint{0.625000in}{1.389847in}}%
\pgfusepath{stroke}%
\end{pgfscope}%
\begin{pgfscope}%
\pgfpathrectangle{\pgfqpoint{0.625000in}{0.550000in}}{\pgfqpoint{3.875000in}{3.850000in}} %
\pgfusepath{clip}%
\pgfsetbuttcap%
\pgfsetroundjoin%
\pgfsetlinewidth{0.250937pt}%
\definecolor{currentstroke}{rgb}{0.000000,0.000000,0.000000}%
\pgfsetstrokecolor{currentstroke}%
\pgfsetdash{}{0pt}%
\pgfpathmoveto{\pgfqpoint{0.625000in}{1.408394in}}%
\pgfpathlineto{\pgfqpoint{0.634712in}{1.405602in}}%
\pgfpathlineto{\pgfqpoint{0.636235in}{1.408772in}}%
\pgfpathlineto{\pgfqpoint{0.634712in}{1.410548in}}%
\pgfpathlineto{\pgfqpoint{0.633568in}{1.418421in}}%
\pgfpathlineto{\pgfqpoint{0.625134in}{1.428070in}}%
\pgfpathlineto{\pgfqpoint{0.633585in}{1.437719in}}%
\pgfpathlineto{\pgfqpoint{0.633206in}{1.447368in}}%
\pgfpathlineto{\pgfqpoint{0.633730in}{1.457018in}}%
\pgfpathlineto{\pgfqpoint{0.634585in}{1.466667in}}%
\pgfpathlineto{\pgfqpoint{0.634712in}{1.468098in}}%
\pgfpathlineto{\pgfqpoint{0.644424in}{1.470740in}}%
\pgfpathlineto{\pgfqpoint{0.653764in}{1.476316in}}%
\pgfpathlineto{\pgfqpoint{0.654135in}{1.476567in}}%
\pgfpathlineto{\pgfqpoint{0.663595in}{1.485965in}}%
\pgfpathlineto{\pgfqpoint{0.663847in}{1.486334in}}%
\pgfpathlineto{\pgfqpoint{0.669456in}{1.495614in}}%
\pgfpathlineto{\pgfqpoint{0.672467in}{1.505263in}}%
\pgfpathlineto{\pgfqpoint{0.673373in}{1.514912in}}%
\pgfpathlineto{\pgfqpoint{0.672467in}{1.524561in}}%
\pgfpathlineto{\pgfqpoint{0.669456in}{1.534211in}}%
\pgfpathlineto{\pgfqpoint{0.663847in}{1.543491in}}%
\pgfpathlineto{\pgfqpoint{0.663595in}{1.543860in}}%
\pgfpathlineto{\pgfqpoint{0.654135in}{1.553258in}}%
\pgfpathlineto{\pgfqpoint{0.653764in}{1.553509in}}%
\pgfpathlineto{\pgfqpoint{0.644424in}{1.559080in}}%
\pgfpathlineto{\pgfqpoint{0.634712in}{1.562448in}}%
\pgfpathlineto{\pgfqpoint{0.634680in}{1.553509in}}%
\pgfpathlineto{\pgfqpoint{0.625000in}{1.543986in}}%
\pgfusepath{stroke}%
\end{pgfscope}%
\begin{pgfscope}%
\pgfpathrectangle{\pgfqpoint{0.625000in}{0.550000in}}{\pgfqpoint{3.875000in}{3.850000in}} %
\pgfusepath{clip}%
\pgfsetbuttcap%
\pgfsetroundjoin%
\pgfsetlinewidth{0.250937pt}%
\definecolor{currentstroke}{rgb}{0.000000,0.000000,0.000000}%
\pgfsetstrokecolor{currentstroke}%
\pgfsetdash{}{0pt}%
\pgfpathmoveto{\pgfqpoint{0.625000in}{1.562732in}}%
\pgfpathlineto{\pgfqpoint{0.634657in}{1.563158in}}%
\pgfpathlineto{\pgfqpoint{0.633547in}{1.572807in}}%
\pgfpathlineto{\pgfqpoint{0.630032in}{1.582456in}}%
\pgfpathlineto{\pgfqpoint{0.632633in}{1.592105in}}%
\pgfpathlineto{\pgfqpoint{0.634432in}{1.601754in}}%
\pgfpathlineto{\pgfqpoint{0.633082in}{1.611404in}}%
\pgfpathlineto{\pgfqpoint{0.625537in}{1.621053in}}%
\pgfpathlineto{\pgfqpoint{0.632332in}{1.630702in}}%
\pgfpathlineto{\pgfqpoint{0.625000in}{1.639023in}}%
\pgfusepath{stroke}%
\end{pgfscope}%
\begin{pgfscope}%
\pgfpathrectangle{\pgfqpoint{0.625000in}{0.550000in}}{\pgfqpoint{3.875000in}{3.850000in}} %
\pgfusepath{clip}%
\pgfsetbuttcap%
\pgfsetroundjoin%
\pgfsetlinewidth{0.250937pt}%
\definecolor{currentstroke}{rgb}{0.000000,0.000000,0.000000}%
\pgfsetstrokecolor{currentstroke}%
\pgfsetdash{}{0pt}%
\pgfpathmoveto{\pgfqpoint{0.625000in}{1.642562in}}%
\pgfpathlineto{\pgfqpoint{0.634712in}{1.642980in}}%
\pgfpathlineto{\pgfqpoint{0.639344in}{1.650000in}}%
\pgfpathlineto{\pgfqpoint{0.637255in}{1.659649in}}%
\pgfpathlineto{\pgfqpoint{0.634712in}{1.663895in}}%
\pgfpathlineto{\pgfqpoint{0.633981in}{1.669298in}}%
\pgfpathlineto{\pgfqpoint{0.632546in}{1.678947in}}%
\pgfpathlineto{\pgfqpoint{0.630474in}{1.688596in}}%
\pgfpathlineto{\pgfqpoint{0.628147in}{1.698246in}}%
\pgfpathlineto{\pgfqpoint{0.625000in}{1.698601in}}%
\pgfusepath{stroke}%
\end{pgfscope}%
\begin{pgfscope}%
\pgfpathrectangle{\pgfqpoint{0.625000in}{0.550000in}}{\pgfqpoint{3.875000in}{3.850000in}} %
\pgfusepath{clip}%
\pgfsetbuttcap%
\pgfsetroundjoin%
\pgfsetlinewidth{0.250937pt}%
\definecolor{currentstroke}{rgb}{0.000000,0.000000,0.000000}%
\pgfsetstrokecolor{currentstroke}%
\pgfsetdash{}{0pt}%
\pgfpathmoveto{\pgfqpoint{0.625000in}{1.717130in}}%
\pgfpathlineto{\pgfqpoint{0.628493in}{1.717544in}}%
\pgfpathlineto{\pgfqpoint{0.630474in}{1.727193in}}%
\pgfpathlineto{\pgfqpoint{0.633553in}{1.736842in}}%
\pgfpathlineto{\pgfqpoint{0.634311in}{1.746491in}}%
\pgfpathlineto{\pgfqpoint{0.634712in}{1.749494in}}%
\pgfpathlineto{\pgfqpoint{0.637831in}{1.756140in}}%
\pgfpathlineto{\pgfqpoint{0.634712in}{1.761831in}}%
\pgfpathlineto{\pgfqpoint{0.633870in}{1.765789in}}%
\pgfpathlineto{\pgfqpoint{0.631479in}{1.775439in}}%
\pgfpathlineto{\pgfqpoint{0.633031in}{1.785088in}}%
\pgfpathlineto{\pgfqpoint{0.634118in}{1.794737in}}%
\pgfpathlineto{\pgfqpoint{0.634565in}{1.804386in}}%
\pgfpathlineto{\pgfqpoint{0.634712in}{1.805885in}}%
\pgfpathlineto{\pgfqpoint{0.644424in}{1.810870in}}%
\pgfpathlineto{\pgfqpoint{0.647943in}{1.814035in}}%
\pgfpathlineto{\pgfqpoint{0.654135in}{1.822962in}}%
\pgfpathlineto{\pgfqpoint{0.654561in}{1.823684in}}%
\pgfpathlineto{\pgfqpoint{0.657371in}{1.833333in}}%
\pgfpathlineto{\pgfqpoint{0.656832in}{1.842982in}}%
\pgfpathlineto{\pgfqpoint{0.654135in}{1.849712in}}%
\pgfpathlineto{\pgfqpoint{0.652961in}{1.852632in}}%
\pgfpathlineto{\pgfqpoint{0.644424in}{1.862115in}}%
\pgfpathlineto{\pgfqpoint{0.644162in}{1.862281in}}%
\pgfpathlineto{\pgfqpoint{0.634712in}{1.867756in}}%
\pgfpathlineto{\pgfqpoint{0.634685in}{1.862281in}}%
\pgfpathlineto{\pgfqpoint{0.625000in}{1.853142in}}%
\pgfusepath{stroke}%
\end{pgfscope}%
\begin{pgfscope}%
\pgfpathrectangle{\pgfqpoint{0.625000in}{0.550000in}}{\pgfqpoint{3.875000in}{3.850000in}} %
\pgfusepath{clip}%
\pgfsetbuttcap%
\pgfsetroundjoin%
\pgfsetlinewidth{0.250937pt}%
\definecolor{currentstroke}{rgb}{0.000000,0.000000,0.000000}%
\pgfsetstrokecolor{currentstroke}%
\pgfsetdash{}{0pt}%
\pgfpathmoveto{\pgfqpoint{0.625000in}{1.871613in}}%
\pgfpathlineto{\pgfqpoint{0.634151in}{1.871930in}}%
\pgfpathlineto{\pgfqpoint{0.632375in}{1.881579in}}%
\pgfpathlineto{\pgfqpoint{0.625000in}{1.885580in}}%
\pgfusepath{stroke}%
\end{pgfscope}%
\begin{pgfscope}%
\pgfpathrectangle{\pgfqpoint{0.625000in}{0.550000in}}{\pgfqpoint{3.875000in}{3.850000in}} %
\pgfusepath{clip}%
\pgfsetbuttcap%
\pgfsetroundjoin%
\pgfsetlinewidth{0.250937pt}%
\definecolor{currentstroke}{rgb}{0.000000,0.000000,0.000000}%
\pgfsetstrokecolor{currentstroke}%
\pgfsetdash{}{0pt}%
\pgfpathmoveto{\pgfqpoint{0.625000in}{1.894770in}}%
\pgfpathlineto{\pgfqpoint{0.634712in}{1.900874in}}%
\pgfpathlineto{\pgfqpoint{0.634714in}{1.900877in}}%
\pgfpathlineto{\pgfqpoint{0.634712in}{1.900880in}}%
\pgfpathlineto{\pgfqpoint{0.629147in}{1.910526in}}%
\pgfpathlineto{\pgfqpoint{0.629062in}{1.920175in}}%
\pgfpathlineto{\pgfqpoint{0.625000in}{1.924503in}}%
\pgfusepath{stroke}%
\end{pgfscope}%
\begin{pgfscope}%
\pgfpathrectangle{\pgfqpoint{0.625000in}{0.550000in}}{\pgfqpoint{3.875000in}{3.850000in}} %
\pgfusepath{clip}%
\pgfsetbuttcap%
\pgfsetroundjoin%
\pgfsetlinewidth{0.250937pt}%
\definecolor{currentstroke}{rgb}{0.000000,0.000000,0.000000}%
\pgfsetstrokecolor{currentstroke}%
\pgfsetdash{}{0pt}%
\pgfpathmoveto{\pgfqpoint{0.625000in}{1.933660in}}%
\pgfpathlineto{\pgfqpoint{0.631528in}{1.939474in}}%
\pgfpathlineto{\pgfqpoint{0.633923in}{1.949123in}}%
\pgfpathlineto{\pgfqpoint{0.633673in}{1.958772in}}%
\pgfpathlineto{\pgfqpoint{0.633352in}{1.968421in}}%
\pgfpathlineto{\pgfqpoint{0.634712in}{1.975627in}}%
\pgfpathlineto{\pgfqpoint{0.639172in}{1.978070in}}%
\pgfpathlineto{\pgfqpoint{0.644424in}{1.983312in}}%
\pgfpathlineto{\pgfqpoint{0.647430in}{1.987719in}}%
\pgfpathlineto{\pgfqpoint{0.649666in}{1.997368in}}%
\pgfpathlineto{\pgfqpoint{0.647431in}{2.007018in}}%
\pgfpathlineto{\pgfqpoint{0.644424in}{2.011427in}}%
\pgfpathlineto{\pgfqpoint{0.639175in}{2.016667in}}%
\pgfpathlineto{\pgfqpoint{0.634712in}{2.019689in}}%
\pgfpathlineto{\pgfqpoint{0.634693in}{2.016667in}}%
\pgfpathlineto{\pgfqpoint{0.625000in}{2.007062in}}%
\pgfusepath{stroke}%
\end{pgfscope}%
\begin{pgfscope}%
\pgfpathrectangle{\pgfqpoint{0.625000in}{0.550000in}}{\pgfqpoint{3.875000in}{3.850000in}} %
\pgfusepath{clip}%
\pgfsetbuttcap%
\pgfsetroundjoin%
\pgfsetlinewidth{0.250937pt}%
\definecolor{currentstroke}{rgb}{0.000000,0.000000,0.000000}%
\pgfsetstrokecolor{currentstroke}%
\pgfsetdash{}{0pt}%
\pgfpathmoveto{\pgfqpoint{0.625000in}{2.027370in}}%
\pgfpathlineto{\pgfqpoint{0.633310in}{2.035965in}}%
\pgfpathlineto{\pgfqpoint{0.630490in}{2.045614in}}%
\pgfpathlineto{\pgfqpoint{0.630392in}{2.055263in}}%
\pgfpathlineto{\pgfqpoint{0.632845in}{2.064912in}}%
\pgfpathlineto{\pgfqpoint{0.634102in}{2.074561in}}%
\pgfpathlineto{\pgfqpoint{0.634712in}{2.076869in}}%
\pgfpathlineto{\pgfqpoint{0.641966in}{2.084211in}}%
\pgfpathlineto{\pgfqpoint{0.644364in}{2.093860in}}%
\pgfpathlineto{\pgfqpoint{0.641969in}{2.103509in}}%
\pgfpathlineto{\pgfqpoint{0.634712in}{2.111153in}}%
\pgfpathlineto{\pgfqpoint{0.634197in}{2.113158in}}%
\pgfpathlineto{\pgfqpoint{0.632389in}{2.122807in}}%
\pgfpathlineto{\pgfqpoint{0.631630in}{2.132456in}}%
\pgfpathlineto{\pgfqpoint{0.634215in}{2.142105in}}%
\pgfpathlineto{\pgfqpoint{0.634712in}{2.145075in}}%
\pgfpathlineto{\pgfqpoint{0.640304in}{2.151754in}}%
\pgfpathlineto{\pgfqpoint{0.641445in}{2.161404in}}%
\pgfpathlineto{\pgfqpoint{0.634712in}{2.170963in}}%
\pgfpathlineto{\pgfqpoint{0.625000in}{2.161605in}}%
\pgfusepath{stroke}%
\end{pgfscope}%
\begin{pgfscope}%
\pgfpathrectangle{\pgfqpoint{0.625000in}{0.550000in}}{\pgfqpoint{3.875000in}{3.850000in}} %
\pgfusepath{clip}%
\pgfsetbuttcap%
\pgfsetroundjoin%
\pgfsetlinewidth{0.250937pt}%
\definecolor{currentstroke}{rgb}{0.000000,0.000000,0.000000}%
\pgfsetstrokecolor{currentstroke}%
\pgfsetdash{}{0pt}%
\pgfpathmoveto{\pgfqpoint{0.625000in}{2.180237in}}%
\pgfpathlineto{\pgfqpoint{0.632848in}{2.180702in}}%
\pgfpathlineto{\pgfqpoint{0.633979in}{2.190351in}}%
\pgfpathlineto{\pgfqpoint{0.634712in}{2.195124in}}%
\pgfpathlineto{\pgfqpoint{0.638847in}{2.200000in}}%
\pgfpathlineto{\pgfqpoint{0.638160in}{2.209649in}}%
\pgfpathlineto{\pgfqpoint{0.634712in}{2.213422in}}%
\pgfpathlineto{\pgfqpoint{0.633448in}{2.219298in}}%
\pgfpathlineto{\pgfqpoint{0.633912in}{2.228947in}}%
\pgfpathlineto{\pgfqpoint{0.634712in}{2.232779in}}%
\pgfpathlineto{\pgfqpoint{0.637660in}{2.238596in}}%
\pgfpathlineto{\pgfqpoint{0.634712in}{2.244795in}}%
\pgfpathlineto{\pgfqpoint{0.634332in}{2.248246in}}%
\pgfpathlineto{\pgfqpoint{0.634147in}{2.257895in}}%
\pgfpathlineto{\pgfqpoint{0.634712in}{2.263740in}}%
\pgfpathlineto{\pgfqpoint{0.635683in}{2.267544in}}%
\pgfpathlineto{\pgfqpoint{0.634712in}{2.270286in}}%
\pgfpathlineto{\pgfqpoint{0.633280in}{2.277193in}}%
\pgfpathlineto{\pgfqpoint{0.634684in}{2.286842in}}%
\pgfpathlineto{\pgfqpoint{0.633544in}{2.296491in}}%
\pgfpathlineto{\pgfqpoint{0.633947in}{2.306140in}}%
\pgfpathlineto{\pgfqpoint{0.625000in}{2.312559in}}%
\pgfusepath{stroke}%
\end{pgfscope}%
\begin{pgfscope}%
\pgfpathrectangle{\pgfqpoint{0.625000in}{0.550000in}}{\pgfqpoint{3.875000in}{3.850000in}} %
\pgfusepath{clip}%
\pgfsetbuttcap%
\pgfsetroundjoin%
\pgfsetlinewidth{0.250937pt}%
\definecolor{currentstroke}{rgb}{0.000000,0.000000,0.000000}%
\pgfsetstrokecolor{currentstroke}%
\pgfsetdash{}{0pt}%
\pgfpathmoveto{\pgfqpoint{0.625000in}{2.334590in}}%
\pgfpathlineto{\pgfqpoint{0.633530in}{2.335088in}}%
\pgfpathlineto{\pgfqpoint{0.632230in}{2.344737in}}%
\pgfpathlineto{\pgfqpoint{0.632149in}{2.354386in}}%
\pgfpathlineto{\pgfqpoint{0.631828in}{2.364035in}}%
\pgfpathlineto{\pgfqpoint{0.628831in}{2.373684in}}%
\pgfpathlineto{\pgfqpoint{0.631309in}{2.383333in}}%
\pgfpathlineto{\pgfqpoint{0.629856in}{2.392982in}}%
\pgfpathlineto{\pgfqpoint{0.629665in}{2.402632in}}%
\pgfpathlineto{\pgfqpoint{0.625154in}{2.412281in}}%
\pgfpathlineto{\pgfqpoint{0.626897in}{2.421930in}}%
\pgfpathlineto{\pgfqpoint{0.626598in}{2.431579in}}%
\pgfpathlineto{\pgfqpoint{0.626206in}{2.441228in}}%
\pgfpathlineto{\pgfqpoint{0.625788in}{2.450877in}}%
\pgfpathlineto{\pgfqpoint{0.625254in}{2.460526in}}%
\pgfpathlineto{\pgfqpoint{0.625000in}{2.463889in}}%
\pgfusepath{stroke}%
\end{pgfscope}%
\begin{pgfscope}%
\pgfpathrectangle{\pgfqpoint{0.625000in}{0.550000in}}{\pgfqpoint{3.875000in}{3.850000in}} %
\pgfusepath{clip}%
\pgfsetbuttcap%
\pgfsetroundjoin%
\pgfsetlinewidth{0.250937pt}%
\definecolor{currentstroke}{rgb}{0.000000,0.000000,0.000000}%
\pgfsetstrokecolor{currentstroke}%
\pgfsetdash{}{0pt}%
\pgfpathmoveto{\pgfqpoint{0.625000in}{2.495760in}}%
\pgfpathlineto{\pgfqpoint{0.625253in}{2.499123in}}%
\pgfpathlineto{\pgfqpoint{0.625788in}{2.508772in}}%
\pgfpathlineto{\pgfqpoint{0.626206in}{2.518421in}}%
\pgfpathlineto{\pgfqpoint{0.626598in}{2.528070in}}%
\pgfpathlineto{\pgfqpoint{0.626897in}{2.537719in}}%
\pgfpathlineto{\pgfqpoint{0.625154in}{2.547368in}}%
\pgfpathlineto{\pgfqpoint{0.629665in}{2.557018in}}%
\pgfpathlineto{\pgfqpoint{0.629856in}{2.566667in}}%
\pgfpathlineto{\pgfqpoint{0.631309in}{2.576316in}}%
\pgfpathlineto{\pgfqpoint{0.628831in}{2.585965in}}%
\pgfpathlineto{\pgfqpoint{0.631828in}{2.595614in}}%
\pgfpathlineto{\pgfqpoint{0.632149in}{2.605263in}}%
\pgfpathlineto{\pgfqpoint{0.632230in}{2.614912in}}%
\pgfpathlineto{\pgfqpoint{0.633530in}{2.624561in}}%
\pgfpathlineto{\pgfqpoint{0.625000in}{2.625060in}}%
\pgfusepath{stroke}%
\end{pgfscope}%
\begin{pgfscope}%
\pgfpathrectangle{\pgfqpoint{0.625000in}{0.550000in}}{\pgfqpoint{3.875000in}{3.850000in}} %
\pgfusepath{clip}%
\pgfsetbuttcap%
\pgfsetroundjoin%
\pgfsetlinewidth{0.250937pt}%
\definecolor{currentstroke}{rgb}{0.000000,0.000000,0.000000}%
\pgfsetstrokecolor{currentstroke}%
\pgfsetdash{}{0pt}%
\pgfpathmoveto{\pgfqpoint{0.625000in}{2.647090in}}%
\pgfpathlineto{\pgfqpoint{0.633947in}{2.653509in}}%
\pgfpathlineto{\pgfqpoint{0.633544in}{2.663158in}}%
\pgfpathlineto{\pgfqpoint{0.634684in}{2.672807in}}%
\pgfpathlineto{\pgfqpoint{0.633280in}{2.682456in}}%
\pgfpathlineto{\pgfqpoint{0.634712in}{2.689363in}}%
\pgfpathlineto{\pgfqpoint{0.635683in}{2.692105in}}%
\pgfpathlineto{\pgfqpoint{0.634712in}{2.695909in}}%
\pgfpathlineto{\pgfqpoint{0.634147in}{2.701754in}}%
\pgfpathlineto{\pgfqpoint{0.634332in}{2.711404in}}%
\pgfpathlineto{\pgfqpoint{0.634712in}{2.714854in}}%
\pgfpathlineto{\pgfqpoint{0.637660in}{2.721053in}}%
\pgfpathlineto{\pgfqpoint{0.634712in}{2.726870in}}%
\pgfpathlineto{\pgfqpoint{0.633912in}{2.730702in}}%
\pgfpathlineto{\pgfqpoint{0.633448in}{2.740351in}}%
\pgfpathlineto{\pgfqpoint{0.634712in}{2.746227in}}%
\pgfpathlineto{\pgfqpoint{0.638160in}{2.750000in}}%
\pgfpathlineto{\pgfqpoint{0.638847in}{2.759649in}}%
\pgfpathlineto{\pgfqpoint{0.634712in}{2.764525in}}%
\pgfpathlineto{\pgfqpoint{0.633979in}{2.769298in}}%
\pgfpathlineto{\pgfqpoint{0.632848in}{2.778947in}}%
\pgfpathlineto{\pgfqpoint{0.625000in}{2.779415in}}%
\pgfusepath{stroke}%
\end{pgfscope}%
\begin{pgfscope}%
\pgfpathrectangle{\pgfqpoint{0.625000in}{0.550000in}}{\pgfqpoint{3.875000in}{3.850000in}} %
\pgfusepath{clip}%
\pgfsetbuttcap%
\pgfsetroundjoin%
\pgfsetlinewidth{0.250937pt}%
\definecolor{currentstroke}{rgb}{0.000000,0.000000,0.000000}%
\pgfsetstrokecolor{currentstroke}%
\pgfsetdash{}{0pt}%
\pgfpathmoveto{\pgfqpoint{0.625000in}{2.798043in}}%
\pgfpathlineto{\pgfqpoint{0.634712in}{2.788686in}}%
\pgfpathlineto{\pgfqpoint{0.641445in}{2.798246in}}%
\pgfpathlineto{\pgfqpoint{0.640304in}{2.807895in}}%
\pgfpathlineto{\pgfqpoint{0.634712in}{2.814574in}}%
\pgfpathlineto{\pgfqpoint{0.634215in}{2.817544in}}%
\pgfpathlineto{\pgfqpoint{0.631630in}{2.827193in}}%
\pgfpathlineto{\pgfqpoint{0.632389in}{2.836842in}}%
\pgfpathlineto{\pgfqpoint{0.634197in}{2.846491in}}%
\pgfpathlineto{\pgfqpoint{0.634712in}{2.848496in}}%
\pgfpathlineto{\pgfqpoint{0.641969in}{2.856140in}}%
\pgfpathlineto{\pgfqpoint{0.644364in}{2.865789in}}%
\pgfpathlineto{\pgfqpoint{0.641966in}{2.875439in}}%
\pgfpathlineto{\pgfqpoint{0.634712in}{2.882780in}}%
\pgfpathlineto{\pgfqpoint{0.634102in}{2.885088in}}%
\pgfpathlineto{\pgfqpoint{0.632845in}{2.894737in}}%
\pgfpathlineto{\pgfqpoint{0.630392in}{2.904386in}}%
\pgfpathlineto{\pgfqpoint{0.630490in}{2.914035in}}%
\pgfpathlineto{\pgfqpoint{0.633310in}{2.923684in}}%
\pgfpathlineto{\pgfqpoint{0.625000in}{2.932279in}}%
\pgfusepath{stroke}%
\end{pgfscope}%
\begin{pgfscope}%
\pgfpathrectangle{\pgfqpoint{0.625000in}{0.550000in}}{\pgfqpoint{3.875000in}{3.850000in}} %
\pgfusepath{clip}%
\pgfsetbuttcap%
\pgfsetroundjoin%
\pgfsetlinewidth{0.250937pt}%
\definecolor{currentstroke}{rgb}{0.000000,0.000000,0.000000}%
\pgfsetstrokecolor{currentstroke}%
\pgfsetdash{}{0pt}%
\pgfpathmoveto{\pgfqpoint{0.625000in}{2.952586in}}%
\pgfpathlineto{\pgfqpoint{0.634693in}{2.942982in}}%
\pgfpathlineto{\pgfqpoint{0.634712in}{2.939960in}}%
\pgfpathlineto{\pgfqpoint{0.639175in}{2.942982in}}%
\pgfpathlineto{\pgfqpoint{0.644424in}{2.948222in}}%
\pgfpathlineto{\pgfqpoint{0.647431in}{2.952632in}}%
\pgfpathlineto{\pgfqpoint{0.649666in}{2.962281in}}%
\pgfpathlineto{\pgfqpoint{0.647430in}{2.971930in}}%
\pgfpathlineto{\pgfqpoint{0.644424in}{2.976337in}}%
\pgfpathlineto{\pgfqpoint{0.639172in}{2.981579in}}%
\pgfpathlineto{\pgfqpoint{0.634712in}{2.984022in}}%
\pgfpathlineto{\pgfqpoint{0.633352in}{2.991228in}}%
\pgfpathlineto{\pgfqpoint{0.633673in}{3.000877in}}%
\pgfpathlineto{\pgfqpoint{0.633923in}{3.010526in}}%
\pgfpathlineto{\pgfqpoint{0.631528in}{3.020175in}}%
\pgfpathlineto{\pgfqpoint{0.625000in}{3.025989in}}%
\pgfusepath{stroke}%
\end{pgfscope}%
\begin{pgfscope}%
\pgfpathrectangle{\pgfqpoint{0.625000in}{0.550000in}}{\pgfqpoint{3.875000in}{3.850000in}} %
\pgfusepath{clip}%
\pgfsetbuttcap%
\pgfsetroundjoin%
\pgfsetlinewidth{0.250937pt}%
\definecolor{currentstroke}{rgb}{0.000000,0.000000,0.000000}%
\pgfsetstrokecolor{currentstroke}%
\pgfsetdash{}{0pt}%
\pgfpathmoveto{\pgfqpoint{0.625000in}{3.035146in}}%
\pgfpathlineto{\pgfqpoint{0.629062in}{3.039474in}}%
\pgfpathlineto{\pgfqpoint{0.629147in}{3.049123in}}%
\pgfpathlineto{\pgfqpoint{0.634712in}{3.058769in}}%
\pgfpathlineto{\pgfqpoint{0.634714in}{3.058772in}}%
\pgfpathlineto{\pgfqpoint{0.634712in}{3.058775in}}%
\pgfpathlineto{\pgfqpoint{0.625000in}{3.064879in}}%
\pgfusepath{stroke}%
\end{pgfscope}%
\begin{pgfscope}%
\pgfpathrectangle{\pgfqpoint{0.625000in}{0.550000in}}{\pgfqpoint{3.875000in}{3.850000in}} %
\pgfusepath{clip}%
\pgfsetbuttcap%
\pgfsetroundjoin%
\pgfsetlinewidth{0.250937pt}%
\definecolor{currentstroke}{rgb}{0.000000,0.000000,0.000000}%
\pgfsetstrokecolor{currentstroke}%
\pgfsetdash{}{0pt}%
\pgfpathmoveto{\pgfqpoint{0.625000in}{3.074069in}}%
\pgfpathlineto{\pgfqpoint{0.632375in}{3.078070in}}%
\pgfpathlineto{\pgfqpoint{0.634151in}{3.087719in}}%
\pgfpathlineto{\pgfqpoint{0.625000in}{3.088041in}}%
\pgfusepath{stroke}%
\end{pgfscope}%
\begin{pgfscope}%
\pgfpathrectangle{\pgfqpoint{0.625000in}{0.550000in}}{\pgfqpoint{3.875000in}{3.850000in}} %
\pgfusepath{clip}%
\pgfsetbuttcap%
\pgfsetroundjoin%
\pgfsetlinewidth{0.250937pt}%
\definecolor{currentstroke}{rgb}{0.000000,0.000000,0.000000}%
\pgfsetstrokecolor{currentstroke}%
\pgfsetdash{}{0pt}%
\pgfpathmoveto{\pgfqpoint{0.625000in}{3.106501in}}%
\pgfpathlineto{\pgfqpoint{0.634685in}{3.097368in}}%
\pgfpathlineto{\pgfqpoint{0.634712in}{3.091894in}}%
\pgfpathlineto{\pgfqpoint{0.644162in}{3.097368in}}%
\pgfpathlineto{\pgfqpoint{0.644424in}{3.097535in}}%
\pgfpathlineto{\pgfqpoint{0.652961in}{3.107018in}}%
\pgfpathlineto{\pgfqpoint{0.654135in}{3.109937in}}%
\pgfpathlineto{\pgfqpoint{0.656832in}{3.116667in}}%
\pgfpathlineto{\pgfqpoint{0.657371in}{3.126316in}}%
\pgfpathlineto{\pgfqpoint{0.654561in}{3.135965in}}%
\pgfpathlineto{\pgfqpoint{0.654135in}{3.136687in}}%
\pgfpathlineto{\pgfqpoint{0.647943in}{3.145614in}}%
\pgfpathlineto{\pgfqpoint{0.644424in}{3.148779in}}%
\pgfpathlineto{\pgfqpoint{0.634712in}{3.153764in}}%
\pgfpathlineto{\pgfqpoint{0.634565in}{3.155263in}}%
\pgfpathlineto{\pgfqpoint{0.634118in}{3.164912in}}%
\pgfpathlineto{\pgfqpoint{0.633031in}{3.174561in}}%
\pgfpathlineto{\pgfqpoint{0.631479in}{3.184211in}}%
\pgfpathlineto{\pgfqpoint{0.633870in}{3.193860in}}%
\pgfpathlineto{\pgfqpoint{0.634712in}{3.197818in}}%
\pgfpathlineto{\pgfqpoint{0.637831in}{3.203509in}}%
\pgfpathlineto{\pgfqpoint{0.634712in}{3.210155in}}%
\pgfpathlineto{\pgfqpoint{0.634311in}{3.213158in}}%
\pgfpathlineto{\pgfqpoint{0.633553in}{3.222807in}}%
\pgfpathlineto{\pgfqpoint{0.630474in}{3.232456in}}%
\pgfpathlineto{\pgfqpoint{0.628493in}{3.242105in}}%
\pgfpathlineto{\pgfqpoint{0.625000in}{3.242539in}}%
\pgfusepath{stroke}%
\end{pgfscope}%
\begin{pgfscope}%
\pgfpathrectangle{\pgfqpoint{0.625000in}{0.550000in}}{\pgfqpoint{3.875000in}{3.850000in}} %
\pgfusepath{clip}%
\pgfsetbuttcap%
\pgfsetroundjoin%
\pgfsetlinewidth{0.250937pt}%
\definecolor{currentstroke}{rgb}{0.000000,0.000000,0.000000}%
\pgfsetstrokecolor{currentstroke}%
\pgfsetdash{}{0pt}%
\pgfpathmoveto{\pgfqpoint{0.625000in}{3.261031in}}%
\pgfpathlineto{\pgfqpoint{0.628147in}{3.261404in}}%
\pgfpathlineto{\pgfqpoint{0.630474in}{3.271053in}}%
\pgfpathlineto{\pgfqpoint{0.632546in}{3.280702in}}%
\pgfpathlineto{\pgfqpoint{0.633981in}{3.290351in}}%
\pgfpathlineto{\pgfqpoint{0.634712in}{3.295754in}}%
\pgfpathlineto{\pgfqpoint{0.637255in}{3.300000in}}%
\pgfpathlineto{\pgfqpoint{0.639344in}{3.309649in}}%
\pgfpathlineto{\pgfqpoint{0.634712in}{3.316669in}}%
\pgfpathlineto{\pgfqpoint{0.625000in}{3.317087in}}%
\pgfusepath{stroke}%
\end{pgfscope}%
\begin{pgfscope}%
\pgfpathrectangle{\pgfqpoint{0.625000in}{0.550000in}}{\pgfqpoint{3.875000in}{3.850000in}} %
\pgfusepath{clip}%
\pgfsetbuttcap%
\pgfsetroundjoin%
\pgfsetlinewidth{0.250937pt}%
\definecolor{currentstroke}{rgb}{0.000000,0.000000,0.000000}%
\pgfsetstrokecolor{currentstroke}%
\pgfsetdash{}{0pt}%
\pgfpathmoveto{\pgfqpoint{0.625000in}{3.320626in}}%
\pgfpathlineto{\pgfqpoint{0.632332in}{3.328947in}}%
\pgfpathlineto{\pgfqpoint{0.625537in}{3.338596in}}%
\pgfpathlineto{\pgfqpoint{0.633082in}{3.348246in}}%
\pgfpathlineto{\pgfqpoint{0.634432in}{3.357895in}}%
\pgfpathlineto{\pgfqpoint{0.632633in}{3.367544in}}%
\pgfpathlineto{\pgfqpoint{0.630032in}{3.377193in}}%
\pgfpathlineto{\pgfqpoint{0.633547in}{3.386842in}}%
\pgfpathlineto{\pgfqpoint{0.634657in}{3.396491in}}%
\pgfpathlineto{\pgfqpoint{0.625000in}{3.396932in}}%
\pgfusepath{stroke}%
\end{pgfscope}%
\begin{pgfscope}%
\pgfpathrectangle{\pgfqpoint{0.625000in}{0.550000in}}{\pgfqpoint{3.875000in}{3.850000in}} %
\pgfusepath{clip}%
\pgfsetbuttcap%
\pgfsetroundjoin%
\pgfsetlinewidth{0.250937pt}%
\definecolor{currentstroke}{rgb}{0.000000,0.000000,0.000000}%
\pgfsetstrokecolor{currentstroke}%
\pgfsetdash{}{0pt}%
\pgfpathmoveto{\pgfqpoint{0.625000in}{3.415658in}}%
\pgfpathlineto{\pgfqpoint{0.634679in}{3.406140in}}%
\pgfpathlineto{\pgfqpoint{0.634712in}{3.397201in}}%
\pgfpathlineto{\pgfqpoint{0.644424in}{3.400569in}}%
\pgfpathlineto{\pgfqpoint{0.653764in}{3.406140in}}%
\pgfpathlineto{\pgfqpoint{0.654135in}{3.406391in}}%
\pgfpathlineto{\pgfqpoint{0.663595in}{3.415789in}}%
\pgfpathlineto{\pgfqpoint{0.663847in}{3.416159in}}%
\pgfpathlineto{\pgfqpoint{0.669456in}{3.425439in}}%
\pgfpathlineto{\pgfqpoint{0.672467in}{3.435088in}}%
\pgfpathlineto{\pgfqpoint{0.673373in}{3.444737in}}%
\pgfpathlineto{\pgfqpoint{0.672467in}{3.454386in}}%
\pgfpathlineto{\pgfqpoint{0.669456in}{3.464035in}}%
\pgfpathlineto{\pgfqpoint{0.663847in}{3.473315in}}%
\pgfpathlineto{\pgfqpoint{0.663595in}{3.473684in}}%
\pgfpathlineto{\pgfqpoint{0.654135in}{3.483083in}}%
\pgfpathlineto{\pgfqpoint{0.653764in}{3.483333in}}%
\pgfpathlineto{\pgfqpoint{0.644424in}{3.488909in}}%
\pgfpathlineto{\pgfqpoint{0.634712in}{3.491551in}}%
\pgfpathlineto{\pgfqpoint{0.634585in}{3.492982in}}%
\pgfpathlineto{\pgfqpoint{0.633730in}{3.502632in}}%
\pgfpathlineto{\pgfqpoint{0.633206in}{3.512281in}}%
\pgfpathlineto{\pgfqpoint{0.633585in}{3.521930in}}%
\pgfpathlineto{\pgfqpoint{0.625134in}{3.531579in}}%
\pgfpathlineto{\pgfqpoint{0.633568in}{3.541228in}}%
\pgfpathlineto{\pgfqpoint{0.634712in}{3.549101in}}%
\pgfpathlineto{\pgfqpoint{0.636235in}{3.550877in}}%
\pgfpathlineto{\pgfqpoint{0.634712in}{3.554047in}}%
\pgfpathlineto{\pgfqpoint{0.625000in}{3.551270in}}%
\pgfusepath{stroke}%
\end{pgfscope}%
\begin{pgfscope}%
\pgfpathrectangle{\pgfqpoint{0.625000in}{0.550000in}}{\pgfqpoint{3.875000in}{3.850000in}} %
\pgfusepath{clip}%
\pgfsetbuttcap%
\pgfsetroundjoin%
\pgfsetlinewidth{0.250937pt}%
\definecolor{currentstroke}{rgb}{0.000000,0.000000,0.000000}%
\pgfsetstrokecolor{currentstroke}%
\pgfsetdash{}{0pt}%
\pgfpathmoveto{\pgfqpoint{0.625000in}{3.569788in}}%
\pgfpathlineto{\pgfqpoint{0.631434in}{3.570175in}}%
\pgfpathlineto{\pgfqpoint{0.627672in}{3.579825in}}%
\pgfpathlineto{\pgfqpoint{0.629141in}{3.589474in}}%
\pgfpathlineto{\pgfqpoint{0.632777in}{3.599123in}}%
\pgfpathlineto{\pgfqpoint{0.633976in}{3.608772in}}%
\pgfpathlineto{\pgfqpoint{0.633947in}{3.618421in}}%
\pgfpathlineto{\pgfqpoint{0.634712in}{3.621181in}}%
\pgfpathlineto{\pgfqpoint{0.641966in}{3.628070in}}%
\pgfpathlineto{\pgfqpoint{0.644364in}{3.637719in}}%
\pgfpathlineto{\pgfqpoint{0.641971in}{3.647368in}}%
\pgfpathlineto{\pgfqpoint{0.634712in}{3.654710in}}%
\pgfpathlineto{\pgfqpoint{0.634102in}{3.657018in}}%
\pgfpathlineto{\pgfqpoint{0.625000in}{3.661259in}}%
\pgfusepath{stroke}%
\end{pgfscope}%
\begin{pgfscope}%
\pgfpathrectangle{\pgfqpoint{0.625000in}{0.550000in}}{\pgfqpoint{3.875000in}{3.850000in}} %
\pgfusepath{clip}%
\pgfsetbuttcap%
\pgfsetroundjoin%
\pgfsetlinewidth{0.250937pt}%
\definecolor{currentstroke}{rgb}{0.000000,0.000000,0.000000}%
\pgfsetstrokecolor{currentstroke}%
\pgfsetdash{}{0pt}%
\pgfpathmoveto{\pgfqpoint{0.625000in}{3.671211in}}%
\pgfpathlineto{\pgfqpoint{0.634091in}{3.676316in}}%
\pgfpathlineto{\pgfqpoint{0.634712in}{3.680345in}}%
\pgfpathlineto{\pgfqpoint{0.637729in}{3.685965in}}%
\pgfpathlineto{\pgfqpoint{0.634712in}{3.692521in}}%
\pgfpathlineto{\pgfqpoint{0.634125in}{3.695614in}}%
\pgfpathlineto{\pgfqpoint{0.629099in}{3.705263in}}%
\pgfpathlineto{\pgfqpoint{0.625000in}{3.705278in}}%
\pgfusepath{stroke}%
\end{pgfscope}%
\begin{pgfscope}%
\pgfpathrectangle{\pgfqpoint{0.625000in}{0.550000in}}{\pgfqpoint{3.875000in}{3.850000in}} %
\pgfusepath{clip}%
\pgfsetbuttcap%
\pgfsetroundjoin%
\pgfsetlinewidth{0.250937pt}%
\definecolor{currentstroke}{rgb}{0.000000,0.000000,0.000000}%
\pgfsetstrokecolor{currentstroke}%
\pgfsetdash{}{0pt}%
\pgfpathmoveto{\pgfqpoint{0.625000in}{3.724264in}}%
\pgfpathlineto{\pgfqpoint{0.632008in}{3.724561in}}%
\pgfpathlineto{\pgfqpoint{0.631562in}{3.734211in}}%
\pgfpathlineto{\pgfqpoint{0.629427in}{3.743860in}}%
\pgfpathlineto{\pgfqpoint{0.626745in}{3.753509in}}%
\pgfpathlineto{\pgfqpoint{0.625000in}{3.760092in}}%
\pgfusepath{stroke}%
\end{pgfscope}%
\begin{pgfscope}%
\pgfpathrectangle{\pgfqpoint{0.625000in}{0.550000in}}{\pgfqpoint{3.875000in}{3.850000in}} %
\pgfusepath{clip}%
\pgfsetbuttcap%
\pgfsetroundjoin%
\pgfsetlinewidth{0.250937pt}%
\definecolor{currentstroke}{rgb}{0.000000,0.000000,0.000000}%
\pgfsetstrokecolor{currentstroke}%
\pgfsetdash{}{0pt}%
\pgfpathmoveto{\pgfqpoint{0.625000in}{3.767018in}}%
\pgfpathlineto{\pgfqpoint{0.625667in}{3.772807in}}%
\pgfpathlineto{\pgfqpoint{0.626760in}{3.782456in}}%
\pgfpathlineto{\pgfqpoint{0.630048in}{3.792105in}}%
\pgfpathlineto{\pgfqpoint{0.630142in}{3.801754in}}%
\pgfpathlineto{\pgfqpoint{0.632375in}{3.811404in}}%
\pgfpathlineto{\pgfqpoint{0.633811in}{3.821053in}}%
\pgfpathlineto{\pgfqpoint{0.634690in}{3.830702in}}%
\pgfpathlineto{\pgfqpoint{0.633895in}{3.840351in}}%
\pgfpathlineto{\pgfqpoint{0.634712in}{3.847725in}}%
\pgfpathlineto{\pgfqpoint{0.636161in}{3.850000in}}%
\pgfpathlineto{\pgfqpoint{0.639572in}{3.859649in}}%
\pgfpathlineto{\pgfqpoint{0.634712in}{3.867706in}}%
\pgfpathlineto{\pgfqpoint{0.625000in}{3.860007in}}%
\pgfusepath{stroke}%
\end{pgfscope}%
\begin{pgfscope}%
\pgfpathrectangle{\pgfqpoint{0.625000in}{0.550000in}}{\pgfqpoint{3.875000in}{3.850000in}} %
\pgfusepath{clip}%
\pgfsetbuttcap%
\pgfsetroundjoin%
\pgfsetlinewidth{0.250937pt}%
\definecolor{currentstroke}{rgb}{0.000000,0.000000,0.000000}%
\pgfsetstrokecolor{currentstroke}%
\pgfsetdash{}{0pt}%
\pgfpathmoveto{\pgfqpoint{0.625000in}{3.878552in}}%
\pgfpathlineto{\pgfqpoint{0.631839in}{3.878947in}}%
\pgfpathlineto{\pgfqpoint{0.631363in}{3.888596in}}%
\pgfpathlineto{\pgfqpoint{0.634063in}{3.898246in}}%
\pgfpathlineto{\pgfqpoint{0.634712in}{3.904073in}}%
\pgfpathlineto{\pgfqpoint{0.639120in}{3.907895in}}%
\pgfpathlineto{\pgfqpoint{0.644424in}{3.913140in}}%
\pgfpathlineto{\pgfqpoint{0.647430in}{3.917544in}}%
\pgfpathlineto{\pgfqpoint{0.649666in}{3.927193in}}%
\pgfpathlineto{\pgfqpoint{0.647430in}{3.936842in}}%
\pgfpathlineto{\pgfqpoint{0.644424in}{3.941244in}}%
\pgfpathlineto{\pgfqpoint{0.639118in}{3.946491in}}%
\pgfpathlineto{\pgfqpoint{0.634712in}{3.949471in}}%
\pgfpathlineto{\pgfqpoint{0.633944in}{3.956140in}}%
\pgfpathlineto{\pgfqpoint{0.631735in}{3.965789in}}%
\pgfpathlineto{\pgfqpoint{0.634590in}{3.975439in}}%
\pgfpathlineto{\pgfqpoint{0.634712in}{3.983301in}}%
\pgfpathlineto{\pgfqpoint{0.634831in}{3.985088in}}%
\pgfpathlineto{\pgfqpoint{0.634712in}{3.985317in}}%
\pgfpathlineto{\pgfqpoint{0.632696in}{3.994737in}}%
\pgfpathlineto{\pgfqpoint{0.630474in}{4.004386in}}%
\pgfpathlineto{\pgfqpoint{0.628383in}{4.014035in}}%
\pgfpathlineto{\pgfqpoint{0.625000in}{4.014442in}}%
\pgfusepath{stroke}%
\end{pgfscope}%
\begin{pgfscope}%
\pgfpathrectangle{\pgfqpoint{0.625000in}{0.550000in}}{\pgfqpoint{3.875000in}{3.850000in}} %
\pgfusepath{clip}%
\pgfsetbuttcap%
\pgfsetroundjoin%
\pgfsetlinewidth{0.250937pt}%
\definecolor{currentstroke}{rgb}{0.000000,0.000000,0.000000}%
\pgfsetstrokecolor{currentstroke}%
\pgfsetdash{}{0pt}%
\pgfpathmoveto{\pgfqpoint{0.625000in}{4.033017in}}%
\pgfpathlineto{\pgfqpoint{0.627829in}{4.033333in}}%
\pgfpathlineto{\pgfqpoint{0.630474in}{4.042982in}}%
\pgfpathlineto{\pgfqpoint{0.625000in}{4.048050in}}%
\pgfusepath{stroke}%
\end{pgfscope}%
\begin{pgfscope}%
\pgfpathrectangle{\pgfqpoint{0.625000in}{0.550000in}}{\pgfqpoint{3.875000in}{3.850000in}} %
\pgfusepath{clip}%
\pgfsetbuttcap%
\pgfsetroundjoin%
\pgfsetlinewidth{0.250937pt}%
\definecolor{currentstroke}{rgb}{0.000000,0.000000,0.000000}%
\pgfsetstrokecolor{currentstroke}%
\pgfsetdash{}{0pt}%
\pgfpathmoveto{\pgfqpoint{0.625000in}{4.056366in}}%
\pgfpathlineto{\pgfqpoint{0.634225in}{4.062281in}}%
\pgfpathlineto{\pgfqpoint{0.634115in}{4.071930in}}%
\pgfpathlineto{\pgfqpoint{0.634712in}{4.075284in}}%
\pgfpathlineto{\pgfqpoint{0.640273in}{4.081579in}}%
\pgfpathlineto{\pgfqpoint{0.641455in}{4.091228in}}%
\pgfpathlineto{\pgfqpoint{0.634712in}{4.100788in}}%
\pgfpathlineto{\pgfqpoint{0.634698in}{4.100877in}}%
\pgfpathlineto{\pgfqpoint{0.633941in}{4.110526in}}%
\pgfpathlineto{\pgfqpoint{0.632541in}{4.120175in}}%
\pgfpathlineto{\pgfqpoint{0.629324in}{4.129825in}}%
\pgfpathlineto{\pgfqpoint{0.629033in}{4.139474in}}%
\pgfpathlineto{\pgfqpoint{0.630338in}{4.149123in}}%
\pgfpathlineto{\pgfqpoint{0.634083in}{4.158772in}}%
\pgfpathlineto{\pgfqpoint{0.634712in}{4.162021in}}%
\pgfpathlineto{\pgfqpoint{0.637905in}{4.168421in}}%
\pgfpathlineto{\pgfqpoint{0.634712in}{4.174421in}}%
\pgfpathlineto{\pgfqpoint{0.625000in}{4.169136in}}%
\pgfusepath{stroke}%
\end{pgfscope}%
\begin{pgfscope}%
\pgfpathrectangle{\pgfqpoint{0.625000in}{0.550000in}}{\pgfqpoint{3.875000in}{3.850000in}} %
\pgfusepath{clip}%
\pgfsetbuttcap%
\pgfsetroundjoin%
\pgfsetlinewidth{0.250937pt}%
\definecolor{currentstroke}{rgb}{0.000000,0.000000,0.000000}%
\pgfsetstrokecolor{currentstroke}%
\pgfsetdash{}{0pt}%
\pgfpathmoveto{\pgfqpoint{0.625000in}{4.187443in}}%
\pgfpathlineto{\pgfqpoint{0.630559in}{4.187719in}}%
\pgfpathlineto{\pgfqpoint{0.629062in}{4.197368in}}%
\pgfpathlineto{\pgfqpoint{0.633342in}{4.207018in}}%
\pgfpathlineto{\pgfqpoint{0.634712in}{4.216615in}}%
\pgfpathlineto{\pgfqpoint{0.634762in}{4.216667in}}%
\pgfpathlineto{\pgfqpoint{0.634712in}{4.216719in}}%
\pgfpathlineto{\pgfqpoint{0.633325in}{4.226316in}}%
\pgfpathlineto{\pgfqpoint{0.630474in}{4.235965in}}%
\pgfpathlineto{\pgfqpoint{0.633796in}{4.245614in}}%
\pgfpathlineto{\pgfqpoint{0.633575in}{4.255263in}}%
\pgfpathlineto{\pgfqpoint{0.632611in}{4.264912in}}%
\pgfpathlineto{\pgfqpoint{0.633561in}{4.274561in}}%
\pgfpathlineto{\pgfqpoint{0.632349in}{4.284211in}}%
\pgfpathlineto{\pgfqpoint{0.633942in}{4.293860in}}%
\pgfpathlineto{\pgfqpoint{0.634368in}{4.303509in}}%
\pgfpathlineto{\pgfqpoint{0.634703in}{4.313158in}}%
\pgfpathlineto{\pgfqpoint{0.634712in}{4.313428in}}%
\pgfpathlineto{\pgfqpoint{0.644424in}{4.315443in}}%
\pgfpathlineto{\pgfqpoint{0.654135in}{4.317929in}}%
\pgfpathlineto{\pgfqpoint{0.663847in}{4.321624in}}%
\pgfpathlineto{\pgfqpoint{0.666247in}{4.322807in}}%
\pgfpathlineto{\pgfqpoint{0.673559in}{4.326495in}}%
\pgfpathlineto{\pgfqpoint{0.682559in}{4.332456in}}%
\pgfpathlineto{\pgfqpoint{0.683271in}{4.332963in}}%
\pgfpathlineto{\pgfqpoint{0.692982in}{4.341315in}}%
\pgfpathlineto{\pgfqpoint{0.693778in}{4.342105in}}%
\pgfpathlineto{\pgfqpoint{0.702184in}{4.351754in}}%
\pgfpathlineto{\pgfqpoint{0.702694in}{4.352461in}}%
\pgfpathlineto{\pgfqpoint{0.708694in}{4.361404in}}%
\pgfpathlineto{\pgfqpoint{0.712406in}{4.368668in}}%
\pgfpathlineto{\pgfqpoint{0.713596in}{4.371053in}}%
\pgfpathlineto{\pgfqpoint{0.717315in}{4.380702in}}%
\pgfpathlineto{\pgfqpoint{0.719819in}{4.390351in}}%
\pgfpathlineto{\pgfqpoint{0.721256in}{4.400000in}}%
\pgfusepath{stroke}%
\end{pgfscope}%
\begin{pgfscope}%
\pgfpathrectangle{\pgfqpoint{0.625000in}{0.550000in}}{\pgfqpoint{3.875000in}{3.850000in}} %
\pgfusepath{clip}%
\pgfsetbuttcap%
\pgfsetroundjoin%
\pgfsetlinewidth{0.250937pt}%
\definecolor{currentstroke}{rgb}{0.000000,0.000000,0.000000}%
\pgfsetstrokecolor{currentstroke}%
\pgfsetdash{}{0pt}%
\pgfpathmoveto{\pgfqpoint{0.625000in}{4.341769in}}%
\pgfpathlineto{\pgfqpoint{0.634673in}{4.332456in}}%
\pgfpathlineto{\pgfqpoint{0.625000in}{4.323112in}}%
\pgfusepath{stroke}%
\end{pgfscope}%
\begin{pgfscope}%
\pgfpathrectangle{\pgfqpoint{0.625000in}{0.550000in}}{\pgfqpoint{3.875000in}{3.850000in}} %
\pgfusepath{clip}%
\pgfsetbuttcap%
\pgfsetroundjoin%
\pgfsetlinewidth{0.250937pt}%
\definecolor{currentstroke}{rgb}{0.000000,0.000000,0.000000}%
\pgfsetstrokecolor{currentstroke}%
\pgfsetdash{}{0pt}%
\pgfpathmoveto{\pgfqpoint{0.683684in}{0.550000in}}%
\pgfpathlineto{\pgfqpoint{0.683271in}{0.554619in}}%
\pgfpathlineto{\pgfqpoint{0.682878in}{0.559649in}}%
\pgfpathlineto{\pgfqpoint{0.680485in}{0.569298in}}%
\pgfpathlineto{\pgfqpoint{0.675994in}{0.578947in}}%
\pgfpathlineto{\pgfqpoint{0.673559in}{0.582569in}}%
\pgfpathlineto{\pgfqpoint{0.668977in}{0.588596in}}%
\pgfpathlineto{\pgfqpoint{0.663847in}{0.593693in}}%
\pgfpathlineto{\pgfqpoint{0.657781in}{0.598246in}}%
\pgfpathlineto{\pgfqpoint{0.654135in}{0.600665in}}%
\pgfpathlineto{\pgfqpoint{0.644424in}{0.605127in}}%
\pgfpathlineto{\pgfqpoint{0.634712in}{0.607504in}}%
\pgfpathlineto{\pgfqpoint{0.634649in}{0.607895in}}%
\pgfpathlineto{\pgfqpoint{0.633745in}{0.617544in}}%
\pgfpathlineto{\pgfqpoint{0.625000in}{0.617760in}}%
\pgfusepath{stroke}%
\end{pgfscope}%
\begin{pgfscope}%
\pgfpathrectangle{\pgfqpoint{0.625000in}{0.550000in}}{\pgfqpoint{3.875000in}{3.850000in}} %
\pgfusepath{clip}%
\pgfsetbuttcap%
\pgfsetroundjoin%
\pgfsetlinewidth{0.250937pt}%
\definecolor{currentstroke}{rgb}{0.000000,0.000000,0.000000}%
\pgfsetstrokecolor{currentstroke}%
\pgfsetdash{}{0pt}%
\pgfpathmoveto{\pgfqpoint{0.625000in}{0.636653in}}%
\pgfpathlineto{\pgfqpoint{0.632330in}{0.636842in}}%
\pgfpathlineto{\pgfqpoint{0.633194in}{0.646491in}}%
\pgfpathlineto{\pgfqpoint{0.632238in}{0.656140in}}%
\pgfpathlineto{\pgfqpoint{0.631997in}{0.665789in}}%
\pgfpathlineto{\pgfqpoint{0.628755in}{0.675439in}}%
\pgfpathlineto{\pgfqpoint{0.630894in}{0.685088in}}%
\pgfpathlineto{\pgfqpoint{0.630768in}{0.694737in}}%
\pgfpathlineto{\pgfqpoint{0.632142in}{0.704386in}}%
\pgfpathlineto{\pgfqpoint{0.631871in}{0.714035in}}%
\pgfpathlineto{\pgfqpoint{0.628769in}{0.723684in}}%
\pgfpathlineto{\pgfqpoint{0.631995in}{0.733333in}}%
\pgfpathlineto{\pgfqpoint{0.633475in}{0.742982in}}%
\pgfpathlineto{\pgfqpoint{0.632028in}{0.752632in}}%
\pgfpathlineto{\pgfqpoint{0.627797in}{0.762281in}}%
\pgfpathlineto{\pgfqpoint{0.628777in}{0.771930in}}%
\pgfpathlineto{\pgfqpoint{0.625000in}{0.772100in}}%
\pgfusepath{stroke}%
\end{pgfscope}%
\begin{pgfscope}%
\pgfpathrectangle{\pgfqpoint{0.625000in}{0.550000in}}{\pgfqpoint{3.875000in}{3.850000in}} %
\pgfusepath{clip}%
\pgfsetbuttcap%
\pgfsetroundjoin%
\pgfsetlinewidth{0.250937pt}%
\definecolor{currentstroke}{rgb}{0.000000,0.000000,0.000000}%
\pgfsetstrokecolor{currentstroke}%
\pgfsetdash{}{0pt}%
\pgfpathmoveto{\pgfqpoint{0.625000in}{0.790654in}}%
\pgfpathlineto{\pgfqpoint{0.634032in}{0.791228in}}%
\pgfpathlineto{\pgfqpoint{0.631253in}{0.800877in}}%
\pgfpathlineto{\pgfqpoint{0.627349in}{0.810526in}}%
\pgfpathlineto{\pgfqpoint{0.628156in}{0.820175in}}%
\pgfpathlineto{\pgfqpoint{0.628608in}{0.829825in}}%
\pgfpathlineto{\pgfqpoint{0.631368in}{0.839474in}}%
\pgfpathlineto{\pgfqpoint{0.632006in}{0.849123in}}%
\pgfpathlineto{\pgfqpoint{0.633077in}{0.858772in}}%
\pgfpathlineto{\pgfqpoint{0.634411in}{0.868421in}}%
\pgfpathlineto{\pgfqpoint{0.633369in}{0.878070in}}%
\pgfpathlineto{\pgfqpoint{0.632380in}{0.887719in}}%
\pgfpathlineto{\pgfqpoint{0.632218in}{0.897368in}}%
\pgfpathlineto{\pgfqpoint{0.625000in}{0.901996in}}%
\pgfusepath{stroke}%
\end{pgfscope}%
\begin{pgfscope}%
\pgfpathrectangle{\pgfqpoint{0.625000in}{0.550000in}}{\pgfqpoint{3.875000in}{3.850000in}} %
\pgfusepath{clip}%
\pgfsetbuttcap%
\pgfsetroundjoin%
\pgfsetlinewidth{0.250937pt}%
\definecolor{currentstroke}{rgb}{0.000000,0.000000,0.000000}%
\pgfsetstrokecolor{currentstroke}%
\pgfsetdash{}{0pt}%
\pgfpathmoveto{\pgfqpoint{0.625000in}{0.913178in}}%
\pgfpathlineto{\pgfqpoint{0.628769in}{0.916667in}}%
\pgfpathlineto{\pgfqpoint{0.627073in}{0.926316in}}%
\pgfpathlineto{\pgfqpoint{0.625000in}{0.926528in}}%
\pgfusepath{stroke}%
\end{pgfscope}%
\begin{pgfscope}%
\pgfpathrectangle{\pgfqpoint{0.625000in}{0.550000in}}{\pgfqpoint{3.875000in}{3.850000in}} %
\pgfusepath{clip}%
\pgfsetbuttcap%
\pgfsetroundjoin%
\pgfsetlinewidth{0.250937pt}%
\definecolor{currentstroke}{rgb}{0.000000,0.000000,0.000000}%
\pgfsetstrokecolor{currentstroke}%
\pgfsetdash{}{0pt}%
\pgfpathmoveto{\pgfqpoint{0.625000in}{0.945318in}}%
\pgfpathlineto{\pgfqpoint{0.627688in}{0.945614in}}%
\pgfpathlineto{\pgfqpoint{0.628769in}{0.955263in}}%
\pgfpathlineto{\pgfqpoint{0.629760in}{0.964912in}}%
\pgfpathlineto{\pgfqpoint{0.632627in}{0.974561in}}%
\pgfpathlineto{\pgfqpoint{0.632940in}{0.984211in}}%
\pgfpathlineto{\pgfqpoint{0.629637in}{0.993860in}}%
\pgfpathlineto{\pgfqpoint{0.632328in}{1.003509in}}%
\pgfpathlineto{\pgfqpoint{0.633386in}{1.013158in}}%
\pgfpathlineto{\pgfqpoint{0.634712in}{1.021432in}}%
\pgfpathlineto{\pgfqpoint{0.636099in}{1.022807in}}%
\pgfpathlineto{\pgfqpoint{0.640485in}{1.032456in}}%
\pgfpathlineto{\pgfqpoint{0.636099in}{1.042105in}}%
\pgfpathlineto{\pgfqpoint{0.634712in}{1.043480in}}%
\pgfpathlineto{\pgfqpoint{0.632967in}{1.051754in}}%
\pgfpathlineto{\pgfqpoint{0.632061in}{1.061404in}}%
\pgfpathlineto{\pgfqpoint{0.629381in}{1.071053in}}%
\pgfpathlineto{\pgfqpoint{0.630340in}{1.080702in}}%
\pgfpathlineto{\pgfqpoint{0.625000in}{1.080983in}}%
\pgfusepath{stroke}%
\end{pgfscope}%
\begin{pgfscope}%
\pgfpathrectangle{\pgfqpoint{0.625000in}{0.550000in}}{\pgfqpoint{3.875000in}{3.850000in}} %
\pgfusepath{clip}%
\pgfsetbuttcap%
\pgfsetroundjoin%
\pgfsetlinewidth{0.250937pt}%
\definecolor{currentstroke}{rgb}{0.000000,0.000000,0.000000}%
\pgfsetstrokecolor{currentstroke}%
\pgfsetdash{}{0pt}%
\pgfpathmoveto{\pgfqpoint{0.625000in}{1.099754in}}%
\pgfpathlineto{\pgfqpoint{0.633817in}{1.100000in}}%
\pgfpathlineto{\pgfqpoint{0.632020in}{1.109649in}}%
\pgfpathlineto{\pgfqpoint{0.632244in}{1.119298in}}%
\pgfpathlineto{\pgfqpoint{0.633450in}{1.128947in}}%
\pgfpathlineto{\pgfqpoint{0.631990in}{1.138596in}}%
\pgfpathlineto{\pgfqpoint{0.630078in}{1.148246in}}%
\pgfpathlineto{\pgfqpoint{0.628182in}{1.157895in}}%
\pgfpathlineto{\pgfqpoint{0.629204in}{1.167544in}}%
\pgfpathlineto{\pgfqpoint{0.626217in}{1.177193in}}%
\pgfpathlineto{\pgfqpoint{0.625459in}{1.186842in}}%
\pgfpathlineto{\pgfqpoint{0.625000in}{1.190828in}}%
\pgfusepath{stroke}%
\end{pgfscope}%
\begin{pgfscope}%
\pgfpathrectangle{\pgfqpoint{0.625000in}{0.550000in}}{\pgfqpoint{3.875000in}{3.850000in}} %
\pgfusepath{clip}%
\pgfsetbuttcap%
\pgfsetroundjoin%
\pgfsetlinewidth{0.250937pt}%
\definecolor{currentstroke}{rgb}{0.000000,0.000000,0.000000}%
\pgfsetstrokecolor{currentstroke}%
\pgfsetdash{}{0pt}%
\pgfpathmoveto{\pgfqpoint{0.625000in}{1.200989in}}%
\pgfpathlineto{\pgfqpoint{0.626366in}{1.206140in}}%
\pgfpathlineto{\pgfqpoint{0.628724in}{1.215789in}}%
\pgfpathlineto{\pgfqpoint{0.630541in}{1.225439in}}%
\pgfpathlineto{\pgfqpoint{0.629993in}{1.235088in}}%
\pgfpathlineto{\pgfqpoint{0.625000in}{1.235287in}}%
\pgfusepath{stroke}%
\end{pgfscope}%
\begin{pgfscope}%
\pgfpathrectangle{\pgfqpoint{0.625000in}{0.550000in}}{\pgfqpoint{3.875000in}{3.850000in}} %
\pgfusepath{clip}%
\pgfsetbuttcap%
\pgfsetroundjoin%
\pgfsetlinewidth{0.250937pt}%
\definecolor{currentstroke}{rgb}{0.000000,0.000000,0.000000}%
\pgfsetstrokecolor{currentstroke}%
\pgfsetdash{}{0pt}%
\pgfpathmoveto{\pgfqpoint{0.625000in}{1.257017in}}%
\pgfpathlineto{\pgfqpoint{0.631283in}{1.264035in}}%
\pgfpathlineto{\pgfqpoint{0.634006in}{1.273684in}}%
\pgfpathlineto{\pgfqpoint{0.632113in}{1.283333in}}%
\pgfpathlineto{\pgfqpoint{0.625000in}{1.287327in}}%
\pgfusepath{stroke}%
\end{pgfscope}%
\begin{pgfscope}%
\pgfpathrectangle{\pgfqpoint{0.625000in}{0.550000in}}{\pgfqpoint{3.875000in}{3.850000in}} %
\pgfusepath{clip}%
\pgfsetbuttcap%
\pgfsetroundjoin%
\pgfsetlinewidth{0.250937pt}%
\definecolor{currentstroke}{rgb}{0.000000,0.000000,0.000000}%
\pgfsetstrokecolor{currentstroke}%
\pgfsetdash{}{0pt}%
\pgfpathmoveto{\pgfqpoint{0.625000in}{1.299711in}}%
\pgfpathlineto{\pgfqpoint{0.631267in}{1.302632in}}%
\pgfpathlineto{\pgfqpoint{0.634046in}{1.312281in}}%
\pgfpathlineto{\pgfqpoint{0.634712in}{1.316671in}}%
\pgfpathlineto{\pgfqpoint{0.637358in}{1.321930in}}%
\pgfpathlineto{\pgfqpoint{0.634712in}{1.327189in}}%
\pgfpathlineto{\pgfqpoint{0.634112in}{1.331579in}}%
\pgfpathlineto{\pgfqpoint{0.631160in}{1.341228in}}%
\pgfpathlineto{\pgfqpoint{0.631706in}{1.350877in}}%
\pgfpathlineto{\pgfqpoint{0.631085in}{1.360526in}}%
\pgfpathlineto{\pgfqpoint{0.628114in}{1.370175in}}%
\pgfpathlineto{\pgfqpoint{0.626840in}{1.379825in}}%
\pgfpathlineto{\pgfqpoint{0.630028in}{1.389474in}}%
\pgfpathlineto{\pgfqpoint{0.625000in}{1.389765in}}%
\pgfusepath{stroke}%
\end{pgfscope}%
\begin{pgfscope}%
\pgfpathrectangle{\pgfqpoint{0.625000in}{0.550000in}}{\pgfqpoint{3.875000in}{3.850000in}} %
\pgfusepath{clip}%
\pgfsetbuttcap%
\pgfsetroundjoin%
\pgfsetlinewidth{0.250937pt}%
\definecolor{currentstroke}{rgb}{0.000000,0.000000,0.000000}%
\pgfsetstrokecolor{currentstroke}%
\pgfsetdash{}{0pt}%
\pgfpathmoveto{\pgfqpoint{0.625000in}{1.408475in}}%
\pgfpathlineto{\pgfqpoint{0.632952in}{1.408772in}}%
\pgfpathlineto{\pgfqpoint{0.632235in}{1.418421in}}%
\pgfpathlineto{\pgfqpoint{0.625000in}{1.426616in}}%
\pgfusepath{stroke}%
\end{pgfscope}%
\begin{pgfscope}%
\pgfpathrectangle{\pgfqpoint{0.625000in}{0.550000in}}{\pgfqpoint{3.875000in}{3.850000in}} %
\pgfusepath{clip}%
\pgfsetbuttcap%
\pgfsetroundjoin%
\pgfsetlinewidth{0.250937pt}%
\definecolor{currentstroke}{rgb}{0.000000,0.000000,0.000000}%
\pgfsetstrokecolor{currentstroke}%
\pgfsetdash{}{0pt}%
\pgfpathmoveto{\pgfqpoint{0.625000in}{1.430109in}}%
\pgfpathlineto{\pgfqpoint{0.631717in}{1.437719in}}%
\pgfpathlineto{\pgfqpoint{0.631013in}{1.447368in}}%
\pgfpathlineto{\pgfqpoint{0.631010in}{1.457018in}}%
\pgfpathlineto{\pgfqpoint{0.632658in}{1.466667in}}%
\pgfpathlineto{\pgfqpoint{0.633632in}{1.476316in}}%
\pgfpathlineto{\pgfqpoint{0.634511in}{1.485965in}}%
\pgfpathlineto{\pgfqpoint{0.634712in}{1.487249in}}%
\pgfpathlineto{\pgfqpoint{0.644424in}{1.493013in}}%
\pgfpathlineto{\pgfqpoint{0.647042in}{1.495614in}}%
\pgfpathlineto{\pgfqpoint{0.652843in}{1.505263in}}%
\pgfpathlineto{\pgfqpoint{0.654135in}{1.513711in}}%
\pgfpathlineto{\pgfqpoint{0.654357in}{1.514912in}}%
\pgfpathlineto{\pgfqpoint{0.654135in}{1.516113in}}%
\pgfpathlineto{\pgfqpoint{0.652843in}{1.524561in}}%
\pgfpathlineto{\pgfqpoint{0.647042in}{1.534211in}}%
\pgfpathlineto{\pgfqpoint{0.644424in}{1.536812in}}%
\pgfpathlineto{\pgfqpoint{0.634712in}{1.542576in}}%
\pgfpathlineto{\pgfqpoint{0.633281in}{1.543860in}}%
\pgfpathlineto{\pgfqpoint{0.625000in}{1.543904in}}%
\pgfusepath{stroke}%
\end{pgfscope}%
\begin{pgfscope}%
\pgfpathrectangle{\pgfqpoint{0.625000in}{0.550000in}}{\pgfqpoint{3.875000in}{3.850000in}} %
\pgfusepath{clip}%
\pgfsetbuttcap%
\pgfsetroundjoin%
\pgfsetlinewidth{0.250937pt}%
\definecolor{currentstroke}{rgb}{0.000000,0.000000,0.000000}%
\pgfsetstrokecolor{currentstroke}%
\pgfsetdash{}{0pt}%
\pgfpathmoveto{\pgfqpoint{0.625000in}{1.562812in}}%
\pgfpathlineto{\pgfqpoint{0.632846in}{1.563158in}}%
\pgfpathlineto{\pgfqpoint{0.630885in}{1.572807in}}%
\pgfpathlineto{\pgfqpoint{0.625000in}{1.581436in}}%
\pgfusepath{stroke}%
\end{pgfscope}%
\begin{pgfscope}%
\pgfpathrectangle{\pgfqpoint{0.625000in}{0.550000in}}{\pgfqpoint{3.875000in}{3.850000in}} %
\pgfusepath{clip}%
\pgfsetbuttcap%
\pgfsetroundjoin%
\pgfsetlinewidth{0.250937pt}%
\definecolor{currentstroke}{rgb}{0.000000,0.000000,0.000000}%
\pgfsetstrokecolor{currentstroke}%
\pgfsetdash{}{0pt}%
\pgfpathmoveto{\pgfqpoint{0.625000in}{1.583110in}}%
\pgfpathlineto{\pgfqpoint{0.630972in}{1.592105in}}%
\pgfpathlineto{\pgfqpoint{0.632698in}{1.601754in}}%
\pgfpathlineto{\pgfqpoint{0.631825in}{1.611404in}}%
\pgfpathlineto{\pgfqpoint{0.625000in}{1.619916in}}%
\pgfusepath{stroke}%
\end{pgfscope}%
\begin{pgfscope}%
\pgfpathrectangle{\pgfqpoint{0.625000in}{0.550000in}}{\pgfqpoint{3.875000in}{3.850000in}} %
\pgfusepath{clip}%
\pgfsetbuttcap%
\pgfsetroundjoin%
\pgfsetlinewidth{0.250937pt}%
\definecolor{currentstroke}{rgb}{0.000000,0.000000,0.000000}%
\pgfsetstrokecolor{currentstroke}%
\pgfsetdash{}{0pt}%
\pgfpathmoveto{\pgfqpoint{0.625000in}{1.622278in}}%
\pgfpathlineto{\pgfqpoint{0.631107in}{1.630702in}}%
\pgfpathlineto{\pgfqpoint{0.625000in}{1.637632in}}%
\pgfusepath{stroke}%
\end{pgfscope}%
\begin{pgfscope}%
\pgfpathrectangle{\pgfqpoint{0.625000in}{0.550000in}}{\pgfqpoint{3.875000in}{3.850000in}} %
\pgfusepath{clip}%
\pgfsetbuttcap%
\pgfsetroundjoin%
\pgfsetlinewidth{0.250937pt}%
\definecolor{currentstroke}{rgb}{0.000000,0.000000,0.000000}%
\pgfsetstrokecolor{currentstroke}%
\pgfsetdash{}{0pt}%
\pgfpathmoveto{\pgfqpoint{0.625000in}{1.644879in}}%
\pgfpathlineto{\pgfqpoint{0.633307in}{1.650000in}}%
\pgfpathlineto{\pgfqpoint{0.633522in}{1.659649in}}%
\pgfpathlineto{\pgfqpoint{0.632027in}{1.669298in}}%
\pgfpathlineto{\pgfqpoint{0.629158in}{1.678947in}}%
\pgfpathlineto{\pgfqpoint{0.628769in}{1.688596in}}%
\pgfpathlineto{\pgfqpoint{0.627426in}{1.698246in}}%
\pgfpathlineto{\pgfqpoint{0.625000in}{1.698520in}}%
\pgfusepath{stroke}%
\end{pgfscope}%
\begin{pgfscope}%
\pgfpathrectangle{\pgfqpoint{0.625000in}{0.550000in}}{\pgfqpoint{3.875000in}{3.850000in}} %
\pgfusepath{clip}%
\pgfsetbuttcap%
\pgfsetroundjoin%
\pgfsetlinewidth{0.250937pt}%
\definecolor{currentstroke}{rgb}{0.000000,0.000000,0.000000}%
\pgfsetstrokecolor{currentstroke}%
\pgfsetdash{}{0pt}%
\pgfpathmoveto{\pgfqpoint{0.625000in}{1.717211in}}%
\pgfpathlineto{\pgfqpoint{0.627810in}{1.717544in}}%
\pgfpathlineto{\pgfqpoint{0.628769in}{1.727193in}}%
\pgfpathlineto{\pgfqpoint{0.631789in}{1.736842in}}%
\pgfpathlineto{\pgfqpoint{0.632285in}{1.746491in}}%
\pgfpathlineto{\pgfqpoint{0.634022in}{1.756140in}}%
\pgfpathlineto{\pgfqpoint{0.631107in}{1.765789in}}%
\pgfpathlineto{\pgfqpoint{0.630093in}{1.775439in}}%
\pgfpathlineto{\pgfqpoint{0.631689in}{1.785088in}}%
\pgfpathlineto{\pgfqpoint{0.632627in}{1.794737in}}%
\pgfpathlineto{\pgfqpoint{0.633077in}{1.804386in}}%
\pgfpathlineto{\pgfqpoint{0.632475in}{1.814035in}}%
\pgfpathlineto{\pgfqpoint{0.634712in}{1.819353in}}%
\pgfpathlineto{\pgfqpoint{0.639721in}{1.823684in}}%
\pgfpathlineto{\pgfqpoint{0.644318in}{1.833333in}}%
\pgfpathlineto{\pgfqpoint{0.643626in}{1.842982in}}%
\pgfpathlineto{\pgfqpoint{0.635879in}{1.852632in}}%
\pgfpathlineto{\pgfqpoint{0.634712in}{1.853521in}}%
\pgfpathlineto{\pgfqpoint{0.625000in}{1.853060in}}%
\pgfusepath{stroke}%
\end{pgfscope}%
\begin{pgfscope}%
\pgfpathrectangle{\pgfqpoint{0.625000in}{0.550000in}}{\pgfqpoint{3.875000in}{3.850000in}} %
\pgfusepath{clip}%
\pgfsetbuttcap%
\pgfsetroundjoin%
\pgfsetlinewidth{0.250937pt}%
\definecolor{currentstroke}{rgb}{0.000000,0.000000,0.000000}%
\pgfsetstrokecolor{currentstroke}%
\pgfsetdash{}{0pt}%
\pgfpathmoveto{\pgfqpoint{0.625000in}{1.871696in}}%
\pgfpathlineto{\pgfqpoint{0.631758in}{1.871930in}}%
\pgfpathlineto{\pgfqpoint{0.630078in}{1.881579in}}%
\pgfpathlineto{\pgfqpoint{0.625000in}{1.884334in}}%
\pgfusepath{stroke}%
\end{pgfscope}%
\begin{pgfscope}%
\pgfpathrectangle{\pgfqpoint{0.625000in}{0.550000in}}{\pgfqpoint{3.875000in}{3.850000in}} %
\pgfusepath{clip}%
\pgfsetbuttcap%
\pgfsetroundjoin%
\pgfsetlinewidth{0.250937pt}%
\definecolor{currentstroke}{rgb}{0.000000,0.000000,0.000000}%
\pgfsetstrokecolor{currentstroke}%
\pgfsetdash{}{0pt}%
\pgfpathmoveto{\pgfqpoint{0.625000in}{1.895551in}}%
\pgfpathlineto{\pgfqpoint{0.633469in}{1.900877in}}%
\pgfpathlineto{\pgfqpoint{0.625000in}{1.910081in}}%
\pgfusepath{stroke}%
\end{pgfscope}%
\begin{pgfscope}%
\pgfpathrectangle{\pgfqpoint{0.625000in}{0.550000in}}{\pgfqpoint{3.875000in}{3.850000in}} %
\pgfusepath{clip}%
\pgfsetbuttcap%
\pgfsetroundjoin%
\pgfsetlinewidth{0.250937pt}%
\definecolor{currentstroke}{rgb}{0.000000,0.000000,0.000000}%
\pgfsetstrokecolor{currentstroke}%
\pgfsetdash{}{0pt}%
\pgfpathmoveto{\pgfqpoint{0.625000in}{1.911779in}}%
\pgfpathlineto{\pgfqpoint{0.627797in}{1.920175in}}%
\pgfpathlineto{\pgfqpoint{0.625000in}{1.923155in}}%
\pgfusepath{stroke}%
\end{pgfscope}%
\begin{pgfscope}%
\pgfpathrectangle{\pgfqpoint{0.625000in}{0.550000in}}{\pgfqpoint{3.875000in}{3.850000in}} %
\pgfusepath{clip}%
\pgfsetbuttcap%
\pgfsetroundjoin%
\pgfsetlinewidth{0.250937pt}%
\definecolor{currentstroke}{rgb}{0.000000,0.000000,0.000000}%
\pgfsetstrokecolor{currentstroke}%
\pgfsetdash{}{0pt}%
\pgfpathmoveto{\pgfqpoint{0.625000in}{1.934632in}}%
\pgfpathlineto{\pgfqpoint{0.630437in}{1.939474in}}%
\pgfpathlineto{\pgfqpoint{0.632335in}{1.949123in}}%
\pgfpathlineto{\pgfqpoint{0.630972in}{1.958772in}}%
\pgfpathlineto{\pgfqpoint{0.631226in}{1.968421in}}%
\pgfpathlineto{\pgfqpoint{0.632975in}{1.978070in}}%
\pgfpathlineto{\pgfqpoint{0.634712in}{1.986338in}}%
\pgfpathlineto{\pgfqpoint{0.636099in}{1.987719in}}%
\pgfpathlineto{\pgfqpoint{0.640485in}{1.997368in}}%
\pgfpathlineto{\pgfqpoint{0.636099in}{2.007018in}}%
\pgfpathlineto{\pgfqpoint{0.634712in}{2.008399in}}%
\pgfpathlineto{\pgfqpoint{0.632646in}{2.007018in}}%
\pgfpathlineto{\pgfqpoint{0.625000in}{2.006506in}}%
\pgfusepath{stroke}%
\end{pgfscope}%
\begin{pgfscope}%
\pgfpathrectangle{\pgfqpoint{0.625000in}{0.550000in}}{\pgfqpoint{3.875000in}{3.850000in}} %
\pgfusepath{clip}%
\pgfsetbuttcap%
\pgfsetroundjoin%
\pgfsetlinewidth{0.250937pt}%
\definecolor{currentstroke}{rgb}{0.000000,0.000000,0.000000}%
\pgfsetstrokecolor{currentstroke}%
\pgfsetdash{}{0pt}%
\pgfpathmoveto{\pgfqpoint{0.625000in}{2.030047in}}%
\pgfpathlineto{\pgfqpoint{0.630722in}{2.035965in}}%
\pgfpathlineto{\pgfqpoint{0.629179in}{2.045614in}}%
\pgfpathlineto{\pgfqpoint{0.629219in}{2.055263in}}%
\pgfpathlineto{\pgfqpoint{0.630127in}{2.064912in}}%
\pgfpathlineto{\pgfqpoint{0.631267in}{2.074561in}}%
\pgfpathlineto{\pgfqpoint{0.633460in}{2.084211in}}%
\pgfpathlineto{\pgfqpoint{0.634712in}{2.088601in}}%
\pgfpathlineto{\pgfqpoint{0.637358in}{2.093860in}}%
\pgfpathlineto{\pgfqpoint{0.634712in}{2.099118in}}%
\pgfpathlineto{\pgfqpoint{0.633471in}{2.103509in}}%
\pgfpathlineto{\pgfqpoint{0.631332in}{2.113158in}}%
\pgfpathlineto{\pgfqpoint{0.630801in}{2.122807in}}%
\pgfpathlineto{\pgfqpoint{0.630188in}{2.132456in}}%
\pgfpathlineto{\pgfqpoint{0.632486in}{2.142105in}}%
\pgfpathlineto{\pgfqpoint{0.633369in}{2.151754in}}%
\pgfpathlineto{\pgfqpoint{0.634105in}{2.161404in}}%
\pgfpathlineto{\pgfqpoint{0.625000in}{2.161521in}}%
\pgfusepath{stroke}%
\end{pgfscope}%
\begin{pgfscope}%
\pgfpathrectangle{\pgfqpoint{0.625000in}{0.550000in}}{\pgfqpoint{3.875000in}{3.850000in}} %
\pgfusepath{clip}%
\pgfsetbuttcap%
\pgfsetroundjoin%
\pgfsetlinewidth{0.250937pt}%
\definecolor{currentstroke}{rgb}{0.000000,0.000000,0.000000}%
\pgfsetstrokecolor{currentstroke}%
\pgfsetdash{}{0pt}%
\pgfpathmoveto{\pgfqpoint{0.625000in}{2.180319in}}%
\pgfpathlineto{\pgfqpoint{0.631459in}{2.180702in}}%
\pgfpathlineto{\pgfqpoint{0.632582in}{2.190351in}}%
\pgfpathlineto{\pgfqpoint{0.633309in}{2.200000in}}%
\pgfpathlineto{\pgfqpoint{0.633367in}{2.209649in}}%
\pgfpathlineto{\pgfqpoint{0.631525in}{2.219298in}}%
\pgfpathlineto{\pgfqpoint{0.631136in}{2.228947in}}%
\pgfpathlineto{\pgfqpoint{0.633994in}{2.238596in}}%
\pgfpathlineto{\pgfqpoint{0.632772in}{2.248246in}}%
\pgfpathlineto{\pgfqpoint{0.631737in}{2.257895in}}%
\pgfpathlineto{\pgfqpoint{0.632020in}{2.267544in}}%
\pgfpathlineto{\pgfqpoint{0.629557in}{2.277193in}}%
\pgfpathlineto{\pgfqpoint{0.633445in}{2.286842in}}%
\pgfpathlineto{\pgfqpoint{0.630505in}{2.296491in}}%
\pgfpathlineto{\pgfqpoint{0.631160in}{2.306140in}}%
\pgfpathlineto{\pgfqpoint{0.625000in}{2.310560in}}%
\pgfusepath{stroke}%
\end{pgfscope}%
\begin{pgfscope}%
\pgfpathrectangle{\pgfqpoint{0.625000in}{0.550000in}}{\pgfqpoint{3.875000in}{3.850000in}} %
\pgfusepath{clip}%
\pgfsetbuttcap%
\pgfsetroundjoin%
\pgfsetlinewidth{0.250937pt}%
\definecolor{currentstroke}{rgb}{0.000000,0.000000,0.000000}%
\pgfsetstrokecolor{currentstroke}%
\pgfsetdash{}{0pt}%
\pgfpathmoveto{\pgfqpoint{0.625000in}{2.334672in}}%
\pgfpathlineto{\pgfqpoint{0.632113in}{2.335088in}}%
\pgfpathlineto{\pgfqpoint{0.629978in}{2.344737in}}%
\pgfpathlineto{\pgfqpoint{0.630398in}{2.354386in}}%
\pgfpathlineto{\pgfqpoint{0.630343in}{2.364035in}}%
\pgfpathlineto{\pgfqpoint{0.626555in}{2.373684in}}%
\pgfpathlineto{\pgfqpoint{0.630328in}{2.383333in}}%
\pgfpathlineto{\pgfqpoint{0.628823in}{2.392982in}}%
\pgfpathlineto{\pgfqpoint{0.628886in}{2.402632in}}%
\pgfpathlineto{\pgfqpoint{0.625000in}{2.410870in}}%
\pgfusepath{stroke}%
\end{pgfscope}%
\begin{pgfscope}%
\pgfpathrectangle{\pgfqpoint{0.625000in}{0.550000in}}{\pgfqpoint{3.875000in}{3.850000in}} %
\pgfusepath{clip}%
\pgfsetbuttcap%
\pgfsetroundjoin%
\pgfsetlinewidth{0.250937pt}%
\definecolor{currentstroke}{rgb}{0.000000,0.000000,0.000000}%
\pgfsetstrokecolor{currentstroke}%
\pgfsetdash{}{0pt}%
\pgfpathmoveto{\pgfqpoint{0.625000in}{2.414969in}}%
\pgfpathlineto{\pgfqpoint{0.626306in}{2.421930in}}%
\pgfpathlineto{\pgfqpoint{0.626193in}{2.431579in}}%
\pgfpathlineto{\pgfqpoint{0.625944in}{2.441228in}}%
\pgfpathlineto{\pgfqpoint{0.625634in}{2.450877in}}%
\pgfpathlineto{\pgfqpoint{0.625175in}{2.460526in}}%
\pgfpathlineto{\pgfqpoint{0.625000in}{2.462841in}}%
\pgfusepath{stroke}%
\end{pgfscope}%
\begin{pgfscope}%
\pgfpathrectangle{\pgfqpoint{0.625000in}{0.550000in}}{\pgfqpoint{3.875000in}{3.850000in}} %
\pgfusepath{clip}%
\pgfsetbuttcap%
\pgfsetroundjoin%
\pgfsetlinewidth{0.250937pt}%
\definecolor{currentstroke}{rgb}{0.000000,0.000000,0.000000}%
\pgfsetstrokecolor{currentstroke}%
\pgfsetdash{}{0pt}%
\pgfpathmoveto{\pgfqpoint{0.625000in}{2.496808in}}%
\pgfpathlineto{\pgfqpoint{0.625174in}{2.499123in}}%
\pgfpathlineto{\pgfqpoint{0.625634in}{2.508772in}}%
\pgfpathlineto{\pgfqpoint{0.625944in}{2.518421in}}%
\pgfpathlineto{\pgfqpoint{0.626193in}{2.528070in}}%
\pgfpathlineto{\pgfqpoint{0.626306in}{2.537719in}}%
\pgfpathlineto{\pgfqpoint{0.625000in}{2.544680in}}%
\pgfusepath{stroke}%
\end{pgfscope}%
\begin{pgfscope}%
\pgfpathrectangle{\pgfqpoint{0.625000in}{0.550000in}}{\pgfqpoint{3.875000in}{3.850000in}} %
\pgfusepath{clip}%
\pgfsetbuttcap%
\pgfsetroundjoin%
\pgfsetlinewidth{0.250937pt}%
\definecolor{currentstroke}{rgb}{0.000000,0.000000,0.000000}%
\pgfsetstrokecolor{currentstroke}%
\pgfsetdash{}{0pt}%
\pgfpathmoveto{\pgfqpoint{0.625000in}{2.548780in}}%
\pgfpathlineto{\pgfqpoint{0.628886in}{2.557018in}}%
\pgfpathlineto{\pgfqpoint{0.628823in}{2.566667in}}%
\pgfpathlineto{\pgfqpoint{0.630328in}{2.576316in}}%
\pgfpathlineto{\pgfqpoint{0.626555in}{2.585965in}}%
\pgfpathlineto{\pgfqpoint{0.630343in}{2.595614in}}%
\pgfpathlineto{\pgfqpoint{0.630398in}{2.605263in}}%
\pgfpathlineto{\pgfqpoint{0.629978in}{2.614912in}}%
\pgfpathlineto{\pgfqpoint{0.632113in}{2.624561in}}%
\pgfpathlineto{\pgfqpoint{0.625000in}{2.624978in}}%
\pgfusepath{stroke}%
\end{pgfscope}%
\begin{pgfscope}%
\pgfpathrectangle{\pgfqpoint{0.625000in}{0.550000in}}{\pgfqpoint{3.875000in}{3.850000in}} %
\pgfusepath{clip}%
\pgfsetbuttcap%
\pgfsetroundjoin%
\pgfsetlinewidth{0.250937pt}%
\definecolor{currentstroke}{rgb}{0.000000,0.000000,0.000000}%
\pgfsetstrokecolor{currentstroke}%
\pgfsetdash{}{0pt}%
\pgfpathmoveto{\pgfqpoint{0.625000in}{2.649089in}}%
\pgfpathlineto{\pgfqpoint{0.631160in}{2.653509in}}%
\pgfpathlineto{\pgfqpoint{0.630505in}{2.663158in}}%
\pgfpathlineto{\pgfqpoint{0.633445in}{2.672807in}}%
\pgfpathlineto{\pgfqpoint{0.629557in}{2.682456in}}%
\pgfpathlineto{\pgfqpoint{0.632020in}{2.692105in}}%
\pgfpathlineto{\pgfqpoint{0.631737in}{2.701754in}}%
\pgfpathlineto{\pgfqpoint{0.632772in}{2.711404in}}%
\pgfpathlineto{\pgfqpoint{0.633994in}{2.721053in}}%
\pgfpathlineto{\pgfqpoint{0.631136in}{2.730702in}}%
\pgfpathlineto{\pgfqpoint{0.631525in}{2.740351in}}%
\pgfpathlineto{\pgfqpoint{0.633367in}{2.750000in}}%
\pgfpathlineto{\pgfqpoint{0.633309in}{2.759649in}}%
\pgfpathlineto{\pgfqpoint{0.632582in}{2.769298in}}%
\pgfpathlineto{\pgfqpoint{0.631459in}{2.778947in}}%
\pgfpathlineto{\pgfqpoint{0.625000in}{2.779332in}}%
\pgfusepath{stroke}%
\end{pgfscope}%
\begin{pgfscope}%
\pgfpathrectangle{\pgfqpoint{0.625000in}{0.550000in}}{\pgfqpoint{3.875000in}{3.850000in}} %
\pgfusepath{clip}%
\pgfsetbuttcap%
\pgfsetroundjoin%
\pgfsetlinewidth{0.250937pt}%
\definecolor{currentstroke}{rgb}{0.000000,0.000000,0.000000}%
\pgfsetstrokecolor{currentstroke}%
\pgfsetdash{}{0pt}%
\pgfpathmoveto{\pgfqpoint{0.625000in}{2.798128in}}%
\pgfpathlineto{\pgfqpoint{0.634105in}{2.798246in}}%
\pgfpathlineto{\pgfqpoint{0.633369in}{2.807895in}}%
\pgfpathlineto{\pgfqpoint{0.632486in}{2.817544in}}%
\pgfpathlineto{\pgfqpoint{0.630188in}{2.827193in}}%
\pgfpathlineto{\pgfqpoint{0.630801in}{2.836842in}}%
\pgfpathlineto{\pgfqpoint{0.631332in}{2.846491in}}%
\pgfpathlineto{\pgfqpoint{0.633471in}{2.856140in}}%
\pgfpathlineto{\pgfqpoint{0.634712in}{2.860531in}}%
\pgfpathlineto{\pgfqpoint{0.637358in}{2.865789in}}%
\pgfpathlineto{\pgfqpoint{0.634712in}{2.871048in}}%
\pgfpathlineto{\pgfqpoint{0.633460in}{2.875439in}}%
\pgfpathlineto{\pgfqpoint{0.631267in}{2.885088in}}%
\pgfpathlineto{\pgfqpoint{0.630127in}{2.894737in}}%
\pgfpathlineto{\pgfqpoint{0.629219in}{2.904386in}}%
\pgfpathlineto{\pgfqpoint{0.629179in}{2.914035in}}%
\pgfpathlineto{\pgfqpoint{0.630722in}{2.923684in}}%
\pgfpathlineto{\pgfqpoint{0.625000in}{2.929602in}}%
\pgfusepath{stroke}%
\end{pgfscope}%
\begin{pgfscope}%
\pgfpathrectangle{\pgfqpoint{0.625000in}{0.550000in}}{\pgfqpoint{3.875000in}{3.850000in}} %
\pgfusepath{clip}%
\pgfsetbuttcap%
\pgfsetroundjoin%
\pgfsetlinewidth{0.250937pt}%
\definecolor{currentstroke}{rgb}{0.000000,0.000000,0.000000}%
\pgfsetstrokecolor{currentstroke}%
\pgfsetdash{}{0pt}%
\pgfpathmoveto{\pgfqpoint{0.625000in}{2.953143in}}%
\pgfpathlineto{\pgfqpoint{0.632646in}{2.952632in}}%
\pgfpathlineto{\pgfqpoint{0.634712in}{2.951250in}}%
\pgfpathlineto{\pgfqpoint{0.636099in}{2.952632in}}%
\pgfpathlineto{\pgfqpoint{0.640485in}{2.962281in}}%
\pgfpathlineto{\pgfqpoint{0.636099in}{2.971930in}}%
\pgfpathlineto{\pgfqpoint{0.634712in}{2.973312in}}%
\pgfpathlineto{\pgfqpoint{0.632975in}{2.981579in}}%
\pgfpathlineto{\pgfqpoint{0.631226in}{2.991228in}}%
\pgfpathlineto{\pgfqpoint{0.630972in}{3.000877in}}%
\pgfpathlineto{\pgfqpoint{0.632335in}{3.010526in}}%
\pgfpathlineto{\pgfqpoint{0.630437in}{3.020175in}}%
\pgfpathlineto{\pgfqpoint{0.625000in}{3.025017in}}%
\pgfusepath{stroke}%
\end{pgfscope}%
\begin{pgfscope}%
\pgfpathrectangle{\pgfqpoint{0.625000in}{0.550000in}}{\pgfqpoint{3.875000in}{3.850000in}} %
\pgfusepath{clip}%
\pgfsetbuttcap%
\pgfsetroundjoin%
\pgfsetlinewidth{0.250937pt}%
\definecolor{currentstroke}{rgb}{0.000000,0.000000,0.000000}%
\pgfsetstrokecolor{currentstroke}%
\pgfsetdash{}{0pt}%
\pgfpathmoveto{\pgfqpoint{0.625000in}{3.036494in}}%
\pgfpathlineto{\pgfqpoint{0.627797in}{3.039474in}}%
\pgfpathlineto{\pgfqpoint{0.625000in}{3.047870in}}%
\pgfusepath{stroke}%
\end{pgfscope}%
\begin{pgfscope}%
\pgfpathrectangle{\pgfqpoint{0.625000in}{0.550000in}}{\pgfqpoint{3.875000in}{3.850000in}} %
\pgfusepath{clip}%
\pgfsetbuttcap%
\pgfsetroundjoin%
\pgfsetlinewidth{0.250937pt}%
\definecolor{currentstroke}{rgb}{0.000000,0.000000,0.000000}%
\pgfsetstrokecolor{currentstroke}%
\pgfsetdash{}{0pt}%
\pgfpathmoveto{\pgfqpoint{0.625000in}{3.049568in}}%
\pgfpathlineto{\pgfqpoint{0.633469in}{3.058772in}}%
\pgfpathlineto{\pgfqpoint{0.625000in}{3.064098in}}%
\pgfusepath{stroke}%
\end{pgfscope}%
\begin{pgfscope}%
\pgfpathrectangle{\pgfqpoint{0.625000in}{0.550000in}}{\pgfqpoint{3.875000in}{3.850000in}} %
\pgfusepath{clip}%
\pgfsetbuttcap%
\pgfsetroundjoin%
\pgfsetlinewidth{0.250937pt}%
\definecolor{currentstroke}{rgb}{0.000000,0.000000,0.000000}%
\pgfsetstrokecolor{currentstroke}%
\pgfsetdash{}{0pt}%
\pgfpathmoveto{\pgfqpoint{0.625000in}{3.075315in}}%
\pgfpathlineto{\pgfqpoint{0.630078in}{3.078070in}}%
\pgfpathlineto{\pgfqpoint{0.631758in}{3.087719in}}%
\pgfpathlineto{\pgfqpoint{0.625000in}{3.087957in}}%
\pgfusepath{stroke}%
\end{pgfscope}%
\begin{pgfscope}%
\pgfpathrectangle{\pgfqpoint{0.625000in}{0.550000in}}{\pgfqpoint{3.875000in}{3.850000in}} %
\pgfusepath{clip}%
\pgfsetbuttcap%
\pgfsetroundjoin%
\pgfsetlinewidth{0.250937pt}%
\definecolor{currentstroke}{rgb}{0.000000,0.000000,0.000000}%
\pgfsetstrokecolor{currentstroke}%
\pgfsetdash{}{0pt}%
\pgfpathmoveto{\pgfqpoint{0.625000in}{3.106583in}}%
\pgfpathlineto{\pgfqpoint{0.634712in}{3.106128in}}%
\pgfpathlineto{\pgfqpoint{0.635879in}{3.107018in}}%
\pgfpathlineto{\pgfqpoint{0.643626in}{3.116667in}}%
\pgfpathlineto{\pgfqpoint{0.644318in}{3.126316in}}%
\pgfpathlineto{\pgfqpoint{0.639721in}{3.135965in}}%
\pgfpathlineto{\pgfqpoint{0.634712in}{3.140296in}}%
\pgfpathlineto{\pgfqpoint{0.632475in}{3.145614in}}%
\pgfpathlineto{\pgfqpoint{0.633077in}{3.155263in}}%
\pgfpathlineto{\pgfqpoint{0.632627in}{3.164912in}}%
\pgfpathlineto{\pgfqpoint{0.631689in}{3.174561in}}%
\pgfpathlineto{\pgfqpoint{0.630093in}{3.184211in}}%
\pgfpathlineto{\pgfqpoint{0.631107in}{3.193860in}}%
\pgfpathlineto{\pgfqpoint{0.634022in}{3.203509in}}%
\pgfpathlineto{\pgfqpoint{0.632285in}{3.213158in}}%
\pgfpathlineto{\pgfqpoint{0.631789in}{3.222807in}}%
\pgfpathlineto{\pgfqpoint{0.628769in}{3.232456in}}%
\pgfpathlineto{\pgfqpoint{0.627810in}{3.242105in}}%
\pgfpathlineto{\pgfqpoint{0.625000in}{3.242454in}}%
\pgfusepath{stroke}%
\end{pgfscope}%
\begin{pgfscope}%
\pgfpathrectangle{\pgfqpoint{0.625000in}{0.550000in}}{\pgfqpoint{3.875000in}{3.850000in}} %
\pgfusepath{clip}%
\pgfsetbuttcap%
\pgfsetroundjoin%
\pgfsetlinewidth{0.250937pt}%
\definecolor{currentstroke}{rgb}{0.000000,0.000000,0.000000}%
\pgfsetstrokecolor{currentstroke}%
\pgfsetdash{}{0pt}%
\pgfpathmoveto{\pgfqpoint{0.625000in}{3.261116in}}%
\pgfpathlineto{\pgfqpoint{0.627426in}{3.261404in}}%
\pgfpathlineto{\pgfqpoint{0.628769in}{3.271053in}}%
\pgfpathlineto{\pgfqpoint{0.629158in}{3.280702in}}%
\pgfpathlineto{\pgfqpoint{0.632027in}{3.290351in}}%
\pgfpathlineto{\pgfqpoint{0.633522in}{3.300000in}}%
\pgfpathlineto{\pgfqpoint{0.633307in}{3.309649in}}%
\pgfpathlineto{\pgfqpoint{0.625000in}{3.314770in}}%
\pgfusepath{stroke}%
\end{pgfscope}%
\begin{pgfscope}%
\pgfpathrectangle{\pgfqpoint{0.625000in}{0.550000in}}{\pgfqpoint{3.875000in}{3.850000in}} %
\pgfusepath{clip}%
\pgfsetbuttcap%
\pgfsetroundjoin%
\pgfsetlinewidth{0.250937pt}%
\definecolor{currentstroke}{rgb}{0.000000,0.000000,0.000000}%
\pgfsetstrokecolor{currentstroke}%
\pgfsetdash{}{0pt}%
\pgfpathmoveto{\pgfqpoint{0.625000in}{3.322017in}}%
\pgfpathlineto{\pgfqpoint{0.631107in}{3.328947in}}%
\pgfpathlineto{\pgfqpoint{0.625000in}{3.337371in}}%
\pgfusepath{stroke}%
\end{pgfscope}%
\begin{pgfscope}%
\pgfpathrectangle{\pgfqpoint{0.625000in}{0.550000in}}{\pgfqpoint{3.875000in}{3.850000in}} %
\pgfusepath{clip}%
\pgfsetbuttcap%
\pgfsetroundjoin%
\pgfsetlinewidth{0.250937pt}%
\definecolor{currentstroke}{rgb}{0.000000,0.000000,0.000000}%
\pgfsetstrokecolor{currentstroke}%
\pgfsetdash{}{0pt}%
\pgfpathmoveto{\pgfqpoint{0.625000in}{3.339733in}}%
\pgfpathlineto{\pgfqpoint{0.631825in}{3.348246in}}%
\pgfpathlineto{\pgfqpoint{0.632698in}{3.357895in}}%
\pgfpathlineto{\pgfqpoint{0.630972in}{3.367544in}}%
\pgfpathlineto{\pgfqpoint{0.625000in}{3.376539in}}%
\pgfusepath{stroke}%
\end{pgfscope}%
\begin{pgfscope}%
\pgfpathrectangle{\pgfqpoint{0.625000in}{0.550000in}}{\pgfqpoint{3.875000in}{3.850000in}} %
\pgfusepath{clip}%
\pgfsetbuttcap%
\pgfsetroundjoin%
\pgfsetlinewidth{0.250937pt}%
\definecolor{currentstroke}{rgb}{0.000000,0.000000,0.000000}%
\pgfsetstrokecolor{currentstroke}%
\pgfsetdash{}{0pt}%
\pgfpathmoveto{\pgfqpoint{0.625000in}{3.378214in}}%
\pgfpathlineto{\pgfqpoint{0.630885in}{3.386842in}}%
\pgfpathlineto{\pgfqpoint{0.632846in}{3.396491in}}%
\pgfpathlineto{\pgfqpoint{0.625000in}{3.396849in}}%
\pgfusepath{stroke}%
\end{pgfscope}%
\begin{pgfscope}%
\pgfpathrectangle{\pgfqpoint{0.625000in}{0.550000in}}{\pgfqpoint{3.875000in}{3.850000in}} %
\pgfusepath{clip}%
\pgfsetbuttcap%
\pgfsetroundjoin%
\pgfsetlinewidth{0.250937pt}%
\definecolor{currentstroke}{rgb}{0.000000,0.000000,0.000000}%
\pgfsetstrokecolor{currentstroke}%
\pgfsetdash{}{0pt}%
\pgfpathmoveto{\pgfqpoint{0.625000in}{3.415744in}}%
\pgfpathlineto{\pgfqpoint{0.633281in}{3.415789in}}%
\pgfpathlineto{\pgfqpoint{0.634712in}{3.417074in}}%
\pgfpathlineto{\pgfqpoint{0.644424in}{3.422837in}}%
\pgfpathlineto{\pgfqpoint{0.647042in}{3.425439in}}%
\pgfpathlineto{\pgfqpoint{0.652843in}{3.435088in}}%
\pgfpathlineto{\pgfqpoint{0.654135in}{3.443536in}}%
\pgfpathlineto{\pgfqpoint{0.654357in}{3.444737in}}%
\pgfpathlineto{\pgfqpoint{0.654135in}{3.445938in}}%
\pgfpathlineto{\pgfqpoint{0.652843in}{3.454386in}}%
\pgfpathlineto{\pgfqpoint{0.647042in}{3.464035in}}%
\pgfpathlineto{\pgfqpoint{0.644424in}{3.466637in}}%
\pgfpathlineto{\pgfqpoint{0.634712in}{3.472400in}}%
\pgfpathlineto{\pgfqpoint{0.634511in}{3.473684in}}%
\pgfpathlineto{\pgfqpoint{0.633632in}{3.483333in}}%
\pgfpathlineto{\pgfqpoint{0.632658in}{3.492982in}}%
\pgfpathlineto{\pgfqpoint{0.631010in}{3.502632in}}%
\pgfpathlineto{\pgfqpoint{0.631013in}{3.512281in}}%
\pgfpathlineto{\pgfqpoint{0.631717in}{3.521930in}}%
\pgfpathlineto{\pgfqpoint{0.625000in}{3.529540in}}%
\pgfusepath{stroke}%
\end{pgfscope}%
\begin{pgfscope}%
\pgfpathrectangle{\pgfqpoint{0.625000in}{0.550000in}}{\pgfqpoint{3.875000in}{3.850000in}} %
\pgfusepath{clip}%
\pgfsetbuttcap%
\pgfsetroundjoin%
\pgfsetlinewidth{0.250937pt}%
\definecolor{currentstroke}{rgb}{0.000000,0.000000,0.000000}%
\pgfsetstrokecolor{currentstroke}%
\pgfsetdash{}{0pt}%
\pgfpathmoveto{\pgfqpoint{0.625000in}{3.533034in}}%
\pgfpathlineto{\pgfqpoint{0.632235in}{3.541228in}}%
\pgfpathlineto{\pgfqpoint{0.632952in}{3.550877in}}%
\pgfpathlineto{\pgfqpoint{0.625000in}{3.551186in}}%
\pgfusepath{stroke}%
\end{pgfscope}%
\begin{pgfscope}%
\pgfpathrectangle{\pgfqpoint{0.625000in}{0.550000in}}{\pgfqpoint{3.875000in}{3.850000in}} %
\pgfusepath{clip}%
\pgfsetbuttcap%
\pgfsetroundjoin%
\pgfsetlinewidth{0.250937pt}%
\definecolor{currentstroke}{rgb}{0.000000,0.000000,0.000000}%
\pgfsetstrokecolor{currentstroke}%
\pgfsetdash{}{0pt}%
\pgfpathmoveto{\pgfqpoint{0.625000in}{3.569872in}}%
\pgfpathlineto{\pgfqpoint{0.630028in}{3.570175in}}%
\pgfpathlineto{\pgfqpoint{0.626840in}{3.579825in}}%
\pgfpathlineto{\pgfqpoint{0.628114in}{3.589474in}}%
\pgfpathlineto{\pgfqpoint{0.631085in}{3.599123in}}%
\pgfpathlineto{\pgfqpoint{0.631706in}{3.608772in}}%
\pgfpathlineto{\pgfqpoint{0.631160in}{3.618421in}}%
\pgfpathlineto{\pgfqpoint{0.634112in}{3.628070in}}%
\pgfpathlineto{\pgfqpoint{0.634712in}{3.632461in}}%
\pgfpathlineto{\pgfqpoint{0.637358in}{3.637719in}}%
\pgfpathlineto{\pgfqpoint{0.634712in}{3.642978in}}%
\pgfpathlineto{\pgfqpoint{0.634046in}{3.647368in}}%
\pgfpathlineto{\pgfqpoint{0.631267in}{3.657018in}}%
\pgfpathlineto{\pgfqpoint{0.625000in}{3.659938in}}%
\pgfusepath{stroke}%
\end{pgfscope}%
\begin{pgfscope}%
\pgfpathrectangle{\pgfqpoint{0.625000in}{0.550000in}}{\pgfqpoint{3.875000in}{3.850000in}} %
\pgfusepath{clip}%
\pgfsetbuttcap%
\pgfsetroundjoin%
\pgfsetlinewidth{0.250937pt}%
\definecolor{currentstroke}{rgb}{0.000000,0.000000,0.000000}%
\pgfsetstrokecolor{currentstroke}%
\pgfsetdash{}{0pt}%
\pgfpathmoveto{\pgfqpoint{0.625000in}{3.672322in}}%
\pgfpathlineto{\pgfqpoint{0.632113in}{3.676316in}}%
\pgfpathlineto{\pgfqpoint{0.634006in}{3.685965in}}%
\pgfpathlineto{\pgfqpoint{0.631283in}{3.695614in}}%
\pgfpathlineto{\pgfqpoint{0.625000in}{3.702632in}}%
\pgfusepath{stroke}%
\end{pgfscope}%
\begin{pgfscope}%
\pgfpathrectangle{\pgfqpoint{0.625000in}{0.550000in}}{\pgfqpoint{3.875000in}{3.850000in}} %
\pgfusepath{clip}%
\pgfsetbuttcap%
\pgfsetroundjoin%
\pgfsetlinewidth{0.250937pt}%
\definecolor{currentstroke}{rgb}{0.000000,0.000000,0.000000}%
\pgfsetstrokecolor{currentstroke}%
\pgfsetdash{}{0pt}%
\pgfpathmoveto{\pgfqpoint{0.625000in}{3.724350in}}%
\pgfpathlineto{\pgfqpoint{0.629993in}{3.724561in}}%
\pgfpathlineto{\pgfqpoint{0.630541in}{3.734211in}}%
\pgfpathlineto{\pgfqpoint{0.628724in}{3.743860in}}%
\pgfpathlineto{\pgfqpoint{0.626366in}{3.753509in}}%
\pgfpathlineto{\pgfqpoint{0.625000in}{3.758660in}}%
\pgfusepath{stroke}%
\end{pgfscope}%
\begin{pgfscope}%
\pgfpathrectangle{\pgfqpoint{0.625000in}{0.550000in}}{\pgfqpoint{3.875000in}{3.850000in}} %
\pgfusepath{clip}%
\pgfsetbuttcap%
\pgfsetroundjoin%
\pgfsetlinewidth{0.250937pt}%
\definecolor{currentstroke}{rgb}{0.000000,0.000000,0.000000}%
\pgfsetstrokecolor{currentstroke}%
\pgfsetdash{}{0pt}%
\pgfpathmoveto{\pgfqpoint{0.625000in}{3.768821in}}%
\pgfpathlineto{\pgfqpoint{0.625459in}{3.772807in}}%
\pgfpathlineto{\pgfqpoint{0.626217in}{3.782456in}}%
\pgfpathlineto{\pgfqpoint{0.629204in}{3.792105in}}%
\pgfpathlineto{\pgfqpoint{0.628182in}{3.801754in}}%
\pgfpathlineto{\pgfqpoint{0.630078in}{3.811404in}}%
\pgfpathlineto{\pgfqpoint{0.631990in}{3.821053in}}%
\pgfpathlineto{\pgfqpoint{0.633450in}{3.830702in}}%
\pgfpathlineto{\pgfqpoint{0.632244in}{3.840351in}}%
\pgfpathlineto{\pgfqpoint{0.632020in}{3.850000in}}%
\pgfpathlineto{\pgfqpoint{0.633817in}{3.859649in}}%
\pgfpathlineto{\pgfqpoint{0.625000in}{3.859920in}}%
\pgfusepath{stroke}%
\end{pgfscope}%
\begin{pgfscope}%
\pgfpathrectangle{\pgfqpoint{0.625000in}{0.550000in}}{\pgfqpoint{3.875000in}{3.850000in}} %
\pgfusepath{clip}%
\pgfsetbuttcap%
\pgfsetroundjoin%
\pgfsetlinewidth{0.250937pt}%
\definecolor{currentstroke}{rgb}{0.000000,0.000000,0.000000}%
\pgfsetstrokecolor{currentstroke}%
\pgfsetdash{}{0pt}%
\pgfpathmoveto{\pgfqpoint{0.625000in}{3.878638in}}%
\pgfpathlineto{\pgfqpoint{0.630340in}{3.878947in}}%
\pgfpathlineto{\pgfqpoint{0.629381in}{3.888596in}}%
\pgfpathlineto{\pgfqpoint{0.632061in}{3.898246in}}%
\pgfpathlineto{\pgfqpoint{0.632967in}{3.907895in}}%
\pgfpathlineto{\pgfqpoint{0.634712in}{3.916169in}}%
\pgfpathlineto{\pgfqpoint{0.636099in}{3.917544in}}%
\pgfpathlineto{\pgfqpoint{0.640485in}{3.927193in}}%
\pgfpathlineto{\pgfqpoint{0.636099in}{3.936842in}}%
\pgfpathlineto{\pgfqpoint{0.634712in}{3.938217in}}%
\pgfpathlineto{\pgfqpoint{0.633386in}{3.946491in}}%
\pgfpathlineto{\pgfqpoint{0.632328in}{3.956140in}}%
\pgfpathlineto{\pgfqpoint{0.629637in}{3.965789in}}%
\pgfpathlineto{\pgfqpoint{0.632940in}{3.975439in}}%
\pgfpathlineto{\pgfqpoint{0.632627in}{3.985088in}}%
\pgfpathlineto{\pgfqpoint{0.629760in}{3.994737in}}%
\pgfpathlineto{\pgfqpoint{0.628769in}{4.004386in}}%
\pgfpathlineto{\pgfqpoint{0.627688in}{4.014035in}}%
\pgfpathlineto{\pgfqpoint{0.625000in}{4.014358in}}%
\pgfusepath{stroke}%
\end{pgfscope}%
\begin{pgfscope}%
\pgfpathrectangle{\pgfqpoint{0.625000in}{0.550000in}}{\pgfqpoint{3.875000in}{3.850000in}} %
\pgfusepath{clip}%
\pgfsetbuttcap%
\pgfsetroundjoin%
\pgfsetlinewidth{0.250937pt}%
\definecolor{currentstroke}{rgb}{0.000000,0.000000,0.000000}%
\pgfsetstrokecolor{currentstroke}%
\pgfsetdash{}{0pt}%
\pgfpathmoveto{\pgfqpoint{0.625000in}{4.033102in}}%
\pgfpathlineto{\pgfqpoint{0.627073in}{4.033333in}}%
\pgfpathlineto{\pgfqpoint{0.628769in}{4.042982in}}%
\pgfpathlineto{\pgfqpoint{0.625000in}{4.046471in}}%
\pgfusepath{stroke}%
\end{pgfscope}%
\begin{pgfscope}%
\pgfpathrectangle{\pgfqpoint{0.625000in}{0.550000in}}{\pgfqpoint{3.875000in}{3.850000in}} %
\pgfusepath{clip}%
\pgfsetbuttcap%
\pgfsetroundjoin%
\pgfsetlinewidth{0.250937pt}%
\definecolor{currentstroke}{rgb}{0.000000,0.000000,0.000000}%
\pgfsetstrokecolor{currentstroke}%
\pgfsetdash{}{0pt}%
\pgfpathmoveto{\pgfqpoint{0.625000in}{4.057653in}}%
\pgfpathlineto{\pgfqpoint{0.632218in}{4.062281in}}%
\pgfpathlineto{\pgfqpoint{0.632380in}{4.071930in}}%
\pgfpathlineto{\pgfqpoint{0.633369in}{4.081579in}}%
\pgfpathlineto{\pgfqpoint{0.634411in}{4.091228in}}%
\pgfpathlineto{\pgfqpoint{0.633077in}{4.100877in}}%
\pgfpathlineto{\pgfqpoint{0.632006in}{4.110526in}}%
\pgfpathlineto{\pgfqpoint{0.631368in}{4.120175in}}%
\pgfpathlineto{\pgfqpoint{0.628608in}{4.129825in}}%
\pgfpathlineto{\pgfqpoint{0.628156in}{4.139474in}}%
\pgfpathlineto{\pgfqpoint{0.627349in}{4.149123in}}%
\pgfpathlineto{\pgfqpoint{0.631253in}{4.158772in}}%
\pgfpathlineto{\pgfqpoint{0.634032in}{4.168421in}}%
\pgfpathlineto{\pgfqpoint{0.625000in}{4.169052in}}%
\pgfusepath{stroke}%
\end{pgfscope}%
\begin{pgfscope}%
\pgfpathrectangle{\pgfqpoint{0.625000in}{0.550000in}}{\pgfqpoint{3.875000in}{3.850000in}} %
\pgfusepath{clip}%
\pgfsetbuttcap%
\pgfsetroundjoin%
\pgfsetlinewidth{0.250937pt}%
\definecolor{currentstroke}{rgb}{0.000000,0.000000,0.000000}%
\pgfsetstrokecolor{currentstroke}%
\pgfsetdash{}{0pt}%
\pgfpathmoveto{\pgfqpoint{0.625000in}{4.187531in}}%
\pgfpathlineto{\pgfqpoint{0.628777in}{4.187719in}}%
\pgfpathlineto{\pgfqpoint{0.627797in}{4.197368in}}%
\pgfpathlineto{\pgfqpoint{0.632028in}{4.207018in}}%
\pgfpathlineto{\pgfqpoint{0.633475in}{4.216667in}}%
\pgfpathlineto{\pgfqpoint{0.631995in}{4.226316in}}%
\pgfpathlineto{\pgfqpoint{0.628769in}{4.235965in}}%
\pgfpathlineto{\pgfqpoint{0.631871in}{4.245614in}}%
\pgfpathlineto{\pgfqpoint{0.632142in}{4.255263in}}%
\pgfpathlineto{\pgfqpoint{0.630768in}{4.264912in}}%
\pgfpathlineto{\pgfqpoint{0.630894in}{4.274561in}}%
\pgfpathlineto{\pgfqpoint{0.628755in}{4.284211in}}%
\pgfpathlineto{\pgfqpoint{0.631997in}{4.293860in}}%
\pgfpathlineto{\pgfqpoint{0.632238in}{4.303509in}}%
\pgfpathlineto{\pgfqpoint{0.633194in}{4.313158in}}%
\pgfpathlineto{\pgfqpoint{0.632330in}{4.322807in}}%
\pgfpathlineto{\pgfqpoint{0.625000in}{4.323025in}}%
\pgfusepath{stroke}%
\end{pgfscope}%
\begin{pgfscope}%
\pgfpathrectangle{\pgfqpoint{0.625000in}{0.550000in}}{\pgfqpoint{3.875000in}{3.850000in}} %
\pgfusepath{clip}%
\pgfsetbuttcap%
\pgfsetroundjoin%
\pgfsetlinewidth{0.250937pt}%
\definecolor{currentstroke}{rgb}{0.000000,0.000000,0.000000}%
\pgfsetstrokecolor{currentstroke}%
\pgfsetdash{}{0pt}%
\pgfpathmoveto{\pgfqpoint{0.625000in}{4.341856in}}%
\pgfpathlineto{\pgfqpoint{0.633745in}{4.342105in}}%
\pgfpathlineto{\pgfqpoint{0.634649in}{4.351754in}}%
\pgfpathlineto{\pgfqpoint{0.634712in}{4.352145in}}%
\pgfpathlineto{\pgfqpoint{0.644424in}{4.354522in}}%
\pgfpathlineto{\pgfqpoint{0.654135in}{4.358984in}}%
\pgfpathlineto{\pgfqpoint{0.657781in}{4.361404in}}%
\pgfpathlineto{\pgfqpoint{0.663847in}{4.365956in}}%
\pgfpathlineto{\pgfqpoint{0.668977in}{4.371053in}}%
\pgfpathlineto{\pgfqpoint{0.673559in}{4.377080in}}%
\pgfpathlineto{\pgfqpoint{0.675994in}{4.380702in}}%
\pgfpathlineto{\pgfqpoint{0.680485in}{4.390351in}}%
\pgfpathlineto{\pgfqpoint{0.682878in}{4.400000in}}%
\pgfusepath{stroke}%
\end{pgfscope}%
\begin{pgfscope}%
\pgfpathrectangle{\pgfqpoint{0.625000in}{0.550000in}}{\pgfqpoint{3.875000in}{3.850000in}} %
\pgfusepath{clip}%
\pgfsetbuttcap%
\pgfsetroundjoin%
\pgfsetlinewidth{0.250937pt}%
\definecolor{currentstroke}{rgb}{0.000000,0.000000,0.000000}%
\pgfsetstrokecolor{currentstroke}%
\pgfsetdash{}{0pt}%
\pgfpathmoveto{\pgfqpoint{0.660877in}{0.550000in}}%
\pgfpathlineto{\pgfqpoint{0.659517in}{0.559649in}}%
\pgfpathlineto{\pgfqpoint{0.654850in}{0.569298in}}%
\pgfpathlineto{\pgfqpoint{0.654135in}{0.570238in}}%
\pgfpathlineto{\pgfqpoint{0.645369in}{0.578947in}}%
\pgfpathlineto{\pgfqpoint{0.644424in}{0.579657in}}%
\pgfpathlineto{\pgfqpoint{0.634712in}{0.584294in}}%
\pgfpathlineto{\pgfqpoint{0.633901in}{0.588596in}}%
\pgfpathlineto{\pgfqpoint{0.633306in}{0.598246in}}%
\pgfpathlineto{\pgfqpoint{0.630284in}{0.607895in}}%
\pgfpathlineto{\pgfqpoint{0.630680in}{0.617544in}}%
\pgfpathlineto{\pgfqpoint{0.625000in}{0.617684in}}%
\pgfusepath{stroke}%
\end{pgfscope}%
\begin{pgfscope}%
\pgfpathrectangle{\pgfqpoint{0.625000in}{0.550000in}}{\pgfqpoint{3.875000in}{3.850000in}} %
\pgfusepath{clip}%
\pgfsetbuttcap%
\pgfsetroundjoin%
\pgfsetlinewidth{0.250937pt}%
\definecolor{currentstroke}{rgb}{0.000000,0.000000,0.000000}%
\pgfsetstrokecolor{currentstroke}%
\pgfsetdash{}{0pt}%
\pgfpathmoveto{\pgfqpoint{0.625000in}{0.636729in}}%
\pgfpathlineto{\pgfqpoint{0.629376in}{0.636842in}}%
\pgfpathlineto{\pgfqpoint{0.631685in}{0.646491in}}%
\pgfpathlineto{\pgfqpoint{0.630108in}{0.656140in}}%
\pgfpathlineto{\pgfqpoint{0.630051in}{0.665789in}}%
\pgfpathlineto{\pgfqpoint{0.625162in}{0.675439in}}%
\pgfpathlineto{\pgfqpoint{0.628228in}{0.685088in}}%
\pgfpathlineto{\pgfqpoint{0.628926in}{0.694737in}}%
\pgfpathlineto{\pgfqpoint{0.630709in}{0.704386in}}%
\pgfpathlineto{\pgfqpoint{0.629945in}{0.714035in}}%
\pgfpathlineto{\pgfqpoint{0.627064in}{0.723684in}}%
\pgfpathlineto{\pgfqpoint{0.630665in}{0.733333in}}%
\pgfpathlineto{\pgfqpoint{0.632232in}{0.742982in}}%
\pgfpathlineto{\pgfqpoint{0.630715in}{0.752632in}}%
\pgfpathlineto{\pgfqpoint{0.626531in}{0.762281in}}%
\pgfpathlineto{\pgfqpoint{0.626996in}{0.771930in}}%
\pgfpathlineto{\pgfqpoint{0.625000in}{0.772020in}}%
\pgfusepath{stroke}%
\end{pgfscope}%
\begin{pgfscope}%
\pgfpathrectangle{\pgfqpoint{0.625000in}{0.550000in}}{\pgfqpoint{3.875000in}{3.850000in}} %
\pgfusepath{clip}%
\pgfsetbuttcap%
\pgfsetroundjoin%
\pgfsetlinewidth{0.250937pt}%
\definecolor{currentstroke}{rgb}{0.000000,0.000000,0.000000}%
\pgfsetstrokecolor{currentstroke}%
\pgfsetdash{}{0pt}%
\pgfpathmoveto{\pgfqpoint{0.625000in}{0.790731in}}%
\pgfpathlineto{\pgfqpoint{0.632822in}{0.791228in}}%
\pgfpathlineto{\pgfqpoint{0.628424in}{0.800877in}}%
\pgfpathlineto{\pgfqpoint{0.625000in}{0.809076in}}%
\pgfusepath{stroke}%
\end{pgfscope}%
\begin{pgfscope}%
\pgfpathrectangle{\pgfqpoint{0.625000in}{0.550000in}}{\pgfqpoint{3.875000in}{3.850000in}} %
\pgfusepath{clip}%
\pgfsetbuttcap%
\pgfsetroundjoin%
\pgfsetlinewidth{0.250937pt}%
\definecolor{currentstroke}{rgb}{0.000000,0.000000,0.000000}%
\pgfsetstrokecolor{currentstroke}%
\pgfsetdash{}{0pt}%
\pgfpathmoveto{\pgfqpoint{0.625000in}{0.811261in}}%
\pgfpathlineto{\pgfqpoint{0.627278in}{0.820175in}}%
\pgfpathlineto{\pgfqpoint{0.627891in}{0.829825in}}%
\pgfpathlineto{\pgfqpoint{0.630195in}{0.839474in}}%
\pgfpathlineto{\pgfqpoint{0.630071in}{0.849123in}}%
\pgfpathlineto{\pgfqpoint{0.631456in}{0.858772in}}%
\pgfpathlineto{\pgfqpoint{0.631147in}{0.868421in}}%
\pgfpathlineto{\pgfqpoint{0.629583in}{0.878070in}}%
\pgfpathlineto{\pgfqpoint{0.630644in}{0.887719in}}%
\pgfpathlineto{\pgfqpoint{0.630211in}{0.897368in}}%
\pgfpathlineto{\pgfqpoint{0.625000in}{0.900709in}}%
\pgfusepath{stroke}%
\end{pgfscope}%
\begin{pgfscope}%
\pgfpathrectangle{\pgfqpoint{0.625000in}{0.550000in}}{\pgfqpoint{3.875000in}{3.850000in}} %
\pgfusepath{clip}%
\pgfsetbuttcap%
\pgfsetroundjoin%
\pgfsetlinewidth{0.250937pt}%
\definecolor{currentstroke}{rgb}{0.000000,0.000000,0.000000}%
\pgfsetstrokecolor{currentstroke}%
\pgfsetdash{}{0pt}%
\pgfpathmoveto{\pgfqpoint{0.625000in}{0.914756in}}%
\pgfpathlineto{\pgfqpoint{0.627064in}{0.916667in}}%
\pgfpathlineto{\pgfqpoint{0.626317in}{0.926316in}}%
\pgfpathlineto{\pgfqpoint{0.625000in}{0.926450in}}%
\pgfusepath{stroke}%
\end{pgfscope}%
\begin{pgfscope}%
\pgfpathrectangle{\pgfqpoint{0.625000in}{0.550000in}}{\pgfqpoint{3.875000in}{3.850000in}} %
\pgfusepath{clip}%
\pgfsetbuttcap%
\pgfsetroundjoin%
\pgfsetlinewidth{0.250937pt}%
\definecolor{currentstroke}{rgb}{0.000000,0.000000,0.000000}%
\pgfsetstrokecolor{currentstroke}%
\pgfsetdash{}{0pt}%
\pgfpathmoveto{\pgfqpoint{0.625000in}{0.945395in}}%
\pgfpathlineto{\pgfqpoint{0.626993in}{0.945614in}}%
\pgfpathlineto{\pgfqpoint{0.627064in}{0.955263in}}%
\pgfpathlineto{\pgfqpoint{0.626824in}{0.964912in}}%
\pgfpathlineto{\pgfqpoint{0.630506in}{0.974561in}}%
\pgfpathlineto{\pgfqpoint{0.631289in}{0.984211in}}%
\pgfpathlineto{\pgfqpoint{0.627539in}{0.993860in}}%
\pgfpathlineto{\pgfqpoint{0.630713in}{1.003509in}}%
\pgfpathlineto{\pgfqpoint{0.631703in}{1.013158in}}%
\pgfpathlineto{\pgfqpoint{0.632245in}{1.022807in}}%
\pgfpathlineto{\pgfqpoint{0.634493in}{1.032456in}}%
\pgfpathlineto{\pgfqpoint{0.625000in}{1.041398in}}%
\pgfusepath{stroke}%
\end{pgfscope}%
\begin{pgfscope}%
\pgfpathrectangle{\pgfqpoint{0.625000in}{0.550000in}}{\pgfqpoint{3.875000in}{3.850000in}} %
\pgfusepath{clip}%
\pgfsetbuttcap%
\pgfsetroundjoin%
\pgfsetlinewidth{0.250937pt}%
\definecolor{currentstroke}{rgb}{0.000000,0.000000,0.000000}%
\pgfsetstrokecolor{currentstroke}%
\pgfsetdash{}{0pt}%
\pgfpathmoveto{\pgfqpoint{0.625000in}{1.043929in}}%
\pgfpathlineto{\pgfqpoint{0.630752in}{1.051754in}}%
\pgfpathlineto{\pgfqpoint{0.630060in}{1.061404in}}%
\pgfpathlineto{\pgfqpoint{0.627399in}{1.071053in}}%
\pgfpathlineto{\pgfqpoint{0.628841in}{1.080702in}}%
\pgfpathlineto{\pgfqpoint{0.625000in}{1.080904in}}%
\pgfusepath{stroke}%
\end{pgfscope}%
\begin{pgfscope}%
\pgfpathrectangle{\pgfqpoint{0.625000in}{0.550000in}}{\pgfqpoint{3.875000in}{3.850000in}} %
\pgfusepath{clip}%
\pgfsetbuttcap%
\pgfsetroundjoin%
\pgfsetlinewidth{0.250937pt}%
\definecolor{currentstroke}{rgb}{0.000000,0.000000,0.000000}%
\pgfsetstrokecolor{currentstroke}%
\pgfsetdash{}{0pt}%
\pgfpathmoveto{\pgfqpoint{0.625000in}{1.099833in}}%
\pgfpathlineto{\pgfqpoint{0.630981in}{1.100000in}}%
\pgfpathlineto{\pgfqpoint{0.628844in}{1.109649in}}%
\pgfpathlineto{\pgfqpoint{0.630594in}{1.119298in}}%
\pgfpathlineto{\pgfqpoint{0.632211in}{1.128947in}}%
\pgfpathlineto{\pgfqpoint{0.630169in}{1.138596in}}%
\pgfpathlineto{\pgfqpoint{0.627781in}{1.148246in}}%
\pgfpathlineto{\pgfqpoint{0.626222in}{1.157895in}}%
\pgfpathlineto{\pgfqpoint{0.628360in}{1.167544in}}%
\pgfpathlineto{\pgfqpoint{0.625674in}{1.177193in}}%
\pgfpathlineto{\pgfqpoint{0.625251in}{1.186842in}}%
\pgfpathlineto{\pgfqpoint{0.625000in}{1.189025in}}%
\pgfusepath{stroke}%
\end{pgfscope}%
\begin{pgfscope}%
\pgfpathrectangle{\pgfqpoint{0.625000in}{0.550000in}}{\pgfqpoint{3.875000in}{3.850000in}} %
\pgfusepath{clip}%
\pgfsetbuttcap%
\pgfsetroundjoin%
\pgfsetlinewidth{0.250937pt}%
\definecolor{currentstroke}{rgb}{0.000000,0.000000,0.000000}%
\pgfsetstrokecolor{currentstroke}%
\pgfsetdash{}{0pt}%
\pgfpathmoveto{\pgfqpoint{0.625000in}{1.202421in}}%
\pgfpathlineto{\pgfqpoint{0.625986in}{1.206140in}}%
\pgfpathlineto{\pgfqpoint{0.628022in}{1.215789in}}%
\pgfpathlineto{\pgfqpoint{0.629520in}{1.225439in}}%
\pgfpathlineto{\pgfqpoint{0.627978in}{1.235088in}}%
\pgfpathlineto{\pgfqpoint{0.625000in}{1.235207in}}%
\pgfusepath{stroke}%
\end{pgfscope}%
\begin{pgfscope}%
\pgfpathrectangle{\pgfqpoint{0.625000in}{0.550000in}}{\pgfqpoint{3.875000in}{3.850000in}} %
\pgfusepath{clip}%
\pgfsetbuttcap%
\pgfsetroundjoin%
\pgfsetlinewidth{0.250937pt}%
\definecolor{currentstroke}{rgb}{0.000000,0.000000,0.000000}%
\pgfsetstrokecolor{currentstroke}%
\pgfsetdash{}{0pt}%
\pgfpathmoveto{\pgfqpoint{0.625000in}{1.260192in}}%
\pgfpathlineto{\pgfqpoint{0.628440in}{1.264035in}}%
\pgfpathlineto{\pgfqpoint{0.632752in}{1.273684in}}%
\pgfpathlineto{\pgfqpoint{0.630136in}{1.283333in}}%
\pgfpathlineto{\pgfqpoint{0.625000in}{1.286217in}}%
\pgfusepath{stroke}%
\end{pgfscope}%
\begin{pgfscope}%
\pgfpathrectangle{\pgfqpoint{0.625000in}{0.550000in}}{\pgfqpoint{3.875000in}{3.850000in}} %
\pgfusepath{clip}%
\pgfsetbuttcap%
\pgfsetroundjoin%
\pgfsetlinewidth{0.250937pt}%
\definecolor{currentstroke}{rgb}{0.000000,0.000000,0.000000}%
\pgfsetstrokecolor{currentstroke}%
\pgfsetdash{}{0pt}%
\pgfpathmoveto{\pgfqpoint{0.625000in}{1.301032in}}%
\pgfpathlineto{\pgfqpoint{0.628431in}{1.302632in}}%
\pgfpathlineto{\pgfqpoint{0.631937in}{1.312281in}}%
\pgfpathlineto{\pgfqpoint{0.633939in}{1.321930in}}%
\pgfpathlineto{\pgfqpoint{0.632209in}{1.331579in}}%
\pgfpathlineto{\pgfqpoint{0.628373in}{1.341228in}}%
\pgfpathlineto{\pgfqpoint{0.629435in}{1.350877in}}%
\pgfpathlineto{\pgfqpoint{0.629393in}{1.360526in}}%
\pgfpathlineto{\pgfqpoint{0.627087in}{1.370175in}}%
\pgfpathlineto{\pgfqpoint{0.626008in}{1.379825in}}%
\pgfpathlineto{\pgfqpoint{0.628622in}{1.389474in}}%
\pgfpathlineto{\pgfqpoint{0.625000in}{1.389684in}}%
\pgfusepath{stroke}%
\end{pgfscope}%
\begin{pgfscope}%
\pgfpathrectangle{\pgfqpoint{0.625000in}{0.550000in}}{\pgfqpoint{3.875000in}{3.850000in}} %
\pgfusepath{clip}%
\pgfsetbuttcap%
\pgfsetroundjoin%
\pgfsetlinewidth{0.250937pt}%
\definecolor{currentstroke}{rgb}{0.000000,0.000000,0.000000}%
\pgfsetstrokecolor{currentstroke}%
\pgfsetdash{}{0pt}%
\pgfpathmoveto{\pgfqpoint{0.625000in}{1.408557in}}%
\pgfpathlineto{\pgfqpoint{0.630769in}{1.408772in}}%
\pgfpathlineto{\pgfqpoint{0.630903in}{1.418421in}}%
\pgfpathlineto{\pgfqpoint{0.625000in}{1.425106in}}%
\pgfusepath{stroke}%
\end{pgfscope}%
\begin{pgfscope}%
\pgfpathrectangle{\pgfqpoint{0.625000in}{0.550000in}}{\pgfqpoint{3.875000in}{3.850000in}} %
\pgfusepath{clip}%
\pgfsetbuttcap%
\pgfsetroundjoin%
\pgfsetlinewidth{0.250937pt}%
\definecolor{currentstroke}{rgb}{0.000000,0.000000,0.000000}%
\pgfsetstrokecolor{currentstroke}%
\pgfsetdash{}{0pt}%
\pgfpathmoveto{\pgfqpoint{0.625000in}{1.432225in}}%
\pgfpathlineto{\pgfqpoint{0.629850in}{1.437719in}}%
\pgfpathlineto{\pgfqpoint{0.628820in}{1.447368in}}%
\pgfpathlineto{\pgfqpoint{0.628291in}{1.457018in}}%
\pgfpathlineto{\pgfqpoint{0.630730in}{1.466667in}}%
\pgfpathlineto{\pgfqpoint{0.631900in}{1.476316in}}%
\pgfpathlineto{\pgfqpoint{0.632330in}{1.485965in}}%
\pgfpathlineto{\pgfqpoint{0.632304in}{1.495614in}}%
\pgfpathlineto{\pgfqpoint{0.634712in}{1.499817in}}%
\pgfpathlineto{\pgfqpoint{0.640194in}{1.505263in}}%
\pgfpathlineto{\pgfqpoint{0.643191in}{1.514912in}}%
\pgfpathlineto{\pgfqpoint{0.640194in}{1.524561in}}%
\pgfpathlineto{\pgfqpoint{0.634712in}{1.530008in}}%
\pgfpathlineto{\pgfqpoint{0.632304in}{1.534211in}}%
\pgfpathlineto{\pgfqpoint{0.625000in}{1.541177in}}%
\pgfusepath{stroke}%
\end{pgfscope}%
\begin{pgfscope}%
\pgfpathrectangle{\pgfqpoint{0.625000in}{0.550000in}}{\pgfqpoint{3.875000in}{3.850000in}} %
\pgfusepath{clip}%
\pgfsetbuttcap%
\pgfsetroundjoin%
\pgfsetlinewidth{0.250937pt}%
\definecolor{currentstroke}{rgb}{0.000000,0.000000,0.000000}%
\pgfsetstrokecolor{currentstroke}%
\pgfsetdash{}{0pt}%
\pgfpathmoveto{\pgfqpoint{0.625000in}{1.562892in}}%
\pgfpathlineto{\pgfqpoint{0.631035in}{1.563158in}}%
\pgfpathlineto{\pgfqpoint{0.628222in}{1.572807in}}%
\pgfpathlineto{\pgfqpoint{0.625000in}{1.577532in}}%
\pgfusepath{stroke}%
\end{pgfscope}%
\begin{pgfscope}%
\pgfpathrectangle{\pgfqpoint{0.625000in}{0.550000in}}{\pgfqpoint{3.875000in}{3.850000in}} %
\pgfusepath{clip}%
\pgfsetbuttcap%
\pgfsetroundjoin%
\pgfsetlinewidth{0.250937pt}%
\definecolor{currentstroke}{rgb}{0.000000,0.000000,0.000000}%
\pgfsetstrokecolor{currentstroke}%
\pgfsetdash{}{0pt}%
\pgfpathmoveto{\pgfqpoint{0.625000in}{1.585611in}}%
\pgfpathlineto{\pgfqpoint{0.629312in}{1.592105in}}%
\pgfpathlineto{\pgfqpoint{0.630964in}{1.601754in}}%
\pgfpathlineto{\pgfqpoint{0.630568in}{1.611404in}}%
\pgfpathlineto{\pgfqpoint{0.625000in}{1.618348in}}%
\pgfusepath{stroke}%
\end{pgfscope}%
\begin{pgfscope}%
\pgfpathrectangle{\pgfqpoint{0.625000in}{0.550000in}}{\pgfqpoint{3.875000in}{3.850000in}} %
\pgfusepath{clip}%
\pgfsetbuttcap%
\pgfsetroundjoin%
\pgfsetlinewidth{0.250937pt}%
\definecolor{currentstroke}{rgb}{0.000000,0.000000,0.000000}%
\pgfsetstrokecolor{currentstroke}%
\pgfsetdash{}{0pt}%
\pgfpathmoveto{\pgfqpoint{0.625000in}{1.623969in}}%
\pgfpathlineto{\pgfqpoint{0.629881in}{1.630702in}}%
\pgfpathlineto{\pgfqpoint{0.625000in}{1.636242in}}%
\pgfusepath{stroke}%
\end{pgfscope}%
\begin{pgfscope}%
\pgfpathrectangle{\pgfqpoint{0.625000in}{0.550000in}}{\pgfqpoint{3.875000in}{3.850000in}} %
\pgfusepath{clip}%
\pgfsetbuttcap%
\pgfsetroundjoin%
\pgfsetlinewidth{0.250937pt}%
\definecolor{currentstroke}{rgb}{0.000000,0.000000,0.000000}%
\pgfsetstrokecolor{currentstroke}%
\pgfsetdash{}{0pt}%
\pgfpathmoveto{\pgfqpoint{0.625000in}{1.647196in}}%
\pgfpathlineto{\pgfqpoint{0.629549in}{1.650000in}}%
\pgfpathlineto{\pgfqpoint{0.631836in}{1.659649in}}%
\pgfpathlineto{\pgfqpoint{0.630073in}{1.669298in}}%
\pgfpathlineto{\pgfqpoint{0.625770in}{1.678947in}}%
\pgfpathlineto{\pgfqpoint{0.627064in}{1.688596in}}%
\pgfpathlineto{\pgfqpoint{0.626706in}{1.698246in}}%
\pgfpathlineto{\pgfqpoint{0.625000in}{1.698438in}}%
\pgfusepath{stroke}%
\end{pgfscope}%
\begin{pgfscope}%
\pgfpathrectangle{\pgfqpoint{0.625000in}{0.550000in}}{\pgfqpoint{3.875000in}{3.850000in}} %
\pgfusepath{clip}%
\pgfsetbuttcap%
\pgfsetroundjoin%
\pgfsetlinewidth{0.250937pt}%
\definecolor{currentstroke}{rgb}{0.000000,0.000000,0.000000}%
\pgfsetstrokecolor{currentstroke}%
\pgfsetdash{}{0pt}%
\pgfpathmoveto{\pgfqpoint{0.625000in}{1.717292in}}%
\pgfpathlineto{\pgfqpoint{0.627127in}{1.717544in}}%
\pgfpathlineto{\pgfqpoint{0.627064in}{1.727193in}}%
\pgfpathlineto{\pgfqpoint{0.630026in}{1.736842in}}%
\pgfpathlineto{\pgfqpoint{0.630260in}{1.746491in}}%
\pgfpathlineto{\pgfqpoint{0.632795in}{1.756140in}}%
\pgfpathlineto{\pgfqpoint{0.628344in}{1.765789in}}%
\pgfpathlineto{\pgfqpoint{0.628706in}{1.775439in}}%
\pgfpathlineto{\pgfqpoint{0.630347in}{1.785088in}}%
\pgfpathlineto{\pgfqpoint{0.631136in}{1.794737in}}%
\pgfpathlineto{\pgfqpoint{0.631590in}{1.804386in}}%
\pgfpathlineto{\pgfqpoint{0.627655in}{1.814035in}}%
\pgfpathlineto{\pgfqpoint{0.632834in}{1.823684in}}%
\pgfpathlineto{\pgfqpoint{0.634712in}{1.830237in}}%
\pgfpathlineto{\pgfqpoint{0.636895in}{1.833333in}}%
\pgfpathlineto{\pgfqpoint{0.634986in}{1.842982in}}%
\pgfpathlineto{\pgfqpoint{0.634712in}{1.843300in}}%
\pgfpathlineto{\pgfqpoint{0.632979in}{1.852632in}}%
\pgfpathlineto{\pgfqpoint{0.625000in}{1.852979in}}%
\pgfusepath{stroke}%
\end{pgfscope}%
\begin{pgfscope}%
\pgfpathrectangle{\pgfqpoint{0.625000in}{0.550000in}}{\pgfqpoint{3.875000in}{3.850000in}} %
\pgfusepath{clip}%
\pgfsetbuttcap%
\pgfsetroundjoin%
\pgfsetlinewidth{0.250937pt}%
\definecolor{currentstroke}{rgb}{0.000000,0.000000,0.000000}%
\pgfsetstrokecolor{currentstroke}%
\pgfsetdash{}{0pt}%
\pgfpathmoveto{\pgfqpoint{0.625000in}{1.871779in}}%
\pgfpathlineto{\pgfqpoint{0.629365in}{1.871930in}}%
\pgfpathlineto{\pgfqpoint{0.627781in}{1.881579in}}%
\pgfpathlineto{\pgfqpoint{0.625000in}{1.883087in}}%
\pgfusepath{stroke}%
\end{pgfscope}%
\begin{pgfscope}%
\pgfpathrectangle{\pgfqpoint{0.625000in}{0.550000in}}{\pgfqpoint{3.875000in}{3.850000in}} %
\pgfusepath{clip}%
\pgfsetbuttcap%
\pgfsetroundjoin%
\pgfsetlinewidth{0.250937pt}%
\definecolor{currentstroke}{rgb}{0.000000,0.000000,0.000000}%
\pgfsetstrokecolor{currentstroke}%
\pgfsetdash{}{0pt}%
\pgfpathmoveto{\pgfqpoint{0.625000in}{1.896333in}}%
\pgfpathlineto{\pgfqpoint{0.632227in}{1.900877in}}%
\pgfpathlineto{\pgfqpoint{0.625000in}{1.908731in}}%
\pgfusepath{stroke}%
\end{pgfscope}%
\begin{pgfscope}%
\pgfpathrectangle{\pgfqpoint{0.625000in}{0.550000in}}{\pgfqpoint{3.875000in}{3.850000in}} %
\pgfusepath{clip}%
\pgfsetbuttcap%
\pgfsetroundjoin%
\pgfsetlinewidth{0.250937pt}%
\definecolor{currentstroke}{rgb}{0.000000,0.000000,0.000000}%
\pgfsetstrokecolor{currentstroke}%
\pgfsetdash{}{0pt}%
\pgfpathmoveto{\pgfqpoint{0.625000in}{1.915578in}}%
\pgfpathlineto{\pgfqpoint{0.626531in}{1.920175in}}%
\pgfpathlineto{\pgfqpoint{0.625000in}{1.921807in}}%
\pgfusepath{stroke}%
\end{pgfscope}%
\begin{pgfscope}%
\pgfpathrectangle{\pgfqpoint{0.625000in}{0.550000in}}{\pgfqpoint{3.875000in}{3.850000in}} %
\pgfusepath{clip}%
\pgfsetbuttcap%
\pgfsetroundjoin%
\pgfsetlinewidth{0.250937pt}%
\definecolor{currentstroke}{rgb}{0.000000,0.000000,0.000000}%
\pgfsetstrokecolor{currentstroke}%
\pgfsetdash{}{0pt}%
\pgfpathmoveto{\pgfqpoint{0.625000in}{1.935603in}}%
\pgfpathlineto{\pgfqpoint{0.629345in}{1.939474in}}%
\pgfpathlineto{\pgfqpoint{0.630747in}{1.949123in}}%
\pgfpathlineto{\pgfqpoint{0.628270in}{1.958772in}}%
\pgfpathlineto{\pgfqpoint{0.629100in}{1.968421in}}%
\pgfpathlineto{\pgfqpoint{0.630758in}{1.978070in}}%
\pgfpathlineto{\pgfqpoint{0.632130in}{1.987719in}}%
\pgfpathlineto{\pgfqpoint{0.634493in}{1.997368in}}%
\pgfpathlineto{\pgfqpoint{0.625000in}{2.005449in}}%
\pgfusepath{stroke}%
\end{pgfscope}%
\begin{pgfscope}%
\pgfpathrectangle{\pgfqpoint{0.625000in}{0.550000in}}{\pgfqpoint{3.875000in}{3.850000in}} %
\pgfusepath{clip}%
\pgfsetbuttcap%
\pgfsetroundjoin%
\pgfsetlinewidth{0.250937pt}%
\definecolor{currentstroke}{rgb}{0.000000,0.000000,0.000000}%
\pgfsetstrokecolor{currentstroke}%
\pgfsetdash{}{0pt}%
\pgfpathmoveto{\pgfqpoint{0.625000in}{2.032725in}}%
\pgfpathlineto{\pgfqpoint{0.628133in}{2.035965in}}%
\pgfpathlineto{\pgfqpoint{0.627869in}{2.045614in}}%
\pgfpathlineto{\pgfqpoint{0.628046in}{2.055263in}}%
\pgfpathlineto{\pgfqpoint{0.627408in}{2.064912in}}%
\pgfpathlineto{\pgfqpoint{0.628431in}{2.074561in}}%
\pgfpathlineto{\pgfqpoint{0.629489in}{2.084211in}}%
\pgfpathlineto{\pgfqpoint{0.633939in}{2.093860in}}%
\pgfpathlineto{\pgfqpoint{0.629538in}{2.103509in}}%
\pgfpathlineto{\pgfqpoint{0.628468in}{2.113158in}}%
\pgfpathlineto{\pgfqpoint{0.629213in}{2.122807in}}%
\pgfpathlineto{\pgfqpoint{0.628745in}{2.132456in}}%
\pgfpathlineto{\pgfqpoint{0.630756in}{2.142105in}}%
\pgfpathlineto{\pgfqpoint{0.629583in}{2.151754in}}%
\pgfpathlineto{\pgfqpoint{0.627527in}{2.161404in}}%
\pgfpathlineto{\pgfqpoint{0.625000in}{2.161436in}}%
\pgfusepath{stroke}%
\end{pgfscope}%
\begin{pgfscope}%
\pgfpathrectangle{\pgfqpoint{0.625000in}{0.550000in}}{\pgfqpoint{3.875000in}{3.850000in}} %
\pgfusepath{clip}%
\pgfsetbuttcap%
\pgfsetroundjoin%
\pgfsetlinewidth{0.250937pt}%
\definecolor{currentstroke}{rgb}{0.000000,0.000000,0.000000}%
\pgfsetstrokecolor{currentstroke}%
\pgfsetdash{}{0pt}%
\pgfpathmoveto{\pgfqpoint{0.625000in}{2.180401in}}%
\pgfpathlineto{\pgfqpoint{0.630070in}{2.180702in}}%
\pgfpathlineto{\pgfqpoint{0.631186in}{2.190351in}}%
\pgfpathlineto{\pgfqpoint{0.630284in}{2.200000in}}%
\pgfpathlineto{\pgfqpoint{0.631041in}{2.209649in}}%
\pgfpathlineto{\pgfqpoint{0.629601in}{2.219298in}}%
\pgfpathlineto{\pgfqpoint{0.628360in}{2.228947in}}%
\pgfpathlineto{\pgfqpoint{0.632718in}{2.238596in}}%
\pgfpathlineto{\pgfqpoint{0.631212in}{2.248246in}}%
\pgfpathlineto{\pgfqpoint{0.629328in}{2.257895in}}%
\pgfpathlineto{\pgfqpoint{0.628844in}{2.267544in}}%
\pgfpathlineto{\pgfqpoint{0.625834in}{2.277193in}}%
\pgfpathlineto{\pgfqpoint{0.632206in}{2.286842in}}%
\pgfpathlineto{\pgfqpoint{0.627467in}{2.296491in}}%
\pgfpathlineto{\pgfqpoint{0.628373in}{2.306140in}}%
\pgfpathlineto{\pgfqpoint{0.625000in}{2.308560in}}%
\pgfusepath{stroke}%
\end{pgfscope}%
\begin{pgfscope}%
\pgfpathrectangle{\pgfqpoint{0.625000in}{0.550000in}}{\pgfqpoint{3.875000in}{3.850000in}} %
\pgfusepath{clip}%
\pgfsetbuttcap%
\pgfsetroundjoin%
\pgfsetlinewidth{0.250937pt}%
\definecolor{currentstroke}{rgb}{0.000000,0.000000,0.000000}%
\pgfsetstrokecolor{currentstroke}%
\pgfsetdash{}{0pt}%
\pgfpathmoveto{\pgfqpoint{0.625000in}{2.334755in}}%
\pgfpathlineto{\pgfqpoint{0.630696in}{2.335088in}}%
\pgfpathlineto{\pgfqpoint{0.627726in}{2.344737in}}%
\pgfpathlineto{\pgfqpoint{0.628647in}{2.354386in}}%
\pgfpathlineto{\pgfqpoint{0.628857in}{2.364035in}}%
\pgfpathlineto{\pgfqpoint{0.625000in}{2.372636in}}%
\pgfusepath{stroke}%
\end{pgfscope}%
\begin{pgfscope}%
\pgfpathrectangle{\pgfqpoint{0.625000in}{0.550000in}}{\pgfqpoint{3.875000in}{3.850000in}} %
\pgfusepath{clip}%
\pgfsetbuttcap%
\pgfsetroundjoin%
\pgfsetlinewidth{0.250937pt}%
\definecolor{currentstroke}{rgb}{0.000000,0.000000,0.000000}%
\pgfsetstrokecolor{currentstroke}%
\pgfsetdash{}{0pt}%
\pgfpathmoveto{\pgfqpoint{0.625000in}{2.374328in}}%
\pgfpathlineto{\pgfqpoint{0.629347in}{2.383333in}}%
\pgfpathlineto{\pgfqpoint{0.627790in}{2.392982in}}%
\pgfpathlineto{\pgfqpoint{0.628106in}{2.402632in}}%
\pgfpathlineto{\pgfqpoint{0.625000in}{2.409216in}}%
\pgfusepath{stroke}%
\end{pgfscope}%
\begin{pgfscope}%
\pgfpathrectangle{\pgfqpoint{0.625000in}{0.550000in}}{\pgfqpoint{3.875000in}{3.850000in}} %
\pgfusepath{clip}%
\pgfsetbuttcap%
\pgfsetroundjoin%
\pgfsetlinewidth{0.250937pt}%
\definecolor{currentstroke}{rgb}{0.000000,0.000000,0.000000}%
\pgfsetstrokecolor{currentstroke}%
\pgfsetdash{}{0pt}%
\pgfpathmoveto{\pgfqpoint{0.625000in}{2.418118in}}%
\pgfpathlineto{\pgfqpoint{0.625715in}{2.421930in}}%
\pgfpathlineto{\pgfqpoint{0.625789in}{2.431579in}}%
\pgfpathlineto{\pgfqpoint{0.625681in}{2.441228in}}%
\pgfpathlineto{\pgfqpoint{0.625480in}{2.450877in}}%
\pgfpathlineto{\pgfqpoint{0.625096in}{2.460526in}}%
\pgfpathlineto{\pgfqpoint{0.625000in}{2.461794in}}%
\pgfusepath{stroke}%
\end{pgfscope}%
\begin{pgfscope}%
\pgfpathrectangle{\pgfqpoint{0.625000in}{0.550000in}}{\pgfqpoint{3.875000in}{3.850000in}} %
\pgfusepath{clip}%
\pgfsetbuttcap%
\pgfsetroundjoin%
\pgfsetlinewidth{0.250937pt}%
\definecolor{currentstroke}{rgb}{0.000000,0.000000,0.000000}%
\pgfsetstrokecolor{currentstroke}%
\pgfsetdash{}{0pt}%
\pgfpathmoveto{\pgfqpoint{0.625000in}{2.497855in}}%
\pgfpathlineto{\pgfqpoint{0.625096in}{2.499123in}}%
\pgfpathlineto{\pgfqpoint{0.625480in}{2.508772in}}%
\pgfpathlineto{\pgfqpoint{0.625681in}{2.518421in}}%
\pgfpathlineto{\pgfqpoint{0.625789in}{2.528070in}}%
\pgfpathlineto{\pgfqpoint{0.625715in}{2.537719in}}%
\pgfpathlineto{\pgfqpoint{0.625000in}{2.541531in}}%
\pgfusepath{stroke}%
\end{pgfscope}%
\begin{pgfscope}%
\pgfpathrectangle{\pgfqpoint{0.625000in}{0.550000in}}{\pgfqpoint{3.875000in}{3.850000in}} %
\pgfusepath{clip}%
\pgfsetbuttcap%
\pgfsetroundjoin%
\pgfsetlinewidth{0.250937pt}%
\definecolor{currentstroke}{rgb}{0.000000,0.000000,0.000000}%
\pgfsetstrokecolor{currentstroke}%
\pgfsetdash{}{0pt}%
\pgfpathmoveto{\pgfqpoint{0.625000in}{2.550433in}}%
\pgfpathlineto{\pgfqpoint{0.628106in}{2.557018in}}%
\pgfpathlineto{\pgfqpoint{0.627790in}{2.566667in}}%
\pgfpathlineto{\pgfqpoint{0.629347in}{2.576316in}}%
\pgfpathlineto{\pgfqpoint{0.625000in}{2.585322in}}%
\pgfusepath{stroke}%
\end{pgfscope}%
\begin{pgfscope}%
\pgfpathrectangle{\pgfqpoint{0.625000in}{0.550000in}}{\pgfqpoint{3.875000in}{3.850000in}} %
\pgfusepath{clip}%
\pgfsetbuttcap%
\pgfsetroundjoin%
\pgfsetlinewidth{0.250937pt}%
\definecolor{currentstroke}{rgb}{0.000000,0.000000,0.000000}%
\pgfsetstrokecolor{currentstroke}%
\pgfsetdash{}{0pt}%
\pgfpathmoveto{\pgfqpoint{0.625000in}{2.587013in}}%
\pgfpathlineto{\pgfqpoint{0.628857in}{2.595614in}}%
\pgfpathlineto{\pgfqpoint{0.628647in}{2.605263in}}%
\pgfpathlineto{\pgfqpoint{0.627726in}{2.614912in}}%
\pgfpathlineto{\pgfqpoint{0.630696in}{2.624561in}}%
\pgfpathlineto{\pgfqpoint{0.625000in}{2.624895in}}%
\pgfusepath{stroke}%
\end{pgfscope}%
\begin{pgfscope}%
\pgfpathrectangle{\pgfqpoint{0.625000in}{0.550000in}}{\pgfqpoint{3.875000in}{3.850000in}} %
\pgfusepath{clip}%
\pgfsetbuttcap%
\pgfsetroundjoin%
\pgfsetlinewidth{0.250937pt}%
\definecolor{currentstroke}{rgb}{0.000000,0.000000,0.000000}%
\pgfsetstrokecolor{currentstroke}%
\pgfsetdash{}{0pt}%
\pgfpathmoveto{\pgfqpoint{0.625000in}{2.651089in}}%
\pgfpathlineto{\pgfqpoint{0.628373in}{2.653509in}}%
\pgfpathlineto{\pgfqpoint{0.627467in}{2.663158in}}%
\pgfpathlineto{\pgfqpoint{0.632206in}{2.672807in}}%
\pgfpathlineto{\pgfqpoint{0.625834in}{2.682456in}}%
\pgfpathlineto{\pgfqpoint{0.628844in}{2.692105in}}%
\pgfpathlineto{\pgfqpoint{0.629328in}{2.701754in}}%
\pgfpathlineto{\pgfqpoint{0.631212in}{2.711404in}}%
\pgfpathlineto{\pgfqpoint{0.632718in}{2.721053in}}%
\pgfpathlineto{\pgfqpoint{0.628360in}{2.730702in}}%
\pgfpathlineto{\pgfqpoint{0.629601in}{2.740351in}}%
\pgfpathlineto{\pgfqpoint{0.631041in}{2.750000in}}%
\pgfpathlineto{\pgfqpoint{0.630284in}{2.759649in}}%
\pgfpathlineto{\pgfqpoint{0.631186in}{2.769298in}}%
\pgfpathlineto{\pgfqpoint{0.630070in}{2.778947in}}%
\pgfpathlineto{\pgfqpoint{0.625000in}{2.779249in}}%
\pgfusepath{stroke}%
\end{pgfscope}%
\begin{pgfscope}%
\pgfpathrectangle{\pgfqpoint{0.625000in}{0.550000in}}{\pgfqpoint{3.875000in}{3.850000in}} %
\pgfusepath{clip}%
\pgfsetbuttcap%
\pgfsetroundjoin%
\pgfsetlinewidth{0.250937pt}%
\definecolor{currentstroke}{rgb}{0.000000,0.000000,0.000000}%
\pgfsetstrokecolor{currentstroke}%
\pgfsetdash{}{0pt}%
\pgfpathmoveto{\pgfqpoint{0.625000in}{2.798213in}}%
\pgfpathlineto{\pgfqpoint{0.627527in}{2.798246in}}%
\pgfpathlineto{\pgfqpoint{0.629583in}{2.807895in}}%
\pgfpathlineto{\pgfqpoint{0.630756in}{2.817544in}}%
\pgfpathlineto{\pgfqpoint{0.628745in}{2.827193in}}%
\pgfpathlineto{\pgfqpoint{0.629213in}{2.836842in}}%
\pgfpathlineto{\pgfqpoint{0.628468in}{2.846491in}}%
\pgfpathlineto{\pgfqpoint{0.629538in}{2.856140in}}%
\pgfpathlineto{\pgfqpoint{0.633939in}{2.865789in}}%
\pgfpathlineto{\pgfqpoint{0.629489in}{2.875439in}}%
\pgfpathlineto{\pgfqpoint{0.628431in}{2.885088in}}%
\pgfpathlineto{\pgfqpoint{0.627408in}{2.894737in}}%
\pgfpathlineto{\pgfqpoint{0.628046in}{2.904386in}}%
\pgfpathlineto{\pgfqpoint{0.627869in}{2.914035in}}%
\pgfpathlineto{\pgfqpoint{0.628133in}{2.923684in}}%
\pgfpathlineto{\pgfqpoint{0.625000in}{2.926924in}}%
\pgfusepath{stroke}%
\end{pgfscope}%
\begin{pgfscope}%
\pgfpathrectangle{\pgfqpoint{0.625000in}{0.550000in}}{\pgfqpoint{3.875000in}{3.850000in}} %
\pgfusepath{clip}%
\pgfsetbuttcap%
\pgfsetroundjoin%
\pgfsetlinewidth{0.250937pt}%
\definecolor{currentstroke}{rgb}{0.000000,0.000000,0.000000}%
\pgfsetstrokecolor{currentstroke}%
\pgfsetdash{}{0pt}%
\pgfpathmoveto{\pgfqpoint{0.625000in}{2.954200in}}%
\pgfpathlineto{\pgfqpoint{0.634493in}{2.962281in}}%
\pgfpathlineto{\pgfqpoint{0.632130in}{2.971930in}}%
\pgfpathlineto{\pgfqpoint{0.630758in}{2.981579in}}%
\pgfpathlineto{\pgfqpoint{0.629100in}{2.991228in}}%
\pgfpathlineto{\pgfqpoint{0.628270in}{3.000877in}}%
\pgfpathlineto{\pgfqpoint{0.630747in}{3.010526in}}%
\pgfpathlineto{\pgfqpoint{0.629345in}{3.020175in}}%
\pgfpathlineto{\pgfqpoint{0.625000in}{3.024046in}}%
\pgfusepath{stroke}%
\end{pgfscope}%
\begin{pgfscope}%
\pgfpathrectangle{\pgfqpoint{0.625000in}{0.550000in}}{\pgfqpoint{3.875000in}{3.850000in}} %
\pgfusepath{clip}%
\pgfsetbuttcap%
\pgfsetroundjoin%
\pgfsetlinewidth{0.250937pt}%
\definecolor{currentstroke}{rgb}{0.000000,0.000000,0.000000}%
\pgfsetstrokecolor{currentstroke}%
\pgfsetdash{}{0pt}%
\pgfpathmoveto{\pgfqpoint{0.625000in}{3.037842in}}%
\pgfpathlineto{\pgfqpoint{0.626531in}{3.039474in}}%
\pgfpathlineto{\pgfqpoint{0.625000in}{3.044072in}}%
\pgfusepath{stroke}%
\end{pgfscope}%
\begin{pgfscope}%
\pgfpathrectangle{\pgfqpoint{0.625000in}{0.550000in}}{\pgfqpoint{3.875000in}{3.850000in}} %
\pgfusepath{clip}%
\pgfsetbuttcap%
\pgfsetroundjoin%
\pgfsetlinewidth{0.250937pt}%
\definecolor{currentstroke}{rgb}{0.000000,0.000000,0.000000}%
\pgfsetstrokecolor{currentstroke}%
\pgfsetdash{}{0pt}%
\pgfpathmoveto{\pgfqpoint{0.625000in}{3.050918in}}%
\pgfpathlineto{\pgfqpoint{0.632227in}{3.058772in}}%
\pgfpathlineto{\pgfqpoint{0.625000in}{3.063317in}}%
\pgfusepath{stroke}%
\end{pgfscope}%
\begin{pgfscope}%
\pgfpathrectangle{\pgfqpoint{0.625000in}{0.550000in}}{\pgfqpoint{3.875000in}{3.850000in}} %
\pgfusepath{clip}%
\pgfsetbuttcap%
\pgfsetroundjoin%
\pgfsetlinewidth{0.250937pt}%
\definecolor{currentstroke}{rgb}{0.000000,0.000000,0.000000}%
\pgfsetstrokecolor{currentstroke}%
\pgfsetdash{}{0pt}%
\pgfpathmoveto{\pgfqpoint{0.625000in}{3.076562in}}%
\pgfpathlineto{\pgfqpoint{0.627781in}{3.078070in}}%
\pgfpathlineto{\pgfqpoint{0.629365in}{3.087719in}}%
\pgfpathlineto{\pgfqpoint{0.625000in}{3.087873in}}%
\pgfusepath{stroke}%
\end{pgfscope}%
\begin{pgfscope}%
\pgfpathrectangle{\pgfqpoint{0.625000in}{0.550000in}}{\pgfqpoint{3.875000in}{3.850000in}} %
\pgfusepath{clip}%
\pgfsetbuttcap%
\pgfsetroundjoin%
\pgfsetlinewidth{0.250937pt}%
\definecolor{currentstroke}{rgb}{0.000000,0.000000,0.000000}%
\pgfsetstrokecolor{currentstroke}%
\pgfsetdash{}{0pt}%
\pgfpathmoveto{\pgfqpoint{0.625000in}{3.106665in}}%
\pgfpathlineto{\pgfqpoint{0.632979in}{3.107018in}}%
\pgfpathlineto{\pgfqpoint{0.634712in}{3.116349in}}%
\pgfpathlineto{\pgfqpoint{0.634986in}{3.116667in}}%
\pgfpathlineto{\pgfqpoint{0.636895in}{3.126316in}}%
\pgfpathlineto{\pgfqpoint{0.634712in}{3.129412in}}%
\pgfpathlineto{\pgfqpoint{0.632834in}{3.135965in}}%
\pgfpathlineto{\pgfqpoint{0.627655in}{3.145614in}}%
\pgfpathlineto{\pgfqpoint{0.631590in}{3.155263in}}%
\pgfpathlineto{\pgfqpoint{0.631136in}{3.164912in}}%
\pgfpathlineto{\pgfqpoint{0.630347in}{3.174561in}}%
\pgfpathlineto{\pgfqpoint{0.628706in}{3.184211in}}%
\pgfpathlineto{\pgfqpoint{0.628344in}{3.193860in}}%
\pgfpathlineto{\pgfqpoint{0.632795in}{3.203509in}}%
\pgfpathlineto{\pgfqpoint{0.630260in}{3.213158in}}%
\pgfpathlineto{\pgfqpoint{0.630026in}{3.222807in}}%
\pgfpathlineto{\pgfqpoint{0.627064in}{3.232456in}}%
\pgfpathlineto{\pgfqpoint{0.627127in}{3.242105in}}%
\pgfpathlineto{\pgfqpoint{0.625000in}{3.242369in}}%
\pgfusepath{stroke}%
\end{pgfscope}%
\begin{pgfscope}%
\pgfpathrectangle{\pgfqpoint{0.625000in}{0.550000in}}{\pgfqpoint{3.875000in}{3.850000in}} %
\pgfusepath{clip}%
\pgfsetbuttcap%
\pgfsetroundjoin%
\pgfsetlinewidth{0.250937pt}%
\definecolor{currentstroke}{rgb}{0.000000,0.000000,0.000000}%
\pgfsetstrokecolor{currentstroke}%
\pgfsetdash{}{0pt}%
\pgfpathmoveto{\pgfqpoint{0.625000in}{3.261202in}}%
\pgfpathlineto{\pgfqpoint{0.626706in}{3.261404in}}%
\pgfpathlineto{\pgfqpoint{0.627064in}{3.271053in}}%
\pgfpathlineto{\pgfqpoint{0.625770in}{3.280702in}}%
\pgfpathlineto{\pgfqpoint{0.630073in}{3.290351in}}%
\pgfpathlineto{\pgfqpoint{0.631836in}{3.300000in}}%
\pgfpathlineto{\pgfqpoint{0.629549in}{3.309649in}}%
\pgfpathlineto{\pgfqpoint{0.625000in}{3.312453in}}%
\pgfusepath{stroke}%
\end{pgfscope}%
\begin{pgfscope}%
\pgfpathrectangle{\pgfqpoint{0.625000in}{0.550000in}}{\pgfqpoint{3.875000in}{3.850000in}} %
\pgfusepath{clip}%
\pgfsetbuttcap%
\pgfsetroundjoin%
\pgfsetlinewidth{0.250937pt}%
\definecolor{currentstroke}{rgb}{0.000000,0.000000,0.000000}%
\pgfsetstrokecolor{currentstroke}%
\pgfsetdash{}{0pt}%
\pgfpathmoveto{\pgfqpoint{0.625000in}{3.323408in}}%
\pgfpathlineto{\pgfqpoint{0.629881in}{3.328947in}}%
\pgfpathlineto{\pgfqpoint{0.625000in}{3.335680in}}%
\pgfusepath{stroke}%
\end{pgfscope}%
\begin{pgfscope}%
\pgfpathrectangle{\pgfqpoint{0.625000in}{0.550000in}}{\pgfqpoint{3.875000in}{3.850000in}} %
\pgfusepath{clip}%
\pgfsetbuttcap%
\pgfsetroundjoin%
\pgfsetlinewidth{0.250937pt}%
\definecolor{currentstroke}{rgb}{0.000000,0.000000,0.000000}%
\pgfsetstrokecolor{currentstroke}%
\pgfsetdash{}{0pt}%
\pgfpathmoveto{\pgfqpoint{0.625000in}{3.341301in}}%
\pgfpathlineto{\pgfqpoint{0.630568in}{3.348246in}}%
\pgfpathlineto{\pgfqpoint{0.630964in}{3.357895in}}%
\pgfpathlineto{\pgfqpoint{0.629312in}{3.367544in}}%
\pgfpathlineto{\pgfqpoint{0.625000in}{3.374038in}}%
\pgfusepath{stroke}%
\end{pgfscope}%
\begin{pgfscope}%
\pgfpathrectangle{\pgfqpoint{0.625000in}{0.550000in}}{\pgfqpoint{3.875000in}{3.850000in}} %
\pgfusepath{clip}%
\pgfsetbuttcap%
\pgfsetroundjoin%
\pgfsetlinewidth{0.250937pt}%
\definecolor{currentstroke}{rgb}{0.000000,0.000000,0.000000}%
\pgfsetstrokecolor{currentstroke}%
\pgfsetdash{}{0pt}%
\pgfpathmoveto{\pgfqpoint{0.625000in}{3.382117in}}%
\pgfpathlineto{\pgfqpoint{0.628222in}{3.386842in}}%
\pgfpathlineto{\pgfqpoint{0.631035in}{3.396491in}}%
\pgfpathlineto{\pgfqpoint{0.625000in}{3.396766in}}%
\pgfusepath{stroke}%
\end{pgfscope}%
\begin{pgfscope}%
\pgfpathrectangle{\pgfqpoint{0.625000in}{0.550000in}}{\pgfqpoint{3.875000in}{3.850000in}} %
\pgfusepath{clip}%
\pgfsetbuttcap%
\pgfsetroundjoin%
\pgfsetlinewidth{0.250937pt}%
\definecolor{currentstroke}{rgb}{0.000000,0.000000,0.000000}%
\pgfsetstrokecolor{currentstroke}%
\pgfsetdash{}{0pt}%
\pgfpathmoveto{\pgfqpoint{0.625000in}{3.418472in}}%
\pgfpathlineto{\pgfqpoint{0.632304in}{3.425439in}}%
\pgfpathlineto{\pgfqpoint{0.634712in}{3.429641in}}%
\pgfpathlineto{\pgfqpoint{0.640194in}{3.435088in}}%
\pgfpathlineto{\pgfqpoint{0.643191in}{3.444737in}}%
\pgfpathlineto{\pgfqpoint{0.640194in}{3.454386in}}%
\pgfpathlineto{\pgfqpoint{0.634712in}{3.459832in}}%
\pgfpathlineto{\pgfqpoint{0.632304in}{3.464035in}}%
\pgfpathlineto{\pgfqpoint{0.632330in}{3.473684in}}%
\pgfpathlineto{\pgfqpoint{0.631900in}{3.483333in}}%
\pgfpathlineto{\pgfqpoint{0.630730in}{3.492982in}}%
\pgfpathlineto{\pgfqpoint{0.628291in}{3.502632in}}%
\pgfpathlineto{\pgfqpoint{0.628820in}{3.512281in}}%
\pgfpathlineto{\pgfqpoint{0.629850in}{3.521930in}}%
\pgfpathlineto{\pgfqpoint{0.625000in}{3.527424in}}%
\pgfusepath{stroke}%
\end{pgfscope}%
\begin{pgfscope}%
\pgfpathrectangle{\pgfqpoint{0.625000in}{0.550000in}}{\pgfqpoint{3.875000in}{3.850000in}} %
\pgfusepath{clip}%
\pgfsetbuttcap%
\pgfsetroundjoin%
\pgfsetlinewidth{0.250937pt}%
\definecolor{currentstroke}{rgb}{0.000000,0.000000,0.000000}%
\pgfsetstrokecolor{currentstroke}%
\pgfsetdash{}{0pt}%
\pgfpathmoveto{\pgfqpoint{0.625000in}{3.534543in}}%
\pgfpathlineto{\pgfqpoint{0.630903in}{3.541228in}}%
\pgfpathlineto{\pgfqpoint{0.630769in}{3.550877in}}%
\pgfpathlineto{\pgfqpoint{0.625000in}{3.551101in}}%
\pgfusepath{stroke}%
\end{pgfscope}%
\begin{pgfscope}%
\pgfpathrectangle{\pgfqpoint{0.625000in}{0.550000in}}{\pgfqpoint{3.875000in}{3.850000in}} %
\pgfusepath{clip}%
\pgfsetbuttcap%
\pgfsetroundjoin%
\pgfsetlinewidth{0.250937pt}%
\definecolor{currentstroke}{rgb}{0.000000,0.000000,0.000000}%
\pgfsetstrokecolor{currentstroke}%
\pgfsetdash{}{0pt}%
\pgfpathmoveto{\pgfqpoint{0.625000in}{3.569957in}}%
\pgfpathlineto{\pgfqpoint{0.628622in}{3.570175in}}%
\pgfpathlineto{\pgfqpoint{0.626008in}{3.579825in}}%
\pgfpathlineto{\pgfqpoint{0.627087in}{3.589474in}}%
\pgfpathlineto{\pgfqpoint{0.629393in}{3.599123in}}%
\pgfpathlineto{\pgfqpoint{0.629435in}{3.608772in}}%
\pgfpathlineto{\pgfqpoint{0.628373in}{3.618421in}}%
\pgfpathlineto{\pgfqpoint{0.632209in}{3.628070in}}%
\pgfpathlineto{\pgfqpoint{0.633939in}{3.637719in}}%
\pgfpathlineto{\pgfqpoint{0.631937in}{3.647368in}}%
\pgfpathlineto{\pgfqpoint{0.628431in}{3.657018in}}%
\pgfpathlineto{\pgfqpoint{0.625000in}{3.658617in}}%
\pgfusepath{stroke}%
\end{pgfscope}%
\begin{pgfscope}%
\pgfpathrectangle{\pgfqpoint{0.625000in}{0.550000in}}{\pgfqpoint{3.875000in}{3.850000in}} %
\pgfusepath{clip}%
\pgfsetbuttcap%
\pgfsetroundjoin%
\pgfsetlinewidth{0.250937pt}%
\definecolor{currentstroke}{rgb}{0.000000,0.000000,0.000000}%
\pgfsetstrokecolor{currentstroke}%
\pgfsetdash{}{0pt}%
\pgfpathmoveto{\pgfqpoint{0.625000in}{3.673432in}}%
\pgfpathlineto{\pgfqpoint{0.630136in}{3.676316in}}%
\pgfpathlineto{\pgfqpoint{0.632752in}{3.685965in}}%
\pgfpathlineto{\pgfqpoint{0.628440in}{3.695614in}}%
\pgfpathlineto{\pgfqpoint{0.625000in}{3.699457in}}%
\pgfusepath{stroke}%
\end{pgfscope}%
\begin{pgfscope}%
\pgfpathrectangle{\pgfqpoint{0.625000in}{0.550000in}}{\pgfqpoint{3.875000in}{3.850000in}} %
\pgfusepath{clip}%
\pgfsetbuttcap%
\pgfsetroundjoin%
\pgfsetlinewidth{0.250937pt}%
\definecolor{currentstroke}{rgb}{0.000000,0.000000,0.000000}%
\pgfsetstrokecolor{currentstroke}%
\pgfsetdash{}{0pt}%
\pgfpathmoveto{\pgfqpoint{0.625000in}{3.724435in}}%
\pgfpathlineto{\pgfqpoint{0.627978in}{3.724561in}}%
\pgfpathlineto{\pgfqpoint{0.629520in}{3.734211in}}%
\pgfpathlineto{\pgfqpoint{0.628022in}{3.743860in}}%
\pgfpathlineto{\pgfqpoint{0.625986in}{3.753509in}}%
\pgfpathlineto{\pgfqpoint{0.625000in}{3.757228in}}%
\pgfusepath{stroke}%
\end{pgfscope}%
\begin{pgfscope}%
\pgfpathrectangle{\pgfqpoint{0.625000in}{0.550000in}}{\pgfqpoint{3.875000in}{3.850000in}} %
\pgfusepath{clip}%
\pgfsetbuttcap%
\pgfsetroundjoin%
\pgfsetlinewidth{0.250937pt}%
\definecolor{currentstroke}{rgb}{0.000000,0.000000,0.000000}%
\pgfsetstrokecolor{currentstroke}%
\pgfsetdash{}{0pt}%
\pgfpathmoveto{\pgfqpoint{0.625000in}{3.770624in}}%
\pgfpathlineto{\pgfqpoint{0.625251in}{3.772807in}}%
\pgfpathlineto{\pgfqpoint{0.625674in}{3.782456in}}%
\pgfpathlineto{\pgfqpoint{0.628360in}{3.792105in}}%
\pgfpathlineto{\pgfqpoint{0.626222in}{3.801754in}}%
\pgfpathlineto{\pgfqpoint{0.627781in}{3.811404in}}%
\pgfpathlineto{\pgfqpoint{0.630169in}{3.821053in}}%
\pgfpathlineto{\pgfqpoint{0.632211in}{3.830702in}}%
\pgfpathlineto{\pgfqpoint{0.630594in}{3.840351in}}%
\pgfpathlineto{\pgfqpoint{0.628844in}{3.850000in}}%
\pgfpathlineto{\pgfqpoint{0.630981in}{3.859649in}}%
\pgfpathlineto{\pgfqpoint{0.625000in}{3.859833in}}%
\pgfusepath{stroke}%
\end{pgfscope}%
\begin{pgfscope}%
\pgfpathrectangle{\pgfqpoint{0.625000in}{0.550000in}}{\pgfqpoint{3.875000in}{3.850000in}} %
\pgfusepath{clip}%
\pgfsetbuttcap%
\pgfsetroundjoin%
\pgfsetlinewidth{0.250937pt}%
\definecolor{currentstroke}{rgb}{0.000000,0.000000,0.000000}%
\pgfsetstrokecolor{currentstroke}%
\pgfsetdash{}{0pt}%
\pgfpathmoveto{\pgfqpoint{0.625000in}{3.878725in}}%
\pgfpathlineto{\pgfqpoint{0.628841in}{3.878947in}}%
\pgfpathlineto{\pgfqpoint{0.627399in}{3.888596in}}%
\pgfpathlineto{\pgfqpoint{0.630060in}{3.898246in}}%
\pgfpathlineto{\pgfqpoint{0.630752in}{3.907895in}}%
\pgfpathlineto{\pgfqpoint{0.625000in}{3.915721in}}%
\pgfusepath{stroke}%
\end{pgfscope}%
\begin{pgfscope}%
\pgfpathrectangle{\pgfqpoint{0.625000in}{0.550000in}}{\pgfqpoint{3.875000in}{3.850000in}} %
\pgfusepath{clip}%
\pgfsetbuttcap%
\pgfsetroundjoin%
\pgfsetlinewidth{0.250937pt}%
\definecolor{currentstroke}{rgb}{0.000000,0.000000,0.000000}%
\pgfsetstrokecolor{currentstroke}%
\pgfsetdash{}{0pt}%
\pgfpathmoveto{\pgfqpoint{0.625000in}{3.918251in}}%
\pgfpathlineto{\pgfqpoint{0.634493in}{3.927193in}}%
\pgfpathlineto{\pgfqpoint{0.632245in}{3.936842in}}%
\pgfpathlineto{\pgfqpoint{0.631703in}{3.946491in}}%
\pgfpathlineto{\pgfqpoint{0.630713in}{3.956140in}}%
\pgfpathlineto{\pgfqpoint{0.627539in}{3.965789in}}%
\pgfpathlineto{\pgfqpoint{0.631289in}{3.975439in}}%
\pgfpathlineto{\pgfqpoint{0.630506in}{3.985088in}}%
\pgfpathlineto{\pgfqpoint{0.626824in}{3.994737in}}%
\pgfpathlineto{\pgfqpoint{0.627064in}{4.004386in}}%
\pgfpathlineto{\pgfqpoint{0.626993in}{4.014035in}}%
\pgfpathlineto{\pgfqpoint{0.625000in}{4.014275in}}%
\pgfusepath{stroke}%
\end{pgfscope}%
\begin{pgfscope}%
\pgfpathrectangle{\pgfqpoint{0.625000in}{0.550000in}}{\pgfqpoint{3.875000in}{3.850000in}} %
\pgfusepath{clip}%
\pgfsetbuttcap%
\pgfsetroundjoin%
\pgfsetlinewidth{0.250937pt}%
\definecolor{currentstroke}{rgb}{0.000000,0.000000,0.000000}%
\pgfsetstrokecolor{currentstroke}%
\pgfsetdash{}{0pt}%
\pgfpathmoveto{\pgfqpoint{0.625000in}{4.033186in}}%
\pgfpathlineto{\pgfqpoint{0.626317in}{4.033333in}}%
\pgfpathlineto{\pgfqpoint{0.627064in}{4.042982in}}%
\pgfpathlineto{\pgfqpoint{0.625000in}{4.044893in}}%
\pgfusepath{stroke}%
\end{pgfscope}%
\begin{pgfscope}%
\pgfpathrectangle{\pgfqpoint{0.625000in}{0.550000in}}{\pgfqpoint{3.875000in}{3.850000in}} %
\pgfusepath{clip}%
\pgfsetbuttcap%
\pgfsetroundjoin%
\pgfsetlinewidth{0.250937pt}%
\definecolor{currentstroke}{rgb}{0.000000,0.000000,0.000000}%
\pgfsetstrokecolor{currentstroke}%
\pgfsetdash{}{0pt}%
\pgfpathmoveto{\pgfqpoint{0.625000in}{4.058940in}}%
\pgfpathlineto{\pgfqpoint{0.630211in}{4.062281in}}%
\pgfpathlineto{\pgfqpoint{0.630644in}{4.071930in}}%
\pgfpathlineto{\pgfqpoint{0.629583in}{4.081579in}}%
\pgfpathlineto{\pgfqpoint{0.631147in}{4.091228in}}%
\pgfpathlineto{\pgfqpoint{0.631456in}{4.100877in}}%
\pgfpathlineto{\pgfqpoint{0.630071in}{4.110526in}}%
\pgfpathlineto{\pgfqpoint{0.630195in}{4.120175in}}%
\pgfpathlineto{\pgfqpoint{0.627891in}{4.129825in}}%
\pgfpathlineto{\pgfqpoint{0.627278in}{4.139474in}}%
\pgfpathlineto{\pgfqpoint{0.625000in}{4.148388in}}%
\pgfusepath{stroke}%
\end{pgfscope}%
\begin{pgfscope}%
\pgfpathrectangle{\pgfqpoint{0.625000in}{0.550000in}}{\pgfqpoint{3.875000in}{3.850000in}} %
\pgfusepath{clip}%
\pgfsetbuttcap%
\pgfsetroundjoin%
\pgfsetlinewidth{0.250937pt}%
\definecolor{currentstroke}{rgb}{0.000000,0.000000,0.000000}%
\pgfsetstrokecolor{currentstroke}%
\pgfsetdash{}{0pt}%
\pgfpathmoveto{\pgfqpoint{0.625000in}{4.150573in}}%
\pgfpathlineto{\pgfqpoint{0.628424in}{4.158772in}}%
\pgfpathlineto{\pgfqpoint{0.632822in}{4.168421in}}%
\pgfpathlineto{\pgfqpoint{0.625000in}{4.168967in}}%
\pgfusepath{stroke}%
\end{pgfscope}%
\begin{pgfscope}%
\pgfpathrectangle{\pgfqpoint{0.625000in}{0.550000in}}{\pgfqpoint{3.875000in}{3.850000in}} %
\pgfusepath{clip}%
\pgfsetbuttcap%
\pgfsetroundjoin%
\pgfsetlinewidth{0.250937pt}%
\definecolor{currentstroke}{rgb}{0.000000,0.000000,0.000000}%
\pgfsetstrokecolor{currentstroke}%
\pgfsetdash{}{0pt}%
\pgfpathmoveto{\pgfqpoint{0.625000in}{4.187620in}}%
\pgfpathlineto{\pgfqpoint{0.626996in}{4.187719in}}%
\pgfpathlineto{\pgfqpoint{0.626531in}{4.197368in}}%
\pgfpathlineto{\pgfqpoint{0.630715in}{4.207018in}}%
\pgfpathlineto{\pgfqpoint{0.632232in}{4.216667in}}%
\pgfpathlineto{\pgfqpoint{0.630665in}{4.226316in}}%
\pgfpathlineto{\pgfqpoint{0.627064in}{4.235965in}}%
\pgfpathlineto{\pgfqpoint{0.629945in}{4.245614in}}%
\pgfpathlineto{\pgfqpoint{0.630709in}{4.255263in}}%
\pgfpathlineto{\pgfqpoint{0.628926in}{4.264912in}}%
\pgfpathlineto{\pgfqpoint{0.628228in}{4.274561in}}%
\pgfpathlineto{\pgfqpoint{0.625162in}{4.284211in}}%
\pgfpathlineto{\pgfqpoint{0.630051in}{4.293860in}}%
\pgfpathlineto{\pgfqpoint{0.630108in}{4.303509in}}%
\pgfpathlineto{\pgfqpoint{0.631685in}{4.313158in}}%
\pgfpathlineto{\pgfqpoint{0.629376in}{4.322807in}}%
\pgfpathlineto{\pgfqpoint{0.625000in}{4.322937in}}%
\pgfusepath{stroke}%
\end{pgfscope}%
\begin{pgfscope}%
\pgfpathrectangle{\pgfqpoint{0.625000in}{0.550000in}}{\pgfqpoint{3.875000in}{3.850000in}} %
\pgfusepath{clip}%
\pgfsetbuttcap%
\pgfsetroundjoin%
\pgfsetlinewidth{0.250937pt}%
\definecolor{currentstroke}{rgb}{0.000000,0.000000,0.000000}%
\pgfsetstrokecolor{currentstroke}%
\pgfsetdash{}{0pt}%
\pgfpathmoveto{\pgfqpoint{0.625000in}{4.341943in}}%
\pgfpathlineto{\pgfqpoint{0.630680in}{4.342105in}}%
\pgfpathlineto{\pgfqpoint{0.630284in}{4.351754in}}%
\pgfpathlineto{\pgfqpoint{0.633306in}{4.361404in}}%
\pgfpathlineto{\pgfqpoint{0.633901in}{4.371053in}}%
\pgfpathlineto{\pgfqpoint{0.634712in}{4.375355in}}%
\pgfpathlineto{\pgfqpoint{0.644424in}{4.379992in}}%
\pgfpathlineto{\pgfqpoint{0.645369in}{4.380702in}}%
\pgfpathlineto{\pgfqpoint{0.654135in}{4.389411in}}%
\pgfpathlineto{\pgfqpoint{0.654850in}{4.390351in}}%
\pgfpathlineto{\pgfqpoint{0.659517in}{4.400000in}}%
\pgfusepath{stroke}%
\end{pgfscope}%
\begin{pgfscope}%
\pgfpathrectangle{\pgfqpoint{0.625000in}{0.550000in}}{\pgfqpoint{3.875000in}{3.850000in}} %
\pgfusepath{clip}%
\pgfsetbuttcap%
\pgfsetroundjoin%
\pgfsetlinewidth{0.250937pt}%
\definecolor{currentstroke}{rgb}{0.000000,0.000000,0.000000}%
\pgfsetstrokecolor{currentstroke}%
\pgfsetdash{}{0pt}%
\pgfpathmoveto{\pgfqpoint{0.646943in}{0.550000in}}%
\pgfpathlineto{\pgfqpoint{0.644424in}{0.559095in}}%
\pgfpathlineto{\pgfqpoint{0.644288in}{0.559649in}}%
\pgfpathlineto{\pgfqpoint{0.634712in}{0.569163in}}%
\pgfpathlineto{\pgfqpoint{0.634155in}{0.569298in}}%
\pgfpathlineto{\pgfqpoint{0.625000in}{0.573515in}}%
\pgfusepath{stroke}%
\end{pgfscope}%
\begin{pgfscope}%
\pgfpathrectangle{\pgfqpoint{0.625000in}{0.550000in}}{\pgfqpoint{3.875000in}{3.850000in}} %
\pgfusepath{clip}%
\pgfsetbuttcap%
\pgfsetroundjoin%
\pgfsetlinewidth{0.250937pt}%
\definecolor{currentstroke}{rgb}{0.000000,0.000000,0.000000}%
\pgfsetstrokecolor{currentstroke}%
\pgfsetdash{}{0pt}%
\pgfpathmoveto{\pgfqpoint{0.625000in}{0.580347in}}%
\pgfpathlineto{\pgfqpoint{0.630473in}{0.588596in}}%
\pgfpathlineto{\pgfqpoint{0.631181in}{0.598246in}}%
\pgfpathlineto{\pgfqpoint{0.625918in}{0.607895in}}%
\pgfpathlineto{\pgfqpoint{0.627615in}{0.617544in}}%
\pgfpathlineto{\pgfqpoint{0.625000in}{0.617609in}}%
\pgfusepath{stroke}%
\end{pgfscope}%
\begin{pgfscope}%
\pgfpathrectangle{\pgfqpoint{0.625000in}{0.550000in}}{\pgfqpoint{3.875000in}{3.850000in}} %
\pgfusepath{clip}%
\pgfsetbuttcap%
\pgfsetroundjoin%
\pgfsetlinewidth{0.250937pt}%
\definecolor{currentstroke}{rgb}{0.000000,0.000000,0.000000}%
\pgfsetstrokecolor{currentstroke}%
\pgfsetdash{}{0pt}%
\pgfpathmoveto{\pgfqpoint{0.625000in}{0.636805in}}%
\pgfpathlineto{\pgfqpoint{0.626422in}{0.636842in}}%
\pgfpathlineto{\pgfqpoint{0.630176in}{0.646491in}}%
\pgfpathlineto{\pgfqpoint{0.627979in}{0.656140in}}%
\pgfpathlineto{\pgfqpoint{0.628106in}{0.665789in}}%
\pgfpathlineto{\pgfqpoint{0.625000in}{0.671827in}}%
\pgfusepath{stroke}%
\end{pgfscope}%
\begin{pgfscope}%
\pgfpathrectangle{\pgfqpoint{0.625000in}{0.550000in}}{\pgfqpoint{3.875000in}{3.850000in}} %
\pgfusepath{clip}%
\pgfsetbuttcap%
\pgfsetroundjoin%
\pgfsetlinewidth{0.250937pt}%
\definecolor{currentstroke}{rgb}{0.000000,0.000000,0.000000}%
\pgfsetstrokecolor{currentstroke}%
\pgfsetdash{}{0pt}%
\pgfpathmoveto{\pgfqpoint{0.625000in}{0.683346in}}%
\pgfpathlineto{\pgfqpoint{0.625561in}{0.685088in}}%
\pgfpathlineto{\pgfqpoint{0.627083in}{0.694737in}}%
\pgfpathlineto{\pgfqpoint{0.629275in}{0.704386in}}%
\pgfpathlineto{\pgfqpoint{0.628020in}{0.714035in}}%
\pgfpathlineto{\pgfqpoint{0.625359in}{0.723684in}}%
\pgfpathlineto{\pgfqpoint{0.629335in}{0.733333in}}%
\pgfpathlineto{\pgfqpoint{0.630988in}{0.742982in}}%
\pgfpathlineto{\pgfqpoint{0.629401in}{0.752632in}}%
\pgfpathlineto{\pgfqpoint{0.625266in}{0.762281in}}%
\pgfpathlineto{\pgfqpoint{0.625215in}{0.771930in}}%
\pgfpathlineto{\pgfqpoint{0.625000in}{0.771940in}}%
\pgfusepath{stroke}%
\end{pgfscope}%
\begin{pgfscope}%
\pgfpathrectangle{\pgfqpoint{0.625000in}{0.550000in}}{\pgfqpoint{3.875000in}{3.850000in}} %
\pgfusepath{clip}%
\pgfsetbuttcap%
\pgfsetroundjoin%
\pgfsetlinewidth{0.250937pt}%
\definecolor{currentstroke}{rgb}{0.000000,0.000000,0.000000}%
\pgfsetstrokecolor{currentstroke}%
\pgfsetdash{}{0pt}%
\pgfpathmoveto{\pgfqpoint{0.625000in}{0.790807in}}%
\pgfpathlineto{\pgfqpoint{0.631613in}{0.791228in}}%
\pgfpathlineto{\pgfqpoint{0.625595in}{0.800877in}}%
\pgfpathlineto{\pgfqpoint{0.625000in}{0.802302in}}%
\pgfusepath{stroke}%
\end{pgfscope}%
\begin{pgfscope}%
\pgfpathrectangle{\pgfqpoint{0.625000in}{0.550000in}}{\pgfqpoint{3.875000in}{3.850000in}} %
\pgfusepath{clip}%
\pgfsetbuttcap%
\pgfsetroundjoin%
\pgfsetlinewidth{0.250937pt}%
\definecolor{currentstroke}{rgb}{0.000000,0.000000,0.000000}%
\pgfsetstrokecolor{currentstroke}%
\pgfsetdash{}{0pt}%
\pgfpathmoveto{\pgfqpoint{0.625000in}{0.814694in}}%
\pgfpathlineto{\pgfqpoint{0.626401in}{0.820175in}}%
\pgfpathlineto{\pgfqpoint{0.627175in}{0.829825in}}%
\pgfpathlineto{\pgfqpoint{0.629022in}{0.839474in}}%
\pgfpathlineto{\pgfqpoint{0.628137in}{0.849123in}}%
\pgfpathlineto{\pgfqpoint{0.629835in}{0.858772in}}%
\pgfpathlineto{\pgfqpoint{0.627883in}{0.868421in}}%
\pgfpathlineto{\pgfqpoint{0.625796in}{0.878070in}}%
\pgfpathlineto{\pgfqpoint{0.628909in}{0.887719in}}%
\pgfpathlineto{\pgfqpoint{0.628204in}{0.897368in}}%
\pgfpathlineto{\pgfqpoint{0.625000in}{0.899423in}}%
\pgfusepath{stroke}%
\end{pgfscope}%
\begin{pgfscope}%
\pgfpathrectangle{\pgfqpoint{0.625000in}{0.550000in}}{\pgfqpoint{3.875000in}{3.850000in}} %
\pgfusepath{clip}%
\pgfsetbuttcap%
\pgfsetroundjoin%
\pgfsetlinewidth{0.250937pt}%
\definecolor{currentstroke}{rgb}{0.000000,0.000000,0.000000}%
\pgfsetstrokecolor{currentstroke}%
\pgfsetdash{}{0pt}%
\pgfpathmoveto{\pgfqpoint{0.625000in}{0.916335in}}%
\pgfpathlineto{\pgfqpoint{0.625359in}{0.916667in}}%
\pgfpathlineto{\pgfqpoint{0.625562in}{0.926316in}}%
\pgfpathlineto{\pgfqpoint{0.625000in}{0.926373in}}%
\pgfusepath{stroke}%
\end{pgfscope}%
\begin{pgfscope}%
\pgfpathrectangle{\pgfqpoint{0.625000in}{0.550000in}}{\pgfqpoint{3.875000in}{3.850000in}} %
\pgfusepath{clip}%
\pgfsetbuttcap%
\pgfsetroundjoin%
\pgfsetlinewidth{0.250937pt}%
\definecolor{currentstroke}{rgb}{0.000000,0.000000,0.000000}%
\pgfsetstrokecolor{currentstroke}%
\pgfsetdash{}{0pt}%
\pgfpathmoveto{\pgfqpoint{0.625000in}{0.945471in}}%
\pgfpathlineto{\pgfqpoint{0.626298in}{0.945614in}}%
\pgfpathlineto{\pgfqpoint{0.625359in}{0.955263in}}%
\pgfpathlineto{\pgfqpoint{0.625000in}{0.958709in}}%
\pgfusepath{stroke}%
\end{pgfscope}%
\begin{pgfscope}%
\pgfpathrectangle{\pgfqpoint{0.625000in}{0.550000in}}{\pgfqpoint{3.875000in}{3.850000in}} %
\pgfusepath{clip}%
\pgfsetbuttcap%
\pgfsetroundjoin%
\pgfsetlinewidth{0.250937pt}%
\definecolor{currentstroke}{rgb}{0.000000,0.000000,0.000000}%
\pgfsetstrokecolor{currentstroke}%
\pgfsetdash{}{0pt}%
\pgfpathmoveto{\pgfqpoint{0.625000in}{0.966762in}}%
\pgfpathlineto{\pgfqpoint{0.628386in}{0.974561in}}%
\pgfpathlineto{\pgfqpoint{0.629639in}{0.984211in}}%
\pgfpathlineto{\pgfqpoint{0.625441in}{0.993860in}}%
\pgfpathlineto{\pgfqpoint{0.629097in}{1.003509in}}%
\pgfpathlineto{\pgfqpoint{0.630020in}{1.013158in}}%
\pgfpathlineto{\pgfqpoint{0.629407in}{1.022807in}}%
\pgfpathlineto{\pgfqpoint{0.633252in}{1.032456in}}%
\pgfpathlineto{\pgfqpoint{0.625000in}{1.040229in}}%
\pgfusepath{stroke}%
\end{pgfscope}%
\begin{pgfscope}%
\pgfpathrectangle{\pgfqpoint{0.625000in}{0.550000in}}{\pgfqpoint{3.875000in}{3.850000in}} %
\pgfusepath{clip}%
\pgfsetbuttcap%
\pgfsetroundjoin%
\pgfsetlinewidth{0.250937pt}%
\definecolor{currentstroke}{rgb}{0.000000,0.000000,0.000000}%
\pgfsetstrokecolor{currentstroke}%
\pgfsetdash{}{0pt}%
\pgfpathmoveto{\pgfqpoint{0.625000in}{1.046942in}}%
\pgfpathlineto{\pgfqpoint{0.628537in}{1.051754in}}%
\pgfpathlineto{\pgfqpoint{0.628058in}{1.061404in}}%
\pgfpathlineto{\pgfqpoint{0.625417in}{1.071053in}}%
\pgfpathlineto{\pgfqpoint{0.627343in}{1.080702in}}%
\pgfpathlineto{\pgfqpoint{0.625000in}{1.080825in}}%
\pgfusepath{stroke}%
\end{pgfscope}%
\begin{pgfscope}%
\pgfpathrectangle{\pgfqpoint{0.625000in}{0.550000in}}{\pgfqpoint{3.875000in}{3.850000in}} %
\pgfusepath{clip}%
\pgfsetbuttcap%
\pgfsetroundjoin%
\pgfsetlinewidth{0.250937pt}%
\definecolor{currentstroke}{rgb}{0.000000,0.000000,0.000000}%
\pgfsetstrokecolor{currentstroke}%
\pgfsetdash{}{0pt}%
\pgfpathmoveto{\pgfqpoint{0.625000in}{1.099912in}}%
\pgfpathlineto{\pgfqpoint{0.628145in}{1.100000in}}%
\pgfpathlineto{\pgfqpoint{0.625668in}{1.109649in}}%
\pgfpathlineto{\pgfqpoint{0.628943in}{1.119298in}}%
\pgfpathlineto{\pgfqpoint{0.630971in}{1.128947in}}%
\pgfpathlineto{\pgfqpoint{0.628348in}{1.138596in}}%
\pgfpathlineto{\pgfqpoint{0.625483in}{1.148246in}}%
\pgfpathlineto{\pgfqpoint{0.625000in}{1.151704in}}%
\pgfusepath{stroke}%
\end{pgfscope}%
\begin{pgfscope}%
\pgfpathrectangle{\pgfqpoint{0.625000in}{0.550000in}}{\pgfqpoint{3.875000in}{3.850000in}} %
\pgfusepath{clip}%
\pgfsetbuttcap%
\pgfsetroundjoin%
\pgfsetlinewidth{0.250937pt}%
\definecolor{currentstroke}{rgb}{0.000000,0.000000,0.000000}%
\pgfsetstrokecolor{currentstroke}%
\pgfsetdash{}{0pt}%
\pgfpathmoveto{\pgfqpoint{0.625000in}{1.158976in}}%
\pgfpathlineto{\pgfqpoint{0.627517in}{1.167544in}}%
\pgfpathlineto{\pgfqpoint{0.625131in}{1.177193in}}%
\pgfpathlineto{\pgfqpoint{0.625044in}{1.186842in}}%
\pgfpathlineto{\pgfqpoint{0.625000in}{1.187221in}}%
\pgfusepath{stroke}%
\end{pgfscope}%
\begin{pgfscope}%
\pgfpathrectangle{\pgfqpoint{0.625000in}{0.550000in}}{\pgfqpoint{3.875000in}{3.850000in}} %
\pgfusepath{clip}%
\pgfsetbuttcap%
\pgfsetroundjoin%
\pgfsetlinewidth{0.250937pt}%
\definecolor{currentstroke}{rgb}{0.000000,0.000000,0.000000}%
\pgfsetstrokecolor{currentstroke}%
\pgfsetdash{}{0pt}%
\pgfpathmoveto{\pgfqpoint{0.625000in}{1.203854in}}%
\pgfpathlineto{\pgfqpoint{0.625606in}{1.206140in}}%
\pgfpathlineto{\pgfqpoint{0.627320in}{1.215789in}}%
\pgfpathlineto{\pgfqpoint{0.628500in}{1.225439in}}%
\pgfpathlineto{\pgfqpoint{0.625962in}{1.235088in}}%
\pgfpathlineto{\pgfqpoint{0.625000in}{1.235126in}}%
\pgfusepath{stroke}%
\end{pgfscope}%
\begin{pgfscope}%
\pgfpathrectangle{\pgfqpoint{0.625000in}{0.550000in}}{\pgfqpoint{3.875000in}{3.850000in}} %
\pgfusepath{clip}%
\pgfsetbuttcap%
\pgfsetroundjoin%
\pgfsetlinewidth{0.250937pt}%
\definecolor{currentstroke}{rgb}{0.000000,0.000000,0.000000}%
\pgfsetstrokecolor{currentstroke}%
\pgfsetdash{}{0pt}%
\pgfpathmoveto{\pgfqpoint{0.625000in}{1.263367in}}%
\pgfpathlineto{\pgfqpoint{0.625598in}{1.264035in}}%
\pgfpathlineto{\pgfqpoint{0.631498in}{1.273684in}}%
\pgfpathlineto{\pgfqpoint{0.628158in}{1.283333in}}%
\pgfpathlineto{\pgfqpoint{0.625000in}{1.285106in}}%
\pgfusepath{stroke}%
\end{pgfscope}%
\begin{pgfscope}%
\pgfpathrectangle{\pgfqpoint{0.625000in}{0.550000in}}{\pgfqpoint{3.875000in}{3.850000in}} %
\pgfusepath{clip}%
\pgfsetbuttcap%
\pgfsetroundjoin%
\pgfsetlinewidth{0.250937pt}%
\definecolor{currentstroke}{rgb}{0.000000,0.000000,0.000000}%
\pgfsetstrokecolor{currentstroke}%
\pgfsetdash{}{0pt}%
\pgfpathmoveto{\pgfqpoint{0.625000in}{1.302354in}}%
\pgfpathlineto{\pgfqpoint{0.625596in}{1.302632in}}%
\pgfpathlineto{\pgfqpoint{0.629827in}{1.312281in}}%
\pgfpathlineto{\pgfqpoint{0.632698in}{1.321930in}}%
\pgfpathlineto{\pgfqpoint{0.630306in}{1.331579in}}%
\pgfpathlineto{\pgfqpoint{0.625586in}{1.341228in}}%
\pgfpathlineto{\pgfqpoint{0.627165in}{1.350877in}}%
\pgfpathlineto{\pgfqpoint{0.627701in}{1.360526in}}%
\pgfpathlineto{\pgfqpoint{0.626061in}{1.370175in}}%
\pgfpathlineto{\pgfqpoint{0.625175in}{1.379825in}}%
\pgfpathlineto{\pgfqpoint{0.627217in}{1.389474in}}%
\pgfpathlineto{\pgfqpoint{0.625000in}{1.389602in}}%
\pgfusepath{stroke}%
\end{pgfscope}%
\begin{pgfscope}%
\pgfpathrectangle{\pgfqpoint{0.625000in}{0.550000in}}{\pgfqpoint{3.875000in}{3.850000in}} %
\pgfusepath{clip}%
\pgfsetbuttcap%
\pgfsetroundjoin%
\pgfsetlinewidth{0.250937pt}%
\definecolor{currentstroke}{rgb}{0.000000,0.000000,0.000000}%
\pgfsetstrokecolor{currentstroke}%
\pgfsetdash{}{0pt}%
\pgfpathmoveto{\pgfqpoint{0.625000in}{1.408638in}}%
\pgfpathlineto{\pgfqpoint{0.628586in}{1.408772in}}%
\pgfpathlineto{\pgfqpoint{0.629570in}{1.418421in}}%
\pgfpathlineto{\pgfqpoint{0.625000in}{1.423597in}}%
\pgfusepath{stroke}%
\end{pgfscope}%
\begin{pgfscope}%
\pgfpathrectangle{\pgfqpoint{0.625000in}{0.550000in}}{\pgfqpoint{3.875000in}{3.850000in}} %
\pgfusepath{clip}%
\pgfsetbuttcap%
\pgfsetroundjoin%
\pgfsetlinewidth{0.250937pt}%
\definecolor{currentstroke}{rgb}{0.000000,0.000000,0.000000}%
\pgfsetstrokecolor{currentstroke}%
\pgfsetdash{}{0pt}%
\pgfpathmoveto{\pgfqpoint{0.625000in}{1.434341in}}%
\pgfpathlineto{\pgfqpoint{0.627982in}{1.437719in}}%
\pgfpathlineto{\pgfqpoint{0.626627in}{1.447368in}}%
\pgfpathlineto{\pgfqpoint{0.625572in}{1.457018in}}%
\pgfpathlineto{\pgfqpoint{0.628802in}{1.466667in}}%
\pgfpathlineto{\pgfqpoint{0.630167in}{1.476316in}}%
\pgfpathlineto{\pgfqpoint{0.630149in}{1.485965in}}%
\pgfpathlineto{\pgfqpoint{0.626269in}{1.495614in}}%
\pgfpathlineto{\pgfqpoint{0.628078in}{1.505263in}}%
\pgfpathlineto{\pgfqpoint{0.634712in}{1.511984in}}%
\pgfpathlineto{\pgfqpoint{0.636185in}{1.514912in}}%
\pgfpathlineto{\pgfqpoint{0.634712in}{1.517840in}}%
\pgfpathlineto{\pgfqpoint{0.630347in}{1.524561in}}%
\pgfpathlineto{\pgfqpoint{0.626269in}{1.534211in}}%
\pgfpathlineto{\pgfqpoint{0.625000in}{1.535421in}}%
\pgfusepath{stroke}%
\end{pgfscope}%
\begin{pgfscope}%
\pgfpathrectangle{\pgfqpoint{0.625000in}{0.550000in}}{\pgfqpoint{3.875000in}{3.850000in}} %
\pgfusepath{clip}%
\pgfsetbuttcap%
\pgfsetroundjoin%
\pgfsetlinewidth{0.250937pt}%
\definecolor{currentstroke}{rgb}{0.000000,0.000000,0.000000}%
\pgfsetstrokecolor{currentstroke}%
\pgfsetdash{}{0pt}%
\pgfpathmoveto{\pgfqpoint{0.625000in}{1.562971in}}%
\pgfpathlineto{\pgfqpoint{0.629223in}{1.563158in}}%
\pgfpathlineto{\pgfqpoint{0.625560in}{1.572807in}}%
\pgfpathlineto{\pgfqpoint{0.625000in}{1.573628in}}%
\pgfusepath{stroke}%
\end{pgfscope}%
\begin{pgfscope}%
\pgfpathrectangle{\pgfqpoint{0.625000in}{0.550000in}}{\pgfqpoint{3.875000in}{3.850000in}} %
\pgfusepath{clip}%
\pgfsetbuttcap%
\pgfsetroundjoin%
\pgfsetlinewidth{0.250937pt}%
\definecolor{currentstroke}{rgb}{0.000000,0.000000,0.000000}%
\pgfsetstrokecolor{currentstroke}%
\pgfsetdash{}{0pt}%
\pgfpathmoveto{\pgfqpoint{0.625000in}{1.588112in}}%
\pgfpathlineto{\pgfqpoint{0.627651in}{1.592105in}}%
\pgfpathlineto{\pgfqpoint{0.629230in}{1.601754in}}%
\pgfpathlineto{\pgfqpoint{0.629311in}{1.611404in}}%
\pgfpathlineto{\pgfqpoint{0.625000in}{1.616780in}}%
\pgfusepath{stroke}%
\end{pgfscope}%
\begin{pgfscope}%
\pgfpathrectangle{\pgfqpoint{0.625000in}{0.550000in}}{\pgfqpoint{3.875000in}{3.850000in}} %
\pgfusepath{clip}%
\pgfsetbuttcap%
\pgfsetroundjoin%
\pgfsetlinewidth{0.250937pt}%
\definecolor{currentstroke}{rgb}{0.000000,0.000000,0.000000}%
\pgfsetstrokecolor{currentstroke}%
\pgfsetdash{}{0pt}%
\pgfpathmoveto{\pgfqpoint{0.625000in}{1.625659in}}%
\pgfpathlineto{\pgfqpoint{0.628656in}{1.630702in}}%
\pgfpathlineto{\pgfqpoint{0.625000in}{1.634851in}}%
\pgfusepath{stroke}%
\end{pgfscope}%
\begin{pgfscope}%
\pgfpathrectangle{\pgfqpoint{0.625000in}{0.550000in}}{\pgfqpoint{3.875000in}{3.850000in}} %
\pgfusepath{clip}%
\pgfsetbuttcap%
\pgfsetroundjoin%
\pgfsetlinewidth{0.250937pt}%
\definecolor{currentstroke}{rgb}{0.000000,0.000000,0.000000}%
\pgfsetstrokecolor{currentstroke}%
\pgfsetdash{}{0pt}%
\pgfpathmoveto{\pgfqpoint{0.625000in}{1.649513in}}%
\pgfpathlineto{\pgfqpoint{0.625791in}{1.650000in}}%
\pgfpathlineto{\pgfqpoint{0.630151in}{1.659649in}}%
\pgfpathlineto{\pgfqpoint{0.628120in}{1.669298in}}%
\pgfpathlineto{\pgfqpoint{0.625000in}{1.675800in}}%
\pgfusepath{stroke}%
\end{pgfscope}%
\begin{pgfscope}%
\pgfpathrectangle{\pgfqpoint{0.625000in}{0.550000in}}{\pgfqpoint{3.875000in}{3.850000in}} %
\pgfusepath{clip}%
\pgfsetbuttcap%
\pgfsetroundjoin%
\pgfsetlinewidth{0.250937pt}%
\definecolor{currentstroke}{rgb}{0.000000,0.000000,0.000000}%
\pgfsetstrokecolor{currentstroke}%
\pgfsetdash{}{0pt}%
\pgfpathmoveto{\pgfqpoint{0.625000in}{1.686532in}}%
\pgfpathlineto{\pgfqpoint{0.625359in}{1.688596in}}%
\pgfpathlineto{\pgfqpoint{0.625985in}{1.698246in}}%
\pgfpathlineto{\pgfqpoint{0.625000in}{1.698357in}}%
\pgfusepath{stroke}%
\end{pgfscope}%
\begin{pgfscope}%
\pgfpathrectangle{\pgfqpoint{0.625000in}{0.550000in}}{\pgfqpoint{3.875000in}{3.850000in}} %
\pgfusepath{clip}%
\pgfsetbuttcap%
\pgfsetroundjoin%
\pgfsetlinewidth{0.250937pt}%
\definecolor{currentstroke}{rgb}{0.000000,0.000000,0.000000}%
\pgfsetstrokecolor{currentstroke}%
\pgfsetdash{}{0pt}%
\pgfpathmoveto{\pgfqpoint{0.625000in}{1.717373in}}%
\pgfpathlineto{\pgfqpoint{0.626445in}{1.717544in}}%
\pgfpathlineto{\pgfqpoint{0.625359in}{1.727193in}}%
\pgfpathlineto{\pgfqpoint{0.628263in}{1.736842in}}%
\pgfpathlineto{\pgfqpoint{0.628234in}{1.746491in}}%
\pgfpathlineto{\pgfqpoint{0.631568in}{1.756140in}}%
\pgfpathlineto{\pgfqpoint{0.625581in}{1.765789in}}%
\pgfpathlineto{\pgfqpoint{0.627320in}{1.775439in}}%
\pgfpathlineto{\pgfqpoint{0.629004in}{1.785088in}}%
\pgfpathlineto{\pgfqpoint{0.629646in}{1.794737in}}%
\pgfpathlineto{\pgfqpoint{0.630102in}{1.804386in}}%
\pgfpathlineto{\pgfqpoint{0.625000in}{1.812917in}}%
\pgfusepath{stroke}%
\end{pgfscope}%
\begin{pgfscope}%
\pgfpathrectangle{\pgfqpoint{0.625000in}{0.550000in}}{\pgfqpoint{3.875000in}{3.850000in}} %
\pgfusepath{clip}%
\pgfsetbuttcap%
\pgfsetroundjoin%
\pgfsetlinewidth{0.250937pt}%
\definecolor{currentstroke}{rgb}{0.000000,0.000000,0.000000}%
\pgfsetstrokecolor{currentstroke}%
\pgfsetdash{}{0pt}%
\pgfpathmoveto{\pgfqpoint{0.625000in}{1.816154in}}%
\pgfpathlineto{\pgfqpoint{0.629817in}{1.823684in}}%
\pgfpathlineto{\pgfqpoint{0.631725in}{1.833333in}}%
\pgfpathlineto{\pgfqpoint{0.626733in}{1.842982in}}%
\pgfpathlineto{\pgfqpoint{0.631115in}{1.852632in}}%
\pgfpathlineto{\pgfqpoint{0.625000in}{1.852898in}}%
\pgfusepath{stroke}%
\end{pgfscope}%
\begin{pgfscope}%
\pgfpathrectangle{\pgfqpoint{0.625000in}{0.550000in}}{\pgfqpoint{3.875000in}{3.850000in}} %
\pgfusepath{clip}%
\pgfsetbuttcap%
\pgfsetroundjoin%
\pgfsetlinewidth{0.250937pt}%
\definecolor{currentstroke}{rgb}{0.000000,0.000000,0.000000}%
\pgfsetstrokecolor{currentstroke}%
\pgfsetdash{}{0pt}%
\pgfpathmoveto{\pgfqpoint{0.625000in}{1.871862in}}%
\pgfpathlineto{\pgfqpoint{0.626972in}{1.871930in}}%
\pgfpathlineto{\pgfqpoint{0.625483in}{1.881579in}}%
\pgfpathlineto{\pgfqpoint{0.625000in}{1.881841in}}%
\pgfusepath{stroke}%
\end{pgfscope}%
\begin{pgfscope}%
\pgfpathrectangle{\pgfqpoint{0.625000in}{0.550000in}}{\pgfqpoint{3.875000in}{3.850000in}} %
\pgfusepath{clip}%
\pgfsetbuttcap%
\pgfsetroundjoin%
\pgfsetlinewidth{0.250937pt}%
\definecolor{currentstroke}{rgb}{0.000000,0.000000,0.000000}%
\pgfsetstrokecolor{currentstroke}%
\pgfsetdash{}{0pt}%
\pgfpathmoveto{\pgfqpoint{0.625000in}{1.897114in}}%
\pgfpathlineto{\pgfqpoint{0.630984in}{1.900877in}}%
\pgfpathlineto{\pgfqpoint{0.625000in}{1.907380in}}%
\pgfusepath{stroke}%
\end{pgfscope}%
\begin{pgfscope}%
\pgfpathrectangle{\pgfqpoint{0.625000in}{0.550000in}}{\pgfqpoint{3.875000in}{3.850000in}} %
\pgfusepath{clip}%
\pgfsetbuttcap%
\pgfsetroundjoin%
\pgfsetlinewidth{0.250937pt}%
\definecolor{currentstroke}{rgb}{0.000000,0.000000,0.000000}%
\pgfsetstrokecolor{currentstroke}%
\pgfsetdash{}{0pt}%
\pgfpathmoveto{\pgfqpoint{0.625000in}{1.919376in}}%
\pgfpathlineto{\pgfqpoint{0.625266in}{1.920175in}}%
\pgfpathlineto{\pgfqpoint{0.625000in}{1.920459in}}%
\pgfusepath{stroke}%
\end{pgfscope}%
\begin{pgfscope}%
\pgfpathrectangle{\pgfqpoint{0.625000in}{0.550000in}}{\pgfqpoint{3.875000in}{3.850000in}} %
\pgfusepath{clip}%
\pgfsetbuttcap%
\pgfsetroundjoin%
\pgfsetlinewidth{0.250937pt}%
\definecolor{currentstroke}{rgb}{0.000000,0.000000,0.000000}%
\pgfsetstrokecolor{currentstroke}%
\pgfsetdash{}{0pt}%
\pgfpathmoveto{\pgfqpoint{0.625000in}{1.936575in}}%
\pgfpathlineto{\pgfqpoint{0.628254in}{1.939474in}}%
\pgfpathlineto{\pgfqpoint{0.629159in}{1.949123in}}%
\pgfpathlineto{\pgfqpoint{0.625568in}{1.958772in}}%
\pgfpathlineto{\pgfqpoint{0.626974in}{1.968421in}}%
\pgfpathlineto{\pgfqpoint{0.628540in}{1.978070in}}%
\pgfpathlineto{\pgfqpoint{0.629160in}{1.987719in}}%
\pgfpathlineto{\pgfqpoint{0.633252in}{1.997368in}}%
\pgfpathlineto{\pgfqpoint{0.625000in}{2.004393in}}%
\pgfusepath{stroke}%
\end{pgfscope}%
\begin{pgfscope}%
\pgfpathrectangle{\pgfqpoint{0.625000in}{0.550000in}}{\pgfqpoint{3.875000in}{3.850000in}} %
\pgfusepath{clip}%
\pgfsetbuttcap%
\pgfsetroundjoin%
\pgfsetlinewidth{0.250937pt}%
\definecolor{currentstroke}{rgb}{0.000000,0.000000,0.000000}%
\pgfsetstrokecolor{currentstroke}%
\pgfsetdash{}{0pt}%
\pgfpathmoveto{\pgfqpoint{0.625000in}{2.035402in}}%
\pgfpathlineto{\pgfqpoint{0.625544in}{2.035965in}}%
\pgfpathlineto{\pgfqpoint{0.626559in}{2.045614in}}%
\pgfpathlineto{\pgfqpoint{0.626873in}{2.055263in}}%
\pgfpathlineto{\pgfqpoint{0.625000in}{2.064269in}}%
\pgfusepath{stroke}%
\end{pgfscope}%
\begin{pgfscope}%
\pgfpathrectangle{\pgfqpoint{0.625000in}{0.550000in}}{\pgfqpoint{3.875000in}{3.850000in}} %
\pgfusepath{clip}%
\pgfsetbuttcap%
\pgfsetroundjoin%
\pgfsetlinewidth{0.250937pt}%
\definecolor{currentstroke}{rgb}{0.000000,0.000000,0.000000}%
\pgfsetstrokecolor{currentstroke}%
\pgfsetdash{}{0pt}%
\pgfpathmoveto{\pgfqpoint{0.625000in}{2.068306in}}%
\pgfpathlineto{\pgfqpoint{0.625596in}{2.074561in}}%
\pgfpathlineto{\pgfqpoint{0.625519in}{2.084211in}}%
\pgfpathlineto{\pgfqpoint{0.632698in}{2.093860in}}%
\pgfpathlineto{\pgfqpoint{0.625605in}{2.103509in}}%
\pgfpathlineto{\pgfqpoint{0.625603in}{2.113158in}}%
\pgfpathlineto{\pgfqpoint{0.627625in}{2.122807in}}%
\pgfpathlineto{\pgfqpoint{0.627303in}{2.132456in}}%
\pgfpathlineto{\pgfqpoint{0.629026in}{2.142105in}}%
\pgfpathlineto{\pgfqpoint{0.625796in}{2.151754in}}%
\pgfpathlineto{\pgfqpoint{0.625000in}{2.154211in}}%
\pgfusepath{stroke}%
\end{pgfscope}%
\begin{pgfscope}%
\pgfpathrectangle{\pgfqpoint{0.625000in}{0.550000in}}{\pgfqpoint{3.875000in}{3.850000in}} %
\pgfusepath{clip}%
\pgfsetbuttcap%
\pgfsetroundjoin%
\pgfsetlinewidth{0.250937pt}%
\definecolor{currentstroke}{rgb}{0.000000,0.000000,0.000000}%
\pgfsetstrokecolor{currentstroke}%
\pgfsetdash{}{0pt}%
\pgfpathmoveto{\pgfqpoint{0.625000in}{2.180484in}}%
\pgfpathlineto{\pgfqpoint{0.628681in}{2.180702in}}%
\pgfpathlineto{\pgfqpoint{0.629789in}{2.190351in}}%
\pgfpathlineto{\pgfqpoint{0.627260in}{2.200000in}}%
\pgfpathlineto{\pgfqpoint{0.628714in}{2.209649in}}%
\pgfpathlineto{\pgfqpoint{0.627678in}{2.219298in}}%
\pgfpathlineto{\pgfqpoint{0.625584in}{2.228947in}}%
\pgfpathlineto{\pgfqpoint{0.631442in}{2.238596in}}%
\pgfpathlineto{\pgfqpoint{0.629653in}{2.248246in}}%
\pgfpathlineto{\pgfqpoint{0.626919in}{2.257895in}}%
\pgfpathlineto{\pgfqpoint{0.625668in}{2.267544in}}%
\pgfpathlineto{\pgfqpoint{0.625000in}{2.269601in}}%
\pgfusepath{stroke}%
\end{pgfscope}%
\begin{pgfscope}%
\pgfpathrectangle{\pgfqpoint{0.625000in}{0.550000in}}{\pgfqpoint{3.875000in}{3.850000in}} %
\pgfusepath{clip}%
\pgfsetbuttcap%
\pgfsetroundjoin%
\pgfsetlinewidth{0.250937pt}%
\definecolor{currentstroke}{rgb}{0.000000,0.000000,0.000000}%
\pgfsetstrokecolor{currentstroke}%
\pgfsetdash{}{0pt}%
\pgfpathmoveto{\pgfqpoint{0.625000in}{2.278532in}}%
\pgfpathlineto{\pgfqpoint{0.630967in}{2.286842in}}%
\pgfpathlineto{\pgfqpoint{0.625000in}{2.296128in}}%
\pgfusepath{stroke}%
\end{pgfscope}%
\begin{pgfscope}%
\pgfpathrectangle{\pgfqpoint{0.625000in}{0.550000in}}{\pgfqpoint{3.875000in}{3.850000in}} %
\pgfusepath{clip}%
\pgfsetbuttcap%
\pgfsetroundjoin%
\pgfsetlinewidth{0.250937pt}%
\definecolor{currentstroke}{rgb}{0.000000,0.000000,0.000000}%
\pgfsetstrokecolor{currentstroke}%
\pgfsetdash{}{0pt}%
\pgfpathmoveto{\pgfqpoint{0.625000in}{2.301048in}}%
\pgfpathlineto{\pgfqpoint{0.625586in}{2.306140in}}%
\pgfpathlineto{\pgfqpoint{0.625000in}{2.306561in}}%
\pgfusepath{stroke}%
\end{pgfscope}%
\begin{pgfscope}%
\pgfpathrectangle{\pgfqpoint{0.625000in}{0.550000in}}{\pgfqpoint{3.875000in}{3.850000in}} %
\pgfusepath{clip}%
\pgfsetbuttcap%
\pgfsetroundjoin%
\pgfsetlinewidth{0.250937pt}%
\definecolor{currentstroke}{rgb}{0.000000,0.000000,0.000000}%
\pgfsetstrokecolor{currentstroke}%
\pgfsetdash{}{0pt}%
\pgfpathmoveto{\pgfqpoint{0.625000in}{2.334838in}}%
\pgfpathlineto{\pgfqpoint{0.629279in}{2.335088in}}%
\pgfpathlineto{\pgfqpoint{0.625474in}{2.344737in}}%
\pgfpathlineto{\pgfqpoint{0.626895in}{2.354386in}}%
\pgfpathlineto{\pgfqpoint{0.627372in}{2.364035in}}%
\pgfpathlineto{\pgfqpoint{0.625000in}{2.369324in}}%
\pgfusepath{stroke}%
\end{pgfscope}%
\begin{pgfscope}%
\pgfpathrectangle{\pgfqpoint{0.625000in}{0.550000in}}{\pgfqpoint{3.875000in}{3.850000in}} %
\pgfusepath{clip}%
\pgfsetbuttcap%
\pgfsetroundjoin%
\pgfsetlinewidth{0.250937pt}%
\definecolor{currentstroke}{rgb}{0.000000,0.000000,0.000000}%
\pgfsetstrokecolor{currentstroke}%
\pgfsetdash{}{0pt}%
\pgfpathmoveto{\pgfqpoint{0.625000in}{2.376361in}}%
\pgfpathlineto{\pgfqpoint{0.628365in}{2.383333in}}%
\pgfpathlineto{\pgfqpoint{0.626757in}{2.392982in}}%
\pgfpathlineto{\pgfqpoint{0.627326in}{2.402632in}}%
\pgfpathlineto{\pgfqpoint{0.625000in}{2.407563in}}%
\pgfusepath{stroke}%
\end{pgfscope}%
\begin{pgfscope}%
\pgfpathrectangle{\pgfqpoint{0.625000in}{0.550000in}}{\pgfqpoint{3.875000in}{3.850000in}} %
\pgfusepath{clip}%
\pgfsetbuttcap%
\pgfsetroundjoin%
\pgfsetlinewidth{0.250937pt}%
\definecolor{currentstroke}{rgb}{0.000000,0.000000,0.000000}%
\pgfsetstrokecolor{currentstroke}%
\pgfsetdash{}{0pt}%
\pgfpathmoveto{\pgfqpoint{0.625000in}{2.421267in}}%
\pgfpathlineto{\pgfqpoint{0.625124in}{2.421930in}}%
\pgfpathlineto{\pgfqpoint{0.625384in}{2.431579in}}%
\pgfpathlineto{\pgfqpoint{0.625419in}{2.441228in}}%
\pgfpathlineto{\pgfqpoint{0.625325in}{2.450877in}}%
\pgfpathlineto{\pgfqpoint{0.625017in}{2.460526in}}%
\pgfpathlineto{\pgfqpoint{0.625000in}{2.460747in}}%
\pgfusepath{stroke}%
\end{pgfscope}%
\begin{pgfscope}%
\pgfpathrectangle{\pgfqpoint{0.625000in}{0.550000in}}{\pgfqpoint{3.875000in}{3.850000in}} %
\pgfusepath{clip}%
\pgfsetbuttcap%
\pgfsetroundjoin%
\pgfsetlinewidth{0.250937pt}%
\definecolor{currentstroke}{rgb}{0.000000,0.000000,0.000000}%
\pgfsetstrokecolor{currentstroke}%
\pgfsetdash{}{0pt}%
\pgfpathmoveto{\pgfqpoint{0.625000in}{2.498902in}}%
\pgfpathlineto{\pgfqpoint{0.625017in}{2.499123in}}%
\pgfpathlineto{\pgfqpoint{0.625325in}{2.508772in}}%
\pgfpathlineto{\pgfqpoint{0.625419in}{2.518421in}}%
\pgfpathlineto{\pgfqpoint{0.625384in}{2.528070in}}%
\pgfpathlineto{\pgfqpoint{0.625124in}{2.537719in}}%
\pgfpathlineto{\pgfqpoint{0.625000in}{2.538382in}}%
\pgfusepath{stroke}%
\end{pgfscope}%
\begin{pgfscope}%
\pgfpathrectangle{\pgfqpoint{0.625000in}{0.550000in}}{\pgfqpoint{3.875000in}{3.850000in}} %
\pgfusepath{clip}%
\pgfsetbuttcap%
\pgfsetroundjoin%
\pgfsetlinewidth{0.250937pt}%
\definecolor{currentstroke}{rgb}{0.000000,0.000000,0.000000}%
\pgfsetstrokecolor{currentstroke}%
\pgfsetdash{}{0pt}%
\pgfpathmoveto{\pgfqpoint{0.625000in}{2.552086in}}%
\pgfpathlineto{\pgfqpoint{0.627326in}{2.557018in}}%
\pgfpathlineto{\pgfqpoint{0.626757in}{2.566667in}}%
\pgfpathlineto{\pgfqpoint{0.628365in}{2.576316in}}%
\pgfpathlineto{\pgfqpoint{0.625000in}{2.583288in}}%
\pgfusepath{stroke}%
\end{pgfscope}%
\begin{pgfscope}%
\pgfpathrectangle{\pgfqpoint{0.625000in}{0.550000in}}{\pgfqpoint{3.875000in}{3.850000in}} %
\pgfusepath{clip}%
\pgfsetbuttcap%
\pgfsetroundjoin%
\pgfsetlinewidth{0.250937pt}%
\definecolor{currentstroke}{rgb}{0.000000,0.000000,0.000000}%
\pgfsetstrokecolor{currentstroke}%
\pgfsetdash{}{0pt}%
\pgfpathmoveto{\pgfqpoint{0.625000in}{2.590325in}}%
\pgfpathlineto{\pgfqpoint{0.627372in}{2.595614in}}%
\pgfpathlineto{\pgfqpoint{0.626895in}{2.605263in}}%
\pgfpathlineto{\pgfqpoint{0.625474in}{2.614912in}}%
\pgfpathlineto{\pgfqpoint{0.629279in}{2.624561in}}%
\pgfpathlineto{\pgfqpoint{0.625000in}{2.624812in}}%
\pgfusepath{stroke}%
\end{pgfscope}%
\begin{pgfscope}%
\pgfpathrectangle{\pgfqpoint{0.625000in}{0.550000in}}{\pgfqpoint{3.875000in}{3.850000in}} %
\pgfusepath{clip}%
\pgfsetbuttcap%
\pgfsetroundjoin%
\pgfsetlinewidth{0.250937pt}%
\definecolor{currentstroke}{rgb}{0.000000,0.000000,0.000000}%
\pgfsetstrokecolor{currentstroke}%
\pgfsetdash{}{0pt}%
\pgfpathmoveto{\pgfqpoint{0.625000in}{2.653088in}}%
\pgfpathlineto{\pgfqpoint{0.625586in}{2.653509in}}%
\pgfpathlineto{\pgfqpoint{0.625000in}{2.658601in}}%
\pgfusepath{stroke}%
\end{pgfscope}%
\begin{pgfscope}%
\pgfpathrectangle{\pgfqpoint{0.625000in}{0.550000in}}{\pgfqpoint{3.875000in}{3.850000in}} %
\pgfusepath{clip}%
\pgfsetbuttcap%
\pgfsetroundjoin%
\pgfsetlinewidth{0.250937pt}%
\definecolor{currentstroke}{rgb}{0.000000,0.000000,0.000000}%
\pgfsetstrokecolor{currentstroke}%
\pgfsetdash{}{0pt}%
\pgfpathmoveto{\pgfqpoint{0.625000in}{2.663521in}}%
\pgfpathlineto{\pgfqpoint{0.630967in}{2.672807in}}%
\pgfpathlineto{\pgfqpoint{0.625000in}{2.681117in}}%
\pgfusepath{stroke}%
\end{pgfscope}%
\begin{pgfscope}%
\pgfpathrectangle{\pgfqpoint{0.625000in}{0.550000in}}{\pgfqpoint{3.875000in}{3.850000in}} %
\pgfusepath{clip}%
\pgfsetbuttcap%
\pgfsetroundjoin%
\pgfsetlinewidth{0.250937pt}%
\definecolor{currentstroke}{rgb}{0.000000,0.000000,0.000000}%
\pgfsetstrokecolor{currentstroke}%
\pgfsetdash{}{0pt}%
\pgfpathmoveto{\pgfqpoint{0.625000in}{2.690048in}}%
\pgfpathlineto{\pgfqpoint{0.625668in}{2.692105in}}%
\pgfpathlineto{\pgfqpoint{0.626919in}{2.701754in}}%
\pgfpathlineto{\pgfqpoint{0.629653in}{2.711404in}}%
\pgfpathlineto{\pgfqpoint{0.631442in}{2.721053in}}%
\pgfpathlineto{\pgfqpoint{0.625584in}{2.730702in}}%
\pgfpathlineto{\pgfqpoint{0.627678in}{2.740351in}}%
\pgfpathlineto{\pgfqpoint{0.628714in}{2.750000in}}%
\pgfpathlineto{\pgfqpoint{0.627260in}{2.759649in}}%
\pgfpathlineto{\pgfqpoint{0.629789in}{2.769298in}}%
\pgfpathlineto{\pgfqpoint{0.628681in}{2.778947in}}%
\pgfpathlineto{\pgfqpoint{0.625000in}{2.779167in}}%
\pgfusepath{stroke}%
\end{pgfscope}%
\begin{pgfscope}%
\pgfpathrectangle{\pgfqpoint{0.625000in}{0.550000in}}{\pgfqpoint{3.875000in}{3.850000in}} %
\pgfusepath{clip}%
\pgfsetbuttcap%
\pgfsetroundjoin%
\pgfsetlinewidth{0.250937pt}%
\definecolor{currentstroke}{rgb}{0.000000,0.000000,0.000000}%
\pgfsetstrokecolor{currentstroke}%
\pgfsetdash{}{0pt}%
\pgfpathmoveto{\pgfqpoint{0.625000in}{2.805438in}}%
\pgfpathlineto{\pgfqpoint{0.625796in}{2.807895in}}%
\pgfpathlineto{\pgfqpoint{0.629026in}{2.817544in}}%
\pgfpathlineto{\pgfqpoint{0.627303in}{2.827193in}}%
\pgfpathlineto{\pgfqpoint{0.627625in}{2.836842in}}%
\pgfpathlineto{\pgfqpoint{0.625603in}{2.846491in}}%
\pgfpathlineto{\pgfqpoint{0.625605in}{2.856140in}}%
\pgfpathlineto{\pgfqpoint{0.632698in}{2.865789in}}%
\pgfpathlineto{\pgfqpoint{0.625519in}{2.875439in}}%
\pgfpathlineto{\pgfqpoint{0.625596in}{2.885088in}}%
\pgfpathlineto{\pgfqpoint{0.625000in}{2.891343in}}%
\pgfusepath{stroke}%
\end{pgfscope}%
\begin{pgfscope}%
\pgfpathrectangle{\pgfqpoint{0.625000in}{0.550000in}}{\pgfqpoint{3.875000in}{3.850000in}} %
\pgfusepath{clip}%
\pgfsetbuttcap%
\pgfsetroundjoin%
\pgfsetlinewidth{0.250937pt}%
\definecolor{currentstroke}{rgb}{0.000000,0.000000,0.000000}%
\pgfsetstrokecolor{currentstroke}%
\pgfsetdash{}{0pt}%
\pgfpathmoveto{\pgfqpoint{0.625000in}{2.895381in}}%
\pgfpathlineto{\pgfqpoint{0.626873in}{2.904386in}}%
\pgfpathlineto{\pgfqpoint{0.626559in}{2.914035in}}%
\pgfpathlineto{\pgfqpoint{0.625544in}{2.923684in}}%
\pgfpathlineto{\pgfqpoint{0.625000in}{2.924247in}}%
\pgfusepath{stroke}%
\end{pgfscope}%
\begin{pgfscope}%
\pgfpathrectangle{\pgfqpoint{0.625000in}{0.550000in}}{\pgfqpoint{3.875000in}{3.850000in}} %
\pgfusepath{clip}%
\pgfsetbuttcap%
\pgfsetroundjoin%
\pgfsetlinewidth{0.250937pt}%
\definecolor{currentstroke}{rgb}{0.000000,0.000000,0.000000}%
\pgfsetstrokecolor{currentstroke}%
\pgfsetdash{}{0pt}%
\pgfpathmoveto{\pgfqpoint{0.625000in}{2.955257in}}%
\pgfpathlineto{\pgfqpoint{0.633252in}{2.962281in}}%
\pgfpathlineto{\pgfqpoint{0.629160in}{2.971930in}}%
\pgfpathlineto{\pgfqpoint{0.628540in}{2.981579in}}%
\pgfpathlineto{\pgfqpoint{0.626974in}{2.991228in}}%
\pgfpathlineto{\pgfqpoint{0.625568in}{3.000877in}}%
\pgfpathlineto{\pgfqpoint{0.629159in}{3.010526in}}%
\pgfpathlineto{\pgfqpoint{0.628254in}{3.020175in}}%
\pgfpathlineto{\pgfqpoint{0.625000in}{3.023074in}}%
\pgfusepath{stroke}%
\end{pgfscope}%
\begin{pgfscope}%
\pgfpathrectangle{\pgfqpoint{0.625000in}{0.550000in}}{\pgfqpoint{3.875000in}{3.850000in}} %
\pgfusepath{clip}%
\pgfsetbuttcap%
\pgfsetroundjoin%
\pgfsetlinewidth{0.250937pt}%
\definecolor{currentstroke}{rgb}{0.000000,0.000000,0.000000}%
\pgfsetstrokecolor{currentstroke}%
\pgfsetdash{}{0pt}%
\pgfpathmoveto{\pgfqpoint{0.625000in}{3.039190in}}%
\pgfpathlineto{\pgfqpoint{0.625266in}{3.039474in}}%
\pgfpathlineto{\pgfqpoint{0.625000in}{3.040273in}}%
\pgfusepath{stroke}%
\end{pgfscope}%
\begin{pgfscope}%
\pgfpathrectangle{\pgfqpoint{0.625000in}{0.550000in}}{\pgfqpoint{3.875000in}{3.850000in}} %
\pgfusepath{clip}%
\pgfsetbuttcap%
\pgfsetroundjoin%
\pgfsetlinewidth{0.250937pt}%
\definecolor{currentstroke}{rgb}{0.000000,0.000000,0.000000}%
\pgfsetstrokecolor{currentstroke}%
\pgfsetdash{}{0pt}%
\pgfpathmoveto{\pgfqpoint{0.625000in}{3.052269in}}%
\pgfpathlineto{\pgfqpoint{0.630984in}{3.058772in}}%
\pgfpathlineto{\pgfqpoint{0.625000in}{3.062535in}}%
\pgfusepath{stroke}%
\end{pgfscope}%
\begin{pgfscope}%
\pgfpathrectangle{\pgfqpoint{0.625000in}{0.550000in}}{\pgfqpoint{3.875000in}{3.850000in}} %
\pgfusepath{clip}%
\pgfsetbuttcap%
\pgfsetroundjoin%
\pgfsetlinewidth{0.250937pt}%
\definecolor{currentstroke}{rgb}{0.000000,0.000000,0.000000}%
\pgfsetstrokecolor{currentstroke}%
\pgfsetdash{}{0pt}%
\pgfpathmoveto{\pgfqpoint{0.625000in}{3.077808in}}%
\pgfpathlineto{\pgfqpoint{0.625483in}{3.078070in}}%
\pgfpathlineto{\pgfqpoint{0.626972in}{3.087719in}}%
\pgfpathlineto{\pgfqpoint{0.625000in}{3.087789in}}%
\pgfusepath{stroke}%
\end{pgfscope}%
\begin{pgfscope}%
\pgfpathrectangle{\pgfqpoint{0.625000in}{0.550000in}}{\pgfqpoint{3.875000in}{3.850000in}} %
\pgfusepath{clip}%
\pgfsetbuttcap%
\pgfsetroundjoin%
\pgfsetlinewidth{0.250937pt}%
\definecolor{currentstroke}{rgb}{0.000000,0.000000,0.000000}%
\pgfsetstrokecolor{currentstroke}%
\pgfsetdash{}{0pt}%
\pgfpathmoveto{\pgfqpoint{0.625000in}{3.106748in}}%
\pgfpathlineto{\pgfqpoint{0.631115in}{3.107018in}}%
\pgfpathlineto{\pgfqpoint{0.626733in}{3.116667in}}%
\pgfpathlineto{\pgfqpoint{0.631725in}{3.126316in}}%
\pgfpathlineto{\pgfqpoint{0.629817in}{3.135965in}}%
\pgfpathlineto{\pgfqpoint{0.625000in}{3.143495in}}%
\pgfusepath{stroke}%
\end{pgfscope}%
\begin{pgfscope}%
\pgfpathrectangle{\pgfqpoint{0.625000in}{0.550000in}}{\pgfqpoint{3.875000in}{3.850000in}} %
\pgfusepath{clip}%
\pgfsetbuttcap%
\pgfsetroundjoin%
\pgfsetlinewidth{0.250937pt}%
\definecolor{currentstroke}{rgb}{0.000000,0.000000,0.000000}%
\pgfsetstrokecolor{currentstroke}%
\pgfsetdash{}{0pt}%
\pgfpathmoveto{\pgfqpoint{0.625000in}{3.146732in}}%
\pgfpathlineto{\pgfqpoint{0.630102in}{3.155263in}}%
\pgfpathlineto{\pgfqpoint{0.629646in}{3.164912in}}%
\pgfpathlineto{\pgfqpoint{0.629004in}{3.174561in}}%
\pgfpathlineto{\pgfqpoint{0.627320in}{3.184211in}}%
\pgfpathlineto{\pgfqpoint{0.625581in}{3.193860in}}%
\pgfpathlineto{\pgfqpoint{0.631568in}{3.203509in}}%
\pgfpathlineto{\pgfqpoint{0.628234in}{3.213158in}}%
\pgfpathlineto{\pgfqpoint{0.628263in}{3.222807in}}%
\pgfpathlineto{\pgfqpoint{0.625359in}{3.232456in}}%
\pgfpathlineto{\pgfqpoint{0.626445in}{3.242105in}}%
\pgfpathlineto{\pgfqpoint{0.625000in}{3.242285in}}%
\pgfusepath{stroke}%
\end{pgfscope}%
\begin{pgfscope}%
\pgfpathrectangle{\pgfqpoint{0.625000in}{0.550000in}}{\pgfqpoint{3.875000in}{3.850000in}} %
\pgfusepath{clip}%
\pgfsetbuttcap%
\pgfsetroundjoin%
\pgfsetlinewidth{0.250937pt}%
\definecolor{currentstroke}{rgb}{0.000000,0.000000,0.000000}%
\pgfsetstrokecolor{currentstroke}%
\pgfsetdash{}{0pt}%
\pgfpathmoveto{\pgfqpoint{0.625000in}{3.261287in}}%
\pgfpathlineto{\pgfqpoint{0.625985in}{3.261404in}}%
\pgfpathlineto{\pgfqpoint{0.625359in}{3.271053in}}%
\pgfpathlineto{\pgfqpoint{0.625000in}{3.273117in}}%
\pgfusepath{stroke}%
\end{pgfscope}%
\begin{pgfscope}%
\pgfpathrectangle{\pgfqpoint{0.625000in}{0.550000in}}{\pgfqpoint{3.875000in}{3.850000in}} %
\pgfusepath{clip}%
\pgfsetbuttcap%
\pgfsetroundjoin%
\pgfsetlinewidth{0.250937pt}%
\definecolor{currentstroke}{rgb}{0.000000,0.000000,0.000000}%
\pgfsetstrokecolor{currentstroke}%
\pgfsetdash{}{0pt}%
\pgfpathmoveto{\pgfqpoint{0.625000in}{3.283849in}}%
\pgfpathlineto{\pgfqpoint{0.628120in}{3.290351in}}%
\pgfpathlineto{\pgfqpoint{0.630151in}{3.300000in}}%
\pgfpathlineto{\pgfqpoint{0.625791in}{3.309649in}}%
\pgfpathlineto{\pgfqpoint{0.625000in}{3.310136in}}%
\pgfusepath{stroke}%
\end{pgfscope}%
\begin{pgfscope}%
\pgfpathrectangle{\pgfqpoint{0.625000in}{0.550000in}}{\pgfqpoint{3.875000in}{3.850000in}} %
\pgfusepath{clip}%
\pgfsetbuttcap%
\pgfsetroundjoin%
\pgfsetlinewidth{0.250937pt}%
\definecolor{currentstroke}{rgb}{0.000000,0.000000,0.000000}%
\pgfsetstrokecolor{currentstroke}%
\pgfsetdash{}{0pt}%
\pgfpathmoveto{\pgfqpoint{0.625000in}{3.324798in}}%
\pgfpathlineto{\pgfqpoint{0.628656in}{3.328947in}}%
\pgfpathlineto{\pgfqpoint{0.625000in}{3.333990in}}%
\pgfusepath{stroke}%
\end{pgfscope}%
\begin{pgfscope}%
\pgfpathrectangle{\pgfqpoint{0.625000in}{0.550000in}}{\pgfqpoint{3.875000in}{3.850000in}} %
\pgfusepath{clip}%
\pgfsetbuttcap%
\pgfsetroundjoin%
\pgfsetlinewidth{0.250937pt}%
\definecolor{currentstroke}{rgb}{0.000000,0.000000,0.000000}%
\pgfsetstrokecolor{currentstroke}%
\pgfsetdash{}{0pt}%
\pgfpathmoveto{\pgfqpoint{0.625000in}{3.342869in}}%
\pgfpathlineto{\pgfqpoint{0.629311in}{3.348246in}}%
\pgfpathlineto{\pgfqpoint{0.629230in}{3.357895in}}%
\pgfpathlineto{\pgfqpoint{0.627651in}{3.367544in}}%
\pgfpathlineto{\pgfqpoint{0.625000in}{3.371537in}}%
\pgfusepath{stroke}%
\end{pgfscope}%
\begin{pgfscope}%
\pgfpathrectangle{\pgfqpoint{0.625000in}{0.550000in}}{\pgfqpoint{3.875000in}{3.850000in}} %
\pgfusepath{clip}%
\pgfsetbuttcap%
\pgfsetroundjoin%
\pgfsetlinewidth{0.250937pt}%
\definecolor{currentstroke}{rgb}{0.000000,0.000000,0.000000}%
\pgfsetstrokecolor{currentstroke}%
\pgfsetdash{}{0pt}%
\pgfpathmoveto{\pgfqpoint{0.625000in}{3.386021in}}%
\pgfpathlineto{\pgfqpoint{0.625560in}{3.386842in}}%
\pgfpathlineto{\pgfqpoint{0.629223in}{3.396491in}}%
\pgfpathlineto{\pgfqpoint{0.625000in}{3.396684in}}%
\pgfusepath{stroke}%
\end{pgfscope}%
\begin{pgfscope}%
\pgfpathrectangle{\pgfqpoint{0.625000in}{0.550000in}}{\pgfqpoint{3.875000in}{3.850000in}} %
\pgfusepath{clip}%
\pgfsetbuttcap%
\pgfsetroundjoin%
\pgfsetlinewidth{0.250937pt}%
\definecolor{currentstroke}{rgb}{0.000000,0.000000,0.000000}%
\pgfsetstrokecolor{currentstroke}%
\pgfsetdash{}{0pt}%
\pgfpathmoveto{\pgfqpoint{0.625000in}{3.424228in}}%
\pgfpathlineto{\pgfqpoint{0.626269in}{3.425439in}}%
\pgfpathlineto{\pgfqpoint{0.630347in}{3.435088in}}%
\pgfpathlineto{\pgfqpoint{0.634712in}{3.441809in}}%
\pgfpathlineto{\pgfqpoint{0.636185in}{3.444737in}}%
\pgfpathlineto{\pgfqpoint{0.634712in}{3.447665in}}%
\pgfpathlineto{\pgfqpoint{0.628078in}{3.454386in}}%
\pgfpathlineto{\pgfqpoint{0.626269in}{3.464035in}}%
\pgfpathlineto{\pgfqpoint{0.630149in}{3.473684in}}%
\pgfpathlineto{\pgfqpoint{0.630167in}{3.483333in}}%
\pgfpathlineto{\pgfqpoint{0.628802in}{3.492982in}}%
\pgfpathlineto{\pgfqpoint{0.625572in}{3.502632in}}%
\pgfpathlineto{\pgfqpoint{0.626627in}{3.512281in}}%
\pgfpathlineto{\pgfqpoint{0.627982in}{3.521930in}}%
\pgfpathlineto{\pgfqpoint{0.625000in}{3.525308in}}%
\pgfusepath{stroke}%
\end{pgfscope}%
\begin{pgfscope}%
\pgfpathrectangle{\pgfqpoint{0.625000in}{0.550000in}}{\pgfqpoint{3.875000in}{3.850000in}} %
\pgfusepath{clip}%
\pgfsetbuttcap%
\pgfsetroundjoin%
\pgfsetlinewidth{0.250937pt}%
\definecolor{currentstroke}{rgb}{0.000000,0.000000,0.000000}%
\pgfsetstrokecolor{currentstroke}%
\pgfsetdash{}{0pt}%
\pgfpathmoveto{\pgfqpoint{0.625000in}{3.536052in}}%
\pgfpathlineto{\pgfqpoint{0.629570in}{3.541228in}}%
\pgfpathlineto{\pgfqpoint{0.628586in}{3.550877in}}%
\pgfpathlineto{\pgfqpoint{0.625000in}{3.551016in}}%
\pgfusepath{stroke}%
\end{pgfscope}%
\begin{pgfscope}%
\pgfpathrectangle{\pgfqpoint{0.625000in}{0.550000in}}{\pgfqpoint{3.875000in}{3.850000in}} %
\pgfusepath{clip}%
\pgfsetbuttcap%
\pgfsetroundjoin%
\pgfsetlinewidth{0.250937pt}%
\definecolor{currentstroke}{rgb}{0.000000,0.000000,0.000000}%
\pgfsetstrokecolor{currentstroke}%
\pgfsetdash{}{0pt}%
\pgfpathmoveto{\pgfqpoint{0.625000in}{3.570042in}}%
\pgfpathlineto{\pgfqpoint{0.627217in}{3.570175in}}%
\pgfpathlineto{\pgfqpoint{0.625175in}{3.579825in}}%
\pgfpathlineto{\pgfqpoint{0.626061in}{3.589474in}}%
\pgfpathlineto{\pgfqpoint{0.627701in}{3.599123in}}%
\pgfpathlineto{\pgfqpoint{0.627165in}{3.608772in}}%
\pgfpathlineto{\pgfqpoint{0.625586in}{3.618421in}}%
\pgfpathlineto{\pgfqpoint{0.630306in}{3.628070in}}%
\pgfpathlineto{\pgfqpoint{0.632698in}{3.637719in}}%
\pgfpathlineto{\pgfqpoint{0.629827in}{3.647368in}}%
\pgfpathlineto{\pgfqpoint{0.625596in}{3.657018in}}%
\pgfpathlineto{\pgfqpoint{0.625000in}{3.657295in}}%
\pgfusepath{stroke}%
\end{pgfscope}%
\begin{pgfscope}%
\pgfpathrectangle{\pgfqpoint{0.625000in}{0.550000in}}{\pgfqpoint{3.875000in}{3.850000in}} %
\pgfusepath{clip}%
\pgfsetbuttcap%
\pgfsetroundjoin%
\pgfsetlinewidth{0.250937pt}%
\definecolor{currentstroke}{rgb}{0.000000,0.000000,0.000000}%
\pgfsetstrokecolor{currentstroke}%
\pgfsetdash{}{0pt}%
\pgfpathmoveto{\pgfqpoint{0.625000in}{3.674543in}}%
\pgfpathlineto{\pgfqpoint{0.628158in}{3.676316in}}%
\pgfpathlineto{\pgfqpoint{0.631498in}{3.685965in}}%
\pgfpathlineto{\pgfqpoint{0.625598in}{3.695614in}}%
\pgfpathlineto{\pgfqpoint{0.625000in}{3.696282in}}%
\pgfusepath{stroke}%
\end{pgfscope}%
\begin{pgfscope}%
\pgfpathrectangle{\pgfqpoint{0.625000in}{0.550000in}}{\pgfqpoint{3.875000in}{3.850000in}} %
\pgfusepath{clip}%
\pgfsetbuttcap%
\pgfsetroundjoin%
\pgfsetlinewidth{0.250937pt}%
\definecolor{currentstroke}{rgb}{0.000000,0.000000,0.000000}%
\pgfsetstrokecolor{currentstroke}%
\pgfsetdash{}{0pt}%
\pgfpathmoveto{\pgfqpoint{0.625000in}{3.724521in}}%
\pgfpathlineto{\pgfqpoint{0.625962in}{3.724561in}}%
\pgfpathlineto{\pgfqpoint{0.628500in}{3.734211in}}%
\pgfpathlineto{\pgfqpoint{0.627320in}{3.743860in}}%
\pgfpathlineto{\pgfqpoint{0.625606in}{3.753509in}}%
\pgfpathlineto{\pgfqpoint{0.625000in}{3.755796in}}%
\pgfusepath{stroke}%
\end{pgfscope}%
\begin{pgfscope}%
\pgfpathrectangle{\pgfqpoint{0.625000in}{0.550000in}}{\pgfqpoint{3.875000in}{3.850000in}} %
\pgfusepath{clip}%
\pgfsetbuttcap%
\pgfsetroundjoin%
\pgfsetlinewidth{0.250937pt}%
\definecolor{currentstroke}{rgb}{0.000000,0.000000,0.000000}%
\pgfsetstrokecolor{currentstroke}%
\pgfsetdash{}{0pt}%
\pgfpathmoveto{\pgfqpoint{0.625000in}{3.772428in}}%
\pgfpathlineto{\pgfqpoint{0.625044in}{3.772807in}}%
\pgfpathlineto{\pgfqpoint{0.625131in}{3.782456in}}%
\pgfpathlineto{\pgfqpoint{0.627517in}{3.792105in}}%
\pgfpathlineto{\pgfqpoint{0.625000in}{3.800673in}}%
\pgfusepath{stroke}%
\end{pgfscope}%
\begin{pgfscope}%
\pgfpathrectangle{\pgfqpoint{0.625000in}{0.550000in}}{\pgfqpoint{3.875000in}{3.850000in}} %
\pgfusepath{clip}%
\pgfsetbuttcap%
\pgfsetroundjoin%
\pgfsetlinewidth{0.250937pt}%
\definecolor{currentstroke}{rgb}{0.000000,0.000000,0.000000}%
\pgfsetstrokecolor{currentstroke}%
\pgfsetdash{}{0pt}%
\pgfpathmoveto{\pgfqpoint{0.625000in}{3.807945in}}%
\pgfpathlineto{\pgfqpoint{0.625483in}{3.811404in}}%
\pgfpathlineto{\pgfqpoint{0.628348in}{3.821053in}}%
\pgfpathlineto{\pgfqpoint{0.630971in}{3.830702in}}%
\pgfpathlineto{\pgfqpoint{0.628943in}{3.840351in}}%
\pgfpathlineto{\pgfqpoint{0.625668in}{3.850000in}}%
\pgfpathlineto{\pgfqpoint{0.628145in}{3.859649in}}%
\pgfpathlineto{\pgfqpoint{0.625000in}{3.859746in}}%
\pgfusepath{stroke}%
\end{pgfscope}%
\begin{pgfscope}%
\pgfpathrectangle{\pgfqpoint{0.625000in}{0.550000in}}{\pgfqpoint{3.875000in}{3.850000in}} %
\pgfusepath{clip}%
\pgfsetbuttcap%
\pgfsetroundjoin%
\pgfsetlinewidth{0.250937pt}%
\definecolor{currentstroke}{rgb}{0.000000,0.000000,0.000000}%
\pgfsetstrokecolor{currentstroke}%
\pgfsetdash{}{0pt}%
\pgfpathmoveto{\pgfqpoint{0.625000in}{3.878812in}}%
\pgfpathlineto{\pgfqpoint{0.627343in}{3.878947in}}%
\pgfpathlineto{\pgfqpoint{0.625417in}{3.888596in}}%
\pgfpathlineto{\pgfqpoint{0.628058in}{3.898246in}}%
\pgfpathlineto{\pgfqpoint{0.628537in}{3.907895in}}%
\pgfpathlineto{\pgfqpoint{0.625000in}{3.912707in}}%
\pgfusepath{stroke}%
\end{pgfscope}%
\begin{pgfscope}%
\pgfpathrectangle{\pgfqpoint{0.625000in}{0.550000in}}{\pgfqpoint{3.875000in}{3.850000in}} %
\pgfusepath{clip}%
\pgfsetbuttcap%
\pgfsetroundjoin%
\pgfsetlinewidth{0.250937pt}%
\definecolor{currentstroke}{rgb}{0.000000,0.000000,0.000000}%
\pgfsetstrokecolor{currentstroke}%
\pgfsetdash{}{0pt}%
\pgfpathmoveto{\pgfqpoint{0.625000in}{3.919420in}}%
\pgfpathlineto{\pgfqpoint{0.633252in}{3.927193in}}%
\pgfpathlineto{\pgfqpoint{0.629407in}{3.936842in}}%
\pgfpathlineto{\pgfqpoint{0.630020in}{3.946491in}}%
\pgfpathlineto{\pgfqpoint{0.629097in}{3.956140in}}%
\pgfpathlineto{\pgfqpoint{0.625441in}{3.965789in}}%
\pgfpathlineto{\pgfqpoint{0.629639in}{3.975439in}}%
\pgfpathlineto{\pgfqpoint{0.628386in}{3.985088in}}%
\pgfpathlineto{\pgfqpoint{0.625000in}{3.992887in}}%
\pgfusepath{stroke}%
\end{pgfscope}%
\begin{pgfscope}%
\pgfpathrectangle{\pgfqpoint{0.625000in}{0.550000in}}{\pgfqpoint{3.875000in}{3.850000in}} %
\pgfusepath{clip}%
\pgfsetbuttcap%
\pgfsetroundjoin%
\pgfsetlinewidth{0.250937pt}%
\definecolor{currentstroke}{rgb}{0.000000,0.000000,0.000000}%
\pgfsetstrokecolor{currentstroke}%
\pgfsetdash{}{0pt}%
\pgfpathmoveto{\pgfqpoint{0.625000in}{4.000940in}}%
\pgfpathlineto{\pgfqpoint{0.625359in}{4.004386in}}%
\pgfpathlineto{\pgfqpoint{0.626298in}{4.014035in}}%
\pgfpathlineto{\pgfqpoint{0.625000in}{4.014191in}}%
\pgfusepath{stroke}%
\end{pgfscope}%
\begin{pgfscope}%
\pgfpathrectangle{\pgfqpoint{0.625000in}{0.550000in}}{\pgfqpoint{3.875000in}{3.850000in}} %
\pgfusepath{clip}%
\pgfsetbuttcap%
\pgfsetroundjoin%
\pgfsetlinewidth{0.250937pt}%
\definecolor{currentstroke}{rgb}{0.000000,0.000000,0.000000}%
\pgfsetstrokecolor{currentstroke}%
\pgfsetdash{}{0pt}%
\pgfpathmoveto{\pgfqpoint{0.625000in}{4.033271in}}%
\pgfpathlineto{\pgfqpoint{0.625562in}{4.033333in}}%
\pgfpathlineto{\pgfqpoint{0.625359in}{4.042982in}}%
\pgfpathlineto{\pgfqpoint{0.625000in}{4.043314in}}%
\pgfusepath{stroke}%
\end{pgfscope}%
\begin{pgfscope}%
\pgfpathrectangle{\pgfqpoint{0.625000in}{0.550000in}}{\pgfqpoint{3.875000in}{3.850000in}} %
\pgfusepath{clip}%
\pgfsetbuttcap%
\pgfsetroundjoin%
\pgfsetlinewidth{0.250937pt}%
\definecolor{currentstroke}{rgb}{0.000000,0.000000,0.000000}%
\pgfsetstrokecolor{currentstroke}%
\pgfsetdash{}{0pt}%
\pgfpathmoveto{\pgfqpoint{0.625000in}{4.060226in}}%
\pgfpathlineto{\pgfqpoint{0.628204in}{4.062281in}}%
\pgfpathlineto{\pgfqpoint{0.628909in}{4.071930in}}%
\pgfpathlineto{\pgfqpoint{0.625796in}{4.081579in}}%
\pgfpathlineto{\pgfqpoint{0.627883in}{4.091228in}}%
\pgfpathlineto{\pgfqpoint{0.629835in}{4.100877in}}%
\pgfpathlineto{\pgfqpoint{0.628137in}{4.110526in}}%
\pgfpathlineto{\pgfqpoint{0.629022in}{4.120175in}}%
\pgfpathlineto{\pgfqpoint{0.627175in}{4.129825in}}%
\pgfpathlineto{\pgfqpoint{0.626401in}{4.139474in}}%
\pgfpathlineto{\pgfqpoint{0.625000in}{4.144955in}}%
\pgfusepath{stroke}%
\end{pgfscope}%
\begin{pgfscope}%
\pgfpathrectangle{\pgfqpoint{0.625000in}{0.550000in}}{\pgfqpoint{3.875000in}{3.850000in}} %
\pgfusepath{clip}%
\pgfsetbuttcap%
\pgfsetroundjoin%
\pgfsetlinewidth{0.250937pt}%
\definecolor{currentstroke}{rgb}{0.000000,0.000000,0.000000}%
\pgfsetstrokecolor{currentstroke}%
\pgfsetdash{}{0pt}%
\pgfpathmoveto{\pgfqpoint{0.625000in}{4.157347in}}%
\pgfpathlineto{\pgfqpoint{0.625595in}{4.158772in}}%
\pgfpathlineto{\pgfqpoint{0.631613in}{4.168421in}}%
\pgfpathlineto{\pgfqpoint{0.625000in}{4.168883in}}%
\pgfusepath{stroke}%
\end{pgfscope}%
\begin{pgfscope}%
\pgfpathrectangle{\pgfqpoint{0.625000in}{0.550000in}}{\pgfqpoint{3.875000in}{3.850000in}} %
\pgfusepath{clip}%
\pgfsetbuttcap%
\pgfsetroundjoin%
\pgfsetlinewidth{0.250937pt}%
\definecolor{currentstroke}{rgb}{0.000000,0.000000,0.000000}%
\pgfsetstrokecolor{currentstroke}%
\pgfsetdash{}{0pt}%
\pgfpathmoveto{\pgfqpoint{0.625000in}{4.187709in}}%
\pgfpathlineto{\pgfqpoint{0.625215in}{4.187719in}}%
\pgfpathlineto{\pgfqpoint{0.625266in}{4.197368in}}%
\pgfpathlineto{\pgfqpoint{0.629401in}{4.207018in}}%
\pgfpathlineto{\pgfqpoint{0.630988in}{4.216667in}}%
\pgfpathlineto{\pgfqpoint{0.629335in}{4.226316in}}%
\pgfpathlineto{\pgfqpoint{0.625359in}{4.235965in}}%
\pgfpathlineto{\pgfqpoint{0.628020in}{4.245614in}}%
\pgfpathlineto{\pgfqpoint{0.629275in}{4.255263in}}%
\pgfpathlineto{\pgfqpoint{0.627083in}{4.264912in}}%
\pgfpathlineto{\pgfqpoint{0.625561in}{4.274561in}}%
\pgfpathlineto{\pgfqpoint{0.625000in}{4.276303in}}%
\pgfusepath{stroke}%
\end{pgfscope}%
\begin{pgfscope}%
\pgfpathrectangle{\pgfqpoint{0.625000in}{0.550000in}}{\pgfqpoint{3.875000in}{3.850000in}} %
\pgfusepath{clip}%
\pgfsetbuttcap%
\pgfsetroundjoin%
\pgfsetlinewidth{0.250937pt}%
\definecolor{currentstroke}{rgb}{0.000000,0.000000,0.000000}%
\pgfsetstrokecolor{currentstroke}%
\pgfsetdash{}{0pt}%
\pgfpathmoveto{\pgfqpoint{0.625000in}{4.287822in}}%
\pgfpathlineto{\pgfqpoint{0.628106in}{4.293860in}}%
\pgfpathlineto{\pgfqpoint{0.627979in}{4.303509in}}%
\pgfpathlineto{\pgfqpoint{0.630176in}{4.313158in}}%
\pgfpathlineto{\pgfqpoint{0.626422in}{4.322807in}}%
\pgfpathlineto{\pgfqpoint{0.625000in}{4.322849in}}%
\pgfusepath{stroke}%
\end{pgfscope}%
\begin{pgfscope}%
\pgfpathrectangle{\pgfqpoint{0.625000in}{0.550000in}}{\pgfqpoint{3.875000in}{3.850000in}} %
\pgfusepath{clip}%
\pgfsetbuttcap%
\pgfsetroundjoin%
\pgfsetlinewidth{0.250937pt}%
\definecolor{currentstroke}{rgb}{0.000000,0.000000,0.000000}%
\pgfsetstrokecolor{currentstroke}%
\pgfsetdash{}{0pt}%
\pgfpathmoveto{\pgfqpoint{0.625000in}{4.342031in}}%
\pgfpathlineto{\pgfqpoint{0.627615in}{4.342105in}}%
\pgfpathlineto{\pgfqpoint{0.625918in}{4.351754in}}%
\pgfpathlineto{\pgfqpoint{0.631181in}{4.361404in}}%
\pgfpathlineto{\pgfqpoint{0.630473in}{4.371053in}}%
\pgfpathlineto{\pgfqpoint{0.625000in}{4.379302in}}%
\pgfusepath{stroke}%
\end{pgfscope}%
\begin{pgfscope}%
\pgfpathrectangle{\pgfqpoint{0.625000in}{0.550000in}}{\pgfqpoint{3.875000in}{3.850000in}} %
\pgfusepath{clip}%
\pgfsetbuttcap%
\pgfsetroundjoin%
\pgfsetlinewidth{0.250937pt}%
\definecolor{currentstroke}{rgb}{0.000000,0.000000,0.000000}%
\pgfsetstrokecolor{currentstroke}%
\pgfsetdash{}{0pt}%
\pgfpathmoveto{\pgfqpoint{0.625000in}{4.386134in}}%
\pgfpathlineto{\pgfqpoint{0.634155in}{4.390351in}}%
\pgfpathlineto{\pgfqpoint{0.634712in}{4.390486in}}%
\pgfpathlineto{\pgfqpoint{0.644288in}{4.400000in}}%
\pgfusepath{stroke}%
\end{pgfscope}%
\begin{pgfscope}%
\pgfpathrectangle{\pgfqpoint{0.625000in}{0.550000in}}{\pgfqpoint{3.875000in}{3.850000in}} %
\pgfusepath{clip}%
\pgfsetbuttcap%
\pgfsetroundjoin%
\pgfsetlinewidth{0.250937pt}%
\definecolor{currentstroke}{rgb}{0.000000,0.000000,0.000000}%
\pgfsetstrokecolor{currentstroke}%
\pgfsetdash{}{0pt}%
\pgfpathmoveto{\pgfqpoint{0.638892in}{0.550000in}}%
\pgfpathlineto{\pgfqpoint{0.634712in}{0.558306in}}%
\pgfpathlineto{\pgfqpoint{0.633658in}{0.559649in}}%
\pgfpathlineto{\pgfqpoint{0.625000in}{0.564485in}}%
\pgfusepath{stroke}%
\end{pgfscope}%
\begin{pgfscope}%
\pgfpathrectangle{\pgfqpoint{0.625000in}{0.550000in}}{\pgfqpoint{3.875000in}{3.850000in}} %
\pgfusepath{clip}%
\pgfsetbuttcap%
\pgfsetroundjoin%
\pgfsetlinewidth{0.250937pt}%
\definecolor{currentstroke}{rgb}{0.000000,0.000000,0.000000}%
\pgfsetstrokecolor{currentstroke}%
\pgfsetdash{}{0pt}%
\pgfpathmoveto{\pgfqpoint{0.625000in}{0.585514in}}%
\pgfpathlineto{\pgfqpoint{0.627045in}{0.588596in}}%
\pgfpathlineto{\pgfqpoint{0.629056in}{0.598246in}}%
\pgfpathlineto{\pgfqpoint{0.625000in}{0.605071in}}%
\pgfusepath{stroke}%
\end{pgfscope}%
\begin{pgfscope}%
\pgfpathrectangle{\pgfqpoint{0.625000in}{0.550000in}}{\pgfqpoint{3.875000in}{3.850000in}} %
\pgfusepath{clip}%
\pgfsetbuttcap%
\pgfsetroundjoin%
\pgfsetlinewidth{0.250937pt}%
\definecolor{currentstroke}{rgb}{0.000000,0.000000,0.000000}%
\pgfsetstrokecolor{currentstroke}%
\pgfsetdash{}{0pt}%
\pgfpathmoveto{\pgfqpoint{0.625000in}{0.638539in}}%
\pgfpathlineto{\pgfqpoint{0.628666in}{0.646491in}}%
\pgfpathlineto{\pgfqpoint{0.625849in}{0.656140in}}%
\pgfpathlineto{\pgfqpoint{0.626161in}{0.665789in}}%
\pgfpathlineto{\pgfqpoint{0.625000in}{0.668046in}}%
\pgfusepath{stroke}%
\end{pgfscope}%
\begin{pgfscope}%
\pgfpathrectangle{\pgfqpoint{0.625000in}{0.550000in}}{\pgfqpoint{3.875000in}{3.850000in}} %
\pgfusepath{clip}%
\pgfsetbuttcap%
\pgfsetroundjoin%
\pgfsetlinewidth{0.250937pt}%
\definecolor{currentstroke}{rgb}{0.000000,0.000000,0.000000}%
\pgfsetstrokecolor{currentstroke}%
\pgfsetdash{}{0pt}%
\pgfpathmoveto{\pgfqpoint{0.625000in}{0.693368in}}%
\pgfpathlineto{\pgfqpoint{0.625241in}{0.694737in}}%
\pgfpathlineto{\pgfqpoint{0.627842in}{0.704386in}}%
\pgfpathlineto{\pgfqpoint{0.626095in}{0.714035in}}%
\pgfpathlineto{\pgfqpoint{0.625000in}{0.718074in}}%
\pgfusepath{stroke}%
\end{pgfscope}%
\begin{pgfscope}%
\pgfpathrectangle{\pgfqpoint{0.625000in}{0.550000in}}{\pgfqpoint{3.875000in}{3.850000in}} %
\pgfusepath{clip}%
\pgfsetbuttcap%
\pgfsetroundjoin%
\pgfsetlinewidth{0.250937pt}%
\definecolor{currentstroke}{rgb}{0.000000,0.000000,0.000000}%
\pgfsetstrokecolor{currentstroke}%
\pgfsetdash{}{0pt}%
\pgfpathmoveto{\pgfqpoint{0.625000in}{0.726183in}}%
\pgfpathlineto{\pgfqpoint{0.628005in}{0.733333in}}%
\pgfpathlineto{\pgfqpoint{0.629745in}{0.742982in}}%
\pgfpathlineto{\pgfqpoint{0.628088in}{0.752632in}}%
\pgfpathlineto{\pgfqpoint{0.625000in}{0.759855in}}%
\pgfusepath{stroke}%
\end{pgfscope}%
\begin{pgfscope}%
\pgfpathrectangle{\pgfqpoint{0.625000in}{0.550000in}}{\pgfqpoint{3.875000in}{3.850000in}} %
\pgfusepath{clip}%
\pgfsetbuttcap%
\pgfsetroundjoin%
\pgfsetlinewidth{0.250937pt}%
\definecolor{currentstroke}{rgb}{0.000000,0.000000,0.000000}%
\pgfsetstrokecolor{currentstroke}%
\pgfsetdash{}{0pt}%
\pgfpathmoveto{\pgfqpoint{0.625000in}{0.790884in}}%
\pgfpathlineto{\pgfqpoint{0.630403in}{0.791228in}}%
\pgfpathlineto{\pgfqpoint{0.625000in}{0.799428in}}%
\pgfusepath{stroke}%
\end{pgfscope}%
\begin{pgfscope}%
\pgfpathrectangle{\pgfqpoint{0.625000in}{0.550000in}}{\pgfqpoint{3.875000in}{3.850000in}} %
\pgfusepath{clip}%
\pgfsetbuttcap%
\pgfsetroundjoin%
\pgfsetlinewidth{0.250937pt}%
\definecolor{currentstroke}{rgb}{0.000000,0.000000,0.000000}%
\pgfsetstrokecolor{currentstroke}%
\pgfsetdash{}{0pt}%
\pgfpathmoveto{\pgfqpoint{0.625000in}{0.818127in}}%
\pgfpathlineto{\pgfqpoint{0.625524in}{0.820175in}}%
\pgfpathlineto{\pgfqpoint{0.626459in}{0.829825in}}%
\pgfpathlineto{\pgfqpoint{0.627849in}{0.839474in}}%
\pgfpathlineto{\pgfqpoint{0.626202in}{0.849123in}}%
\pgfpathlineto{\pgfqpoint{0.628214in}{0.858772in}}%
\pgfpathlineto{\pgfqpoint{0.625000in}{0.867885in}}%
\pgfusepath{stroke}%
\end{pgfscope}%
\begin{pgfscope}%
\pgfpathrectangle{\pgfqpoint{0.625000in}{0.550000in}}{\pgfqpoint{3.875000in}{3.850000in}} %
\pgfusepath{clip}%
\pgfsetbuttcap%
\pgfsetroundjoin%
\pgfsetlinewidth{0.250937pt}%
\definecolor{currentstroke}{rgb}{0.000000,0.000000,0.000000}%
\pgfsetstrokecolor{currentstroke}%
\pgfsetdash{}{0pt}%
\pgfpathmoveto{\pgfqpoint{0.625000in}{0.881801in}}%
\pgfpathlineto{\pgfqpoint{0.627174in}{0.887719in}}%
\pgfpathlineto{\pgfqpoint{0.626197in}{0.897368in}}%
\pgfpathlineto{\pgfqpoint{0.625000in}{0.898136in}}%
\pgfusepath{stroke}%
\end{pgfscope}%
\begin{pgfscope}%
\pgfpathrectangle{\pgfqpoint{0.625000in}{0.550000in}}{\pgfqpoint{3.875000in}{3.850000in}} %
\pgfusepath{clip}%
\pgfsetbuttcap%
\pgfsetroundjoin%
\pgfsetlinewidth{0.250937pt}%
\definecolor{currentstroke}{rgb}{0.000000,0.000000,0.000000}%
\pgfsetstrokecolor{currentstroke}%
\pgfsetdash{}{0pt}%
\pgfpathmoveto{\pgfqpoint{0.625000in}{0.945548in}}%
\pgfpathlineto{\pgfqpoint{0.625603in}{0.945614in}}%
\pgfpathlineto{\pgfqpoint{0.625000in}{0.950668in}}%
\pgfusepath{stroke}%
\end{pgfscope}%
\begin{pgfscope}%
\pgfpathrectangle{\pgfqpoint{0.625000in}{0.550000in}}{\pgfqpoint{3.875000in}{3.850000in}} %
\pgfusepath{clip}%
\pgfsetbuttcap%
\pgfsetroundjoin%
\pgfsetlinewidth{0.250937pt}%
\definecolor{currentstroke}{rgb}{0.000000,0.000000,0.000000}%
\pgfsetstrokecolor{currentstroke}%
\pgfsetdash{}{0pt}%
\pgfpathmoveto{\pgfqpoint{0.625000in}{0.971647in}}%
\pgfpathlineto{\pgfqpoint{0.626265in}{0.974561in}}%
\pgfpathlineto{\pgfqpoint{0.627988in}{0.984211in}}%
\pgfpathlineto{\pgfqpoint{0.625000in}{0.990929in}}%
\pgfusepath{stroke}%
\end{pgfscope}%
\begin{pgfscope}%
\pgfpathrectangle{\pgfqpoint{0.625000in}{0.550000in}}{\pgfqpoint{3.875000in}{3.850000in}} %
\pgfusepath{clip}%
\pgfsetbuttcap%
\pgfsetroundjoin%
\pgfsetlinewidth{0.250937pt}%
\definecolor{currentstroke}{rgb}{0.000000,0.000000,0.000000}%
\pgfsetstrokecolor{currentstroke}%
\pgfsetdash{}{0pt}%
\pgfpathmoveto{\pgfqpoint{0.625000in}{0.997136in}}%
\pgfpathlineto{\pgfqpoint{0.627482in}{1.003509in}}%
\pgfpathlineto{\pgfqpoint{0.628337in}{1.013158in}}%
\pgfpathlineto{\pgfqpoint{0.626569in}{1.022807in}}%
\pgfpathlineto{\pgfqpoint{0.632011in}{1.032456in}}%
\pgfpathlineto{\pgfqpoint{0.625000in}{1.039060in}}%
\pgfusepath{stroke}%
\end{pgfscope}%
\begin{pgfscope}%
\pgfpathrectangle{\pgfqpoint{0.625000in}{0.550000in}}{\pgfqpoint{3.875000in}{3.850000in}} %
\pgfusepath{clip}%
\pgfsetbuttcap%
\pgfsetroundjoin%
\pgfsetlinewidth{0.250937pt}%
\definecolor{currentstroke}{rgb}{0.000000,0.000000,0.000000}%
\pgfsetstrokecolor{currentstroke}%
\pgfsetdash{}{0pt}%
\pgfpathmoveto{\pgfqpoint{0.625000in}{1.049956in}}%
\pgfpathlineto{\pgfqpoint{0.626322in}{1.051754in}}%
\pgfpathlineto{\pgfqpoint{0.626056in}{1.061404in}}%
\pgfpathlineto{\pgfqpoint{0.625000in}{1.065268in}}%
\pgfusepath{stroke}%
\end{pgfscope}%
\begin{pgfscope}%
\pgfpathrectangle{\pgfqpoint{0.625000in}{0.550000in}}{\pgfqpoint{3.875000in}{3.850000in}} %
\pgfusepath{clip}%
\pgfsetbuttcap%
\pgfsetroundjoin%
\pgfsetlinewidth{0.250937pt}%
\definecolor{currentstroke}{rgb}{0.000000,0.000000,0.000000}%
\pgfsetstrokecolor{currentstroke}%
\pgfsetdash{}{0pt}%
\pgfpathmoveto{\pgfqpoint{0.625000in}{1.076684in}}%
\pgfpathlineto{\pgfqpoint{0.625844in}{1.080702in}}%
\pgfpathlineto{\pgfqpoint{0.625000in}{1.080746in}}%
\pgfusepath{stroke}%
\end{pgfscope}%
\begin{pgfscope}%
\pgfpathrectangle{\pgfqpoint{0.625000in}{0.550000in}}{\pgfqpoint{3.875000in}{3.850000in}} %
\pgfusepath{clip}%
\pgfsetbuttcap%
\pgfsetroundjoin%
\pgfsetlinewidth{0.250937pt}%
\definecolor{currentstroke}{rgb}{0.000000,0.000000,0.000000}%
\pgfsetstrokecolor{currentstroke}%
\pgfsetdash{}{0pt}%
\pgfpathmoveto{\pgfqpoint{0.625000in}{1.099991in}}%
\pgfpathlineto{\pgfqpoint{0.625309in}{1.100000in}}%
\pgfpathlineto{\pgfqpoint{0.625000in}{1.101170in}}%
\pgfusepath{stroke}%
\end{pgfscope}%
\begin{pgfscope}%
\pgfpathrectangle{\pgfqpoint{0.625000in}{0.550000in}}{\pgfqpoint{3.875000in}{3.850000in}} %
\pgfusepath{clip}%
\pgfsetbuttcap%
\pgfsetroundjoin%
\pgfsetlinewidth{0.250937pt}%
\definecolor{currentstroke}{rgb}{0.000000,0.000000,0.000000}%
\pgfsetstrokecolor{currentstroke}%
\pgfsetdash{}{0pt}%
\pgfpathmoveto{\pgfqpoint{0.625000in}{1.113147in}}%
\pgfpathlineto{\pgfqpoint{0.627292in}{1.119298in}}%
\pgfpathlineto{\pgfqpoint{0.629731in}{1.128947in}}%
\pgfpathlineto{\pgfqpoint{0.626526in}{1.138596in}}%
\pgfpathlineto{\pgfqpoint{0.625000in}{1.143565in}}%
\pgfusepath{stroke}%
\end{pgfscope}%
\begin{pgfscope}%
\pgfpathrectangle{\pgfqpoint{0.625000in}{0.550000in}}{\pgfqpoint{3.875000in}{3.850000in}} %
\pgfusepath{clip}%
\pgfsetbuttcap%
\pgfsetroundjoin%
\pgfsetlinewidth{0.250937pt}%
\definecolor{currentstroke}{rgb}{0.000000,0.000000,0.000000}%
\pgfsetstrokecolor{currentstroke}%
\pgfsetdash{}{0pt}%
\pgfpathmoveto{\pgfqpoint{0.625000in}{1.161848in}}%
\pgfpathlineto{\pgfqpoint{0.626673in}{1.167544in}}%
\pgfpathlineto{\pgfqpoint{0.625000in}{1.174521in}}%
\pgfusepath{stroke}%
\end{pgfscope}%
\begin{pgfscope}%
\pgfpathrectangle{\pgfqpoint{0.625000in}{0.550000in}}{\pgfqpoint{3.875000in}{3.850000in}} %
\pgfusepath{clip}%
\pgfsetbuttcap%
\pgfsetroundjoin%
\pgfsetlinewidth{0.250937pt}%
\definecolor{currentstroke}{rgb}{0.000000,0.000000,0.000000}%
\pgfsetstrokecolor{currentstroke}%
\pgfsetdash{}{0pt}%
\pgfpathmoveto{\pgfqpoint{0.625000in}{1.205286in}}%
\pgfpathlineto{\pgfqpoint{0.625227in}{1.206140in}}%
\pgfpathlineto{\pgfqpoint{0.626617in}{1.215789in}}%
\pgfpathlineto{\pgfqpoint{0.627479in}{1.225439in}}%
\pgfpathlineto{\pgfqpoint{0.625000in}{1.233380in}}%
\pgfusepath{stroke}%
\end{pgfscope}%
\begin{pgfscope}%
\pgfpathrectangle{\pgfqpoint{0.625000in}{0.550000in}}{\pgfqpoint{3.875000in}{3.850000in}} %
\pgfusepath{clip}%
\pgfsetbuttcap%
\pgfsetroundjoin%
\pgfsetlinewidth{0.250937pt}%
\definecolor{currentstroke}{rgb}{0.000000,0.000000,0.000000}%
\pgfsetstrokecolor{currentstroke}%
\pgfsetdash{}{0pt}%
\pgfpathmoveto{\pgfqpoint{0.625000in}{1.265568in}}%
\pgfpathlineto{\pgfqpoint{0.630243in}{1.273684in}}%
\pgfpathlineto{\pgfqpoint{0.626180in}{1.283333in}}%
\pgfpathlineto{\pgfqpoint{0.625000in}{1.283996in}}%
\pgfusepath{stroke}%
\end{pgfscope}%
\begin{pgfscope}%
\pgfpathrectangle{\pgfqpoint{0.625000in}{0.550000in}}{\pgfqpoint{3.875000in}{3.850000in}} %
\pgfusepath{clip}%
\pgfsetbuttcap%
\pgfsetroundjoin%
\pgfsetlinewidth{0.250937pt}%
\definecolor{currentstroke}{rgb}{0.000000,0.000000,0.000000}%
\pgfsetstrokecolor{currentstroke}%
\pgfsetdash{}{0pt}%
\pgfpathmoveto{\pgfqpoint{0.625000in}{1.306298in}}%
\pgfpathlineto{\pgfqpoint{0.627718in}{1.312281in}}%
\pgfpathlineto{\pgfqpoint{0.631457in}{1.321930in}}%
\pgfpathlineto{\pgfqpoint{0.628403in}{1.331579in}}%
\pgfpathlineto{\pgfqpoint{0.625000in}{1.338273in}}%
\pgfusepath{stroke}%
\end{pgfscope}%
\begin{pgfscope}%
\pgfpathrectangle{\pgfqpoint{0.625000in}{0.550000in}}{\pgfqpoint{3.875000in}{3.850000in}} %
\pgfusepath{clip}%
\pgfsetbuttcap%
\pgfsetroundjoin%
\pgfsetlinewidth{0.250937pt}%
\definecolor{currentstroke}{rgb}{0.000000,0.000000,0.000000}%
\pgfsetstrokecolor{currentstroke}%
\pgfsetdash{}{0pt}%
\pgfpathmoveto{\pgfqpoint{0.625000in}{1.351577in}}%
\pgfpathlineto{\pgfqpoint{0.626009in}{1.360526in}}%
\pgfpathlineto{\pgfqpoint{0.625034in}{1.370175in}}%
\pgfpathlineto{\pgfqpoint{0.625000in}{1.370564in}}%
\pgfusepath{stroke}%
\end{pgfscope}%
\begin{pgfscope}%
\pgfpathrectangle{\pgfqpoint{0.625000in}{0.550000in}}{\pgfqpoint{3.875000in}{3.850000in}} %
\pgfusepath{clip}%
\pgfsetbuttcap%
\pgfsetroundjoin%
\pgfsetlinewidth{0.250937pt}%
\definecolor{currentstroke}{rgb}{0.000000,0.000000,0.000000}%
\pgfsetstrokecolor{currentstroke}%
\pgfsetdash{}{0pt}%
\pgfpathmoveto{\pgfqpoint{0.625000in}{1.385401in}}%
\pgfpathlineto{\pgfqpoint{0.625811in}{1.389474in}}%
\pgfpathlineto{\pgfqpoint{0.625000in}{1.389521in}}%
\pgfusepath{stroke}%
\end{pgfscope}%
\begin{pgfscope}%
\pgfpathrectangle{\pgfqpoint{0.625000in}{0.550000in}}{\pgfqpoint{3.875000in}{3.850000in}} %
\pgfusepath{clip}%
\pgfsetbuttcap%
\pgfsetroundjoin%
\pgfsetlinewidth{0.250937pt}%
\definecolor{currentstroke}{rgb}{0.000000,0.000000,0.000000}%
\pgfsetstrokecolor{currentstroke}%
\pgfsetdash{}{0pt}%
\pgfpathmoveto{\pgfqpoint{0.625000in}{1.408720in}}%
\pgfpathlineto{\pgfqpoint{0.626404in}{1.408772in}}%
\pgfpathlineto{\pgfqpoint{0.628237in}{1.418421in}}%
\pgfpathlineto{\pgfqpoint{0.625000in}{1.422088in}}%
\pgfusepath{stroke}%
\end{pgfscope}%
\begin{pgfscope}%
\pgfpathrectangle{\pgfqpoint{0.625000in}{0.550000in}}{\pgfqpoint{3.875000in}{3.850000in}} %
\pgfusepath{clip}%
\pgfsetbuttcap%
\pgfsetroundjoin%
\pgfsetlinewidth{0.250937pt}%
\definecolor{currentstroke}{rgb}{0.000000,0.000000,0.000000}%
\pgfsetstrokecolor{currentstroke}%
\pgfsetdash{}{0pt}%
\pgfpathmoveto{\pgfqpoint{0.625000in}{1.436457in}}%
\pgfpathlineto{\pgfqpoint{0.626114in}{1.437719in}}%
\pgfpathlineto{\pgfqpoint{0.625000in}{1.444456in}}%
\pgfusepath{stroke}%
\end{pgfscope}%
\begin{pgfscope}%
\pgfpathrectangle{\pgfqpoint{0.625000in}{0.550000in}}{\pgfqpoint{3.875000in}{3.850000in}} %
\pgfusepath{clip}%
\pgfsetbuttcap%
\pgfsetroundjoin%
\pgfsetlinewidth{0.250937pt}%
\definecolor{currentstroke}{rgb}{0.000000,0.000000,0.000000}%
\pgfsetstrokecolor{currentstroke}%
\pgfsetdash{}{0pt}%
\pgfpathmoveto{\pgfqpoint{0.625000in}{1.461342in}}%
\pgfpathlineto{\pgfqpoint{0.626874in}{1.466667in}}%
\pgfpathlineto{\pgfqpoint{0.628435in}{1.476316in}}%
\pgfpathlineto{\pgfqpoint{0.627969in}{1.485965in}}%
\pgfpathlineto{\pgfqpoint{0.625000in}{1.492072in}}%
\pgfusepath{stroke}%
\end{pgfscope}%
\begin{pgfscope}%
\pgfpathrectangle{\pgfqpoint{0.625000in}{0.550000in}}{\pgfqpoint{3.875000in}{3.850000in}} %
\pgfusepath{clip}%
\pgfsetbuttcap%
\pgfsetroundjoin%
\pgfsetlinewidth{0.250937pt}%
\definecolor{currentstroke}{rgb}{0.000000,0.000000,0.000000}%
\pgfsetstrokecolor{currentstroke}%
\pgfsetdash{}{0pt}%
\pgfpathmoveto{\pgfqpoint{0.625000in}{1.506222in}}%
\pgfpathlineto{\pgfqpoint{0.633732in}{1.514912in}}%
\pgfpathlineto{\pgfqpoint{0.625000in}{1.524032in}}%
\pgfusepath{stroke}%
\end{pgfscope}%
\begin{pgfscope}%
\pgfpathrectangle{\pgfqpoint{0.625000in}{0.550000in}}{\pgfqpoint{3.875000in}{3.850000in}} %
\pgfusepath{clip}%
\pgfsetbuttcap%
\pgfsetroundjoin%
\pgfsetlinewidth{0.250937pt}%
\definecolor{currentstroke}{rgb}{0.000000,0.000000,0.000000}%
\pgfsetstrokecolor{currentstroke}%
\pgfsetdash{}{0pt}%
\pgfpathmoveto{\pgfqpoint{0.625000in}{1.563051in}}%
\pgfpathlineto{\pgfqpoint{0.627412in}{1.563158in}}%
\pgfpathlineto{\pgfqpoint{0.625000in}{1.569215in}}%
\pgfusepath{stroke}%
\end{pgfscope}%
\begin{pgfscope}%
\pgfpathrectangle{\pgfqpoint{0.625000in}{0.550000in}}{\pgfqpoint{3.875000in}{3.850000in}} %
\pgfusepath{clip}%
\pgfsetbuttcap%
\pgfsetroundjoin%
\pgfsetlinewidth{0.250937pt}%
\definecolor{currentstroke}{rgb}{0.000000,0.000000,0.000000}%
\pgfsetstrokecolor{currentstroke}%
\pgfsetdash{}{0pt}%
\pgfpathmoveto{\pgfqpoint{0.625000in}{1.590613in}}%
\pgfpathlineto{\pgfqpoint{0.625991in}{1.592105in}}%
\pgfpathlineto{\pgfqpoint{0.627496in}{1.601754in}}%
\pgfpathlineto{\pgfqpoint{0.628054in}{1.611404in}}%
\pgfpathlineto{\pgfqpoint{0.625000in}{1.615212in}}%
\pgfusepath{stroke}%
\end{pgfscope}%
\begin{pgfscope}%
\pgfpathrectangle{\pgfqpoint{0.625000in}{0.550000in}}{\pgfqpoint{3.875000in}{3.850000in}} %
\pgfusepath{clip}%
\pgfsetbuttcap%
\pgfsetroundjoin%
\pgfsetlinewidth{0.250937pt}%
\definecolor{currentstroke}{rgb}{0.000000,0.000000,0.000000}%
\pgfsetstrokecolor{currentstroke}%
\pgfsetdash{}{0pt}%
\pgfpathmoveto{\pgfqpoint{0.625000in}{1.627350in}}%
\pgfpathlineto{\pgfqpoint{0.627430in}{1.630702in}}%
\pgfpathlineto{\pgfqpoint{0.625000in}{1.633460in}}%
\pgfusepath{stroke}%
\end{pgfscope}%
\begin{pgfscope}%
\pgfpathrectangle{\pgfqpoint{0.625000in}{0.550000in}}{\pgfqpoint{3.875000in}{3.850000in}} %
\pgfusepath{clip}%
\pgfsetbuttcap%
\pgfsetroundjoin%
\pgfsetlinewidth{0.250937pt}%
\definecolor{currentstroke}{rgb}{0.000000,0.000000,0.000000}%
\pgfsetstrokecolor{currentstroke}%
\pgfsetdash{}{0pt}%
\pgfpathmoveto{\pgfqpoint{0.625000in}{1.652677in}}%
\pgfpathlineto{\pgfqpoint{0.628466in}{1.659649in}}%
\pgfpathlineto{\pgfqpoint{0.626166in}{1.669298in}}%
\pgfpathlineto{\pgfqpoint{0.625000in}{1.671728in}}%
\pgfusepath{stroke}%
\end{pgfscope}%
\begin{pgfscope}%
\pgfpathrectangle{\pgfqpoint{0.625000in}{0.550000in}}{\pgfqpoint{3.875000in}{3.850000in}} %
\pgfusepath{clip}%
\pgfsetbuttcap%
\pgfsetroundjoin%
\pgfsetlinewidth{0.250937pt}%
\definecolor{currentstroke}{rgb}{0.000000,0.000000,0.000000}%
\pgfsetstrokecolor{currentstroke}%
\pgfsetdash{}{0pt}%
\pgfpathmoveto{\pgfqpoint{0.625000in}{1.695187in}}%
\pgfpathlineto{\pgfqpoint{0.625264in}{1.698246in}}%
\pgfpathlineto{\pgfqpoint{0.625000in}{1.698275in}}%
\pgfusepath{stroke}%
\end{pgfscope}%
\begin{pgfscope}%
\pgfpathrectangle{\pgfqpoint{0.625000in}{0.550000in}}{\pgfqpoint{3.875000in}{3.850000in}} %
\pgfusepath{clip}%
\pgfsetbuttcap%
\pgfsetroundjoin%
\pgfsetlinewidth{0.250937pt}%
\definecolor{currentstroke}{rgb}{0.000000,0.000000,0.000000}%
\pgfsetstrokecolor{currentstroke}%
\pgfsetdash{}{0pt}%
\pgfpathmoveto{\pgfqpoint{0.625000in}{1.717454in}}%
\pgfpathlineto{\pgfqpoint{0.625762in}{1.717544in}}%
\pgfpathlineto{\pgfqpoint{0.625000in}{1.723195in}}%
\pgfusepath{stroke}%
\end{pgfscope}%
\begin{pgfscope}%
\pgfpathrectangle{\pgfqpoint{0.625000in}{0.550000in}}{\pgfqpoint{3.875000in}{3.850000in}} %
\pgfusepath{clip}%
\pgfsetbuttcap%
\pgfsetroundjoin%
\pgfsetlinewidth{0.250937pt}%
\definecolor{currentstroke}{rgb}{0.000000,0.000000,0.000000}%
\pgfsetstrokecolor{currentstroke}%
\pgfsetdash{}{0pt}%
\pgfpathmoveto{\pgfqpoint{0.625000in}{1.731838in}}%
\pgfpathlineto{\pgfqpoint{0.626500in}{1.736842in}}%
\pgfpathlineto{\pgfqpoint{0.626209in}{1.746491in}}%
\pgfpathlineto{\pgfqpoint{0.630341in}{1.756140in}}%
\pgfpathlineto{\pgfqpoint{0.625000in}{1.764308in}}%
\pgfusepath{stroke}%
\end{pgfscope}%
\begin{pgfscope}%
\pgfpathrectangle{\pgfqpoint{0.625000in}{0.550000in}}{\pgfqpoint{3.875000in}{3.850000in}} %
\pgfusepath{clip}%
\pgfsetbuttcap%
\pgfsetroundjoin%
\pgfsetlinewidth{0.250937pt}%
\definecolor{currentstroke}{rgb}{0.000000,0.000000,0.000000}%
\pgfsetstrokecolor{currentstroke}%
\pgfsetdash{}{0pt}%
\pgfpathmoveto{\pgfqpoint{0.625000in}{1.770999in}}%
\pgfpathlineto{\pgfqpoint{0.625933in}{1.775439in}}%
\pgfpathlineto{\pgfqpoint{0.627662in}{1.785088in}}%
\pgfpathlineto{\pgfqpoint{0.628155in}{1.794737in}}%
\pgfpathlineto{\pgfqpoint{0.628614in}{1.804386in}}%
\pgfpathlineto{\pgfqpoint{0.625000in}{1.810429in}}%
\pgfusepath{stroke}%
\end{pgfscope}%
\begin{pgfscope}%
\pgfpathrectangle{\pgfqpoint{0.625000in}{0.550000in}}{\pgfqpoint{3.875000in}{3.850000in}} %
\pgfusepath{clip}%
\pgfsetbuttcap%
\pgfsetroundjoin%
\pgfsetlinewidth{0.250937pt}%
\definecolor{currentstroke}{rgb}{0.000000,0.000000,0.000000}%
\pgfsetstrokecolor{currentstroke}%
\pgfsetdash{}{0pt}%
\pgfpathmoveto{\pgfqpoint{0.625000in}{1.820870in}}%
\pgfpathlineto{\pgfqpoint{0.626800in}{1.823684in}}%
\pgfpathlineto{\pgfqpoint{0.627494in}{1.833333in}}%
\pgfpathlineto{\pgfqpoint{0.625000in}{1.837457in}}%
\pgfusepath{stroke}%
\end{pgfscope}%
\begin{pgfscope}%
\pgfpathrectangle{\pgfqpoint{0.625000in}{0.550000in}}{\pgfqpoint{3.875000in}{3.850000in}} %
\pgfusepath{clip}%
\pgfsetbuttcap%
\pgfsetroundjoin%
\pgfsetlinewidth{0.250937pt}%
\definecolor{currentstroke}{rgb}{0.000000,0.000000,0.000000}%
\pgfsetstrokecolor{currentstroke}%
\pgfsetdash{}{0pt}%
\pgfpathmoveto{\pgfqpoint{0.625000in}{1.845464in}}%
\pgfpathlineto{\pgfqpoint{0.629252in}{1.852632in}}%
\pgfpathlineto{\pgfqpoint{0.625000in}{1.852817in}}%
\pgfusepath{stroke}%
\end{pgfscope}%
\begin{pgfscope}%
\pgfpathrectangle{\pgfqpoint{0.625000in}{0.550000in}}{\pgfqpoint{3.875000in}{3.850000in}} %
\pgfusepath{clip}%
\pgfsetbuttcap%
\pgfsetroundjoin%
\pgfsetlinewidth{0.250937pt}%
\definecolor{currentstroke}{rgb}{0.000000,0.000000,0.000000}%
\pgfsetstrokecolor{currentstroke}%
\pgfsetdash{}{0pt}%
\pgfpathmoveto{\pgfqpoint{0.625000in}{1.897895in}}%
\pgfpathlineto{\pgfqpoint{0.629741in}{1.900877in}}%
\pgfpathlineto{\pgfqpoint{0.625000in}{1.906030in}}%
\pgfusepath{stroke}%
\end{pgfscope}%
\begin{pgfscope}%
\pgfpathrectangle{\pgfqpoint{0.625000in}{0.550000in}}{\pgfqpoint{3.875000in}{3.850000in}} %
\pgfusepath{clip}%
\pgfsetbuttcap%
\pgfsetroundjoin%
\pgfsetlinewidth{0.250937pt}%
\definecolor{currentstroke}{rgb}{0.000000,0.000000,0.000000}%
\pgfsetstrokecolor{currentstroke}%
\pgfsetdash{}{0pt}%
\pgfpathmoveto{\pgfqpoint{0.625000in}{1.937547in}}%
\pgfpathlineto{\pgfqpoint{0.627163in}{1.939474in}}%
\pgfpathlineto{\pgfqpoint{0.627571in}{1.949123in}}%
\pgfpathlineto{\pgfqpoint{0.625000in}{1.955608in}}%
\pgfusepath{stroke}%
\end{pgfscope}%
\begin{pgfscope}%
\pgfpathrectangle{\pgfqpoint{0.625000in}{0.550000in}}{\pgfqpoint{3.875000in}{3.850000in}} %
\pgfusepath{clip}%
\pgfsetbuttcap%
\pgfsetroundjoin%
\pgfsetlinewidth{0.250937pt}%
\definecolor{currentstroke}{rgb}{0.000000,0.000000,0.000000}%
\pgfsetstrokecolor{currentstroke}%
\pgfsetdash{}{0pt}%
\pgfpathmoveto{\pgfqpoint{0.625000in}{1.969454in}}%
\pgfpathlineto{\pgfqpoint{0.626323in}{1.978070in}}%
\pgfpathlineto{\pgfqpoint{0.626190in}{1.987719in}}%
\pgfpathlineto{\pgfqpoint{0.632011in}{1.997368in}}%
\pgfpathlineto{\pgfqpoint{0.625000in}{2.003336in}}%
\pgfusepath{stroke}%
\end{pgfscope}%
\begin{pgfscope}%
\pgfpathrectangle{\pgfqpoint{0.625000in}{0.550000in}}{\pgfqpoint{3.875000in}{3.850000in}} %
\pgfusepath{clip}%
\pgfsetbuttcap%
\pgfsetroundjoin%
\pgfsetlinewidth{0.250937pt}%
\definecolor{currentstroke}{rgb}{0.000000,0.000000,0.000000}%
\pgfsetstrokecolor{currentstroke}%
\pgfsetdash{}{0pt}%
\pgfpathmoveto{\pgfqpoint{0.625000in}{2.043747in}}%
\pgfpathlineto{\pgfqpoint{0.625248in}{2.045614in}}%
\pgfpathlineto{\pgfqpoint{0.625700in}{2.055263in}}%
\pgfpathlineto{\pgfqpoint{0.625000in}{2.058628in}}%
\pgfusepath{stroke}%
\end{pgfscope}%
\begin{pgfscope}%
\pgfpathrectangle{\pgfqpoint{0.625000in}{0.550000in}}{\pgfqpoint{3.875000in}{3.850000in}} %
\pgfusepath{clip}%
\pgfsetbuttcap%
\pgfsetroundjoin%
\pgfsetlinewidth{0.250937pt}%
\definecolor{currentstroke}{rgb}{0.000000,0.000000,0.000000}%
\pgfsetstrokecolor{currentstroke}%
\pgfsetdash{}{0pt}%
\pgfpathmoveto{\pgfqpoint{0.625000in}{2.085592in}}%
\pgfpathlineto{\pgfqpoint{0.631457in}{2.093860in}}%
\pgfpathlineto{\pgfqpoint{0.625000in}{2.102159in}}%
\pgfusepath{stroke}%
\end{pgfscope}%
\begin{pgfscope}%
\pgfpathrectangle{\pgfqpoint{0.625000in}{0.550000in}}{\pgfqpoint{3.875000in}{3.850000in}} %
\pgfusepath{clip}%
\pgfsetbuttcap%
\pgfsetroundjoin%
\pgfsetlinewidth{0.250937pt}%
\definecolor{currentstroke}{rgb}{0.000000,0.000000,0.000000}%
\pgfsetstrokecolor{currentstroke}%
\pgfsetdash{}{0pt}%
\pgfpathmoveto{\pgfqpoint{0.625000in}{2.118437in}}%
\pgfpathlineto{\pgfqpoint{0.626038in}{2.122807in}}%
\pgfpathlineto{\pgfqpoint{0.625861in}{2.132456in}}%
\pgfpathlineto{\pgfqpoint{0.627297in}{2.142105in}}%
\pgfpathlineto{\pgfqpoint{0.625000in}{2.148156in}}%
\pgfusepath{stroke}%
\end{pgfscope}%
\begin{pgfscope}%
\pgfpathrectangle{\pgfqpoint{0.625000in}{0.550000in}}{\pgfqpoint{3.875000in}{3.850000in}} %
\pgfusepath{clip}%
\pgfsetbuttcap%
\pgfsetroundjoin%
\pgfsetlinewidth{0.250937pt}%
\definecolor{currentstroke}{rgb}{0.000000,0.000000,0.000000}%
\pgfsetstrokecolor{currentstroke}%
\pgfsetdash{}{0pt}%
\pgfpathmoveto{\pgfqpoint{0.625000in}{2.180566in}}%
\pgfpathlineto{\pgfqpoint{0.627292in}{2.180702in}}%
\pgfpathlineto{\pgfqpoint{0.628393in}{2.190351in}}%
\pgfpathlineto{\pgfqpoint{0.625000in}{2.199091in}}%
\pgfusepath{stroke}%
\end{pgfscope}%
\begin{pgfscope}%
\pgfpathrectangle{\pgfqpoint{0.625000in}{0.550000in}}{\pgfqpoint{3.875000in}{3.850000in}} %
\pgfusepath{clip}%
\pgfsetbuttcap%
\pgfsetroundjoin%
\pgfsetlinewidth{0.250937pt}%
\definecolor{currentstroke}{rgb}{0.000000,0.000000,0.000000}%
\pgfsetstrokecolor{currentstroke}%
\pgfsetdash{}{0pt}%
\pgfpathmoveto{\pgfqpoint{0.625000in}{2.202871in}}%
\pgfpathlineto{\pgfqpoint{0.626388in}{2.209649in}}%
\pgfpathlineto{\pgfqpoint{0.625755in}{2.219298in}}%
\pgfpathlineto{\pgfqpoint{0.625000in}{2.222503in}}%
\pgfusepath{stroke}%
\end{pgfscope}%
\begin{pgfscope}%
\pgfpathrectangle{\pgfqpoint{0.625000in}{0.550000in}}{\pgfqpoint{3.875000in}{3.850000in}} %
\pgfusepath{clip}%
\pgfsetbuttcap%
\pgfsetroundjoin%
\pgfsetlinewidth{0.250937pt}%
\definecolor{currentstroke}{rgb}{0.000000,0.000000,0.000000}%
\pgfsetstrokecolor{currentstroke}%
\pgfsetdash{}{0pt}%
\pgfpathmoveto{\pgfqpoint{0.625000in}{2.230522in}}%
\pgfpathlineto{\pgfqpoint{0.630166in}{2.238596in}}%
\pgfpathlineto{\pgfqpoint{0.628093in}{2.248246in}}%
\pgfpathlineto{\pgfqpoint{0.625000in}{2.256996in}}%
\pgfusepath{stroke}%
\end{pgfscope}%
\begin{pgfscope}%
\pgfpathrectangle{\pgfqpoint{0.625000in}{0.550000in}}{\pgfqpoint{3.875000in}{3.850000in}} %
\pgfusepath{clip}%
\pgfsetbuttcap%
\pgfsetroundjoin%
\pgfsetlinewidth{0.250937pt}%
\definecolor{currentstroke}{rgb}{0.000000,0.000000,0.000000}%
\pgfsetstrokecolor{currentstroke}%
\pgfsetdash{}{0pt}%
\pgfpathmoveto{\pgfqpoint{0.625000in}{2.280258in}}%
\pgfpathlineto{\pgfqpoint{0.629727in}{2.286842in}}%
\pgfpathlineto{\pgfqpoint{0.625000in}{2.294200in}}%
\pgfusepath{stroke}%
\end{pgfscope}%
\begin{pgfscope}%
\pgfpathrectangle{\pgfqpoint{0.625000in}{0.550000in}}{\pgfqpoint{3.875000in}{3.850000in}} %
\pgfusepath{clip}%
\pgfsetbuttcap%
\pgfsetroundjoin%
\pgfsetlinewidth{0.250937pt}%
\definecolor{currentstroke}{rgb}{0.000000,0.000000,0.000000}%
\pgfsetstrokecolor{currentstroke}%
\pgfsetdash{}{0pt}%
\pgfpathmoveto{\pgfqpoint{0.625000in}{2.334921in}}%
\pgfpathlineto{\pgfqpoint{0.627862in}{2.335088in}}%
\pgfpathlineto{\pgfqpoint{0.625000in}{2.342025in}}%
\pgfusepath{stroke}%
\end{pgfscope}%
\begin{pgfscope}%
\pgfpathrectangle{\pgfqpoint{0.625000in}{0.550000in}}{\pgfqpoint{3.875000in}{3.850000in}} %
\pgfusepath{clip}%
\pgfsetbuttcap%
\pgfsetroundjoin%
\pgfsetlinewidth{0.250937pt}%
\definecolor{currentstroke}{rgb}{0.000000,0.000000,0.000000}%
\pgfsetstrokecolor{currentstroke}%
\pgfsetdash{}{0pt}%
\pgfpathmoveto{\pgfqpoint{0.625000in}{2.353476in}}%
\pgfpathlineto{\pgfqpoint{0.625144in}{2.354386in}}%
\pgfpathlineto{\pgfqpoint{0.625886in}{2.364035in}}%
\pgfpathlineto{\pgfqpoint{0.625000in}{2.366011in}}%
\pgfusepath{stroke}%
\end{pgfscope}%
\begin{pgfscope}%
\pgfpathrectangle{\pgfqpoint{0.625000in}{0.550000in}}{\pgfqpoint{3.875000in}{3.850000in}} %
\pgfusepath{clip}%
\pgfsetbuttcap%
\pgfsetroundjoin%
\pgfsetlinewidth{0.250937pt}%
\definecolor{currentstroke}{rgb}{0.000000,0.000000,0.000000}%
\pgfsetstrokecolor{currentstroke}%
\pgfsetdash{}{0pt}%
\pgfpathmoveto{\pgfqpoint{0.625000in}{2.378394in}}%
\pgfpathlineto{\pgfqpoint{0.627384in}{2.383333in}}%
\pgfpathlineto{\pgfqpoint{0.625724in}{2.392982in}}%
\pgfpathlineto{\pgfqpoint{0.626546in}{2.402632in}}%
\pgfpathlineto{\pgfqpoint{0.625000in}{2.405910in}}%
\pgfusepath{stroke}%
\end{pgfscope}%
\begin{pgfscope}%
\pgfpathrectangle{\pgfqpoint{0.625000in}{0.550000in}}{\pgfqpoint{3.875000in}{3.850000in}} %
\pgfusepath{clip}%
\pgfsetbuttcap%
\pgfsetroundjoin%
\pgfsetlinewidth{0.250937pt}%
\definecolor{currentstroke}{rgb}{0.000000,0.000000,0.000000}%
\pgfsetstrokecolor{currentstroke}%
\pgfsetdash{}{0pt}%
\pgfpathmoveto{\pgfqpoint{0.625000in}{2.432340in}}%
\pgfpathlineto{\pgfqpoint{0.625157in}{2.441228in}}%
\pgfpathlineto{\pgfqpoint{0.625171in}{2.450877in}}%
\pgfpathlineto{\pgfqpoint{0.625000in}{2.456503in}}%
\pgfusepath{stroke}%
\end{pgfscope}%
\begin{pgfscope}%
\pgfpathrectangle{\pgfqpoint{0.625000in}{0.550000in}}{\pgfqpoint{3.875000in}{3.850000in}} %
\pgfusepath{clip}%
\pgfsetbuttcap%
\pgfsetroundjoin%
\pgfsetlinewidth{0.250937pt}%
\definecolor{currentstroke}{rgb}{0.000000,0.000000,0.000000}%
\pgfsetstrokecolor{currentstroke}%
\pgfsetdash{}{0pt}%
\pgfpathmoveto{\pgfqpoint{0.625000in}{2.503146in}}%
\pgfpathlineto{\pgfqpoint{0.625171in}{2.508772in}}%
\pgfpathlineto{\pgfqpoint{0.625157in}{2.518421in}}%
\pgfpathlineto{\pgfqpoint{0.625000in}{2.527309in}}%
\pgfusepath{stroke}%
\end{pgfscope}%
\begin{pgfscope}%
\pgfpathrectangle{\pgfqpoint{0.625000in}{0.550000in}}{\pgfqpoint{3.875000in}{3.850000in}} %
\pgfusepath{clip}%
\pgfsetbuttcap%
\pgfsetroundjoin%
\pgfsetlinewidth{0.250937pt}%
\definecolor{currentstroke}{rgb}{0.000000,0.000000,0.000000}%
\pgfsetstrokecolor{currentstroke}%
\pgfsetdash{}{0pt}%
\pgfpathmoveto{\pgfqpoint{0.625000in}{2.553739in}}%
\pgfpathlineto{\pgfqpoint{0.626546in}{2.557018in}}%
\pgfpathlineto{\pgfqpoint{0.625724in}{2.566667in}}%
\pgfpathlineto{\pgfqpoint{0.627384in}{2.576316in}}%
\pgfpathlineto{\pgfqpoint{0.625000in}{2.581255in}}%
\pgfusepath{stroke}%
\end{pgfscope}%
\begin{pgfscope}%
\pgfpathrectangle{\pgfqpoint{0.625000in}{0.550000in}}{\pgfqpoint{3.875000in}{3.850000in}} %
\pgfusepath{clip}%
\pgfsetbuttcap%
\pgfsetroundjoin%
\pgfsetlinewidth{0.250937pt}%
\definecolor{currentstroke}{rgb}{0.000000,0.000000,0.000000}%
\pgfsetstrokecolor{currentstroke}%
\pgfsetdash{}{0pt}%
\pgfpathmoveto{\pgfqpoint{0.625000in}{2.593638in}}%
\pgfpathlineto{\pgfqpoint{0.625886in}{2.595614in}}%
\pgfpathlineto{\pgfqpoint{0.625144in}{2.605263in}}%
\pgfpathlineto{\pgfqpoint{0.625000in}{2.606173in}}%
\pgfusepath{stroke}%
\end{pgfscope}%
\begin{pgfscope}%
\pgfpathrectangle{\pgfqpoint{0.625000in}{0.550000in}}{\pgfqpoint{3.875000in}{3.850000in}} %
\pgfusepath{clip}%
\pgfsetbuttcap%
\pgfsetroundjoin%
\pgfsetlinewidth{0.250937pt}%
\definecolor{currentstroke}{rgb}{0.000000,0.000000,0.000000}%
\pgfsetstrokecolor{currentstroke}%
\pgfsetdash{}{0pt}%
\pgfpathmoveto{\pgfqpoint{0.625000in}{2.617624in}}%
\pgfpathlineto{\pgfqpoint{0.627862in}{2.624561in}}%
\pgfpathlineto{\pgfqpoint{0.625000in}{2.624729in}}%
\pgfusepath{stroke}%
\end{pgfscope}%
\begin{pgfscope}%
\pgfpathrectangle{\pgfqpoint{0.625000in}{0.550000in}}{\pgfqpoint{3.875000in}{3.850000in}} %
\pgfusepath{clip}%
\pgfsetbuttcap%
\pgfsetroundjoin%
\pgfsetlinewidth{0.250937pt}%
\definecolor{currentstroke}{rgb}{0.000000,0.000000,0.000000}%
\pgfsetstrokecolor{currentstroke}%
\pgfsetdash{}{0pt}%
\pgfpathmoveto{\pgfqpoint{0.625000in}{2.665449in}}%
\pgfpathlineto{\pgfqpoint{0.629727in}{2.672807in}}%
\pgfpathlineto{\pgfqpoint{0.625000in}{2.679391in}}%
\pgfusepath{stroke}%
\end{pgfscope}%
\begin{pgfscope}%
\pgfpathrectangle{\pgfqpoint{0.625000in}{0.550000in}}{\pgfqpoint{3.875000in}{3.850000in}} %
\pgfusepath{clip}%
\pgfsetbuttcap%
\pgfsetroundjoin%
\pgfsetlinewidth{0.250937pt}%
\definecolor{currentstroke}{rgb}{0.000000,0.000000,0.000000}%
\pgfsetstrokecolor{currentstroke}%
\pgfsetdash{}{0pt}%
\pgfpathmoveto{\pgfqpoint{0.625000in}{2.702653in}}%
\pgfpathlineto{\pgfqpoint{0.628093in}{2.711404in}}%
\pgfpathlineto{\pgfqpoint{0.630166in}{2.721053in}}%
\pgfpathlineto{\pgfqpoint{0.625000in}{2.729127in}}%
\pgfusepath{stroke}%
\end{pgfscope}%
\begin{pgfscope}%
\pgfpathrectangle{\pgfqpoint{0.625000in}{0.550000in}}{\pgfqpoint{3.875000in}{3.850000in}} %
\pgfusepath{clip}%
\pgfsetbuttcap%
\pgfsetroundjoin%
\pgfsetlinewidth{0.250937pt}%
\definecolor{currentstroke}{rgb}{0.000000,0.000000,0.000000}%
\pgfsetstrokecolor{currentstroke}%
\pgfsetdash{}{0pt}%
\pgfpathmoveto{\pgfqpoint{0.625000in}{2.737146in}}%
\pgfpathlineto{\pgfqpoint{0.625755in}{2.740351in}}%
\pgfpathlineto{\pgfqpoint{0.626388in}{2.750000in}}%
\pgfpathlineto{\pgfqpoint{0.625000in}{2.756778in}}%
\pgfusepath{stroke}%
\end{pgfscope}%
\begin{pgfscope}%
\pgfpathrectangle{\pgfqpoint{0.625000in}{0.550000in}}{\pgfqpoint{3.875000in}{3.850000in}} %
\pgfusepath{clip}%
\pgfsetbuttcap%
\pgfsetroundjoin%
\pgfsetlinewidth{0.250937pt}%
\definecolor{currentstroke}{rgb}{0.000000,0.000000,0.000000}%
\pgfsetstrokecolor{currentstroke}%
\pgfsetdash{}{0pt}%
\pgfpathmoveto{\pgfqpoint{0.625000in}{2.760559in}}%
\pgfpathlineto{\pgfqpoint{0.628393in}{2.769298in}}%
\pgfpathlineto{\pgfqpoint{0.627292in}{2.778947in}}%
\pgfpathlineto{\pgfqpoint{0.625000in}{2.779084in}}%
\pgfusepath{stroke}%
\end{pgfscope}%
\begin{pgfscope}%
\pgfpathrectangle{\pgfqpoint{0.625000in}{0.550000in}}{\pgfqpoint{3.875000in}{3.850000in}} %
\pgfusepath{clip}%
\pgfsetbuttcap%
\pgfsetroundjoin%
\pgfsetlinewidth{0.250937pt}%
\definecolor{currentstroke}{rgb}{0.000000,0.000000,0.000000}%
\pgfsetstrokecolor{currentstroke}%
\pgfsetdash{}{0pt}%
\pgfpathmoveto{\pgfqpoint{0.625000in}{2.811493in}}%
\pgfpathlineto{\pgfqpoint{0.627297in}{2.817544in}}%
\pgfpathlineto{\pgfqpoint{0.625861in}{2.827193in}}%
\pgfpathlineto{\pgfqpoint{0.626038in}{2.836842in}}%
\pgfpathlineto{\pgfqpoint{0.625000in}{2.841212in}}%
\pgfusepath{stroke}%
\end{pgfscope}%
\begin{pgfscope}%
\pgfpathrectangle{\pgfqpoint{0.625000in}{0.550000in}}{\pgfqpoint{3.875000in}{3.850000in}} %
\pgfusepath{clip}%
\pgfsetbuttcap%
\pgfsetroundjoin%
\pgfsetlinewidth{0.250937pt}%
\definecolor{currentstroke}{rgb}{0.000000,0.000000,0.000000}%
\pgfsetstrokecolor{currentstroke}%
\pgfsetdash{}{0pt}%
\pgfpathmoveto{\pgfqpoint{0.625000in}{2.857490in}}%
\pgfpathlineto{\pgfqpoint{0.631457in}{2.865789in}}%
\pgfpathlineto{\pgfqpoint{0.625000in}{2.874057in}}%
\pgfusepath{stroke}%
\end{pgfscope}%
\begin{pgfscope}%
\pgfpathrectangle{\pgfqpoint{0.625000in}{0.550000in}}{\pgfqpoint{3.875000in}{3.850000in}} %
\pgfusepath{clip}%
\pgfsetbuttcap%
\pgfsetroundjoin%
\pgfsetlinewidth{0.250937pt}%
\definecolor{currentstroke}{rgb}{0.000000,0.000000,0.000000}%
\pgfsetstrokecolor{currentstroke}%
\pgfsetdash{}{0pt}%
\pgfpathmoveto{\pgfqpoint{0.625000in}{2.901021in}}%
\pgfpathlineto{\pgfqpoint{0.625700in}{2.904386in}}%
\pgfpathlineto{\pgfqpoint{0.625248in}{2.914035in}}%
\pgfpathlineto{\pgfqpoint{0.625000in}{2.915902in}}%
\pgfusepath{stroke}%
\end{pgfscope}%
\begin{pgfscope}%
\pgfpathrectangle{\pgfqpoint{0.625000in}{0.550000in}}{\pgfqpoint{3.875000in}{3.850000in}} %
\pgfusepath{clip}%
\pgfsetbuttcap%
\pgfsetroundjoin%
\pgfsetlinewidth{0.250937pt}%
\definecolor{currentstroke}{rgb}{0.000000,0.000000,0.000000}%
\pgfsetstrokecolor{currentstroke}%
\pgfsetdash{}{0pt}%
\pgfpathmoveto{\pgfqpoint{0.625000in}{2.956313in}}%
\pgfpathlineto{\pgfqpoint{0.632011in}{2.962281in}}%
\pgfpathlineto{\pgfqpoint{0.626190in}{2.971930in}}%
\pgfpathlineto{\pgfqpoint{0.626323in}{2.981579in}}%
\pgfpathlineto{\pgfqpoint{0.625000in}{2.990195in}}%
\pgfusepath{stroke}%
\end{pgfscope}%
\begin{pgfscope}%
\pgfpathrectangle{\pgfqpoint{0.625000in}{0.550000in}}{\pgfqpoint{3.875000in}{3.850000in}} %
\pgfusepath{clip}%
\pgfsetbuttcap%
\pgfsetroundjoin%
\pgfsetlinewidth{0.250937pt}%
\definecolor{currentstroke}{rgb}{0.000000,0.000000,0.000000}%
\pgfsetstrokecolor{currentstroke}%
\pgfsetdash{}{0pt}%
\pgfpathmoveto{\pgfqpoint{0.625000in}{3.004041in}}%
\pgfpathlineto{\pgfqpoint{0.627571in}{3.010526in}}%
\pgfpathlineto{\pgfqpoint{0.627163in}{3.020175in}}%
\pgfpathlineto{\pgfqpoint{0.625000in}{3.022102in}}%
\pgfusepath{stroke}%
\end{pgfscope}%
\begin{pgfscope}%
\pgfpathrectangle{\pgfqpoint{0.625000in}{0.550000in}}{\pgfqpoint{3.875000in}{3.850000in}} %
\pgfusepath{clip}%
\pgfsetbuttcap%
\pgfsetroundjoin%
\pgfsetlinewidth{0.250937pt}%
\definecolor{currentstroke}{rgb}{0.000000,0.000000,0.000000}%
\pgfsetstrokecolor{currentstroke}%
\pgfsetdash{}{0pt}%
\pgfpathmoveto{\pgfqpoint{0.625000in}{3.053619in}}%
\pgfpathlineto{\pgfqpoint{0.629741in}{3.058772in}}%
\pgfpathlineto{\pgfqpoint{0.625000in}{3.061754in}}%
\pgfusepath{stroke}%
\end{pgfscope}%
\begin{pgfscope}%
\pgfpathrectangle{\pgfqpoint{0.625000in}{0.550000in}}{\pgfqpoint{3.875000in}{3.850000in}} %
\pgfusepath{clip}%
\pgfsetbuttcap%
\pgfsetroundjoin%
\pgfsetlinewidth{0.250937pt}%
\definecolor{currentstroke}{rgb}{0.000000,0.000000,0.000000}%
\pgfsetstrokecolor{currentstroke}%
\pgfsetdash{}{0pt}%
\pgfpathmoveto{\pgfqpoint{0.625000in}{3.106830in}}%
\pgfpathlineto{\pgfqpoint{0.629252in}{3.107018in}}%
\pgfpathlineto{\pgfqpoint{0.625000in}{3.114186in}}%
\pgfusepath{stroke}%
\end{pgfscope}%
\begin{pgfscope}%
\pgfpathrectangle{\pgfqpoint{0.625000in}{0.550000in}}{\pgfqpoint{3.875000in}{3.850000in}} %
\pgfusepath{clip}%
\pgfsetbuttcap%
\pgfsetroundjoin%
\pgfsetlinewidth{0.250937pt}%
\definecolor{currentstroke}{rgb}{0.000000,0.000000,0.000000}%
\pgfsetstrokecolor{currentstroke}%
\pgfsetdash{}{0pt}%
\pgfpathmoveto{\pgfqpoint{0.625000in}{3.122192in}}%
\pgfpathlineto{\pgfqpoint{0.627494in}{3.126316in}}%
\pgfpathlineto{\pgfqpoint{0.626800in}{3.135965in}}%
\pgfpathlineto{\pgfqpoint{0.625000in}{3.138779in}}%
\pgfusepath{stroke}%
\end{pgfscope}%
\begin{pgfscope}%
\pgfpathrectangle{\pgfqpoint{0.625000in}{0.550000in}}{\pgfqpoint{3.875000in}{3.850000in}} %
\pgfusepath{clip}%
\pgfsetbuttcap%
\pgfsetroundjoin%
\pgfsetlinewidth{0.250937pt}%
\definecolor{currentstroke}{rgb}{0.000000,0.000000,0.000000}%
\pgfsetstrokecolor{currentstroke}%
\pgfsetdash{}{0pt}%
\pgfpathmoveto{\pgfqpoint{0.625000in}{3.149220in}}%
\pgfpathlineto{\pgfqpoint{0.628614in}{3.155263in}}%
\pgfpathlineto{\pgfqpoint{0.628155in}{3.164912in}}%
\pgfpathlineto{\pgfqpoint{0.627662in}{3.174561in}}%
\pgfpathlineto{\pgfqpoint{0.625933in}{3.184211in}}%
\pgfpathlineto{\pgfqpoint{0.625000in}{3.188650in}}%
\pgfusepath{stroke}%
\end{pgfscope}%
\begin{pgfscope}%
\pgfpathrectangle{\pgfqpoint{0.625000in}{0.550000in}}{\pgfqpoint{3.875000in}{3.850000in}} %
\pgfusepath{clip}%
\pgfsetbuttcap%
\pgfsetroundjoin%
\pgfsetlinewidth{0.250937pt}%
\definecolor{currentstroke}{rgb}{0.000000,0.000000,0.000000}%
\pgfsetstrokecolor{currentstroke}%
\pgfsetdash{}{0pt}%
\pgfpathmoveto{\pgfqpoint{0.625000in}{3.195341in}}%
\pgfpathlineto{\pgfqpoint{0.630341in}{3.203509in}}%
\pgfpathlineto{\pgfqpoint{0.626209in}{3.213158in}}%
\pgfpathlineto{\pgfqpoint{0.626500in}{3.222807in}}%
\pgfpathlineto{\pgfqpoint{0.625000in}{3.227811in}}%
\pgfusepath{stroke}%
\end{pgfscope}%
\begin{pgfscope}%
\pgfpathrectangle{\pgfqpoint{0.625000in}{0.550000in}}{\pgfqpoint{3.875000in}{3.850000in}} %
\pgfusepath{clip}%
\pgfsetbuttcap%
\pgfsetroundjoin%
\pgfsetlinewidth{0.250937pt}%
\definecolor{currentstroke}{rgb}{0.000000,0.000000,0.000000}%
\pgfsetstrokecolor{currentstroke}%
\pgfsetdash{}{0pt}%
\pgfpathmoveto{\pgfqpoint{0.625000in}{3.236455in}}%
\pgfpathlineto{\pgfqpoint{0.625762in}{3.242105in}}%
\pgfpathlineto{\pgfqpoint{0.625000in}{3.242200in}}%
\pgfusepath{stroke}%
\end{pgfscope}%
\begin{pgfscope}%
\pgfpathrectangle{\pgfqpoint{0.625000in}{0.550000in}}{\pgfqpoint{3.875000in}{3.850000in}} %
\pgfusepath{clip}%
\pgfsetbuttcap%
\pgfsetroundjoin%
\pgfsetlinewidth{0.250937pt}%
\definecolor{currentstroke}{rgb}{0.000000,0.000000,0.000000}%
\pgfsetstrokecolor{currentstroke}%
\pgfsetdash{}{0pt}%
\pgfpathmoveto{\pgfqpoint{0.625000in}{3.261372in}}%
\pgfpathlineto{\pgfqpoint{0.625264in}{3.261404in}}%
\pgfpathlineto{\pgfqpoint{0.625000in}{3.264463in}}%
\pgfusepath{stroke}%
\end{pgfscope}%
\begin{pgfscope}%
\pgfpathrectangle{\pgfqpoint{0.625000in}{0.550000in}}{\pgfqpoint{3.875000in}{3.850000in}} %
\pgfusepath{clip}%
\pgfsetbuttcap%
\pgfsetroundjoin%
\pgfsetlinewidth{0.250937pt}%
\definecolor{currentstroke}{rgb}{0.000000,0.000000,0.000000}%
\pgfsetstrokecolor{currentstroke}%
\pgfsetdash{}{0pt}%
\pgfpathmoveto{\pgfqpoint{0.625000in}{3.287921in}}%
\pgfpathlineto{\pgfqpoint{0.626166in}{3.290351in}}%
\pgfpathlineto{\pgfqpoint{0.628466in}{3.300000in}}%
\pgfpathlineto{\pgfqpoint{0.625000in}{3.306972in}}%
\pgfusepath{stroke}%
\end{pgfscope}%
\begin{pgfscope}%
\pgfpathrectangle{\pgfqpoint{0.625000in}{0.550000in}}{\pgfqpoint{3.875000in}{3.850000in}} %
\pgfusepath{clip}%
\pgfsetbuttcap%
\pgfsetroundjoin%
\pgfsetlinewidth{0.250937pt}%
\definecolor{currentstroke}{rgb}{0.000000,0.000000,0.000000}%
\pgfsetstrokecolor{currentstroke}%
\pgfsetdash{}{0pt}%
\pgfpathmoveto{\pgfqpoint{0.625000in}{3.326189in}}%
\pgfpathlineto{\pgfqpoint{0.627430in}{3.328947in}}%
\pgfpathlineto{\pgfqpoint{0.625000in}{3.332299in}}%
\pgfusepath{stroke}%
\end{pgfscope}%
\begin{pgfscope}%
\pgfpathrectangle{\pgfqpoint{0.625000in}{0.550000in}}{\pgfqpoint{3.875000in}{3.850000in}} %
\pgfusepath{clip}%
\pgfsetbuttcap%
\pgfsetroundjoin%
\pgfsetlinewidth{0.250937pt}%
\definecolor{currentstroke}{rgb}{0.000000,0.000000,0.000000}%
\pgfsetstrokecolor{currentstroke}%
\pgfsetdash{}{0pt}%
\pgfpathmoveto{\pgfqpoint{0.625000in}{3.344437in}}%
\pgfpathlineto{\pgfqpoint{0.628054in}{3.348246in}}%
\pgfpathlineto{\pgfqpoint{0.627496in}{3.357895in}}%
\pgfpathlineto{\pgfqpoint{0.625991in}{3.367544in}}%
\pgfpathlineto{\pgfqpoint{0.625000in}{3.369036in}}%
\pgfusepath{stroke}%
\end{pgfscope}%
\begin{pgfscope}%
\pgfpathrectangle{\pgfqpoint{0.625000in}{0.550000in}}{\pgfqpoint{3.875000in}{3.850000in}} %
\pgfusepath{clip}%
\pgfsetbuttcap%
\pgfsetroundjoin%
\pgfsetlinewidth{0.250937pt}%
\definecolor{currentstroke}{rgb}{0.000000,0.000000,0.000000}%
\pgfsetstrokecolor{currentstroke}%
\pgfsetdash{}{0pt}%
\pgfpathmoveto{\pgfqpoint{0.625000in}{3.390435in}}%
\pgfpathlineto{\pgfqpoint{0.627412in}{3.396491in}}%
\pgfpathlineto{\pgfqpoint{0.625000in}{3.396601in}}%
\pgfusepath{stroke}%
\end{pgfscope}%
\begin{pgfscope}%
\pgfpathrectangle{\pgfqpoint{0.625000in}{0.550000in}}{\pgfqpoint{3.875000in}{3.850000in}} %
\pgfusepath{clip}%
\pgfsetbuttcap%
\pgfsetroundjoin%
\pgfsetlinewidth{0.250937pt}%
\definecolor{currentstroke}{rgb}{0.000000,0.000000,0.000000}%
\pgfsetstrokecolor{currentstroke}%
\pgfsetdash{}{0pt}%
\pgfpathmoveto{\pgfqpoint{0.625000in}{3.435617in}}%
\pgfpathlineto{\pgfqpoint{0.633732in}{3.444737in}}%
\pgfpathlineto{\pgfqpoint{0.625000in}{3.453427in}}%
\pgfusepath{stroke}%
\end{pgfscope}%
\begin{pgfscope}%
\pgfpathrectangle{\pgfqpoint{0.625000in}{0.550000in}}{\pgfqpoint{3.875000in}{3.850000in}} %
\pgfusepath{clip}%
\pgfsetbuttcap%
\pgfsetroundjoin%
\pgfsetlinewidth{0.250937pt}%
\definecolor{currentstroke}{rgb}{0.000000,0.000000,0.000000}%
\pgfsetstrokecolor{currentstroke}%
\pgfsetdash{}{0pt}%
\pgfpathmoveto{\pgfqpoint{0.625000in}{3.467577in}}%
\pgfpathlineto{\pgfqpoint{0.627969in}{3.473684in}}%
\pgfpathlineto{\pgfqpoint{0.628435in}{3.483333in}}%
\pgfpathlineto{\pgfqpoint{0.626874in}{3.492982in}}%
\pgfpathlineto{\pgfqpoint{0.625000in}{3.498307in}}%
\pgfusepath{stroke}%
\end{pgfscope}%
\begin{pgfscope}%
\pgfpathrectangle{\pgfqpoint{0.625000in}{0.550000in}}{\pgfqpoint{3.875000in}{3.850000in}} %
\pgfusepath{clip}%
\pgfsetbuttcap%
\pgfsetroundjoin%
\pgfsetlinewidth{0.250937pt}%
\definecolor{currentstroke}{rgb}{0.000000,0.000000,0.000000}%
\pgfsetstrokecolor{currentstroke}%
\pgfsetdash{}{0pt}%
\pgfpathmoveto{\pgfqpoint{0.625000in}{3.515193in}}%
\pgfpathlineto{\pgfqpoint{0.626114in}{3.521930in}}%
\pgfpathlineto{\pgfqpoint{0.625000in}{3.523192in}}%
\pgfusepath{stroke}%
\end{pgfscope}%
\begin{pgfscope}%
\pgfpathrectangle{\pgfqpoint{0.625000in}{0.550000in}}{\pgfqpoint{3.875000in}{3.850000in}} %
\pgfusepath{clip}%
\pgfsetbuttcap%
\pgfsetroundjoin%
\pgfsetlinewidth{0.250937pt}%
\definecolor{currentstroke}{rgb}{0.000000,0.000000,0.000000}%
\pgfsetstrokecolor{currentstroke}%
\pgfsetdash{}{0pt}%
\pgfpathmoveto{\pgfqpoint{0.625000in}{3.537562in}}%
\pgfpathlineto{\pgfqpoint{0.628237in}{3.541228in}}%
\pgfpathlineto{\pgfqpoint{0.626404in}{3.550877in}}%
\pgfpathlineto{\pgfqpoint{0.625000in}{3.550932in}}%
\pgfusepath{stroke}%
\end{pgfscope}%
\begin{pgfscope}%
\pgfpathrectangle{\pgfqpoint{0.625000in}{0.550000in}}{\pgfqpoint{3.875000in}{3.850000in}} %
\pgfusepath{clip}%
\pgfsetbuttcap%
\pgfsetroundjoin%
\pgfsetlinewidth{0.250937pt}%
\definecolor{currentstroke}{rgb}{0.000000,0.000000,0.000000}%
\pgfsetstrokecolor{currentstroke}%
\pgfsetdash{}{0pt}%
\pgfpathmoveto{\pgfqpoint{0.625000in}{3.570127in}}%
\pgfpathlineto{\pgfqpoint{0.625811in}{3.570175in}}%
\pgfpathlineto{\pgfqpoint{0.625000in}{3.574248in}}%
\pgfusepath{stroke}%
\end{pgfscope}%
\begin{pgfscope}%
\pgfpathrectangle{\pgfqpoint{0.625000in}{0.550000in}}{\pgfqpoint{3.875000in}{3.850000in}} %
\pgfusepath{clip}%
\pgfsetbuttcap%
\pgfsetroundjoin%
\pgfsetlinewidth{0.250937pt}%
\definecolor{currentstroke}{rgb}{0.000000,0.000000,0.000000}%
\pgfsetstrokecolor{currentstroke}%
\pgfsetdash{}{0pt}%
\pgfpathmoveto{\pgfqpoint{0.625000in}{3.589085in}}%
\pgfpathlineto{\pgfqpoint{0.625034in}{3.589474in}}%
\pgfpathlineto{\pgfqpoint{0.626009in}{3.599123in}}%
\pgfpathlineto{\pgfqpoint{0.625000in}{3.608072in}}%
\pgfusepath{stroke}%
\end{pgfscope}%
\begin{pgfscope}%
\pgfpathrectangle{\pgfqpoint{0.625000in}{0.550000in}}{\pgfqpoint{3.875000in}{3.850000in}} %
\pgfusepath{clip}%
\pgfsetbuttcap%
\pgfsetroundjoin%
\pgfsetlinewidth{0.250937pt}%
\definecolor{currentstroke}{rgb}{0.000000,0.000000,0.000000}%
\pgfsetstrokecolor{currentstroke}%
\pgfsetdash{}{0pt}%
\pgfpathmoveto{\pgfqpoint{0.625000in}{3.621377in}}%
\pgfpathlineto{\pgfqpoint{0.628403in}{3.628070in}}%
\pgfpathlineto{\pgfqpoint{0.631457in}{3.637719in}}%
\pgfpathlineto{\pgfqpoint{0.627718in}{3.647368in}}%
\pgfpathlineto{\pgfqpoint{0.625000in}{3.653351in}}%
\pgfusepath{stroke}%
\end{pgfscope}%
\begin{pgfscope}%
\pgfpathrectangle{\pgfqpoint{0.625000in}{0.550000in}}{\pgfqpoint{3.875000in}{3.850000in}} %
\pgfusepath{clip}%
\pgfsetbuttcap%
\pgfsetroundjoin%
\pgfsetlinewidth{0.250937pt}%
\definecolor{currentstroke}{rgb}{0.000000,0.000000,0.000000}%
\pgfsetstrokecolor{currentstroke}%
\pgfsetdash{}{0pt}%
\pgfpathmoveto{\pgfqpoint{0.625000in}{3.675653in}}%
\pgfpathlineto{\pgfqpoint{0.626180in}{3.676316in}}%
\pgfpathlineto{\pgfqpoint{0.630243in}{3.685965in}}%
\pgfpathlineto{\pgfqpoint{0.625000in}{3.694081in}}%
\pgfusepath{stroke}%
\end{pgfscope}%
\begin{pgfscope}%
\pgfpathrectangle{\pgfqpoint{0.625000in}{0.550000in}}{\pgfqpoint{3.875000in}{3.850000in}} %
\pgfusepath{clip}%
\pgfsetbuttcap%
\pgfsetroundjoin%
\pgfsetlinewidth{0.250937pt}%
\definecolor{currentstroke}{rgb}{0.000000,0.000000,0.000000}%
\pgfsetstrokecolor{currentstroke}%
\pgfsetdash{}{0pt}%
\pgfpathmoveto{\pgfqpoint{0.625000in}{3.726269in}}%
\pgfpathlineto{\pgfqpoint{0.627479in}{3.734211in}}%
\pgfpathlineto{\pgfqpoint{0.626617in}{3.743860in}}%
\pgfpathlineto{\pgfqpoint{0.625227in}{3.753509in}}%
\pgfpathlineto{\pgfqpoint{0.625000in}{3.754363in}}%
\pgfusepath{stroke}%
\end{pgfscope}%
\begin{pgfscope}%
\pgfpathrectangle{\pgfqpoint{0.625000in}{0.550000in}}{\pgfqpoint{3.875000in}{3.850000in}} %
\pgfusepath{clip}%
\pgfsetbuttcap%
\pgfsetroundjoin%
\pgfsetlinewidth{0.250937pt}%
\definecolor{currentstroke}{rgb}{0.000000,0.000000,0.000000}%
\pgfsetstrokecolor{currentstroke}%
\pgfsetdash{}{0pt}%
\pgfpathmoveto{\pgfqpoint{0.625000in}{3.785129in}}%
\pgfpathlineto{\pgfqpoint{0.626673in}{3.792105in}}%
\pgfpathlineto{\pgfqpoint{0.625000in}{3.797801in}}%
\pgfusepath{stroke}%
\end{pgfscope}%
\begin{pgfscope}%
\pgfpathrectangle{\pgfqpoint{0.625000in}{0.550000in}}{\pgfqpoint{3.875000in}{3.850000in}} %
\pgfusepath{clip}%
\pgfsetbuttcap%
\pgfsetroundjoin%
\pgfsetlinewidth{0.250937pt}%
\definecolor{currentstroke}{rgb}{0.000000,0.000000,0.000000}%
\pgfsetstrokecolor{currentstroke}%
\pgfsetdash{}{0pt}%
\pgfpathmoveto{\pgfqpoint{0.625000in}{3.816084in}}%
\pgfpathlineto{\pgfqpoint{0.626526in}{3.821053in}}%
\pgfpathlineto{\pgfqpoint{0.629731in}{3.830702in}}%
\pgfpathlineto{\pgfqpoint{0.627292in}{3.840351in}}%
\pgfpathlineto{\pgfqpoint{0.625000in}{3.846502in}}%
\pgfusepath{stroke}%
\end{pgfscope}%
\begin{pgfscope}%
\pgfpathrectangle{\pgfqpoint{0.625000in}{0.550000in}}{\pgfqpoint{3.875000in}{3.850000in}} %
\pgfusepath{clip}%
\pgfsetbuttcap%
\pgfsetroundjoin%
\pgfsetlinewidth{0.250937pt}%
\definecolor{currentstroke}{rgb}{0.000000,0.000000,0.000000}%
\pgfsetstrokecolor{currentstroke}%
\pgfsetdash{}{0pt}%
\pgfpathmoveto{\pgfqpoint{0.625000in}{3.858479in}}%
\pgfpathlineto{\pgfqpoint{0.625309in}{3.859649in}}%
\pgfpathlineto{\pgfqpoint{0.625000in}{3.859659in}}%
\pgfusepath{stroke}%
\end{pgfscope}%
\begin{pgfscope}%
\pgfpathrectangle{\pgfqpoint{0.625000in}{0.550000in}}{\pgfqpoint{3.875000in}{3.850000in}} %
\pgfusepath{clip}%
\pgfsetbuttcap%
\pgfsetroundjoin%
\pgfsetlinewidth{0.250937pt}%
\definecolor{currentstroke}{rgb}{0.000000,0.000000,0.000000}%
\pgfsetstrokecolor{currentstroke}%
\pgfsetdash{}{0pt}%
\pgfpathmoveto{\pgfqpoint{0.625000in}{3.878899in}}%
\pgfpathlineto{\pgfqpoint{0.625844in}{3.878947in}}%
\pgfpathlineto{\pgfqpoint{0.625000in}{3.882965in}}%
\pgfusepath{stroke}%
\end{pgfscope}%
\begin{pgfscope}%
\pgfpathrectangle{\pgfqpoint{0.625000in}{0.550000in}}{\pgfqpoint{3.875000in}{3.850000in}} %
\pgfusepath{clip}%
\pgfsetbuttcap%
\pgfsetroundjoin%
\pgfsetlinewidth{0.250937pt}%
\definecolor{currentstroke}{rgb}{0.000000,0.000000,0.000000}%
\pgfsetstrokecolor{currentstroke}%
\pgfsetdash{}{0pt}%
\pgfpathmoveto{\pgfqpoint{0.625000in}{3.894381in}}%
\pgfpathlineto{\pgfqpoint{0.626056in}{3.898246in}}%
\pgfpathlineto{\pgfqpoint{0.626322in}{3.907895in}}%
\pgfpathlineto{\pgfqpoint{0.625000in}{3.909693in}}%
\pgfusepath{stroke}%
\end{pgfscope}%
\begin{pgfscope}%
\pgfpathrectangle{\pgfqpoint{0.625000in}{0.550000in}}{\pgfqpoint{3.875000in}{3.850000in}} %
\pgfusepath{clip}%
\pgfsetbuttcap%
\pgfsetroundjoin%
\pgfsetlinewidth{0.250937pt}%
\definecolor{currentstroke}{rgb}{0.000000,0.000000,0.000000}%
\pgfsetstrokecolor{currentstroke}%
\pgfsetdash{}{0pt}%
\pgfpathmoveto{\pgfqpoint{0.625000in}{3.920589in}}%
\pgfpathlineto{\pgfqpoint{0.632011in}{3.927193in}}%
\pgfpathlineto{\pgfqpoint{0.626569in}{3.936842in}}%
\pgfpathlineto{\pgfqpoint{0.628337in}{3.946491in}}%
\pgfpathlineto{\pgfqpoint{0.627482in}{3.956140in}}%
\pgfpathlineto{\pgfqpoint{0.625000in}{3.962513in}}%
\pgfusepath{stroke}%
\end{pgfscope}%
\begin{pgfscope}%
\pgfpathrectangle{\pgfqpoint{0.625000in}{0.550000in}}{\pgfqpoint{3.875000in}{3.850000in}} %
\pgfusepath{clip}%
\pgfsetbuttcap%
\pgfsetroundjoin%
\pgfsetlinewidth{0.250937pt}%
\definecolor{currentstroke}{rgb}{0.000000,0.000000,0.000000}%
\pgfsetstrokecolor{currentstroke}%
\pgfsetdash{}{0pt}%
\pgfpathmoveto{\pgfqpoint{0.625000in}{3.968720in}}%
\pgfpathlineto{\pgfqpoint{0.627988in}{3.975439in}}%
\pgfpathlineto{\pgfqpoint{0.626265in}{3.985088in}}%
\pgfpathlineto{\pgfqpoint{0.625000in}{3.988002in}}%
\pgfusepath{stroke}%
\end{pgfscope}%
\begin{pgfscope}%
\pgfpathrectangle{\pgfqpoint{0.625000in}{0.550000in}}{\pgfqpoint{3.875000in}{3.850000in}} %
\pgfusepath{clip}%
\pgfsetbuttcap%
\pgfsetroundjoin%
\pgfsetlinewidth{0.250937pt}%
\definecolor{currentstroke}{rgb}{0.000000,0.000000,0.000000}%
\pgfsetstrokecolor{currentstroke}%
\pgfsetdash{}{0pt}%
\pgfpathmoveto{\pgfqpoint{0.625000in}{4.008981in}}%
\pgfpathlineto{\pgfqpoint{0.625603in}{4.014035in}}%
\pgfpathlineto{\pgfqpoint{0.625000in}{4.014108in}}%
\pgfusepath{stroke}%
\end{pgfscope}%
\begin{pgfscope}%
\pgfpathrectangle{\pgfqpoint{0.625000in}{0.550000in}}{\pgfqpoint{3.875000in}{3.850000in}} %
\pgfusepath{clip}%
\pgfsetbuttcap%
\pgfsetroundjoin%
\pgfsetlinewidth{0.250937pt}%
\definecolor{currentstroke}{rgb}{0.000000,0.000000,0.000000}%
\pgfsetstrokecolor{currentstroke}%
\pgfsetdash{}{0pt}%
\pgfpathmoveto{\pgfqpoint{0.625000in}{4.061513in}}%
\pgfpathlineto{\pgfqpoint{0.626197in}{4.062281in}}%
\pgfpathlineto{\pgfqpoint{0.627174in}{4.071930in}}%
\pgfpathlineto{\pgfqpoint{0.625000in}{4.077848in}}%
\pgfusepath{stroke}%
\end{pgfscope}%
\begin{pgfscope}%
\pgfpathrectangle{\pgfqpoint{0.625000in}{0.550000in}}{\pgfqpoint{3.875000in}{3.850000in}} %
\pgfusepath{clip}%
\pgfsetbuttcap%
\pgfsetroundjoin%
\pgfsetlinewidth{0.250937pt}%
\definecolor{currentstroke}{rgb}{0.000000,0.000000,0.000000}%
\pgfsetstrokecolor{currentstroke}%
\pgfsetdash{}{0pt}%
\pgfpathmoveto{\pgfqpoint{0.625000in}{4.091764in}}%
\pgfpathlineto{\pgfqpoint{0.628214in}{4.100877in}}%
\pgfpathlineto{\pgfqpoint{0.626202in}{4.110526in}}%
\pgfpathlineto{\pgfqpoint{0.627849in}{4.120175in}}%
\pgfpathlineto{\pgfqpoint{0.626459in}{4.129825in}}%
\pgfpathlineto{\pgfqpoint{0.625524in}{4.139474in}}%
\pgfpathlineto{\pgfqpoint{0.625000in}{4.141522in}}%
\pgfusepath{stroke}%
\end{pgfscope}%
\begin{pgfscope}%
\pgfpathrectangle{\pgfqpoint{0.625000in}{0.550000in}}{\pgfqpoint{3.875000in}{3.850000in}} %
\pgfusepath{clip}%
\pgfsetbuttcap%
\pgfsetroundjoin%
\pgfsetlinewidth{0.250937pt}%
\definecolor{currentstroke}{rgb}{0.000000,0.000000,0.000000}%
\pgfsetstrokecolor{currentstroke}%
\pgfsetdash{}{0pt}%
\pgfpathmoveto{\pgfqpoint{0.625000in}{4.160222in}}%
\pgfpathlineto{\pgfqpoint{0.630403in}{4.168421in}}%
\pgfpathlineto{\pgfqpoint{0.625000in}{4.168798in}}%
\pgfusepath{stroke}%
\end{pgfscope}%
\begin{pgfscope}%
\pgfpathrectangle{\pgfqpoint{0.625000in}{0.550000in}}{\pgfqpoint{3.875000in}{3.850000in}} %
\pgfusepath{clip}%
\pgfsetbuttcap%
\pgfsetroundjoin%
\pgfsetlinewidth{0.250937pt}%
\definecolor{currentstroke}{rgb}{0.000000,0.000000,0.000000}%
\pgfsetstrokecolor{currentstroke}%
\pgfsetdash{}{0pt}%
\pgfpathmoveto{\pgfqpoint{0.625000in}{4.199794in}}%
\pgfpathlineto{\pgfqpoint{0.628088in}{4.207018in}}%
\pgfpathlineto{\pgfqpoint{0.629745in}{4.216667in}}%
\pgfpathlineto{\pgfqpoint{0.628005in}{4.226316in}}%
\pgfpathlineto{\pgfqpoint{0.625000in}{4.233466in}}%
\pgfusepath{stroke}%
\end{pgfscope}%
\begin{pgfscope}%
\pgfpathrectangle{\pgfqpoint{0.625000in}{0.550000in}}{\pgfqpoint{3.875000in}{3.850000in}} %
\pgfusepath{clip}%
\pgfsetbuttcap%
\pgfsetroundjoin%
\pgfsetlinewidth{0.250937pt}%
\definecolor{currentstroke}{rgb}{0.000000,0.000000,0.000000}%
\pgfsetstrokecolor{currentstroke}%
\pgfsetdash{}{0pt}%
\pgfpathmoveto{\pgfqpoint{0.625000in}{4.241575in}}%
\pgfpathlineto{\pgfqpoint{0.626095in}{4.245614in}}%
\pgfpathlineto{\pgfqpoint{0.627842in}{4.255263in}}%
\pgfpathlineto{\pgfqpoint{0.625241in}{4.264912in}}%
\pgfpathlineto{\pgfqpoint{0.625000in}{4.266281in}}%
\pgfusepath{stroke}%
\end{pgfscope}%
\begin{pgfscope}%
\pgfpathrectangle{\pgfqpoint{0.625000in}{0.550000in}}{\pgfqpoint{3.875000in}{3.850000in}} %
\pgfusepath{clip}%
\pgfsetbuttcap%
\pgfsetroundjoin%
\pgfsetlinewidth{0.250937pt}%
\definecolor{currentstroke}{rgb}{0.000000,0.000000,0.000000}%
\pgfsetstrokecolor{currentstroke}%
\pgfsetdash{}{0pt}%
\pgfpathmoveto{\pgfqpoint{0.625000in}{4.291603in}}%
\pgfpathlineto{\pgfqpoint{0.626161in}{4.293860in}}%
\pgfpathlineto{\pgfqpoint{0.625849in}{4.303509in}}%
\pgfpathlineto{\pgfqpoint{0.628666in}{4.313158in}}%
\pgfpathlineto{\pgfqpoint{0.625000in}{4.321110in}}%
\pgfusepath{stroke}%
\end{pgfscope}%
\begin{pgfscope}%
\pgfpathrectangle{\pgfqpoint{0.625000in}{0.550000in}}{\pgfqpoint{3.875000in}{3.850000in}} %
\pgfusepath{clip}%
\pgfsetbuttcap%
\pgfsetroundjoin%
\pgfsetlinewidth{0.250937pt}%
\definecolor{currentstroke}{rgb}{0.000000,0.000000,0.000000}%
\pgfsetstrokecolor{currentstroke}%
\pgfsetdash{}{0pt}%
\pgfpathmoveto{\pgfqpoint{0.625000in}{4.354578in}}%
\pgfpathlineto{\pgfqpoint{0.629056in}{4.361404in}}%
\pgfpathlineto{\pgfqpoint{0.627045in}{4.371053in}}%
\pgfpathlineto{\pgfqpoint{0.625000in}{4.374135in}}%
\pgfusepath{stroke}%
\end{pgfscope}%
\begin{pgfscope}%
\pgfpathrectangle{\pgfqpoint{0.625000in}{0.550000in}}{\pgfqpoint{3.875000in}{3.850000in}} %
\pgfusepath{clip}%
\pgfsetbuttcap%
\pgfsetroundjoin%
\pgfsetlinewidth{0.250937pt}%
\definecolor{currentstroke}{rgb}{0.000000,0.000000,0.000000}%
\pgfsetstrokecolor{currentstroke}%
\pgfsetdash{}{0pt}%
\pgfpathmoveto{\pgfqpoint{0.625000in}{4.395164in}}%
\pgfpathlineto{\pgfqpoint{0.633658in}{4.400000in}}%
\pgfusepath{stroke}%
\end{pgfscope}%
\begin{pgfscope}%
\pgfpathrectangle{\pgfqpoint{0.625000in}{0.550000in}}{\pgfqpoint{3.875000in}{3.850000in}} %
\pgfusepath{clip}%
\pgfsetbuttcap%
\pgfsetroundjoin%
\pgfsetlinewidth{0.250937pt}%
\definecolor{currentstroke}{rgb}{0.000000,0.000000,0.000000}%
\pgfsetstrokecolor{currentstroke}%
\pgfsetdash{}{0pt}%
\pgfpathmoveto{\pgfqpoint{0.634211in}{0.550000in}}%
\pgfpathlineto{\pgfqpoint{0.625000in}{0.559388in}}%
\pgfusepath{stroke}%
\end{pgfscope}%
\begin{pgfscope}%
\pgfpathrectangle{\pgfqpoint{0.625000in}{0.550000in}}{\pgfqpoint{3.875000in}{3.850000in}} %
\pgfusepath{clip}%
\pgfsetbuttcap%
\pgfsetroundjoin%
\pgfsetlinewidth{0.250937pt}%
\definecolor{currentstroke}{rgb}{0.000000,0.000000,0.000000}%
\pgfsetstrokecolor{currentstroke}%
\pgfsetdash{}{0pt}%
\pgfpathmoveto{\pgfqpoint{0.625000in}{0.591563in}}%
\pgfpathlineto{\pgfqpoint{0.626931in}{0.598246in}}%
\pgfpathlineto{\pgfqpoint{0.625000in}{0.601495in}}%
\pgfusepath{stroke}%
\end{pgfscope}%
\begin{pgfscope}%
\pgfpathrectangle{\pgfqpoint{0.625000in}{0.550000in}}{\pgfqpoint{3.875000in}{3.850000in}} %
\pgfusepath{clip}%
\pgfsetbuttcap%
\pgfsetroundjoin%
\pgfsetlinewidth{0.250937pt}%
\definecolor{currentstroke}{rgb}{0.000000,0.000000,0.000000}%
\pgfsetstrokecolor{currentstroke}%
\pgfsetdash{}{0pt}%
\pgfpathmoveto{\pgfqpoint{0.625000in}{0.641813in}}%
\pgfpathlineto{\pgfqpoint{0.627157in}{0.646491in}}%
\pgfpathlineto{\pgfqpoint{0.625000in}{0.653283in}}%
\pgfusepath{stroke}%
\end{pgfscope}%
\begin{pgfscope}%
\pgfpathrectangle{\pgfqpoint{0.625000in}{0.550000in}}{\pgfqpoint{3.875000in}{3.850000in}} %
\pgfusepath{clip}%
\pgfsetbuttcap%
\pgfsetroundjoin%
\pgfsetlinewidth{0.250937pt}%
\definecolor{currentstroke}{rgb}{0.000000,0.000000,0.000000}%
\pgfsetstrokecolor{currentstroke}%
\pgfsetdash{}{0pt}%
\pgfpathmoveto{\pgfqpoint{0.625000in}{0.699266in}}%
\pgfpathlineto{\pgfqpoint{0.626409in}{0.704386in}}%
\pgfpathlineto{\pgfqpoint{0.625000in}{0.711091in}}%
\pgfusepath{stroke}%
\end{pgfscope}%
\begin{pgfscope}%
\pgfpathrectangle{\pgfqpoint{0.625000in}{0.550000in}}{\pgfqpoint{3.875000in}{3.850000in}} %
\pgfusepath{clip}%
\pgfsetbuttcap%
\pgfsetroundjoin%
\pgfsetlinewidth{0.250937pt}%
\definecolor{currentstroke}{rgb}{0.000000,0.000000,0.000000}%
\pgfsetstrokecolor{currentstroke}%
\pgfsetdash{}{0pt}%
\pgfpathmoveto{\pgfqpoint{0.625000in}{0.729348in}}%
\pgfpathlineto{\pgfqpoint{0.626675in}{0.733333in}}%
\pgfpathlineto{\pgfqpoint{0.628501in}{0.742982in}}%
\pgfpathlineto{\pgfqpoint{0.626775in}{0.752632in}}%
\pgfpathlineto{\pgfqpoint{0.625000in}{0.756783in}}%
\pgfusepath{stroke}%
\end{pgfscope}%
\begin{pgfscope}%
\pgfpathrectangle{\pgfqpoint{0.625000in}{0.550000in}}{\pgfqpoint{3.875000in}{3.850000in}} %
\pgfusepath{clip}%
\pgfsetbuttcap%
\pgfsetroundjoin%
\pgfsetlinewidth{0.250937pt}%
\definecolor{currentstroke}{rgb}{0.000000,0.000000,0.000000}%
\pgfsetstrokecolor{currentstroke}%
\pgfsetdash{}{0pt}%
\pgfpathmoveto{\pgfqpoint{0.625000in}{0.790961in}}%
\pgfpathlineto{\pgfqpoint{0.629193in}{0.791228in}}%
\pgfpathlineto{\pgfqpoint{0.625000in}{0.797592in}}%
\pgfusepath{stroke}%
\end{pgfscope}%
\begin{pgfscope}%
\pgfpathrectangle{\pgfqpoint{0.625000in}{0.550000in}}{\pgfqpoint{3.875000in}{3.850000in}} %
\pgfusepath{clip}%
\pgfsetbuttcap%
\pgfsetroundjoin%
\pgfsetlinewidth{0.250937pt}%
\definecolor{currentstroke}{rgb}{0.000000,0.000000,0.000000}%
\pgfsetstrokecolor{currentstroke}%
\pgfsetdash{}{0pt}%
\pgfpathmoveto{\pgfqpoint{0.625000in}{0.822876in}}%
\pgfpathlineto{\pgfqpoint{0.625743in}{0.829825in}}%
\pgfpathlineto{\pgfqpoint{0.626676in}{0.839474in}}%
\pgfpathlineto{\pgfqpoint{0.625000in}{0.847101in}}%
\pgfusepath{stroke}%
\end{pgfscope}%
\begin{pgfscope}%
\pgfpathrectangle{\pgfqpoint{0.625000in}{0.550000in}}{\pgfqpoint{3.875000in}{3.850000in}} %
\pgfusepath{clip}%
\pgfsetbuttcap%
\pgfsetroundjoin%
\pgfsetlinewidth{0.250937pt}%
\definecolor{currentstroke}{rgb}{0.000000,0.000000,0.000000}%
\pgfsetstrokecolor{currentstroke}%
\pgfsetdash{}{0pt}%
\pgfpathmoveto{\pgfqpoint{0.625000in}{0.851807in}}%
\pgfpathlineto{\pgfqpoint{0.626593in}{0.858772in}}%
\pgfpathlineto{\pgfqpoint{0.625000in}{0.863289in}}%
\pgfusepath{stroke}%
\end{pgfscope}%
\begin{pgfscope}%
\pgfpathrectangle{\pgfqpoint{0.625000in}{0.550000in}}{\pgfqpoint{3.875000in}{3.850000in}} %
\pgfusepath{clip}%
\pgfsetbuttcap%
\pgfsetroundjoin%
\pgfsetlinewidth{0.250937pt}%
\definecolor{currentstroke}{rgb}{0.000000,0.000000,0.000000}%
\pgfsetstrokecolor{currentstroke}%
\pgfsetdash{}{0pt}%
\pgfpathmoveto{\pgfqpoint{0.625000in}{0.886526in}}%
\pgfpathlineto{\pgfqpoint{0.625438in}{0.887719in}}%
\pgfpathlineto{\pgfqpoint{0.625000in}{0.891434in}}%
\pgfusepath{stroke}%
\end{pgfscope}%
\begin{pgfscope}%
\pgfpathrectangle{\pgfqpoint{0.625000in}{0.550000in}}{\pgfqpoint{3.875000in}{3.850000in}} %
\pgfusepath{clip}%
\pgfsetbuttcap%
\pgfsetroundjoin%
\pgfsetlinewidth{0.250937pt}%
\definecolor{currentstroke}{rgb}{0.000000,0.000000,0.000000}%
\pgfsetstrokecolor{currentstroke}%
\pgfsetdash{}{0pt}%
\pgfpathmoveto{\pgfqpoint{0.625000in}{0.977768in}}%
\pgfpathlineto{\pgfqpoint{0.626338in}{0.984211in}}%
\pgfpathlineto{\pgfqpoint{0.625000in}{0.987218in}}%
\pgfusepath{stroke}%
\end{pgfscope}%
\begin{pgfscope}%
\pgfpathrectangle{\pgfqpoint{0.625000in}{0.550000in}}{\pgfqpoint{3.875000in}{3.850000in}} %
\pgfusepath{clip}%
\pgfsetbuttcap%
\pgfsetroundjoin%
\pgfsetlinewidth{0.250937pt}%
\definecolor{currentstroke}{rgb}{0.000000,0.000000,0.000000}%
\pgfsetstrokecolor{currentstroke}%
\pgfsetdash{}{0pt}%
\pgfpathmoveto{\pgfqpoint{0.625000in}{1.001285in}}%
\pgfpathlineto{\pgfqpoint{0.625866in}{1.003509in}}%
\pgfpathlineto{\pgfqpoint{0.626654in}{1.013158in}}%
\pgfpathlineto{\pgfqpoint{0.625000in}{1.019791in}}%
\pgfusepath{stroke}%
\end{pgfscope}%
\begin{pgfscope}%
\pgfpathrectangle{\pgfqpoint{0.625000in}{0.550000in}}{\pgfqpoint{3.875000in}{3.850000in}} %
\pgfusepath{clip}%
\pgfsetbuttcap%
\pgfsetroundjoin%
\pgfsetlinewidth{0.250937pt}%
\definecolor{currentstroke}{rgb}{0.000000,0.000000,0.000000}%
\pgfsetstrokecolor{currentstroke}%
\pgfsetdash{}{0pt}%
\pgfpathmoveto{\pgfqpoint{0.625000in}{1.023654in}}%
\pgfpathlineto{\pgfqpoint{0.630770in}{1.032456in}}%
\pgfpathlineto{\pgfqpoint{0.625000in}{1.037890in}}%
\pgfusepath{stroke}%
\end{pgfscope}%
\begin{pgfscope}%
\pgfpathrectangle{\pgfqpoint{0.625000in}{0.550000in}}{\pgfqpoint{3.875000in}{3.850000in}} %
\pgfusepath{clip}%
\pgfsetbuttcap%
\pgfsetroundjoin%
\pgfsetlinewidth{0.250937pt}%
\definecolor{currentstroke}{rgb}{0.000000,0.000000,0.000000}%
\pgfsetstrokecolor{currentstroke}%
\pgfsetdash{}{0pt}%
\pgfpathmoveto{\pgfqpoint{0.625000in}{1.117577in}}%
\pgfpathlineto{\pgfqpoint{0.625641in}{1.119298in}}%
\pgfpathlineto{\pgfqpoint{0.628491in}{1.128947in}}%
\pgfpathlineto{\pgfqpoint{0.625000in}{1.138072in}}%
\pgfusepath{stroke}%
\end{pgfscope}%
\begin{pgfscope}%
\pgfpathrectangle{\pgfqpoint{0.625000in}{0.550000in}}{\pgfqpoint{3.875000in}{3.850000in}} %
\pgfusepath{clip}%
\pgfsetbuttcap%
\pgfsetroundjoin%
\pgfsetlinewidth{0.250937pt}%
\definecolor{currentstroke}{rgb}{0.000000,0.000000,0.000000}%
\pgfsetstrokecolor{currentstroke}%
\pgfsetdash{}{0pt}%
\pgfpathmoveto{\pgfqpoint{0.625000in}{1.164721in}}%
\pgfpathlineto{\pgfqpoint{0.625829in}{1.167544in}}%
\pgfpathlineto{\pgfqpoint{0.625000in}{1.171002in}}%
\pgfusepath{stroke}%
\end{pgfscope}%
\begin{pgfscope}%
\pgfpathrectangle{\pgfqpoint{0.625000in}{0.550000in}}{\pgfqpoint{3.875000in}{3.850000in}} %
\pgfusepath{clip}%
\pgfsetbuttcap%
\pgfsetroundjoin%
\pgfsetlinewidth{0.250937pt}%
\definecolor{currentstroke}{rgb}{0.000000,0.000000,0.000000}%
\pgfsetstrokecolor{currentstroke}%
\pgfsetdash{}{0pt}%
\pgfpathmoveto{\pgfqpoint{0.625000in}{1.208422in}}%
\pgfpathlineto{\pgfqpoint{0.625915in}{1.215789in}}%
\pgfpathlineto{\pgfqpoint{0.626459in}{1.225439in}}%
\pgfpathlineto{\pgfqpoint{0.625000in}{1.230111in}}%
\pgfusepath{stroke}%
\end{pgfscope}%
\begin{pgfscope}%
\pgfpathrectangle{\pgfqpoint{0.625000in}{0.550000in}}{\pgfqpoint{3.875000in}{3.850000in}} %
\pgfusepath{clip}%
\pgfsetbuttcap%
\pgfsetroundjoin%
\pgfsetlinewidth{0.250937pt}%
\definecolor{currentstroke}{rgb}{0.000000,0.000000,0.000000}%
\pgfsetstrokecolor{currentstroke}%
\pgfsetdash{}{0pt}%
\pgfpathmoveto{\pgfqpoint{0.625000in}{1.267510in}}%
\pgfpathlineto{\pgfqpoint{0.628989in}{1.273684in}}%
\pgfpathlineto{\pgfqpoint{0.625000in}{1.282247in}}%
\pgfusepath{stroke}%
\end{pgfscope}%
\begin{pgfscope}%
\pgfpathrectangle{\pgfqpoint{0.625000in}{0.550000in}}{\pgfqpoint{3.875000in}{3.850000in}} %
\pgfusepath{clip}%
\pgfsetbuttcap%
\pgfsetroundjoin%
\pgfsetlinewidth{0.250937pt}%
\definecolor{currentstroke}{rgb}{0.000000,0.000000,0.000000}%
\pgfsetstrokecolor{currentstroke}%
\pgfsetdash{}{0pt}%
\pgfpathmoveto{\pgfqpoint{0.625000in}{1.310942in}}%
\pgfpathlineto{\pgfqpoint{0.625608in}{1.312281in}}%
\pgfpathlineto{\pgfqpoint{0.630216in}{1.321930in}}%
\pgfpathlineto{\pgfqpoint{0.626500in}{1.331579in}}%
\pgfpathlineto{\pgfqpoint{0.625000in}{1.334530in}}%
\pgfusepath{stroke}%
\end{pgfscope}%
\begin{pgfscope}%
\pgfpathrectangle{\pgfqpoint{0.625000in}{0.550000in}}{\pgfqpoint{3.875000in}{3.850000in}} %
\pgfusepath{clip}%
\pgfsetbuttcap%
\pgfsetroundjoin%
\pgfsetlinewidth{0.250937pt}%
\definecolor{currentstroke}{rgb}{0.000000,0.000000,0.000000}%
\pgfsetstrokecolor{currentstroke}%
\pgfsetdash{}{0pt}%
\pgfpathmoveto{\pgfqpoint{0.625000in}{1.410700in}}%
\pgfpathlineto{\pgfqpoint{0.626905in}{1.418421in}}%
\pgfpathlineto{\pgfqpoint{0.625000in}{1.420578in}}%
\pgfusepath{stroke}%
\end{pgfscope}%
\begin{pgfscope}%
\pgfpathrectangle{\pgfqpoint{0.625000in}{0.550000in}}{\pgfqpoint{3.875000in}{3.850000in}} %
\pgfusepath{clip}%
\pgfsetbuttcap%
\pgfsetroundjoin%
\pgfsetlinewidth{0.250937pt}%
\definecolor{currentstroke}{rgb}{0.000000,0.000000,0.000000}%
\pgfsetstrokecolor{currentstroke}%
\pgfsetdash{}{0pt}%
\pgfpathmoveto{\pgfqpoint{0.625000in}{1.466930in}}%
\pgfpathlineto{\pgfqpoint{0.626703in}{1.476316in}}%
\pgfpathlineto{\pgfqpoint{0.625788in}{1.485965in}}%
\pgfpathlineto{\pgfqpoint{0.625000in}{1.487586in}}%
\pgfusepath{stroke}%
\end{pgfscope}%
\begin{pgfscope}%
\pgfpathrectangle{\pgfqpoint{0.625000in}{0.550000in}}{\pgfqpoint{3.875000in}{3.850000in}} %
\pgfusepath{clip}%
\pgfsetbuttcap%
\pgfsetroundjoin%
\pgfsetlinewidth{0.250937pt}%
\definecolor{currentstroke}{rgb}{0.000000,0.000000,0.000000}%
\pgfsetstrokecolor{currentstroke}%
\pgfsetdash{}{0pt}%
\pgfpathmoveto{\pgfqpoint{0.625000in}{1.507457in}}%
\pgfpathlineto{\pgfqpoint{0.632490in}{1.514912in}}%
\pgfpathlineto{\pgfqpoint{0.625000in}{1.522735in}}%
\pgfusepath{stroke}%
\end{pgfscope}%
\begin{pgfscope}%
\pgfpathrectangle{\pgfqpoint{0.625000in}{0.550000in}}{\pgfqpoint{3.875000in}{3.850000in}} %
\pgfusepath{clip}%
\pgfsetbuttcap%
\pgfsetroundjoin%
\pgfsetlinewidth{0.250937pt}%
\definecolor{currentstroke}{rgb}{0.000000,0.000000,0.000000}%
\pgfsetstrokecolor{currentstroke}%
\pgfsetdash{}{0pt}%
\pgfpathmoveto{\pgfqpoint{0.625000in}{1.563131in}}%
\pgfpathlineto{\pgfqpoint{0.625600in}{1.563158in}}%
\pgfpathlineto{\pgfqpoint{0.625000in}{1.564665in}}%
\pgfusepath{stroke}%
\end{pgfscope}%
\begin{pgfscope}%
\pgfpathrectangle{\pgfqpoint{0.625000in}{0.550000in}}{\pgfqpoint{3.875000in}{3.850000in}} %
\pgfusepath{clip}%
\pgfsetbuttcap%
\pgfsetroundjoin%
\pgfsetlinewidth{0.250937pt}%
\definecolor{currentstroke}{rgb}{0.000000,0.000000,0.000000}%
\pgfsetstrokecolor{currentstroke}%
\pgfsetdash{}{0pt}%
\pgfpathmoveto{\pgfqpoint{0.625000in}{1.596725in}}%
\pgfpathlineto{\pgfqpoint{0.625761in}{1.601754in}}%
\pgfpathlineto{\pgfqpoint{0.626797in}{1.611404in}}%
\pgfpathlineto{\pgfqpoint{0.625000in}{1.613644in}}%
\pgfusepath{stroke}%
\end{pgfscope}%
\begin{pgfscope}%
\pgfpathrectangle{\pgfqpoint{0.625000in}{0.550000in}}{\pgfqpoint{3.875000in}{3.850000in}} %
\pgfusepath{clip}%
\pgfsetbuttcap%
\pgfsetroundjoin%
\pgfsetlinewidth{0.250937pt}%
\definecolor{currentstroke}{rgb}{0.000000,0.000000,0.000000}%
\pgfsetstrokecolor{currentstroke}%
\pgfsetdash{}{0pt}%
\pgfpathmoveto{\pgfqpoint{0.625000in}{1.629040in}}%
\pgfpathlineto{\pgfqpoint{0.626205in}{1.630702in}}%
\pgfpathlineto{\pgfqpoint{0.625000in}{1.632069in}}%
\pgfusepath{stroke}%
\end{pgfscope}%
\begin{pgfscope}%
\pgfpathrectangle{\pgfqpoint{0.625000in}{0.550000in}}{\pgfqpoint{3.875000in}{3.850000in}} %
\pgfusepath{clip}%
\pgfsetbuttcap%
\pgfsetroundjoin%
\pgfsetlinewidth{0.250937pt}%
\definecolor{currentstroke}{rgb}{0.000000,0.000000,0.000000}%
\pgfsetstrokecolor{currentstroke}%
\pgfsetdash{}{0pt}%
\pgfpathmoveto{\pgfqpoint{0.625000in}{1.656067in}}%
\pgfpathlineto{\pgfqpoint{0.626781in}{1.659649in}}%
\pgfpathlineto{\pgfqpoint{0.625000in}{1.666632in}}%
\pgfusepath{stroke}%
\end{pgfscope}%
\begin{pgfscope}%
\pgfpathrectangle{\pgfqpoint{0.625000in}{0.550000in}}{\pgfqpoint{3.875000in}{3.850000in}} %
\pgfusepath{clip}%
\pgfsetbuttcap%
\pgfsetroundjoin%
\pgfsetlinewidth{0.250937pt}%
\definecolor{currentstroke}{rgb}{0.000000,0.000000,0.000000}%
\pgfsetstrokecolor{currentstroke}%
\pgfsetdash{}{0pt}%
\pgfpathmoveto{\pgfqpoint{0.625000in}{1.717534in}}%
\pgfpathlineto{\pgfqpoint{0.625079in}{1.717544in}}%
\pgfpathlineto{\pgfqpoint{0.625000in}{1.718131in}}%
\pgfusepath{stroke}%
\end{pgfscope}%
\begin{pgfscope}%
\pgfpathrectangle{\pgfqpoint{0.625000in}{0.550000in}}{\pgfqpoint{3.875000in}{3.850000in}} %
\pgfusepath{clip}%
\pgfsetbuttcap%
\pgfsetroundjoin%
\pgfsetlinewidth{0.250937pt}%
\definecolor{currentstroke}{rgb}{0.000000,0.000000,0.000000}%
\pgfsetstrokecolor{currentstroke}%
\pgfsetdash{}{0pt}%
\pgfpathmoveto{\pgfqpoint{0.625000in}{1.747527in}}%
\pgfpathlineto{\pgfqpoint{0.629114in}{1.756140in}}%
\pgfpathlineto{\pgfqpoint{0.625000in}{1.762431in}}%
\pgfusepath{stroke}%
\end{pgfscope}%
\begin{pgfscope}%
\pgfpathrectangle{\pgfqpoint{0.625000in}{0.550000in}}{\pgfqpoint{3.875000in}{3.850000in}} %
\pgfusepath{clip}%
\pgfsetbuttcap%
\pgfsetroundjoin%
\pgfsetlinewidth{0.250937pt}%
\definecolor{currentstroke}{rgb}{0.000000,0.000000,0.000000}%
\pgfsetstrokecolor{currentstroke}%
\pgfsetdash{}{0pt}%
\pgfpathmoveto{\pgfqpoint{0.625000in}{1.777847in}}%
\pgfpathlineto{\pgfqpoint{0.626319in}{1.785088in}}%
\pgfpathlineto{\pgfqpoint{0.626664in}{1.794737in}}%
\pgfpathlineto{\pgfqpoint{0.627126in}{1.804386in}}%
\pgfpathlineto{\pgfqpoint{0.625000in}{1.807942in}}%
\pgfusepath{stroke}%
\end{pgfscope}%
\begin{pgfscope}%
\pgfpathrectangle{\pgfqpoint{0.625000in}{0.550000in}}{\pgfqpoint{3.875000in}{3.850000in}} %
\pgfusepath{clip}%
\pgfsetbuttcap%
\pgfsetroundjoin%
\pgfsetlinewidth{0.250937pt}%
\definecolor{currentstroke}{rgb}{0.000000,0.000000,0.000000}%
\pgfsetstrokecolor{currentstroke}%
\pgfsetdash{}{0pt}%
\pgfpathmoveto{\pgfqpoint{0.625000in}{1.848606in}}%
\pgfpathlineto{\pgfqpoint{0.627388in}{1.852632in}}%
\pgfpathlineto{\pgfqpoint{0.625000in}{1.852736in}}%
\pgfusepath{stroke}%
\end{pgfscope}%
\begin{pgfscope}%
\pgfpathrectangle{\pgfqpoint{0.625000in}{0.550000in}}{\pgfqpoint{3.875000in}{3.850000in}} %
\pgfusepath{clip}%
\pgfsetbuttcap%
\pgfsetroundjoin%
\pgfsetlinewidth{0.250937pt}%
\definecolor{currentstroke}{rgb}{0.000000,0.000000,0.000000}%
\pgfsetstrokecolor{currentstroke}%
\pgfsetdash{}{0pt}%
\pgfpathmoveto{\pgfqpoint{0.625000in}{1.898677in}}%
\pgfpathlineto{\pgfqpoint{0.628499in}{1.900877in}}%
\pgfpathlineto{\pgfqpoint{0.625000in}{1.904679in}}%
\pgfusepath{stroke}%
\end{pgfscope}%
\begin{pgfscope}%
\pgfpathrectangle{\pgfqpoint{0.625000in}{0.550000in}}{\pgfqpoint{3.875000in}{3.850000in}} %
\pgfusepath{clip}%
\pgfsetbuttcap%
\pgfsetroundjoin%
\pgfsetlinewidth{0.250937pt}%
\definecolor{currentstroke}{rgb}{0.000000,0.000000,0.000000}%
\pgfsetstrokecolor{currentstroke}%
\pgfsetdash{}{0pt}%
\pgfpathmoveto{\pgfqpoint{0.625000in}{1.938519in}}%
\pgfpathlineto{\pgfqpoint{0.626072in}{1.939474in}}%
\pgfpathlineto{\pgfqpoint{0.625983in}{1.949123in}}%
\pgfpathlineto{\pgfqpoint{0.625000in}{1.951602in}}%
\pgfusepath{stroke}%
\end{pgfscope}%
\begin{pgfscope}%
\pgfpathrectangle{\pgfqpoint{0.625000in}{0.550000in}}{\pgfqpoint{3.875000in}{3.850000in}} %
\pgfusepath{clip}%
\pgfsetbuttcap%
\pgfsetroundjoin%
\pgfsetlinewidth{0.250937pt}%
\definecolor{currentstroke}{rgb}{0.000000,0.000000,0.000000}%
\pgfsetstrokecolor{currentstroke}%
\pgfsetdash{}{0pt}%
\pgfpathmoveto{\pgfqpoint{0.625000in}{1.988822in}}%
\pgfpathlineto{\pgfqpoint{0.630770in}{1.997368in}}%
\pgfpathlineto{\pgfqpoint{0.625000in}{2.002279in}}%
\pgfusepath{stroke}%
\end{pgfscope}%
\begin{pgfscope}%
\pgfpathrectangle{\pgfqpoint{0.625000in}{0.550000in}}{\pgfqpoint{3.875000in}{3.850000in}} %
\pgfusepath{clip}%
\pgfsetbuttcap%
\pgfsetroundjoin%
\pgfsetlinewidth{0.250937pt}%
\definecolor{currentstroke}{rgb}{0.000000,0.000000,0.000000}%
\pgfsetstrokecolor{currentstroke}%
\pgfsetdash{}{0pt}%
\pgfpathmoveto{\pgfqpoint{0.625000in}{2.087181in}}%
\pgfpathlineto{\pgfqpoint{0.630216in}{2.093860in}}%
\pgfpathlineto{\pgfqpoint{0.625000in}{2.100563in}}%
\pgfusepath{stroke}%
\end{pgfscope}%
\begin{pgfscope}%
\pgfpathrectangle{\pgfqpoint{0.625000in}{0.550000in}}{\pgfqpoint{3.875000in}{3.850000in}} %
\pgfusepath{clip}%
\pgfsetbuttcap%
\pgfsetroundjoin%
\pgfsetlinewidth{0.250937pt}%
\definecolor{currentstroke}{rgb}{0.000000,0.000000,0.000000}%
\pgfsetstrokecolor{currentstroke}%
\pgfsetdash{}{0pt}%
\pgfpathmoveto{\pgfqpoint{0.625000in}{2.137779in}}%
\pgfpathlineto{\pgfqpoint{0.625567in}{2.142105in}}%
\pgfpathlineto{\pgfqpoint{0.625000in}{2.143599in}}%
\pgfusepath{stroke}%
\end{pgfscope}%
\begin{pgfscope}%
\pgfpathrectangle{\pgfqpoint{0.625000in}{0.550000in}}{\pgfqpoint{3.875000in}{3.850000in}} %
\pgfusepath{clip}%
\pgfsetbuttcap%
\pgfsetroundjoin%
\pgfsetlinewidth{0.250937pt}%
\definecolor{currentstroke}{rgb}{0.000000,0.000000,0.000000}%
\pgfsetstrokecolor{currentstroke}%
\pgfsetdash{}{0pt}%
\pgfpathmoveto{\pgfqpoint{0.625000in}{2.180648in}}%
\pgfpathlineto{\pgfqpoint{0.625903in}{2.180702in}}%
\pgfpathlineto{\pgfqpoint{0.626996in}{2.190351in}}%
\pgfpathlineto{\pgfqpoint{0.625000in}{2.195493in}}%
\pgfusepath{stroke}%
\end{pgfscope}%
\begin{pgfscope}%
\pgfpathrectangle{\pgfqpoint{0.625000in}{0.550000in}}{\pgfqpoint{3.875000in}{3.850000in}} %
\pgfusepath{clip}%
\pgfsetbuttcap%
\pgfsetroundjoin%
\pgfsetlinewidth{0.250937pt}%
\definecolor{currentstroke}{rgb}{0.000000,0.000000,0.000000}%
\pgfsetstrokecolor{currentstroke}%
\pgfsetdash{}{0pt}%
\pgfpathmoveto{\pgfqpoint{0.625000in}{2.232517in}}%
\pgfpathlineto{\pgfqpoint{0.628890in}{2.238596in}}%
\pgfpathlineto{\pgfqpoint{0.626533in}{2.248246in}}%
\pgfpathlineto{\pgfqpoint{0.625000in}{2.252583in}}%
\pgfusepath{stroke}%
\end{pgfscope}%
\begin{pgfscope}%
\pgfpathrectangle{\pgfqpoint{0.625000in}{0.550000in}}{\pgfqpoint{3.875000in}{3.850000in}} %
\pgfusepath{clip}%
\pgfsetbuttcap%
\pgfsetroundjoin%
\pgfsetlinewidth{0.250937pt}%
\definecolor{currentstroke}{rgb}{0.000000,0.000000,0.000000}%
\pgfsetstrokecolor{currentstroke}%
\pgfsetdash{}{0pt}%
\pgfpathmoveto{\pgfqpoint{0.625000in}{2.281984in}}%
\pgfpathlineto{\pgfqpoint{0.628488in}{2.286842in}}%
\pgfpathlineto{\pgfqpoint{0.625000in}{2.292271in}}%
\pgfusepath{stroke}%
\end{pgfscope}%
\begin{pgfscope}%
\pgfpathrectangle{\pgfqpoint{0.625000in}{0.550000in}}{\pgfqpoint{3.875000in}{3.850000in}} %
\pgfusepath{clip}%
\pgfsetbuttcap%
\pgfsetroundjoin%
\pgfsetlinewidth{0.250937pt}%
\definecolor{currentstroke}{rgb}{0.000000,0.000000,0.000000}%
\pgfsetstrokecolor{currentstroke}%
\pgfsetdash{}{0pt}%
\pgfpathmoveto{\pgfqpoint{0.625000in}{2.335003in}}%
\pgfpathlineto{\pgfqpoint{0.626445in}{2.335088in}}%
\pgfpathlineto{\pgfqpoint{0.625000in}{2.338590in}}%
\pgfusepath{stroke}%
\end{pgfscope}%
\begin{pgfscope}%
\pgfpathrectangle{\pgfqpoint{0.625000in}{0.550000in}}{\pgfqpoint{3.875000in}{3.850000in}} %
\pgfusepath{clip}%
\pgfsetbuttcap%
\pgfsetroundjoin%
\pgfsetlinewidth{0.250937pt}%
\definecolor{currentstroke}{rgb}{0.000000,0.000000,0.000000}%
\pgfsetstrokecolor{currentstroke}%
\pgfsetdash{}{0pt}%
\pgfpathmoveto{\pgfqpoint{0.625000in}{2.380427in}}%
\pgfpathlineto{\pgfqpoint{0.626403in}{2.383333in}}%
\pgfpathlineto{\pgfqpoint{0.625000in}{2.391313in}}%
\pgfusepath{stroke}%
\end{pgfscope}%
\begin{pgfscope}%
\pgfpathrectangle{\pgfqpoint{0.625000in}{0.550000in}}{\pgfqpoint{3.875000in}{3.850000in}} %
\pgfusepath{clip}%
\pgfsetbuttcap%
\pgfsetroundjoin%
\pgfsetlinewidth{0.250937pt}%
\definecolor{currentstroke}{rgb}{0.000000,0.000000,0.000000}%
\pgfsetstrokecolor{currentstroke}%
\pgfsetdash{}{0pt}%
\pgfpathmoveto{\pgfqpoint{0.625000in}{2.395233in}}%
\pgfpathlineto{\pgfqpoint{0.625766in}{2.402632in}}%
\pgfpathlineto{\pgfqpoint{0.625000in}{2.404257in}}%
\pgfusepath{stroke}%
\end{pgfscope}%
\begin{pgfscope}%
\pgfpathrectangle{\pgfqpoint{0.625000in}{0.550000in}}{\pgfqpoint{3.875000in}{3.850000in}} %
\pgfusepath{clip}%
\pgfsetbuttcap%
\pgfsetroundjoin%
\pgfsetlinewidth{0.250937pt}%
\definecolor{currentstroke}{rgb}{0.000000,0.000000,0.000000}%
\pgfsetstrokecolor{currentstroke}%
\pgfsetdash{}{0pt}%
\pgfpathmoveto{\pgfqpoint{0.625000in}{2.448894in}}%
\pgfpathlineto{\pgfqpoint{0.625016in}{2.450877in}}%
\pgfpathlineto{\pgfqpoint{0.625000in}{2.451409in}}%
\pgfusepath{stroke}%
\end{pgfscope}%
\begin{pgfscope}%
\pgfpathrectangle{\pgfqpoint{0.625000in}{0.550000in}}{\pgfqpoint{3.875000in}{3.850000in}} %
\pgfusepath{clip}%
\pgfsetbuttcap%
\pgfsetroundjoin%
\pgfsetlinewidth{0.250937pt}%
\definecolor{currentstroke}{rgb}{0.000000,0.000000,0.000000}%
\pgfsetstrokecolor{currentstroke}%
\pgfsetdash{}{0pt}%
\pgfpathmoveto{\pgfqpoint{0.625000in}{2.508240in}}%
\pgfpathlineto{\pgfqpoint{0.625016in}{2.508772in}}%
\pgfpathlineto{\pgfqpoint{0.625000in}{2.510755in}}%
\pgfusepath{stroke}%
\end{pgfscope}%
\begin{pgfscope}%
\pgfpathrectangle{\pgfqpoint{0.625000in}{0.550000in}}{\pgfqpoint{3.875000in}{3.850000in}} %
\pgfusepath{clip}%
\pgfsetbuttcap%
\pgfsetroundjoin%
\pgfsetlinewidth{0.250937pt}%
\definecolor{currentstroke}{rgb}{0.000000,0.000000,0.000000}%
\pgfsetstrokecolor{currentstroke}%
\pgfsetdash{}{0pt}%
\pgfpathmoveto{\pgfqpoint{0.625000in}{2.555393in}}%
\pgfpathlineto{\pgfqpoint{0.625766in}{2.557018in}}%
\pgfpathlineto{\pgfqpoint{0.625000in}{2.564416in}}%
\pgfusepath{stroke}%
\end{pgfscope}%
\begin{pgfscope}%
\pgfpathrectangle{\pgfqpoint{0.625000in}{0.550000in}}{\pgfqpoint{3.875000in}{3.850000in}} %
\pgfusepath{clip}%
\pgfsetbuttcap%
\pgfsetroundjoin%
\pgfsetlinewidth{0.250937pt}%
\definecolor{currentstroke}{rgb}{0.000000,0.000000,0.000000}%
\pgfsetstrokecolor{currentstroke}%
\pgfsetdash{}{0pt}%
\pgfpathmoveto{\pgfqpoint{0.625000in}{2.568336in}}%
\pgfpathlineto{\pgfqpoint{0.626403in}{2.576316in}}%
\pgfpathlineto{\pgfqpoint{0.625000in}{2.579222in}}%
\pgfusepath{stroke}%
\end{pgfscope}%
\begin{pgfscope}%
\pgfpathrectangle{\pgfqpoint{0.625000in}{0.550000in}}{\pgfqpoint{3.875000in}{3.850000in}} %
\pgfusepath{clip}%
\pgfsetbuttcap%
\pgfsetroundjoin%
\pgfsetlinewidth{0.250937pt}%
\definecolor{currentstroke}{rgb}{0.000000,0.000000,0.000000}%
\pgfsetstrokecolor{currentstroke}%
\pgfsetdash{}{0pt}%
\pgfpathmoveto{\pgfqpoint{0.625000in}{2.621059in}}%
\pgfpathlineto{\pgfqpoint{0.626445in}{2.624561in}}%
\pgfpathlineto{\pgfqpoint{0.625000in}{2.624646in}}%
\pgfusepath{stroke}%
\end{pgfscope}%
\begin{pgfscope}%
\pgfpathrectangle{\pgfqpoint{0.625000in}{0.550000in}}{\pgfqpoint{3.875000in}{3.850000in}} %
\pgfusepath{clip}%
\pgfsetbuttcap%
\pgfsetroundjoin%
\pgfsetlinewidth{0.250937pt}%
\definecolor{currentstroke}{rgb}{0.000000,0.000000,0.000000}%
\pgfsetstrokecolor{currentstroke}%
\pgfsetdash{}{0pt}%
\pgfpathmoveto{\pgfqpoint{0.625000in}{2.667378in}}%
\pgfpathlineto{\pgfqpoint{0.628488in}{2.672807in}}%
\pgfpathlineto{\pgfqpoint{0.625000in}{2.677666in}}%
\pgfusepath{stroke}%
\end{pgfscope}%
\begin{pgfscope}%
\pgfpathrectangle{\pgfqpoint{0.625000in}{0.550000in}}{\pgfqpoint{3.875000in}{3.850000in}} %
\pgfusepath{clip}%
\pgfsetbuttcap%
\pgfsetroundjoin%
\pgfsetlinewidth{0.250937pt}%
\definecolor{currentstroke}{rgb}{0.000000,0.000000,0.000000}%
\pgfsetstrokecolor{currentstroke}%
\pgfsetdash{}{0pt}%
\pgfpathmoveto{\pgfqpoint{0.625000in}{2.707066in}}%
\pgfpathlineto{\pgfqpoint{0.626533in}{2.711404in}}%
\pgfpathlineto{\pgfqpoint{0.628890in}{2.721053in}}%
\pgfpathlineto{\pgfqpoint{0.625000in}{2.727132in}}%
\pgfusepath{stroke}%
\end{pgfscope}%
\begin{pgfscope}%
\pgfpathrectangle{\pgfqpoint{0.625000in}{0.550000in}}{\pgfqpoint{3.875000in}{3.850000in}} %
\pgfusepath{clip}%
\pgfsetbuttcap%
\pgfsetroundjoin%
\pgfsetlinewidth{0.250937pt}%
\definecolor{currentstroke}{rgb}{0.000000,0.000000,0.000000}%
\pgfsetstrokecolor{currentstroke}%
\pgfsetdash{}{0pt}%
\pgfpathmoveto{\pgfqpoint{0.625000in}{2.764156in}}%
\pgfpathlineto{\pgfqpoint{0.626996in}{2.769298in}}%
\pgfpathlineto{\pgfqpoint{0.625903in}{2.778947in}}%
\pgfpathlineto{\pgfqpoint{0.625000in}{2.779001in}}%
\pgfusepath{stroke}%
\end{pgfscope}%
\begin{pgfscope}%
\pgfpathrectangle{\pgfqpoint{0.625000in}{0.550000in}}{\pgfqpoint{3.875000in}{3.850000in}} %
\pgfusepath{clip}%
\pgfsetbuttcap%
\pgfsetroundjoin%
\pgfsetlinewidth{0.250937pt}%
\definecolor{currentstroke}{rgb}{0.000000,0.000000,0.000000}%
\pgfsetstrokecolor{currentstroke}%
\pgfsetdash{}{0pt}%
\pgfpathmoveto{\pgfqpoint{0.625000in}{2.816050in}}%
\pgfpathlineto{\pgfqpoint{0.625567in}{2.817544in}}%
\pgfpathlineto{\pgfqpoint{0.625000in}{2.821870in}}%
\pgfusepath{stroke}%
\end{pgfscope}%
\begin{pgfscope}%
\pgfpathrectangle{\pgfqpoint{0.625000in}{0.550000in}}{\pgfqpoint{3.875000in}{3.850000in}} %
\pgfusepath{clip}%
\pgfsetbuttcap%
\pgfsetroundjoin%
\pgfsetlinewidth{0.250937pt}%
\definecolor{currentstroke}{rgb}{0.000000,0.000000,0.000000}%
\pgfsetstrokecolor{currentstroke}%
\pgfsetdash{}{0pt}%
\pgfpathmoveto{\pgfqpoint{0.625000in}{2.859086in}}%
\pgfpathlineto{\pgfqpoint{0.630216in}{2.865789in}}%
\pgfpathlineto{\pgfqpoint{0.625000in}{2.872468in}}%
\pgfusepath{stroke}%
\end{pgfscope}%
\begin{pgfscope}%
\pgfpathrectangle{\pgfqpoint{0.625000in}{0.550000in}}{\pgfqpoint{3.875000in}{3.850000in}} %
\pgfusepath{clip}%
\pgfsetbuttcap%
\pgfsetroundjoin%
\pgfsetlinewidth{0.250937pt}%
\definecolor{currentstroke}{rgb}{0.000000,0.000000,0.000000}%
\pgfsetstrokecolor{currentstroke}%
\pgfsetdash{}{0pt}%
\pgfpathmoveto{\pgfqpoint{0.625000in}{2.957370in}}%
\pgfpathlineto{\pgfqpoint{0.630770in}{2.962281in}}%
\pgfpathlineto{\pgfqpoint{0.625000in}{2.970828in}}%
\pgfusepath{stroke}%
\end{pgfscope}%
\begin{pgfscope}%
\pgfpathrectangle{\pgfqpoint{0.625000in}{0.550000in}}{\pgfqpoint{3.875000in}{3.850000in}} %
\pgfusepath{clip}%
\pgfsetbuttcap%
\pgfsetroundjoin%
\pgfsetlinewidth{0.250937pt}%
\definecolor{currentstroke}{rgb}{0.000000,0.000000,0.000000}%
\pgfsetstrokecolor{currentstroke}%
\pgfsetdash{}{0pt}%
\pgfpathmoveto{\pgfqpoint{0.625000in}{3.008047in}}%
\pgfpathlineto{\pgfqpoint{0.625983in}{3.010526in}}%
\pgfpathlineto{\pgfqpoint{0.626072in}{3.020175in}}%
\pgfpathlineto{\pgfqpoint{0.625000in}{3.021131in}}%
\pgfusepath{stroke}%
\end{pgfscope}%
\begin{pgfscope}%
\pgfpathrectangle{\pgfqpoint{0.625000in}{0.550000in}}{\pgfqpoint{3.875000in}{3.850000in}} %
\pgfusepath{clip}%
\pgfsetbuttcap%
\pgfsetroundjoin%
\pgfsetlinewidth{0.250937pt}%
\definecolor{currentstroke}{rgb}{0.000000,0.000000,0.000000}%
\pgfsetstrokecolor{currentstroke}%
\pgfsetdash{}{0pt}%
\pgfpathmoveto{\pgfqpoint{0.625000in}{3.054970in}}%
\pgfpathlineto{\pgfqpoint{0.628499in}{3.058772in}}%
\pgfpathlineto{\pgfqpoint{0.625000in}{3.060972in}}%
\pgfusepath{stroke}%
\end{pgfscope}%
\begin{pgfscope}%
\pgfpathrectangle{\pgfqpoint{0.625000in}{0.550000in}}{\pgfqpoint{3.875000in}{3.850000in}} %
\pgfusepath{clip}%
\pgfsetbuttcap%
\pgfsetroundjoin%
\pgfsetlinewidth{0.250937pt}%
\definecolor{currentstroke}{rgb}{0.000000,0.000000,0.000000}%
\pgfsetstrokecolor{currentstroke}%
\pgfsetdash{}{0pt}%
\pgfpathmoveto{\pgfqpoint{0.625000in}{3.106912in}}%
\pgfpathlineto{\pgfqpoint{0.627388in}{3.107018in}}%
\pgfpathlineto{\pgfqpoint{0.625000in}{3.111044in}}%
\pgfusepath{stroke}%
\end{pgfscope}%
\begin{pgfscope}%
\pgfpathrectangle{\pgfqpoint{0.625000in}{0.550000in}}{\pgfqpoint{3.875000in}{3.850000in}} %
\pgfusepath{clip}%
\pgfsetbuttcap%
\pgfsetroundjoin%
\pgfsetlinewidth{0.250937pt}%
\definecolor{currentstroke}{rgb}{0.000000,0.000000,0.000000}%
\pgfsetstrokecolor{currentstroke}%
\pgfsetdash{}{0pt}%
\pgfpathmoveto{\pgfqpoint{0.625000in}{3.151708in}}%
\pgfpathlineto{\pgfqpoint{0.627126in}{3.155263in}}%
\pgfpathlineto{\pgfqpoint{0.626664in}{3.164912in}}%
\pgfpathlineto{\pgfqpoint{0.626319in}{3.174561in}}%
\pgfpathlineto{\pgfqpoint{0.625000in}{3.181802in}}%
\pgfusepath{stroke}%
\end{pgfscope}%
\begin{pgfscope}%
\pgfpathrectangle{\pgfqpoint{0.625000in}{0.550000in}}{\pgfqpoint{3.875000in}{3.850000in}} %
\pgfusepath{clip}%
\pgfsetbuttcap%
\pgfsetroundjoin%
\pgfsetlinewidth{0.250937pt}%
\definecolor{currentstroke}{rgb}{0.000000,0.000000,0.000000}%
\pgfsetstrokecolor{currentstroke}%
\pgfsetdash{}{0pt}%
\pgfpathmoveto{\pgfqpoint{0.625000in}{3.197218in}}%
\pgfpathlineto{\pgfqpoint{0.629114in}{3.203509in}}%
\pgfpathlineto{\pgfqpoint{0.625000in}{3.212122in}}%
\pgfusepath{stroke}%
\end{pgfscope}%
\begin{pgfscope}%
\pgfpathrectangle{\pgfqpoint{0.625000in}{0.550000in}}{\pgfqpoint{3.875000in}{3.850000in}} %
\pgfusepath{clip}%
\pgfsetbuttcap%
\pgfsetroundjoin%
\pgfsetlinewidth{0.250937pt}%
\definecolor{currentstroke}{rgb}{0.000000,0.000000,0.000000}%
\pgfsetstrokecolor{currentstroke}%
\pgfsetdash{}{0pt}%
\pgfpathmoveto{\pgfqpoint{0.625000in}{3.241518in}}%
\pgfpathlineto{\pgfqpoint{0.625079in}{3.242105in}}%
\pgfpathlineto{\pgfqpoint{0.625000in}{3.242115in}}%
\pgfusepath{stroke}%
\end{pgfscope}%
\begin{pgfscope}%
\pgfpathrectangle{\pgfqpoint{0.625000in}{0.550000in}}{\pgfqpoint{3.875000in}{3.850000in}} %
\pgfusepath{clip}%
\pgfsetbuttcap%
\pgfsetroundjoin%
\pgfsetlinewidth{0.250937pt}%
\definecolor{currentstroke}{rgb}{0.000000,0.000000,0.000000}%
\pgfsetstrokecolor{currentstroke}%
\pgfsetdash{}{0pt}%
\pgfpathmoveto{\pgfqpoint{0.625000in}{3.293017in}}%
\pgfpathlineto{\pgfqpoint{0.626781in}{3.300000in}}%
\pgfpathlineto{\pgfqpoint{0.625000in}{3.303582in}}%
\pgfusepath{stroke}%
\end{pgfscope}%
\begin{pgfscope}%
\pgfpathrectangle{\pgfqpoint{0.625000in}{0.550000in}}{\pgfqpoint{3.875000in}{3.850000in}} %
\pgfusepath{clip}%
\pgfsetbuttcap%
\pgfsetroundjoin%
\pgfsetlinewidth{0.250937pt}%
\definecolor{currentstroke}{rgb}{0.000000,0.000000,0.000000}%
\pgfsetstrokecolor{currentstroke}%
\pgfsetdash{}{0pt}%
\pgfpathmoveto{\pgfqpoint{0.625000in}{3.327580in}}%
\pgfpathlineto{\pgfqpoint{0.626205in}{3.328947in}}%
\pgfpathlineto{\pgfqpoint{0.625000in}{3.330609in}}%
\pgfusepath{stroke}%
\end{pgfscope}%
\begin{pgfscope}%
\pgfpathrectangle{\pgfqpoint{0.625000in}{0.550000in}}{\pgfqpoint{3.875000in}{3.850000in}} %
\pgfusepath{clip}%
\pgfsetbuttcap%
\pgfsetroundjoin%
\pgfsetlinewidth{0.250937pt}%
\definecolor{currentstroke}{rgb}{0.000000,0.000000,0.000000}%
\pgfsetstrokecolor{currentstroke}%
\pgfsetdash{}{0pt}%
\pgfpathmoveto{\pgfqpoint{0.625000in}{3.346005in}}%
\pgfpathlineto{\pgfqpoint{0.626797in}{3.348246in}}%
\pgfpathlineto{\pgfqpoint{0.625761in}{3.357895in}}%
\pgfpathlineto{\pgfqpoint{0.625000in}{3.362924in}}%
\pgfusepath{stroke}%
\end{pgfscope}%
\begin{pgfscope}%
\pgfpathrectangle{\pgfqpoint{0.625000in}{0.550000in}}{\pgfqpoint{3.875000in}{3.850000in}} %
\pgfusepath{clip}%
\pgfsetbuttcap%
\pgfsetroundjoin%
\pgfsetlinewidth{0.250937pt}%
\definecolor{currentstroke}{rgb}{0.000000,0.000000,0.000000}%
\pgfsetstrokecolor{currentstroke}%
\pgfsetdash{}{0pt}%
\pgfpathmoveto{\pgfqpoint{0.625000in}{3.394984in}}%
\pgfpathlineto{\pgfqpoint{0.625600in}{3.396491in}}%
\pgfpathlineto{\pgfqpoint{0.625000in}{3.396519in}}%
\pgfusepath{stroke}%
\end{pgfscope}%
\begin{pgfscope}%
\pgfpathrectangle{\pgfqpoint{0.625000in}{0.550000in}}{\pgfqpoint{3.875000in}{3.850000in}} %
\pgfusepath{clip}%
\pgfsetbuttcap%
\pgfsetroundjoin%
\pgfsetlinewidth{0.250937pt}%
\definecolor{currentstroke}{rgb}{0.000000,0.000000,0.000000}%
\pgfsetstrokecolor{currentstroke}%
\pgfsetdash{}{0pt}%
\pgfpathmoveto{\pgfqpoint{0.625000in}{3.436914in}}%
\pgfpathlineto{\pgfqpoint{0.632490in}{3.444737in}}%
\pgfpathlineto{\pgfqpoint{0.625000in}{3.452192in}}%
\pgfusepath{stroke}%
\end{pgfscope}%
\begin{pgfscope}%
\pgfpathrectangle{\pgfqpoint{0.625000in}{0.550000in}}{\pgfqpoint{3.875000in}{3.850000in}} %
\pgfusepath{clip}%
\pgfsetbuttcap%
\pgfsetroundjoin%
\pgfsetlinewidth{0.250937pt}%
\definecolor{currentstroke}{rgb}{0.000000,0.000000,0.000000}%
\pgfsetstrokecolor{currentstroke}%
\pgfsetdash{}{0pt}%
\pgfpathmoveto{\pgfqpoint{0.625000in}{3.472063in}}%
\pgfpathlineto{\pgfqpoint{0.625788in}{3.473684in}}%
\pgfpathlineto{\pgfqpoint{0.626703in}{3.483333in}}%
\pgfpathlineto{\pgfqpoint{0.625000in}{3.492719in}}%
\pgfusepath{stroke}%
\end{pgfscope}%
\begin{pgfscope}%
\pgfpathrectangle{\pgfqpoint{0.625000in}{0.550000in}}{\pgfqpoint{3.875000in}{3.850000in}} %
\pgfusepath{clip}%
\pgfsetbuttcap%
\pgfsetroundjoin%
\pgfsetlinewidth{0.250937pt}%
\definecolor{currentstroke}{rgb}{0.000000,0.000000,0.000000}%
\pgfsetstrokecolor{currentstroke}%
\pgfsetdash{}{0pt}%
\pgfpathmoveto{\pgfqpoint{0.625000in}{3.539071in}}%
\pgfpathlineto{\pgfqpoint{0.626905in}{3.541228in}}%
\pgfpathlineto{\pgfqpoint{0.625000in}{3.548949in}}%
\pgfusepath{stroke}%
\end{pgfscope}%
\begin{pgfscope}%
\pgfpathrectangle{\pgfqpoint{0.625000in}{0.550000in}}{\pgfqpoint{3.875000in}{3.850000in}} %
\pgfusepath{clip}%
\pgfsetbuttcap%
\pgfsetroundjoin%
\pgfsetlinewidth{0.250937pt}%
\definecolor{currentstroke}{rgb}{0.000000,0.000000,0.000000}%
\pgfsetstrokecolor{currentstroke}%
\pgfsetdash{}{0pt}%
\pgfpathmoveto{\pgfqpoint{0.625000in}{3.625119in}}%
\pgfpathlineto{\pgfqpoint{0.626500in}{3.628070in}}%
\pgfpathlineto{\pgfqpoint{0.630216in}{3.637719in}}%
\pgfpathlineto{\pgfqpoint{0.625608in}{3.647368in}}%
\pgfpathlineto{\pgfqpoint{0.625000in}{3.648707in}}%
\pgfusepath{stroke}%
\end{pgfscope}%
\begin{pgfscope}%
\pgfpathrectangle{\pgfqpoint{0.625000in}{0.550000in}}{\pgfqpoint{3.875000in}{3.850000in}} %
\pgfusepath{clip}%
\pgfsetbuttcap%
\pgfsetroundjoin%
\pgfsetlinewidth{0.250937pt}%
\definecolor{currentstroke}{rgb}{0.000000,0.000000,0.000000}%
\pgfsetstrokecolor{currentstroke}%
\pgfsetdash{}{0pt}%
\pgfpathmoveto{\pgfqpoint{0.625000in}{3.677402in}}%
\pgfpathlineto{\pgfqpoint{0.628989in}{3.685965in}}%
\pgfpathlineto{\pgfqpoint{0.625000in}{3.692139in}}%
\pgfusepath{stroke}%
\end{pgfscope}%
\begin{pgfscope}%
\pgfpathrectangle{\pgfqpoint{0.625000in}{0.550000in}}{\pgfqpoint{3.875000in}{3.850000in}} %
\pgfusepath{clip}%
\pgfsetbuttcap%
\pgfsetroundjoin%
\pgfsetlinewidth{0.250937pt}%
\definecolor{currentstroke}{rgb}{0.000000,0.000000,0.000000}%
\pgfsetstrokecolor{currentstroke}%
\pgfsetdash{}{0pt}%
\pgfpathmoveto{\pgfqpoint{0.625000in}{3.729538in}}%
\pgfpathlineto{\pgfqpoint{0.626459in}{3.734211in}}%
\pgfpathlineto{\pgfqpoint{0.625915in}{3.743860in}}%
\pgfpathlineto{\pgfqpoint{0.625000in}{3.751227in}}%
\pgfusepath{stroke}%
\end{pgfscope}%
\begin{pgfscope}%
\pgfpathrectangle{\pgfqpoint{0.625000in}{0.550000in}}{\pgfqpoint{3.875000in}{3.850000in}} %
\pgfusepath{clip}%
\pgfsetbuttcap%
\pgfsetroundjoin%
\pgfsetlinewidth{0.250937pt}%
\definecolor{currentstroke}{rgb}{0.000000,0.000000,0.000000}%
\pgfsetstrokecolor{currentstroke}%
\pgfsetdash{}{0pt}%
\pgfpathmoveto{\pgfqpoint{0.625000in}{3.788647in}}%
\pgfpathlineto{\pgfqpoint{0.625829in}{3.792105in}}%
\pgfpathlineto{\pgfqpoint{0.625000in}{3.794928in}}%
\pgfusepath{stroke}%
\end{pgfscope}%
\begin{pgfscope}%
\pgfpathrectangle{\pgfqpoint{0.625000in}{0.550000in}}{\pgfqpoint{3.875000in}{3.850000in}} %
\pgfusepath{clip}%
\pgfsetbuttcap%
\pgfsetroundjoin%
\pgfsetlinewidth{0.250937pt}%
\definecolor{currentstroke}{rgb}{0.000000,0.000000,0.000000}%
\pgfsetstrokecolor{currentstroke}%
\pgfsetdash{}{0pt}%
\pgfpathmoveto{\pgfqpoint{0.625000in}{3.821577in}}%
\pgfpathlineto{\pgfqpoint{0.628491in}{3.830702in}}%
\pgfpathlineto{\pgfqpoint{0.625641in}{3.840351in}}%
\pgfpathlineto{\pgfqpoint{0.625000in}{3.842072in}}%
\pgfusepath{stroke}%
\end{pgfscope}%
\begin{pgfscope}%
\pgfpathrectangle{\pgfqpoint{0.625000in}{0.550000in}}{\pgfqpoint{3.875000in}{3.850000in}} %
\pgfusepath{clip}%
\pgfsetbuttcap%
\pgfsetroundjoin%
\pgfsetlinewidth{0.250937pt}%
\definecolor{currentstroke}{rgb}{0.000000,0.000000,0.000000}%
\pgfsetstrokecolor{currentstroke}%
\pgfsetdash{}{0pt}%
\pgfpathmoveto{\pgfqpoint{0.625000in}{3.921759in}}%
\pgfpathlineto{\pgfqpoint{0.630770in}{3.927193in}}%
\pgfpathlineto{\pgfqpoint{0.625000in}{3.935996in}}%
\pgfusepath{stroke}%
\end{pgfscope}%
\begin{pgfscope}%
\pgfpathrectangle{\pgfqpoint{0.625000in}{0.550000in}}{\pgfqpoint{3.875000in}{3.850000in}} %
\pgfusepath{clip}%
\pgfsetbuttcap%
\pgfsetroundjoin%
\pgfsetlinewidth{0.250937pt}%
\definecolor{currentstroke}{rgb}{0.000000,0.000000,0.000000}%
\pgfsetstrokecolor{currentstroke}%
\pgfsetdash{}{0pt}%
\pgfpathmoveto{\pgfqpoint{0.625000in}{3.939859in}}%
\pgfpathlineto{\pgfqpoint{0.626654in}{3.946491in}}%
\pgfpathlineto{\pgfqpoint{0.625866in}{3.956140in}}%
\pgfpathlineto{\pgfqpoint{0.625000in}{3.958364in}}%
\pgfusepath{stroke}%
\end{pgfscope}%
\begin{pgfscope}%
\pgfpathrectangle{\pgfqpoint{0.625000in}{0.550000in}}{\pgfqpoint{3.875000in}{3.850000in}} %
\pgfusepath{clip}%
\pgfsetbuttcap%
\pgfsetroundjoin%
\pgfsetlinewidth{0.250937pt}%
\definecolor{currentstroke}{rgb}{0.000000,0.000000,0.000000}%
\pgfsetstrokecolor{currentstroke}%
\pgfsetdash{}{0pt}%
\pgfpathmoveto{\pgfqpoint{0.625000in}{3.972431in}}%
\pgfpathlineto{\pgfqpoint{0.626338in}{3.975439in}}%
\pgfpathlineto{\pgfqpoint{0.625000in}{3.981881in}}%
\pgfusepath{stroke}%
\end{pgfscope}%
\begin{pgfscope}%
\pgfpathrectangle{\pgfqpoint{0.625000in}{0.550000in}}{\pgfqpoint{3.875000in}{3.850000in}} %
\pgfusepath{clip}%
\pgfsetbuttcap%
\pgfsetroundjoin%
\pgfsetlinewidth{0.250937pt}%
\definecolor{currentstroke}{rgb}{0.000000,0.000000,0.000000}%
\pgfsetstrokecolor{currentstroke}%
\pgfsetdash{}{0pt}%
\pgfpathmoveto{\pgfqpoint{0.625000in}{4.068215in}}%
\pgfpathlineto{\pgfqpoint{0.625438in}{4.071930in}}%
\pgfpathlineto{\pgfqpoint{0.625000in}{4.073123in}}%
\pgfusepath{stroke}%
\end{pgfscope}%
\begin{pgfscope}%
\pgfpathrectangle{\pgfqpoint{0.625000in}{0.550000in}}{\pgfqpoint{3.875000in}{3.850000in}} %
\pgfusepath{clip}%
\pgfsetbuttcap%
\pgfsetroundjoin%
\pgfsetlinewidth{0.250937pt}%
\definecolor{currentstroke}{rgb}{0.000000,0.000000,0.000000}%
\pgfsetstrokecolor{currentstroke}%
\pgfsetdash{}{0pt}%
\pgfpathmoveto{\pgfqpoint{0.625000in}{4.096360in}}%
\pgfpathlineto{\pgfqpoint{0.626593in}{4.100877in}}%
\pgfpathlineto{\pgfqpoint{0.625000in}{4.107842in}}%
\pgfusepath{stroke}%
\end{pgfscope}%
\begin{pgfscope}%
\pgfpathrectangle{\pgfqpoint{0.625000in}{0.550000in}}{\pgfqpoint{3.875000in}{3.850000in}} %
\pgfusepath{clip}%
\pgfsetbuttcap%
\pgfsetroundjoin%
\pgfsetlinewidth{0.250937pt}%
\definecolor{currentstroke}{rgb}{0.000000,0.000000,0.000000}%
\pgfsetstrokecolor{currentstroke}%
\pgfsetdash{}{0pt}%
\pgfpathmoveto{\pgfqpoint{0.625000in}{4.112548in}}%
\pgfpathlineto{\pgfqpoint{0.626676in}{4.120175in}}%
\pgfpathlineto{\pgfqpoint{0.625743in}{4.129825in}}%
\pgfpathlineto{\pgfqpoint{0.625000in}{4.136773in}}%
\pgfusepath{stroke}%
\end{pgfscope}%
\begin{pgfscope}%
\pgfpathrectangle{\pgfqpoint{0.625000in}{0.550000in}}{\pgfqpoint{3.875000in}{3.850000in}} %
\pgfusepath{clip}%
\pgfsetbuttcap%
\pgfsetroundjoin%
\pgfsetlinewidth{0.250937pt}%
\definecolor{currentstroke}{rgb}{0.000000,0.000000,0.000000}%
\pgfsetstrokecolor{currentstroke}%
\pgfsetdash{}{0pt}%
\pgfpathmoveto{\pgfqpoint{0.625000in}{4.162057in}}%
\pgfpathlineto{\pgfqpoint{0.629193in}{4.168421in}}%
\pgfpathlineto{\pgfqpoint{0.625000in}{4.168714in}}%
\pgfusepath{stroke}%
\end{pgfscope}%
\begin{pgfscope}%
\pgfpathrectangle{\pgfqpoint{0.625000in}{0.550000in}}{\pgfqpoint{3.875000in}{3.850000in}} %
\pgfusepath{clip}%
\pgfsetbuttcap%
\pgfsetroundjoin%
\pgfsetlinewidth{0.250937pt}%
\definecolor{currentstroke}{rgb}{0.000000,0.000000,0.000000}%
\pgfsetstrokecolor{currentstroke}%
\pgfsetdash{}{0pt}%
\pgfpathmoveto{\pgfqpoint{0.625000in}{4.202867in}}%
\pgfpathlineto{\pgfqpoint{0.626775in}{4.207018in}}%
\pgfpathlineto{\pgfqpoint{0.628501in}{4.216667in}}%
\pgfpathlineto{\pgfqpoint{0.626675in}{4.226316in}}%
\pgfpathlineto{\pgfqpoint{0.625000in}{4.230301in}}%
\pgfusepath{stroke}%
\end{pgfscope}%
\begin{pgfscope}%
\pgfpathrectangle{\pgfqpoint{0.625000in}{0.550000in}}{\pgfqpoint{3.875000in}{3.850000in}} %
\pgfusepath{clip}%
\pgfsetbuttcap%
\pgfsetroundjoin%
\pgfsetlinewidth{0.250937pt}%
\definecolor{currentstroke}{rgb}{0.000000,0.000000,0.000000}%
\pgfsetstrokecolor{currentstroke}%
\pgfsetdash{}{0pt}%
\pgfpathmoveto{\pgfqpoint{0.625000in}{4.248558in}}%
\pgfpathlineto{\pgfqpoint{0.626409in}{4.255263in}}%
\pgfpathlineto{\pgfqpoint{0.625000in}{4.260383in}}%
\pgfusepath{stroke}%
\end{pgfscope}%
\begin{pgfscope}%
\pgfpathrectangle{\pgfqpoint{0.625000in}{0.550000in}}{\pgfqpoint{3.875000in}{3.850000in}} %
\pgfusepath{clip}%
\pgfsetbuttcap%
\pgfsetroundjoin%
\pgfsetlinewidth{0.250937pt}%
\definecolor{currentstroke}{rgb}{0.000000,0.000000,0.000000}%
\pgfsetstrokecolor{currentstroke}%
\pgfsetdash{}{0pt}%
\pgfpathmoveto{\pgfqpoint{0.625000in}{4.306367in}}%
\pgfpathlineto{\pgfqpoint{0.627157in}{4.313158in}}%
\pgfpathlineto{\pgfqpoint{0.625000in}{4.317836in}}%
\pgfusepath{stroke}%
\end{pgfscope}%
\begin{pgfscope}%
\pgfpathrectangle{\pgfqpoint{0.625000in}{0.550000in}}{\pgfqpoint{3.875000in}{3.850000in}} %
\pgfusepath{clip}%
\pgfsetbuttcap%
\pgfsetroundjoin%
\pgfsetlinewidth{0.250937pt}%
\definecolor{currentstroke}{rgb}{0.000000,0.000000,0.000000}%
\pgfsetstrokecolor{currentstroke}%
\pgfsetdash{}{0pt}%
\pgfpathmoveto{\pgfqpoint{0.625000in}{4.358154in}}%
\pgfpathlineto{\pgfqpoint{0.626931in}{4.361404in}}%
\pgfpathlineto{\pgfqpoint{0.625000in}{4.368086in}}%
\pgfusepath{stroke}%
\end{pgfscope}%
\begin{pgfscope}%
\pgfpathrectangle{\pgfqpoint{0.625000in}{0.550000in}}{\pgfqpoint{3.875000in}{3.850000in}} %
\pgfusepath{clip}%
\pgfsetbuttcap%
\pgfsetroundjoin%
\pgfsetlinewidth{0.250937pt}%
\definecolor{currentstroke}{rgb}{0.000000,0.000000,0.000000}%
\pgfsetstrokecolor{currentstroke}%
\pgfsetdash{}{0pt}%
\pgfpathmoveto{\pgfqpoint{0.632970in}{0.550000in}}%
\pgfpathlineto{\pgfqpoint{0.625000in}{0.558123in}}%
\pgfusepath{stroke}%
\end{pgfscope}%
\begin{pgfscope}%
\pgfpathrectangle{\pgfqpoint{0.625000in}{0.550000in}}{\pgfqpoint{3.875000in}{3.850000in}} %
\pgfusepath{clip}%
\pgfsetbuttcap%
\pgfsetroundjoin%
\pgfsetlinewidth{0.250937pt}%
\definecolor{currentstroke}{rgb}{0.000000,0.000000,0.000000}%
\pgfsetstrokecolor{currentstroke}%
\pgfsetdash{}{0pt}%
\pgfpathmoveto{\pgfqpoint{0.625000in}{0.645086in}}%
\pgfpathlineto{\pgfqpoint{0.625648in}{0.646491in}}%
\pgfpathlineto{\pgfqpoint{0.625000in}{0.648531in}}%
\pgfusepath{stroke}%
\end{pgfscope}%
\begin{pgfscope}%
\pgfpathrectangle{\pgfqpoint{0.625000in}{0.550000in}}{\pgfqpoint{3.875000in}{3.850000in}} %
\pgfusepath{clip}%
\pgfsetbuttcap%
\pgfsetroundjoin%
\pgfsetlinewidth{0.250937pt}%
\definecolor{currentstroke}{rgb}{0.000000,0.000000,0.000000}%
\pgfsetstrokecolor{currentstroke}%
\pgfsetdash{}{0pt}%
\pgfpathmoveto{\pgfqpoint{0.625000in}{0.732513in}}%
\pgfpathlineto{\pgfqpoint{0.625345in}{0.733333in}}%
\pgfpathlineto{\pgfqpoint{0.627258in}{0.742982in}}%
\pgfpathlineto{\pgfqpoint{0.625461in}{0.752632in}}%
\pgfpathlineto{\pgfqpoint{0.625000in}{0.753710in}}%
\pgfusepath{stroke}%
\end{pgfscope}%
\begin{pgfscope}%
\pgfpathrectangle{\pgfqpoint{0.625000in}{0.550000in}}{\pgfqpoint{3.875000in}{3.850000in}} %
\pgfusepath{clip}%
\pgfsetbuttcap%
\pgfsetroundjoin%
\pgfsetlinewidth{0.250937pt}%
\definecolor{currentstroke}{rgb}{0.000000,0.000000,0.000000}%
\pgfsetstrokecolor{currentstroke}%
\pgfsetdash{}{0pt}%
\pgfpathmoveto{\pgfqpoint{0.625000in}{0.791038in}}%
\pgfpathlineto{\pgfqpoint{0.627984in}{0.791228in}}%
\pgfpathlineto{\pgfqpoint{0.625000in}{0.795756in}}%
\pgfusepath{stroke}%
\end{pgfscope}%
\begin{pgfscope}%
\pgfpathrectangle{\pgfqpoint{0.625000in}{0.550000in}}{\pgfqpoint{3.875000in}{3.850000in}} %
\pgfusepath{clip}%
\pgfsetbuttcap%
\pgfsetroundjoin%
\pgfsetlinewidth{0.250937pt}%
\definecolor{currentstroke}{rgb}{0.000000,0.000000,0.000000}%
\pgfsetstrokecolor{currentstroke}%
\pgfsetdash{}{0pt}%
\pgfpathmoveto{\pgfqpoint{0.625000in}{0.829571in}}%
\pgfpathlineto{\pgfqpoint{0.625027in}{0.829825in}}%
\pgfpathlineto{\pgfqpoint{0.625503in}{0.839474in}}%
\pgfpathlineto{\pgfqpoint{0.625000in}{0.841764in}}%
\pgfusepath{stroke}%
\end{pgfscope}%
\begin{pgfscope}%
\pgfpathrectangle{\pgfqpoint{0.625000in}{0.550000in}}{\pgfqpoint{3.875000in}{3.850000in}} %
\pgfusepath{clip}%
\pgfsetbuttcap%
\pgfsetroundjoin%
\pgfsetlinewidth{0.250937pt}%
\definecolor{currentstroke}{rgb}{0.000000,0.000000,0.000000}%
\pgfsetstrokecolor{currentstroke}%
\pgfsetdash{}{0pt}%
\pgfpathmoveto{\pgfqpoint{0.625000in}{1.025547in}}%
\pgfpathlineto{\pgfqpoint{0.629528in}{1.032456in}}%
\pgfpathlineto{\pgfqpoint{0.625000in}{1.036721in}}%
\pgfusepath{stroke}%
\end{pgfscope}%
\begin{pgfscope}%
\pgfpathrectangle{\pgfqpoint{0.625000in}{0.550000in}}{\pgfqpoint{3.875000in}{3.850000in}} %
\pgfusepath{clip}%
\pgfsetbuttcap%
\pgfsetroundjoin%
\pgfsetlinewidth{0.250937pt}%
\definecolor{currentstroke}{rgb}{0.000000,0.000000,0.000000}%
\pgfsetstrokecolor{currentstroke}%
\pgfsetdash{}{0pt}%
\pgfpathmoveto{\pgfqpoint{0.625000in}{1.121730in}}%
\pgfpathlineto{\pgfqpoint{0.627251in}{1.128947in}}%
\pgfpathlineto{\pgfqpoint{0.625000in}{1.134831in}}%
\pgfusepath{stroke}%
\end{pgfscope}%
\begin{pgfscope}%
\pgfpathrectangle{\pgfqpoint{0.625000in}{0.550000in}}{\pgfqpoint{3.875000in}{3.850000in}} %
\pgfusepath{clip}%
\pgfsetbuttcap%
\pgfsetroundjoin%
\pgfsetlinewidth{0.250937pt}%
\definecolor{currentstroke}{rgb}{0.000000,0.000000,0.000000}%
\pgfsetstrokecolor{currentstroke}%
\pgfsetdash{}{0pt}%
\pgfpathmoveto{\pgfqpoint{0.625000in}{1.214079in}}%
\pgfpathlineto{\pgfqpoint{0.625212in}{1.215789in}}%
\pgfpathlineto{\pgfqpoint{0.625438in}{1.225439in}}%
\pgfpathlineto{\pgfqpoint{0.625000in}{1.226842in}}%
\pgfusepath{stroke}%
\end{pgfscope}%
\begin{pgfscope}%
\pgfpathrectangle{\pgfqpoint{0.625000in}{0.550000in}}{\pgfqpoint{3.875000in}{3.850000in}} %
\pgfusepath{clip}%
\pgfsetbuttcap%
\pgfsetroundjoin%
\pgfsetlinewidth{0.250937pt}%
\definecolor{currentstroke}{rgb}{0.000000,0.000000,0.000000}%
\pgfsetstrokecolor{currentstroke}%
\pgfsetdash{}{0pt}%
\pgfpathmoveto{\pgfqpoint{0.625000in}{1.269452in}}%
\pgfpathlineto{\pgfqpoint{0.627734in}{1.273684in}}%
\pgfpathlineto{\pgfqpoint{0.625000in}{1.279554in}}%
\pgfusepath{stroke}%
\end{pgfscope}%
\begin{pgfscope}%
\pgfpathrectangle{\pgfqpoint{0.625000in}{0.550000in}}{\pgfqpoint{3.875000in}{3.850000in}} %
\pgfusepath{clip}%
\pgfsetbuttcap%
\pgfsetroundjoin%
\pgfsetlinewidth{0.250937pt}%
\definecolor{currentstroke}{rgb}{0.000000,0.000000,0.000000}%
\pgfsetstrokecolor{currentstroke}%
\pgfsetdash{}{0pt}%
\pgfpathmoveto{\pgfqpoint{0.625000in}{1.314036in}}%
\pgfpathlineto{\pgfqpoint{0.628974in}{1.321930in}}%
\pgfpathlineto{\pgfqpoint{0.625000in}{1.330981in}}%
\pgfusepath{stroke}%
\end{pgfscope}%
\begin{pgfscope}%
\pgfpathrectangle{\pgfqpoint{0.625000in}{0.550000in}}{\pgfqpoint{3.875000in}{3.850000in}} %
\pgfusepath{clip}%
\pgfsetbuttcap%
\pgfsetroundjoin%
\pgfsetlinewidth{0.250937pt}%
\definecolor{currentstroke}{rgb}{0.000000,0.000000,0.000000}%
\pgfsetstrokecolor{currentstroke}%
\pgfsetdash{}{0pt}%
\pgfpathmoveto{\pgfqpoint{0.625000in}{1.416102in}}%
\pgfpathlineto{\pgfqpoint{0.625572in}{1.418421in}}%
\pgfpathlineto{\pgfqpoint{0.625000in}{1.419069in}}%
\pgfusepath{stroke}%
\end{pgfscope}%
\begin{pgfscope}%
\pgfpathrectangle{\pgfqpoint{0.625000in}{0.550000in}}{\pgfqpoint{3.875000in}{3.850000in}} %
\pgfusepath{clip}%
\pgfsetbuttcap%
\pgfsetroundjoin%
\pgfsetlinewidth{0.250937pt}%
\definecolor{currentstroke}{rgb}{0.000000,0.000000,0.000000}%
\pgfsetstrokecolor{currentstroke}%
\pgfsetdash{}{0pt}%
\pgfpathmoveto{\pgfqpoint{0.625000in}{1.508693in}}%
\pgfpathlineto{\pgfqpoint{0.631249in}{1.514912in}}%
\pgfpathlineto{\pgfqpoint{0.625000in}{1.521439in}}%
\pgfusepath{stroke}%
\end{pgfscope}%
\begin{pgfscope}%
\pgfpathrectangle{\pgfqpoint{0.625000in}{0.550000in}}{\pgfqpoint{3.875000in}{3.850000in}} %
\pgfusepath{clip}%
\pgfsetbuttcap%
\pgfsetroundjoin%
\pgfsetlinewidth{0.250937pt}%
\definecolor{currentstroke}{rgb}{0.000000,0.000000,0.000000}%
\pgfsetstrokecolor{currentstroke}%
\pgfsetdash{}{0pt}%
\pgfpathmoveto{\pgfqpoint{0.625000in}{1.607220in}}%
\pgfpathlineto{\pgfqpoint{0.625540in}{1.611404in}}%
\pgfpathlineto{\pgfqpoint{0.625000in}{1.612076in}}%
\pgfusepath{stroke}%
\end{pgfscope}%
\begin{pgfscope}%
\pgfpathrectangle{\pgfqpoint{0.625000in}{0.550000in}}{\pgfqpoint{3.875000in}{3.850000in}} %
\pgfusepath{clip}%
\pgfsetbuttcap%
\pgfsetroundjoin%
\pgfsetlinewidth{0.250937pt}%
\definecolor{currentstroke}{rgb}{0.000000,0.000000,0.000000}%
\pgfsetstrokecolor{currentstroke}%
\pgfsetdash{}{0pt}%
\pgfpathmoveto{\pgfqpoint{0.625000in}{1.659457in}}%
\pgfpathlineto{\pgfqpoint{0.625095in}{1.659649in}}%
\pgfpathlineto{\pgfqpoint{0.625000in}{1.660023in}}%
\pgfusepath{stroke}%
\end{pgfscope}%
\begin{pgfscope}%
\pgfpathrectangle{\pgfqpoint{0.625000in}{0.550000in}}{\pgfqpoint{3.875000in}{3.850000in}} %
\pgfusepath{clip}%
\pgfsetbuttcap%
\pgfsetroundjoin%
\pgfsetlinewidth{0.250937pt}%
\definecolor{currentstroke}{rgb}{0.000000,0.000000,0.000000}%
\pgfsetstrokecolor{currentstroke}%
\pgfsetdash{}{0pt}%
\pgfpathmoveto{\pgfqpoint{0.625000in}{1.750096in}}%
\pgfpathlineto{\pgfqpoint{0.627887in}{1.756140in}}%
\pgfpathlineto{\pgfqpoint{0.625000in}{1.760555in}}%
\pgfusepath{stroke}%
\end{pgfscope}%
\begin{pgfscope}%
\pgfpathrectangle{\pgfqpoint{0.625000in}{0.550000in}}{\pgfqpoint{3.875000in}{3.850000in}} %
\pgfusepath{clip}%
\pgfsetbuttcap%
\pgfsetroundjoin%
\pgfsetlinewidth{0.250937pt}%
\definecolor{currentstroke}{rgb}{0.000000,0.000000,0.000000}%
\pgfsetstrokecolor{currentstroke}%
\pgfsetdash{}{0pt}%
\pgfpathmoveto{\pgfqpoint{0.625000in}{1.786321in}}%
\pgfpathlineto{\pgfqpoint{0.625174in}{1.794737in}}%
\pgfpathlineto{\pgfqpoint{0.625639in}{1.804386in}}%
\pgfpathlineto{\pgfqpoint{0.625000in}{1.805454in}}%
\pgfusepath{stroke}%
\end{pgfscope}%
\begin{pgfscope}%
\pgfpathrectangle{\pgfqpoint{0.625000in}{0.550000in}}{\pgfqpoint{3.875000in}{3.850000in}} %
\pgfusepath{clip}%
\pgfsetbuttcap%
\pgfsetroundjoin%
\pgfsetlinewidth{0.250937pt}%
\definecolor{currentstroke}{rgb}{0.000000,0.000000,0.000000}%
\pgfsetstrokecolor{currentstroke}%
\pgfsetdash{}{0pt}%
\pgfpathmoveto{\pgfqpoint{0.625000in}{1.851748in}}%
\pgfpathlineto{\pgfqpoint{0.625524in}{1.852632in}}%
\pgfpathlineto{\pgfqpoint{0.625000in}{1.852654in}}%
\pgfusepath{stroke}%
\end{pgfscope}%
\begin{pgfscope}%
\pgfpathrectangle{\pgfqpoint{0.625000in}{0.550000in}}{\pgfqpoint{3.875000in}{3.850000in}} %
\pgfusepath{clip}%
\pgfsetbuttcap%
\pgfsetroundjoin%
\pgfsetlinewidth{0.250937pt}%
\definecolor{currentstroke}{rgb}{0.000000,0.000000,0.000000}%
\pgfsetstrokecolor{currentstroke}%
\pgfsetdash{}{0pt}%
\pgfpathmoveto{\pgfqpoint{0.625000in}{1.899458in}}%
\pgfpathlineto{\pgfqpoint{0.627256in}{1.900877in}}%
\pgfpathlineto{\pgfqpoint{0.625000in}{1.903329in}}%
\pgfusepath{stroke}%
\end{pgfscope}%
\begin{pgfscope}%
\pgfpathrectangle{\pgfqpoint{0.625000in}{0.550000in}}{\pgfqpoint{3.875000in}{3.850000in}} %
\pgfusepath{clip}%
\pgfsetbuttcap%
\pgfsetroundjoin%
\pgfsetlinewidth{0.250937pt}%
\definecolor{currentstroke}{rgb}{0.000000,0.000000,0.000000}%
\pgfsetstrokecolor{currentstroke}%
\pgfsetdash{}{0pt}%
\pgfpathmoveto{\pgfqpoint{0.625000in}{1.990660in}}%
\pgfpathlineto{\pgfqpoint{0.629528in}{1.997368in}}%
\pgfpathlineto{\pgfqpoint{0.625000in}{2.001223in}}%
\pgfusepath{stroke}%
\end{pgfscope}%
\begin{pgfscope}%
\pgfpathrectangle{\pgfqpoint{0.625000in}{0.550000in}}{\pgfqpoint{3.875000in}{3.850000in}} %
\pgfusepath{clip}%
\pgfsetbuttcap%
\pgfsetroundjoin%
\pgfsetlinewidth{0.250937pt}%
\definecolor{currentstroke}{rgb}{0.000000,0.000000,0.000000}%
\pgfsetstrokecolor{currentstroke}%
\pgfsetdash{}{0pt}%
\pgfpathmoveto{\pgfqpoint{0.625000in}{2.088771in}}%
\pgfpathlineto{\pgfqpoint{0.628974in}{2.093860in}}%
\pgfpathlineto{\pgfqpoint{0.625000in}{2.098968in}}%
\pgfusepath{stroke}%
\end{pgfscope}%
\begin{pgfscope}%
\pgfpathrectangle{\pgfqpoint{0.625000in}{0.550000in}}{\pgfqpoint{3.875000in}{3.850000in}} %
\pgfusepath{clip}%
\pgfsetbuttcap%
\pgfsetroundjoin%
\pgfsetlinewidth{0.250937pt}%
\definecolor{currentstroke}{rgb}{0.000000,0.000000,0.000000}%
\pgfsetstrokecolor{currentstroke}%
\pgfsetdash{}{0pt}%
\pgfpathmoveto{\pgfqpoint{0.625000in}{2.185034in}}%
\pgfpathlineto{\pgfqpoint{0.625599in}{2.190351in}}%
\pgfpathlineto{\pgfqpoint{0.625000in}{2.191895in}}%
\pgfusepath{stroke}%
\end{pgfscope}%
\begin{pgfscope}%
\pgfpathrectangle{\pgfqpoint{0.625000in}{0.550000in}}{\pgfqpoint{3.875000in}{3.850000in}} %
\pgfusepath{clip}%
\pgfsetbuttcap%
\pgfsetroundjoin%
\pgfsetlinewidth{0.250937pt}%
\definecolor{currentstroke}{rgb}{0.000000,0.000000,0.000000}%
\pgfsetstrokecolor{currentstroke}%
\pgfsetdash{}{0pt}%
\pgfpathmoveto{\pgfqpoint{0.625000in}{2.234511in}}%
\pgfpathlineto{\pgfqpoint{0.627614in}{2.238596in}}%
\pgfpathlineto{\pgfqpoint{0.625000in}{2.248166in}}%
\pgfusepath{stroke}%
\end{pgfscope}%
\begin{pgfscope}%
\pgfpathrectangle{\pgfqpoint{0.625000in}{0.550000in}}{\pgfqpoint{3.875000in}{3.850000in}} %
\pgfusepath{clip}%
\pgfsetbuttcap%
\pgfsetroundjoin%
\pgfsetlinewidth{0.250937pt}%
\definecolor{currentstroke}{rgb}{0.000000,0.000000,0.000000}%
\pgfsetstrokecolor{currentstroke}%
\pgfsetdash{}{0pt}%
\pgfpathmoveto{\pgfqpoint{0.625000in}{2.283709in}}%
\pgfpathlineto{\pgfqpoint{0.627249in}{2.286842in}}%
\pgfpathlineto{\pgfqpoint{0.625000in}{2.290343in}}%
\pgfusepath{stroke}%
\end{pgfscope}%
\begin{pgfscope}%
\pgfpathrectangle{\pgfqpoint{0.625000in}{0.550000in}}{\pgfqpoint{3.875000in}{3.850000in}} %
\pgfusepath{clip}%
\pgfsetbuttcap%
\pgfsetroundjoin%
\pgfsetlinewidth{0.250937pt}%
\definecolor{currentstroke}{rgb}{0.000000,0.000000,0.000000}%
\pgfsetstrokecolor{currentstroke}%
\pgfsetdash{}{0pt}%
\pgfpathmoveto{\pgfqpoint{0.625000in}{2.335086in}}%
\pgfpathlineto{\pgfqpoint{0.625028in}{2.335088in}}%
\pgfpathlineto{\pgfqpoint{0.625000in}{2.335156in}}%
\pgfusepath{stroke}%
\end{pgfscope}%
\begin{pgfscope}%
\pgfpathrectangle{\pgfqpoint{0.625000in}{0.550000in}}{\pgfqpoint{3.875000in}{3.850000in}} %
\pgfusepath{clip}%
\pgfsetbuttcap%
\pgfsetroundjoin%
\pgfsetlinewidth{0.250937pt}%
\definecolor{currentstroke}{rgb}{0.000000,0.000000,0.000000}%
\pgfsetstrokecolor{currentstroke}%
\pgfsetdash{}{0pt}%
\pgfpathmoveto{\pgfqpoint{0.625000in}{2.382461in}}%
\pgfpathlineto{\pgfqpoint{0.625421in}{2.383333in}}%
\pgfpathlineto{\pgfqpoint{0.625000in}{2.385730in}}%
\pgfusepath{stroke}%
\end{pgfscope}%
\begin{pgfscope}%
\pgfpathrectangle{\pgfqpoint{0.625000in}{0.550000in}}{\pgfqpoint{3.875000in}{3.850000in}} %
\pgfusepath{clip}%
\pgfsetbuttcap%
\pgfsetroundjoin%
\pgfsetlinewidth{0.250937pt}%
\definecolor{currentstroke}{rgb}{0.000000,0.000000,0.000000}%
\pgfsetstrokecolor{currentstroke}%
\pgfsetdash{}{0pt}%
\pgfpathmoveto{\pgfqpoint{0.625000in}{2.573919in}}%
\pgfpathlineto{\pgfqpoint{0.625421in}{2.576316in}}%
\pgfpathlineto{\pgfqpoint{0.625000in}{2.577188in}}%
\pgfusepath{stroke}%
\end{pgfscope}%
\begin{pgfscope}%
\pgfpathrectangle{\pgfqpoint{0.625000in}{0.550000in}}{\pgfqpoint{3.875000in}{3.850000in}} %
\pgfusepath{clip}%
\pgfsetbuttcap%
\pgfsetroundjoin%
\pgfsetlinewidth{0.250937pt}%
\definecolor{currentstroke}{rgb}{0.000000,0.000000,0.000000}%
\pgfsetstrokecolor{currentstroke}%
\pgfsetdash{}{0pt}%
\pgfpathmoveto{\pgfqpoint{0.625000in}{2.624493in}}%
\pgfpathlineto{\pgfqpoint{0.625028in}{2.624561in}}%
\pgfpathlineto{\pgfqpoint{0.625000in}{2.624563in}}%
\pgfusepath{stroke}%
\end{pgfscope}%
\begin{pgfscope}%
\pgfpathrectangle{\pgfqpoint{0.625000in}{0.550000in}}{\pgfqpoint{3.875000in}{3.850000in}} %
\pgfusepath{clip}%
\pgfsetbuttcap%
\pgfsetroundjoin%
\pgfsetlinewidth{0.250937pt}%
\definecolor{currentstroke}{rgb}{0.000000,0.000000,0.000000}%
\pgfsetstrokecolor{currentstroke}%
\pgfsetdash{}{0pt}%
\pgfpathmoveto{\pgfqpoint{0.625000in}{2.669306in}}%
\pgfpathlineto{\pgfqpoint{0.627249in}{2.672807in}}%
\pgfpathlineto{\pgfqpoint{0.625000in}{2.675940in}}%
\pgfusepath{stroke}%
\end{pgfscope}%
\begin{pgfscope}%
\pgfpathrectangle{\pgfqpoint{0.625000in}{0.550000in}}{\pgfqpoint{3.875000in}{3.850000in}} %
\pgfusepath{clip}%
\pgfsetbuttcap%
\pgfsetroundjoin%
\pgfsetlinewidth{0.250937pt}%
\definecolor{currentstroke}{rgb}{0.000000,0.000000,0.000000}%
\pgfsetstrokecolor{currentstroke}%
\pgfsetdash{}{0pt}%
\pgfpathmoveto{\pgfqpoint{0.625000in}{2.711483in}}%
\pgfpathlineto{\pgfqpoint{0.627614in}{2.721053in}}%
\pgfpathlineto{\pgfqpoint{0.625000in}{2.725138in}}%
\pgfusepath{stroke}%
\end{pgfscope}%
\begin{pgfscope}%
\pgfpathrectangle{\pgfqpoint{0.625000in}{0.550000in}}{\pgfqpoint{3.875000in}{3.850000in}} %
\pgfusepath{clip}%
\pgfsetbuttcap%
\pgfsetroundjoin%
\pgfsetlinewidth{0.250937pt}%
\definecolor{currentstroke}{rgb}{0.000000,0.000000,0.000000}%
\pgfsetstrokecolor{currentstroke}%
\pgfsetdash{}{0pt}%
\pgfpathmoveto{\pgfqpoint{0.625000in}{2.767754in}}%
\pgfpathlineto{\pgfqpoint{0.625599in}{2.769298in}}%
\pgfpathlineto{\pgfqpoint{0.625000in}{2.774615in}}%
\pgfusepath{stroke}%
\end{pgfscope}%
\begin{pgfscope}%
\pgfpathrectangle{\pgfqpoint{0.625000in}{0.550000in}}{\pgfqpoint{3.875000in}{3.850000in}} %
\pgfusepath{clip}%
\pgfsetbuttcap%
\pgfsetroundjoin%
\pgfsetlinewidth{0.250937pt}%
\definecolor{currentstroke}{rgb}{0.000000,0.000000,0.000000}%
\pgfsetstrokecolor{currentstroke}%
\pgfsetdash{}{0pt}%
\pgfpathmoveto{\pgfqpoint{0.625000in}{2.860681in}}%
\pgfpathlineto{\pgfqpoint{0.628974in}{2.865789in}}%
\pgfpathlineto{\pgfqpoint{0.625000in}{2.870878in}}%
\pgfusepath{stroke}%
\end{pgfscope}%
\begin{pgfscope}%
\pgfpathrectangle{\pgfqpoint{0.625000in}{0.550000in}}{\pgfqpoint{3.875000in}{3.850000in}} %
\pgfusepath{clip}%
\pgfsetbuttcap%
\pgfsetroundjoin%
\pgfsetlinewidth{0.250937pt}%
\definecolor{currentstroke}{rgb}{0.000000,0.000000,0.000000}%
\pgfsetstrokecolor{currentstroke}%
\pgfsetdash{}{0pt}%
\pgfpathmoveto{\pgfqpoint{0.625000in}{2.958426in}}%
\pgfpathlineto{\pgfqpoint{0.629528in}{2.962281in}}%
\pgfpathlineto{\pgfqpoint{0.625000in}{2.968989in}}%
\pgfusepath{stroke}%
\end{pgfscope}%
\begin{pgfscope}%
\pgfpathrectangle{\pgfqpoint{0.625000in}{0.550000in}}{\pgfqpoint{3.875000in}{3.850000in}} %
\pgfusepath{clip}%
\pgfsetbuttcap%
\pgfsetroundjoin%
\pgfsetlinewidth{0.250937pt}%
\definecolor{currentstroke}{rgb}{0.000000,0.000000,0.000000}%
\pgfsetstrokecolor{currentstroke}%
\pgfsetdash{}{0pt}%
\pgfpathmoveto{\pgfqpoint{0.625000in}{3.056320in}}%
\pgfpathlineto{\pgfqpoint{0.627256in}{3.058772in}}%
\pgfpathlineto{\pgfqpoint{0.625000in}{3.060191in}}%
\pgfusepath{stroke}%
\end{pgfscope}%
\begin{pgfscope}%
\pgfpathrectangle{\pgfqpoint{0.625000in}{0.550000in}}{\pgfqpoint{3.875000in}{3.850000in}} %
\pgfusepath{clip}%
\pgfsetbuttcap%
\pgfsetroundjoin%
\pgfsetlinewidth{0.250937pt}%
\definecolor{currentstroke}{rgb}{0.000000,0.000000,0.000000}%
\pgfsetstrokecolor{currentstroke}%
\pgfsetdash{}{0pt}%
\pgfpathmoveto{\pgfqpoint{0.625000in}{3.106994in}}%
\pgfpathlineto{\pgfqpoint{0.625524in}{3.107018in}}%
\pgfpathlineto{\pgfqpoint{0.625000in}{3.107902in}}%
\pgfusepath{stroke}%
\end{pgfscope}%
\begin{pgfscope}%
\pgfpathrectangle{\pgfqpoint{0.625000in}{0.550000in}}{\pgfqpoint{3.875000in}{3.850000in}} %
\pgfusepath{clip}%
\pgfsetbuttcap%
\pgfsetroundjoin%
\pgfsetlinewidth{0.250937pt}%
\definecolor{currentstroke}{rgb}{0.000000,0.000000,0.000000}%
\pgfsetstrokecolor{currentstroke}%
\pgfsetdash{}{0pt}%
\pgfpathmoveto{\pgfqpoint{0.625000in}{3.154195in}}%
\pgfpathlineto{\pgfqpoint{0.625639in}{3.155263in}}%
\pgfpathlineto{\pgfqpoint{0.625174in}{3.164912in}}%
\pgfpathlineto{\pgfqpoint{0.625000in}{3.173328in}}%
\pgfusepath{stroke}%
\end{pgfscope}%
\begin{pgfscope}%
\pgfpathrectangle{\pgfqpoint{0.625000in}{0.550000in}}{\pgfqpoint{3.875000in}{3.850000in}} %
\pgfusepath{clip}%
\pgfsetbuttcap%
\pgfsetroundjoin%
\pgfsetlinewidth{0.250937pt}%
\definecolor{currentstroke}{rgb}{0.000000,0.000000,0.000000}%
\pgfsetstrokecolor{currentstroke}%
\pgfsetdash{}{0pt}%
\pgfpathmoveto{\pgfqpoint{0.625000in}{3.199094in}}%
\pgfpathlineto{\pgfqpoint{0.627887in}{3.203509in}}%
\pgfpathlineto{\pgfqpoint{0.625000in}{3.209553in}}%
\pgfusepath{stroke}%
\end{pgfscope}%
\begin{pgfscope}%
\pgfpathrectangle{\pgfqpoint{0.625000in}{0.550000in}}{\pgfqpoint{3.875000in}{3.850000in}} %
\pgfusepath{clip}%
\pgfsetbuttcap%
\pgfsetroundjoin%
\pgfsetlinewidth{0.250937pt}%
\definecolor{currentstroke}{rgb}{0.000000,0.000000,0.000000}%
\pgfsetstrokecolor{currentstroke}%
\pgfsetdash{}{0pt}%
\pgfpathmoveto{\pgfqpoint{0.625000in}{3.299626in}}%
\pgfpathlineto{\pgfqpoint{0.625095in}{3.300000in}}%
\pgfpathlineto{\pgfqpoint{0.625000in}{3.300192in}}%
\pgfusepath{stroke}%
\end{pgfscope}%
\begin{pgfscope}%
\pgfpathrectangle{\pgfqpoint{0.625000in}{0.550000in}}{\pgfqpoint{3.875000in}{3.850000in}} %
\pgfusepath{clip}%
\pgfsetbuttcap%
\pgfsetroundjoin%
\pgfsetlinewidth{0.250937pt}%
\definecolor{currentstroke}{rgb}{0.000000,0.000000,0.000000}%
\pgfsetstrokecolor{currentstroke}%
\pgfsetdash{}{0pt}%
\pgfpathmoveto{\pgfqpoint{0.625000in}{3.347573in}}%
\pgfpathlineto{\pgfqpoint{0.625540in}{3.348246in}}%
\pgfpathlineto{\pgfqpoint{0.625000in}{3.352429in}}%
\pgfusepath{stroke}%
\end{pgfscope}%
\begin{pgfscope}%
\pgfpathrectangle{\pgfqpoint{0.625000in}{0.550000in}}{\pgfqpoint{3.875000in}{3.850000in}} %
\pgfusepath{clip}%
\pgfsetbuttcap%
\pgfsetroundjoin%
\pgfsetlinewidth{0.250937pt}%
\definecolor{currentstroke}{rgb}{0.000000,0.000000,0.000000}%
\pgfsetstrokecolor{currentstroke}%
\pgfsetdash{}{0pt}%
\pgfpathmoveto{\pgfqpoint{0.625000in}{3.438210in}}%
\pgfpathlineto{\pgfqpoint{0.631249in}{3.444737in}}%
\pgfpathlineto{\pgfqpoint{0.625000in}{3.450956in}}%
\pgfusepath{stroke}%
\end{pgfscope}%
\begin{pgfscope}%
\pgfpathrectangle{\pgfqpoint{0.625000in}{0.550000in}}{\pgfqpoint{3.875000in}{3.850000in}} %
\pgfusepath{clip}%
\pgfsetbuttcap%
\pgfsetroundjoin%
\pgfsetlinewidth{0.250937pt}%
\definecolor{currentstroke}{rgb}{0.000000,0.000000,0.000000}%
\pgfsetstrokecolor{currentstroke}%
\pgfsetdash{}{0pt}%
\pgfpathmoveto{\pgfqpoint{0.625000in}{3.540580in}}%
\pgfpathlineto{\pgfqpoint{0.625572in}{3.541228in}}%
\pgfpathlineto{\pgfqpoint{0.625000in}{3.543547in}}%
\pgfusepath{stroke}%
\end{pgfscope}%
\begin{pgfscope}%
\pgfpathrectangle{\pgfqpoint{0.625000in}{0.550000in}}{\pgfqpoint{3.875000in}{3.850000in}} %
\pgfusepath{clip}%
\pgfsetbuttcap%
\pgfsetroundjoin%
\pgfsetlinewidth{0.250937pt}%
\definecolor{currentstroke}{rgb}{0.000000,0.000000,0.000000}%
\pgfsetstrokecolor{currentstroke}%
\pgfsetdash{}{0pt}%
\pgfpathmoveto{\pgfqpoint{0.625000in}{3.628668in}}%
\pgfpathlineto{\pgfqpoint{0.628974in}{3.637719in}}%
\pgfpathlineto{\pgfqpoint{0.625000in}{3.645614in}}%
\pgfusepath{stroke}%
\end{pgfscope}%
\begin{pgfscope}%
\pgfpathrectangle{\pgfqpoint{0.625000in}{0.550000in}}{\pgfqpoint{3.875000in}{3.850000in}} %
\pgfusepath{clip}%
\pgfsetbuttcap%
\pgfsetroundjoin%
\pgfsetlinewidth{0.250937pt}%
\definecolor{currentstroke}{rgb}{0.000000,0.000000,0.000000}%
\pgfsetstrokecolor{currentstroke}%
\pgfsetdash{}{0pt}%
\pgfpathmoveto{\pgfqpoint{0.625000in}{3.680095in}}%
\pgfpathlineto{\pgfqpoint{0.627734in}{3.685965in}}%
\pgfpathlineto{\pgfqpoint{0.625000in}{3.690198in}}%
\pgfusepath{stroke}%
\end{pgfscope}%
\begin{pgfscope}%
\pgfpathrectangle{\pgfqpoint{0.625000in}{0.550000in}}{\pgfqpoint{3.875000in}{3.850000in}} %
\pgfusepath{clip}%
\pgfsetbuttcap%
\pgfsetroundjoin%
\pgfsetlinewidth{0.250937pt}%
\definecolor{currentstroke}{rgb}{0.000000,0.000000,0.000000}%
\pgfsetstrokecolor{currentstroke}%
\pgfsetdash{}{0pt}%
\pgfpathmoveto{\pgfqpoint{0.625000in}{3.732807in}}%
\pgfpathlineto{\pgfqpoint{0.625438in}{3.734211in}}%
\pgfpathlineto{\pgfqpoint{0.625212in}{3.743860in}}%
\pgfpathlineto{\pgfqpoint{0.625000in}{3.745570in}}%
\pgfusepath{stroke}%
\end{pgfscope}%
\begin{pgfscope}%
\pgfpathrectangle{\pgfqpoint{0.625000in}{0.550000in}}{\pgfqpoint{3.875000in}{3.850000in}} %
\pgfusepath{clip}%
\pgfsetbuttcap%
\pgfsetroundjoin%
\pgfsetlinewidth{0.250937pt}%
\definecolor{currentstroke}{rgb}{0.000000,0.000000,0.000000}%
\pgfsetstrokecolor{currentstroke}%
\pgfsetdash{}{0pt}%
\pgfpathmoveto{\pgfqpoint{0.625000in}{3.824818in}}%
\pgfpathlineto{\pgfqpoint{0.627251in}{3.830702in}}%
\pgfpathlineto{\pgfqpoint{0.625000in}{3.837920in}}%
\pgfusepath{stroke}%
\end{pgfscope}%
\begin{pgfscope}%
\pgfpathrectangle{\pgfqpoint{0.625000in}{0.550000in}}{\pgfqpoint{3.875000in}{3.850000in}} %
\pgfusepath{clip}%
\pgfsetbuttcap%
\pgfsetroundjoin%
\pgfsetlinewidth{0.250937pt}%
\definecolor{currentstroke}{rgb}{0.000000,0.000000,0.000000}%
\pgfsetstrokecolor{currentstroke}%
\pgfsetdash{}{0pt}%
\pgfpathmoveto{\pgfqpoint{0.625000in}{3.922928in}}%
\pgfpathlineto{\pgfqpoint{0.629528in}{3.927193in}}%
\pgfpathlineto{\pgfqpoint{0.625000in}{3.934102in}}%
\pgfusepath{stroke}%
\end{pgfscope}%
\begin{pgfscope}%
\pgfpathrectangle{\pgfqpoint{0.625000in}{0.550000in}}{\pgfqpoint{3.875000in}{3.850000in}} %
\pgfusepath{clip}%
\pgfsetbuttcap%
\pgfsetroundjoin%
\pgfsetlinewidth{0.250937pt}%
\definecolor{currentstroke}{rgb}{0.000000,0.000000,0.000000}%
\pgfsetstrokecolor{currentstroke}%
\pgfsetdash{}{0pt}%
\pgfpathmoveto{\pgfqpoint{0.625000in}{4.117885in}}%
\pgfpathlineto{\pgfqpoint{0.625503in}{4.120175in}}%
\pgfpathlineto{\pgfqpoint{0.625027in}{4.129825in}}%
\pgfpathlineto{\pgfqpoint{0.625000in}{4.130078in}}%
\pgfusepath{stroke}%
\end{pgfscope}%
\begin{pgfscope}%
\pgfpathrectangle{\pgfqpoint{0.625000in}{0.550000in}}{\pgfqpoint{3.875000in}{3.850000in}} %
\pgfusepath{clip}%
\pgfsetbuttcap%
\pgfsetroundjoin%
\pgfsetlinewidth{0.250937pt}%
\definecolor{currentstroke}{rgb}{0.000000,0.000000,0.000000}%
\pgfsetstrokecolor{currentstroke}%
\pgfsetdash{}{0pt}%
\pgfpathmoveto{\pgfqpoint{0.625000in}{4.163893in}}%
\pgfpathlineto{\pgfqpoint{0.627984in}{4.168421in}}%
\pgfpathlineto{\pgfqpoint{0.625000in}{4.168629in}}%
\pgfusepath{stroke}%
\end{pgfscope}%
\begin{pgfscope}%
\pgfpathrectangle{\pgfqpoint{0.625000in}{0.550000in}}{\pgfqpoint{3.875000in}{3.850000in}} %
\pgfusepath{clip}%
\pgfsetbuttcap%
\pgfsetroundjoin%
\pgfsetlinewidth{0.250937pt}%
\definecolor{currentstroke}{rgb}{0.000000,0.000000,0.000000}%
\pgfsetstrokecolor{currentstroke}%
\pgfsetdash{}{0pt}%
\pgfpathmoveto{\pgfqpoint{0.625000in}{4.205939in}}%
\pgfpathlineto{\pgfqpoint{0.625461in}{4.207018in}}%
\pgfpathlineto{\pgfqpoint{0.627258in}{4.216667in}}%
\pgfpathlineto{\pgfqpoint{0.625345in}{4.226316in}}%
\pgfpathlineto{\pgfqpoint{0.625000in}{4.227136in}}%
\pgfusepath{stroke}%
\end{pgfscope}%
\begin{pgfscope}%
\pgfpathrectangle{\pgfqpoint{0.625000in}{0.550000in}}{\pgfqpoint{3.875000in}{3.850000in}} %
\pgfusepath{clip}%
\pgfsetbuttcap%
\pgfsetroundjoin%
\pgfsetlinewidth{0.250937pt}%
\definecolor{currentstroke}{rgb}{0.000000,0.000000,0.000000}%
\pgfsetstrokecolor{currentstroke}%
\pgfsetdash{}{0pt}%
\pgfpathmoveto{\pgfqpoint{0.625000in}{4.311118in}}%
\pgfpathlineto{\pgfqpoint{0.625648in}{4.313158in}}%
\pgfpathlineto{\pgfqpoint{0.625000in}{4.314563in}}%
\pgfusepath{stroke}%
\end{pgfscope}%
\begin{pgfscope}%
\pgfpathrectangle{\pgfqpoint{0.625000in}{0.550000in}}{\pgfqpoint{3.875000in}{3.850000in}} %
\pgfusepath{clip}%
\pgfsetbuttcap%
\pgfsetroundjoin%
\pgfsetlinewidth{0.250937pt}%
\definecolor{currentstroke}{rgb}{0.000000,0.000000,0.000000}%
\pgfsetstrokecolor{currentstroke}%
\pgfsetdash{}{0pt}%
\pgfpathmoveto{\pgfqpoint{0.631729in}{0.550000in}}%
\pgfpathlineto{\pgfqpoint{0.625000in}{0.556858in}}%
\pgfusepath{stroke}%
\end{pgfscope}%
\begin{pgfscope}%
\pgfpathrectangle{\pgfqpoint{0.625000in}{0.550000in}}{\pgfqpoint{3.875000in}{3.850000in}} %
\pgfusepath{clip}%
\pgfsetbuttcap%
\pgfsetroundjoin%
\pgfsetlinewidth{0.250937pt}%
\definecolor{currentstroke}{rgb}{0.000000,0.000000,0.000000}%
\pgfsetstrokecolor{currentstroke}%
\pgfsetdash{}{0pt}%
\pgfpathmoveto{\pgfqpoint{0.625000in}{0.737926in}}%
\pgfpathlineto{\pgfqpoint{0.626014in}{0.742982in}}%
\pgfpathlineto{\pgfqpoint{0.625000in}{0.748356in}}%
\pgfusepath{stroke}%
\end{pgfscope}%
\begin{pgfscope}%
\pgfpathrectangle{\pgfqpoint{0.625000in}{0.550000in}}{\pgfqpoint{3.875000in}{3.850000in}} %
\pgfusepath{clip}%
\pgfsetbuttcap%
\pgfsetroundjoin%
\pgfsetlinewidth{0.250937pt}%
\definecolor{currentstroke}{rgb}{0.000000,0.000000,0.000000}%
\pgfsetstrokecolor{currentstroke}%
\pgfsetdash{}{0pt}%
\pgfpathmoveto{\pgfqpoint{0.625000in}{0.791115in}}%
\pgfpathlineto{\pgfqpoint{0.626774in}{0.791228in}}%
\pgfpathlineto{\pgfqpoint{0.625000in}{0.793920in}}%
\pgfusepath{stroke}%
\end{pgfscope}%
\begin{pgfscope}%
\pgfpathrectangle{\pgfqpoint{0.625000in}{0.550000in}}{\pgfqpoint{3.875000in}{3.850000in}} %
\pgfusepath{clip}%
\pgfsetbuttcap%
\pgfsetroundjoin%
\pgfsetlinewidth{0.250937pt}%
\definecolor{currentstroke}{rgb}{0.000000,0.000000,0.000000}%
\pgfsetstrokecolor{currentstroke}%
\pgfsetdash{}{0pt}%
\pgfpathmoveto{\pgfqpoint{0.625000in}{1.027441in}}%
\pgfpathlineto{\pgfqpoint{0.628287in}{1.032456in}}%
\pgfpathlineto{\pgfqpoint{0.625000in}{1.035552in}}%
\pgfusepath{stroke}%
\end{pgfscope}%
\begin{pgfscope}%
\pgfpathrectangle{\pgfqpoint{0.625000in}{0.550000in}}{\pgfqpoint{3.875000in}{3.850000in}} %
\pgfusepath{clip}%
\pgfsetbuttcap%
\pgfsetroundjoin%
\pgfsetlinewidth{0.250937pt}%
\definecolor{currentstroke}{rgb}{0.000000,0.000000,0.000000}%
\pgfsetstrokecolor{currentstroke}%
\pgfsetdash{}{0pt}%
\pgfpathmoveto{\pgfqpoint{0.625000in}{1.125705in}}%
\pgfpathlineto{\pgfqpoint{0.626011in}{1.128947in}}%
\pgfpathlineto{\pgfqpoint{0.625000in}{1.131590in}}%
\pgfusepath{stroke}%
\end{pgfscope}%
\begin{pgfscope}%
\pgfpathrectangle{\pgfqpoint{0.625000in}{0.550000in}}{\pgfqpoint{3.875000in}{3.850000in}} %
\pgfusepath{clip}%
\pgfsetbuttcap%
\pgfsetroundjoin%
\pgfsetlinewidth{0.250937pt}%
\definecolor{currentstroke}{rgb}{0.000000,0.000000,0.000000}%
\pgfsetstrokecolor{currentstroke}%
\pgfsetdash{}{0pt}%
\pgfpathmoveto{\pgfqpoint{0.625000in}{1.271393in}}%
\pgfpathlineto{\pgfqpoint{0.626480in}{1.273684in}}%
\pgfpathlineto{\pgfqpoint{0.625000in}{1.276862in}}%
\pgfusepath{stroke}%
\end{pgfscope}%
\begin{pgfscope}%
\pgfpathrectangle{\pgfqpoint{0.625000in}{0.550000in}}{\pgfqpoint{3.875000in}{3.850000in}} %
\pgfusepath{clip}%
\pgfsetbuttcap%
\pgfsetroundjoin%
\pgfsetlinewidth{0.250937pt}%
\definecolor{currentstroke}{rgb}{0.000000,0.000000,0.000000}%
\pgfsetstrokecolor{currentstroke}%
\pgfsetdash{}{0pt}%
\pgfpathmoveto{\pgfqpoint{0.625000in}{1.316501in}}%
\pgfpathlineto{\pgfqpoint{0.627733in}{1.321930in}}%
\pgfpathlineto{\pgfqpoint{0.625000in}{1.328154in}}%
\pgfusepath{stroke}%
\end{pgfscope}%
\begin{pgfscope}%
\pgfpathrectangle{\pgfqpoint{0.625000in}{0.550000in}}{\pgfqpoint{3.875000in}{3.850000in}} %
\pgfusepath{clip}%
\pgfsetbuttcap%
\pgfsetroundjoin%
\pgfsetlinewidth{0.250937pt}%
\definecolor{currentstroke}{rgb}{0.000000,0.000000,0.000000}%
\pgfsetstrokecolor{currentstroke}%
\pgfsetdash{}{0pt}%
\pgfpathmoveto{\pgfqpoint{0.625000in}{1.509928in}}%
\pgfpathlineto{\pgfqpoint{0.630008in}{1.514912in}}%
\pgfpathlineto{\pgfqpoint{0.625000in}{1.520142in}}%
\pgfusepath{stroke}%
\end{pgfscope}%
\begin{pgfscope}%
\pgfpathrectangle{\pgfqpoint{0.625000in}{0.550000in}}{\pgfqpoint{3.875000in}{3.850000in}} %
\pgfusepath{clip}%
\pgfsetbuttcap%
\pgfsetroundjoin%
\pgfsetlinewidth{0.250937pt}%
\definecolor{currentstroke}{rgb}{0.000000,0.000000,0.000000}%
\pgfsetstrokecolor{currentstroke}%
\pgfsetdash{}{0pt}%
\pgfpathmoveto{\pgfqpoint{0.625000in}{1.752665in}}%
\pgfpathlineto{\pgfqpoint{0.626660in}{1.756140in}}%
\pgfpathlineto{\pgfqpoint{0.625000in}{1.758679in}}%
\pgfusepath{stroke}%
\end{pgfscope}%
\begin{pgfscope}%
\pgfpathrectangle{\pgfqpoint{0.625000in}{0.550000in}}{\pgfqpoint{3.875000in}{3.850000in}} %
\pgfusepath{clip}%
\pgfsetbuttcap%
\pgfsetroundjoin%
\pgfsetlinewidth{0.250937pt}%
\definecolor{currentstroke}{rgb}{0.000000,0.000000,0.000000}%
\pgfsetstrokecolor{currentstroke}%
\pgfsetdash{}{0pt}%
\pgfpathmoveto{\pgfqpoint{0.625000in}{1.900240in}}%
\pgfpathlineto{\pgfqpoint{0.626013in}{1.900877in}}%
\pgfpathlineto{\pgfqpoint{0.625000in}{1.901978in}}%
\pgfusepath{stroke}%
\end{pgfscope}%
\begin{pgfscope}%
\pgfpathrectangle{\pgfqpoint{0.625000in}{0.550000in}}{\pgfqpoint{3.875000in}{3.850000in}} %
\pgfusepath{clip}%
\pgfsetbuttcap%
\pgfsetroundjoin%
\pgfsetlinewidth{0.250937pt}%
\definecolor{currentstroke}{rgb}{0.000000,0.000000,0.000000}%
\pgfsetstrokecolor{currentstroke}%
\pgfsetdash{}{0pt}%
\pgfpathmoveto{\pgfqpoint{0.625000in}{1.992499in}}%
\pgfpathlineto{\pgfqpoint{0.628287in}{1.997368in}}%
\pgfpathlineto{\pgfqpoint{0.625000in}{2.000166in}}%
\pgfusepath{stroke}%
\end{pgfscope}%
\begin{pgfscope}%
\pgfpathrectangle{\pgfqpoint{0.625000in}{0.550000in}}{\pgfqpoint{3.875000in}{3.850000in}} %
\pgfusepath{clip}%
\pgfsetbuttcap%
\pgfsetroundjoin%
\pgfsetlinewidth{0.250937pt}%
\definecolor{currentstroke}{rgb}{0.000000,0.000000,0.000000}%
\pgfsetstrokecolor{currentstroke}%
\pgfsetdash{}{0pt}%
\pgfpathmoveto{\pgfqpoint{0.625000in}{2.090360in}}%
\pgfpathlineto{\pgfqpoint{0.627733in}{2.093860in}}%
\pgfpathlineto{\pgfqpoint{0.625000in}{2.097372in}}%
\pgfusepath{stroke}%
\end{pgfscope}%
\begin{pgfscope}%
\pgfpathrectangle{\pgfqpoint{0.625000in}{0.550000in}}{\pgfqpoint{3.875000in}{3.850000in}} %
\pgfusepath{clip}%
\pgfsetbuttcap%
\pgfsetroundjoin%
\pgfsetlinewidth{0.250937pt}%
\definecolor{currentstroke}{rgb}{0.000000,0.000000,0.000000}%
\pgfsetstrokecolor{currentstroke}%
\pgfsetdash{}{0pt}%
\pgfpathmoveto{\pgfqpoint{0.625000in}{2.236506in}}%
\pgfpathlineto{\pgfqpoint{0.626337in}{2.238596in}}%
\pgfpathlineto{\pgfqpoint{0.625000in}{2.243494in}}%
\pgfusepath{stroke}%
\end{pgfscope}%
\begin{pgfscope}%
\pgfpathrectangle{\pgfqpoint{0.625000in}{0.550000in}}{\pgfqpoint{3.875000in}{3.850000in}} %
\pgfusepath{clip}%
\pgfsetbuttcap%
\pgfsetroundjoin%
\pgfsetlinewidth{0.250937pt}%
\definecolor{currentstroke}{rgb}{0.000000,0.000000,0.000000}%
\pgfsetstrokecolor{currentstroke}%
\pgfsetdash{}{0pt}%
\pgfpathmoveto{\pgfqpoint{0.625000in}{2.285435in}}%
\pgfpathlineto{\pgfqpoint{0.626010in}{2.286842in}}%
\pgfpathlineto{\pgfqpoint{0.625000in}{2.288415in}}%
\pgfusepath{stroke}%
\end{pgfscope}%
\begin{pgfscope}%
\pgfpathrectangle{\pgfqpoint{0.625000in}{0.550000in}}{\pgfqpoint{3.875000in}{3.850000in}} %
\pgfusepath{clip}%
\pgfsetbuttcap%
\pgfsetroundjoin%
\pgfsetlinewidth{0.250937pt}%
\definecolor{currentstroke}{rgb}{0.000000,0.000000,0.000000}%
\pgfsetstrokecolor{currentstroke}%
\pgfsetdash{}{0pt}%
\pgfpathmoveto{\pgfqpoint{0.625000in}{2.671234in}}%
\pgfpathlineto{\pgfqpoint{0.626010in}{2.672807in}}%
\pgfpathlineto{\pgfqpoint{0.625000in}{2.674214in}}%
\pgfusepath{stroke}%
\end{pgfscope}%
\begin{pgfscope}%
\pgfpathrectangle{\pgfqpoint{0.625000in}{0.550000in}}{\pgfqpoint{3.875000in}{3.850000in}} %
\pgfusepath{clip}%
\pgfsetbuttcap%
\pgfsetroundjoin%
\pgfsetlinewidth{0.250937pt}%
\definecolor{currentstroke}{rgb}{0.000000,0.000000,0.000000}%
\pgfsetstrokecolor{currentstroke}%
\pgfsetdash{}{0pt}%
\pgfpathmoveto{\pgfqpoint{0.625000in}{2.716156in}}%
\pgfpathlineto{\pgfqpoint{0.626337in}{2.721053in}}%
\pgfpathlineto{\pgfqpoint{0.625000in}{2.723143in}}%
\pgfusepath{stroke}%
\end{pgfscope}%
\begin{pgfscope}%
\pgfpathrectangle{\pgfqpoint{0.625000in}{0.550000in}}{\pgfqpoint{3.875000in}{3.850000in}} %
\pgfusepath{clip}%
\pgfsetbuttcap%
\pgfsetroundjoin%
\pgfsetlinewidth{0.250937pt}%
\definecolor{currentstroke}{rgb}{0.000000,0.000000,0.000000}%
\pgfsetstrokecolor{currentstroke}%
\pgfsetdash{}{0pt}%
\pgfpathmoveto{\pgfqpoint{0.625000in}{2.862277in}}%
\pgfpathlineto{\pgfqpoint{0.627733in}{2.865789in}}%
\pgfpathlineto{\pgfqpoint{0.625000in}{2.869289in}}%
\pgfusepath{stroke}%
\end{pgfscope}%
\begin{pgfscope}%
\pgfpathrectangle{\pgfqpoint{0.625000in}{0.550000in}}{\pgfqpoint{3.875000in}{3.850000in}} %
\pgfusepath{clip}%
\pgfsetbuttcap%
\pgfsetroundjoin%
\pgfsetlinewidth{0.250937pt}%
\definecolor{currentstroke}{rgb}{0.000000,0.000000,0.000000}%
\pgfsetstrokecolor{currentstroke}%
\pgfsetdash{}{0pt}%
\pgfpathmoveto{\pgfqpoint{0.625000in}{2.959483in}}%
\pgfpathlineto{\pgfqpoint{0.628287in}{2.962281in}}%
\pgfpathlineto{\pgfqpoint{0.625000in}{2.967150in}}%
\pgfusepath{stroke}%
\end{pgfscope}%
\begin{pgfscope}%
\pgfpathrectangle{\pgfqpoint{0.625000in}{0.550000in}}{\pgfqpoint{3.875000in}{3.850000in}} %
\pgfusepath{clip}%
\pgfsetbuttcap%
\pgfsetroundjoin%
\pgfsetlinewidth{0.250937pt}%
\definecolor{currentstroke}{rgb}{0.000000,0.000000,0.000000}%
\pgfsetstrokecolor{currentstroke}%
\pgfsetdash{}{0pt}%
\pgfpathmoveto{\pgfqpoint{0.625000in}{3.057671in}}%
\pgfpathlineto{\pgfqpoint{0.626013in}{3.058772in}}%
\pgfpathlineto{\pgfqpoint{0.625000in}{3.059409in}}%
\pgfusepath{stroke}%
\end{pgfscope}%
\begin{pgfscope}%
\pgfpathrectangle{\pgfqpoint{0.625000in}{0.550000in}}{\pgfqpoint{3.875000in}{3.850000in}} %
\pgfusepath{clip}%
\pgfsetbuttcap%
\pgfsetroundjoin%
\pgfsetlinewidth{0.250937pt}%
\definecolor{currentstroke}{rgb}{0.000000,0.000000,0.000000}%
\pgfsetstrokecolor{currentstroke}%
\pgfsetdash{}{0pt}%
\pgfpathmoveto{\pgfqpoint{0.625000in}{3.200971in}}%
\pgfpathlineto{\pgfqpoint{0.626660in}{3.203509in}}%
\pgfpathlineto{\pgfqpoint{0.625000in}{3.206984in}}%
\pgfusepath{stroke}%
\end{pgfscope}%
\begin{pgfscope}%
\pgfpathrectangle{\pgfqpoint{0.625000in}{0.550000in}}{\pgfqpoint{3.875000in}{3.850000in}} %
\pgfusepath{clip}%
\pgfsetbuttcap%
\pgfsetroundjoin%
\pgfsetlinewidth{0.250937pt}%
\definecolor{currentstroke}{rgb}{0.000000,0.000000,0.000000}%
\pgfsetstrokecolor{currentstroke}%
\pgfsetdash{}{0pt}%
\pgfpathmoveto{\pgfqpoint{0.625000in}{3.439507in}}%
\pgfpathlineto{\pgfqpoint{0.630008in}{3.444737in}}%
\pgfpathlineto{\pgfqpoint{0.625000in}{3.449721in}}%
\pgfusepath{stroke}%
\end{pgfscope}%
\begin{pgfscope}%
\pgfpathrectangle{\pgfqpoint{0.625000in}{0.550000in}}{\pgfqpoint{3.875000in}{3.850000in}} %
\pgfusepath{clip}%
\pgfsetbuttcap%
\pgfsetroundjoin%
\pgfsetlinewidth{0.250937pt}%
\definecolor{currentstroke}{rgb}{0.000000,0.000000,0.000000}%
\pgfsetstrokecolor{currentstroke}%
\pgfsetdash{}{0pt}%
\pgfpathmoveto{\pgfqpoint{0.625000in}{3.631495in}}%
\pgfpathlineto{\pgfqpoint{0.627733in}{3.637719in}}%
\pgfpathlineto{\pgfqpoint{0.625000in}{3.643148in}}%
\pgfusepath{stroke}%
\end{pgfscope}%
\begin{pgfscope}%
\pgfpathrectangle{\pgfqpoint{0.625000in}{0.550000in}}{\pgfqpoint{3.875000in}{3.850000in}} %
\pgfusepath{clip}%
\pgfsetbuttcap%
\pgfsetroundjoin%
\pgfsetlinewidth{0.250937pt}%
\definecolor{currentstroke}{rgb}{0.000000,0.000000,0.000000}%
\pgfsetstrokecolor{currentstroke}%
\pgfsetdash{}{0pt}%
\pgfpathmoveto{\pgfqpoint{0.625000in}{3.682788in}}%
\pgfpathlineto{\pgfqpoint{0.626480in}{3.685965in}}%
\pgfpathlineto{\pgfqpoint{0.625000in}{3.688256in}}%
\pgfusepath{stroke}%
\end{pgfscope}%
\begin{pgfscope}%
\pgfpathrectangle{\pgfqpoint{0.625000in}{0.550000in}}{\pgfqpoint{3.875000in}{3.850000in}} %
\pgfusepath{clip}%
\pgfsetbuttcap%
\pgfsetroundjoin%
\pgfsetlinewidth{0.250937pt}%
\definecolor{currentstroke}{rgb}{0.000000,0.000000,0.000000}%
\pgfsetstrokecolor{currentstroke}%
\pgfsetdash{}{0pt}%
\pgfpathmoveto{\pgfqpoint{0.625000in}{3.828059in}}%
\pgfpathlineto{\pgfqpoint{0.626011in}{3.830702in}}%
\pgfpathlineto{\pgfqpoint{0.625000in}{3.833944in}}%
\pgfusepath{stroke}%
\end{pgfscope}%
\begin{pgfscope}%
\pgfpathrectangle{\pgfqpoint{0.625000in}{0.550000in}}{\pgfqpoint{3.875000in}{3.850000in}} %
\pgfusepath{clip}%
\pgfsetbuttcap%
\pgfsetroundjoin%
\pgfsetlinewidth{0.250937pt}%
\definecolor{currentstroke}{rgb}{0.000000,0.000000,0.000000}%
\pgfsetstrokecolor{currentstroke}%
\pgfsetdash{}{0pt}%
\pgfpathmoveto{\pgfqpoint{0.625000in}{3.924097in}}%
\pgfpathlineto{\pgfqpoint{0.628287in}{3.927193in}}%
\pgfpathlineto{\pgfqpoint{0.625000in}{3.932208in}}%
\pgfusepath{stroke}%
\end{pgfscope}%
\begin{pgfscope}%
\pgfpathrectangle{\pgfqpoint{0.625000in}{0.550000in}}{\pgfqpoint{3.875000in}{3.850000in}} %
\pgfusepath{clip}%
\pgfsetbuttcap%
\pgfsetroundjoin%
\pgfsetlinewidth{0.250937pt}%
\definecolor{currentstroke}{rgb}{0.000000,0.000000,0.000000}%
\pgfsetstrokecolor{currentstroke}%
\pgfsetdash{}{0pt}%
\pgfpathmoveto{\pgfqpoint{0.625000in}{4.165729in}}%
\pgfpathlineto{\pgfqpoint{0.626774in}{4.168421in}}%
\pgfpathlineto{\pgfqpoint{0.625000in}{4.168545in}}%
\pgfusepath{stroke}%
\end{pgfscope}%
\begin{pgfscope}%
\pgfpathrectangle{\pgfqpoint{0.625000in}{0.550000in}}{\pgfqpoint{3.875000in}{3.850000in}} %
\pgfusepath{clip}%
\pgfsetbuttcap%
\pgfsetroundjoin%
\pgfsetlinewidth{0.250937pt}%
\definecolor{currentstroke}{rgb}{0.000000,0.000000,0.000000}%
\pgfsetstrokecolor{currentstroke}%
\pgfsetdash{}{0pt}%
\pgfpathmoveto{\pgfqpoint{0.625000in}{4.211293in}}%
\pgfpathlineto{\pgfqpoint{0.626014in}{4.216667in}}%
\pgfpathlineto{\pgfqpoint{0.625000in}{4.221723in}}%
\pgfusepath{stroke}%
\end{pgfscope}%
\begin{pgfscope}%
\pgfpathrectangle{\pgfqpoint{0.625000in}{0.550000in}}{\pgfqpoint{3.875000in}{3.850000in}} %
\pgfusepath{clip}%
\pgfsetbuttcap%
\pgfsetroundjoin%
\pgfsetlinewidth{0.250937pt}%
\definecolor{currentstroke}{rgb}{0.000000,0.000000,0.000000}%
\pgfsetstrokecolor{currentstroke}%
\pgfsetdash{}{0pt}%
\pgfpathmoveto{\pgfqpoint{0.630487in}{0.550000in}}%
\pgfpathlineto{\pgfqpoint{0.625000in}{0.555592in}}%
\pgfusepath{stroke}%
\end{pgfscope}%
\begin{pgfscope}%
\pgfpathrectangle{\pgfqpoint{0.625000in}{0.550000in}}{\pgfqpoint{3.875000in}{3.850000in}} %
\pgfusepath{clip}%
\pgfsetbuttcap%
\pgfsetroundjoin%
\pgfsetlinewidth{0.250937pt}%
\definecolor{currentstroke}{rgb}{0.000000,0.000000,0.000000}%
\pgfsetstrokecolor{currentstroke}%
\pgfsetdash{}{0pt}%
\pgfpathmoveto{\pgfqpoint{0.625000in}{0.791192in}}%
\pgfpathlineto{\pgfqpoint{0.625564in}{0.791228in}}%
\pgfpathlineto{\pgfqpoint{0.625000in}{0.792084in}}%
\pgfusepath{stroke}%
\end{pgfscope}%
\begin{pgfscope}%
\pgfpathrectangle{\pgfqpoint{0.625000in}{0.550000in}}{\pgfqpoint{3.875000in}{3.850000in}} %
\pgfusepath{clip}%
\pgfsetbuttcap%
\pgfsetroundjoin%
\pgfsetlinewidth{0.250937pt}%
\definecolor{currentstroke}{rgb}{0.000000,0.000000,0.000000}%
\pgfsetstrokecolor{currentstroke}%
\pgfsetdash{}{0pt}%
\pgfpathmoveto{\pgfqpoint{0.625000in}{1.029335in}}%
\pgfpathlineto{\pgfqpoint{0.627046in}{1.032456in}}%
\pgfpathlineto{\pgfqpoint{0.625000in}{1.034383in}}%
\pgfusepath{stroke}%
\end{pgfscope}%
\begin{pgfscope}%
\pgfpathrectangle{\pgfqpoint{0.625000in}{0.550000in}}{\pgfqpoint{3.875000in}{3.850000in}} %
\pgfusepath{clip}%
\pgfsetbuttcap%
\pgfsetroundjoin%
\pgfsetlinewidth{0.250937pt}%
\definecolor{currentstroke}{rgb}{0.000000,0.000000,0.000000}%
\pgfsetstrokecolor{currentstroke}%
\pgfsetdash{}{0pt}%
\pgfpathmoveto{\pgfqpoint{0.625000in}{1.273335in}}%
\pgfpathlineto{\pgfqpoint{0.625226in}{1.273684in}}%
\pgfpathlineto{\pgfqpoint{0.625000in}{1.274169in}}%
\pgfusepath{stroke}%
\end{pgfscope}%
\begin{pgfscope}%
\pgfpathrectangle{\pgfqpoint{0.625000in}{0.550000in}}{\pgfqpoint{3.875000in}{3.850000in}} %
\pgfusepath{clip}%
\pgfsetbuttcap%
\pgfsetroundjoin%
\pgfsetlinewidth{0.250937pt}%
\definecolor{currentstroke}{rgb}{0.000000,0.000000,0.000000}%
\pgfsetstrokecolor{currentstroke}%
\pgfsetdash{}{0pt}%
\pgfpathmoveto{\pgfqpoint{0.625000in}{1.318967in}}%
\pgfpathlineto{\pgfqpoint{0.626492in}{1.321930in}}%
\pgfpathlineto{\pgfqpoint{0.625000in}{1.325327in}}%
\pgfusepath{stroke}%
\end{pgfscope}%
\begin{pgfscope}%
\pgfpathrectangle{\pgfqpoint{0.625000in}{0.550000in}}{\pgfqpoint{3.875000in}{3.850000in}} %
\pgfusepath{clip}%
\pgfsetbuttcap%
\pgfsetroundjoin%
\pgfsetlinewidth{0.250937pt}%
\definecolor{currentstroke}{rgb}{0.000000,0.000000,0.000000}%
\pgfsetstrokecolor{currentstroke}%
\pgfsetdash{}{0pt}%
\pgfpathmoveto{\pgfqpoint{0.625000in}{1.511164in}}%
\pgfpathlineto{\pgfqpoint{0.628767in}{1.514912in}}%
\pgfpathlineto{\pgfqpoint{0.625000in}{1.518846in}}%
\pgfusepath{stroke}%
\end{pgfscope}%
\begin{pgfscope}%
\pgfpathrectangle{\pgfqpoint{0.625000in}{0.550000in}}{\pgfqpoint{3.875000in}{3.850000in}} %
\pgfusepath{clip}%
\pgfsetbuttcap%
\pgfsetroundjoin%
\pgfsetlinewidth{0.250937pt}%
\definecolor{currentstroke}{rgb}{0.000000,0.000000,0.000000}%
\pgfsetstrokecolor{currentstroke}%
\pgfsetdash{}{0pt}%
\pgfpathmoveto{\pgfqpoint{0.625000in}{1.755234in}}%
\pgfpathlineto{\pgfqpoint{0.625433in}{1.756140in}}%
\pgfpathlineto{\pgfqpoint{0.625000in}{1.756802in}}%
\pgfusepath{stroke}%
\end{pgfscope}%
\begin{pgfscope}%
\pgfpathrectangle{\pgfqpoint{0.625000in}{0.550000in}}{\pgfqpoint{3.875000in}{3.850000in}} %
\pgfusepath{clip}%
\pgfsetbuttcap%
\pgfsetroundjoin%
\pgfsetlinewidth{0.250937pt}%
\definecolor{currentstroke}{rgb}{0.000000,0.000000,0.000000}%
\pgfsetstrokecolor{currentstroke}%
\pgfsetdash{}{0pt}%
\pgfpathmoveto{\pgfqpoint{0.625000in}{1.994338in}}%
\pgfpathlineto{\pgfqpoint{0.627046in}{1.997368in}}%
\pgfpathlineto{\pgfqpoint{0.625000in}{1.999110in}}%
\pgfusepath{stroke}%
\end{pgfscope}%
\begin{pgfscope}%
\pgfpathrectangle{\pgfqpoint{0.625000in}{0.550000in}}{\pgfqpoint{3.875000in}{3.850000in}} %
\pgfusepath{clip}%
\pgfsetbuttcap%
\pgfsetroundjoin%
\pgfsetlinewidth{0.250937pt}%
\definecolor{currentstroke}{rgb}{0.000000,0.000000,0.000000}%
\pgfsetstrokecolor{currentstroke}%
\pgfsetdash{}{0pt}%
\pgfpathmoveto{\pgfqpoint{0.625000in}{2.091950in}}%
\pgfpathlineto{\pgfqpoint{0.626492in}{2.093860in}}%
\pgfpathlineto{\pgfqpoint{0.625000in}{2.095777in}}%
\pgfusepath{stroke}%
\end{pgfscope}%
\begin{pgfscope}%
\pgfpathrectangle{\pgfqpoint{0.625000in}{0.550000in}}{\pgfqpoint{3.875000in}{3.850000in}} %
\pgfusepath{clip}%
\pgfsetbuttcap%
\pgfsetroundjoin%
\pgfsetlinewidth{0.250937pt}%
\definecolor{currentstroke}{rgb}{0.000000,0.000000,0.000000}%
\pgfsetstrokecolor{currentstroke}%
\pgfsetdash{}{0pt}%
\pgfpathmoveto{\pgfqpoint{0.625000in}{2.238500in}}%
\pgfpathlineto{\pgfqpoint{0.625061in}{2.238596in}}%
\pgfpathlineto{\pgfqpoint{0.625000in}{2.238821in}}%
\pgfusepath{stroke}%
\end{pgfscope}%
\begin{pgfscope}%
\pgfpathrectangle{\pgfqpoint{0.625000in}{0.550000in}}{\pgfqpoint{3.875000in}{3.850000in}} %
\pgfusepath{clip}%
\pgfsetbuttcap%
\pgfsetroundjoin%
\pgfsetlinewidth{0.250937pt}%
\definecolor{currentstroke}{rgb}{0.000000,0.000000,0.000000}%
\pgfsetstrokecolor{currentstroke}%
\pgfsetdash{}{0pt}%
\pgfpathmoveto{\pgfqpoint{0.625000in}{2.720828in}}%
\pgfpathlineto{\pgfqpoint{0.625061in}{2.721053in}}%
\pgfpathlineto{\pgfqpoint{0.625000in}{2.721149in}}%
\pgfusepath{stroke}%
\end{pgfscope}%
\begin{pgfscope}%
\pgfpathrectangle{\pgfqpoint{0.625000in}{0.550000in}}{\pgfqpoint{3.875000in}{3.850000in}} %
\pgfusepath{clip}%
\pgfsetbuttcap%
\pgfsetroundjoin%
\pgfsetlinewidth{0.250937pt}%
\definecolor{currentstroke}{rgb}{0.000000,0.000000,0.000000}%
\pgfsetstrokecolor{currentstroke}%
\pgfsetdash{}{0pt}%
\pgfpathmoveto{\pgfqpoint{0.625000in}{2.863872in}}%
\pgfpathlineto{\pgfqpoint{0.626492in}{2.865789in}}%
\pgfpathlineto{\pgfqpoint{0.625000in}{2.867700in}}%
\pgfusepath{stroke}%
\end{pgfscope}%
\begin{pgfscope}%
\pgfpathrectangle{\pgfqpoint{0.625000in}{0.550000in}}{\pgfqpoint{3.875000in}{3.850000in}} %
\pgfusepath{clip}%
\pgfsetbuttcap%
\pgfsetroundjoin%
\pgfsetlinewidth{0.250937pt}%
\definecolor{currentstroke}{rgb}{0.000000,0.000000,0.000000}%
\pgfsetstrokecolor{currentstroke}%
\pgfsetdash{}{0pt}%
\pgfpathmoveto{\pgfqpoint{0.625000in}{2.960539in}}%
\pgfpathlineto{\pgfqpoint{0.627046in}{2.962281in}}%
\pgfpathlineto{\pgfqpoint{0.625000in}{2.965311in}}%
\pgfusepath{stroke}%
\end{pgfscope}%
\begin{pgfscope}%
\pgfpathrectangle{\pgfqpoint{0.625000in}{0.550000in}}{\pgfqpoint{3.875000in}{3.850000in}} %
\pgfusepath{clip}%
\pgfsetbuttcap%
\pgfsetroundjoin%
\pgfsetlinewidth{0.250937pt}%
\definecolor{currentstroke}{rgb}{0.000000,0.000000,0.000000}%
\pgfsetstrokecolor{currentstroke}%
\pgfsetdash{}{0pt}%
\pgfpathmoveto{\pgfqpoint{0.625000in}{3.202847in}}%
\pgfpathlineto{\pgfqpoint{0.625433in}{3.203509in}}%
\pgfpathlineto{\pgfqpoint{0.625000in}{3.204415in}}%
\pgfusepath{stroke}%
\end{pgfscope}%
\begin{pgfscope}%
\pgfpathrectangle{\pgfqpoint{0.625000in}{0.550000in}}{\pgfqpoint{3.875000in}{3.850000in}} %
\pgfusepath{clip}%
\pgfsetbuttcap%
\pgfsetroundjoin%
\pgfsetlinewidth{0.250937pt}%
\definecolor{currentstroke}{rgb}{0.000000,0.000000,0.000000}%
\pgfsetstrokecolor{currentstroke}%
\pgfsetdash{}{0pt}%
\pgfpathmoveto{\pgfqpoint{0.625000in}{3.440803in}}%
\pgfpathlineto{\pgfqpoint{0.628767in}{3.444737in}}%
\pgfpathlineto{\pgfqpoint{0.625000in}{3.448486in}}%
\pgfusepath{stroke}%
\end{pgfscope}%
\begin{pgfscope}%
\pgfpathrectangle{\pgfqpoint{0.625000in}{0.550000in}}{\pgfqpoint{3.875000in}{3.850000in}} %
\pgfusepath{clip}%
\pgfsetbuttcap%
\pgfsetroundjoin%
\pgfsetlinewidth{0.250937pt}%
\definecolor{currentstroke}{rgb}{0.000000,0.000000,0.000000}%
\pgfsetstrokecolor{currentstroke}%
\pgfsetdash{}{0pt}%
\pgfpathmoveto{\pgfqpoint{0.625000in}{3.634322in}}%
\pgfpathlineto{\pgfqpoint{0.626492in}{3.637719in}}%
\pgfpathlineto{\pgfqpoint{0.625000in}{3.640682in}}%
\pgfusepath{stroke}%
\end{pgfscope}%
\begin{pgfscope}%
\pgfpathrectangle{\pgfqpoint{0.625000in}{0.550000in}}{\pgfqpoint{3.875000in}{3.850000in}} %
\pgfusepath{clip}%
\pgfsetbuttcap%
\pgfsetroundjoin%
\pgfsetlinewidth{0.250937pt}%
\definecolor{currentstroke}{rgb}{0.000000,0.000000,0.000000}%
\pgfsetstrokecolor{currentstroke}%
\pgfsetdash{}{0pt}%
\pgfpathmoveto{\pgfqpoint{0.625000in}{3.685480in}}%
\pgfpathlineto{\pgfqpoint{0.625226in}{3.685965in}}%
\pgfpathlineto{\pgfqpoint{0.625000in}{3.686314in}}%
\pgfusepath{stroke}%
\end{pgfscope}%
\begin{pgfscope}%
\pgfpathrectangle{\pgfqpoint{0.625000in}{0.550000in}}{\pgfqpoint{3.875000in}{3.850000in}} %
\pgfusepath{clip}%
\pgfsetbuttcap%
\pgfsetroundjoin%
\pgfsetlinewidth{0.250937pt}%
\definecolor{currentstroke}{rgb}{0.000000,0.000000,0.000000}%
\pgfsetstrokecolor{currentstroke}%
\pgfsetdash{}{0pt}%
\pgfpathmoveto{\pgfqpoint{0.625000in}{3.925266in}}%
\pgfpathlineto{\pgfqpoint{0.627046in}{3.927193in}}%
\pgfpathlineto{\pgfqpoint{0.625000in}{3.930314in}}%
\pgfusepath{stroke}%
\end{pgfscope}%
\begin{pgfscope}%
\pgfpathrectangle{\pgfqpoint{0.625000in}{0.550000in}}{\pgfqpoint{3.875000in}{3.850000in}} %
\pgfusepath{clip}%
\pgfsetbuttcap%
\pgfsetroundjoin%
\pgfsetlinewidth{0.250937pt}%
\definecolor{currentstroke}{rgb}{0.000000,0.000000,0.000000}%
\pgfsetstrokecolor{currentstroke}%
\pgfsetdash{}{0pt}%
\pgfpathmoveto{\pgfqpoint{0.625000in}{4.167565in}}%
\pgfpathlineto{\pgfqpoint{0.625564in}{4.168421in}}%
\pgfpathlineto{\pgfqpoint{0.625000in}{4.168460in}}%
\pgfusepath{stroke}%
\end{pgfscope}%
\begin{pgfscope}%
\pgfpathrectangle{\pgfqpoint{0.625000in}{0.550000in}}{\pgfqpoint{3.875000in}{3.850000in}} %
\pgfusepath{clip}%
\pgfsetbuttcap%
\pgfsetroundjoin%
\pgfsetlinewidth{0.250937pt}%
\definecolor{currentstroke}{rgb}{0.000000,0.000000,0.000000}%
\pgfsetstrokecolor{currentstroke}%
\pgfsetdash{}{0pt}%
\pgfpathmoveto{\pgfqpoint{0.629246in}{0.550000in}}%
\pgfpathlineto{\pgfqpoint{0.625000in}{0.554327in}}%
\pgfusepath{stroke}%
\end{pgfscope}%
\begin{pgfscope}%
\pgfpathrectangle{\pgfqpoint{0.625000in}{0.550000in}}{\pgfqpoint{3.875000in}{3.850000in}} %
\pgfusepath{clip}%
\pgfsetbuttcap%
\pgfsetroundjoin%
\pgfsetlinewidth{0.250937pt}%
\definecolor{currentstroke}{rgb}{0.000000,0.000000,0.000000}%
\pgfsetstrokecolor{currentstroke}%
\pgfsetdash{}{0pt}%
\pgfpathmoveto{\pgfqpoint{0.625000in}{1.031229in}}%
\pgfpathlineto{\pgfqpoint{0.625804in}{1.032456in}}%
\pgfpathlineto{\pgfqpoint{0.625000in}{1.033214in}}%
\pgfusepath{stroke}%
\end{pgfscope}%
\begin{pgfscope}%
\pgfpathrectangle{\pgfqpoint{0.625000in}{0.550000in}}{\pgfqpoint{3.875000in}{3.850000in}} %
\pgfusepath{clip}%
\pgfsetbuttcap%
\pgfsetroundjoin%
\pgfsetlinewidth{0.250937pt}%
\definecolor{currentstroke}{rgb}{0.000000,0.000000,0.000000}%
\pgfsetstrokecolor{currentstroke}%
\pgfsetdash{}{0pt}%
\pgfpathmoveto{\pgfqpoint{0.625000in}{1.321432in}}%
\pgfpathlineto{\pgfqpoint{0.625250in}{1.321930in}}%
\pgfpathlineto{\pgfqpoint{0.625000in}{1.322500in}}%
\pgfusepath{stroke}%
\end{pgfscope}%
\begin{pgfscope}%
\pgfpathrectangle{\pgfqpoint{0.625000in}{0.550000in}}{\pgfqpoint{3.875000in}{3.850000in}} %
\pgfusepath{clip}%
\pgfsetbuttcap%
\pgfsetroundjoin%
\pgfsetlinewidth{0.250937pt}%
\definecolor{currentstroke}{rgb}{0.000000,0.000000,0.000000}%
\pgfsetstrokecolor{currentstroke}%
\pgfsetdash{}{0pt}%
\pgfpathmoveto{\pgfqpoint{0.625000in}{1.512399in}}%
\pgfpathlineto{\pgfqpoint{0.627525in}{1.514912in}}%
\pgfpathlineto{\pgfqpoint{0.625000in}{1.517550in}}%
\pgfusepath{stroke}%
\end{pgfscope}%
\begin{pgfscope}%
\pgfpathrectangle{\pgfqpoint{0.625000in}{0.550000in}}{\pgfqpoint{3.875000in}{3.850000in}} %
\pgfusepath{clip}%
\pgfsetbuttcap%
\pgfsetroundjoin%
\pgfsetlinewidth{0.250937pt}%
\definecolor{currentstroke}{rgb}{0.000000,0.000000,0.000000}%
\pgfsetstrokecolor{currentstroke}%
\pgfsetdash{}{0pt}%
\pgfpathmoveto{\pgfqpoint{0.625000in}{1.996177in}}%
\pgfpathlineto{\pgfqpoint{0.625804in}{1.997368in}}%
\pgfpathlineto{\pgfqpoint{0.625000in}{1.998053in}}%
\pgfusepath{stroke}%
\end{pgfscope}%
\begin{pgfscope}%
\pgfpathrectangle{\pgfqpoint{0.625000in}{0.550000in}}{\pgfqpoint{3.875000in}{3.850000in}} %
\pgfusepath{clip}%
\pgfsetbuttcap%
\pgfsetroundjoin%
\pgfsetlinewidth{0.250937pt}%
\definecolor{currentstroke}{rgb}{0.000000,0.000000,0.000000}%
\pgfsetstrokecolor{currentstroke}%
\pgfsetdash{}{0pt}%
\pgfpathmoveto{\pgfqpoint{0.625000in}{2.093539in}}%
\pgfpathlineto{\pgfqpoint{0.625250in}{2.093860in}}%
\pgfpathlineto{\pgfqpoint{0.625000in}{2.094182in}}%
\pgfusepath{stroke}%
\end{pgfscope}%
\begin{pgfscope}%
\pgfpathrectangle{\pgfqpoint{0.625000in}{0.550000in}}{\pgfqpoint{3.875000in}{3.850000in}} %
\pgfusepath{clip}%
\pgfsetbuttcap%
\pgfsetroundjoin%
\pgfsetlinewidth{0.250937pt}%
\definecolor{currentstroke}{rgb}{0.000000,0.000000,0.000000}%
\pgfsetstrokecolor{currentstroke}%
\pgfsetdash{}{0pt}%
\pgfpathmoveto{\pgfqpoint{0.625000in}{2.865468in}}%
\pgfpathlineto{\pgfqpoint{0.625250in}{2.865789in}}%
\pgfpathlineto{\pgfqpoint{0.625000in}{2.866110in}}%
\pgfusepath{stroke}%
\end{pgfscope}%
\begin{pgfscope}%
\pgfpathrectangle{\pgfqpoint{0.625000in}{0.550000in}}{\pgfqpoint{3.875000in}{3.850000in}} %
\pgfusepath{clip}%
\pgfsetbuttcap%
\pgfsetroundjoin%
\pgfsetlinewidth{0.250937pt}%
\definecolor{currentstroke}{rgb}{0.000000,0.000000,0.000000}%
\pgfsetstrokecolor{currentstroke}%
\pgfsetdash{}{0pt}%
\pgfpathmoveto{\pgfqpoint{0.625000in}{2.961596in}}%
\pgfpathlineto{\pgfqpoint{0.625804in}{2.962281in}}%
\pgfpathlineto{\pgfqpoint{0.625000in}{2.963472in}}%
\pgfusepath{stroke}%
\end{pgfscope}%
\begin{pgfscope}%
\pgfpathrectangle{\pgfqpoint{0.625000in}{0.550000in}}{\pgfqpoint{3.875000in}{3.850000in}} %
\pgfusepath{clip}%
\pgfsetbuttcap%
\pgfsetroundjoin%
\pgfsetlinewidth{0.250937pt}%
\definecolor{currentstroke}{rgb}{0.000000,0.000000,0.000000}%
\pgfsetstrokecolor{currentstroke}%
\pgfsetdash{}{0pt}%
\pgfpathmoveto{\pgfqpoint{0.625000in}{3.442099in}}%
\pgfpathlineto{\pgfqpoint{0.627525in}{3.444737in}}%
\pgfpathlineto{\pgfqpoint{0.625000in}{3.447250in}}%
\pgfusepath{stroke}%
\end{pgfscope}%
\begin{pgfscope}%
\pgfpathrectangle{\pgfqpoint{0.625000in}{0.550000in}}{\pgfqpoint{3.875000in}{3.850000in}} %
\pgfusepath{clip}%
\pgfsetbuttcap%
\pgfsetroundjoin%
\pgfsetlinewidth{0.250937pt}%
\definecolor{currentstroke}{rgb}{0.000000,0.000000,0.000000}%
\pgfsetstrokecolor{currentstroke}%
\pgfsetdash{}{0pt}%
\pgfpathmoveto{\pgfqpoint{0.625000in}{3.637149in}}%
\pgfpathlineto{\pgfqpoint{0.625250in}{3.637719in}}%
\pgfpathlineto{\pgfqpoint{0.625000in}{3.638217in}}%
\pgfusepath{stroke}%
\end{pgfscope}%
\begin{pgfscope}%
\pgfpathrectangle{\pgfqpoint{0.625000in}{0.550000in}}{\pgfqpoint{3.875000in}{3.850000in}} %
\pgfusepath{clip}%
\pgfsetbuttcap%
\pgfsetroundjoin%
\pgfsetlinewidth{0.250937pt}%
\definecolor{currentstroke}{rgb}{0.000000,0.000000,0.000000}%
\pgfsetstrokecolor{currentstroke}%
\pgfsetdash{}{0pt}%
\pgfpathmoveto{\pgfqpoint{0.625000in}{3.926435in}}%
\pgfpathlineto{\pgfqpoint{0.625804in}{3.927193in}}%
\pgfpathlineto{\pgfqpoint{0.625000in}{3.928420in}}%
\pgfusepath{stroke}%
\end{pgfscope}%
\begin{pgfscope}%
\pgfpathrectangle{\pgfqpoint{0.625000in}{0.550000in}}{\pgfqpoint{3.875000in}{3.850000in}} %
\pgfusepath{clip}%
\pgfsetbuttcap%
\pgfsetroundjoin%
\pgfsetlinewidth{0.250937pt}%
\definecolor{currentstroke}{rgb}{0.000000,0.000000,0.000000}%
\pgfsetstrokecolor{currentstroke}%
\pgfsetdash{}{0pt}%
\pgfpathmoveto{\pgfqpoint{0.628005in}{0.550000in}}%
\pgfpathlineto{\pgfqpoint{0.625000in}{0.553062in}}%
\pgfusepath{stroke}%
\end{pgfscope}%
\begin{pgfscope}%
\pgfpathrectangle{\pgfqpoint{0.625000in}{0.550000in}}{\pgfqpoint{3.875000in}{3.850000in}} %
\pgfusepath{clip}%
\pgfsetbuttcap%
\pgfsetroundjoin%
\pgfsetlinewidth{0.250937pt}%
\definecolor{currentstroke}{rgb}{0.000000,0.000000,0.000000}%
\pgfsetstrokecolor{currentstroke}%
\pgfsetdash{}{0pt}%
\pgfpathmoveto{\pgfqpoint{0.625000in}{1.513634in}}%
\pgfpathlineto{\pgfqpoint{0.626284in}{1.514912in}}%
\pgfpathlineto{\pgfqpoint{0.625000in}{1.516253in}}%
\pgfusepath{stroke}%
\end{pgfscope}%
\begin{pgfscope}%
\pgfpathrectangle{\pgfqpoint{0.625000in}{0.550000in}}{\pgfqpoint{3.875000in}{3.850000in}} %
\pgfusepath{clip}%
\pgfsetbuttcap%
\pgfsetroundjoin%
\pgfsetlinewidth{0.250937pt}%
\definecolor{currentstroke}{rgb}{0.000000,0.000000,0.000000}%
\pgfsetstrokecolor{currentstroke}%
\pgfsetdash{}{0pt}%
\pgfpathmoveto{\pgfqpoint{0.625000in}{3.443396in}}%
\pgfpathlineto{\pgfqpoint{0.626284in}{3.444737in}}%
\pgfpathlineto{\pgfqpoint{0.625000in}{3.446015in}}%
\pgfusepath{stroke}%
\end{pgfscope}%
\begin{pgfscope}%
\pgfpathrectangle{\pgfqpoint{0.625000in}{0.550000in}}{\pgfqpoint{3.875000in}{3.850000in}} %
\pgfusepath{clip}%
\pgfsetbuttcap%
\pgfsetroundjoin%
\pgfsetlinewidth{0.250937pt}%
\definecolor{currentstroke}{rgb}{0.000000,0.000000,0.000000}%
\pgfsetstrokecolor{currentstroke}%
\pgfsetdash{}{0pt}%
\pgfpathmoveto{\pgfqpoint{0.626763in}{0.550000in}}%
\pgfpathlineto{\pgfqpoint{0.625000in}{0.551797in}}%
\pgfusepath{stroke}%
\end{pgfscope}%
\begin{pgfscope}%
\pgfpathrectangle{\pgfqpoint{0.625000in}{0.550000in}}{\pgfqpoint{3.875000in}{3.850000in}} %
\pgfusepath{clip}%
\pgfsetbuttcap%
\pgfsetroundjoin%
\pgfsetlinewidth{0.250937pt}%
\definecolor{currentstroke}{rgb}{0.000000,0.000000,0.000000}%
\pgfsetstrokecolor{currentstroke}%
\pgfsetdash{}{0pt}%
\pgfpathmoveto{\pgfqpoint{0.625000in}{1.514870in}}%
\pgfpathlineto{\pgfqpoint{0.625043in}{1.514912in}}%
\pgfpathlineto{\pgfqpoint{0.625000in}{1.514957in}}%
\pgfusepath{stroke}%
\end{pgfscope}%
\begin{pgfscope}%
\pgfpathrectangle{\pgfqpoint{0.625000in}{0.550000in}}{\pgfqpoint{3.875000in}{3.850000in}} %
\pgfusepath{clip}%
\pgfsetbuttcap%
\pgfsetroundjoin%
\pgfsetlinewidth{0.250937pt}%
\definecolor{currentstroke}{rgb}{0.000000,0.000000,0.000000}%
\pgfsetstrokecolor{currentstroke}%
\pgfsetdash{}{0pt}%
\pgfpathmoveto{\pgfqpoint{0.625000in}{3.444692in}}%
\pgfpathlineto{\pgfqpoint{0.625043in}{3.444737in}}%
\pgfpathlineto{\pgfqpoint{0.625000in}{3.444779in}}%
\pgfusepath{stroke}%
\end{pgfscope}%
\begin{pgfscope}%
\pgfpathrectangle{\pgfqpoint{0.625000in}{0.550000in}}{\pgfqpoint{3.875000in}{3.850000in}} %
\pgfusepath{clip}%
\pgfsetbuttcap%
\pgfsetroundjoin%
\pgfsetlinewidth{0.250937pt}%
\definecolor{currentstroke}{rgb}{0.000000,0.000000,0.000000}%
\pgfsetstrokecolor{currentstroke}%
\pgfsetdash{}{0pt}%
\pgfpathmoveto{\pgfqpoint{0.625522in}{0.550000in}}%
\pgfpathlineto{\pgfqpoint{0.625000in}{0.550532in}}%
\pgfusepath{stroke}%
\end{pgfscope}%
\begin{pgfscope}%
\pgfpathrectangle{\pgfqpoint{0.625000in}{0.550000in}}{\pgfqpoint{3.875000in}{3.850000in}} %
\pgfusepath{clip}%
\pgfsetbuttcap%
\pgfsetroundjoin%
\pgfsetlinewidth{0.501875pt}%
\definecolor{currentstroke}{rgb}{0.000000,0.000000,0.000000}%
\pgfsetstrokecolor{currentstroke}%
\pgfsetdash{}{0pt}%
\pgfpathmoveto{\pgfqpoint{0.625000in}{0.552570in}}%
\pgfpathlineto{\pgfqpoint{0.644424in}{0.558173in}}%
\pgfpathlineto{\pgfqpoint{0.654135in}{0.562220in}}%
\pgfpathlineto{\pgfqpoint{0.929864in}{0.675439in}}%
\pgfpathlineto{\pgfqpoint{1.081454in}{0.737844in}}%
\pgfpathlineto{\pgfqpoint{1.363095in}{0.854399in}}%
\pgfpathlineto{\pgfqpoint{1.440789in}{0.888759in}}%
\pgfpathlineto{\pgfqpoint{1.508772in}{0.921029in}}%
\pgfpathlineto{\pgfqpoint{1.567043in}{0.951011in}}%
\pgfpathlineto{\pgfqpoint{1.615602in}{0.978122in}}%
\pgfpathlineto{\pgfqpoint{1.664160in}{1.007631in}}%
\pgfpathlineto{\pgfqpoint{1.703008in}{1.033315in}}%
\pgfpathlineto{\pgfqpoint{1.742110in}{1.061404in}}%
\pgfpathlineto{\pgfqpoint{1.780702in}{1.091755in}}%
\pgfpathlineto{\pgfqpoint{1.819549in}{1.125492in}}%
\pgfpathlineto{\pgfqpoint{1.848684in}{1.153338in}}%
\pgfpathlineto{\pgfqpoint{1.877820in}{1.183848in}}%
\pgfpathlineto{\pgfqpoint{1.897403in}{1.206140in}}%
\pgfpathlineto{\pgfqpoint{1.926378in}{1.242428in}}%
\pgfpathlineto{\pgfqpoint{1.948707in}{1.273684in}}%
\pgfpathlineto{\pgfqpoint{1.973097in}{1.312281in}}%
\pgfpathlineto{\pgfqpoint{1.994361in}{1.351139in}}%
\pgfpathlineto{\pgfqpoint{2.012332in}{1.389474in}}%
\pgfpathlineto{\pgfqpoint{2.027553in}{1.428070in}}%
\pgfpathlineto{\pgfqpoint{2.040016in}{1.466667in}}%
\pgfpathlineto{\pgfqpoint{2.049835in}{1.505263in}}%
\pgfpathlineto{\pgfqpoint{2.057092in}{1.543860in}}%
\pgfpathlineto{\pgfqpoint{2.061823in}{1.582456in}}%
\pgfpathlineto{\pgfqpoint{2.064101in}{1.621053in}}%
\pgfpathlineto{\pgfqpoint{2.063912in}{1.659649in}}%
\pgfpathlineto{\pgfqpoint{2.061259in}{1.698246in}}%
\pgfpathlineto{\pgfqpoint{2.056133in}{1.736842in}}%
\pgfpathlineto{\pgfqpoint{2.048465in}{1.775439in}}%
\pgfpathlineto{\pgfqpoint{2.038197in}{1.814035in}}%
\pgfpathlineto{\pgfqpoint{2.025225in}{1.852632in}}%
\pgfpathlineto{\pgfqpoint{2.009458in}{1.891228in}}%
\pgfpathlineto{\pgfqpoint{1.990710in}{1.929825in}}%
\pgfpathlineto{\pgfqpoint{1.968797in}{1.968421in}}%
\pgfpathlineto{\pgfqpoint{1.943465in}{2.007018in}}%
\pgfpathlineto{\pgfqpoint{1.916667in}{2.042748in}}%
\pgfpathlineto{\pgfqpoint{1.897243in}{2.066188in}}%
\pgfpathlineto{\pgfqpoint{1.868108in}{2.098154in}}%
\pgfpathlineto{\pgfqpoint{1.838972in}{2.127018in}}%
\pgfpathlineto{\pgfqpoint{1.809837in}{2.153317in}}%
\pgfpathlineto{\pgfqpoint{1.770990in}{2.185018in}}%
\pgfpathlineto{\pgfqpoint{1.737727in}{2.209649in}}%
\pgfpathlineto{\pgfqpoint{1.693296in}{2.239339in}}%
\pgfpathlineto{\pgfqpoint{1.644737in}{2.268333in}}%
\pgfpathlineto{\pgfqpoint{1.596178in}{2.294213in}}%
\pgfpathlineto{\pgfqpoint{1.557331in}{2.312969in}}%
\pgfpathlineto{\pgfqpoint{1.518484in}{2.330176in}}%
\pgfpathlineto{\pgfqpoint{1.469925in}{2.349726in}}%
\pgfpathlineto{\pgfqpoint{1.411654in}{2.370611in}}%
\pgfpathlineto{\pgfqpoint{1.353383in}{2.388987in}}%
\pgfpathlineto{\pgfqpoint{1.295113in}{2.405126in}}%
\pgfpathlineto{\pgfqpoint{1.227130in}{2.421416in}}%
\pgfpathlineto{\pgfqpoint{1.178313in}{2.431579in}}%
\pgfpathlineto{\pgfqpoint{1.110589in}{2.443729in}}%
\pgfpathlineto{\pgfqpoint{1.032895in}{2.455153in}}%
\pgfpathlineto{\pgfqpoint{0.955201in}{2.464141in}}%
\pgfpathlineto{\pgfqpoint{0.877506in}{2.470921in}}%
\pgfpathlineto{\pgfqpoint{0.790100in}{2.476126in}}%
\pgfpathlineto{\pgfqpoint{0.702694in}{2.479020in}}%
\pgfpathlineto{\pgfqpoint{0.683271in}{2.478931in}}%
\pgfpathlineto{\pgfqpoint{0.673559in}{2.474271in}}%
\pgfpathlineto{\pgfqpoint{0.672289in}{2.479825in}}%
\pgfpathlineto{\pgfqpoint{0.673559in}{2.485395in}}%
\pgfpathlineto{\pgfqpoint{0.683271in}{2.480718in}}%
\pgfpathlineto{\pgfqpoint{0.712406in}{2.480843in}}%
\pgfpathlineto{\pgfqpoint{0.799812in}{2.483983in}}%
\pgfpathlineto{\pgfqpoint{0.887432in}{2.489474in}}%
\pgfpathlineto{\pgfqpoint{0.964912in}{2.496504in}}%
\pgfpathlineto{\pgfqpoint{1.042607in}{2.505784in}}%
\pgfpathlineto{\pgfqpoint{1.110589in}{2.515921in}}%
\pgfpathlineto{\pgfqpoint{1.178571in}{2.528122in}}%
\pgfpathlineto{\pgfqpoint{1.246554in}{2.542633in}}%
\pgfpathlineto{\pgfqpoint{1.304825in}{2.557057in}}%
\pgfpathlineto{\pgfqpoint{1.372181in}{2.576316in}}%
\pgfpathlineto{\pgfqpoint{1.440789in}{2.599148in}}%
\pgfpathlineto{\pgfqpoint{1.499060in}{2.621401in}}%
\pgfpathlineto{\pgfqpoint{1.547619in}{2.642234in}}%
\pgfpathlineto{\pgfqpoint{1.586466in}{2.660594in}}%
\pgfpathlineto{\pgfqpoint{1.628763in}{2.682456in}}%
\pgfpathlineto{\pgfqpoint{1.679013in}{2.711404in}}%
\pgfpathlineto{\pgfqpoint{1.723822in}{2.740351in}}%
\pgfpathlineto{\pgfqpoint{1.764078in}{2.769298in}}%
\pgfpathlineto{\pgfqpoint{1.800352in}{2.798246in}}%
\pgfpathlineto{\pgfqpoint{1.838972in}{2.832632in}}%
\pgfpathlineto{\pgfqpoint{1.868108in}{2.861495in}}%
\pgfpathlineto{\pgfqpoint{1.889949in}{2.885088in}}%
\pgfpathlineto{\pgfqpoint{1.916667in}{2.916901in}}%
\pgfpathlineto{\pgfqpoint{1.936551in}{2.942982in}}%
\pgfpathlineto{\pgfqpoint{1.956564in}{2.971930in}}%
\pgfpathlineto{\pgfqpoint{1.980165in}{3.010526in}}%
\pgfpathlineto{\pgfqpoint{2.000466in}{3.049123in}}%
\pgfpathlineto{\pgfqpoint{2.017706in}{3.087719in}}%
\pgfpathlineto{\pgfqpoint{2.032046in}{3.126316in}}%
\pgfpathlineto{\pgfqpoint{2.043646in}{3.164912in}}%
\pgfpathlineto{\pgfqpoint{2.052632in}{3.203678in}}%
\pgfpathlineto{\pgfqpoint{2.059014in}{3.242105in}}%
\pgfpathlineto{\pgfqpoint{2.062890in}{3.280702in}}%
\pgfpathlineto{\pgfqpoint{2.064315in}{3.319298in}}%
\pgfpathlineto{\pgfqpoint{2.063268in}{3.357895in}}%
\pgfpathlineto{\pgfqpoint{2.059772in}{3.396491in}}%
\pgfpathlineto{\pgfqpoint{2.053772in}{3.435088in}}%
\pgfpathlineto{\pgfqpoint{2.045249in}{3.473684in}}%
\pgfpathlineto{\pgfqpoint{2.033208in}{3.515000in}}%
\pgfpathlineto{\pgfqpoint{2.020296in}{3.550877in}}%
\pgfpathlineto{\pgfqpoint{2.003649in}{3.589474in}}%
\pgfpathlineto{\pgfqpoint{1.984055in}{3.628070in}}%
\pgfpathlineto{\pgfqpoint{1.961326in}{3.666667in}}%
\pgfpathlineto{\pgfqpoint{1.935208in}{3.705263in}}%
\pgfpathlineto{\pgfqpoint{1.906955in}{3.742009in}}%
\pgfpathlineto{\pgfqpoint{1.887531in}{3.764942in}}%
\pgfpathlineto{\pgfqpoint{1.858396in}{3.796463in}}%
\pgfpathlineto{\pgfqpoint{1.829261in}{3.825141in}}%
\pgfpathlineto{\pgfqpoint{1.800125in}{3.851469in}}%
\pgfpathlineto{\pgfqpoint{1.761278in}{3.883519in}}%
\pgfpathlineto{\pgfqpoint{1.722431in}{3.912679in}}%
\pgfpathlineto{\pgfqpoint{1.683584in}{3.939429in}}%
\pgfpathlineto{\pgfqpoint{1.635025in}{3.970039in}}%
\pgfpathlineto{\pgfqpoint{1.586466in}{3.998052in}}%
\pgfpathlineto{\pgfqpoint{1.528195in}{4.028898in}}%
\pgfpathlineto{\pgfqpoint{1.469925in}{4.057362in}}%
\pgfpathlineto{\pgfqpoint{1.395410in}{4.091228in}}%
\pgfpathlineto{\pgfqpoint{1.281898in}{4.139474in}}%
\pgfpathlineto{\pgfqpoint{0.649279in}{4.400000in}}%
\pgfpathlineto{\pgfqpoint{0.649279in}{4.400000in}}%
\pgfusepath{stroke}%
\end{pgfscope}%
\begin{pgfscope}%
\pgfpathrectangle{\pgfqpoint{0.625000in}{0.550000in}}{\pgfqpoint{3.875000in}{3.850000in}} %
\pgfusepath{clip}%
\pgfsetbuttcap%
\pgfsetroundjoin%
\pgfsetlinewidth{0.501875pt}%
\definecolor{currentstroke}{rgb}{0.000000,0.000000,0.000000}%
\pgfsetstrokecolor{currentstroke}%
\pgfsetdash{}{0pt}%
\pgfpathmoveto{\pgfqpoint{0.625000in}{0.706284in}}%
\pgfpathlineto{\pgfqpoint{0.628239in}{0.704386in}}%
\pgfpathlineto{\pgfqpoint{0.625000in}{0.703447in}}%
\pgfusepath{stroke}%
\end{pgfscope}%
\begin{pgfscope}%
\pgfpathrectangle{\pgfqpoint{0.625000in}{0.550000in}}{\pgfqpoint{3.875000in}{3.850000in}} %
\pgfusepath{clip}%
\pgfsetbuttcap%
\pgfsetroundjoin%
\pgfsetlinewidth{0.501875pt}%
\definecolor{currentstroke}{rgb}{0.000000,0.000000,0.000000}%
\pgfsetstrokecolor{currentstroke}%
\pgfsetdash{}{0pt}%
\pgfpathmoveto{\pgfqpoint{0.625000in}{0.868789in}}%
\pgfpathlineto{\pgfqpoint{0.625503in}{0.868421in}}%
\pgfpathlineto{\pgfqpoint{0.630859in}{0.858772in}}%
\pgfpathlineto{\pgfqpoint{0.625000in}{0.855747in}}%
\pgfusepath{stroke}%
\end{pgfscope}%
\begin{pgfscope}%
\pgfpathrectangle{\pgfqpoint{0.625000in}{0.550000in}}{\pgfqpoint{3.875000in}{3.850000in}} %
\pgfusepath{clip}%
\pgfsetbuttcap%
\pgfsetroundjoin%
\pgfsetlinewidth{0.501875pt}%
\definecolor{currentstroke}{rgb}{0.000000,0.000000,0.000000}%
\pgfsetstrokecolor{currentstroke}%
\pgfsetdash{}{0pt}%
\pgfpathmoveto{\pgfqpoint{0.625000in}{0.929966in}}%
\pgfpathlineto{\pgfqpoint{0.625589in}{0.926316in}}%
\pgfpathlineto{\pgfqpoint{0.625000in}{0.925592in}}%
\pgfusepath{stroke}%
\end{pgfscope}%
\begin{pgfscope}%
\pgfpathrectangle{\pgfqpoint{0.625000in}{0.550000in}}{\pgfqpoint{3.875000in}{3.850000in}} %
\pgfusepath{clip}%
\pgfsetbuttcap%
\pgfsetroundjoin%
\pgfsetlinewidth{0.501875pt}%
\definecolor{currentstroke}{rgb}{0.000000,0.000000,0.000000}%
\pgfsetstrokecolor{currentstroke}%
\pgfsetdash{}{0pt}%
\pgfpathmoveto{\pgfqpoint{0.625000in}{0.966231in}}%
\pgfpathlineto{\pgfqpoint{0.626724in}{0.964912in}}%
\pgfpathlineto{\pgfqpoint{0.625000in}{0.964513in}}%
\pgfusepath{stroke}%
\end{pgfscope}%
\begin{pgfscope}%
\pgfpathrectangle{\pgfqpoint{0.625000in}{0.550000in}}{\pgfqpoint{3.875000in}{3.850000in}} %
\pgfusepath{clip}%
\pgfsetbuttcap%
\pgfsetroundjoin%
\pgfsetlinewidth{0.501875pt}%
\definecolor{currentstroke}{rgb}{0.000000,0.000000,0.000000}%
\pgfsetstrokecolor{currentstroke}%
\pgfsetdash{}{0pt}%
\pgfpathmoveto{\pgfqpoint{0.625000in}{1.017657in}}%
\pgfpathlineto{\pgfqpoint{0.633206in}{1.013158in}}%
\pgfpathlineto{\pgfqpoint{0.625000in}{1.010398in}}%
\pgfusepath{stroke}%
\end{pgfscope}%
\begin{pgfscope}%
\pgfpathrectangle{\pgfqpoint{0.625000in}{0.550000in}}{\pgfqpoint{3.875000in}{3.850000in}} %
\pgfusepath{clip}%
\pgfsetbuttcap%
\pgfsetroundjoin%
\pgfsetlinewidth{0.501875pt}%
\definecolor{currentstroke}{rgb}{0.000000,0.000000,0.000000}%
\pgfsetstrokecolor{currentstroke}%
\pgfsetdash{}{0pt}%
\pgfpathmoveto{\pgfqpoint{0.625000in}{1.168342in}}%
\pgfpathlineto{\pgfqpoint{0.628039in}{1.167544in}}%
\pgfpathlineto{\pgfqpoint{0.625000in}{1.166463in}}%
\pgfusepath{stroke}%
\end{pgfscope}%
\begin{pgfscope}%
\pgfpathrectangle{\pgfqpoint{0.625000in}{0.550000in}}{\pgfqpoint{3.875000in}{3.850000in}} %
\pgfusepath{clip}%
\pgfsetbuttcap%
\pgfsetroundjoin%
\pgfsetlinewidth{0.501875pt}%
\definecolor{currentstroke}{rgb}{0.000000,0.000000,0.000000}%
\pgfsetstrokecolor{currentstroke}%
\pgfsetdash{}{0pt}%
\pgfpathmoveto{\pgfqpoint{0.625000in}{1.323143in}}%
\pgfpathlineto{\pgfqpoint{0.634712in}{1.326754in}}%
\pgfpathlineto{\pgfqpoint{0.644424in}{1.330101in}}%
\pgfpathlineto{\pgfqpoint{0.649215in}{1.331579in}}%
\pgfpathlineto{\pgfqpoint{0.654135in}{1.334265in}}%
\pgfpathlineto{\pgfqpoint{0.663847in}{1.338961in}}%
\pgfpathlineto{\pgfqpoint{0.669318in}{1.341228in}}%
\pgfpathlineto{\pgfqpoint{0.673559in}{1.344532in}}%
\pgfpathlineto{\pgfqpoint{0.683271in}{1.348980in}}%
\pgfpathlineto{\pgfqpoint{0.692982in}{1.350310in}}%
\pgfpathlineto{\pgfqpoint{0.702694in}{1.348614in}}%
\pgfpathlineto{\pgfqpoint{0.712406in}{1.344130in}}%
\pgfpathlineto{\pgfqpoint{0.716592in}{1.341228in}}%
\pgfpathlineto{\pgfqpoint{0.722118in}{1.337317in}}%
\pgfpathlineto{\pgfqpoint{0.728270in}{1.331579in}}%
\pgfpathlineto{\pgfqpoint{0.731830in}{1.327791in}}%
\pgfpathlineto{\pgfqpoint{0.736513in}{1.321930in}}%
\pgfpathlineto{\pgfqpoint{0.741541in}{1.313935in}}%
\pgfpathlineto{\pgfqpoint{0.742499in}{1.312281in}}%
\pgfpathlineto{\pgfqpoint{0.746842in}{1.302632in}}%
\pgfpathlineto{\pgfqpoint{0.749574in}{1.292982in}}%
\pgfpathlineto{\pgfqpoint{0.750884in}{1.283333in}}%
\pgfpathlineto{\pgfqpoint{0.750890in}{1.273684in}}%
\pgfpathlineto{\pgfqpoint{0.749597in}{1.264035in}}%
\pgfpathlineto{\pgfqpoint{0.746840in}{1.254386in}}%
\pgfpathlineto{\pgfqpoint{0.742160in}{1.244737in}}%
\pgfpathlineto{\pgfqpoint{0.741541in}{1.243893in}}%
\pgfpathlineto{\pgfqpoint{0.735926in}{1.235088in}}%
\pgfpathlineto{\pgfqpoint{0.731830in}{1.230889in}}%
\pgfpathlineto{\pgfqpoint{0.727077in}{1.225439in}}%
\pgfpathlineto{\pgfqpoint{0.722118in}{1.221682in}}%
\pgfpathlineto{\pgfqpoint{0.714680in}{1.215789in}}%
\pgfpathlineto{\pgfqpoint{0.712406in}{1.214588in}}%
\pgfpathlineto{\pgfqpoint{0.702694in}{1.209260in}}%
\pgfpathlineto{\pgfqpoint{0.696840in}{1.206140in}}%
\pgfpathlineto{\pgfqpoint{0.692982in}{1.204818in}}%
\pgfpathlineto{\pgfqpoint{0.683271in}{1.201503in}}%
\pgfpathlineto{\pgfqpoint{0.674480in}{1.206140in}}%
\pgfpathlineto{\pgfqpoint{0.683271in}{1.212886in}}%
\pgfpathlineto{\pgfqpoint{0.686841in}{1.215789in}}%
\pgfpathlineto{\pgfqpoint{0.692982in}{1.223875in}}%
\pgfpathlineto{\pgfqpoint{0.694463in}{1.225439in}}%
\pgfpathlineto{\pgfqpoint{0.699640in}{1.235088in}}%
\pgfpathlineto{\pgfqpoint{0.702276in}{1.244737in}}%
\pgfpathlineto{\pgfqpoint{0.702694in}{1.249594in}}%
\pgfpathlineto{\pgfqpoint{0.703252in}{1.254386in}}%
\pgfpathlineto{\pgfqpoint{0.702694in}{1.260353in}}%
\pgfpathlineto{\pgfqpoint{0.702362in}{1.264035in}}%
\pgfpathlineto{\pgfqpoint{0.699626in}{1.273684in}}%
\pgfpathlineto{\pgfqpoint{0.694402in}{1.283333in}}%
\pgfpathlineto{\pgfqpoint{0.692982in}{1.285226in}}%
\pgfpathlineto{\pgfqpoint{0.685965in}{1.292982in}}%
\pgfpathlineto{\pgfqpoint{0.683271in}{1.295276in}}%
\pgfpathlineto{\pgfqpoint{0.673559in}{1.301661in}}%
\pgfpathlineto{\pgfqpoint{0.671711in}{1.302632in}}%
\pgfpathlineto{\pgfqpoint{0.663847in}{1.306132in}}%
\pgfpathlineto{\pgfqpoint{0.654135in}{1.309845in}}%
\pgfpathlineto{\pgfqpoint{0.649323in}{1.312281in}}%
\pgfpathlineto{\pgfqpoint{0.644424in}{1.313757in}}%
\pgfpathlineto{\pgfqpoint{0.634712in}{1.317105in}}%
\pgfpathlineto{\pgfqpoint{0.625000in}{1.315539in}}%
\pgfusepath{stroke}%
\end{pgfscope}%
\begin{pgfscope}%
\pgfpathrectangle{\pgfqpoint{0.625000in}{0.550000in}}{\pgfqpoint{3.875000in}{3.850000in}} %
\pgfusepath{clip}%
\pgfsetbuttcap%
\pgfsetroundjoin%
\pgfsetlinewidth{0.501875pt}%
\definecolor{currentstroke}{rgb}{0.000000,0.000000,0.000000}%
\pgfsetstrokecolor{currentstroke}%
\pgfsetdash{}{0pt}%
\pgfpathmoveto{\pgfqpoint{0.625000in}{1.480978in}}%
\pgfpathlineto{\pgfqpoint{0.633309in}{1.485965in}}%
\pgfpathlineto{\pgfqpoint{0.634712in}{1.490789in}}%
\pgfpathlineto{\pgfqpoint{0.637300in}{1.485965in}}%
\pgfpathlineto{\pgfqpoint{0.641277in}{1.476316in}}%
\pgfpathlineto{\pgfqpoint{0.644424in}{1.468079in}}%
\pgfpathlineto{\pgfqpoint{0.645162in}{1.466667in}}%
\pgfpathlineto{\pgfqpoint{0.649278in}{1.457018in}}%
\pgfpathlineto{\pgfqpoint{0.653259in}{1.447368in}}%
\pgfpathlineto{\pgfqpoint{0.654135in}{1.444982in}}%
\pgfpathlineto{\pgfqpoint{0.657206in}{1.437719in}}%
\pgfpathlineto{\pgfqpoint{0.660336in}{1.428070in}}%
\pgfpathlineto{\pgfqpoint{0.663847in}{1.422666in}}%
\pgfpathlineto{\pgfqpoint{0.666354in}{1.418421in}}%
\pgfpathlineto{\pgfqpoint{0.671693in}{1.408772in}}%
\pgfpathlineto{\pgfqpoint{0.673559in}{1.400742in}}%
\pgfpathlineto{\pgfqpoint{0.674019in}{1.399123in}}%
\pgfpathlineto{\pgfqpoint{0.673559in}{1.395605in}}%
\pgfpathlineto{\pgfqpoint{0.672919in}{1.389474in}}%
\pgfpathlineto{\pgfqpoint{0.663847in}{1.381209in}}%
\pgfpathlineto{\pgfqpoint{0.661416in}{1.379825in}}%
\pgfpathlineto{\pgfqpoint{0.654135in}{1.378453in}}%
\pgfpathlineto{\pgfqpoint{0.644424in}{1.377353in}}%
\pgfpathlineto{\pgfqpoint{0.641470in}{1.379825in}}%
\pgfpathlineto{\pgfqpoint{0.644424in}{1.388001in}}%
\pgfpathlineto{\pgfqpoint{0.645644in}{1.389474in}}%
\pgfpathlineto{\pgfqpoint{0.644424in}{1.392240in}}%
\pgfpathlineto{\pgfqpoint{0.642233in}{1.399123in}}%
\pgfpathlineto{\pgfqpoint{0.634712in}{1.404696in}}%
\pgfpathlineto{\pgfqpoint{0.633429in}{1.408772in}}%
\pgfpathlineto{\pgfqpoint{0.634712in}{1.412504in}}%
\pgfpathlineto{\pgfqpoint{0.643563in}{1.418421in}}%
\pgfpathlineto{\pgfqpoint{0.634712in}{1.424607in}}%
\pgfpathlineto{\pgfqpoint{0.634307in}{1.428070in}}%
\pgfpathlineto{\pgfqpoint{0.633140in}{1.437719in}}%
\pgfpathlineto{\pgfqpoint{0.634330in}{1.447368in}}%
\pgfpathlineto{\pgfqpoint{0.632841in}{1.457018in}}%
\pgfpathlineto{\pgfqpoint{0.632523in}{1.466667in}}%
\pgfpathlineto{\pgfqpoint{0.625000in}{1.472793in}}%
\pgfusepath{stroke}%
\end{pgfscope}%
\begin{pgfscope}%
\pgfpathrectangle{\pgfqpoint{0.625000in}{0.550000in}}{\pgfqpoint{3.875000in}{3.850000in}} %
\pgfusepath{clip}%
\pgfsetbuttcap%
\pgfsetroundjoin%
\pgfsetlinewidth{0.501875pt}%
\definecolor{currentstroke}{rgb}{0.000000,0.000000,0.000000}%
\pgfsetstrokecolor{currentstroke}%
\pgfsetdash{}{0pt}%
\pgfpathmoveto{\pgfqpoint{0.625000in}{1.506622in}}%
\pgfpathlineto{\pgfqpoint{0.628202in}{1.505263in}}%
\pgfpathlineto{\pgfqpoint{0.625000in}{1.503904in}}%
\pgfusepath{stroke}%
\end{pgfscope}%
\begin{pgfscope}%
\pgfpathrectangle{\pgfqpoint{0.625000in}{0.550000in}}{\pgfqpoint{3.875000in}{3.850000in}} %
\pgfusepath{clip}%
\pgfsetbuttcap%
\pgfsetroundjoin%
\pgfsetlinewidth{0.501875pt}%
\definecolor{currentstroke}{rgb}{0.000000,0.000000,0.000000}%
\pgfsetstrokecolor{currentstroke}%
\pgfsetdash{}{0pt}%
\pgfpathmoveto{\pgfqpoint{0.625000in}{1.584023in}}%
\pgfpathlineto{\pgfqpoint{0.625737in}{1.582456in}}%
\pgfpathlineto{\pgfqpoint{0.625000in}{1.581972in}}%
\pgfusepath{stroke}%
\end{pgfscope}%
\begin{pgfscope}%
\pgfpathrectangle{\pgfqpoint{0.625000in}{0.550000in}}{\pgfqpoint{3.875000in}{3.850000in}} %
\pgfusepath{clip}%
\pgfsetbuttcap%
\pgfsetroundjoin%
\pgfsetlinewidth{0.501875pt}%
\definecolor{currentstroke}{rgb}{0.000000,0.000000,0.000000}%
\pgfsetstrokecolor{currentstroke}%
\pgfsetdash{}{0pt}%
\pgfpathmoveto{\pgfqpoint{0.625000in}{1.633661in}}%
\pgfpathlineto{\pgfqpoint{0.627394in}{1.630702in}}%
\pgfpathlineto{\pgfqpoint{0.625000in}{1.627787in}}%
\pgfusepath{stroke}%
\end{pgfscope}%
\begin{pgfscope}%
\pgfpathrectangle{\pgfqpoint{0.625000in}{0.550000in}}{\pgfqpoint{3.875000in}{3.850000in}} %
\pgfusepath{clip}%
\pgfsetbuttcap%
\pgfsetroundjoin%
\pgfsetlinewidth{0.501875pt}%
\definecolor{currentstroke}{rgb}{0.000000,0.000000,0.000000}%
\pgfsetstrokecolor{currentstroke}%
\pgfsetdash{}{0pt}%
\pgfpathmoveto{\pgfqpoint{0.625000in}{1.786562in}}%
\pgfpathlineto{\pgfqpoint{0.627738in}{1.785088in}}%
\pgfpathlineto{\pgfqpoint{0.625000in}{1.783387in}}%
\pgfusepath{stroke}%
\end{pgfscope}%
\begin{pgfscope}%
\pgfpathrectangle{\pgfqpoint{0.625000in}{0.550000in}}{\pgfqpoint{3.875000in}{3.850000in}} %
\pgfusepath{clip}%
\pgfsetbuttcap%
\pgfsetroundjoin%
\pgfsetlinewidth{0.501875pt}%
\definecolor{currentstroke}{rgb}{0.000000,0.000000,0.000000}%
\pgfsetstrokecolor{currentstroke}%
\pgfsetdash{}{0pt}%
\pgfpathmoveto{\pgfqpoint{0.625000in}{1.872219in}}%
\pgfpathlineto{\pgfqpoint{0.625312in}{1.871930in}}%
\pgfpathlineto{\pgfqpoint{0.625000in}{1.870872in}}%
\pgfusepath{stroke}%
\end{pgfscope}%
\begin{pgfscope}%
\pgfpathrectangle{\pgfqpoint{0.625000in}{0.550000in}}{\pgfqpoint{3.875000in}{3.850000in}} %
\pgfusepath{clip}%
\pgfsetbuttcap%
\pgfsetroundjoin%
\pgfsetlinewidth{0.501875pt}%
\definecolor{currentstroke}{rgb}{0.000000,0.000000,0.000000}%
\pgfsetstrokecolor{currentstroke}%
\pgfsetdash{}{0pt}%
\pgfpathmoveto{\pgfqpoint{0.625000in}{1.941441in}}%
\pgfpathlineto{\pgfqpoint{0.627225in}{1.939474in}}%
\pgfpathlineto{\pgfqpoint{0.625000in}{1.937598in}}%
\pgfusepath{stroke}%
\end{pgfscope}%
\begin{pgfscope}%
\pgfpathrectangle{\pgfqpoint{0.625000in}{0.550000in}}{\pgfqpoint{3.875000in}{3.850000in}} %
\pgfusepath{clip}%
\pgfsetbuttcap%
\pgfsetroundjoin%
\pgfsetlinewidth{0.501875pt}%
\definecolor{currentstroke}{rgb}{0.000000,0.000000,0.000000}%
\pgfsetstrokecolor{currentstroke}%
\pgfsetdash{}{0pt}%
\pgfpathmoveto{\pgfqpoint{0.625000in}{2.260018in}}%
\pgfpathlineto{\pgfqpoint{0.630617in}{2.257895in}}%
\pgfpathlineto{\pgfqpoint{0.629684in}{2.248246in}}%
\pgfpathlineto{\pgfqpoint{0.625000in}{2.245901in}}%
\pgfusepath{stroke}%
\end{pgfscope}%
\begin{pgfscope}%
\pgfpathrectangle{\pgfqpoint{0.625000in}{0.550000in}}{\pgfqpoint{3.875000in}{3.850000in}} %
\pgfusepath{clip}%
\pgfsetbuttcap%
\pgfsetroundjoin%
\pgfsetlinewidth{0.501875pt}%
\definecolor{currentstroke}{rgb}{0.000000,0.000000,0.000000}%
\pgfsetstrokecolor{currentstroke}%
\pgfsetdash{}{0pt}%
\pgfpathmoveto{\pgfqpoint{0.625000in}{2.403769in}}%
\pgfpathlineto{\pgfqpoint{0.632949in}{2.402632in}}%
\pgfpathlineto{\pgfqpoint{0.625000in}{2.401410in}}%
\pgfusepath{stroke}%
\end{pgfscope}%
\begin{pgfscope}%
\pgfpathrectangle{\pgfqpoint{0.625000in}{0.550000in}}{\pgfqpoint{3.875000in}{3.850000in}} %
\pgfusepath{clip}%
\pgfsetbuttcap%
\pgfsetroundjoin%
\pgfsetlinewidth{0.501875pt}%
\definecolor{currentstroke}{rgb}{0.000000,0.000000,0.000000}%
\pgfsetstrokecolor{currentstroke}%
\pgfsetdash{}{0pt}%
\pgfpathmoveto{\pgfqpoint{0.625000in}{2.558240in}}%
\pgfpathlineto{\pgfqpoint{0.632949in}{2.557018in}}%
\pgfpathlineto{\pgfqpoint{0.625000in}{2.555880in}}%
\pgfusepath{stroke}%
\end{pgfscope}%
\begin{pgfscope}%
\pgfpathrectangle{\pgfqpoint{0.625000in}{0.550000in}}{\pgfqpoint{3.875000in}{3.850000in}} %
\pgfusepath{clip}%
\pgfsetbuttcap%
\pgfsetroundjoin%
\pgfsetlinewidth{0.501875pt}%
\definecolor{currentstroke}{rgb}{0.000000,0.000000,0.000000}%
\pgfsetstrokecolor{currentstroke}%
\pgfsetdash{}{0pt}%
\pgfpathmoveto{\pgfqpoint{0.625000in}{2.713748in}}%
\pgfpathlineto{\pgfqpoint{0.629684in}{2.711404in}}%
\pgfpathlineto{\pgfqpoint{0.630617in}{2.701754in}}%
\pgfpathlineto{\pgfqpoint{0.625000in}{2.699631in}}%
\pgfusepath{stroke}%
\end{pgfscope}%
\begin{pgfscope}%
\pgfpathrectangle{\pgfqpoint{0.625000in}{0.550000in}}{\pgfqpoint{3.875000in}{3.850000in}} %
\pgfusepath{clip}%
\pgfsetbuttcap%
\pgfsetroundjoin%
\pgfsetlinewidth{0.501875pt}%
\definecolor{currentstroke}{rgb}{0.000000,0.000000,0.000000}%
\pgfsetstrokecolor{currentstroke}%
\pgfsetdash{}{0pt}%
\pgfpathmoveto{\pgfqpoint{0.625000in}{3.022051in}}%
\pgfpathlineto{\pgfqpoint{0.627225in}{3.020175in}}%
\pgfpathlineto{\pgfqpoint{0.625000in}{3.018208in}}%
\pgfusepath{stroke}%
\end{pgfscope}%
\begin{pgfscope}%
\pgfpathrectangle{\pgfqpoint{0.625000in}{0.550000in}}{\pgfqpoint{3.875000in}{3.850000in}} %
\pgfusepath{clip}%
\pgfsetbuttcap%
\pgfsetroundjoin%
\pgfsetlinewidth{0.501875pt}%
\definecolor{currentstroke}{rgb}{0.000000,0.000000,0.000000}%
\pgfsetstrokecolor{currentstroke}%
\pgfsetdash{}{0pt}%
\pgfpathmoveto{\pgfqpoint{0.625000in}{3.087995in}}%
\pgfpathlineto{\pgfqpoint{0.625312in}{3.087719in}}%
\pgfpathlineto{\pgfqpoint{0.625000in}{3.087430in}}%
\pgfusepath{stroke}%
\end{pgfscope}%
\begin{pgfscope}%
\pgfpathrectangle{\pgfqpoint{0.625000in}{0.550000in}}{\pgfqpoint{3.875000in}{3.850000in}} %
\pgfusepath{clip}%
\pgfsetbuttcap%
\pgfsetroundjoin%
\pgfsetlinewidth{0.501875pt}%
\definecolor{currentstroke}{rgb}{0.000000,0.000000,0.000000}%
\pgfsetstrokecolor{currentstroke}%
\pgfsetdash{}{0pt}%
\pgfpathmoveto{\pgfqpoint{0.625000in}{3.176262in}}%
\pgfpathlineto{\pgfqpoint{0.627738in}{3.174561in}}%
\pgfpathlineto{\pgfqpoint{0.625000in}{3.173087in}}%
\pgfusepath{stroke}%
\end{pgfscope}%
\begin{pgfscope}%
\pgfpathrectangle{\pgfqpoint{0.625000in}{0.550000in}}{\pgfqpoint{3.875000in}{3.850000in}} %
\pgfusepath{clip}%
\pgfsetbuttcap%
\pgfsetroundjoin%
\pgfsetlinewidth{0.501875pt}%
\definecolor{currentstroke}{rgb}{0.000000,0.000000,0.000000}%
\pgfsetstrokecolor{currentstroke}%
\pgfsetdash{}{0pt}%
\pgfpathmoveto{\pgfqpoint{0.625000in}{3.331862in}}%
\pgfpathlineto{\pgfqpoint{0.627394in}{3.328947in}}%
\pgfpathlineto{\pgfqpoint{0.625000in}{3.325988in}}%
\pgfusepath{stroke}%
\end{pgfscope}%
\begin{pgfscope}%
\pgfpathrectangle{\pgfqpoint{0.625000in}{0.550000in}}{\pgfqpoint{3.875000in}{3.850000in}} %
\pgfusepath{clip}%
\pgfsetbuttcap%
\pgfsetroundjoin%
\pgfsetlinewidth{0.501875pt}%
\definecolor{currentstroke}{rgb}{0.000000,0.000000,0.000000}%
\pgfsetstrokecolor{currentstroke}%
\pgfsetdash{}{0pt}%
\pgfpathmoveto{\pgfqpoint{0.625000in}{3.377677in}}%
\pgfpathlineto{\pgfqpoint{0.625737in}{3.377193in}}%
\pgfpathlineto{\pgfqpoint{0.625000in}{3.375626in}}%
\pgfusepath{stroke}%
\end{pgfscope}%
\begin{pgfscope}%
\pgfpathrectangle{\pgfqpoint{0.625000in}{0.550000in}}{\pgfqpoint{3.875000in}{3.850000in}} %
\pgfusepath{clip}%
\pgfsetbuttcap%
\pgfsetroundjoin%
\pgfsetlinewidth{0.501875pt}%
\definecolor{currentstroke}{rgb}{0.000000,0.000000,0.000000}%
\pgfsetstrokecolor{currentstroke}%
\pgfsetdash{}{0pt}%
\pgfpathmoveto{\pgfqpoint{0.625000in}{3.455745in}}%
\pgfpathlineto{\pgfqpoint{0.628202in}{3.454386in}}%
\pgfpathlineto{\pgfqpoint{0.625000in}{3.453027in}}%
\pgfusepath{stroke}%
\end{pgfscope}%
\begin{pgfscope}%
\pgfpathrectangle{\pgfqpoint{0.625000in}{0.550000in}}{\pgfqpoint{3.875000in}{3.850000in}} %
\pgfusepath{clip}%
\pgfsetbuttcap%
\pgfsetroundjoin%
\pgfsetlinewidth{0.501875pt}%
\definecolor{currentstroke}{rgb}{0.000000,0.000000,0.000000}%
\pgfsetstrokecolor{currentstroke}%
\pgfsetdash{}{0pt}%
\pgfpathmoveto{\pgfqpoint{0.625000in}{3.486856in}}%
\pgfpathlineto{\pgfqpoint{0.632523in}{3.492982in}}%
\pgfpathlineto{\pgfqpoint{0.632841in}{3.502632in}}%
\pgfpathlineto{\pgfqpoint{0.634330in}{3.512281in}}%
\pgfpathlineto{\pgfqpoint{0.633140in}{3.521930in}}%
\pgfpathlineto{\pgfqpoint{0.634307in}{3.531579in}}%
\pgfpathlineto{\pgfqpoint{0.634712in}{3.535042in}}%
\pgfpathlineto{\pgfqpoint{0.643563in}{3.541228in}}%
\pgfpathlineto{\pgfqpoint{0.634712in}{3.547145in}}%
\pgfpathlineto{\pgfqpoint{0.633429in}{3.550877in}}%
\pgfpathlineto{\pgfqpoint{0.634712in}{3.554954in}}%
\pgfpathlineto{\pgfqpoint{0.642233in}{3.560526in}}%
\pgfpathlineto{\pgfqpoint{0.644424in}{3.567409in}}%
\pgfpathlineto{\pgfqpoint{0.645644in}{3.570175in}}%
\pgfpathlineto{\pgfqpoint{0.644424in}{3.571648in}}%
\pgfpathlineto{\pgfqpoint{0.641470in}{3.579825in}}%
\pgfpathlineto{\pgfqpoint{0.644424in}{3.582296in}}%
\pgfpathlineto{\pgfqpoint{0.654135in}{3.581196in}}%
\pgfpathlineto{\pgfqpoint{0.661416in}{3.579825in}}%
\pgfpathlineto{\pgfqpoint{0.663847in}{3.578440in}}%
\pgfpathlineto{\pgfqpoint{0.672919in}{3.570175in}}%
\pgfpathlineto{\pgfqpoint{0.673559in}{3.564044in}}%
\pgfpathlineto{\pgfqpoint{0.674019in}{3.560526in}}%
\pgfpathlineto{\pgfqpoint{0.673559in}{3.558907in}}%
\pgfpathlineto{\pgfqpoint{0.671693in}{3.550877in}}%
\pgfpathlineto{\pgfqpoint{0.666354in}{3.541228in}}%
\pgfpathlineto{\pgfqpoint{0.663847in}{3.536983in}}%
\pgfpathlineto{\pgfqpoint{0.660336in}{3.531579in}}%
\pgfpathlineto{\pgfqpoint{0.657206in}{3.521930in}}%
\pgfpathlineto{\pgfqpoint{0.654135in}{3.514667in}}%
\pgfpathlineto{\pgfqpoint{0.653259in}{3.512281in}}%
\pgfpathlineto{\pgfqpoint{0.649278in}{3.502632in}}%
\pgfpathlineto{\pgfqpoint{0.645162in}{3.492982in}}%
\pgfpathlineto{\pgfqpoint{0.644424in}{3.491570in}}%
\pgfpathlineto{\pgfqpoint{0.641277in}{3.483333in}}%
\pgfpathlineto{\pgfqpoint{0.637300in}{3.473684in}}%
\pgfpathlineto{\pgfqpoint{0.634712in}{3.468860in}}%
\pgfpathlineto{\pgfqpoint{0.633309in}{3.473684in}}%
\pgfpathlineto{\pgfqpoint{0.625000in}{3.478671in}}%
\pgfusepath{stroke}%
\end{pgfscope}%
\begin{pgfscope}%
\pgfpathrectangle{\pgfqpoint{0.625000in}{0.550000in}}{\pgfqpoint{3.875000in}{3.850000in}} %
\pgfusepath{clip}%
\pgfsetbuttcap%
\pgfsetroundjoin%
\pgfsetlinewidth{0.501875pt}%
\definecolor{currentstroke}{rgb}{0.000000,0.000000,0.000000}%
\pgfsetstrokecolor{currentstroke}%
\pgfsetdash{}{0pt}%
\pgfpathmoveto{\pgfqpoint{0.625000in}{3.644110in}}%
\pgfpathlineto{\pgfqpoint{0.634712in}{3.642544in}}%
\pgfpathlineto{\pgfqpoint{0.644424in}{3.645892in}}%
\pgfpathlineto{\pgfqpoint{0.649323in}{3.647368in}}%
\pgfpathlineto{\pgfqpoint{0.654135in}{3.649804in}}%
\pgfpathlineto{\pgfqpoint{0.663847in}{3.653517in}}%
\pgfpathlineto{\pgfqpoint{0.671711in}{3.657018in}}%
\pgfpathlineto{\pgfqpoint{0.673559in}{3.657989in}}%
\pgfpathlineto{\pgfqpoint{0.683271in}{3.664374in}}%
\pgfpathlineto{\pgfqpoint{0.685965in}{3.666667in}}%
\pgfpathlineto{\pgfqpoint{0.692982in}{3.674423in}}%
\pgfpathlineto{\pgfqpoint{0.694402in}{3.676316in}}%
\pgfpathlineto{\pgfqpoint{0.699626in}{3.685965in}}%
\pgfpathlineto{\pgfqpoint{0.702362in}{3.695614in}}%
\pgfpathlineto{\pgfqpoint{0.702694in}{3.699296in}}%
\pgfpathlineto{\pgfqpoint{0.703252in}{3.705263in}}%
\pgfpathlineto{\pgfqpoint{0.702694in}{3.710055in}}%
\pgfpathlineto{\pgfqpoint{0.702276in}{3.714912in}}%
\pgfpathlineto{\pgfqpoint{0.699640in}{3.724561in}}%
\pgfpathlineto{\pgfqpoint{0.694463in}{3.734211in}}%
\pgfpathlineto{\pgfqpoint{0.692982in}{3.735774in}}%
\pgfpathlineto{\pgfqpoint{0.686841in}{3.743860in}}%
\pgfpathlineto{\pgfqpoint{0.683271in}{3.746763in}}%
\pgfpathlineto{\pgfqpoint{0.674480in}{3.753509in}}%
\pgfpathlineto{\pgfqpoint{0.683271in}{3.758146in}}%
\pgfpathlineto{\pgfqpoint{0.692982in}{3.754831in}}%
\pgfpathlineto{\pgfqpoint{0.696840in}{3.753509in}}%
\pgfpathlineto{\pgfqpoint{0.702694in}{3.750389in}}%
\pgfpathlineto{\pgfqpoint{0.712406in}{3.745061in}}%
\pgfpathlineto{\pgfqpoint{0.714680in}{3.743860in}}%
\pgfpathlineto{\pgfqpoint{0.722118in}{3.737967in}}%
\pgfpathlineto{\pgfqpoint{0.727077in}{3.734211in}}%
\pgfpathlineto{\pgfqpoint{0.731830in}{3.728760in}}%
\pgfpathlineto{\pgfqpoint{0.735926in}{3.724561in}}%
\pgfpathlineto{\pgfqpoint{0.741541in}{3.715756in}}%
\pgfpathlineto{\pgfqpoint{0.742160in}{3.714912in}}%
\pgfpathlineto{\pgfqpoint{0.746840in}{3.705263in}}%
\pgfpathlineto{\pgfqpoint{0.749597in}{3.695614in}}%
\pgfpathlineto{\pgfqpoint{0.750890in}{3.685965in}}%
\pgfpathlineto{\pgfqpoint{0.750884in}{3.676316in}}%
\pgfpathlineto{\pgfqpoint{0.749574in}{3.666667in}}%
\pgfpathlineto{\pgfqpoint{0.746842in}{3.657018in}}%
\pgfpathlineto{\pgfqpoint{0.742499in}{3.647368in}}%
\pgfpathlineto{\pgfqpoint{0.741541in}{3.645714in}}%
\pgfpathlineto{\pgfqpoint{0.736513in}{3.637719in}}%
\pgfpathlineto{\pgfqpoint{0.731830in}{3.631858in}}%
\pgfpathlineto{\pgfqpoint{0.728270in}{3.628070in}}%
\pgfpathlineto{\pgfqpoint{0.722118in}{3.622332in}}%
\pgfpathlineto{\pgfqpoint{0.716592in}{3.618421in}}%
\pgfpathlineto{\pgfqpoint{0.712406in}{3.615519in}}%
\pgfpathlineto{\pgfqpoint{0.702694in}{3.611035in}}%
\pgfpathlineto{\pgfqpoint{0.692982in}{3.609339in}}%
\pgfpathlineto{\pgfqpoint{0.683271in}{3.610669in}}%
\pgfpathlineto{\pgfqpoint{0.673559in}{3.615117in}}%
\pgfpathlineto{\pgfqpoint{0.669318in}{3.618421in}}%
\pgfpathlineto{\pgfqpoint{0.663847in}{3.620688in}}%
\pgfpathlineto{\pgfqpoint{0.654135in}{3.625384in}}%
\pgfpathlineto{\pgfqpoint{0.649215in}{3.628070in}}%
\pgfpathlineto{\pgfqpoint{0.644424in}{3.629549in}}%
\pgfpathlineto{\pgfqpoint{0.634712in}{3.632895in}}%
\pgfpathlineto{\pgfqpoint{0.625000in}{3.636506in}}%
\pgfusepath{stroke}%
\end{pgfscope}%
\begin{pgfscope}%
\pgfpathrectangle{\pgfqpoint{0.625000in}{0.550000in}}{\pgfqpoint{3.875000in}{3.850000in}} %
\pgfusepath{clip}%
\pgfsetbuttcap%
\pgfsetroundjoin%
\pgfsetlinewidth{0.501875pt}%
\definecolor{currentstroke}{rgb}{0.000000,0.000000,0.000000}%
\pgfsetstrokecolor{currentstroke}%
\pgfsetdash{}{0pt}%
\pgfpathmoveto{\pgfqpoint{0.625000in}{3.793186in}}%
\pgfpathlineto{\pgfqpoint{0.628039in}{3.792105in}}%
\pgfpathlineto{\pgfqpoint{0.625000in}{3.791307in}}%
\pgfusepath{stroke}%
\end{pgfscope}%
\begin{pgfscope}%
\pgfpathrectangle{\pgfqpoint{0.625000in}{0.550000in}}{\pgfqpoint{3.875000in}{3.850000in}} %
\pgfusepath{clip}%
\pgfsetbuttcap%
\pgfsetroundjoin%
\pgfsetlinewidth{0.501875pt}%
\definecolor{currentstroke}{rgb}{0.000000,0.000000,0.000000}%
\pgfsetstrokecolor{currentstroke}%
\pgfsetdash{}{0pt}%
\pgfpathmoveto{\pgfqpoint{0.625000in}{3.949252in}}%
\pgfpathlineto{\pgfqpoint{0.633206in}{3.946491in}}%
\pgfpathlineto{\pgfqpoint{0.625000in}{3.941992in}}%
\pgfusepath{stroke}%
\end{pgfscope}%
\begin{pgfscope}%
\pgfpathrectangle{\pgfqpoint{0.625000in}{0.550000in}}{\pgfqpoint{3.875000in}{3.850000in}} %
\pgfusepath{clip}%
\pgfsetbuttcap%
\pgfsetroundjoin%
\pgfsetlinewidth{0.501875pt}%
\definecolor{currentstroke}{rgb}{0.000000,0.000000,0.000000}%
\pgfsetstrokecolor{currentstroke}%
\pgfsetdash{}{0pt}%
\pgfpathmoveto{\pgfqpoint{0.625000in}{3.995136in}}%
\pgfpathlineto{\pgfqpoint{0.626724in}{3.994737in}}%
\pgfpathlineto{\pgfqpoint{0.625000in}{3.993418in}}%
\pgfusepath{stroke}%
\end{pgfscope}%
\begin{pgfscope}%
\pgfpathrectangle{\pgfqpoint{0.625000in}{0.550000in}}{\pgfqpoint{3.875000in}{3.850000in}} %
\pgfusepath{clip}%
\pgfsetbuttcap%
\pgfsetroundjoin%
\pgfsetlinewidth{0.501875pt}%
\definecolor{currentstroke}{rgb}{0.000000,0.000000,0.000000}%
\pgfsetstrokecolor{currentstroke}%
\pgfsetdash{}{0pt}%
\pgfpathmoveto{\pgfqpoint{0.625000in}{4.034057in}}%
\pgfpathlineto{\pgfqpoint{0.625589in}{4.033333in}}%
\pgfpathlineto{\pgfqpoint{0.625000in}{4.032113in}}%
\pgfusepath{stroke}%
\end{pgfscope}%
\begin{pgfscope}%
\pgfpathrectangle{\pgfqpoint{0.625000in}{0.550000in}}{\pgfqpoint{3.875000in}{3.850000in}} %
\pgfusepath{clip}%
\pgfsetbuttcap%
\pgfsetroundjoin%
\pgfsetlinewidth{0.501875pt}%
\definecolor{currentstroke}{rgb}{0.000000,0.000000,0.000000}%
\pgfsetstrokecolor{currentstroke}%
\pgfsetdash{}{0pt}%
\pgfpathmoveto{\pgfqpoint{0.625000in}{4.103902in}}%
\pgfpathlineto{\pgfqpoint{0.630859in}{4.100877in}}%
\pgfpathlineto{\pgfqpoint{0.625503in}{4.091228in}}%
\pgfpathlineto{\pgfqpoint{0.625000in}{4.090860in}}%
\pgfusepath{stroke}%
\end{pgfscope}%
\begin{pgfscope}%
\pgfpathrectangle{\pgfqpoint{0.625000in}{0.550000in}}{\pgfqpoint{3.875000in}{3.850000in}} %
\pgfusepath{clip}%
\pgfsetbuttcap%
\pgfsetroundjoin%
\pgfsetlinewidth{0.501875pt}%
\definecolor{currentstroke}{rgb}{0.000000,0.000000,0.000000}%
\pgfsetstrokecolor{currentstroke}%
\pgfsetdash{}{0pt}%
\pgfpathmoveto{\pgfqpoint{0.625000in}{4.256202in}}%
\pgfpathlineto{\pgfqpoint{0.628239in}{4.255263in}}%
\pgfpathlineto{\pgfqpoint{0.625000in}{4.253365in}}%
\pgfusepath{stroke}%
\end{pgfscope}%
\begin{pgfscope}%
\pgfpathrectangle{\pgfqpoint{0.625000in}{0.550000in}}{\pgfqpoint{3.875000in}{3.850000in}} %
\pgfusepath{clip}%
\pgfsetbuttcap%
\pgfsetroundjoin%
\pgfsetlinewidth{0.501875pt}%
\definecolor{currentstroke}{rgb}{0.000000,0.000000,0.000000}%
\pgfsetstrokecolor{currentstroke}%
\pgfsetdash{}{0pt}%
\pgfpathmoveto{\pgfqpoint{0.634712in}{0.770790in}}%
\pgfpathlineto{\pgfqpoint{0.634076in}{0.771930in}}%
\pgfpathlineto{\pgfqpoint{0.634712in}{0.775516in}}%
\pgfpathlineto{\pgfqpoint{0.636946in}{0.771930in}}%
\pgfpathlineto{\pgfqpoint{0.634712in}{0.770790in}}%
\pgfusepath{stroke}%
\end{pgfscope}%
\begin{pgfscope}%
\pgfpathrectangle{\pgfqpoint{0.625000in}{0.550000in}}{\pgfqpoint{3.875000in}{3.850000in}} %
\pgfusepath{clip}%
\pgfsetbuttcap%
\pgfsetroundjoin%
\pgfsetlinewidth{0.501875pt}%
\definecolor{currentstroke}{rgb}{0.000000,0.000000,0.000000}%
\pgfsetstrokecolor{currentstroke}%
\pgfsetdash{}{0pt}%
\pgfpathmoveto{\pgfqpoint{0.634712in}{0.846796in}}%
\pgfpathlineto{\pgfqpoint{0.633705in}{0.849123in}}%
\pgfpathlineto{\pgfqpoint{0.634712in}{0.851806in}}%
\pgfpathlineto{\pgfqpoint{0.637448in}{0.849123in}}%
\pgfpathlineto{\pgfqpoint{0.634712in}{0.846796in}}%
\pgfusepath{stroke}%
\end{pgfscope}%
\begin{pgfscope}%
\pgfpathrectangle{\pgfqpoint{0.625000in}{0.550000in}}{\pgfqpoint{3.875000in}{3.850000in}} %
\pgfusepath{clip}%
\pgfsetbuttcap%
\pgfsetroundjoin%
\pgfsetlinewidth{0.501875pt}%
\definecolor{currentstroke}{rgb}{0.000000,0.000000,0.000000}%
\pgfsetstrokecolor{currentstroke}%
\pgfsetdash{}{0pt}%
\pgfpathmoveto{\pgfqpoint{0.654135in}{0.944882in}}%
\pgfpathlineto{\pgfqpoint{0.652673in}{0.945614in}}%
\pgfpathlineto{\pgfqpoint{0.651214in}{0.955263in}}%
\pgfpathlineto{\pgfqpoint{0.648702in}{0.964912in}}%
\pgfpathlineto{\pgfqpoint{0.644424in}{0.969851in}}%
\pgfpathlineto{\pgfqpoint{0.635765in}{0.974561in}}%
\pgfpathlineto{\pgfqpoint{0.640159in}{0.984211in}}%
\pgfpathlineto{\pgfqpoint{0.644424in}{0.988020in}}%
\pgfpathlineto{\pgfqpoint{0.651553in}{0.984211in}}%
\pgfpathlineto{\pgfqpoint{0.654135in}{0.983067in}}%
\pgfpathlineto{\pgfqpoint{0.661378in}{0.974561in}}%
\pgfpathlineto{\pgfqpoint{0.663847in}{0.965048in}}%
\pgfpathlineto{\pgfqpoint{0.663882in}{0.964912in}}%
\pgfpathlineto{\pgfqpoint{0.663847in}{0.964769in}}%
\pgfpathlineto{\pgfqpoint{0.662010in}{0.955263in}}%
\pgfpathlineto{\pgfqpoint{0.655699in}{0.945614in}}%
\pgfpathlineto{\pgfqpoint{0.654135in}{0.944882in}}%
\pgfusepath{stroke}%
\end{pgfscope}%
\begin{pgfscope}%
\pgfpathrectangle{\pgfqpoint{0.625000in}{0.550000in}}{\pgfqpoint{3.875000in}{3.850000in}} %
\pgfusepath{clip}%
\pgfsetbuttcap%
\pgfsetroundjoin%
\pgfsetlinewidth{0.501875pt}%
\definecolor{currentstroke}{rgb}{0.000000,0.000000,0.000000}%
\pgfsetstrokecolor{currentstroke}%
\pgfsetdash{}{0pt}%
\pgfpathmoveto{\pgfqpoint{0.634712in}{0.999574in}}%
\pgfpathlineto{\pgfqpoint{0.633425in}{1.003509in}}%
\pgfpathlineto{\pgfqpoint{0.634712in}{1.010023in}}%
\pgfpathlineto{\pgfqpoint{0.638823in}{1.003509in}}%
\pgfpathlineto{\pgfqpoint{0.634712in}{0.999574in}}%
\pgfusepath{stroke}%
\end{pgfscope}%
\begin{pgfscope}%
\pgfpathrectangle{\pgfqpoint{0.625000in}{0.550000in}}{\pgfqpoint{3.875000in}{3.850000in}} %
\pgfusepath{clip}%
\pgfsetbuttcap%
\pgfsetroundjoin%
\pgfsetlinewidth{0.501875pt}%
\definecolor{currentstroke}{rgb}{0.000000,0.000000,0.000000}%
\pgfsetstrokecolor{currentstroke}%
\pgfsetdash{}{0pt}%
\pgfpathmoveto{\pgfqpoint{0.654135in}{1.195660in}}%
\pgfpathlineto{\pgfqpoint{0.652363in}{1.196491in}}%
\pgfpathlineto{\pgfqpoint{0.654135in}{1.197457in}}%
\pgfpathlineto{\pgfqpoint{0.663328in}{1.196491in}}%
\pgfpathlineto{\pgfqpoint{0.654135in}{1.195660in}}%
\pgfusepath{stroke}%
\end{pgfscope}%
\begin{pgfscope}%
\pgfpathrectangle{\pgfqpoint{0.625000in}{0.550000in}}{\pgfqpoint{3.875000in}{3.850000in}} %
\pgfusepath{clip}%
\pgfsetbuttcap%
\pgfsetroundjoin%
\pgfsetlinewidth{0.501875pt}%
\definecolor{currentstroke}{rgb}{0.000000,0.000000,0.000000}%
\pgfsetstrokecolor{currentstroke}%
\pgfsetdash{}{0pt}%
\pgfpathmoveto{\pgfqpoint{0.644424in}{1.719777in}}%
\pgfpathlineto{\pgfqpoint{0.642591in}{1.727193in}}%
\pgfpathlineto{\pgfqpoint{0.644424in}{1.732382in}}%
\pgfpathlineto{\pgfqpoint{0.646687in}{1.727193in}}%
\pgfpathlineto{\pgfqpoint{0.644424in}{1.719777in}}%
\pgfusepath{stroke}%
\end{pgfscope}%
\begin{pgfscope}%
\pgfpathrectangle{\pgfqpoint{0.625000in}{0.550000in}}{\pgfqpoint{3.875000in}{3.850000in}} %
\pgfusepath{clip}%
\pgfsetbuttcap%
\pgfsetroundjoin%
\pgfsetlinewidth{0.501875pt}%
\definecolor{currentstroke}{rgb}{0.000000,0.000000,0.000000}%
\pgfsetstrokecolor{currentstroke}%
\pgfsetdash{}{0pt}%
\pgfpathmoveto{\pgfqpoint{0.634712in}{1.734171in}}%
\pgfpathlineto{\pgfqpoint{0.632555in}{1.736842in}}%
\pgfpathlineto{\pgfqpoint{0.634712in}{1.741672in}}%
\pgfpathlineto{\pgfqpoint{0.641092in}{1.736842in}}%
\pgfpathlineto{\pgfqpoint{0.634712in}{1.734171in}}%
\pgfusepath{stroke}%
\end{pgfscope}%
\begin{pgfscope}%
\pgfpathrectangle{\pgfqpoint{0.625000in}{0.550000in}}{\pgfqpoint{3.875000in}{3.850000in}} %
\pgfusepath{clip}%
\pgfsetbuttcap%
\pgfsetroundjoin%
\pgfsetlinewidth{0.501875pt}%
\definecolor{currentstroke}{rgb}{0.000000,0.000000,0.000000}%
\pgfsetstrokecolor{currentstroke}%
\pgfsetdash{}{0pt}%
\pgfpathmoveto{\pgfqpoint{0.634712in}{2.263543in}}%
\pgfpathlineto{\pgfqpoint{0.633477in}{2.267544in}}%
\pgfpathlineto{\pgfqpoint{0.634712in}{2.270243in}}%
\pgfpathlineto{\pgfqpoint{0.644424in}{2.271516in}}%
\pgfpathlineto{\pgfqpoint{0.654135in}{2.273129in}}%
\pgfpathlineto{\pgfqpoint{0.663847in}{2.274607in}}%
\pgfpathlineto{\pgfqpoint{0.667784in}{2.277193in}}%
\pgfpathlineto{\pgfqpoint{0.673559in}{2.279292in}}%
\pgfpathlineto{\pgfqpoint{0.683271in}{2.283743in}}%
\pgfpathlineto{\pgfqpoint{0.687776in}{2.286842in}}%
\pgfpathlineto{\pgfqpoint{0.692982in}{2.290154in}}%
\pgfpathlineto{\pgfqpoint{0.700899in}{2.296491in}}%
\pgfpathlineto{\pgfqpoint{0.702694in}{2.298012in}}%
\pgfpathlineto{\pgfqpoint{0.710630in}{2.306140in}}%
\pgfpathlineto{\pgfqpoint{0.712406in}{2.308160in}}%
\pgfpathlineto{\pgfqpoint{0.718163in}{2.315789in}}%
\pgfpathlineto{\pgfqpoint{0.722118in}{2.322009in}}%
\pgfpathlineto{\pgfqpoint{0.726364in}{2.325439in}}%
\pgfpathlineto{\pgfqpoint{0.728603in}{2.335088in}}%
\pgfpathlineto{\pgfqpoint{0.731830in}{2.344380in}}%
\pgfpathlineto{\pgfqpoint{0.736504in}{2.335088in}}%
\pgfpathlineto{\pgfqpoint{0.734384in}{2.325439in}}%
\pgfpathlineto{\pgfqpoint{0.731830in}{2.319940in}}%
\pgfpathlineto{\pgfqpoint{0.728441in}{2.315789in}}%
\pgfpathlineto{\pgfqpoint{0.723763in}{2.306140in}}%
\pgfpathlineto{\pgfqpoint{0.722118in}{2.303806in}}%
\pgfpathlineto{\pgfqpoint{0.716228in}{2.296491in}}%
\pgfpathlineto{\pgfqpoint{0.712406in}{2.292256in}}%
\pgfpathlineto{\pgfqpoint{0.706643in}{2.286842in}}%
\pgfpathlineto{\pgfqpoint{0.702694in}{2.283376in}}%
\pgfpathlineto{\pgfqpoint{0.694102in}{2.277193in}}%
\pgfpathlineto{\pgfqpoint{0.692982in}{2.276413in}}%
\pgfpathlineto{\pgfqpoint{0.683271in}{2.270813in}}%
\pgfpathlineto{\pgfqpoint{0.675847in}{2.267544in}}%
\pgfpathlineto{\pgfqpoint{0.673559in}{2.266525in}}%
\pgfpathlineto{\pgfqpoint{0.663847in}{2.263190in}}%
\pgfpathlineto{\pgfqpoint{0.654135in}{2.261076in}}%
\pgfpathlineto{\pgfqpoint{0.644424in}{2.260727in}}%
\pgfpathlineto{\pgfqpoint{0.634712in}{2.263543in}}%
\pgfusepath{stroke}%
\end{pgfscope}%
\begin{pgfscope}%
\pgfpathrectangle{\pgfqpoint{0.625000in}{0.550000in}}{\pgfqpoint{3.875000in}{3.850000in}} %
\pgfusepath{clip}%
\pgfsetbuttcap%
\pgfsetroundjoin%
\pgfsetlinewidth{0.501875pt}%
\definecolor{currentstroke}{rgb}{0.000000,0.000000,0.000000}%
\pgfsetstrokecolor{currentstroke}%
\pgfsetdash{}{0pt}%
\pgfpathmoveto{\pgfqpoint{0.741541in}{2.351206in}}%
\pgfpathlineto{\pgfqpoint{0.739975in}{2.354386in}}%
\pgfpathlineto{\pgfqpoint{0.739307in}{2.364035in}}%
\pgfpathlineto{\pgfqpoint{0.739704in}{2.373684in}}%
\pgfpathlineto{\pgfqpoint{0.741413in}{2.383333in}}%
\pgfpathlineto{\pgfqpoint{0.741541in}{2.383594in}}%
\pgfpathlineto{\pgfqpoint{0.741600in}{2.383333in}}%
\pgfpathlineto{\pgfqpoint{0.742660in}{2.373684in}}%
\pgfpathlineto{\pgfqpoint{0.742806in}{2.364035in}}%
\pgfpathlineto{\pgfqpoint{0.742057in}{2.354386in}}%
\pgfpathlineto{\pgfqpoint{0.741541in}{2.351206in}}%
\pgfusepath{stroke}%
\end{pgfscope}%
\begin{pgfscope}%
\pgfpathrectangle{\pgfqpoint{0.625000in}{0.550000in}}{\pgfqpoint{3.875000in}{3.850000in}} %
\pgfusepath{clip}%
\pgfsetbuttcap%
\pgfsetroundjoin%
\pgfsetlinewidth{0.501875pt}%
\definecolor{currentstroke}{rgb}{0.000000,0.000000,0.000000}%
\pgfsetstrokecolor{currentstroke}%
\pgfsetdash{}{0pt}%
\pgfpathmoveto{\pgfqpoint{0.654135in}{2.372258in}}%
\pgfpathlineto{\pgfqpoint{0.651677in}{2.373684in}}%
\pgfpathlineto{\pgfqpoint{0.654135in}{2.374738in}}%
\pgfpathlineto{\pgfqpoint{0.656574in}{2.373684in}}%
\pgfpathlineto{\pgfqpoint{0.654135in}{2.372258in}}%
\pgfusepath{stroke}%
\end{pgfscope}%
\begin{pgfscope}%
\pgfpathrectangle{\pgfqpoint{0.625000in}{0.550000in}}{\pgfqpoint{3.875000in}{3.850000in}} %
\pgfusepath{clip}%
\pgfsetbuttcap%
\pgfsetroundjoin%
\pgfsetlinewidth{0.501875pt}%
\definecolor{currentstroke}{rgb}{0.000000,0.000000,0.000000}%
\pgfsetstrokecolor{currentstroke}%
\pgfsetdash{}{0pt}%
\pgfpathmoveto{\pgfqpoint{0.731830in}{2.402561in}}%
\pgfpathlineto{\pgfqpoint{0.731815in}{2.402632in}}%
\pgfpathlineto{\pgfqpoint{0.730458in}{2.412281in}}%
\pgfpathlineto{\pgfqpoint{0.731830in}{2.414469in}}%
\pgfpathlineto{\pgfqpoint{0.733093in}{2.412281in}}%
\pgfpathlineto{\pgfqpoint{0.731867in}{2.402632in}}%
\pgfpathlineto{\pgfqpoint{0.731830in}{2.402561in}}%
\pgfusepath{stroke}%
\end{pgfscope}%
\begin{pgfscope}%
\pgfpathrectangle{\pgfqpoint{0.625000in}{0.550000in}}{\pgfqpoint{3.875000in}{3.850000in}} %
\pgfusepath{clip}%
\pgfsetbuttcap%
\pgfsetroundjoin%
\pgfsetlinewidth{0.501875pt}%
\definecolor{currentstroke}{rgb}{0.000000,0.000000,0.000000}%
\pgfsetstrokecolor{currentstroke}%
\pgfsetdash{}{0pt}%
\pgfpathmoveto{\pgfqpoint{0.683271in}{2.409045in}}%
\pgfpathlineto{\pgfqpoint{0.682525in}{2.412281in}}%
\pgfpathlineto{\pgfqpoint{0.682996in}{2.421930in}}%
\pgfpathlineto{\pgfqpoint{0.682705in}{2.431579in}}%
\pgfpathlineto{\pgfqpoint{0.683271in}{2.433642in}}%
\pgfpathlineto{\pgfqpoint{0.683960in}{2.431579in}}%
\pgfpathlineto{\pgfqpoint{0.683644in}{2.421930in}}%
\pgfpathlineto{\pgfqpoint{0.684051in}{2.412281in}}%
\pgfpathlineto{\pgfqpoint{0.683271in}{2.409045in}}%
\pgfusepath{stroke}%
\end{pgfscope}%
\begin{pgfscope}%
\pgfpathrectangle{\pgfqpoint{0.625000in}{0.550000in}}{\pgfqpoint{3.875000in}{3.850000in}} %
\pgfusepath{clip}%
\pgfsetbuttcap%
\pgfsetroundjoin%
\pgfsetlinewidth{0.501875pt}%
\definecolor{currentstroke}{rgb}{0.000000,0.000000,0.000000}%
\pgfsetstrokecolor{currentstroke}%
\pgfsetdash{}{0pt}%
\pgfpathmoveto{\pgfqpoint{0.663847in}{2.431087in}}%
\pgfpathlineto{\pgfqpoint{0.663626in}{2.431579in}}%
\pgfpathlineto{\pgfqpoint{0.663847in}{2.433685in}}%
\pgfpathlineto{\pgfqpoint{0.664051in}{2.431579in}}%
\pgfpathlineto{\pgfqpoint{0.663847in}{2.431087in}}%
\pgfusepath{stroke}%
\end{pgfscope}%
\begin{pgfscope}%
\pgfpathrectangle{\pgfqpoint{0.625000in}{0.550000in}}{\pgfqpoint{3.875000in}{3.850000in}} %
\pgfusepath{clip}%
\pgfsetbuttcap%
\pgfsetroundjoin%
\pgfsetlinewidth{0.501875pt}%
\definecolor{currentstroke}{rgb}{0.000000,0.000000,0.000000}%
\pgfsetstrokecolor{currentstroke}%
\pgfsetdash{}{0pt}%
\pgfpathmoveto{\pgfqpoint{0.712406in}{2.439186in}}%
\pgfpathlineto{\pgfqpoint{0.711074in}{2.441228in}}%
\pgfpathlineto{\pgfqpoint{0.712406in}{2.442696in}}%
\pgfpathlineto{\pgfqpoint{0.714098in}{2.441228in}}%
\pgfpathlineto{\pgfqpoint{0.712406in}{2.439186in}}%
\pgfusepath{stroke}%
\end{pgfscope}%
\begin{pgfscope}%
\pgfpathrectangle{\pgfqpoint{0.625000in}{0.550000in}}{\pgfqpoint{3.875000in}{3.850000in}} %
\pgfusepath{clip}%
\pgfsetbuttcap%
\pgfsetroundjoin%
\pgfsetlinewidth{0.501875pt}%
\definecolor{currentstroke}{rgb}{0.000000,0.000000,0.000000}%
\pgfsetstrokecolor{currentstroke}%
\pgfsetdash{}{0pt}%
\pgfpathmoveto{\pgfqpoint{0.702694in}{2.449507in}}%
\pgfpathlineto{\pgfqpoint{0.701569in}{2.450877in}}%
\pgfpathlineto{\pgfqpoint{0.702694in}{2.451903in}}%
\pgfpathlineto{\pgfqpoint{0.704224in}{2.450877in}}%
\pgfpathlineto{\pgfqpoint{0.702694in}{2.449507in}}%
\pgfusepath{stroke}%
\end{pgfscope}%
\begin{pgfscope}%
\pgfpathrectangle{\pgfqpoint{0.625000in}{0.550000in}}{\pgfqpoint{3.875000in}{3.850000in}} %
\pgfusepath{clip}%
\pgfsetbuttcap%
\pgfsetroundjoin%
\pgfsetlinewidth{0.501875pt}%
\definecolor{currentstroke}{rgb}{0.000000,0.000000,0.000000}%
\pgfsetstrokecolor{currentstroke}%
\pgfsetdash{}{0pt}%
\pgfpathmoveto{\pgfqpoint{0.634712in}{2.469203in}}%
\pgfpathlineto{\pgfqpoint{0.634567in}{2.470175in}}%
\pgfpathlineto{\pgfqpoint{0.634712in}{2.470472in}}%
\pgfpathlineto{\pgfqpoint{0.637327in}{2.470175in}}%
\pgfpathlineto{\pgfqpoint{0.634712in}{2.469203in}}%
\pgfusepath{stroke}%
\end{pgfscope}%
\begin{pgfscope}%
\pgfpathrectangle{\pgfqpoint{0.625000in}{0.550000in}}{\pgfqpoint{3.875000in}{3.850000in}} %
\pgfusepath{clip}%
\pgfsetbuttcap%
\pgfsetroundjoin%
\pgfsetlinewidth{0.501875pt}%
\definecolor{currentstroke}{rgb}{0.000000,0.000000,0.000000}%
\pgfsetstrokecolor{currentstroke}%
\pgfsetdash{}{0pt}%
\pgfpathmoveto{\pgfqpoint{0.702694in}{2.507746in}}%
\pgfpathlineto{\pgfqpoint{0.701569in}{2.508772in}}%
\pgfpathlineto{\pgfqpoint{0.702694in}{2.510142in}}%
\pgfpathlineto{\pgfqpoint{0.704224in}{2.508772in}}%
\pgfpathlineto{\pgfqpoint{0.702694in}{2.507746in}}%
\pgfusepath{stroke}%
\end{pgfscope}%
\begin{pgfscope}%
\pgfpathrectangle{\pgfqpoint{0.625000in}{0.550000in}}{\pgfqpoint{3.875000in}{3.850000in}} %
\pgfusepath{clip}%
\pgfsetbuttcap%
\pgfsetroundjoin%
\pgfsetlinewidth{0.501875pt}%
\definecolor{currentstroke}{rgb}{0.000000,0.000000,0.000000}%
\pgfsetstrokecolor{currentstroke}%
\pgfsetdash{}{0pt}%
\pgfpathmoveto{\pgfqpoint{0.712406in}{2.516953in}}%
\pgfpathlineto{\pgfqpoint{0.711074in}{2.518421in}}%
\pgfpathlineto{\pgfqpoint{0.712406in}{2.520463in}}%
\pgfpathlineto{\pgfqpoint{0.714098in}{2.518421in}}%
\pgfpathlineto{\pgfqpoint{0.712406in}{2.516953in}}%
\pgfusepath{stroke}%
\end{pgfscope}%
\begin{pgfscope}%
\pgfpathrectangle{\pgfqpoint{0.625000in}{0.550000in}}{\pgfqpoint{3.875000in}{3.850000in}} %
\pgfusepath{clip}%
\pgfsetbuttcap%
\pgfsetroundjoin%
\pgfsetlinewidth{0.501875pt}%
\definecolor{currentstroke}{rgb}{0.000000,0.000000,0.000000}%
\pgfsetstrokecolor{currentstroke}%
\pgfsetdash{}{0pt}%
\pgfpathmoveto{\pgfqpoint{0.663847in}{2.525964in}}%
\pgfpathlineto{\pgfqpoint{0.663626in}{2.528070in}}%
\pgfpathlineto{\pgfqpoint{0.663847in}{2.528562in}}%
\pgfpathlineto{\pgfqpoint{0.664051in}{2.528070in}}%
\pgfpathlineto{\pgfqpoint{0.663847in}{2.525964in}}%
\pgfusepath{stroke}%
\end{pgfscope}%
\begin{pgfscope}%
\pgfpathrectangle{\pgfqpoint{0.625000in}{0.550000in}}{\pgfqpoint{3.875000in}{3.850000in}} %
\pgfusepath{clip}%
\pgfsetbuttcap%
\pgfsetroundjoin%
\pgfsetlinewidth{0.501875pt}%
\definecolor{currentstroke}{rgb}{0.000000,0.000000,0.000000}%
\pgfsetstrokecolor{currentstroke}%
\pgfsetdash{}{0pt}%
\pgfpathmoveto{\pgfqpoint{0.683271in}{2.526007in}}%
\pgfpathlineto{\pgfqpoint{0.682705in}{2.528070in}}%
\pgfpathlineto{\pgfqpoint{0.682996in}{2.537719in}}%
\pgfpathlineto{\pgfqpoint{0.682525in}{2.547368in}}%
\pgfpathlineto{\pgfqpoint{0.683271in}{2.550604in}}%
\pgfpathlineto{\pgfqpoint{0.684051in}{2.547368in}}%
\pgfpathlineto{\pgfqpoint{0.683644in}{2.537719in}}%
\pgfpathlineto{\pgfqpoint{0.683960in}{2.528070in}}%
\pgfpathlineto{\pgfqpoint{0.683271in}{2.526007in}}%
\pgfusepath{stroke}%
\end{pgfscope}%
\begin{pgfscope}%
\pgfpathrectangle{\pgfqpoint{0.625000in}{0.550000in}}{\pgfqpoint{3.875000in}{3.850000in}} %
\pgfusepath{clip}%
\pgfsetbuttcap%
\pgfsetroundjoin%
\pgfsetlinewidth{0.501875pt}%
\definecolor{currentstroke}{rgb}{0.000000,0.000000,0.000000}%
\pgfsetstrokecolor{currentstroke}%
\pgfsetdash{}{0pt}%
\pgfpathmoveto{\pgfqpoint{0.731830in}{2.545181in}}%
\pgfpathlineto{\pgfqpoint{0.730458in}{2.547368in}}%
\pgfpathlineto{\pgfqpoint{0.731815in}{2.557018in}}%
\pgfpathlineto{\pgfqpoint{0.731830in}{2.557088in}}%
\pgfpathlineto{\pgfqpoint{0.731867in}{2.557018in}}%
\pgfpathlineto{\pgfqpoint{0.733093in}{2.547368in}}%
\pgfpathlineto{\pgfqpoint{0.731830in}{2.545181in}}%
\pgfusepath{stroke}%
\end{pgfscope}%
\begin{pgfscope}%
\pgfpathrectangle{\pgfqpoint{0.625000in}{0.550000in}}{\pgfqpoint{3.875000in}{3.850000in}} %
\pgfusepath{clip}%
\pgfsetbuttcap%
\pgfsetroundjoin%
\pgfsetlinewidth{0.501875pt}%
\definecolor{currentstroke}{rgb}{0.000000,0.000000,0.000000}%
\pgfsetstrokecolor{currentstroke}%
\pgfsetdash{}{0pt}%
\pgfpathmoveto{\pgfqpoint{0.741541in}{2.576055in}}%
\pgfpathlineto{\pgfqpoint{0.741413in}{2.576316in}}%
\pgfpathlineto{\pgfqpoint{0.739704in}{2.585965in}}%
\pgfpathlineto{\pgfqpoint{0.739307in}{2.595614in}}%
\pgfpathlineto{\pgfqpoint{0.739975in}{2.605263in}}%
\pgfpathlineto{\pgfqpoint{0.741541in}{2.608443in}}%
\pgfpathlineto{\pgfqpoint{0.742057in}{2.605263in}}%
\pgfpathlineto{\pgfqpoint{0.742806in}{2.595614in}}%
\pgfpathlineto{\pgfqpoint{0.742660in}{2.585965in}}%
\pgfpathlineto{\pgfqpoint{0.741600in}{2.576316in}}%
\pgfpathlineto{\pgfqpoint{0.741541in}{2.576055in}}%
\pgfusepath{stroke}%
\end{pgfscope}%
\begin{pgfscope}%
\pgfpathrectangle{\pgfqpoint{0.625000in}{0.550000in}}{\pgfqpoint{3.875000in}{3.850000in}} %
\pgfusepath{clip}%
\pgfsetbuttcap%
\pgfsetroundjoin%
\pgfsetlinewidth{0.501875pt}%
\definecolor{currentstroke}{rgb}{0.000000,0.000000,0.000000}%
\pgfsetstrokecolor{currentstroke}%
\pgfsetdash{}{0pt}%
\pgfpathmoveto{\pgfqpoint{0.654135in}{2.584911in}}%
\pgfpathlineto{\pgfqpoint{0.651677in}{2.585965in}}%
\pgfpathlineto{\pgfqpoint{0.654135in}{2.587391in}}%
\pgfpathlineto{\pgfqpoint{0.656574in}{2.585965in}}%
\pgfpathlineto{\pgfqpoint{0.654135in}{2.584911in}}%
\pgfusepath{stroke}%
\end{pgfscope}%
\begin{pgfscope}%
\pgfpathrectangle{\pgfqpoint{0.625000in}{0.550000in}}{\pgfqpoint{3.875000in}{3.850000in}} %
\pgfusepath{clip}%
\pgfsetbuttcap%
\pgfsetroundjoin%
\pgfsetlinewidth{0.501875pt}%
\definecolor{currentstroke}{rgb}{0.000000,0.000000,0.000000}%
\pgfsetstrokecolor{currentstroke}%
\pgfsetdash{}{0pt}%
\pgfpathmoveto{\pgfqpoint{0.731830in}{2.615269in}}%
\pgfpathlineto{\pgfqpoint{0.728603in}{2.624561in}}%
\pgfpathlineto{\pgfqpoint{0.726364in}{2.634211in}}%
\pgfpathlineto{\pgfqpoint{0.722118in}{2.637640in}}%
\pgfpathlineto{\pgfqpoint{0.718163in}{2.643860in}}%
\pgfpathlineto{\pgfqpoint{0.712406in}{2.651489in}}%
\pgfpathlineto{\pgfqpoint{0.710630in}{2.653509in}}%
\pgfpathlineto{\pgfqpoint{0.702694in}{2.661637in}}%
\pgfpathlineto{\pgfqpoint{0.700899in}{2.663158in}}%
\pgfpathlineto{\pgfqpoint{0.692982in}{2.669495in}}%
\pgfpathlineto{\pgfqpoint{0.687776in}{2.672807in}}%
\pgfpathlineto{\pgfqpoint{0.683271in}{2.675906in}}%
\pgfpathlineto{\pgfqpoint{0.673559in}{2.680357in}}%
\pgfpathlineto{\pgfqpoint{0.667784in}{2.682456in}}%
\pgfpathlineto{\pgfqpoint{0.663847in}{2.685042in}}%
\pgfpathlineto{\pgfqpoint{0.654135in}{2.686520in}}%
\pgfpathlineto{\pgfqpoint{0.644424in}{2.688134in}}%
\pgfpathlineto{\pgfqpoint{0.634712in}{2.689406in}}%
\pgfpathlineto{\pgfqpoint{0.633477in}{2.692105in}}%
\pgfpathlineto{\pgfqpoint{0.634712in}{2.696106in}}%
\pgfpathlineto{\pgfqpoint{0.644424in}{2.698922in}}%
\pgfpathlineto{\pgfqpoint{0.654135in}{2.698573in}}%
\pgfpathlineto{\pgfqpoint{0.663847in}{2.696460in}}%
\pgfpathlineto{\pgfqpoint{0.673559in}{2.693124in}}%
\pgfpathlineto{\pgfqpoint{0.675847in}{2.692105in}}%
\pgfpathlineto{\pgfqpoint{0.683271in}{2.688837in}}%
\pgfpathlineto{\pgfqpoint{0.692982in}{2.683236in}}%
\pgfpathlineto{\pgfqpoint{0.694102in}{2.682456in}}%
\pgfpathlineto{\pgfqpoint{0.702694in}{2.676273in}}%
\pgfpathlineto{\pgfqpoint{0.706643in}{2.672807in}}%
\pgfpathlineto{\pgfqpoint{0.712406in}{2.667393in}}%
\pgfpathlineto{\pgfqpoint{0.716228in}{2.663158in}}%
\pgfpathlineto{\pgfqpoint{0.722118in}{2.655843in}}%
\pgfpathlineto{\pgfqpoint{0.723763in}{2.653509in}}%
\pgfpathlineto{\pgfqpoint{0.728441in}{2.643860in}}%
\pgfpathlineto{\pgfqpoint{0.731830in}{2.639709in}}%
\pgfpathlineto{\pgfqpoint{0.734384in}{2.634211in}}%
\pgfpathlineto{\pgfqpoint{0.736504in}{2.624561in}}%
\pgfpathlineto{\pgfqpoint{0.731830in}{2.615269in}}%
\pgfusepath{stroke}%
\end{pgfscope}%
\begin{pgfscope}%
\pgfpathrectangle{\pgfqpoint{0.625000in}{0.550000in}}{\pgfqpoint{3.875000in}{3.850000in}} %
\pgfusepath{clip}%
\pgfsetbuttcap%
\pgfsetroundjoin%
\pgfsetlinewidth{0.501875pt}%
\definecolor{currentstroke}{rgb}{0.000000,0.000000,0.000000}%
\pgfsetstrokecolor{currentstroke}%
\pgfsetdash{}{0pt}%
\pgfpathmoveto{\pgfqpoint{0.634712in}{3.217977in}}%
\pgfpathlineto{\pgfqpoint{0.632555in}{3.222807in}}%
\pgfpathlineto{\pgfqpoint{0.634712in}{3.225478in}}%
\pgfpathlineto{\pgfqpoint{0.641092in}{3.222807in}}%
\pgfpathlineto{\pgfqpoint{0.634712in}{3.217977in}}%
\pgfusepath{stroke}%
\end{pgfscope}%
\begin{pgfscope}%
\pgfpathrectangle{\pgfqpoint{0.625000in}{0.550000in}}{\pgfqpoint{3.875000in}{3.850000in}} %
\pgfusepath{clip}%
\pgfsetbuttcap%
\pgfsetroundjoin%
\pgfsetlinewidth{0.501875pt}%
\definecolor{currentstroke}{rgb}{0.000000,0.000000,0.000000}%
\pgfsetstrokecolor{currentstroke}%
\pgfsetdash{}{0pt}%
\pgfpathmoveto{\pgfqpoint{0.644424in}{3.227268in}}%
\pgfpathlineto{\pgfqpoint{0.642591in}{3.232456in}}%
\pgfpathlineto{\pgfqpoint{0.644424in}{3.239872in}}%
\pgfpathlineto{\pgfqpoint{0.646687in}{3.232456in}}%
\pgfpathlineto{\pgfqpoint{0.644424in}{3.227268in}}%
\pgfusepath{stroke}%
\end{pgfscope}%
\begin{pgfscope}%
\pgfpathrectangle{\pgfqpoint{0.625000in}{0.550000in}}{\pgfqpoint{3.875000in}{3.850000in}} %
\pgfusepath{clip}%
\pgfsetbuttcap%
\pgfsetroundjoin%
\pgfsetlinewidth{0.501875pt}%
\definecolor{currentstroke}{rgb}{0.000000,0.000000,0.000000}%
\pgfsetstrokecolor{currentstroke}%
\pgfsetdash{}{0pt}%
\pgfpathmoveto{\pgfqpoint{0.654135in}{3.762193in}}%
\pgfpathlineto{\pgfqpoint{0.652363in}{3.763158in}}%
\pgfpathlineto{\pgfqpoint{0.654135in}{3.763989in}}%
\pgfpathlineto{\pgfqpoint{0.663328in}{3.763158in}}%
\pgfpathlineto{\pgfqpoint{0.654135in}{3.762193in}}%
\pgfusepath{stroke}%
\end{pgfscope}%
\begin{pgfscope}%
\pgfpathrectangle{\pgfqpoint{0.625000in}{0.550000in}}{\pgfqpoint{3.875000in}{3.850000in}} %
\pgfusepath{clip}%
\pgfsetbuttcap%
\pgfsetroundjoin%
\pgfsetlinewidth{0.501875pt}%
\definecolor{currentstroke}{rgb}{0.000000,0.000000,0.000000}%
\pgfsetstrokecolor{currentstroke}%
\pgfsetdash{}{0pt}%
\pgfpathmoveto{\pgfqpoint{0.634712in}{3.949626in}}%
\pgfpathlineto{\pgfqpoint{0.633425in}{3.956140in}}%
\pgfpathlineto{\pgfqpoint{0.634712in}{3.960075in}}%
\pgfpathlineto{\pgfqpoint{0.638823in}{3.956140in}}%
\pgfpathlineto{\pgfqpoint{0.634712in}{3.949626in}}%
\pgfusepath{stroke}%
\end{pgfscope}%
\begin{pgfscope}%
\pgfpathrectangle{\pgfqpoint{0.625000in}{0.550000in}}{\pgfqpoint{3.875000in}{3.850000in}} %
\pgfusepath{clip}%
\pgfsetbuttcap%
\pgfsetroundjoin%
\pgfsetlinewidth{0.501875pt}%
\definecolor{currentstroke}{rgb}{0.000000,0.000000,0.000000}%
\pgfsetstrokecolor{currentstroke}%
\pgfsetdash{}{0pt}%
\pgfpathmoveto{\pgfqpoint{0.644424in}{3.971629in}}%
\pgfpathlineto{\pgfqpoint{0.640159in}{3.975439in}}%
\pgfpathlineto{\pgfqpoint{0.635765in}{3.985088in}}%
\pgfpathlineto{\pgfqpoint{0.644424in}{3.989798in}}%
\pgfpathlineto{\pgfqpoint{0.648702in}{3.994737in}}%
\pgfpathlineto{\pgfqpoint{0.651214in}{4.004386in}}%
\pgfpathlineto{\pgfqpoint{0.652673in}{4.014035in}}%
\pgfpathlineto{\pgfqpoint{0.654135in}{4.014767in}}%
\pgfpathlineto{\pgfqpoint{0.655699in}{4.014035in}}%
\pgfpathlineto{\pgfqpoint{0.662010in}{4.004386in}}%
\pgfpathlineto{\pgfqpoint{0.663847in}{3.994880in}}%
\pgfpathlineto{\pgfqpoint{0.663882in}{3.994737in}}%
\pgfpathlineto{\pgfqpoint{0.663847in}{3.994601in}}%
\pgfpathlineto{\pgfqpoint{0.661378in}{3.985088in}}%
\pgfpathlineto{\pgfqpoint{0.654135in}{3.976582in}}%
\pgfpathlineto{\pgfqpoint{0.651553in}{3.975439in}}%
\pgfpathlineto{\pgfqpoint{0.644424in}{3.971629in}}%
\pgfusepath{stroke}%
\end{pgfscope}%
\begin{pgfscope}%
\pgfpathrectangle{\pgfqpoint{0.625000in}{0.550000in}}{\pgfqpoint{3.875000in}{3.850000in}} %
\pgfusepath{clip}%
\pgfsetbuttcap%
\pgfsetroundjoin%
\pgfsetlinewidth{0.501875pt}%
\definecolor{currentstroke}{rgb}{0.000000,0.000000,0.000000}%
\pgfsetstrokecolor{currentstroke}%
\pgfsetdash{}{0pt}%
\pgfpathmoveto{\pgfqpoint{0.634712in}{4.107843in}}%
\pgfpathlineto{\pgfqpoint{0.633705in}{4.110526in}}%
\pgfpathlineto{\pgfqpoint{0.634712in}{4.112853in}}%
\pgfpathlineto{\pgfqpoint{0.637448in}{4.110526in}}%
\pgfpathlineto{\pgfqpoint{0.634712in}{4.107843in}}%
\pgfusepath{stroke}%
\end{pgfscope}%
\begin{pgfscope}%
\pgfpathrectangle{\pgfqpoint{0.625000in}{0.550000in}}{\pgfqpoint{3.875000in}{3.850000in}} %
\pgfusepath{clip}%
\pgfsetbuttcap%
\pgfsetroundjoin%
\pgfsetlinewidth{0.501875pt}%
\definecolor{currentstroke}{rgb}{0.000000,0.000000,0.000000}%
\pgfsetstrokecolor{currentstroke}%
\pgfsetdash{}{0pt}%
\pgfpathmoveto{\pgfqpoint{0.634712in}{4.184133in}}%
\pgfpathlineto{\pgfqpoint{0.634076in}{4.187719in}}%
\pgfpathlineto{\pgfqpoint{0.634712in}{4.188859in}}%
\pgfpathlineto{\pgfqpoint{0.636946in}{4.187719in}}%
\pgfpathlineto{\pgfqpoint{0.634712in}{4.184133in}}%
\pgfusepath{stroke}%
\end{pgfscope}%
\begin{pgfscope}%
\pgfpathrectangle{\pgfqpoint{0.625000in}{0.550000in}}{\pgfqpoint{3.875000in}{3.850000in}} %
\pgfusepath{clip}%
\pgfsetbuttcap%
\pgfsetroundjoin%
\pgfsetlinewidth{0.501875pt}%
\definecolor{currentstroke}{rgb}{0.000000,0.000000,0.000000}%
\pgfsetstrokecolor{currentstroke}%
\pgfsetdash{}{0pt}%
\pgfpathmoveto{\pgfqpoint{0.625000in}{0.557711in}}%
\pgfpathlineto{\pgfqpoint{0.629174in}{0.559649in}}%
\pgfpathlineto{\pgfqpoint{0.761937in}{0.878070in}}%
\pgfpathlineto{\pgfqpoint{0.769700in}{0.907018in}}%
\pgfpathlineto{\pgfqpoint{0.774917in}{0.926316in}}%
\pgfpathlineto{\pgfqpoint{0.780388in}{0.938810in}}%
\pgfpathlineto{\pgfqpoint{0.784230in}{0.945614in}}%
\pgfpathlineto{\pgfqpoint{0.799812in}{0.965509in}}%
\pgfpathlineto{\pgfqpoint{0.822287in}{0.993860in}}%
\pgfpathlineto{\pgfqpoint{0.834598in}{1.013158in}}%
\pgfpathlineto{\pgfqpoint{0.844040in}{1.032456in}}%
\pgfpathlineto{\pgfqpoint{0.850610in}{1.051754in}}%
\pgfpathlineto{\pgfqpoint{0.854326in}{1.071053in}}%
\pgfpathlineto{\pgfqpoint{0.854899in}{1.090351in}}%
\pgfpathlineto{\pgfqpoint{0.851841in}{1.109649in}}%
\pgfpathlineto{\pgfqpoint{0.844071in}{1.128947in}}%
\pgfpathlineto{\pgfqpoint{0.837667in}{1.138596in}}%
\pgfpathlineto{\pgfqpoint{0.828947in}{1.148319in}}%
\pgfpathlineto{\pgfqpoint{0.817420in}{1.157895in}}%
\pgfpathlineto{\pgfqpoint{0.799812in}{1.168005in}}%
\pgfpathlineto{\pgfqpoint{0.776238in}{1.177193in}}%
\pgfpathlineto{\pgfqpoint{0.751253in}{1.183383in}}%
\pgfpathlineto{\pgfqpoint{0.722118in}{1.188173in}}%
\pgfpathlineto{\pgfqpoint{0.683271in}{1.191663in}}%
\pgfpathlineto{\pgfqpoint{0.654135in}{1.193271in}}%
\pgfpathlineto{\pgfqpoint{0.647274in}{1.196491in}}%
\pgfpathlineto{\pgfqpoint{0.664689in}{1.206140in}}%
\pgfpathlineto{\pgfqpoint{0.673761in}{1.215789in}}%
\pgfpathlineto{\pgfqpoint{0.679473in}{1.225439in}}%
\pgfpathlineto{\pgfqpoint{0.681836in}{1.235088in}}%
\pgfpathlineto{\pgfqpoint{0.682054in}{1.244737in}}%
\pgfpathlineto{\pgfqpoint{0.680045in}{1.254386in}}%
\pgfpathlineto{\pgfqpoint{0.673559in}{1.266247in}}%
\pgfpathlineto{\pgfqpoint{0.663847in}{1.275608in}}%
\pgfpathlineto{\pgfqpoint{0.654135in}{1.280688in}}%
\pgfpathlineto{\pgfqpoint{0.644424in}{1.281764in}}%
\pgfpathlineto{\pgfqpoint{0.634712in}{1.278544in}}%
\pgfpathlineto{\pgfqpoint{0.629892in}{1.283333in}}%
\pgfpathlineto{\pgfqpoint{0.633858in}{1.292982in}}%
\pgfpathlineto{\pgfqpoint{0.633692in}{1.302632in}}%
\pgfpathlineto{\pgfqpoint{0.625000in}{1.306507in}}%
\pgfpathlineto{\pgfqpoint{0.625000in}{1.306507in}}%
\pgfusepath{stroke}%
\end{pgfscope}%
\begin{pgfscope}%
\pgfpathrectangle{\pgfqpoint{0.625000in}{0.550000in}}{\pgfqpoint{3.875000in}{3.850000in}} %
\pgfusepath{clip}%
\pgfsetbuttcap%
\pgfsetroundjoin%
\pgfsetlinewidth{0.501875pt}%
\definecolor{currentstroke}{rgb}{0.000000,0.000000,0.000000}%
\pgfsetstrokecolor{currentstroke}%
\pgfsetdash{}{0pt}%
\pgfpathmoveto{\pgfqpoint{0.625000in}{0.617985in}}%
\pgfpathlineto{\pgfqpoint{0.629622in}{0.617544in}}%
\pgfpathlineto{\pgfqpoint{0.625000in}{0.616960in}}%
\pgfusepath{stroke}%
\end{pgfscope}%
\begin{pgfscope}%
\pgfpathrectangle{\pgfqpoint{0.625000in}{0.550000in}}{\pgfqpoint{3.875000in}{3.850000in}} %
\pgfusepath{clip}%
\pgfsetbuttcap%
\pgfsetroundjoin%
\pgfsetlinewidth{0.501875pt}%
\definecolor{currentstroke}{rgb}{0.000000,0.000000,0.000000}%
\pgfsetstrokecolor{currentstroke}%
\pgfsetdash{}{0pt}%
\pgfpathmoveto{\pgfqpoint{0.625000in}{0.641079in}}%
\pgfpathlineto{\pgfqpoint{0.630063in}{0.636842in}}%
\pgfpathlineto{\pgfqpoint{0.625000in}{0.636256in}}%
\pgfusepath{stroke}%
\end{pgfscope}%
\begin{pgfscope}%
\pgfpathrectangle{\pgfqpoint{0.625000in}{0.550000in}}{\pgfqpoint{3.875000in}{3.850000in}} %
\pgfusepath{clip}%
\pgfsetbuttcap%
\pgfsetroundjoin%
\pgfsetlinewidth{0.501875pt}%
\definecolor{currentstroke}{rgb}{0.000000,0.000000,0.000000}%
\pgfsetstrokecolor{currentstroke}%
\pgfsetdash{}{0pt}%
\pgfpathmoveto{\pgfqpoint{0.625000in}{0.661981in}}%
\pgfpathlineto{\pgfqpoint{0.631485in}{0.656140in}}%
\pgfpathlineto{\pgfqpoint{0.625000in}{0.650300in}}%
\pgfusepath{stroke}%
\end{pgfscope}%
\begin{pgfscope}%
\pgfpathrectangle{\pgfqpoint{0.625000in}{0.550000in}}{\pgfqpoint{3.875000in}{3.850000in}} %
\pgfusepath{clip}%
\pgfsetbuttcap%
\pgfsetroundjoin%
\pgfsetlinewidth{0.501875pt}%
\definecolor{currentstroke}{rgb}{0.000000,0.000000,0.000000}%
\pgfsetstrokecolor{currentstroke}%
\pgfsetdash{}{0pt}%
\pgfpathmoveto{\pgfqpoint{0.625000in}{0.710081in}}%
\pgfpathlineto{\pgfqpoint{0.634712in}{0.704395in}}%
\pgfpathlineto{\pgfqpoint{0.634736in}{0.704386in}}%
\pgfpathlineto{\pgfqpoint{0.634712in}{0.704375in}}%
\pgfpathlineto{\pgfqpoint{0.625000in}{0.701569in}}%
\pgfusepath{stroke}%
\end{pgfscope}%
\begin{pgfscope}%
\pgfpathrectangle{\pgfqpoint{0.625000in}{0.550000in}}{\pgfqpoint{3.875000in}{3.850000in}} %
\pgfusepath{clip}%
\pgfsetbuttcap%
\pgfsetroundjoin%
\pgfsetlinewidth{0.501875pt}%
\definecolor{currentstroke}{rgb}{0.000000,0.000000,0.000000}%
\pgfsetstrokecolor{currentstroke}%
\pgfsetdash{}{0pt}%
\pgfpathmoveto{\pgfqpoint{0.625000in}{0.736862in}}%
\pgfpathlineto{\pgfqpoint{0.634712in}{0.737786in}}%
\pgfpathlineto{\pgfqpoint{0.641187in}{0.733333in}}%
\pgfpathlineto{\pgfqpoint{0.634712in}{0.729587in}}%
\pgfpathlineto{\pgfqpoint{0.625000in}{0.729804in}}%
\pgfusepath{stroke}%
\end{pgfscope}%
\begin{pgfscope}%
\pgfpathrectangle{\pgfqpoint{0.625000in}{0.550000in}}{\pgfqpoint{3.875000in}{3.850000in}} %
\pgfusepath{clip}%
\pgfsetbuttcap%
\pgfsetroundjoin%
\pgfsetlinewidth{0.501875pt}%
\definecolor{currentstroke}{rgb}{0.000000,0.000000,0.000000}%
\pgfsetstrokecolor{currentstroke}%
\pgfsetdash{}{0pt}%
\pgfpathmoveto{\pgfqpoint{0.625000in}{0.836196in}}%
\pgfpathlineto{\pgfqpoint{0.629409in}{0.839474in}}%
\pgfpathlineto{\pgfqpoint{0.625754in}{0.849123in}}%
\pgfpathlineto{\pgfqpoint{0.625000in}{0.849697in}}%
\pgfusepath{stroke}%
\end{pgfscope}%
\begin{pgfscope}%
\pgfpathrectangle{\pgfqpoint{0.625000in}{0.550000in}}{\pgfqpoint{3.875000in}{3.850000in}} %
\pgfusepath{clip}%
\pgfsetbuttcap%
\pgfsetroundjoin%
\pgfsetlinewidth{0.501875pt}%
\definecolor{currentstroke}{rgb}{0.000000,0.000000,0.000000}%
\pgfsetstrokecolor{currentstroke}%
\pgfsetdash{}{0pt}%
\pgfpathmoveto{\pgfqpoint{0.625000in}{0.874977in}}%
\pgfpathlineto{\pgfqpoint{0.633949in}{0.868421in}}%
\pgfpathlineto{\pgfqpoint{0.634712in}{0.867276in}}%
\pgfpathlineto{\pgfqpoint{0.644424in}{0.863263in}}%
\pgfpathlineto{\pgfqpoint{0.653874in}{0.858772in}}%
\pgfpathlineto{\pgfqpoint{0.654135in}{0.858494in}}%
\pgfpathlineto{\pgfqpoint{0.663193in}{0.849123in}}%
\pgfpathlineto{\pgfqpoint{0.663749in}{0.839474in}}%
\pgfpathlineto{\pgfqpoint{0.654135in}{0.830277in}}%
\pgfpathlineto{\pgfqpoint{0.653778in}{0.829825in}}%
\pgfpathlineto{\pgfqpoint{0.644424in}{0.827185in}}%
\pgfpathlineto{\pgfqpoint{0.634712in}{0.827881in}}%
\pgfpathlineto{\pgfqpoint{0.625000in}{0.823453in}}%
\pgfusepath{stroke}%
\end{pgfscope}%
\begin{pgfscope}%
\pgfpathrectangle{\pgfqpoint{0.625000in}{0.550000in}}{\pgfqpoint{3.875000in}{3.850000in}} %
\pgfusepath{clip}%
\pgfsetbuttcap%
\pgfsetroundjoin%
\pgfsetlinewidth{0.501875pt}%
\definecolor{currentstroke}{rgb}{0.000000,0.000000,0.000000}%
\pgfsetstrokecolor{currentstroke}%
\pgfsetdash{}{0pt}%
\pgfpathmoveto{\pgfqpoint{0.625000in}{0.936998in}}%
\pgfpathlineto{\pgfqpoint{0.632069in}{0.935965in}}%
\pgfpathlineto{\pgfqpoint{0.626911in}{0.926316in}}%
\pgfpathlineto{\pgfqpoint{0.625000in}{0.923970in}}%
\pgfusepath{stroke}%
\end{pgfscope}%
\begin{pgfscope}%
\pgfpathrectangle{\pgfqpoint{0.625000in}{0.550000in}}{\pgfqpoint{3.875000in}{3.850000in}} %
\pgfusepath{clip}%
\pgfsetbuttcap%
\pgfsetroundjoin%
\pgfsetlinewidth{0.501875pt}%
\definecolor{currentstroke}{rgb}{0.000000,0.000000,0.000000}%
\pgfsetstrokecolor{currentstroke}%
\pgfsetdash{}{0pt}%
\pgfpathmoveto{\pgfqpoint{0.625000in}{0.971785in}}%
\pgfpathlineto{\pgfqpoint{0.628325in}{0.974561in}}%
\pgfpathlineto{\pgfqpoint{0.626908in}{0.984211in}}%
\pgfpathlineto{\pgfqpoint{0.628970in}{0.993860in}}%
\pgfpathlineto{\pgfqpoint{0.626673in}{1.003509in}}%
\pgfpathlineto{\pgfqpoint{0.625000in}{1.004877in}}%
\pgfusepath{stroke}%
\end{pgfscope}%
\begin{pgfscope}%
\pgfpathrectangle{\pgfqpoint{0.625000in}{0.550000in}}{\pgfqpoint{3.875000in}{3.850000in}} %
\pgfusepath{clip}%
\pgfsetbuttcap%
\pgfsetroundjoin%
\pgfsetlinewidth{0.501875pt}%
\definecolor{currentstroke}{rgb}{0.000000,0.000000,0.000000}%
\pgfsetstrokecolor{currentstroke}%
\pgfsetdash{}{0pt}%
\pgfpathmoveto{\pgfqpoint{0.625000in}{1.027256in}}%
\pgfpathlineto{\pgfqpoint{0.634712in}{1.027632in}}%
\pgfpathlineto{\pgfqpoint{0.644424in}{1.024284in}}%
\pgfpathlineto{\pgfqpoint{0.649279in}{1.022807in}}%
\pgfpathlineto{\pgfqpoint{0.654135in}{1.020240in}}%
\pgfpathlineto{\pgfqpoint{0.663847in}{1.016283in}}%
\pgfpathlineto{\pgfqpoint{0.671726in}{1.013158in}}%
\pgfpathlineto{\pgfqpoint{0.673559in}{1.012154in}}%
\pgfpathlineto{\pgfqpoint{0.683271in}{1.007203in}}%
\pgfpathlineto{\pgfqpoint{0.690146in}{1.003509in}}%
\pgfpathlineto{\pgfqpoint{0.692982in}{1.001257in}}%
\pgfpathlineto{\pgfqpoint{0.702588in}{0.993860in}}%
\pgfpathlineto{\pgfqpoint{0.702694in}{0.993720in}}%
\pgfpathlineto{\pgfqpoint{0.711364in}{0.984211in}}%
\pgfpathlineto{\pgfqpoint{0.712406in}{0.981742in}}%
\pgfpathlineto{\pgfqpoint{0.716793in}{0.974561in}}%
\pgfpathlineto{\pgfqpoint{0.718189in}{0.964912in}}%
\pgfpathlineto{\pgfqpoint{0.714175in}{0.955263in}}%
\pgfpathlineto{\pgfqpoint{0.712406in}{0.953862in}}%
\pgfpathlineto{\pgfqpoint{0.702694in}{0.946662in}}%
\pgfpathlineto{\pgfqpoint{0.701198in}{0.945614in}}%
\pgfpathlineto{\pgfqpoint{0.692982in}{0.942768in}}%
\pgfpathlineto{\pgfqpoint{0.683271in}{0.940288in}}%
\pgfpathlineto{\pgfqpoint{0.673559in}{0.938613in}}%
\pgfpathlineto{\pgfqpoint{0.663847in}{0.937498in}}%
\pgfpathlineto{\pgfqpoint{0.654135in}{0.936776in}}%
\pgfpathlineto{\pgfqpoint{0.644424in}{0.937599in}}%
\pgfpathlineto{\pgfqpoint{0.643339in}{0.945614in}}%
\pgfpathlineto{\pgfqpoint{0.643936in}{0.955263in}}%
\pgfpathlineto{\pgfqpoint{0.635990in}{0.964912in}}%
\pgfpathlineto{\pgfqpoint{0.634712in}{0.965821in}}%
\pgfpathlineto{\pgfqpoint{0.633987in}{0.964912in}}%
\pgfpathlineto{\pgfqpoint{0.625000in}{0.962831in}}%
\pgfusepath{stroke}%
\end{pgfscope}%
\begin{pgfscope}%
\pgfpathrectangle{\pgfqpoint{0.625000in}{0.550000in}}{\pgfqpoint{3.875000in}{3.850000in}} %
\pgfusepath{clip}%
\pgfsetbuttcap%
\pgfsetroundjoin%
\pgfsetlinewidth{0.501875pt}%
\definecolor{currentstroke}{rgb}{0.000000,0.000000,0.000000}%
\pgfsetstrokecolor{currentstroke}%
\pgfsetdash{}{0pt}%
\pgfpathmoveto{\pgfqpoint{0.625000in}{1.046094in}}%
\pgfpathlineto{\pgfqpoint{0.626848in}{1.042105in}}%
\pgfpathlineto{\pgfqpoint{0.625000in}{1.038116in}}%
\pgfusepath{stroke}%
\end{pgfscope}%
\begin{pgfscope}%
\pgfpathrectangle{\pgfqpoint{0.625000in}{0.550000in}}{\pgfqpoint{3.875000in}{3.850000in}} %
\pgfusepath{clip}%
\pgfsetbuttcap%
\pgfsetroundjoin%
\pgfsetlinewidth{0.501875pt}%
\definecolor{currentstroke}{rgb}{0.000000,0.000000,0.000000}%
\pgfsetstrokecolor{currentstroke}%
\pgfsetdash{}{0pt}%
\pgfpathmoveto{\pgfqpoint{0.625000in}{1.081164in}}%
\pgfpathlineto{\pgfqpoint{0.625756in}{1.080702in}}%
\pgfpathlineto{\pgfqpoint{0.625000in}{1.080512in}}%
\pgfusepath{stroke}%
\end{pgfscope}%
\begin{pgfscope}%
\pgfpathrectangle{\pgfqpoint{0.625000in}{0.550000in}}{\pgfqpoint{3.875000in}{3.850000in}} %
\pgfusepath{clip}%
\pgfsetbuttcap%
\pgfsetroundjoin%
\pgfsetlinewidth{0.501875pt}%
\definecolor{currentstroke}{rgb}{0.000000,0.000000,0.000000}%
\pgfsetstrokecolor{currentstroke}%
\pgfsetdash{}{0pt}%
\pgfpathmoveto{\pgfqpoint{0.625000in}{1.119716in}}%
\pgfpathlineto{\pgfqpoint{0.625511in}{1.119298in}}%
\pgfpathlineto{\pgfqpoint{0.625000in}{1.118880in}}%
\pgfusepath{stroke}%
\end{pgfscope}%
\begin{pgfscope}%
\pgfpathrectangle{\pgfqpoint{0.625000in}{0.550000in}}{\pgfqpoint{3.875000in}{3.850000in}} %
\pgfusepath{clip}%
\pgfsetbuttcap%
\pgfsetroundjoin%
\pgfsetlinewidth{0.501875pt}%
\definecolor{currentstroke}{rgb}{0.000000,0.000000,0.000000}%
\pgfsetstrokecolor{currentstroke}%
\pgfsetdash{}{0pt}%
\pgfpathmoveto{\pgfqpoint{0.625000in}{1.169939in}}%
\pgfpathlineto{\pgfqpoint{0.634116in}{1.167544in}}%
\pgfpathlineto{\pgfqpoint{0.625000in}{1.164301in}}%
\pgfusepath{stroke}%
\end{pgfscope}%
\begin{pgfscope}%
\pgfpathrectangle{\pgfqpoint{0.625000in}{0.550000in}}{\pgfqpoint{3.875000in}{3.850000in}} %
\pgfusepath{clip}%
\pgfsetbuttcap%
\pgfsetroundjoin%
\pgfsetlinewidth{0.501875pt}%
\definecolor{currentstroke}{rgb}{0.000000,0.000000,0.000000}%
\pgfsetstrokecolor{currentstroke}%
\pgfsetdash{}{0pt}%
\pgfpathmoveto{\pgfqpoint{0.625000in}{1.215939in}}%
\pgfpathlineto{\pgfqpoint{0.625532in}{1.215789in}}%
\pgfpathlineto{\pgfqpoint{0.625000in}{1.214590in}}%
\pgfusepath{stroke}%
\end{pgfscope}%
\begin{pgfscope}%
\pgfpathrectangle{\pgfqpoint{0.625000in}{0.550000in}}{\pgfqpoint{3.875000in}{3.850000in}} %
\pgfusepath{clip}%
\pgfsetbuttcap%
\pgfsetroundjoin%
\pgfsetlinewidth{0.501875pt}%
\definecolor{currentstroke}{rgb}{0.000000,0.000000,0.000000}%
\pgfsetstrokecolor{currentstroke}%
\pgfsetdash{}{0pt}%
\pgfpathmoveto{\pgfqpoint{0.625000in}{1.325571in}}%
\pgfpathlineto{\pgfqpoint{0.633080in}{1.331579in}}%
\pgfpathlineto{\pgfqpoint{0.632958in}{1.341228in}}%
\pgfpathlineto{\pgfqpoint{0.625000in}{1.344463in}}%
\pgfusepath{stroke}%
\end{pgfscope}%
\begin{pgfscope}%
\pgfpathrectangle{\pgfqpoint{0.625000in}{0.550000in}}{\pgfqpoint{3.875000in}{3.850000in}} %
\pgfusepath{clip}%
\pgfsetbuttcap%
\pgfsetroundjoin%
\pgfsetlinewidth{0.501875pt}%
\definecolor{currentstroke}{rgb}{0.000000,0.000000,0.000000}%
\pgfsetstrokecolor{currentstroke}%
\pgfsetdash{}{0pt}%
\pgfpathmoveto{\pgfqpoint{0.625000in}{1.357291in}}%
\pgfpathlineto{\pgfqpoint{0.633901in}{1.350877in}}%
\pgfpathlineto{\pgfqpoint{0.634712in}{1.346530in}}%
\pgfpathlineto{\pgfqpoint{0.638473in}{1.350877in}}%
\pgfpathlineto{\pgfqpoint{0.644424in}{1.355051in}}%
\pgfpathlineto{\pgfqpoint{0.647419in}{1.360526in}}%
\pgfpathlineto{\pgfqpoint{0.644424in}{1.367484in}}%
\pgfpathlineto{\pgfqpoint{0.644106in}{1.370175in}}%
\pgfpathlineto{\pgfqpoint{0.634712in}{1.378747in}}%
\pgfpathlineto{\pgfqpoint{0.634163in}{1.379825in}}%
\pgfpathlineto{\pgfqpoint{0.634712in}{1.382436in}}%
\pgfpathlineto{\pgfqpoint{0.638938in}{1.389474in}}%
\pgfpathlineto{\pgfqpoint{0.634712in}{1.395148in}}%
\pgfpathlineto{\pgfqpoint{0.633199in}{1.399123in}}%
\pgfpathlineto{\pgfqpoint{0.629862in}{1.408772in}}%
\pgfpathlineto{\pgfqpoint{0.630221in}{1.418421in}}%
\pgfpathlineto{\pgfqpoint{0.633038in}{1.428070in}}%
\pgfpathlineto{\pgfqpoint{0.627713in}{1.437719in}}%
\pgfpathlineto{\pgfqpoint{0.630503in}{1.447368in}}%
\pgfpathlineto{\pgfqpoint{0.627614in}{1.457018in}}%
\pgfpathlineto{\pgfqpoint{0.625000in}{1.464670in}}%
\pgfusepath{stroke}%
\end{pgfscope}%
\begin{pgfscope}%
\pgfpathrectangle{\pgfqpoint{0.625000in}{0.550000in}}{\pgfqpoint{3.875000in}{3.850000in}} %
\pgfusepath{clip}%
\pgfsetbuttcap%
\pgfsetroundjoin%
\pgfsetlinewidth{0.501875pt}%
\definecolor{currentstroke}{rgb}{0.000000,0.000000,0.000000}%
\pgfsetstrokecolor{currentstroke}%
\pgfsetdash{}{0pt}%
\pgfpathmoveto{\pgfqpoint{0.625000in}{1.490615in}}%
\pgfpathlineto{\pgfqpoint{0.628445in}{1.495614in}}%
\pgfpathlineto{\pgfqpoint{0.625000in}{1.498377in}}%
\pgfusepath{stroke}%
\end{pgfscope}%
\begin{pgfscope}%
\pgfpathrectangle{\pgfqpoint{0.625000in}{0.550000in}}{\pgfqpoint{3.875000in}{3.850000in}} %
\pgfusepath{clip}%
\pgfsetbuttcap%
\pgfsetroundjoin%
\pgfsetlinewidth{0.501875pt}%
\definecolor{currentstroke}{rgb}{0.000000,0.000000,0.000000}%
\pgfsetstrokecolor{currentstroke}%
\pgfsetdash{}{0pt}%
\pgfpathmoveto{\pgfqpoint{0.625000in}{1.512149in}}%
\pgfpathlineto{\pgfqpoint{0.634712in}{1.510088in}}%
\pgfpathlineto{\pgfqpoint{0.654135in}{1.502692in}}%
\pgfpathlineto{\pgfqpoint{0.799812in}{1.442439in}}%
\pgfpathlineto{\pgfqpoint{0.852919in}{1.418421in}}%
\pgfpathlineto{\pgfqpoint{0.896930in}{1.397874in}}%
\pgfpathlineto{\pgfqpoint{0.926065in}{1.385787in}}%
\pgfpathlineto{\pgfqpoint{0.955201in}{1.376520in}}%
\pgfpathlineto{\pgfqpoint{0.974624in}{1.372431in}}%
\pgfpathlineto{\pgfqpoint{0.994814in}{1.370175in}}%
\pgfpathlineto{\pgfqpoint{1.013471in}{1.369771in}}%
\pgfpathlineto{\pgfqpoint{1.032895in}{1.371087in}}%
\pgfpathlineto{\pgfqpoint{1.052318in}{1.374016in}}%
\pgfpathlineto{\pgfqpoint{1.081454in}{1.381254in}}%
\pgfpathlineto{\pgfqpoint{1.110589in}{1.391505in}}%
\pgfpathlineto{\pgfqpoint{1.139724in}{1.404575in}}%
\pgfpathlineto{\pgfqpoint{1.168860in}{1.420397in}}%
\pgfpathlineto{\pgfqpoint{1.197995in}{1.438999in}}%
\pgfpathlineto{\pgfqpoint{1.227130in}{1.460563in}}%
\pgfpathlineto{\pgfqpoint{1.256811in}{1.485965in}}%
\pgfpathlineto{\pgfqpoint{1.286016in}{1.514912in}}%
\pgfpathlineto{\pgfqpoint{1.311281in}{1.543860in}}%
\pgfpathlineto{\pgfqpoint{1.333960in}{1.573811in}}%
\pgfpathlineto{\pgfqpoint{1.353383in}{1.603407in}}%
\pgfpathlineto{\pgfqpoint{1.368968in}{1.630702in}}%
\pgfpathlineto{\pgfqpoint{1.383238in}{1.659649in}}%
\pgfpathlineto{\pgfqpoint{1.395389in}{1.688596in}}%
\pgfpathlineto{\pgfqpoint{1.405535in}{1.717544in}}%
\pgfpathlineto{\pgfqpoint{1.413780in}{1.746491in}}%
\pgfpathlineto{\pgfqpoint{1.421366in}{1.781852in}}%
\pgfpathlineto{\pgfqpoint{1.424887in}{1.804386in}}%
\pgfpathlineto{\pgfqpoint{1.427840in}{1.833333in}}%
\pgfpathlineto{\pgfqpoint{1.429097in}{1.862281in}}%
\pgfpathlineto{\pgfqpoint{1.428682in}{1.891228in}}%
\pgfpathlineto{\pgfqpoint{1.426591in}{1.920175in}}%
\pgfpathlineto{\pgfqpoint{1.421366in}{1.957345in}}%
\pgfpathlineto{\pgfqpoint{1.417280in}{1.978070in}}%
\pgfpathlineto{\pgfqpoint{1.409985in}{2.007018in}}%
\pgfpathlineto{\pgfqpoint{1.400870in}{2.035965in}}%
\pgfpathlineto{\pgfqpoint{1.389868in}{2.064912in}}%
\pgfpathlineto{\pgfqpoint{1.376860in}{2.093860in}}%
\pgfpathlineto{\pgfqpoint{1.361696in}{2.122807in}}%
\pgfpathlineto{\pgfqpoint{1.343672in}{2.152613in}}%
\pgfpathlineto{\pgfqpoint{1.324248in}{2.180779in}}%
\pgfpathlineto{\pgfqpoint{1.301659in}{2.209649in}}%
\pgfpathlineto{\pgfqpoint{1.275689in}{2.238752in}}%
\pgfpathlineto{\pgfqpoint{1.246520in}{2.267544in}}%
\pgfpathlineto{\pgfqpoint{1.213098in}{2.296491in}}%
\pgfpathlineto{\pgfqpoint{1.174598in}{2.325439in}}%
\pgfpathlineto{\pgfqpoint{1.139724in}{2.348271in}}%
\pgfpathlineto{\pgfqpoint{1.100877in}{2.370667in}}%
\pgfpathlineto{\pgfqpoint{1.071742in}{2.385575in}}%
\pgfpathlineto{\pgfqpoint{1.032895in}{2.403226in}}%
\pgfpathlineto{\pgfqpoint{0.984754in}{2.421930in}}%
\pgfpathlineto{\pgfqpoint{0.926065in}{2.440520in}}%
\pgfpathlineto{\pgfqpoint{0.896930in}{2.448199in}}%
\pgfpathlineto{\pgfqpoint{0.848371in}{2.458961in}}%
\pgfpathlineto{\pgfqpoint{0.809524in}{2.465806in}}%
\pgfpathlineto{\pgfqpoint{0.760965in}{2.472363in}}%
\pgfpathlineto{\pgfqpoint{0.712406in}{2.476769in}}%
\pgfpathlineto{\pgfqpoint{0.692982in}{2.477974in}}%
\pgfpathlineto{\pgfqpoint{0.683271in}{2.477144in}}%
\pgfpathlineto{\pgfqpoint{0.673559in}{2.468657in}}%
\pgfpathlineto{\pgfqpoint{0.672095in}{2.470175in}}%
\pgfpathlineto{\pgfqpoint{0.669727in}{2.479825in}}%
\pgfpathlineto{\pgfqpoint{0.673559in}{2.490997in}}%
\pgfpathlineto{\pgfqpoint{0.674923in}{2.489474in}}%
\pgfpathlineto{\pgfqpoint{0.683271in}{2.482505in}}%
\pgfpathlineto{\pgfqpoint{0.692982in}{2.481676in}}%
\pgfpathlineto{\pgfqpoint{0.741541in}{2.485281in}}%
\pgfpathlineto{\pgfqpoint{0.790100in}{2.490951in}}%
\pgfpathlineto{\pgfqpoint{0.840400in}{2.499123in}}%
\pgfpathlineto{\pgfqpoint{0.887218in}{2.509074in}}%
\pgfpathlineto{\pgfqpoint{0.935777in}{2.521936in}}%
\pgfpathlineto{\pgfqpoint{0.974624in}{2.534232in}}%
\pgfpathlineto{\pgfqpoint{1.013471in}{2.548490in}}%
\pgfpathlineto{\pgfqpoint{1.056202in}{2.566667in}}%
\pgfpathlineto{\pgfqpoint{1.100877in}{2.588983in}}%
\pgfpathlineto{\pgfqpoint{1.139724in}{2.611378in}}%
\pgfpathlineto{\pgfqpoint{1.168860in}{2.630320in}}%
\pgfpathlineto{\pgfqpoint{1.197995in}{2.651371in}}%
\pgfpathlineto{\pgfqpoint{1.217419in}{2.666766in}}%
\pgfpathlineto{\pgfqpoint{1.246554in}{2.692137in}}%
\pgfpathlineto{\pgfqpoint{1.275843in}{2.721053in}}%
\pgfpathlineto{\pgfqpoint{1.301659in}{2.750000in}}%
\pgfpathlineto{\pgfqpoint{1.324307in}{2.778947in}}%
\pgfpathlineto{\pgfqpoint{1.344248in}{2.807895in}}%
\pgfpathlineto{\pgfqpoint{1.363095in}{2.839434in}}%
\pgfpathlineto{\pgfqpoint{1.376860in}{2.865789in}}%
\pgfpathlineto{\pgfqpoint{1.392231in}{2.900674in}}%
\pgfpathlineto{\pgfqpoint{1.401942in}{2.926926in}}%
\pgfpathlineto{\pgfqpoint{1.411654in}{2.958840in}}%
\pgfpathlineto{\pgfqpoint{1.417280in}{2.981579in}}%
\pgfpathlineto{\pgfqpoint{1.424247in}{3.020175in}}%
\pgfpathlineto{\pgfqpoint{1.427476in}{3.049123in}}%
\pgfpathlineto{\pgfqpoint{1.429006in}{3.078070in}}%
\pgfpathlineto{\pgfqpoint{1.428865in}{3.107018in}}%
\pgfpathlineto{\pgfqpoint{1.427046in}{3.135965in}}%
\pgfpathlineto{\pgfqpoint{1.423518in}{3.164912in}}%
\pgfpathlineto{\pgfqpoint{1.418268in}{3.193860in}}%
\pgfpathlineto{\pgfqpoint{1.411227in}{3.222807in}}%
\pgfpathlineto{\pgfqpoint{1.401942in}{3.252947in}}%
\pgfpathlineto{\pgfqpoint{1.391561in}{3.280702in}}%
\pgfpathlineto{\pgfqpoint{1.378731in}{3.309649in}}%
\pgfpathlineto{\pgfqpoint{1.363095in}{3.339665in}}%
\pgfpathlineto{\pgfqpoint{1.343672in}{3.371565in}}%
\pgfpathlineto{\pgfqpoint{1.324248in}{3.399190in}}%
\pgfpathlineto{\pgfqpoint{1.303251in}{3.425439in}}%
\pgfpathlineto{\pgfqpoint{1.275689in}{3.455475in}}%
\pgfpathlineto{\pgfqpoint{1.246034in}{3.483333in}}%
\pgfpathlineto{\pgfqpoint{1.217419in}{3.506625in}}%
\pgfpathlineto{\pgfqpoint{1.188283in}{3.527176in}}%
\pgfpathlineto{\pgfqpoint{1.159148in}{3.544834in}}%
\pgfpathlineto{\pgfqpoint{1.128220in}{3.560526in}}%
\pgfpathlineto{\pgfqpoint{1.100877in}{3.571882in}}%
\pgfpathlineto{\pgfqpoint{1.071742in}{3.581153in}}%
\pgfpathlineto{\pgfqpoint{1.042607in}{3.587296in}}%
\pgfpathlineto{\pgfqpoint{1.022021in}{3.589474in}}%
\pgfpathlineto{\pgfqpoint{1.003759in}{3.589890in}}%
\pgfpathlineto{\pgfqpoint{0.984336in}{3.588559in}}%
\pgfpathlineto{\pgfqpoint{0.964912in}{3.585410in}}%
\pgfpathlineto{\pgfqpoint{0.943860in}{3.579825in}}%
\pgfpathlineto{\pgfqpoint{0.926065in}{3.573862in}}%
\pgfpathlineto{\pgfqpoint{0.887218in}{3.557408in}}%
\pgfpathlineto{\pgfqpoint{0.780388in}{3.508891in}}%
\pgfpathlineto{\pgfqpoint{0.663847in}{3.460904in}}%
\pgfpathlineto{\pgfqpoint{0.634712in}{3.449561in}}%
\pgfpathlineto{\pgfqpoint{0.625000in}{3.447500in}}%
\pgfpathlineto{\pgfqpoint{0.625000in}{3.447500in}}%
\pgfusepath{stroke}%
\end{pgfscope}%
\begin{pgfscope}%
\pgfpathrectangle{\pgfqpoint{0.625000in}{0.550000in}}{\pgfqpoint{3.875000in}{3.850000in}} %
\pgfusepath{clip}%
\pgfsetbuttcap%
\pgfsetroundjoin%
\pgfsetlinewidth{0.501875pt}%
\definecolor{currentstroke}{rgb}{0.000000,0.000000,0.000000}%
\pgfsetstrokecolor{currentstroke}%
\pgfsetdash{}{0pt}%
\pgfpathmoveto{\pgfqpoint{0.625000in}{1.589411in}}%
\pgfpathlineto{\pgfqpoint{0.628273in}{1.582456in}}%
\pgfpathlineto{\pgfqpoint{0.625000in}{1.580305in}}%
\pgfusepath{stroke}%
\end{pgfscope}%
\begin{pgfscope}%
\pgfpathrectangle{\pgfqpoint{0.625000in}{0.550000in}}{\pgfqpoint{3.875000in}{3.850000in}} %
\pgfusepath{clip}%
\pgfsetbuttcap%
\pgfsetroundjoin%
\pgfsetlinewidth{0.501875pt}%
\definecolor{currentstroke}{rgb}{0.000000,0.000000,0.000000}%
\pgfsetstrokecolor{currentstroke}%
\pgfsetdash{}{0pt}%
\pgfpathmoveto{\pgfqpoint{0.625000in}{1.639580in}}%
\pgfpathlineto{\pgfqpoint{0.632182in}{1.630702in}}%
\pgfpathlineto{\pgfqpoint{0.625000in}{1.621957in}}%
\pgfusepath{stroke}%
\end{pgfscope}%
\begin{pgfscope}%
\pgfpathrectangle{\pgfqpoint{0.625000in}{0.550000in}}{\pgfqpoint{3.875000in}{3.850000in}} %
\pgfusepath{clip}%
\pgfsetbuttcap%
\pgfsetroundjoin%
\pgfsetlinewidth{0.501875pt}%
\definecolor{currentstroke}{rgb}{0.000000,0.000000,0.000000}%
\pgfsetstrokecolor{currentstroke}%
\pgfsetdash{}{0pt}%
\pgfpathmoveto{\pgfqpoint{0.625000in}{1.709564in}}%
\pgfpathlineto{\pgfqpoint{0.627915in}{1.707895in}}%
\pgfpathlineto{\pgfqpoint{0.625000in}{1.701645in}}%
\pgfusepath{stroke}%
\end{pgfscope}%
\begin{pgfscope}%
\pgfpathrectangle{\pgfqpoint{0.625000in}{0.550000in}}{\pgfqpoint{3.875000in}{3.850000in}} %
\pgfusepath{clip}%
\pgfsetbuttcap%
\pgfsetroundjoin%
\pgfsetlinewidth{0.501875pt}%
\definecolor{currentstroke}{rgb}{0.000000,0.000000,0.000000}%
\pgfsetstrokecolor{currentstroke}%
\pgfsetdash{}{0pt}%
\pgfpathmoveto{\pgfqpoint{0.625000in}{1.789510in}}%
\pgfpathlineto{\pgfqpoint{0.633213in}{1.785088in}}%
\pgfpathlineto{\pgfqpoint{0.625000in}{1.779985in}}%
\pgfusepath{stroke}%
\end{pgfscope}%
\begin{pgfscope}%
\pgfpathrectangle{\pgfqpoint{0.625000in}{0.550000in}}{\pgfqpoint{3.875000in}{3.850000in}} %
\pgfusepath{clip}%
\pgfsetbuttcap%
\pgfsetroundjoin%
\pgfsetlinewidth{0.501875pt}%
\definecolor{currentstroke}{rgb}{0.000000,0.000000,0.000000}%
\pgfsetstrokecolor{currentstroke}%
\pgfsetdash{}{0pt}%
\pgfpathmoveto{\pgfqpoint{0.625000in}{1.873921in}}%
\pgfpathlineto{\pgfqpoint{0.627145in}{1.871930in}}%
\pgfpathlineto{\pgfqpoint{0.625000in}{1.864648in}}%
\pgfusepath{stroke}%
\end{pgfscope}%
\begin{pgfscope}%
\pgfpathrectangle{\pgfqpoint{0.625000in}{0.550000in}}{\pgfqpoint{3.875000in}{3.850000in}} %
\pgfusepath{clip}%
\pgfsetbuttcap%
\pgfsetroundjoin%
\pgfsetlinewidth{0.501875pt}%
\definecolor{currentstroke}{rgb}{0.000000,0.000000,0.000000}%
\pgfsetstrokecolor{currentstroke}%
\pgfsetdash{}{0pt}%
\pgfpathmoveto{\pgfqpoint{0.625000in}{1.945377in}}%
\pgfpathlineto{\pgfqpoint{0.631676in}{1.939474in}}%
\pgfpathlineto{\pgfqpoint{0.625000in}{1.933847in}}%
\pgfusepath{stroke}%
\end{pgfscope}%
\begin{pgfscope}%
\pgfpathrectangle{\pgfqpoint{0.625000in}{0.550000in}}{\pgfqpoint{3.875000in}{3.850000in}} %
\pgfusepath{clip}%
\pgfsetbuttcap%
\pgfsetroundjoin%
\pgfsetlinewidth{0.501875pt}%
\definecolor{currentstroke}{rgb}{0.000000,0.000000,0.000000}%
\pgfsetstrokecolor{currentstroke}%
\pgfsetdash{}{0pt}%
\pgfpathmoveto{\pgfqpoint{0.625000in}{1.973943in}}%
\pgfpathlineto{\pgfqpoint{0.626816in}{1.968421in}}%
\pgfpathlineto{\pgfqpoint{0.625000in}{1.962899in}}%
\pgfusepath{stroke}%
\end{pgfscope}%
\begin{pgfscope}%
\pgfpathrectangle{\pgfqpoint{0.625000in}{0.550000in}}{\pgfqpoint{3.875000in}{3.850000in}} %
\pgfusepath{clip}%
\pgfsetbuttcap%
\pgfsetroundjoin%
\pgfsetlinewidth{0.501875pt}%
\definecolor{currentstroke}{rgb}{0.000000,0.000000,0.000000}%
\pgfsetstrokecolor{currentstroke}%
\pgfsetdash{}{0pt}%
\pgfpathmoveto{\pgfqpoint{0.625000in}{2.084387in}}%
\pgfpathlineto{\pgfqpoint{0.625209in}{2.084211in}}%
\pgfpathlineto{\pgfqpoint{0.625000in}{2.082328in}}%
\pgfusepath{stroke}%
\end{pgfscope}%
\begin{pgfscope}%
\pgfpathrectangle{\pgfqpoint{0.625000in}{0.550000in}}{\pgfqpoint{3.875000in}{3.850000in}} %
\pgfusepath{clip}%
\pgfsetbuttcap%
\pgfsetroundjoin%
\pgfsetlinewidth{0.501875pt}%
\definecolor{currentstroke}{rgb}{0.000000,0.000000,0.000000}%
\pgfsetstrokecolor{currentstroke}%
\pgfsetdash{}{0pt}%
\pgfpathmoveto{\pgfqpoint{0.625000in}{2.172766in}}%
\pgfpathlineto{\pgfqpoint{0.626459in}{2.171053in}}%
\pgfpathlineto{\pgfqpoint{0.625000in}{2.169487in}}%
\pgfusepath{stroke}%
\end{pgfscope}%
\begin{pgfscope}%
\pgfpathrectangle{\pgfqpoint{0.625000in}{0.550000in}}{\pgfqpoint{3.875000in}{3.850000in}} %
\pgfusepath{clip}%
\pgfsetbuttcap%
\pgfsetroundjoin%
\pgfsetlinewidth{0.501875pt}%
\definecolor{currentstroke}{rgb}{0.000000,0.000000,0.000000}%
\pgfsetstrokecolor{currentstroke}%
\pgfsetdash{}{0pt}%
\pgfpathmoveto{\pgfqpoint{0.625000in}{2.225240in}}%
\pgfpathlineto{\pgfqpoint{0.628659in}{2.219298in}}%
\pgfpathlineto{\pgfqpoint{0.625000in}{2.213357in}}%
\pgfusepath{stroke}%
\end{pgfscope}%
\begin{pgfscope}%
\pgfpathrectangle{\pgfqpoint{0.625000in}{0.550000in}}{\pgfqpoint{3.875000in}{3.850000in}} %
\pgfusepath{clip}%
\pgfsetbuttcap%
\pgfsetroundjoin%
\pgfsetlinewidth{0.501875pt}%
\definecolor{currentstroke}{rgb}{0.000000,0.000000,0.000000}%
\pgfsetstrokecolor{currentstroke}%
\pgfsetdash{}{0pt}%
\pgfpathmoveto{\pgfqpoint{0.625000in}{2.265035in}}%
\pgfpathlineto{\pgfqpoint{0.627826in}{2.267544in}}%
\pgfpathlineto{\pgfqpoint{0.633802in}{2.277193in}}%
\pgfpathlineto{\pgfqpoint{0.634712in}{2.281708in}}%
\pgfpathlineto{\pgfqpoint{0.644424in}{2.282400in}}%
\pgfpathlineto{\pgfqpoint{0.654135in}{2.284174in}}%
\pgfpathlineto{\pgfqpoint{0.663196in}{2.286842in}}%
\pgfpathlineto{\pgfqpoint{0.663847in}{2.287026in}}%
\pgfpathlineto{\pgfqpoint{0.673559in}{2.290833in}}%
\pgfpathlineto{\pgfqpoint{0.683271in}{2.296099in}}%
\pgfpathlineto{\pgfqpoint{0.683857in}{2.296491in}}%
\pgfpathlineto{\pgfqpoint{0.692982in}{2.302776in}}%
\pgfpathlineto{\pgfqpoint{0.696878in}{2.306140in}}%
\pgfpathlineto{\pgfqpoint{0.702694in}{2.311539in}}%
\pgfpathlineto{\pgfqpoint{0.706501in}{2.315789in}}%
\pgfpathlineto{\pgfqpoint{0.712406in}{2.323293in}}%
\pgfpathlineto{\pgfqpoint{0.713866in}{2.325439in}}%
\pgfpathlineto{\pgfqpoint{0.719526in}{2.335088in}}%
\pgfpathlineto{\pgfqpoint{0.722118in}{2.340698in}}%
\pgfpathlineto{\pgfqpoint{0.723813in}{2.344737in}}%
\pgfpathlineto{\pgfqpoint{0.726941in}{2.354386in}}%
\pgfpathlineto{\pgfqpoint{0.728965in}{2.364035in}}%
\pgfpathlineto{\pgfqpoint{0.729976in}{2.373684in}}%
\pgfpathlineto{\pgfqpoint{0.730036in}{2.383333in}}%
\pgfpathlineto{\pgfqpoint{0.729163in}{2.392982in}}%
\pgfpathlineto{\pgfqpoint{0.727311in}{2.402632in}}%
\pgfpathlineto{\pgfqpoint{0.724951in}{2.412281in}}%
\pgfpathlineto{\pgfqpoint{0.722118in}{2.416983in}}%
\pgfpathlineto{\pgfqpoint{0.720245in}{2.421930in}}%
\pgfpathlineto{\pgfqpoint{0.716704in}{2.431579in}}%
\pgfpathlineto{\pgfqpoint{0.712406in}{2.434769in}}%
\pgfpathlineto{\pgfqpoint{0.708194in}{2.441228in}}%
\pgfpathlineto{\pgfqpoint{0.712406in}{2.445872in}}%
\pgfpathlineto{\pgfqpoint{0.717758in}{2.441228in}}%
\pgfpathlineto{\pgfqpoint{0.722118in}{2.435757in}}%
\pgfpathlineto{\pgfqpoint{0.726037in}{2.431579in}}%
\pgfpathlineto{\pgfqpoint{0.731830in}{2.422903in}}%
\pgfpathlineto{\pgfqpoint{0.732597in}{2.421930in}}%
\pgfpathlineto{\pgfqpoint{0.738168in}{2.412281in}}%
\pgfpathlineto{\pgfqpoint{0.741541in}{2.404270in}}%
\pgfpathlineto{\pgfqpoint{0.742364in}{2.402632in}}%
\pgfpathlineto{\pgfqpoint{0.745802in}{2.392982in}}%
\pgfpathlineto{\pgfqpoint{0.748189in}{2.383333in}}%
\pgfpathlineto{\pgfqpoint{0.749694in}{2.373684in}}%
\pgfpathlineto{\pgfqpoint{0.750391in}{2.364035in}}%
\pgfpathlineto{\pgfqpoint{0.750302in}{2.354386in}}%
\pgfpathlineto{\pgfqpoint{0.749408in}{2.344737in}}%
\pgfpathlineto{\pgfqpoint{0.747668in}{2.335088in}}%
\pgfpathlineto{\pgfqpoint{0.745019in}{2.325439in}}%
\pgfpathlineto{\pgfqpoint{0.741541in}{2.316167in}}%
\pgfpathlineto{\pgfqpoint{0.741389in}{2.315789in}}%
\pgfpathlineto{\pgfqpoint{0.736760in}{2.306140in}}%
\pgfpathlineto{\pgfqpoint{0.731830in}{2.297891in}}%
\pgfpathlineto{\pgfqpoint{0.730895in}{2.296491in}}%
\pgfpathlineto{\pgfqpoint{0.723592in}{2.286842in}}%
\pgfpathlineto{\pgfqpoint{0.722118in}{2.285091in}}%
\pgfpathlineto{\pgfqpoint{0.714434in}{2.277193in}}%
\pgfpathlineto{\pgfqpoint{0.712406in}{2.275265in}}%
\pgfpathlineto{\pgfqpoint{0.702750in}{2.267544in}}%
\pgfpathlineto{\pgfqpoint{0.702694in}{2.267502in}}%
\pgfpathlineto{\pgfqpoint{0.692982in}{2.261170in}}%
\pgfpathlineto{\pgfqpoint{0.686604in}{2.257895in}}%
\pgfpathlineto{\pgfqpoint{0.683271in}{2.256185in}}%
\pgfpathlineto{\pgfqpoint{0.673559in}{2.252217in}}%
\pgfpathlineto{\pgfqpoint{0.663847in}{2.249294in}}%
\pgfpathlineto{\pgfqpoint{0.658968in}{2.248246in}}%
\pgfpathlineto{\pgfqpoint{0.654135in}{2.247173in}}%
\pgfpathlineto{\pgfqpoint{0.644424in}{2.245637in}}%
\pgfpathlineto{\pgfqpoint{0.634712in}{2.243606in}}%
\pgfpathlineto{\pgfqpoint{0.625000in}{2.241213in}}%
\pgfusepath{stroke}%
\end{pgfscope}%
\begin{pgfscope}%
\pgfpathrectangle{\pgfqpoint{0.625000in}{0.550000in}}{\pgfqpoint{3.875000in}{3.850000in}} %
\pgfusepath{clip}%
\pgfsetbuttcap%
\pgfsetroundjoin%
\pgfsetlinewidth{0.501875pt}%
\definecolor{currentstroke}{rgb}{0.000000,0.000000,0.000000}%
\pgfsetstrokecolor{currentstroke}%
\pgfsetdash{}{0pt}%
\pgfpathmoveto{\pgfqpoint{0.625000in}{2.360626in}}%
\pgfpathlineto{\pgfqpoint{0.628258in}{2.354386in}}%
\pgfpathlineto{\pgfqpoint{0.625000in}{2.352755in}}%
\pgfusepath{stroke}%
\end{pgfscope}%
\begin{pgfscope}%
\pgfpathrectangle{\pgfqpoint{0.625000in}{0.550000in}}{\pgfqpoint{3.875000in}{3.850000in}} %
\pgfusepath{clip}%
\pgfsetbuttcap%
\pgfsetroundjoin%
\pgfsetlinewidth{0.501875pt}%
\definecolor{currentstroke}{rgb}{0.000000,0.000000,0.000000}%
\pgfsetstrokecolor{currentstroke}%
\pgfsetdash{}{0pt}%
\pgfpathmoveto{\pgfqpoint{0.625000in}{2.406043in}}%
\pgfpathlineto{\pgfqpoint{0.634712in}{2.404172in}}%
\pgfpathlineto{\pgfqpoint{0.638990in}{2.402632in}}%
\pgfpathlineto{\pgfqpoint{0.634712in}{2.401027in}}%
\pgfpathlineto{\pgfqpoint{0.625000in}{2.398965in}}%
\pgfusepath{stroke}%
\end{pgfscope}%
\begin{pgfscope}%
\pgfpathrectangle{\pgfqpoint{0.625000in}{0.550000in}}{\pgfqpoint{3.875000in}{3.850000in}} %
\pgfusepath{clip}%
\pgfsetbuttcap%
\pgfsetroundjoin%
\pgfsetlinewidth{0.501875pt}%
\definecolor{currentstroke}{rgb}{0.000000,0.000000,0.000000}%
\pgfsetstrokecolor{currentstroke}%
\pgfsetdash{}{0pt}%
\pgfpathmoveto{\pgfqpoint{0.625000in}{2.560684in}}%
\pgfpathlineto{\pgfqpoint{0.634712in}{2.558622in}}%
\pgfpathlineto{\pgfqpoint{0.638990in}{2.557018in}}%
\pgfpathlineto{\pgfqpoint{0.634712in}{2.555477in}}%
\pgfpathlineto{\pgfqpoint{0.625000in}{2.553606in}}%
\pgfusepath{stroke}%
\end{pgfscope}%
\begin{pgfscope}%
\pgfpathrectangle{\pgfqpoint{0.625000in}{0.550000in}}{\pgfqpoint{3.875000in}{3.850000in}} %
\pgfusepath{clip}%
\pgfsetbuttcap%
\pgfsetroundjoin%
\pgfsetlinewidth{0.501875pt}%
\definecolor{currentstroke}{rgb}{0.000000,0.000000,0.000000}%
\pgfsetstrokecolor{currentstroke}%
\pgfsetdash{}{0pt}%
\pgfpathmoveto{\pgfqpoint{0.625000in}{2.606894in}}%
\pgfpathlineto{\pgfqpoint{0.628258in}{2.605263in}}%
\pgfpathlineto{\pgfqpoint{0.625000in}{2.599023in}}%
\pgfusepath{stroke}%
\end{pgfscope}%
\begin{pgfscope}%
\pgfpathrectangle{\pgfqpoint{0.625000in}{0.550000in}}{\pgfqpoint{3.875000in}{3.850000in}} %
\pgfusepath{clip}%
\pgfsetbuttcap%
\pgfsetroundjoin%
\pgfsetlinewidth{0.501875pt}%
\definecolor{currentstroke}{rgb}{0.000000,0.000000,0.000000}%
\pgfsetstrokecolor{currentstroke}%
\pgfsetdash{}{0pt}%
\pgfpathmoveto{\pgfqpoint{0.625000in}{2.718436in}}%
\pgfpathlineto{\pgfqpoint{0.634712in}{2.716043in}}%
\pgfpathlineto{\pgfqpoint{0.644424in}{2.714012in}}%
\pgfpathlineto{\pgfqpoint{0.654135in}{2.712476in}}%
\pgfpathlineto{\pgfqpoint{0.658968in}{2.711404in}}%
\pgfpathlineto{\pgfqpoint{0.663847in}{2.710355in}}%
\pgfpathlineto{\pgfqpoint{0.673559in}{2.707432in}}%
\pgfpathlineto{\pgfqpoint{0.683271in}{2.703465in}}%
\pgfpathlineto{\pgfqpoint{0.686604in}{2.701754in}}%
\pgfpathlineto{\pgfqpoint{0.692982in}{2.698479in}}%
\pgfpathlineto{\pgfqpoint{0.702694in}{2.692148in}}%
\pgfpathlineto{\pgfqpoint{0.702750in}{2.692105in}}%
\pgfpathlineto{\pgfqpoint{0.712406in}{2.684384in}}%
\pgfpathlineto{\pgfqpoint{0.714434in}{2.682456in}}%
\pgfpathlineto{\pgfqpoint{0.722118in}{2.674558in}}%
\pgfpathlineto{\pgfqpoint{0.723592in}{2.672807in}}%
\pgfpathlineto{\pgfqpoint{0.730895in}{2.663158in}}%
\pgfpathlineto{\pgfqpoint{0.731830in}{2.661758in}}%
\pgfpathlineto{\pgfqpoint{0.736760in}{2.653509in}}%
\pgfpathlineto{\pgfqpoint{0.741389in}{2.643860in}}%
\pgfpathlineto{\pgfqpoint{0.741541in}{2.643482in}}%
\pgfpathlineto{\pgfqpoint{0.745019in}{2.634211in}}%
\pgfpathlineto{\pgfqpoint{0.747668in}{2.624561in}}%
\pgfpathlineto{\pgfqpoint{0.749408in}{2.614912in}}%
\pgfpathlineto{\pgfqpoint{0.750302in}{2.605263in}}%
\pgfpathlineto{\pgfqpoint{0.750391in}{2.595614in}}%
\pgfpathlineto{\pgfqpoint{0.749694in}{2.585965in}}%
\pgfpathlineto{\pgfqpoint{0.748189in}{2.576316in}}%
\pgfpathlineto{\pgfqpoint{0.745802in}{2.566667in}}%
\pgfpathlineto{\pgfqpoint{0.742364in}{2.557018in}}%
\pgfpathlineto{\pgfqpoint{0.741541in}{2.555379in}}%
\pgfpathlineto{\pgfqpoint{0.738168in}{2.547368in}}%
\pgfpathlineto{\pgfqpoint{0.732597in}{2.537719in}}%
\pgfpathlineto{\pgfqpoint{0.731830in}{2.536746in}}%
\pgfpathlineto{\pgfqpoint{0.726037in}{2.528070in}}%
\pgfpathlineto{\pgfqpoint{0.722118in}{2.523892in}}%
\pgfpathlineto{\pgfqpoint{0.717758in}{2.518421in}}%
\pgfpathlineto{\pgfqpoint{0.712406in}{2.513777in}}%
\pgfpathlineto{\pgfqpoint{0.708194in}{2.518421in}}%
\pgfpathlineto{\pgfqpoint{0.712406in}{2.524880in}}%
\pgfpathlineto{\pgfqpoint{0.716704in}{2.528070in}}%
\pgfpathlineto{\pgfqpoint{0.720245in}{2.537719in}}%
\pgfpathlineto{\pgfqpoint{0.722118in}{2.542666in}}%
\pgfpathlineto{\pgfqpoint{0.724951in}{2.547368in}}%
\pgfpathlineto{\pgfqpoint{0.727311in}{2.557018in}}%
\pgfpathlineto{\pgfqpoint{0.729163in}{2.566667in}}%
\pgfpathlineto{\pgfqpoint{0.730036in}{2.576316in}}%
\pgfpathlineto{\pgfqpoint{0.729976in}{2.585965in}}%
\pgfpathlineto{\pgfqpoint{0.728965in}{2.595614in}}%
\pgfpathlineto{\pgfqpoint{0.726941in}{2.605263in}}%
\pgfpathlineto{\pgfqpoint{0.723813in}{2.614912in}}%
\pgfpathlineto{\pgfqpoint{0.722118in}{2.618951in}}%
\pgfpathlineto{\pgfqpoint{0.719526in}{2.624561in}}%
\pgfpathlineto{\pgfqpoint{0.713866in}{2.634211in}}%
\pgfpathlineto{\pgfqpoint{0.712406in}{2.636357in}}%
\pgfpathlineto{\pgfqpoint{0.706501in}{2.643860in}}%
\pgfpathlineto{\pgfqpoint{0.702694in}{2.648110in}}%
\pgfpathlineto{\pgfqpoint{0.696878in}{2.653509in}}%
\pgfpathlineto{\pgfqpoint{0.692982in}{2.656873in}}%
\pgfpathlineto{\pgfqpoint{0.683857in}{2.663158in}}%
\pgfpathlineto{\pgfqpoint{0.683271in}{2.663550in}}%
\pgfpathlineto{\pgfqpoint{0.673559in}{2.668816in}}%
\pgfpathlineto{\pgfqpoint{0.663847in}{2.672623in}}%
\pgfpathlineto{\pgfqpoint{0.663196in}{2.672807in}}%
\pgfpathlineto{\pgfqpoint{0.654135in}{2.675475in}}%
\pgfpathlineto{\pgfqpoint{0.644424in}{2.677249in}}%
\pgfpathlineto{\pgfqpoint{0.634712in}{2.677941in}}%
\pgfpathlineto{\pgfqpoint{0.633802in}{2.682456in}}%
\pgfpathlineto{\pgfqpoint{0.627826in}{2.692105in}}%
\pgfpathlineto{\pgfqpoint{0.625000in}{2.694614in}}%
\pgfusepath{stroke}%
\end{pgfscope}%
\begin{pgfscope}%
\pgfpathrectangle{\pgfqpoint{0.625000in}{0.550000in}}{\pgfqpoint{3.875000in}{3.850000in}} %
\pgfusepath{clip}%
\pgfsetbuttcap%
\pgfsetroundjoin%
\pgfsetlinewidth{0.501875pt}%
\definecolor{currentstroke}{rgb}{0.000000,0.000000,0.000000}%
\pgfsetstrokecolor{currentstroke}%
\pgfsetdash{}{0pt}%
\pgfpathmoveto{\pgfqpoint{0.625000in}{2.746293in}}%
\pgfpathlineto{\pgfqpoint{0.628659in}{2.740351in}}%
\pgfpathlineto{\pgfqpoint{0.625000in}{2.734409in}}%
\pgfusepath{stroke}%
\end{pgfscope}%
\begin{pgfscope}%
\pgfpathrectangle{\pgfqpoint{0.625000in}{0.550000in}}{\pgfqpoint{3.875000in}{3.850000in}} %
\pgfusepath{clip}%
\pgfsetbuttcap%
\pgfsetroundjoin%
\pgfsetlinewidth{0.501875pt}%
\definecolor{currentstroke}{rgb}{0.000000,0.000000,0.000000}%
\pgfsetstrokecolor{currentstroke}%
\pgfsetdash{}{0pt}%
\pgfpathmoveto{\pgfqpoint{0.625000in}{2.877321in}}%
\pgfpathlineto{\pgfqpoint{0.625209in}{2.875439in}}%
\pgfpathlineto{\pgfqpoint{0.625000in}{2.875262in}}%
\pgfusepath{stroke}%
\end{pgfscope}%
\begin{pgfscope}%
\pgfpathrectangle{\pgfqpoint{0.625000in}{0.550000in}}{\pgfqpoint{3.875000in}{3.850000in}} %
\pgfusepath{clip}%
\pgfsetbuttcap%
\pgfsetroundjoin%
\pgfsetlinewidth{0.501875pt}%
\definecolor{currentstroke}{rgb}{0.000000,0.000000,0.000000}%
\pgfsetstrokecolor{currentstroke}%
\pgfsetdash{}{0pt}%
\pgfpathmoveto{\pgfqpoint{0.625000in}{2.944627in}}%
\pgfpathlineto{\pgfqpoint{0.626175in}{2.942982in}}%
\pgfpathlineto{\pgfqpoint{0.625000in}{2.941767in}}%
\pgfusepath{stroke}%
\end{pgfscope}%
\begin{pgfscope}%
\pgfpathrectangle{\pgfqpoint{0.625000in}{0.550000in}}{\pgfqpoint{3.875000in}{3.850000in}} %
\pgfusepath{clip}%
\pgfsetbuttcap%
\pgfsetroundjoin%
\pgfsetlinewidth{0.501875pt}%
\definecolor{currentstroke}{rgb}{0.000000,0.000000,0.000000}%
\pgfsetstrokecolor{currentstroke}%
\pgfsetdash{}{0pt}%
\pgfpathmoveto{\pgfqpoint{0.625000in}{2.996750in}}%
\pgfpathlineto{\pgfqpoint{0.626816in}{2.991228in}}%
\pgfpathlineto{\pgfqpoint{0.625000in}{2.985706in}}%
\pgfusepath{stroke}%
\end{pgfscope}%
\begin{pgfscope}%
\pgfpathrectangle{\pgfqpoint{0.625000in}{0.550000in}}{\pgfqpoint{3.875000in}{3.850000in}} %
\pgfusepath{clip}%
\pgfsetbuttcap%
\pgfsetroundjoin%
\pgfsetlinewidth{0.501875pt}%
\definecolor{currentstroke}{rgb}{0.000000,0.000000,0.000000}%
\pgfsetstrokecolor{currentstroke}%
\pgfsetdash{}{0pt}%
\pgfpathmoveto{\pgfqpoint{0.625000in}{3.025802in}}%
\pgfpathlineto{\pgfqpoint{0.631676in}{3.020175in}}%
\pgfpathlineto{\pgfqpoint{0.625000in}{3.014272in}}%
\pgfusepath{stroke}%
\end{pgfscope}%
\begin{pgfscope}%
\pgfpathrectangle{\pgfqpoint{0.625000in}{0.550000in}}{\pgfqpoint{3.875000in}{3.850000in}} %
\pgfusepath{clip}%
\pgfsetbuttcap%
\pgfsetroundjoin%
\pgfsetlinewidth{0.501875pt}%
\definecolor{currentstroke}{rgb}{0.000000,0.000000,0.000000}%
\pgfsetstrokecolor{currentstroke}%
\pgfsetdash{}{0pt}%
\pgfpathmoveto{\pgfqpoint{0.625000in}{3.089615in}}%
\pgfpathlineto{\pgfqpoint{0.627145in}{3.087719in}}%
\pgfpathlineto{\pgfqpoint{0.625000in}{3.085728in}}%
\pgfusepath{stroke}%
\end{pgfscope}%
\begin{pgfscope}%
\pgfpathrectangle{\pgfqpoint{0.625000in}{0.550000in}}{\pgfqpoint{3.875000in}{3.850000in}} %
\pgfusepath{clip}%
\pgfsetbuttcap%
\pgfsetroundjoin%
\pgfsetlinewidth{0.501875pt}%
\definecolor{currentstroke}{rgb}{0.000000,0.000000,0.000000}%
\pgfsetstrokecolor{currentstroke}%
\pgfsetdash{}{0pt}%
\pgfpathmoveto{\pgfqpoint{0.625000in}{3.179664in}}%
\pgfpathlineto{\pgfqpoint{0.633213in}{3.174561in}}%
\pgfpathlineto{\pgfqpoint{0.625000in}{3.170139in}}%
\pgfusepath{stroke}%
\end{pgfscope}%
\begin{pgfscope}%
\pgfpathrectangle{\pgfqpoint{0.625000in}{0.550000in}}{\pgfqpoint{3.875000in}{3.850000in}} %
\pgfusepath{clip}%
\pgfsetbuttcap%
\pgfsetroundjoin%
\pgfsetlinewidth{0.501875pt}%
\definecolor{currentstroke}{rgb}{0.000000,0.000000,0.000000}%
\pgfsetstrokecolor{currentstroke}%
\pgfsetdash{}{0pt}%
\pgfpathmoveto{\pgfqpoint{0.625000in}{3.337693in}}%
\pgfpathlineto{\pgfqpoint{0.632182in}{3.328947in}}%
\pgfpathlineto{\pgfqpoint{0.625000in}{3.320069in}}%
\pgfusepath{stroke}%
\end{pgfscope}%
\begin{pgfscope}%
\pgfpathrectangle{\pgfqpoint{0.625000in}{0.550000in}}{\pgfqpoint{3.875000in}{3.850000in}} %
\pgfusepath{clip}%
\pgfsetbuttcap%
\pgfsetroundjoin%
\pgfsetlinewidth{0.501875pt}%
\definecolor{currentstroke}{rgb}{0.000000,0.000000,0.000000}%
\pgfsetstrokecolor{currentstroke}%
\pgfsetdash{}{0pt}%
\pgfpathmoveto{\pgfqpoint{0.625000in}{3.379344in}}%
\pgfpathlineto{\pgfqpoint{0.628273in}{3.377193in}}%
\pgfpathlineto{\pgfqpoint{0.625000in}{3.370238in}}%
\pgfusepath{stroke}%
\end{pgfscope}%
\begin{pgfscope}%
\pgfpathrectangle{\pgfqpoint{0.625000in}{0.550000in}}{\pgfqpoint{3.875000in}{3.850000in}} %
\pgfusepath{clip}%
\pgfsetbuttcap%
\pgfsetroundjoin%
\pgfsetlinewidth{0.501875pt}%
\definecolor{currentstroke}{rgb}{0.000000,0.000000,0.000000}%
\pgfsetstrokecolor{currentstroke}%
\pgfsetdash{}{0pt}%
\pgfpathmoveto{\pgfqpoint{0.625000in}{3.461272in}}%
\pgfpathlineto{\pgfqpoint{0.628445in}{3.464035in}}%
\pgfpathlineto{\pgfqpoint{0.625000in}{3.469034in}}%
\pgfusepath{stroke}%
\end{pgfscope}%
\begin{pgfscope}%
\pgfpathrectangle{\pgfqpoint{0.625000in}{0.550000in}}{\pgfqpoint{3.875000in}{3.850000in}} %
\pgfusepath{clip}%
\pgfsetbuttcap%
\pgfsetroundjoin%
\pgfsetlinewidth{0.501875pt}%
\definecolor{currentstroke}{rgb}{0.000000,0.000000,0.000000}%
\pgfsetstrokecolor{currentstroke}%
\pgfsetdash{}{0pt}%
\pgfpathmoveto{\pgfqpoint{0.625000in}{3.494979in}}%
\pgfpathlineto{\pgfqpoint{0.627614in}{3.502632in}}%
\pgfpathlineto{\pgfqpoint{0.630503in}{3.512281in}}%
\pgfpathlineto{\pgfqpoint{0.627713in}{3.521930in}}%
\pgfpathlineto{\pgfqpoint{0.633038in}{3.531579in}}%
\pgfpathlineto{\pgfqpoint{0.630221in}{3.541228in}}%
\pgfpathlineto{\pgfqpoint{0.629862in}{3.550877in}}%
\pgfpathlineto{\pgfqpoint{0.625000in}{3.560034in}}%
\pgfusepath{stroke}%
\end{pgfscope}%
\begin{pgfscope}%
\pgfpathrectangle{\pgfqpoint{0.625000in}{0.550000in}}{\pgfqpoint{3.875000in}{3.850000in}} %
\pgfusepath{clip}%
\pgfsetbuttcap%
\pgfsetroundjoin%
\pgfsetlinewidth{0.501875pt}%
\definecolor{currentstroke}{rgb}{0.000000,0.000000,0.000000}%
\pgfsetstrokecolor{currentstroke}%
\pgfsetdash{}{0pt}%
\pgfpathmoveto{\pgfqpoint{0.625000in}{3.561110in}}%
\pgfpathlineto{\pgfqpoint{0.634712in}{3.564501in}}%
\pgfpathlineto{\pgfqpoint{0.638938in}{3.570175in}}%
\pgfpathlineto{\pgfqpoint{0.634712in}{3.577213in}}%
\pgfpathlineto{\pgfqpoint{0.634163in}{3.579825in}}%
\pgfpathlineto{\pgfqpoint{0.634712in}{3.580902in}}%
\pgfpathlineto{\pgfqpoint{0.644106in}{3.589474in}}%
\pgfpathlineto{\pgfqpoint{0.644424in}{3.592165in}}%
\pgfpathlineto{\pgfqpoint{0.647419in}{3.599123in}}%
\pgfpathlineto{\pgfqpoint{0.644424in}{3.604598in}}%
\pgfpathlineto{\pgfqpoint{0.638473in}{3.608772in}}%
\pgfpathlineto{\pgfqpoint{0.634712in}{3.613119in}}%
\pgfpathlineto{\pgfqpoint{0.633901in}{3.608772in}}%
\pgfpathlineto{\pgfqpoint{0.625000in}{3.602358in}}%
\pgfusepath{stroke}%
\end{pgfscope}%
\begin{pgfscope}%
\pgfpathrectangle{\pgfqpoint{0.625000in}{0.550000in}}{\pgfqpoint{3.875000in}{3.850000in}} %
\pgfusepath{clip}%
\pgfsetbuttcap%
\pgfsetroundjoin%
\pgfsetlinewidth{0.501875pt}%
\definecolor{currentstroke}{rgb}{0.000000,0.000000,0.000000}%
\pgfsetstrokecolor{currentstroke}%
\pgfsetdash{}{0pt}%
\pgfpathmoveto{\pgfqpoint{0.625000in}{3.615186in}}%
\pgfpathlineto{\pgfqpoint{0.632958in}{3.618421in}}%
\pgfpathlineto{\pgfqpoint{0.633080in}{3.628070in}}%
\pgfpathlineto{\pgfqpoint{0.625000in}{3.634079in}}%
\pgfusepath{stroke}%
\end{pgfscope}%
\begin{pgfscope}%
\pgfpathrectangle{\pgfqpoint{0.625000in}{0.550000in}}{\pgfqpoint{3.875000in}{3.850000in}} %
\pgfusepath{clip}%
\pgfsetbuttcap%
\pgfsetroundjoin%
\pgfsetlinewidth{0.501875pt}%
\definecolor{currentstroke}{rgb}{0.000000,0.000000,0.000000}%
\pgfsetstrokecolor{currentstroke}%
\pgfsetdash{}{0pt}%
\pgfpathmoveto{\pgfqpoint{0.625000in}{3.653142in}}%
\pgfpathlineto{\pgfqpoint{0.633692in}{3.657018in}}%
\pgfpathlineto{\pgfqpoint{0.633858in}{3.666667in}}%
\pgfpathlineto{\pgfqpoint{0.629892in}{3.676316in}}%
\pgfpathlineto{\pgfqpoint{0.634712in}{3.681105in}}%
\pgfpathlineto{\pgfqpoint{0.644424in}{3.677885in}}%
\pgfpathlineto{\pgfqpoint{0.654135in}{3.678961in}}%
\pgfpathlineto{\pgfqpoint{0.666201in}{3.685965in}}%
\pgfpathlineto{\pgfqpoint{0.675181in}{3.695614in}}%
\pgfpathlineto{\pgfqpoint{0.680045in}{3.705263in}}%
\pgfpathlineto{\pgfqpoint{0.682054in}{3.714912in}}%
\pgfpathlineto{\pgfqpoint{0.681836in}{3.724561in}}%
\pgfpathlineto{\pgfqpoint{0.679473in}{3.734211in}}%
\pgfpathlineto{\pgfqpoint{0.673559in}{3.744030in}}%
\pgfpathlineto{\pgfqpoint{0.663847in}{3.754323in}}%
\pgfpathlineto{\pgfqpoint{0.647274in}{3.763158in}}%
\pgfpathlineto{\pgfqpoint{0.654135in}{3.766378in}}%
\pgfpathlineto{\pgfqpoint{0.731899in}{3.772807in}}%
\pgfpathlineto{\pgfqpoint{0.760965in}{3.778421in}}%
\pgfpathlineto{\pgfqpoint{0.780388in}{3.783911in}}%
\pgfpathlineto{\pgfqpoint{0.800878in}{3.792105in}}%
\pgfpathlineto{\pgfqpoint{0.819236in}{3.803216in}}%
\pgfpathlineto{\pgfqpoint{0.829033in}{3.811404in}}%
\pgfpathlineto{\pgfqpoint{0.838659in}{3.822581in}}%
\pgfpathlineto{\pgfqpoint{0.844071in}{3.830702in}}%
\pgfpathlineto{\pgfqpoint{0.851841in}{3.850000in}}%
\pgfpathlineto{\pgfqpoint{0.854899in}{3.869298in}}%
\pgfpathlineto{\pgfqpoint{0.854326in}{3.888596in}}%
\pgfpathlineto{\pgfqpoint{0.850610in}{3.907895in}}%
\pgfpathlineto{\pgfqpoint{0.844040in}{3.927193in}}%
\pgfpathlineto{\pgfqpoint{0.834598in}{3.946491in}}%
\pgfpathlineto{\pgfqpoint{0.819236in}{3.970056in}}%
\pgfpathlineto{\pgfqpoint{0.799262in}{3.994737in}}%
\pgfpathlineto{\pgfqpoint{0.784230in}{4.014035in}}%
\pgfpathlineto{\pgfqpoint{0.778818in}{4.023684in}}%
\pgfpathlineto{\pgfqpoint{0.772083in}{4.042982in}}%
\pgfpathlineto{\pgfqpoint{0.758656in}{4.091228in}}%
\pgfpathlineto{\pgfqpoint{0.746375in}{4.120175in}}%
\pgfpathlineto{\pgfqpoint{0.729392in}{4.158772in}}%
\pgfpathlineto{\pgfqpoint{0.697461in}{4.235965in}}%
\pgfpathlineto{\pgfqpoint{0.629174in}{4.400000in}}%
\pgfpathlineto{\pgfqpoint{0.629174in}{4.400000in}}%
\pgfusepath{stroke}%
\end{pgfscope}%
\begin{pgfscope}%
\pgfpathrectangle{\pgfqpoint{0.625000in}{0.550000in}}{\pgfqpoint{3.875000in}{3.850000in}} %
\pgfusepath{clip}%
\pgfsetbuttcap%
\pgfsetroundjoin%
\pgfsetlinewidth{0.501875pt}%
\definecolor{currentstroke}{rgb}{0.000000,0.000000,0.000000}%
\pgfsetstrokecolor{currentstroke}%
\pgfsetdash{}{0pt}%
\pgfpathmoveto{\pgfqpoint{0.625000in}{3.745059in}}%
\pgfpathlineto{\pgfqpoint{0.625532in}{3.743860in}}%
\pgfpathlineto{\pgfqpoint{0.625000in}{3.743710in}}%
\pgfusepath{stroke}%
\end{pgfscope}%
\begin{pgfscope}%
\pgfpathrectangle{\pgfqpoint{0.625000in}{0.550000in}}{\pgfqpoint{3.875000in}{3.850000in}} %
\pgfusepath{clip}%
\pgfsetbuttcap%
\pgfsetroundjoin%
\pgfsetlinewidth{0.501875pt}%
\definecolor{currentstroke}{rgb}{0.000000,0.000000,0.000000}%
\pgfsetstrokecolor{currentstroke}%
\pgfsetdash{}{0pt}%
\pgfpathmoveto{\pgfqpoint{0.625000in}{3.795349in}}%
\pgfpathlineto{\pgfqpoint{0.634116in}{3.792105in}}%
\pgfpathlineto{\pgfqpoint{0.625000in}{3.789710in}}%
\pgfusepath{stroke}%
\end{pgfscope}%
\begin{pgfscope}%
\pgfpathrectangle{\pgfqpoint{0.625000in}{0.550000in}}{\pgfqpoint{3.875000in}{3.850000in}} %
\pgfusepath{clip}%
\pgfsetbuttcap%
\pgfsetroundjoin%
\pgfsetlinewidth{0.501875pt}%
\definecolor{currentstroke}{rgb}{0.000000,0.000000,0.000000}%
\pgfsetstrokecolor{currentstroke}%
\pgfsetdash{}{0pt}%
\pgfpathmoveto{\pgfqpoint{0.625000in}{3.840769in}}%
\pgfpathlineto{\pgfqpoint{0.625511in}{3.840351in}}%
\pgfpathlineto{\pgfqpoint{0.625000in}{3.839933in}}%
\pgfusepath{stroke}%
\end{pgfscope}%
\begin{pgfscope}%
\pgfpathrectangle{\pgfqpoint{0.625000in}{0.550000in}}{\pgfqpoint{3.875000in}{3.850000in}} %
\pgfusepath{clip}%
\pgfsetbuttcap%
\pgfsetroundjoin%
\pgfsetlinewidth{0.501875pt}%
\definecolor{currentstroke}{rgb}{0.000000,0.000000,0.000000}%
\pgfsetstrokecolor{currentstroke}%
\pgfsetdash{}{0pt}%
\pgfpathmoveto{\pgfqpoint{0.625000in}{3.879137in}}%
\pgfpathlineto{\pgfqpoint{0.625756in}{3.878947in}}%
\pgfpathlineto{\pgfqpoint{0.625000in}{3.878747in}}%
\pgfusepath{stroke}%
\end{pgfscope}%
\begin{pgfscope}%
\pgfpathrectangle{\pgfqpoint{0.625000in}{0.550000in}}{\pgfqpoint{3.875000in}{3.850000in}} %
\pgfusepath{clip}%
\pgfsetbuttcap%
\pgfsetroundjoin%
\pgfsetlinewidth{0.501875pt}%
\definecolor{currentstroke}{rgb}{0.000000,0.000000,0.000000}%
\pgfsetstrokecolor{currentstroke}%
\pgfsetdash{}{0pt}%
\pgfpathmoveto{\pgfqpoint{0.625000in}{3.921533in}}%
\pgfpathlineto{\pgfqpoint{0.626848in}{3.917544in}}%
\pgfpathlineto{\pgfqpoint{0.625000in}{3.913555in}}%
\pgfusepath{stroke}%
\end{pgfscope}%
\begin{pgfscope}%
\pgfpathrectangle{\pgfqpoint{0.625000in}{0.550000in}}{\pgfqpoint{3.875000in}{3.850000in}} %
\pgfusepath{clip}%
\pgfsetbuttcap%
\pgfsetroundjoin%
\pgfsetlinewidth{0.501875pt}%
\definecolor{currentstroke}{rgb}{0.000000,0.000000,0.000000}%
\pgfsetstrokecolor{currentstroke}%
\pgfsetdash{}{0pt}%
\pgfpathmoveto{\pgfqpoint{0.625000in}{3.954772in}}%
\pgfpathlineto{\pgfqpoint{0.626673in}{3.956140in}}%
\pgfpathlineto{\pgfqpoint{0.628970in}{3.965789in}}%
\pgfpathlineto{\pgfqpoint{0.626908in}{3.975439in}}%
\pgfpathlineto{\pgfqpoint{0.628325in}{3.985088in}}%
\pgfpathlineto{\pgfqpoint{0.625000in}{3.987865in}}%
\pgfusepath{stroke}%
\end{pgfscope}%
\begin{pgfscope}%
\pgfpathrectangle{\pgfqpoint{0.625000in}{0.550000in}}{\pgfqpoint{3.875000in}{3.850000in}} %
\pgfusepath{clip}%
\pgfsetbuttcap%
\pgfsetroundjoin%
\pgfsetlinewidth{0.501875pt}%
\definecolor{currentstroke}{rgb}{0.000000,0.000000,0.000000}%
\pgfsetstrokecolor{currentstroke}%
\pgfsetdash{}{0pt}%
\pgfpathmoveto{\pgfqpoint{0.625000in}{3.996818in}}%
\pgfpathlineto{\pgfqpoint{0.633987in}{3.994737in}}%
\pgfpathlineto{\pgfqpoint{0.634712in}{3.993829in}}%
\pgfpathlineto{\pgfqpoint{0.635990in}{3.994737in}}%
\pgfpathlineto{\pgfqpoint{0.643936in}{4.004386in}}%
\pgfpathlineto{\pgfqpoint{0.643339in}{4.014035in}}%
\pgfpathlineto{\pgfqpoint{0.644424in}{4.022050in}}%
\pgfpathlineto{\pgfqpoint{0.654135in}{4.022874in}}%
\pgfpathlineto{\pgfqpoint{0.663847in}{4.022151in}}%
\pgfpathlineto{\pgfqpoint{0.673559in}{4.021036in}}%
\pgfpathlineto{\pgfqpoint{0.683271in}{4.019361in}}%
\pgfpathlineto{\pgfqpoint{0.692982in}{4.016881in}}%
\pgfpathlineto{\pgfqpoint{0.701198in}{4.014035in}}%
\pgfpathlineto{\pgfqpoint{0.702694in}{4.012987in}}%
\pgfpathlineto{\pgfqpoint{0.712406in}{4.005787in}}%
\pgfpathlineto{\pgfqpoint{0.714175in}{4.004386in}}%
\pgfpathlineto{\pgfqpoint{0.718189in}{3.994737in}}%
\pgfpathlineto{\pgfqpoint{0.716793in}{3.985088in}}%
\pgfpathlineto{\pgfqpoint{0.712406in}{3.977907in}}%
\pgfpathlineto{\pgfqpoint{0.711364in}{3.975439in}}%
\pgfpathlineto{\pgfqpoint{0.702694in}{3.965929in}}%
\pgfpathlineto{\pgfqpoint{0.702588in}{3.965789in}}%
\pgfpathlineto{\pgfqpoint{0.692982in}{3.958392in}}%
\pgfpathlineto{\pgfqpoint{0.690146in}{3.956140in}}%
\pgfpathlineto{\pgfqpoint{0.683271in}{3.952447in}}%
\pgfpathlineto{\pgfqpoint{0.673559in}{3.947495in}}%
\pgfpathlineto{\pgfqpoint{0.671726in}{3.946491in}}%
\pgfpathlineto{\pgfqpoint{0.663847in}{3.943366in}}%
\pgfpathlineto{\pgfqpoint{0.654135in}{3.939409in}}%
\pgfpathlineto{\pgfqpoint{0.649279in}{3.936842in}}%
\pgfpathlineto{\pgfqpoint{0.644424in}{3.935366in}}%
\pgfpathlineto{\pgfqpoint{0.634712in}{3.932018in}}%
\pgfpathlineto{\pgfqpoint{0.625000in}{3.932393in}}%
\pgfusepath{stroke}%
\end{pgfscope}%
\begin{pgfscope}%
\pgfpathrectangle{\pgfqpoint{0.625000in}{0.550000in}}{\pgfqpoint{3.875000in}{3.850000in}} %
\pgfusepath{clip}%
\pgfsetbuttcap%
\pgfsetroundjoin%
\pgfsetlinewidth{0.501875pt}%
\definecolor{currentstroke}{rgb}{0.000000,0.000000,0.000000}%
\pgfsetstrokecolor{currentstroke}%
\pgfsetdash{}{0pt}%
\pgfpathmoveto{\pgfqpoint{0.625000in}{4.035680in}}%
\pgfpathlineto{\pgfqpoint{0.626911in}{4.033333in}}%
\pgfpathlineto{\pgfqpoint{0.625000in}{4.029377in}}%
\pgfusepath{stroke}%
\end{pgfscope}%
\begin{pgfscope}%
\pgfpathrectangle{\pgfqpoint{0.625000in}{0.550000in}}{\pgfqpoint{3.875000in}{3.850000in}} %
\pgfusepath{clip}%
\pgfsetbuttcap%
\pgfsetroundjoin%
\pgfsetlinewidth{0.501875pt}%
\definecolor{currentstroke}{rgb}{0.000000,0.000000,0.000000}%
\pgfsetstrokecolor{currentstroke}%
\pgfsetdash{}{0pt}%
\pgfpathmoveto{\pgfqpoint{0.625000in}{4.109952in}}%
\pgfpathlineto{\pgfqpoint{0.625754in}{4.110526in}}%
\pgfpathlineto{\pgfqpoint{0.629409in}{4.120175in}}%
\pgfpathlineto{\pgfqpoint{0.625000in}{4.123453in}}%
\pgfusepath{stroke}%
\end{pgfscope}%
\begin{pgfscope}%
\pgfpathrectangle{\pgfqpoint{0.625000in}{0.550000in}}{\pgfqpoint{3.875000in}{3.850000in}} %
\pgfusepath{clip}%
\pgfsetbuttcap%
\pgfsetroundjoin%
\pgfsetlinewidth{0.501875pt}%
\definecolor{currentstroke}{rgb}{0.000000,0.000000,0.000000}%
\pgfsetstrokecolor{currentstroke}%
\pgfsetdash{}{0pt}%
\pgfpathmoveto{\pgfqpoint{0.625000in}{4.136196in}}%
\pgfpathlineto{\pgfqpoint{0.634712in}{4.131768in}}%
\pgfpathlineto{\pgfqpoint{0.644424in}{4.132464in}}%
\pgfpathlineto{\pgfqpoint{0.653778in}{4.129825in}}%
\pgfpathlineto{\pgfqpoint{0.654135in}{4.129372in}}%
\pgfpathlineto{\pgfqpoint{0.663749in}{4.120175in}}%
\pgfpathlineto{\pgfqpoint{0.663193in}{4.110526in}}%
\pgfpathlineto{\pgfqpoint{0.654135in}{4.101155in}}%
\pgfpathlineto{\pgfqpoint{0.653874in}{4.100877in}}%
\pgfpathlineto{\pgfqpoint{0.644424in}{4.096386in}}%
\pgfpathlineto{\pgfqpoint{0.634712in}{4.092373in}}%
\pgfpathlineto{\pgfqpoint{0.633949in}{4.091228in}}%
\pgfpathlineto{\pgfqpoint{0.625000in}{4.084673in}}%
\pgfusepath{stroke}%
\end{pgfscope}%
\begin{pgfscope}%
\pgfpathrectangle{\pgfqpoint{0.625000in}{0.550000in}}{\pgfqpoint{3.875000in}{3.850000in}} %
\pgfusepath{clip}%
\pgfsetbuttcap%
\pgfsetroundjoin%
\pgfsetlinewidth{0.501875pt}%
\definecolor{currentstroke}{rgb}{0.000000,0.000000,0.000000}%
\pgfsetstrokecolor{currentstroke}%
\pgfsetdash{}{0pt}%
\pgfpathmoveto{\pgfqpoint{0.625000in}{4.229845in}}%
\pgfpathlineto{\pgfqpoint{0.634712in}{4.230062in}}%
\pgfpathlineto{\pgfqpoint{0.641187in}{4.226316in}}%
\pgfpathlineto{\pgfqpoint{0.634712in}{4.221863in}}%
\pgfpathlineto{\pgfqpoint{0.625000in}{4.222787in}}%
\pgfusepath{stroke}%
\end{pgfscope}%
\begin{pgfscope}%
\pgfpathrectangle{\pgfqpoint{0.625000in}{0.550000in}}{\pgfqpoint{3.875000in}{3.850000in}} %
\pgfusepath{clip}%
\pgfsetbuttcap%
\pgfsetroundjoin%
\pgfsetlinewidth{0.501875pt}%
\definecolor{currentstroke}{rgb}{0.000000,0.000000,0.000000}%
\pgfsetstrokecolor{currentstroke}%
\pgfsetdash{}{0pt}%
\pgfpathmoveto{\pgfqpoint{0.625000in}{4.258080in}}%
\pgfpathlineto{\pgfqpoint{0.634712in}{4.255274in}}%
\pgfpathlineto{\pgfqpoint{0.634736in}{4.255263in}}%
\pgfpathlineto{\pgfqpoint{0.634712in}{4.255254in}}%
\pgfpathlineto{\pgfqpoint{0.625000in}{4.249568in}}%
\pgfusepath{stroke}%
\end{pgfscope}%
\begin{pgfscope}%
\pgfpathrectangle{\pgfqpoint{0.625000in}{0.550000in}}{\pgfqpoint{3.875000in}{3.850000in}} %
\pgfusepath{clip}%
\pgfsetbuttcap%
\pgfsetroundjoin%
\pgfsetlinewidth{0.501875pt}%
\definecolor{currentstroke}{rgb}{0.000000,0.000000,0.000000}%
\pgfsetstrokecolor{currentstroke}%
\pgfsetdash{}{0pt}%
\pgfpathmoveto{\pgfqpoint{0.625000in}{4.309349in}}%
\pgfpathlineto{\pgfqpoint{0.631485in}{4.303509in}}%
\pgfpathlineto{\pgfqpoint{0.625000in}{4.297668in}}%
\pgfusepath{stroke}%
\end{pgfscope}%
\begin{pgfscope}%
\pgfpathrectangle{\pgfqpoint{0.625000in}{0.550000in}}{\pgfqpoint{3.875000in}{3.850000in}} %
\pgfusepath{clip}%
\pgfsetbuttcap%
\pgfsetroundjoin%
\pgfsetlinewidth{0.501875pt}%
\definecolor{currentstroke}{rgb}{0.000000,0.000000,0.000000}%
\pgfsetstrokecolor{currentstroke}%
\pgfsetdash{}{0pt}%
\pgfpathmoveto{\pgfqpoint{0.625000in}{4.342689in}}%
\pgfpathlineto{\pgfqpoint{0.629622in}{4.342105in}}%
\pgfpathlineto{\pgfqpoint{0.625376in}{4.332456in}}%
\pgfpathlineto{\pgfqpoint{0.630063in}{4.322807in}}%
\pgfpathlineto{\pgfqpoint{0.625000in}{4.318570in}}%
\pgfusepath{stroke}%
\end{pgfscope}%
\begin{pgfscope}%
\pgfpathrectangle{\pgfqpoint{0.625000in}{0.550000in}}{\pgfqpoint{3.875000in}{3.850000in}} %
\pgfusepath{clip}%
\pgfsetbuttcap%
\pgfsetroundjoin%
\pgfsetlinewidth{0.501875pt}%
\definecolor{currentstroke}{rgb}{0.000000,0.000000,0.000000}%
\pgfsetstrokecolor{currentstroke}%
\pgfsetdash{}{0pt}%
\pgfpathmoveto{\pgfqpoint{0.634712in}{0.766629in}}%
\pgfpathlineto{\pgfqpoint{0.631756in}{0.771930in}}%
\pgfpathlineto{\pgfqpoint{0.632473in}{0.781579in}}%
\pgfpathlineto{\pgfqpoint{0.634712in}{0.786576in}}%
\pgfpathlineto{\pgfqpoint{0.644247in}{0.781579in}}%
\pgfpathlineto{\pgfqpoint{0.644424in}{0.780551in}}%
\pgfpathlineto{\pgfqpoint{0.647476in}{0.771930in}}%
\pgfpathlineto{\pgfqpoint{0.644424in}{0.770290in}}%
\pgfpathlineto{\pgfqpoint{0.634712in}{0.766629in}}%
\pgfusepath{stroke}%
\end{pgfscope}%
\begin{pgfscope}%
\pgfpathrectangle{\pgfqpoint{0.625000in}{0.550000in}}{\pgfqpoint{3.875000in}{3.850000in}} %
\pgfusepath{clip}%
\pgfsetbuttcap%
\pgfsetroundjoin%
\pgfsetlinewidth{0.501875pt}%
\definecolor{currentstroke}{rgb}{0.000000,0.000000,0.000000}%
\pgfsetstrokecolor{currentstroke}%
\pgfsetdash{}{0pt}%
\pgfpathmoveto{\pgfqpoint{0.644424in}{1.714180in}}%
\pgfpathlineto{\pgfqpoint{0.641671in}{1.717544in}}%
\pgfpathlineto{\pgfqpoint{0.636484in}{1.727193in}}%
\pgfpathlineto{\pgfqpoint{0.634712in}{1.728763in}}%
\pgfpathlineto{\pgfqpoint{0.628188in}{1.736842in}}%
\pgfpathlineto{\pgfqpoint{0.629821in}{1.746491in}}%
\pgfpathlineto{\pgfqpoint{0.634712in}{1.751349in}}%
\pgfpathlineto{\pgfqpoint{0.644424in}{1.747886in}}%
\pgfpathlineto{\pgfqpoint{0.646530in}{1.746491in}}%
\pgfpathlineto{\pgfqpoint{0.653711in}{1.736842in}}%
\pgfpathlineto{\pgfqpoint{0.654135in}{1.729845in}}%
\pgfpathlineto{\pgfqpoint{0.654313in}{1.727193in}}%
\pgfpathlineto{\pgfqpoint{0.654135in}{1.726956in}}%
\pgfpathlineto{\pgfqpoint{0.650648in}{1.717544in}}%
\pgfpathlineto{\pgfqpoint{0.644424in}{1.714180in}}%
\pgfusepath{stroke}%
\end{pgfscope}%
\begin{pgfscope}%
\pgfpathrectangle{\pgfqpoint{0.625000in}{0.550000in}}{\pgfqpoint{3.875000in}{3.850000in}} %
\pgfusepath{clip}%
\pgfsetbuttcap%
\pgfsetroundjoin%
\pgfsetlinewidth{0.501875pt}%
\definecolor{currentstroke}{rgb}{0.000000,0.000000,0.000000}%
\pgfsetstrokecolor{currentstroke}%
\pgfsetdash{}{0pt}%
\pgfpathmoveto{\pgfqpoint{0.634712in}{2.361321in}}%
\pgfpathlineto{\pgfqpoint{0.629016in}{2.364035in}}%
\pgfpathlineto{\pgfqpoint{0.634712in}{2.368957in}}%
\pgfpathlineto{\pgfqpoint{0.643507in}{2.373684in}}%
\pgfpathlineto{\pgfqpoint{0.644424in}{2.373893in}}%
\pgfpathlineto{\pgfqpoint{0.654135in}{2.378085in}}%
\pgfpathlineto{\pgfqpoint{0.661385in}{2.383333in}}%
\pgfpathlineto{\pgfqpoint{0.663847in}{2.385252in}}%
\pgfpathlineto{\pgfqpoint{0.670707in}{2.392982in}}%
\pgfpathlineto{\pgfqpoint{0.673559in}{2.397164in}}%
\pgfpathlineto{\pgfqpoint{0.678182in}{2.392982in}}%
\pgfpathlineto{\pgfqpoint{0.673911in}{2.383333in}}%
\pgfpathlineto{\pgfqpoint{0.673559in}{2.382877in}}%
\pgfpathlineto{\pgfqpoint{0.664161in}{2.373684in}}%
\pgfpathlineto{\pgfqpoint{0.663847in}{2.373411in}}%
\pgfpathlineto{\pgfqpoint{0.654135in}{2.367729in}}%
\pgfpathlineto{\pgfqpoint{0.645870in}{2.364035in}}%
\pgfpathlineto{\pgfqpoint{0.644424in}{2.363459in}}%
\pgfpathlineto{\pgfqpoint{0.634712in}{2.361321in}}%
\pgfusepath{stroke}%
\end{pgfscope}%
\begin{pgfscope}%
\pgfpathrectangle{\pgfqpoint{0.625000in}{0.550000in}}{\pgfqpoint{3.875000in}{3.850000in}} %
\pgfusepath{clip}%
\pgfsetbuttcap%
\pgfsetroundjoin%
\pgfsetlinewidth{0.501875pt}%
\definecolor{currentstroke}{rgb}{0.000000,0.000000,0.000000}%
\pgfsetstrokecolor{currentstroke}%
\pgfsetdash{}{0pt}%
\pgfpathmoveto{\pgfqpoint{0.683271in}{2.400264in}}%
\pgfpathlineto{\pgfqpoint{0.681183in}{2.402632in}}%
\pgfpathlineto{\pgfqpoint{0.680353in}{2.412281in}}%
\pgfpathlineto{\pgfqpoint{0.681349in}{2.421930in}}%
\pgfpathlineto{\pgfqpoint{0.681298in}{2.431579in}}%
\pgfpathlineto{\pgfqpoint{0.683271in}{2.438770in}}%
\pgfpathlineto{\pgfqpoint{0.685675in}{2.431579in}}%
\pgfpathlineto{\pgfqpoint{0.685885in}{2.421930in}}%
\pgfpathlineto{\pgfqpoint{0.686325in}{2.412281in}}%
\pgfpathlineto{\pgfqpoint{0.684136in}{2.402632in}}%
\pgfpathlineto{\pgfqpoint{0.683271in}{2.400264in}}%
\pgfusepath{stroke}%
\end{pgfscope}%
\begin{pgfscope}%
\pgfpathrectangle{\pgfqpoint{0.625000in}{0.550000in}}{\pgfqpoint{3.875000in}{3.850000in}} %
\pgfusepath{clip}%
\pgfsetbuttcap%
\pgfsetroundjoin%
\pgfsetlinewidth{0.501875pt}%
\definecolor{currentstroke}{rgb}{0.000000,0.000000,0.000000}%
\pgfsetstrokecolor{currentstroke}%
\pgfsetdash{}{0pt}%
\pgfpathmoveto{\pgfqpoint{0.654135in}{2.411656in}}%
\pgfpathlineto{\pgfqpoint{0.652703in}{2.412281in}}%
\pgfpathlineto{\pgfqpoint{0.654135in}{2.413316in}}%
\pgfpathlineto{\pgfqpoint{0.654695in}{2.412281in}}%
\pgfpathlineto{\pgfqpoint{0.654135in}{2.411656in}}%
\pgfusepath{stroke}%
\end{pgfscope}%
\begin{pgfscope}%
\pgfpathrectangle{\pgfqpoint{0.625000in}{0.550000in}}{\pgfqpoint{3.875000in}{3.850000in}} %
\pgfusepath{clip}%
\pgfsetbuttcap%
\pgfsetroundjoin%
\pgfsetlinewidth{0.501875pt}%
\definecolor{currentstroke}{rgb}{0.000000,0.000000,0.000000}%
\pgfsetstrokecolor{currentstroke}%
\pgfsetdash{}{0pt}%
\pgfpathmoveto{\pgfqpoint{0.634712in}{2.420874in}}%
\pgfpathlineto{\pgfqpoint{0.629517in}{2.421930in}}%
\pgfpathlineto{\pgfqpoint{0.634712in}{2.423353in}}%
\pgfpathlineto{\pgfqpoint{0.635909in}{2.421930in}}%
\pgfpathlineto{\pgfqpoint{0.634712in}{2.420874in}}%
\pgfusepath{stroke}%
\end{pgfscope}%
\begin{pgfscope}%
\pgfpathrectangle{\pgfqpoint{0.625000in}{0.550000in}}{\pgfqpoint{3.875000in}{3.850000in}} %
\pgfusepath{clip}%
\pgfsetbuttcap%
\pgfsetroundjoin%
\pgfsetlinewidth{0.501875pt}%
\definecolor{currentstroke}{rgb}{0.000000,0.000000,0.000000}%
\pgfsetstrokecolor{currentstroke}%
\pgfsetdash{}{0pt}%
\pgfpathmoveto{\pgfqpoint{0.663847in}{2.427561in}}%
\pgfpathlineto{\pgfqpoint{0.662041in}{2.431579in}}%
\pgfpathlineto{\pgfqpoint{0.654135in}{2.440648in}}%
\pgfpathlineto{\pgfqpoint{0.644424in}{2.440801in}}%
\pgfpathlineto{\pgfqpoint{0.644018in}{2.441228in}}%
\pgfpathlineto{\pgfqpoint{0.644424in}{2.441475in}}%
\pgfpathlineto{\pgfqpoint{0.654135in}{2.447569in}}%
\pgfpathlineto{\pgfqpoint{0.656045in}{2.450877in}}%
\pgfpathlineto{\pgfqpoint{0.663847in}{2.451875in}}%
\pgfpathlineto{\pgfqpoint{0.667547in}{2.450877in}}%
\pgfpathlineto{\pgfqpoint{0.664787in}{2.441228in}}%
\pgfpathlineto{\pgfqpoint{0.665514in}{2.431579in}}%
\pgfpathlineto{\pgfqpoint{0.663847in}{2.427561in}}%
\pgfusepath{stroke}%
\end{pgfscope}%
\begin{pgfscope}%
\pgfpathrectangle{\pgfqpoint{0.625000in}{0.550000in}}{\pgfqpoint{3.875000in}{3.850000in}} %
\pgfusepath{clip}%
\pgfsetbuttcap%
\pgfsetroundjoin%
\pgfsetlinewidth{0.501875pt}%
\definecolor{currentstroke}{rgb}{0.000000,0.000000,0.000000}%
\pgfsetstrokecolor{currentstroke}%
\pgfsetdash{}{0pt}%
\pgfpathmoveto{\pgfqpoint{0.702694in}{2.446662in}}%
\pgfpathlineto{\pgfqpoint{0.699232in}{2.450877in}}%
\pgfpathlineto{\pgfqpoint{0.702694in}{2.454034in}}%
\pgfpathlineto{\pgfqpoint{0.707400in}{2.450877in}}%
\pgfpathlineto{\pgfqpoint{0.702694in}{2.446662in}}%
\pgfusepath{stroke}%
\end{pgfscope}%
\begin{pgfscope}%
\pgfpathrectangle{\pgfqpoint{0.625000in}{0.550000in}}{\pgfqpoint{3.875000in}{3.850000in}} %
\pgfusepath{clip}%
\pgfsetbuttcap%
\pgfsetroundjoin%
\pgfsetlinewidth{0.501875pt}%
\definecolor{currentstroke}{rgb}{0.000000,0.000000,0.000000}%
\pgfsetstrokecolor{currentstroke}%
\pgfsetdash{}{0pt}%
\pgfpathmoveto{\pgfqpoint{0.692982in}{2.459732in}}%
\pgfpathlineto{\pgfqpoint{0.691982in}{2.460526in}}%
\pgfpathlineto{\pgfqpoint{0.692982in}{2.460879in}}%
\pgfpathlineto{\pgfqpoint{0.693692in}{2.460526in}}%
\pgfpathlineto{\pgfqpoint{0.692982in}{2.459732in}}%
\pgfusepath{stroke}%
\end{pgfscope}%
\begin{pgfscope}%
\pgfpathrectangle{\pgfqpoint{0.625000in}{0.550000in}}{\pgfqpoint{3.875000in}{3.850000in}} %
\pgfusepath{clip}%
\pgfsetbuttcap%
\pgfsetroundjoin%
\pgfsetlinewidth{0.501875pt}%
\definecolor{currentstroke}{rgb}{0.000000,0.000000,0.000000}%
\pgfsetstrokecolor{currentstroke}%
\pgfsetdash{}{0pt}%
\pgfpathmoveto{\pgfqpoint{0.634712in}{2.460567in}}%
\pgfpathlineto{\pgfqpoint{0.633285in}{2.470175in}}%
\pgfpathlineto{\pgfqpoint{0.634712in}{2.473105in}}%
\pgfpathlineto{\pgfqpoint{0.644424in}{2.472275in}}%
\pgfpathlineto{\pgfqpoint{0.654135in}{2.471265in}}%
\pgfpathlineto{\pgfqpoint{0.654923in}{2.470175in}}%
\pgfpathlineto{\pgfqpoint{0.654135in}{2.465497in}}%
\pgfpathlineto{\pgfqpoint{0.644424in}{2.462596in}}%
\pgfpathlineto{\pgfqpoint{0.634712in}{2.460567in}}%
\pgfusepath{stroke}%
\end{pgfscope}%
\begin{pgfscope}%
\pgfpathrectangle{\pgfqpoint{0.625000in}{0.550000in}}{\pgfqpoint{3.875000in}{3.850000in}} %
\pgfusepath{clip}%
\pgfsetbuttcap%
\pgfsetroundjoin%
\pgfsetlinewidth{0.501875pt}%
\definecolor{currentstroke}{rgb}{0.000000,0.000000,0.000000}%
\pgfsetstrokecolor{currentstroke}%
\pgfsetdash{}{0pt}%
\pgfpathmoveto{\pgfqpoint{0.634712in}{2.498778in}}%
\pgfpathlineto{\pgfqpoint{0.634276in}{2.499123in}}%
\pgfpathlineto{\pgfqpoint{0.634712in}{2.499521in}}%
\pgfpathlineto{\pgfqpoint{0.636331in}{2.499123in}}%
\pgfpathlineto{\pgfqpoint{0.634712in}{2.498778in}}%
\pgfusepath{stroke}%
\end{pgfscope}%
\begin{pgfscope}%
\pgfpathrectangle{\pgfqpoint{0.625000in}{0.550000in}}{\pgfqpoint{3.875000in}{3.850000in}} %
\pgfusepath{clip}%
\pgfsetbuttcap%
\pgfsetroundjoin%
\pgfsetlinewidth{0.501875pt}%
\definecolor{currentstroke}{rgb}{0.000000,0.000000,0.000000}%
\pgfsetstrokecolor{currentstroke}%
\pgfsetdash{}{0pt}%
\pgfpathmoveto{\pgfqpoint{0.654135in}{2.498470in}}%
\pgfpathlineto{\pgfqpoint{0.651362in}{2.499123in}}%
\pgfpathlineto{\pgfqpoint{0.654135in}{2.506037in}}%
\pgfpathlineto{\pgfqpoint{0.654785in}{2.499123in}}%
\pgfpathlineto{\pgfqpoint{0.654135in}{2.498470in}}%
\pgfusepath{stroke}%
\end{pgfscope}%
\begin{pgfscope}%
\pgfpathrectangle{\pgfqpoint{0.625000in}{0.550000in}}{\pgfqpoint{3.875000in}{3.850000in}} %
\pgfusepath{clip}%
\pgfsetbuttcap%
\pgfsetroundjoin%
\pgfsetlinewidth{0.501875pt}%
\definecolor{currentstroke}{rgb}{0.000000,0.000000,0.000000}%
\pgfsetstrokecolor{currentstroke}%
\pgfsetdash{}{0pt}%
\pgfpathmoveto{\pgfqpoint{0.692982in}{2.498770in}}%
\pgfpathlineto{\pgfqpoint{0.691982in}{2.499123in}}%
\pgfpathlineto{\pgfqpoint{0.692982in}{2.499917in}}%
\pgfpathlineto{\pgfqpoint{0.693692in}{2.499123in}}%
\pgfpathlineto{\pgfqpoint{0.692982in}{2.498770in}}%
\pgfusepath{stroke}%
\end{pgfscope}%
\begin{pgfscope}%
\pgfpathrectangle{\pgfqpoint{0.625000in}{0.550000in}}{\pgfqpoint{3.875000in}{3.850000in}} %
\pgfusepath{clip}%
\pgfsetbuttcap%
\pgfsetroundjoin%
\pgfsetlinewidth{0.501875pt}%
\definecolor{currentstroke}{rgb}{0.000000,0.000000,0.000000}%
\pgfsetstrokecolor{currentstroke}%
\pgfsetdash{}{0pt}%
\pgfpathmoveto{\pgfqpoint{0.663847in}{2.507771in}}%
\pgfpathlineto{\pgfqpoint{0.656047in}{2.508772in}}%
\pgfpathlineto{\pgfqpoint{0.654135in}{2.512083in}}%
\pgfpathlineto{\pgfqpoint{0.644424in}{2.518174in}}%
\pgfpathlineto{\pgfqpoint{0.644018in}{2.518421in}}%
\pgfpathlineto{\pgfqpoint{0.644424in}{2.518848in}}%
\pgfpathlineto{\pgfqpoint{0.654135in}{2.519002in}}%
\pgfpathlineto{\pgfqpoint{0.662041in}{2.528070in}}%
\pgfpathlineto{\pgfqpoint{0.663847in}{2.532088in}}%
\pgfpathlineto{\pgfqpoint{0.665514in}{2.528070in}}%
\pgfpathlineto{\pgfqpoint{0.664787in}{2.518421in}}%
\pgfpathlineto{\pgfqpoint{0.667547in}{2.508772in}}%
\pgfpathlineto{\pgfqpoint{0.663847in}{2.507771in}}%
\pgfusepath{stroke}%
\end{pgfscope}%
\begin{pgfscope}%
\pgfpathrectangle{\pgfqpoint{0.625000in}{0.550000in}}{\pgfqpoint{3.875000in}{3.850000in}} %
\pgfusepath{clip}%
\pgfsetbuttcap%
\pgfsetroundjoin%
\pgfsetlinewidth{0.501875pt}%
\definecolor{currentstroke}{rgb}{0.000000,0.000000,0.000000}%
\pgfsetstrokecolor{currentstroke}%
\pgfsetdash{}{0pt}%
\pgfpathmoveto{\pgfqpoint{0.702694in}{2.505615in}}%
\pgfpathlineto{\pgfqpoint{0.699232in}{2.508772in}}%
\pgfpathlineto{\pgfqpoint{0.702694in}{2.512987in}}%
\pgfpathlineto{\pgfqpoint{0.707400in}{2.508772in}}%
\pgfpathlineto{\pgfqpoint{0.702694in}{2.505615in}}%
\pgfusepath{stroke}%
\end{pgfscope}%
\begin{pgfscope}%
\pgfpathrectangle{\pgfqpoint{0.625000in}{0.550000in}}{\pgfqpoint{3.875000in}{3.850000in}} %
\pgfusepath{clip}%
\pgfsetbuttcap%
\pgfsetroundjoin%
\pgfsetlinewidth{0.501875pt}%
\definecolor{currentstroke}{rgb}{0.000000,0.000000,0.000000}%
\pgfsetstrokecolor{currentstroke}%
\pgfsetdash{}{0pt}%
\pgfpathmoveto{\pgfqpoint{0.683271in}{2.520879in}}%
\pgfpathlineto{\pgfqpoint{0.681298in}{2.528070in}}%
\pgfpathlineto{\pgfqpoint{0.681349in}{2.537719in}}%
\pgfpathlineto{\pgfqpoint{0.680353in}{2.547368in}}%
\pgfpathlineto{\pgfqpoint{0.681183in}{2.557018in}}%
\pgfpathlineto{\pgfqpoint{0.683271in}{2.559385in}}%
\pgfpathlineto{\pgfqpoint{0.684136in}{2.557018in}}%
\pgfpathlineto{\pgfqpoint{0.686325in}{2.547368in}}%
\pgfpathlineto{\pgfqpoint{0.685885in}{2.537719in}}%
\pgfpathlineto{\pgfqpoint{0.685675in}{2.528070in}}%
\pgfpathlineto{\pgfqpoint{0.683271in}{2.520879in}}%
\pgfusepath{stroke}%
\end{pgfscope}%
\begin{pgfscope}%
\pgfpathrectangle{\pgfqpoint{0.625000in}{0.550000in}}{\pgfqpoint{3.875000in}{3.850000in}} %
\pgfusepath{clip}%
\pgfsetbuttcap%
\pgfsetroundjoin%
\pgfsetlinewidth{0.501875pt}%
\definecolor{currentstroke}{rgb}{0.000000,0.000000,0.000000}%
\pgfsetstrokecolor{currentstroke}%
\pgfsetdash{}{0pt}%
\pgfpathmoveto{\pgfqpoint{0.634712in}{2.536297in}}%
\pgfpathlineto{\pgfqpoint{0.629517in}{2.537719in}}%
\pgfpathlineto{\pgfqpoint{0.634712in}{2.538775in}}%
\pgfpathlineto{\pgfqpoint{0.635909in}{2.537719in}}%
\pgfpathlineto{\pgfqpoint{0.634712in}{2.536297in}}%
\pgfusepath{stroke}%
\end{pgfscope}%
\begin{pgfscope}%
\pgfpathrectangle{\pgfqpoint{0.625000in}{0.550000in}}{\pgfqpoint{3.875000in}{3.850000in}} %
\pgfusepath{clip}%
\pgfsetbuttcap%
\pgfsetroundjoin%
\pgfsetlinewidth{0.501875pt}%
\definecolor{currentstroke}{rgb}{0.000000,0.000000,0.000000}%
\pgfsetstrokecolor{currentstroke}%
\pgfsetdash{}{0pt}%
\pgfpathmoveto{\pgfqpoint{0.654135in}{2.546333in}}%
\pgfpathlineto{\pgfqpoint{0.652703in}{2.547368in}}%
\pgfpathlineto{\pgfqpoint{0.654135in}{2.547993in}}%
\pgfpathlineto{\pgfqpoint{0.654695in}{2.547368in}}%
\pgfpathlineto{\pgfqpoint{0.654135in}{2.546333in}}%
\pgfusepath{stroke}%
\end{pgfscope}%
\begin{pgfscope}%
\pgfpathrectangle{\pgfqpoint{0.625000in}{0.550000in}}{\pgfqpoint{3.875000in}{3.850000in}} %
\pgfusepath{clip}%
\pgfsetbuttcap%
\pgfsetroundjoin%
\pgfsetlinewidth{0.501875pt}%
\definecolor{currentstroke}{rgb}{0.000000,0.000000,0.000000}%
\pgfsetstrokecolor{currentstroke}%
\pgfsetdash{}{0pt}%
\pgfpathmoveto{\pgfqpoint{0.673559in}{2.562485in}}%
\pgfpathlineto{\pgfqpoint{0.670707in}{2.566667in}}%
\pgfpathlineto{\pgfqpoint{0.663847in}{2.574397in}}%
\pgfpathlineto{\pgfqpoint{0.661385in}{2.576316in}}%
\pgfpathlineto{\pgfqpoint{0.654135in}{2.581565in}}%
\pgfpathlineto{\pgfqpoint{0.644424in}{2.585756in}}%
\pgfpathlineto{\pgfqpoint{0.643507in}{2.585965in}}%
\pgfpathlineto{\pgfqpoint{0.634712in}{2.590692in}}%
\pgfpathlineto{\pgfqpoint{0.629016in}{2.595614in}}%
\pgfpathlineto{\pgfqpoint{0.634712in}{2.598328in}}%
\pgfpathlineto{\pgfqpoint{0.644424in}{2.596190in}}%
\pgfpathlineto{\pgfqpoint{0.645870in}{2.595614in}}%
\pgfpathlineto{\pgfqpoint{0.654135in}{2.591920in}}%
\pgfpathlineto{\pgfqpoint{0.663847in}{2.586238in}}%
\pgfpathlineto{\pgfqpoint{0.664161in}{2.585965in}}%
\pgfpathlineto{\pgfqpoint{0.673559in}{2.576772in}}%
\pgfpathlineto{\pgfqpoint{0.673911in}{2.576316in}}%
\pgfpathlineto{\pgfqpoint{0.678182in}{2.566667in}}%
\pgfpathlineto{\pgfqpoint{0.673559in}{2.562485in}}%
\pgfusepath{stroke}%
\end{pgfscope}%
\begin{pgfscope}%
\pgfpathrectangle{\pgfqpoint{0.625000in}{0.550000in}}{\pgfqpoint{3.875000in}{3.850000in}} %
\pgfusepath{clip}%
\pgfsetbuttcap%
\pgfsetroundjoin%
\pgfsetlinewidth{0.501875pt}%
\definecolor{currentstroke}{rgb}{0.000000,0.000000,0.000000}%
\pgfsetstrokecolor{currentstroke}%
\pgfsetdash{}{0pt}%
\pgfpathmoveto{\pgfqpoint{0.634712in}{3.208300in}}%
\pgfpathlineto{\pgfqpoint{0.629821in}{3.213158in}}%
\pgfpathlineto{\pgfqpoint{0.628188in}{3.222807in}}%
\pgfpathlineto{\pgfqpoint{0.634712in}{3.230886in}}%
\pgfpathlineto{\pgfqpoint{0.636484in}{3.232456in}}%
\pgfpathlineto{\pgfqpoint{0.641671in}{3.242105in}}%
\pgfpathlineto{\pgfqpoint{0.644424in}{3.245469in}}%
\pgfpathlineto{\pgfqpoint{0.650648in}{3.242105in}}%
\pgfpathlineto{\pgfqpoint{0.654135in}{3.232693in}}%
\pgfpathlineto{\pgfqpoint{0.654313in}{3.232456in}}%
\pgfpathlineto{\pgfqpoint{0.654135in}{3.229804in}}%
\pgfpathlineto{\pgfqpoint{0.653711in}{3.222807in}}%
\pgfpathlineto{\pgfqpoint{0.646530in}{3.213158in}}%
\pgfpathlineto{\pgfqpoint{0.644424in}{3.211763in}}%
\pgfpathlineto{\pgfqpoint{0.634712in}{3.208300in}}%
\pgfusepath{stroke}%
\end{pgfscope}%
\begin{pgfscope}%
\pgfpathrectangle{\pgfqpoint{0.625000in}{0.550000in}}{\pgfqpoint{3.875000in}{3.850000in}} %
\pgfusepath{clip}%
\pgfsetbuttcap%
\pgfsetroundjoin%
\pgfsetlinewidth{0.501875pt}%
\definecolor{currentstroke}{rgb}{0.000000,0.000000,0.000000}%
\pgfsetstrokecolor{currentstroke}%
\pgfsetdash{}{0pt}%
\pgfpathmoveto{\pgfqpoint{0.634712in}{4.173073in}}%
\pgfpathlineto{\pgfqpoint{0.633333in}{4.178070in}}%
\pgfpathlineto{\pgfqpoint{0.631756in}{4.187719in}}%
\pgfpathlineto{\pgfqpoint{0.634712in}{4.193020in}}%
\pgfpathlineto{\pgfqpoint{0.644424in}{4.189359in}}%
\pgfpathlineto{\pgfqpoint{0.647476in}{4.187719in}}%
\pgfpathlineto{\pgfqpoint{0.644424in}{4.179098in}}%
\pgfpathlineto{\pgfqpoint{0.644247in}{4.178070in}}%
\pgfpathlineto{\pgfqpoint{0.634712in}{4.173073in}}%
\pgfusepath{stroke}%
\end{pgfscope}%
\begin{pgfscope}%
\pgfpathrectangle{\pgfqpoint{0.625000in}{0.550000in}}{\pgfqpoint{3.875000in}{3.850000in}} %
\pgfusepath{clip}%
\pgfsetbuttcap%
\pgfsetroundjoin%
\pgfsetlinewidth{0.501875pt}%
\definecolor{currentstroke}{rgb}{0.000000,0.000000,0.000000}%
\pgfsetstrokecolor{currentstroke}%
\pgfsetdash{}{0pt}%
\pgfpathmoveto{\pgfqpoint{0.625000in}{0.599652in}}%
\pgfpathlineto{\pgfqpoint{0.633073in}{0.607895in}}%
\pgfpathlineto{\pgfqpoint{0.625000in}{0.614945in}}%
\pgfusepath{stroke}%
\end{pgfscope}%
\begin{pgfscope}%
\pgfpathrectangle{\pgfqpoint{0.625000in}{0.550000in}}{\pgfqpoint{3.875000in}{3.850000in}} %
\pgfusepath{clip}%
\pgfsetbuttcap%
\pgfsetroundjoin%
\pgfsetlinewidth{0.501875pt}%
\definecolor{currentstroke}{rgb}{0.000000,0.000000,0.000000}%
\pgfsetstrokecolor{currentstroke}%
\pgfsetdash{}{0pt}%
\pgfpathmoveto{\pgfqpoint{0.625000in}{0.619505in}}%
\pgfpathlineto{\pgfqpoint{0.633593in}{0.627193in}}%
\pgfpathlineto{\pgfqpoint{0.625000in}{0.634760in}}%
\pgfusepath{stroke}%
\end{pgfscope}%
\begin{pgfscope}%
\pgfpathrectangle{\pgfqpoint{0.625000in}{0.550000in}}{\pgfqpoint{3.875000in}{3.850000in}} %
\pgfusepath{clip}%
\pgfsetbuttcap%
\pgfsetroundjoin%
\pgfsetlinewidth{0.501875pt}%
\definecolor{currentstroke}{rgb}{0.000000,0.000000,0.000000}%
\pgfsetstrokecolor{currentstroke}%
\pgfsetdash{}{0pt}%
\pgfpathmoveto{\pgfqpoint{0.625000in}{0.667558in}}%
\pgfpathlineto{\pgfqpoint{0.632950in}{0.675439in}}%
\pgfpathlineto{\pgfqpoint{0.625000in}{0.683320in}}%
\pgfusepath{stroke}%
\end{pgfscope}%
\begin{pgfscope}%
\pgfpathrectangle{\pgfqpoint{0.625000in}{0.550000in}}{\pgfqpoint{3.875000in}{3.850000in}} %
\pgfusepath{clip}%
\pgfsetbuttcap%
\pgfsetroundjoin%
\pgfsetlinewidth{0.501875pt}%
\definecolor{currentstroke}{rgb}{0.000000,0.000000,0.000000}%
\pgfsetstrokecolor{currentstroke}%
\pgfsetdash{}{0pt}%
\pgfpathmoveto{\pgfqpoint{0.625000in}{0.686625in}}%
\pgfpathlineto{\pgfqpoint{0.633647in}{0.694737in}}%
\pgfpathlineto{\pgfqpoint{0.625000in}{0.699692in}}%
\pgfusepath{stroke}%
\end{pgfscope}%
\begin{pgfscope}%
\pgfpathrectangle{\pgfqpoint{0.625000in}{0.550000in}}{\pgfqpoint{3.875000in}{3.850000in}} %
\pgfusepath{clip}%
\pgfsetbuttcap%
\pgfsetroundjoin%
\pgfsetlinewidth{0.501875pt}%
\definecolor{currentstroke}{rgb}{0.000000,0.000000,0.000000}%
\pgfsetstrokecolor{currentstroke}%
\pgfsetdash{}{0pt}%
\pgfpathmoveto{\pgfqpoint{0.625000in}{0.713878in}}%
\pgfpathlineto{\pgfqpoint{0.626735in}{0.714035in}}%
\pgfpathlineto{\pgfqpoint{0.625000in}{0.714774in}}%
\pgfusepath{stroke}%
\end{pgfscope}%
\begin{pgfscope}%
\pgfpathrectangle{\pgfqpoint{0.625000in}{0.550000in}}{\pgfqpoint{3.875000in}{3.850000in}} %
\pgfusepath{clip}%
\pgfsetbuttcap%
\pgfsetroundjoin%
\pgfsetlinewidth{0.501875pt}%
\definecolor{currentstroke}{rgb}{0.000000,0.000000,0.000000}%
\pgfsetstrokecolor{currentstroke}%
\pgfsetdash{}{0pt}%
\pgfpathmoveto{\pgfqpoint{0.625000in}{0.753631in}}%
\pgfpathlineto{\pgfqpoint{0.629013in}{0.752632in}}%
\pgfpathlineto{\pgfqpoint{0.634712in}{0.747488in}}%
\pgfpathlineto{\pgfqpoint{0.640957in}{0.752632in}}%
\pgfpathlineto{\pgfqpoint{0.635400in}{0.762281in}}%
\pgfpathlineto{\pgfqpoint{0.634712in}{0.762468in}}%
\pgfpathlineto{\pgfqpoint{0.629435in}{0.771930in}}%
\pgfpathlineto{\pgfqpoint{0.628303in}{0.781579in}}%
\pgfpathlineto{\pgfqpoint{0.625000in}{0.787264in}}%
\pgfusepath{stroke}%
\end{pgfscope}%
\begin{pgfscope}%
\pgfpathrectangle{\pgfqpoint{0.625000in}{0.550000in}}{\pgfqpoint{3.875000in}{3.850000in}} %
\pgfusepath{clip}%
\pgfsetbuttcap%
\pgfsetroundjoin%
\pgfsetlinewidth{0.501875pt}%
\definecolor{currentstroke}{rgb}{0.000000,0.000000,0.000000}%
\pgfsetstrokecolor{currentstroke}%
\pgfsetdash{}{0pt}%
\pgfpathmoveto{\pgfqpoint{0.625000in}{0.792543in}}%
\pgfpathlineto{\pgfqpoint{0.634712in}{0.795908in}}%
\pgfpathlineto{\pgfqpoint{0.644424in}{0.800562in}}%
\pgfpathlineto{\pgfqpoint{0.644887in}{0.800877in}}%
\pgfpathlineto{\pgfqpoint{0.649808in}{0.810526in}}%
\pgfpathlineto{\pgfqpoint{0.644424in}{0.818720in}}%
\pgfpathlineto{\pgfqpoint{0.644142in}{0.820175in}}%
\pgfpathlineto{\pgfqpoint{0.634712in}{0.825371in}}%
\pgfpathlineto{\pgfqpoint{0.626890in}{0.820175in}}%
\pgfpathlineto{\pgfqpoint{0.625000in}{0.818620in}}%
\pgfusepath{stroke}%
\end{pgfscope}%
\begin{pgfscope}%
\pgfpathrectangle{\pgfqpoint{0.625000in}{0.550000in}}{\pgfqpoint{3.875000in}{3.850000in}} %
\pgfusepath{clip}%
\pgfsetbuttcap%
\pgfsetroundjoin%
\pgfsetlinewidth{0.501875pt}%
\definecolor{currentstroke}{rgb}{0.000000,0.000000,0.000000}%
\pgfsetstrokecolor{currentstroke}%
\pgfsetdash{}{0pt}%
\pgfpathmoveto{\pgfqpoint{0.625000in}{0.879624in}}%
\pgfpathlineto{\pgfqpoint{0.631486in}{0.878070in}}%
\pgfpathlineto{\pgfqpoint{0.634712in}{0.875999in}}%
\pgfpathlineto{\pgfqpoint{0.641835in}{0.878070in}}%
\pgfpathlineto{\pgfqpoint{0.644424in}{0.879799in}}%
\pgfpathlineto{\pgfqpoint{0.654135in}{0.884230in}}%
\pgfpathlineto{\pgfqpoint{0.659553in}{0.887719in}}%
\pgfpathlineto{\pgfqpoint{0.663847in}{0.892847in}}%
\pgfpathlineto{\pgfqpoint{0.666977in}{0.897368in}}%
\pgfpathlineto{\pgfqpoint{0.669006in}{0.907018in}}%
\pgfpathlineto{\pgfqpoint{0.666574in}{0.916667in}}%
\pgfpathlineto{\pgfqpoint{0.663847in}{0.919447in}}%
\pgfpathlineto{\pgfqpoint{0.659161in}{0.926316in}}%
\pgfpathlineto{\pgfqpoint{0.654135in}{0.928669in}}%
\pgfpathlineto{\pgfqpoint{0.644424in}{0.932637in}}%
\pgfpathlineto{\pgfqpoint{0.634712in}{0.934468in}}%
\pgfpathlineto{\pgfqpoint{0.628234in}{0.926316in}}%
\pgfpathlineto{\pgfqpoint{0.625000in}{0.922347in}}%
\pgfusepath{stroke}%
\end{pgfscope}%
\begin{pgfscope}%
\pgfpathrectangle{\pgfqpoint{0.625000in}{0.550000in}}{\pgfqpoint{3.875000in}{3.850000in}} %
\pgfusepath{clip}%
\pgfsetbuttcap%
\pgfsetroundjoin%
\pgfsetlinewidth{0.501875pt}%
\definecolor{currentstroke}{rgb}{0.000000,0.000000,0.000000}%
\pgfsetstrokecolor{currentstroke}%
\pgfsetdash{}{0pt}%
\pgfpathmoveto{\pgfqpoint{0.625000in}{0.908129in}}%
\pgfpathlineto{\pgfqpoint{0.626591in}{0.907018in}}%
\pgfpathlineto{\pgfqpoint{0.628126in}{0.897368in}}%
\pgfpathlineto{\pgfqpoint{0.625000in}{0.895815in}}%
\pgfusepath{stroke}%
\end{pgfscope}%
\begin{pgfscope}%
\pgfpathrectangle{\pgfqpoint{0.625000in}{0.550000in}}{\pgfqpoint{3.875000in}{3.850000in}} %
\pgfusepath{clip}%
\pgfsetbuttcap%
\pgfsetroundjoin%
\pgfsetlinewidth{0.501875pt}%
\definecolor{currentstroke}{rgb}{0.000000,0.000000,0.000000}%
\pgfsetstrokecolor{currentstroke}%
\pgfsetdash{}{0pt}%
\pgfpathmoveto{\pgfqpoint{0.625000in}{0.940861in}}%
\pgfpathlineto{\pgfqpoint{0.634712in}{0.938250in}}%
\pgfpathlineto{\pgfqpoint{0.641127in}{0.945614in}}%
\pgfpathlineto{\pgfqpoint{0.639056in}{0.955263in}}%
\pgfpathlineto{\pgfqpoint{0.634712in}{0.960061in}}%
\pgfpathlineto{\pgfqpoint{0.625000in}{0.961149in}}%
\pgfusepath{stroke}%
\end{pgfscope}%
\begin{pgfscope}%
\pgfpathrectangle{\pgfqpoint{0.625000in}{0.550000in}}{\pgfqpoint{3.875000in}{3.850000in}} %
\pgfusepath{clip}%
\pgfsetbuttcap%
\pgfsetroundjoin%
\pgfsetlinewidth{0.501875pt}%
\definecolor{currentstroke}{rgb}{0.000000,0.000000,0.000000}%
\pgfsetstrokecolor{currentstroke}%
\pgfsetdash{}{0pt}%
\pgfpathmoveto{\pgfqpoint{0.625000in}{1.053903in}}%
\pgfpathlineto{\pgfqpoint{0.628441in}{1.051754in}}%
\pgfpathlineto{\pgfqpoint{0.632090in}{1.042105in}}%
\pgfpathlineto{\pgfqpoint{0.634712in}{1.037281in}}%
\pgfpathlineto{\pgfqpoint{0.644424in}{1.040629in}}%
\pgfpathlineto{\pgfqpoint{0.649280in}{1.042105in}}%
\pgfpathlineto{\pgfqpoint{0.654135in}{1.044678in}}%
\pgfpathlineto{\pgfqpoint{0.663847in}{1.048578in}}%
\pgfpathlineto{\pgfqpoint{0.672319in}{1.051754in}}%
\pgfpathlineto{\pgfqpoint{0.673559in}{1.052358in}}%
\pgfpathlineto{\pgfqpoint{0.683271in}{1.056537in}}%
\pgfpathlineto{\pgfqpoint{0.692982in}{1.061343in}}%
\pgfpathlineto{\pgfqpoint{0.693095in}{1.061404in}}%
\pgfpathlineto{\pgfqpoint{0.702694in}{1.067602in}}%
\pgfpathlineto{\pgfqpoint{0.707109in}{1.071053in}}%
\pgfpathlineto{\pgfqpoint{0.712406in}{1.076234in}}%
\pgfpathlineto{\pgfqpoint{0.716387in}{1.080702in}}%
\pgfpathlineto{\pgfqpoint{0.722118in}{1.089614in}}%
\pgfpathlineto{\pgfqpoint{0.722557in}{1.090351in}}%
\pgfpathlineto{\pgfqpoint{0.726699in}{1.100000in}}%
\pgfpathlineto{\pgfqpoint{0.728903in}{1.109649in}}%
\pgfpathlineto{\pgfqpoint{0.729480in}{1.119298in}}%
\pgfpathlineto{\pgfqpoint{0.728539in}{1.128947in}}%
\pgfpathlineto{\pgfqpoint{0.725955in}{1.138596in}}%
\pgfpathlineto{\pgfqpoint{0.722118in}{1.146419in}}%
\pgfpathlineto{\pgfqpoint{0.721397in}{1.148246in}}%
\pgfpathlineto{\pgfqpoint{0.715044in}{1.157895in}}%
\pgfpathlineto{\pgfqpoint{0.712406in}{1.160405in}}%
\pgfpathlineto{\pgfqpoint{0.705843in}{1.167544in}}%
\pgfpathlineto{\pgfqpoint{0.702694in}{1.169674in}}%
\pgfpathlineto{\pgfqpoint{0.692982in}{1.176563in}}%
\pgfpathlineto{\pgfqpoint{0.692226in}{1.177193in}}%
\pgfpathlineto{\pgfqpoint{0.683271in}{1.181499in}}%
\pgfpathlineto{\pgfqpoint{0.673559in}{1.185534in}}%
\pgfpathlineto{\pgfqpoint{0.670772in}{1.186842in}}%
\pgfpathlineto{\pgfqpoint{0.663847in}{1.188547in}}%
\pgfpathlineto{\pgfqpoint{0.654135in}{1.190883in}}%
\pgfpathlineto{\pgfqpoint{0.644424in}{1.195724in}}%
\pgfpathlineto{\pgfqpoint{0.642605in}{1.196491in}}%
\pgfpathlineto{\pgfqpoint{0.644424in}{1.197228in}}%
\pgfpathlineto{\pgfqpoint{0.654135in}{1.203001in}}%
\pgfpathlineto{\pgfqpoint{0.659521in}{1.206140in}}%
\pgfpathlineto{\pgfqpoint{0.663847in}{1.213220in}}%
\pgfpathlineto{\pgfqpoint{0.666289in}{1.215789in}}%
\pgfpathlineto{\pgfqpoint{0.669761in}{1.225439in}}%
\pgfpathlineto{\pgfqpoint{0.670215in}{1.235088in}}%
\pgfpathlineto{\pgfqpoint{0.667936in}{1.244737in}}%
\pgfpathlineto{\pgfqpoint{0.663847in}{1.252090in}}%
\pgfpathlineto{\pgfqpoint{0.662123in}{1.254386in}}%
\pgfpathlineto{\pgfqpoint{0.654135in}{1.261687in}}%
\pgfpathlineto{\pgfqpoint{0.649509in}{1.264035in}}%
\pgfpathlineto{\pgfqpoint{0.644424in}{1.266292in}}%
\pgfpathlineto{\pgfqpoint{0.634712in}{1.268757in}}%
\pgfpathlineto{\pgfqpoint{0.625000in}{1.272201in}}%
\pgfusepath{stroke}%
\end{pgfscope}%
\begin{pgfscope}%
\pgfpathrectangle{\pgfqpoint{0.625000in}{0.550000in}}{\pgfqpoint{3.875000in}{3.850000in}} %
\pgfusepath{clip}%
\pgfsetbuttcap%
\pgfsetroundjoin%
\pgfsetlinewidth{0.501875pt}%
\definecolor{currentstroke}{rgb}{0.000000,0.000000,0.000000}%
\pgfsetstrokecolor{currentstroke}%
\pgfsetdash{}{0pt}%
\pgfpathmoveto{\pgfqpoint{0.625000in}{1.086284in}}%
\pgfpathlineto{\pgfqpoint{0.634121in}{1.080702in}}%
\pgfpathlineto{\pgfqpoint{0.625000in}{1.078410in}}%
\pgfusepath{stroke}%
\end{pgfscope}%
\begin{pgfscope}%
\pgfpathrectangle{\pgfqpoint{0.625000in}{0.550000in}}{\pgfqpoint{3.875000in}{3.850000in}} %
\pgfusepath{clip}%
\pgfsetbuttcap%
\pgfsetroundjoin%
\pgfsetlinewidth{0.501875pt}%
\definecolor{currentstroke}{rgb}{0.000000,0.000000,0.000000}%
\pgfsetstrokecolor{currentstroke}%
\pgfsetdash{}{0pt}%
\pgfpathmoveto{\pgfqpoint{0.625000in}{1.101363in}}%
\pgfpathlineto{\pgfqpoint{0.627951in}{1.100000in}}%
\pgfpathlineto{\pgfqpoint{0.625000in}{1.095945in}}%
\pgfusepath{stroke}%
\end{pgfscope}%
\begin{pgfscope}%
\pgfpathrectangle{\pgfqpoint{0.625000in}{0.550000in}}{\pgfqpoint{3.875000in}{3.850000in}} %
\pgfusepath{clip}%
\pgfsetbuttcap%
\pgfsetroundjoin%
\pgfsetlinewidth{0.501875pt}%
\definecolor{currentstroke}{rgb}{0.000000,0.000000,0.000000}%
\pgfsetstrokecolor{currentstroke}%
\pgfsetdash{}{0pt}%
\pgfpathmoveto{\pgfqpoint{0.625000in}{1.121768in}}%
\pgfpathlineto{\pgfqpoint{0.628018in}{1.119298in}}%
\pgfpathlineto{\pgfqpoint{0.625000in}{1.116829in}}%
\pgfusepath{stroke}%
\end{pgfscope}%
\begin{pgfscope}%
\pgfpathrectangle{\pgfqpoint{0.625000in}{0.550000in}}{\pgfqpoint{3.875000in}{3.850000in}} %
\pgfusepath{clip}%
\pgfsetbuttcap%
\pgfsetroundjoin%
\pgfsetlinewidth{0.501875pt}%
\definecolor{currentstroke}{rgb}{0.000000,0.000000,0.000000}%
\pgfsetstrokecolor{currentstroke}%
\pgfsetdash{}{0pt}%
\pgfpathmoveto{\pgfqpoint{0.625000in}{1.150205in}}%
\pgfpathlineto{\pgfqpoint{0.625918in}{1.148246in}}%
\pgfpathlineto{\pgfqpoint{0.625586in}{1.138596in}}%
\pgfpathlineto{\pgfqpoint{0.625000in}{1.137835in}}%
\pgfusepath{stroke}%
\end{pgfscope}%
\begin{pgfscope}%
\pgfpathrectangle{\pgfqpoint{0.625000in}{0.550000in}}{\pgfqpoint{3.875000in}{3.850000in}} %
\pgfusepath{clip}%
\pgfsetbuttcap%
\pgfsetroundjoin%
\pgfsetlinewidth{0.501875pt}%
\definecolor{currentstroke}{rgb}{0.000000,0.000000,0.000000}%
\pgfsetstrokecolor{currentstroke}%
\pgfsetdash{}{0pt}%
\pgfpathmoveto{\pgfqpoint{0.625000in}{1.171536in}}%
\pgfpathlineto{\pgfqpoint{0.634712in}{1.171829in}}%
\pgfpathlineto{\pgfqpoint{0.637157in}{1.167544in}}%
\pgfpathlineto{\pgfqpoint{0.634712in}{1.165019in}}%
\pgfpathlineto{\pgfqpoint{0.625000in}{1.162138in}}%
\pgfusepath{stroke}%
\end{pgfscope}%
\begin{pgfscope}%
\pgfpathrectangle{\pgfqpoint{0.625000in}{0.550000in}}{\pgfqpoint{3.875000in}{3.850000in}} %
\pgfusepath{clip}%
\pgfsetbuttcap%
\pgfsetroundjoin%
\pgfsetlinewidth{0.501875pt}%
\definecolor{currentstroke}{rgb}{0.000000,0.000000,0.000000}%
\pgfsetstrokecolor{currentstroke}%
\pgfsetdash{}{0pt}%
\pgfpathmoveto{\pgfqpoint{0.625000in}{1.218050in}}%
\pgfpathlineto{\pgfqpoint{0.633020in}{1.215789in}}%
\pgfpathlineto{\pgfqpoint{0.630927in}{1.206140in}}%
\pgfpathlineto{\pgfqpoint{0.627769in}{1.196491in}}%
\pgfpathlineto{\pgfqpoint{0.626568in}{1.186842in}}%
\pgfpathlineto{\pgfqpoint{0.625000in}{1.185649in}}%
\pgfusepath{stroke}%
\end{pgfscope}%
\begin{pgfscope}%
\pgfpathrectangle{\pgfqpoint{0.625000in}{0.550000in}}{\pgfqpoint{3.875000in}{3.850000in}} %
\pgfusepath{clip}%
\pgfsetbuttcap%
\pgfsetroundjoin%
\pgfsetlinewidth{0.501875pt}%
\definecolor{currentstroke}{rgb}{0.000000,0.000000,0.000000}%
\pgfsetstrokecolor{currentstroke}%
\pgfsetdash{}{0pt}%
\pgfpathmoveto{\pgfqpoint{0.625000in}{1.256129in}}%
\pgfpathlineto{\pgfqpoint{0.628946in}{1.254386in}}%
\pgfpathlineto{\pgfqpoint{0.625000in}{1.252739in}}%
\pgfusepath{stroke}%
\end{pgfscope}%
\begin{pgfscope}%
\pgfpathrectangle{\pgfqpoint{0.625000in}{0.550000in}}{\pgfqpoint{3.875000in}{3.850000in}} %
\pgfusepath{clip}%
\pgfsetbuttcap%
\pgfsetroundjoin%
\pgfsetlinewidth{0.501875pt}%
\definecolor{currentstroke}{rgb}{0.000000,0.000000,0.000000}%
\pgfsetstrokecolor{currentstroke}%
\pgfsetdash{}{0pt}%
\pgfpathmoveto{\pgfqpoint{0.625000in}{1.286674in}}%
\pgfpathlineto{\pgfqpoint{0.629302in}{1.292982in}}%
\pgfpathlineto{\pgfqpoint{0.625000in}{1.299291in}}%
\pgfusepath{stroke}%
\end{pgfscope}%
\begin{pgfscope}%
\pgfpathrectangle{\pgfqpoint{0.625000in}{0.550000in}}{\pgfqpoint{3.875000in}{3.850000in}} %
\pgfusepath{clip}%
\pgfsetbuttcap%
\pgfsetroundjoin%
\pgfsetlinewidth{0.501875pt}%
\definecolor{currentstroke}{rgb}{0.000000,0.000000,0.000000}%
\pgfsetstrokecolor{currentstroke}%
\pgfsetdash{}{0pt}%
\pgfpathmoveto{\pgfqpoint{0.625000in}{1.327998in}}%
\pgfpathlineto{\pgfqpoint{0.629816in}{1.331579in}}%
\pgfpathlineto{\pgfqpoint{0.625000in}{1.338786in}}%
\pgfusepath{stroke}%
\end{pgfscope}%
\begin{pgfscope}%
\pgfpathrectangle{\pgfqpoint{0.625000in}{0.550000in}}{\pgfqpoint{3.875000in}{3.850000in}} %
\pgfusepath{clip}%
\pgfsetbuttcap%
\pgfsetroundjoin%
\pgfsetlinewidth{0.501875pt}%
\definecolor{currentstroke}{rgb}{0.000000,0.000000,0.000000}%
\pgfsetstrokecolor{currentstroke}%
\pgfsetdash{}{0pt}%
\pgfpathmoveto{\pgfqpoint{0.625000in}{1.361525in}}%
\pgfpathlineto{\pgfqpoint{0.631084in}{1.360526in}}%
\pgfpathlineto{\pgfqpoint{0.634712in}{1.358144in}}%
\pgfpathlineto{\pgfqpoint{0.637319in}{1.360526in}}%
\pgfpathlineto{\pgfqpoint{0.639722in}{1.370175in}}%
\pgfpathlineto{\pgfqpoint{0.634712in}{1.374747in}}%
\pgfpathlineto{\pgfqpoint{0.632127in}{1.379825in}}%
\pgfpathlineto{\pgfqpoint{0.628267in}{1.389474in}}%
\pgfpathlineto{\pgfqpoint{0.630224in}{1.399123in}}%
\pgfpathlineto{\pgfqpoint{0.626295in}{1.408772in}}%
\pgfpathlineto{\pgfqpoint{0.625000in}{1.412831in}}%
\pgfusepath{stroke}%
\end{pgfscope}%
\begin{pgfscope}%
\pgfpathrectangle{\pgfqpoint{0.625000in}{0.550000in}}{\pgfqpoint{3.875000in}{3.850000in}} %
\pgfusepath{clip}%
\pgfsetbuttcap%
\pgfsetroundjoin%
\pgfsetlinewidth{0.501875pt}%
\definecolor{currentstroke}{rgb}{0.000000,0.000000,0.000000}%
\pgfsetstrokecolor{currentstroke}%
\pgfsetdash{}{0pt}%
\pgfpathmoveto{\pgfqpoint{0.625000in}{1.419248in}}%
\pgfpathlineto{\pgfqpoint{0.631769in}{1.428070in}}%
\pgfpathlineto{\pgfqpoint{0.625000in}{1.436892in}}%
\pgfusepath{stroke}%
\end{pgfscope}%
\begin{pgfscope}%
\pgfpathrectangle{\pgfqpoint{0.625000in}{0.550000in}}{\pgfqpoint{3.875000in}{3.850000in}} %
\pgfusepath{clip}%
\pgfsetbuttcap%
\pgfsetroundjoin%
\pgfsetlinewidth{0.501875pt}%
\definecolor{currentstroke}{rgb}{0.000000,0.000000,0.000000}%
\pgfsetstrokecolor{currentstroke}%
\pgfsetdash{}{0pt}%
\pgfpathmoveto{\pgfqpoint{0.625000in}{1.442864in}}%
\pgfpathlineto{\pgfqpoint{0.626675in}{1.447368in}}%
\pgfpathlineto{\pgfqpoint{0.625000in}{1.451873in}}%
\pgfusepath{stroke}%
\end{pgfscope}%
\begin{pgfscope}%
\pgfpathrectangle{\pgfqpoint{0.625000in}{0.550000in}}{\pgfqpoint{3.875000in}{3.850000in}} %
\pgfusepath{clip}%
\pgfsetbuttcap%
\pgfsetroundjoin%
\pgfsetlinewidth{0.501875pt}%
\definecolor{currentstroke}{rgb}{0.000000,0.000000,0.000000}%
\pgfsetstrokecolor{currentstroke}%
\pgfsetdash{}{0pt}%
\pgfpathmoveto{\pgfqpoint{0.625000in}{1.517329in}}%
\pgfpathlineto{\pgfqpoint{0.644424in}{1.523085in}}%
\pgfpathlineto{\pgfqpoint{0.654135in}{1.527133in}}%
\pgfpathlineto{\pgfqpoint{0.840406in}{1.601754in}}%
\pgfpathlineto{\pgfqpoint{0.916353in}{1.631198in}}%
\pgfpathlineto{\pgfqpoint{0.955201in}{1.649286in}}%
\pgfpathlineto{\pgfqpoint{0.984336in}{1.665291in}}%
\pgfpathlineto{\pgfqpoint{1.013471in}{1.684039in}}%
\pgfpathlineto{\pgfqpoint{1.032895in}{1.698397in}}%
\pgfpathlineto{\pgfqpoint{1.055701in}{1.717544in}}%
\pgfpathlineto{\pgfqpoint{1.075754in}{1.736842in}}%
\pgfpathlineto{\pgfqpoint{1.093363in}{1.756140in}}%
\pgfpathlineto{\pgfqpoint{1.110589in}{1.777709in}}%
\pgfpathlineto{\pgfqpoint{1.128852in}{1.804386in}}%
\pgfpathlineto{\pgfqpoint{1.140213in}{1.823684in}}%
\pgfpathlineto{\pgfqpoint{1.154694in}{1.852632in}}%
\pgfpathlineto{\pgfqpoint{1.166373in}{1.881579in}}%
\pgfpathlineto{\pgfqpoint{1.175511in}{1.910526in}}%
\pgfpathlineto{\pgfqpoint{1.182276in}{1.939474in}}%
\pgfpathlineto{\pgfqpoint{1.186778in}{1.968421in}}%
\pgfpathlineto{\pgfqpoint{1.189112in}{1.997368in}}%
\pgfpathlineto{\pgfqpoint{1.189326in}{2.026316in}}%
\pgfpathlineto{\pgfqpoint{1.187423in}{2.055263in}}%
\pgfpathlineto{\pgfqpoint{1.183420in}{2.084211in}}%
\pgfpathlineto{\pgfqpoint{1.177201in}{2.113158in}}%
\pgfpathlineto{\pgfqpoint{1.168733in}{2.142105in}}%
\pgfpathlineto{\pgfqpoint{1.157935in}{2.171053in}}%
\pgfpathlineto{\pgfqpoint{1.144621in}{2.200000in}}%
\pgfpathlineto{\pgfqpoint{1.128511in}{2.228947in}}%
\pgfpathlineto{\pgfqpoint{1.109371in}{2.257895in}}%
\pgfpathlineto{\pgfqpoint{1.086803in}{2.286842in}}%
\pgfpathlineto{\pgfqpoint{1.060118in}{2.315789in}}%
\pgfpathlineto{\pgfqpoint{1.032895in}{2.340908in}}%
\pgfpathlineto{\pgfqpoint{1.013471in}{2.356790in}}%
\pgfpathlineto{\pgfqpoint{0.984336in}{2.377826in}}%
\pgfpathlineto{\pgfqpoint{0.955201in}{2.396075in}}%
\pgfpathlineto{\pgfqpoint{0.925463in}{2.412281in}}%
\pgfpathlineto{\pgfqpoint{0.887218in}{2.429809in}}%
\pgfpathlineto{\pgfqpoint{0.867794in}{2.437490in}}%
\pgfpathlineto{\pgfqpoint{0.838659in}{2.447634in}}%
\pgfpathlineto{\pgfqpoint{0.799812in}{2.458748in}}%
\pgfpathlineto{\pgfqpoint{0.770677in}{2.465380in}}%
\pgfpathlineto{\pgfqpoint{0.731830in}{2.472199in}}%
\pgfpathlineto{\pgfqpoint{0.692982in}{2.476740in}}%
\pgfpathlineto{\pgfqpoint{0.683271in}{2.475357in}}%
\pgfpathlineto{\pgfqpoint{0.673559in}{2.466265in}}%
\pgfpathlineto{\pgfqpoint{0.669789in}{2.470175in}}%
\pgfpathlineto{\pgfqpoint{0.667165in}{2.479825in}}%
\pgfpathlineto{\pgfqpoint{0.670456in}{2.489474in}}%
\pgfpathlineto{\pgfqpoint{0.673559in}{2.493388in}}%
\pgfpathlineto{\pgfqpoint{0.677063in}{2.489474in}}%
\pgfpathlineto{\pgfqpoint{0.683271in}{2.484292in}}%
\pgfpathlineto{\pgfqpoint{0.692982in}{2.482910in}}%
\pgfpathlineto{\pgfqpoint{0.731830in}{2.487451in}}%
\pgfpathlineto{\pgfqpoint{0.770677in}{2.494269in}}%
\pgfpathlineto{\pgfqpoint{0.809524in}{2.503448in}}%
\pgfpathlineto{\pgfqpoint{0.848371in}{2.515212in}}%
\pgfpathlineto{\pgfqpoint{0.883135in}{2.528070in}}%
\pgfpathlineto{\pgfqpoint{0.926065in}{2.547694in}}%
\pgfpathlineto{\pgfqpoint{0.960621in}{2.566667in}}%
\pgfpathlineto{\pgfqpoint{0.994048in}{2.588492in}}%
\pgfpathlineto{\pgfqpoint{1.023183in}{2.610608in}}%
\pgfpathlineto{\pgfqpoint{1.042607in}{2.627299in}}%
\pgfpathlineto{\pgfqpoint{1.062030in}{2.645873in}}%
\pgfpathlineto{\pgfqpoint{1.081454in}{2.666742in}}%
\pgfpathlineto{\pgfqpoint{1.100877in}{2.690389in}}%
\pgfpathlineto{\pgfqpoint{1.110589in}{2.703543in}}%
\pgfpathlineto{\pgfqpoint{1.128511in}{2.730702in}}%
\pgfpathlineto{\pgfqpoint{1.144621in}{2.759649in}}%
\pgfpathlineto{\pgfqpoint{1.159148in}{2.791688in}}%
\pgfpathlineto{\pgfqpoint{1.168860in}{2.817953in}}%
\pgfpathlineto{\pgfqpoint{1.177201in}{2.846491in}}%
\pgfpathlineto{\pgfqpoint{1.183420in}{2.875439in}}%
\pgfpathlineto{\pgfqpoint{1.187423in}{2.904386in}}%
\pgfpathlineto{\pgfqpoint{1.189326in}{2.933333in}}%
\pgfpathlineto{\pgfqpoint{1.189112in}{2.962281in}}%
\pgfpathlineto{\pgfqpoint{1.186778in}{2.991228in}}%
\pgfpathlineto{\pgfqpoint{1.182276in}{3.020175in}}%
\pgfpathlineto{\pgfqpoint{1.175511in}{3.049123in}}%
\pgfpathlineto{\pgfqpoint{1.166373in}{3.078070in}}%
\pgfpathlineto{\pgfqpoint{1.154694in}{3.107018in}}%
\pgfpathlineto{\pgfqpoint{1.139724in}{3.136839in}}%
\pgfpathlineto{\pgfqpoint{1.122610in}{3.164912in}}%
\pgfpathlineto{\pgfqpoint{1.108892in}{3.184211in}}%
\pgfpathlineto{\pgfqpoint{1.091165in}{3.206053in}}%
\pgfpathlineto{\pgfqpoint{1.071742in}{3.226882in}}%
\pgfpathlineto{\pgfqpoint{1.052318in}{3.245127in}}%
\pgfpathlineto{\pgfqpoint{1.032700in}{3.261404in}}%
\pgfpathlineto{\pgfqpoint{1.003759in}{3.282200in}}%
\pgfpathlineto{\pgfqpoint{0.974556in}{3.300000in}}%
\pgfpathlineto{\pgfqpoint{0.945489in}{3.315196in}}%
\pgfpathlineto{\pgfqpoint{0.906642in}{3.332527in}}%
\pgfpathlineto{\pgfqpoint{0.858083in}{3.351313in}}%
\pgfpathlineto{\pgfqpoint{0.770677in}{3.384821in}}%
\pgfpathlineto{\pgfqpoint{0.695550in}{3.415789in}}%
\pgfpathlineto{\pgfqpoint{0.672135in}{3.425439in}}%
\pgfpathlineto{\pgfqpoint{0.625000in}{3.442320in}}%
\pgfpathlineto{\pgfqpoint{0.625000in}{3.442320in}}%
\pgfusepath{stroke}%
\end{pgfscope}%
\begin{pgfscope}%
\pgfpathrectangle{\pgfqpoint{0.625000in}{0.550000in}}{\pgfqpoint{3.875000in}{3.850000in}} %
\pgfusepath{clip}%
\pgfsetbuttcap%
\pgfsetroundjoin%
\pgfsetlinewidth{0.501875pt}%
\definecolor{currentstroke}{rgb}{0.000000,0.000000,0.000000}%
\pgfsetstrokecolor{currentstroke}%
\pgfsetdash{}{0pt}%
\pgfpathmoveto{\pgfqpoint{0.625000in}{1.603897in}}%
\pgfpathlineto{\pgfqpoint{0.629076in}{1.601754in}}%
\pgfpathlineto{\pgfqpoint{0.628452in}{1.592105in}}%
\pgfpathlineto{\pgfqpoint{0.630809in}{1.582456in}}%
\pgfpathlineto{\pgfqpoint{0.625000in}{1.578639in}}%
\pgfusepath{stroke}%
\end{pgfscope}%
\begin{pgfscope}%
\pgfpathrectangle{\pgfqpoint{0.625000in}{0.550000in}}{\pgfqpoint{3.875000in}{3.850000in}} %
\pgfusepath{clip}%
\pgfsetbuttcap%
\pgfsetroundjoin%
\pgfsetlinewidth{0.501875pt}%
\definecolor{currentstroke}{rgb}{0.000000,0.000000,0.000000}%
\pgfsetstrokecolor{currentstroke}%
\pgfsetdash{}{0pt}%
\pgfpathmoveto{\pgfqpoint{0.625000in}{1.651377in}}%
\pgfpathlineto{\pgfqpoint{0.627951in}{1.650000in}}%
\pgfpathlineto{\pgfqpoint{0.632168in}{1.640351in}}%
\pgfpathlineto{\pgfqpoint{0.634712in}{1.636534in}}%
\pgfpathlineto{\pgfqpoint{0.638705in}{1.630702in}}%
\pgfpathlineto{\pgfqpoint{0.634712in}{1.628342in}}%
\pgfpathlineto{\pgfqpoint{0.628565in}{1.621053in}}%
\pgfpathlineto{\pgfqpoint{0.625000in}{1.619176in}}%
\pgfusepath{stroke}%
\end{pgfscope}%
\begin{pgfscope}%
\pgfpathrectangle{\pgfqpoint{0.625000in}{0.550000in}}{\pgfqpoint{3.875000in}{3.850000in}} %
\pgfusepath{clip}%
\pgfsetbuttcap%
\pgfsetroundjoin%
\pgfsetlinewidth{0.501875pt}%
\definecolor{currentstroke}{rgb}{0.000000,0.000000,0.000000}%
\pgfsetstrokecolor{currentstroke}%
\pgfsetdash{}{0pt}%
\pgfpathmoveto{\pgfqpoint{0.625000in}{1.671160in}}%
\pgfpathlineto{\pgfqpoint{0.626475in}{1.669298in}}%
\pgfpathlineto{\pgfqpoint{0.625000in}{1.667922in}}%
\pgfusepath{stroke}%
\end{pgfscope}%
\begin{pgfscope}%
\pgfpathrectangle{\pgfqpoint{0.625000in}{0.550000in}}{\pgfqpoint{3.875000in}{3.850000in}} %
\pgfusepath{clip}%
\pgfsetbuttcap%
\pgfsetroundjoin%
\pgfsetlinewidth{0.501875pt}%
\definecolor{currentstroke}{rgb}{0.000000,0.000000,0.000000}%
\pgfsetstrokecolor{currentstroke}%
\pgfsetdash{}{0pt}%
\pgfpathmoveto{\pgfqpoint{0.625000in}{1.711853in}}%
\pgfpathlineto{\pgfqpoint{0.631913in}{1.707895in}}%
\pgfpathlineto{\pgfqpoint{0.626501in}{1.698246in}}%
\pgfpathlineto{\pgfqpoint{0.625000in}{1.696827in}}%
\pgfusepath{stroke}%
\end{pgfscope}%
\begin{pgfscope}%
\pgfpathrectangle{\pgfqpoint{0.625000in}{0.550000in}}{\pgfqpoint{3.875000in}{3.850000in}} %
\pgfusepath{clip}%
\pgfsetbuttcap%
\pgfsetroundjoin%
\pgfsetlinewidth{0.501875pt}%
\definecolor{currentstroke}{rgb}{0.000000,0.000000,0.000000}%
\pgfsetstrokecolor{currentstroke}%
\pgfsetdash{}{0pt}%
\pgfpathmoveto{\pgfqpoint{0.625000in}{1.771556in}}%
\pgfpathlineto{\pgfqpoint{0.629755in}{1.765789in}}%
\pgfpathlineto{\pgfqpoint{0.634712in}{1.760864in}}%
\pgfpathlineto{\pgfqpoint{0.644424in}{1.765367in}}%
\pgfpathlineto{\pgfqpoint{0.647645in}{1.765789in}}%
\pgfpathlineto{\pgfqpoint{0.654135in}{1.768605in}}%
\pgfpathlineto{\pgfqpoint{0.658310in}{1.765789in}}%
\pgfpathlineto{\pgfqpoint{0.663847in}{1.762747in}}%
\pgfpathlineto{\pgfqpoint{0.669627in}{1.756140in}}%
\pgfpathlineto{\pgfqpoint{0.673559in}{1.747221in}}%
\pgfpathlineto{\pgfqpoint{0.673857in}{1.746491in}}%
\pgfpathlineto{\pgfqpoint{0.674518in}{1.736842in}}%
\pgfpathlineto{\pgfqpoint{0.673559in}{1.734594in}}%
\pgfpathlineto{\pgfqpoint{0.671543in}{1.727193in}}%
\pgfpathlineto{\pgfqpoint{0.663847in}{1.719338in}}%
\pgfpathlineto{\pgfqpoint{0.662622in}{1.717544in}}%
\pgfpathlineto{\pgfqpoint{0.654135in}{1.713569in}}%
\pgfpathlineto{\pgfqpoint{0.644424in}{1.710483in}}%
\pgfpathlineto{\pgfqpoint{0.638645in}{1.717544in}}%
\pgfpathlineto{\pgfqpoint{0.634712in}{1.723785in}}%
\pgfpathlineto{\pgfqpoint{0.633074in}{1.727193in}}%
\pgfpathlineto{\pgfqpoint{0.625000in}{1.736151in}}%
\pgfusepath{stroke}%
\end{pgfscope}%
\begin{pgfscope}%
\pgfpathrectangle{\pgfqpoint{0.625000in}{0.550000in}}{\pgfqpoint{3.875000in}{3.850000in}} %
\pgfusepath{clip}%
\pgfsetbuttcap%
\pgfsetroundjoin%
\pgfsetlinewidth{0.501875pt}%
\definecolor{currentstroke}{rgb}{0.000000,0.000000,0.000000}%
\pgfsetstrokecolor{currentstroke}%
\pgfsetdash{}{0pt}%
\pgfpathmoveto{\pgfqpoint{0.625000in}{1.792458in}}%
\pgfpathlineto{\pgfqpoint{0.633344in}{1.794737in}}%
\pgfpathlineto{\pgfqpoint{0.625000in}{1.800596in}}%
\pgfusepath{stroke}%
\end{pgfscope}%
\begin{pgfscope}%
\pgfpathrectangle{\pgfqpoint{0.625000in}{0.550000in}}{\pgfqpoint{3.875000in}{3.850000in}} %
\pgfusepath{clip}%
\pgfsetbuttcap%
\pgfsetroundjoin%
\pgfsetlinewidth{0.501875pt}%
\definecolor{currentstroke}{rgb}{0.000000,0.000000,0.000000}%
\pgfsetstrokecolor{currentstroke}%
\pgfsetdash{}{0pt}%
\pgfpathmoveto{\pgfqpoint{0.625000in}{1.805257in}}%
\pgfpathlineto{\pgfqpoint{0.634712in}{1.813105in}}%
\pgfpathlineto{\pgfqpoint{0.638930in}{1.804386in}}%
\pgfpathlineto{\pgfqpoint{0.642107in}{1.794737in}}%
\pgfpathlineto{\pgfqpoint{0.644424in}{1.789941in}}%
\pgfpathlineto{\pgfqpoint{0.646267in}{1.785088in}}%
\pgfpathlineto{\pgfqpoint{0.644424in}{1.782412in}}%
\pgfpathlineto{\pgfqpoint{0.634712in}{1.779578in}}%
\pgfpathlineto{\pgfqpoint{0.625000in}{1.776584in}}%
\pgfusepath{stroke}%
\end{pgfscope}%
\begin{pgfscope}%
\pgfpathrectangle{\pgfqpoint{0.625000in}{0.550000in}}{\pgfqpoint{3.875000in}{3.850000in}} %
\pgfusepath{clip}%
\pgfsetbuttcap%
\pgfsetroundjoin%
\pgfsetlinewidth{0.501875pt}%
\definecolor{currentstroke}{rgb}{0.000000,0.000000,0.000000}%
\pgfsetstrokecolor{currentstroke}%
\pgfsetdash{}{0pt}%
\pgfpathmoveto{\pgfqpoint{0.625000in}{1.833405in}}%
\pgfpathlineto{\pgfqpoint{0.625229in}{1.833333in}}%
\pgfpathlineto{\pgfqpoint{0.630927in}{1.823684in}}%
\pgfpathlineto{\pgfqpoint{0.625000in}{1.822813in}}%
\pgfusepath{stroke}%
\end{pgfscope}%
\begin{pgfscope}%
\pgfpathrectangle{\pgfqpoint{0.625000in}{0.550000in}}{\pgfqpoint{3.875000in}{3.850000in}} %
\pgfusepath{clip}%
\pgfsetbuttcap%
\pgfsetroundjoin%
\pgfsetlinewidth{0.501875pt}%
\definecolor{currentstroke}{rgb}{0.000000,0.000000,0.000000}%
\pgfsetstrokecolor{currentstroke}%
\pgfsetdash{}{0pt}%
\pgfpathmoveto{\pgfqpoint{0.625000in}{1.875623in}}%
\pgfpathlineto{\pgfqpoint{0.628978in}{1.871930in}}%
\pgfpathlineto{\pgfqpoint{0.626643in}{1.862281in}}%
\pgfpathlineto{\pgfqpoint{0.625000in}{1.860822in}}%
\pgfusepath{stroke}%
\end{pgfscope}%
\begin{pgfscope}%
\pgfpathrectangle{\pgfqpoint{0.625000in}{0.550000in}}{\pgfqpoint{3.875000in}{3.850000in}} %
\pgfusepath{clip}%
\pgfsetbuttcap%
\pgfsetroundjoin%
\pgfsetlinewidth{0.501875pt}%
\definecolor{currentstroke}{rgb}{0.000000,0.000000,0.000000}%
\pgfsetstrokecolor{currentstroke}%
\pgfsetdash{}{0pt}%
\pgfpathmoveto{\pgfqpoint{0.625000in}{1.928605in}}%
\pgfpathlineto{\pgfqpoint{0.625848in}{1.920175in}}%
\pgfpathlineto{\pgfqpoint{0.625000in}{1.919251in}}%
\pgfusepath{stroke}%
\end{pgfscope}%
\begin{pgfscope}%
\pgfpathrectangle{\pgfqpoint{0.625000in}{0.550000in}}{\pgfqpoint{3.875000in}{3.850000in}} %
\pgfusepath{clip}%
\pgfsetbuttcap%
\pgfsetroundjoin%
\pgfsetlinewidth{0.501875pt}%
\definecolor{currentstroke}{rgb}{0.000000,0.000000,0.000000}%
\pgfsetstrokecolor{currentstroke}%
\pgfsetdash{}{0pt}%
\pgfpathmoveto{\pgfqpoint{0.625000in}{1.982173in}}%
\pgfpathlineto{\pgfqpoint{0.626874in}{1.978070in}}%
\pgfpathlineto{\pgfqpoint{0.629529in}{1.968421in}}%
\pgfpathlineto{\pgfqpoint{0.626899in}{1.958772in}}%
\pgfpathlineto{\pgfqpoint{0.625465in}{1.949123in}}%
\pgfpathlineto{\pgfqpoint{0.634712in}{1.941874in}}%
\pgfpathlineto{\pgfqpoint{0.636369in}{1.939474in}}%
\pgfpathlineto{\pgfqpoint{0.634712in}{1.937939in}}%
\pgfpathlineto{\pgfqpoint{0.625000in}{1.930096in}}%
\pgfusepath{stroke}%
\end{pgfscope}%
\begin{pgfscope}%
\pgfpathrectangle{\pgfqpoint{0.625000in}{0.550000in}}{\pgfqpoint{3.875000in}{3.850000in}} %
\pgfusepath{clip}%
\pgfsetbuttcap%
\pgfsetroundjoin%
\pgfsetlinewidth{0.501875pt}%
\definecolor{currentstroke}{rgb}{0.000000,0.000000,0.000000}%
\pgfsetstrokecolor{currentstroke}%
\pgfsetdash{}{0pt}%
\pgfpathmoveto{\pgfqpoint{0.625000in}{2.058264in}}%
\pgfpathlineto{\pgfqpoint{0.626274in}{2.055263in}}%
\pgfpathlineto{\pgfqpoint{0.625130in}{2.045614in}}%
\pgfpathlineto{\pgfqpoint{0.625000in}{2.045433in}}%
\pgfusepath{stroke}%
\end{pgfscope}%
\begin{pgfscope}%
\pgfpathrectangle{\pgfqpoint{0.625000in}{0.550000in}}{\pgfqpoint{3.875000in}{3.850000in}} %
\pgfusepath{clip}%
\pgfsetbuttcap%
\pgfsetroundjoin%
\pgfsetlinewidth{0.501875pt}%
\definecolor{currentstroke}{rgb}{0.000000,0.000000,0.000000}%
\pgfsetstrokecolor{currentstroke}%
\pgfsetdash{}{0pt}%
\pgfpathmoveto{\pgfqpoint{0.625000in}{2.085844in}}%
\pgfpathlineto{\pgfqpoint{0.626937in}{2.084211in}}%
\pgfpathlineto{\pgfqpoint{0.626051in}{2.074561in}}%
\pgfpathlineto{\pgfqpoint{0.625000in}{2.071560in}}%
\pgfusepath{stroke}%
\end{pgfscope}%
\begin{pgfscope}%
\pgfpathrectangle{\pgfqpoint{0.625000in}{0.550000in}}{\pgfqpoint{3.875000in}{3.850000in}} %
\pgfusepath{clip}%
\pgfsetbuttcap%
\pgfsetroundjoin%
\pgfsetlinewidth{0.501875pt}%
\definecolor{currentstroke}{rgb}{0.000000,0.000000,0.000000}%
\pgfsetstrokecolor{currentstroke}%
\pgfsetdash{}{0pt}%
\pgfpathmoveto{\pgfqpoint{0.625000in}{2.133445in}}%
\pgfpathlineto{\pgfqpoint{0.625879in}{2.132456in}}%
\pgfpathlineto{\pgfqpoint{0.625294in}{2.122807in}}%
\pgfpathlineto{\pgfqpoint{0.626058in}{2.113158in}}%
\pgfpathlineto{\pgfqpoint{0.625000in}{2.111121in}}%
\pgfusepath{stroke}%
\end{pgfscope}%
\begin{pgfscope}%
\pgfpathrectangle{\pgfqpoint{0.625000in}{0.550000in}}{\pgfqpoint{3.875000in}{3.850000in}} %
\pgfusepath{clip}%
\pgfsetbuttcap%
\pgfsetroundjoin%
\pgfsetlinewidth{0.501875pt}%
\definecolor{currentstroke}{rgb}{0.000000,0.000000,0.000000}%
\pgfsetstrokecolor{currentstroke}%
\pgfsetdash{}{0pt}%
\pgfpathmoveto{\pgfqpoint{0.625000in}{2.175692in}}%
\pgfpathlineto{\pgfqpoint{0.628953in}{2.171053in}}%
\pgfpathlineto{\pgfqpoint{0.625000in}{2.166812in}}%
\pgfusepath{stroke}%
\end{pgfscope}%
\begin{pgfscope}%
\pgfpathrectangle{\pgfqpoint{0.625000in}{0.550000in}}{\pgfqpoint{3.875000in}{3.850000in}} %
\pgfusepath{clip}%
\pgfsetbuttcap%
\pgfsetroundjoin%
\pgfsetlinewidth{0.501875pt}%
\definecolor{currentstroke}{rgb}{0.000000,0.000000,0.000000}%
\pgfsetstrokecolor{currentstroke}%
\pgfsetdash{}{0pt}%
\pgfpathmoveto{\pgfqpoint{0.625000in}{2.268837in}}%
\pgfpathlineto{\pgfqpoint{0.631722in}{2.277193in}}%
\pgfpathlineto{\pgfqpoint{0.625000in}{2.285549in}}%
\pgfusepath{stroke}%
\end{pgfscope}%
\begin{pgfscope}%
\pgfpathrectangle{\pgfqpoint{0.625000in}{0.550000in}}{\pgfqpoint{3.875000in}{3.850000in}} %
\pgfusepath{clip}%
\pgfsetbuttcap%
\pgfsetroundjoin%
\pgfsetlinewidth{0.501875pt}%
\definecolor{currentstroke}{rgb}{0.000000,0.000000,0.000000}%
\pgfsetstrokecolor{currentstroke}%
\pgfsetdash{}{0pt}%
\pgfpathmoveto{\pgfqpoint{0.625000in}{2.287653in}}%
\pgfpathlineto{\pgfqpoint{0.634712in}{2.291396in}}%
\pgfpathlineto{\pgfqpoint{0.644424in}{2.292215in}}%
\pgfpathlineto{\pgfqpoint{0.654135in}{2.294154in}}%
\pgfpathlineto{\pgfqpoint{0.661496in}{2.296491in}}%
\pgfpathlineto{\pgfqpoint{0.663847in}{2.297208in}}%
\pgfpathlineto{\pgfqpoint{0.673559in}{2.301356in}}%
\pgfpathlineto{\pgfqpoint{0.681720in}{2.306140in}}%
\pgfpathlineto{\pgfqpoint{0.683271in}{2.307062in}}%
\pgfpathlineto{\pgfqpoint{0.692982in}{2.314461in}}%
\pgfpathlineto{\pgfqpoint{0.694401in}{2.315789in}}%
\pgfpathlineto{\pgfqpoint{0.702694in}{2.324302in}}%
\pgfpathlineto{\pgfqpoint{0.703621in}{2.325439in}}%
\pgfpathlineto{\pgfqpoint{0.710586in}{2.335088in}}%
\pgfpathlineto{\pgfqpoint{0.712406in}{2.338126in}}%
\pgfpathlineto{\pgfqpoint{0.715897in}{2.344737in}}%
\pgfpathlineto{\pgfqpoint{0.719808in}{2.354386in}}%
\pgfpathlineto{\pgfqpoint{0.722118in}{2.362636in}}%
\pgfpathlineto{\pgfqpoint{0.722485in}{2.364035in}}%
\pgfpathlineto{\pgfqpoint{0.724150in}{2.373684in}}%
\pgfpathlineto{\pgfqpoint{0.724757in}{2.383333in}}%
\pgfpathlineto{\pgfqpoint{0.724326in}{2.392982in}}%
\pgfpathlineto{\pgfqpoint{0.722807in}{2.402632in}}%
\pgfpathlineto{\pgfqpoint{0.722118in}{2.404944in}}%
\pgfpathlineto{\pgfqpoint{0.720359in}{2.412281in}}%
\pgfpathlineto{\pgfqpoint{0.716783in}{2.421930in}}%
\pgfpathlineto{\pgfqpoint{0.712406in}{2.429841in}}%
\pgfpathlineto{\pgfqpoint{0.711644in}{2.431579in}}%
\pgfpathlineto{\pgfqpoint{0.705314in}{2.441228in}}%
\pgfpathlineto{\pgfqpoint{0.702694in}{2.443816in}}%
\pgfpathlineto{\pgfqpoint{0.696895in}{2.450877in}}%
\pgfpathlineto{\pgfqpoint{0.692982in}{2.456420in}}%
\pgfpathlineto{\pgfqpoint{0.687815in}{2.460526in}}%
\pgfpathlineto{\pgfqpoint{0.692982in}{2.462350in}}%
\pgfpathlineto{\pgfqpoint{0.696646in}{2.460526in}}%
\pgfpathlineto{\pgfqpoint{0.702694in}{2.456164in}}%
\pgfpathlineto{\pgfqpoint{0.710576in}{2.450877in}}%
\pgfpathlineto{\pgfqpoint{0.712406in}{2.449047in}}%
\pgfpathlineto{\pgfqpoint{0.721417in}{2.441228in}}%
\pgfpathlineto{\pgfqpoint{0.722118in}{2.440349in}}%
\pgfpathlineto{\pgfqpoint{0.730345in}{2.431579in}}%
\pgfpathlineto{\pgfqpoint{0.731830in}{2.429356in}}%
\pgfpathlineto{\pgfqpoint{0.737688in}{2.421930in}}%
\pgfpathlineto{\pgfqpoint{0.741541in}{2.415243in}}%
\pgfpathlineto{\pgfqpoint{0.743552in}{2.412281in}}%
\pgfpathlineto{\pgfqpoint{0.748417in}{2.402632in}}%
\pgfpathlineto{\pgfqpoint{0.751253in}{2.395066in}}%
\pgfpathlineto{\pgfqpoint{0.752174in}{2.392982in}}%
\pgfpathlineto{\pgfqpoint{0.755216in}{2.383333in}}%
\pgfpathlineto{\pgfqpoint{0.757319in}{2.373684in}}%
\pgfpathlineto{\pgfqpoint{0.758604in}{2.364035in}}%
\pgfpathlineto{\pgfqpoint{0.759128in}{2.354386in}}%
\pgfpathlineto{\pgfqpoint{0.758904in}{2.344737in}}%
\pgfpathlineto{\pgfqpoint{0.757914in}{2.335088in}}%
\pgfpathlineto{\pgfqpoint{0.756121in}{2.325439in}}%
\pgfpathlineto{\pgfqpoint{0.753473in}{2.315789in}}%
\pgfpathlineto{\pgfqpoint{0.751253in}{2.309587in}}%
\pgfpathlineto{\pgfqpoint{0.749938in}{2.306140in}}%
\pgfpathlineto{\pgfqpoint{0.745448in}{2.296491in}}%
\pgfpathlineto{\pgfqpoint{0.741541in}{2.289587in}}%
\pgfpathlineto{\pgfqpoint{0.739829in}{2.286842in}}%
\pgfpathlineto{\pgfqpoint{0.732924in}{2.277193in}}%
\pgfpathlineto{\pgfqpoint{0.731830in}{2.275809in}}%
\pgfpathlineto{\pgfqpoint{0.724397in}{2.267544in}}%
\pgfpathlineto{\pgfqpoint{0.722118in}{2.265217in}}%
\pgfpathlineto{\pgfqpoint{0.713759in}{2.257895in}}%
\pgfpathlineto{\pgfqpoint{0.712406in}{2.256771in}}%
\pgfpathlineto{\pgfqpoint{0.702694in}{2.249888in}}%
\pgfpathlineto{\pgfqpoint{0.699879in}{2.248246in}}%
\pgfpathlineto{\pgfqpoint{0.692982in}{2.244268in}}%
\pgfpathlineto{\pgfqpoint{0.683271in}{2.239808in}}%
\pgfpathlineto{\pgfqpoint{0.679921in}{2.238596in}}%
\pgfpathlineto{\pgfqpoint{0.673559in}{2.236247in}}%
\pgfpathlineto{\pgfqpoint{0.663847in}{2.233565in}}%
\pgfpathlineto{\pgfqpoint{0.654135in}{2.231804in}}%
\pgfpathlineto{\pgfqpoint{0.644424in}{2.231332in}}%
\pgfpathlineto{\pgfqpoint{0.634712in}{2.233615in}}%
\pgfpathlineto{\pgfqpoint{0.630014in}{2.228947in}}%
\pgfpathlineto{\pgfqpoint{0.633225in}{2.219298in}}%
\pgfpathlineto{\pgfqpoint{0.628626in}{2.209649in}}%
\pgfpathlineto{\pgfqpoint{0.625000in}{2.208609in}}%
\pgfusepath{stroke}%
\end{pgfscope}%
\begin{pgfscope}%
\pgfpathrectangle{\pgfqpoint{0.625000in}{0.550000in}}{\pgfqpoint{3.875000in}{3.850000in}} %
\pgfusepath{clip}%
\pgfsetbuttcap%
\pgfsetroundjoin%
\pgfsetlinewidth{0.501875pt}%
\definecolor{currentstroke}{rgb}{0.000000,0.000000,0.000000}%
\pgfsetstrokecolor{currentstroke}%
\pgfsetdash{}{0pt}%
\pgfpathmoveto{\pgfqpoint{0.625000in}{2.326313in}}%
\pgfpathlineto{\pgfqpoint{0.625798in}{2.325439in}}%
\pgfpathlineto{\pgfqpoint{0.627159in}{2.315789in}}%
\pgfpathlineto{\pgfqpoint{0.625000in}{2.313896in}}%
\pgfusepath{stroke}%
\end{pgfscope}%
\begin{pgfscope}%
\pgfpathrectangle{\pgfqpoint{0.625000in}{0.550000in}}{\pgfqpoint{3.875000in}{3.850000in}} %
\pgfusepath{clip}%
\pgfsetbuttcap%
\pgfsetroundjoin%
\pgfsetlinewidth{0.501875pt}%
\definecolor{currentstroke}{rgb}{0.000000,0.000000,0.000000}%
\pgfsetstrokecolor{currentstroke}%
\pgfsetdash{}{0pt}%
\pgfpathmoveto{\pgfqpoint{0.625000in}{2.365582in}}%
\pgfpathlineto{\pgfqpoint{0.633657in}{2.373684in}}%
\pgfpathlineto{\pgfqpoint{0.634712in}{2.374551in}}%
\pgfpathlineto{\pgfqpoint{0.644424in}{2.376831in}}%
\pgfpathlineto{\pgfqpoint{0.654135in}{2.381431in}}%
\pgfpathlineto{\pgfqpoint{0.656762in}{2.383333in}}%
\pgfpathlineto{\pgfqpoint{0.663847in}{2.388855in}}%
\pgfpathlineto{\pgfqpoint{0.667510in}{2.392982in}}%
\pgfpathlineto{\pgfqpoint{0.673559in}{2.401852in}}%
\pgfpathlineto{\pgfqpoint{0.674646in}{2.402632in}}%
\pgfpathlineto{\pgfqpoint{0.678180in}{2.412281in}}%
\pgfpathlineto{\pgfqpoint{0.679701in}{2.421930in}}%
\pgfpathlineto{\pgfqpoint{0.679891in}{2.431579in}}%
\pgfpathlineto{\pgfqpoint{0.682063in}{2.441228in}}%
\pgfpathlineto{\pgfqpoint{0.683271in}{2.442502in}}%
\pgfpathlineto{\pgfqpoint{0.684117in}{2.441228in}}%
\pgfpathlineto{\pgfqpoint{0.687389in}{2.431579in}}%
\pgfpathlineto{\pgfqpoint{0.688126in}{2.421930in}}%
\pgfpathlineto{\pgfqpoint{0.688598in}{2.412281in}}%
\pgfpathlineto{\pgfqpoint{0.686844in}{2.402632in}}%
\pgfpathlineto{\pgfqpoint{0.683329in}{2.392982in}}%
\pgfpathlineto{\pgfqpoint{0.683271in}{2.392861in}}%
\pgfpathlineto{\pgfqpoint{0.677811in}{2.383333in}}%
\pgfpathlineto{\pgfqpoint{0.673559in}{2.377826in}}%
\pgfpathlineto{\pgfqpoint{0.669325in}{2.373684in}}%
\pgfpathlineto{\pgfqpoint{0.663847in}{2.368926in}}%
\pgfpathlineto{\pgfqpoint{0.655541in}{2.364035in}}%
\pgfpathlineto{\pgfqpoint{0.654135in}{2.363219in}}%
\pgfpathlineto{\pgfqpoint{0.644424in}{2.359423in}}%
\pgfpathlineto{\pgfqpoint{0.634712in}{2.357495in}}%
\pgfpathlineto{\pgfqpoint{0.631819in}{2.354386in}}%
\pgfpathlineto{\pgfqpoint{0.625000in}{2.350973in}}%
\pgfusepath{stroke}%
\end{pgfscope}%
\begin{pgfscope}%
\pgfpathrectangle{\pgfqpoint{0.625000in}{0.550000in}}{\pgfqpoint{3.875000in}{3.850000in}} %
\pgfusepath{clip}%
\pgfsetbuttcap%
\pgfsetroundjoin%
\pgfsetlinewidth{0.501875pt}%
\definecolor{currentstroke}{rgb}{0.000000,0.000000,0.000000}%
\pgfsetstrokecolor{currentstroke}%
\pgfsetdash{}{0pt}%
\pgfpathmoveto{\pgfqpoint{0.625000in}{2.408317in}}%
\pgfpathlineto{\pgfqpoint{0.634712in}{2.405904in}}%
\pgfpathlineto{\pgfqpoint{0.643802in}{2.402632in}}%
\pgfpathlineto{\pgfqpoint{0.634712in}{2.399222in}}%
\pgfpathlineto{\pgfqpoint{0.625000in}{2.396521in}}%
\pgfusepath{stroke}%
\end{pgfscope}%
\begin{pgfscope}%
\pgfpathrectangle{\pgfqpoint{0.625000in}{0.550000in}}{\pgfqpoint{3.875000in}{3.850000in}} %
\pgfusepath{clip}%
\pgfsetbuttcap%
\pgfsetroundjoin%
\pgfsetlinewidth{0.501875pt}%
\definecolor{currentstroke}{rgb}{0.000000,0.000000,0.000000}%
\pgfsetstrokecolor{currentstroke}%
\pgfsetdash{}{0pt}%
\pgfpathmoveto{\pgfqpoint{0.625000in}{2.444479in}}%
\pgfpathlineto{\pgfqpoint{0.628092in}{2.441228in}}%
\pgfpathlineto{\pgfqpoint{0.626242in}{2.431579in}}%
\pgfpathlineto{\pgfqpoint{0.634712in}{2.425826in}}%
\pgfpathlineto{\pgfqpoint{0.637991in}{2.421930in}}%
\pgfpathlineto{\pgfqpoint{0.634712in}{2.419038in}}%
\pgfpathlineto{\pgfqpoint{0.625000in}{2.419779in}}%
\pgfusepath{stroke}%
\end{pgfscope}%
\begin{pgfscope}%
\pgfpathrectangle{\pgfqpoint{0.625000in}{0.550000in}}{\pgfqpoint{3.875000in}{3.850000in}} %
\pgfusepath{clip}%
\pgfsetbuttcap%
\pgfsetroundjoin%
\pgfsetlinewidth{0.501875pt}%
\definecolor{currentstroke}{rgb}{0.000000,0.000000,0.000000}%
\pgfsetstrokecolor{currentstroke}%
\pgfsetdash{}{0pt}%
\pgfpathmoveto{\pgfqpoint{0.625000in}{2.539871in}}%
\pgfpathlineto{\pgfqpoint{0.634712in}{2.540611in}}%
\pgfpathlineto{\pgfqpoint{0.637991in}{2.537719in}}%
\pgfpathlineto{\pgfqpoint{0.634712in}{2.533823in}}%
\pgfpathlineto{\pgfqpoint{0.626242in}{2.528070in}}%
\pgfpathlineto{\pgfqpoint{0.628092in}{2.518421in}}%
\pgfpathlineto{\pgfqpoint{0.625000in}{2.515170in}}%
\pgfusepath{stroke}%
\end{pgfscope}%
\begin{pgfscope}%
\pgfpathrectangle{\pgfqpoint{0.625000in}{0.550000in}}{\pgfqpoint{3.875000in}{3.850000in}} %
\pgfusepath{clip}%
\pgfsetbuttcap%
\pgfsetroundjoin%
\pgfsetlinewidth{0.501875pt}%
\definecolor{currentstroke}{rgb}{0.000000,0.000000,0.000000}%
\pgfsetstrokecolor{currentstroke}%
\pgfsetdash{}{0pt}%
\pgfpathmoveto{\pgfqpoint{0.625000in}{2.563128in}}%
\pgfpathlineto{\pgfqpoint{0.634712in}{2.560427in}}%
\pgfpathlineto{\pgfqpoint{0.643802in}{2.557018in}}%
\pgfpathlineto{\pgfqpoint{0.634712in}{2.553745in}}%
\pgfpathlineto{\pgfqpoint{0.625000in}{2.551332in}}%
\pgfusepath{stroke}%
\end{pgfscope}%
\begin{pgfscope}%
\pgfpathrectangle{\pgfqpoint{0.625000in}{0.550000in}}{\pgfqpoint{3.875000in}{3.850000in}} %
\pgfusepath{clip}%
\pgfsetbuttcap%
\pgfsetroundjoin%
\pgfsetlinewidth{0.501875pt}%
\definecolor{currentstroke}{rgb}{0.000000,0.000000,0.000000}%
\pgfsetstrokecolor{currentstroke}%
\pgfsetdash{}{0pt}%
\pgfpathmoveto{\pgfqpoint{0.625000in}{2.608676in}}%
\pgfpathlineto{\pgfqpoint{0.631819in}{2.605263in}}%
\pgfpathlineto{\pgfqpoint{0.634712in}{2.602154in}}%
\pgfpathlineto{\pgfqpoint{0.644424in}{2.600226in}}%
\pgfpathlineto{\pgfqpoint{0.654135in}{2.596430in}}%
\pgfpathlineto{\pgfqpoint{0.655541in}{2.595614in}}%
\pgfpathlineto{\pgfqpoint{0.663847in}{2.590723in}}%
\pgfpathlineto{\pgfqpoint{0.669325in}{2.585965in}}%
\pgfpathlineto{\pgfqpoint{0.673559in}{2.581824in}}%
\pgfpathlineto{\pgfqpoint{0.677811in}{2.576316in}}%
\pgfpathlineto{\pgfqpoint{0.683271in}{2.566788in}}%
\pgfpathlineto{\pgfqpoint{0.683329in}{2.566667in}}%
\pgfpathlineto{\pgfqpoint{0.686844in}{2.557018in}}%
\pgfpathlineto{\pgfqpoint{0.688598in}{2.547368in}}%
\pgfpathlineto{\pgfqpoint{0.688126in}{2.537719in}}%
\pgfpathlineto{\pgfqpoint{0.687389in}{2.528070in}}%
\pgfpathlineto{\pgfqpoint{0.684117in}{2.518421in}}%
\pgfpathlineto{\pgfqpoint{0.683271in}{2.517148in}}%
\pgfpathlineto{\pgfqpoint{0.682063in}{2.518421in}}%
\pgfpathlineto{\pgfqpoint{0.679891in}{2.528070in}}%
\pgfpathlineto{\pgfqpoint{0.679701in}{2.537719in}}%
\pgfpathlineto{\pgfqpoint{0.678180in}{2.547368in}}%
\pgfpathlineto{\pgfqpoint{0.674646in}{2.557018in}}%
\pgfpathlineto{\pgfqpoint{0.673559in}{2.557797in}}%
\pgfpathlineto{\pgfqpoint{0.667510in}{2.566667in}}%
\pgfpathlineto{\pgfqpoint{0.663847in}{2.570794in}}%
\pgfpathlineto{\pgfqpoint{0.656762in}{2.576316in}}%
\pgfpathlineto{\pgfqpoint{0.654135in}{2.578218in}}%
\pgfpathlineto{\pgfqpoint{0.644424in}{2.582818in}}%
\pgfpathlineto{\pgfqpoint{0.634712in}{2.585098in}}%
\pgfpathlineto{\pgfqpoint{0.633657in}{2.585965in}}%
\pgfpathlineto{\pgfqpoint{0.625000in}{2.594067in}}%
\pgfusepath{stroke}%
\end{pgfscope}%
\begin{pgfscope}%
\pgfpathrectangle{\pgfqpoint{0.625000in}{0.550000in}}{\pgfqpoint{3.875000in}{3.850000in}} %
\pgfusepath{clip}%
\pgfsetbuttcap%
\pgfsetroundjoin%
\pgfsetlinewidth{0.501875pt}%
\definecolor{currentstroke}{rgb}{0.000000,0.000000,0.000000}%
\pgfsetstrokecolor{currentstroke}%
\pgfsetdash{}{0pt}%
\pgfpathmoveto{\pgfqpoint{0.625000in}{2.645754in}}%
\pgfpathlineto{\pgfqpoint{0.627159in}{2.643860in}}%
\pgfpathlineto{\pgfqpoint{0.625000in}{2.641707in}}%
\pgfusepath{stroke}%
\end{pgfscope}%
\begin{pgfscope}%
\pgfpathrectangle{\pgfqpoint{0.625000in}{0.550000in}}{\pgfqpoint{3.875000in}{3.850000in}} %
\pgfusepath{clip}%
\pgfsetbuttcap%
\pgfsetroundjoin%
\pgfsetlinewidth{0.501875pt}%
\definecolor{currentstroke}{rgb}{0.000000,0.000000,0.000000}%
\pgfsetstrokecolor{currentstroke}%
\pgfsetdash{}{0pt}%
\pgfpathmoveto{\pgfqpoint{0.625000in}{2.674100in}}%
\pgfpathlineto{\pgfqpoint{0.631722in}{2.682456in}}%
\pgfpathlineto{\pgfqpoint{0.625000in}{2.690812in}}%
\pgfusepath{stroke}%
\end{pgfscope}%
\begin{pgfscope}%
\pgfpathrectangle{\pgfqpoint{0.625000in}{0.550000in}}{\pgfqpoint{3.875000in}{3.850000in}} %
\pgfusepath{clip}%
\pgfsetbuttcap%
\pgfsetroundjoin%
\pgfsetlinewidth{0.501875pt}%
\definecolor{currentstroke}{rgb}{0.000000,0.000000,0.000000}%
\pgfsetstrokecolor{currentstroke}%
\pgfsetdash{}{0pt}%
\pgfpathmoveto{\pgfqpoint{0.625000in}{2.751040in}}%
\pgfpathlineto{\pgfqpoint{0.628626in}{2.750000in}}%
\pgfpathlineto{\pgfqpoint{0.633225in}{2.740351in}}%
\pgfpathlineto{\pgfqpoint{0.630014in}{2.730702in}}%
\pgfpathlineto{\pgfqpoint{0.634712in}{2.726034in}}%
\pgfpathlineto{\pgfqpoint{0.644424in}{2.728317in}}%
\pgfpathlineto{\pgfqpoint{0.654135in}{2.727845in}}%
\pgfpathlineto{\pgfqpoint{0.663847in}{2.726084in}}%
\pgfpathlineto{\pgfqpoint{0.673559in}{2.723402in}}%
\pgfpathlineto{\pgfqpoint{0.679921in}{2.721053in}}%
\pgfpathlineto{\pgfqpoint{0.683271in}{2.719841in}}%
\pgfpathlineto{\pgfqpoint{0.692982in}{2.715381in}}%
\pgfpathlineto{\pgfqpoint{0.699879in}{2.711404in}}%
\pgfpathlineto{\pgfqpoint{0.702694in}{2.709762in}}%
\pgfpathlineto{\pgfqpoint{0.712406in}{2.702878in}}%
\pgfpathlineto{\pgfqpoint{0.713759in}{2.701754in}}%
\pgfpathlineto{\pgfqpoint{0.722118in}{2.694432in}}%
\pgfpathlineto{\pgfqpoint{0.724397in}{2.692105in}}%
\pgfpathlineto{\pgfqpoint{0.731830in}{2.683840in}}%
\pgfpathlineto{\pgfqpoint{0.732924in}{2.682456in}}%
\pgfpathlineto{\pgfqpoint{0.739829in}{2.672807in}}%
\pgfpathlineto{\pgfqpoint{0.741541in}{2.670062in}}%
\pgfpathlineto{\pgfqpoint{0.745448in}{2.663158in}}%
\pgfpathlineto{\pgfqpoint{0.749938in}{2.653509in}}%
\pgfpathlineto{\pgfqpoint{0.751253in}{2.650062in}}%
\pgfpathlineto{\pgfqpoint{0.753473in}{2.643860in}}%
\pgfpathlineto{\pgfqpoint{0.756121in}{2.634211in}}%
\pgfpathlineto{\pgfqpoint{0.757914in}{2.624561in}}%
\pgfpathlineto{\pgfqpoint{0.758904in}{2.614912in}}%
\pgfpathlineto{\pgfqpoint{0.759128in}{2.605263in}}%
\pgfpathlineto{\pgfqpoint{0.758604in}{2.595614in}}%
\pgfpathlineto{\pgfqpoint{0.757319in}{2.585965in}}%
\pgfpathlineto{\pgfqpoint{0.755216in}{2.576316in}}%
\pgfpathlineto{\pgfqpoint{0.752174in}{2.566667in}}%
\pgfpathlineto{\pgfqpoint{0.751253in}{2.564583in}}%
\pgfpathlineto{\pgfqpoint{0.748417in}{2.557018in}}%
\pgfpathlineto{\pgfqpoint{0.743552in}{2.547368in}}%
\pgfpathlineto{\pgfqpoint{0.741541in}{2.544406in}}%
\pgfpathlineto{\pgfqpoint{0.737688in}{2.537719in}}%
\pgfpathlineto{\pgfqpoint{0.731830in}{2.530294in}}%
\pgfpathlineto{\pgfqpoint{0.730345in}{2.528070in}}%
\pgfpathlineto{\pgfqpoint{0.722118in}{2.519300in}}%
\pgfpathlineto{\pgfqpoint{0.721417in}{2.518421in}}%
\pgfpathlineto{\pgfqpoint{0.712406in}{2.510602in}}%
\pgfpathlineto{\pgfqpoint{0.710576in}{2.508772in}}%
\pgfpathlineto{\pgfqpoint{0.702694in}{2.503485in}}%
\pgfpathlineto{\pgfqpoint{0.696646in}{2.499123in}}%
\pgfpathlineto{\pgfqpoint{0.692982in}{2.497299in}}%
\pgfpathlineto{\pgfqpoint{0.687815in}{2.499123in}}%
\pgfpathlineto{\pgfqpoint{0.692982in}{2.503229in}}%
\pgfpathlineto{\pgfqpoint{0.696895in}{2.508772in}}%
\pgfpathlineto{\pgfqpoint{0.702694in}{2.515833in}}%
\pgfpathlineto{\pgfqpoint{0.705314in}{2.518421in}}%
\pgfpathlineto{\pgfqpoint{0.711644in}{2.528070in}}%
\pgfpathlineto{\pgfqpoint{0.712406in}{2.529808in}}%
\pgfpathlineto{\pgfqpoint{0.716783in}{2.537719in}}%
\pgfpathlineto{\pgfqpoint{0.720359in}{2.547368in}}%
\pgfpathlineto{\pgfqpoint{0.722118in}{2.554705in}}%
\pgfpathlineto{\pgfqpoint{0.722807in}{2.557018in}}%
\pgfpathlineto{\pgfqpoint{0.724326in}{2.566667in}}%
\pgfpathlineto{\pgfqpoint{0.724757in}{2.576316in}}%
\pgfpathlineto{\pgfqpoint{0.724150in}{2.585965in}}%
\pgfpathlineto{\pgfqpoint{0.722485in}{2.595614in}}%
\pgfpathlineto{\pgfqpoint{0.722118in}{2.597013in}}%
\pgfpathlineto{\pgfqpoint{0.719808in}{2.605263in}}%
\pgfpathlineto{\pgfqpoint{0.715897in}{2.614912in}}%
\pgfpathlineto{\pgfqpoint{0.712406in}{2.621523in}}%
\pgfpathlineto{\pgfqpoint{0.710586in}{2.624561in}}%
\pgfpathlineto{\pgfqpoint{0.703621in}{2.634211in}}%
\pgfpathlineto{\pgfqpoint{0.702694in}{2.635347in}}%
\pgfpathlineto{\pgfqpoint{0.694401in}{2.643860in}}%
\pgfpathlineto{\pgfqpoint{0.692982in}{2.645188in}}%
\pgfpathlineto{\pgfqpoint{0.683271in}{2.652587in}}%
\pgfpathlineto{\pgfqpoint{0.681720in}{2.653509in}}%
\pgfpathlineto{\pgfqpoint{0.673559in}{2.658294in}}%
\pgfpathlineto{\pgfqpoint{0.663847in}{2.662441in}}%
\pgfpathlineto{\pgfqpoint{0.661496in}{2.663158in}}%
\pgfpathlineto{\pgfqpoint{0.654135in}{2.665495in}}%
\pgfpathlineto{\pgfqpoint{0.644424in}{2.667434in}}%
\pgfpathlineto{\pgfqpoint{0.634712in}{2.668254in}}%
\pgfpathlineto{\pgfqpoint{0.625000in}{2.671996in}}%
\pgfusepath{stroke}%
\end{pgfscope}%
\begin{pgfscope}%
\pgfpathrectangle{\pgfqpoint{0.625000in}{0.550000in}}{\pgfqpoint{3.875000in}{3.850000in}} %
\pgfusepath{clip}%
\pgfsetbuttcap%
\pgfsetroundjoin%
\pgfsetlinewidth{0.501875pt}%
\definecolor{currentstroke}{rgb}{0.000000,0.000000,0.000000}%
\pgfsetstrokecolor{currentstroke}%
\pgfsetdash{}{0pt}%
\pgfpathmoveto{\pgfqpoint{0.625000in}{2.848528in}}%
\pgfpathlineto{\pgfqpoint{0.626058in}{2.846491in}}%
\pgfpathlineto{\pgfqpoint{0.625294in}{2.836842in}}%
\pgfpathlineto{\pgfqpoint{0.625879in}{2.827193in}}%
\pgfpathlineto{\pgfqpoint{0.625000in}{2.826204in}}%
\pgfusepath{stroke}%
\end{pgfscope}%
\begin{pgfscope}%
\pgfpathrectangle{\pgfqpoint{0.625000in}{0.550000in}}{\pgfqpoint{3.875000in}{3.850000in}} %
\pgfusepath{clip}%
\pgfsetbuttcap%
\pgfsetroundjoin%
\pgfsetlinewidth{0.501875pt}%
\definecolor{currentstroke}{rgb}{0.000000,0.000000,0.000000}%
\pgfsetstrokecolor{currentstroke}%
\pgfsetdash{}{0pt}%
\pgfpathmoveto{\pgfqpoint{0.625000in}{2.888089in}}%
\pgfpathlineto{\pgfqpoint{0.626051in}{2.885088in}}%
\pgfpathlineto{\pgfqpoint{0.626937in}{2.875439in}}%
\pgfpathlineto{\pgfqpoint{0.625000in}{2.873805in}}%
\pgfusepath{stroke}%
\end{pgfscope}%
\begin{pgfscope}%
\pgfpathrectangle{\pgfqpoint{0.625000in}{0.550000in}}{\pgfqpoint{3.875000in}{3.850000in}} %
\pgfusepath{clip}%
\pgfsetbuttcap%
\pgfsetroundjoin%
\pgfsetlinewidth{0.501875pt}%
\definecolor{currentstroke}{rgb}{0.000000,0.000000,0.000000}%
\pgfsetstrokecolor{currentstroke}%
\pgfsetdash{}{0pt}%
\pgfpathmoveto{\pgfqpoint{0.625000in}{2.914216in}}%
\pgfpathlineto{\pgfqpoint{0.625130in}{2.914035in}}%
\pgfpathlineto{\pgfqpoint{0.626274in}{2.904386in}}%
\pgfpathlineto{\pgfqpoint{0.625000in}{2.901385in}}%
\pgfusepath{stroke}%
\end{pgfscope}%
\begin{pgfscope}%
\pgfpathrectangle{\pgfqpoint{0.625000in}{0.550000in}}{\pgfqpoint{3.875000in}{3.850000in}} %
\pgfusepath{clip}%
\pgfsetbuttcap%
\pgfsetroundjoin%
\pgfsetlinewidth{0.501875pt}%
\definecolor{currentstroke}{rgb}{0.000000,0.000000,0.000000}%
\pgfsetstrokecolor{currentstroke}%
\pgfsetdash{}{0pt}%
\pgfpathmoveto{\pgfqpoint{0.625000in}{2.946648in}}%
\pgfpathlineto{\pgfqpoint{0.627619in}{2.942982in}}%
\pgfpathlineto{\pgfqpoint{0.625000in}{2.940274in}}%
\pgfusepath{stroke}%
\end{pgfscope}%
\begin{pgfscope}%
\pgfpathrectangle{\pgfqpoint{0.625000in}{0.550000in}}{\pgfqpoint{3.875000in}{3.850000in}} %
\pgfusepath{clip}%
\pgfsetbuttcap%
\pgfsetroundjoin%
\pgfsetlinewidth{0.501875pt}%
\definecolor{currentstroke}{rgb}{0.000000,0.000000,0.000000}%
\pgfsetstrokecolor{currentstroke}%
\pgfsetdash{}{0pt}%
\pgfpathmoveto{\pgfqpoint{0.625000in}{3.029553in}}%
\pgfpathlineto{\pgfqpoint{0.634712in}{3.021710in}}%
\pgfpathlineto{\pgfqpoint{0.636369in}{3.020175in}}%
\pgfpathlineto{\pgfqpoint{0.634712in}{3.017775in}}%
\pgfpathlineto{\pgfqpoint{0.625465in}{3.010526in}}%
\pgfpathlineto{\pgfqpoint{0.626899in}{3.000877in}}%
\pgfpathlineto{\pgfqpoint{0.629529in}{2.991228in}}%
\pgfpathlineto{\pgfqpoint{0.626874in}{2.981579in}}%
\pgfpathlineto{\pgfqpoint{0.625000in}{2.977476in}}%
\pgfusepath{stroke}%
\end{pgfscope}%
\begin{pgfscope}%
\pgfpathrectangle{\pgfqpoint{0.625000in}{0.550000in}}{\pgfqpoint{3.875000in}{3.850000in}} %
\pgfusepath{clip}%
\pgfsetbuttcap%
\pgfsetroundjoin%
\pgfsetlinewidth{0.501875pt}%
\definecolor{currentstroke}{rgb}{0.000000,0.000000,0.000000}%
\pgfsetstrokecolor{currentstroke}%
\pgfsetdash{}{0pt}%
\pgfpathmoveto{\pgfqpoint{0.625000in}{3.040398in}}%
\pgfpathlineto{\pgfqpoint{0.625848in}{3.039474in}}%
\pgfpathlineto{\pgfqpoint{0.625000in}{3.031044in}}%
\pgfusepath{stroke}%
\end{pgfscope}%
\begin{pgfscope}%
\pgfpathrectangle{\pgfqpoint{0.625000in}{0.550000in}}{\pgfqpoint{3.875000in}{3.850000in}} %
\pgfusepath{clip}%
\pgfsetbuttcap%
\pgfsetroundjoin%
\pgfsetlinewidth{0.501875pt}%
\definecolor{currentstroke}{rgb}{0.000000,0.000000,0.000000}%
\pgfsetstrokecolor{currentstroke}%
\pgfsetdash{}{0pt}%
\pgfpathmoveto{\pgfqpoint{0.625000in}{3.091235in}}%
\pgfpathlineto{\pgfqpoint{0.628978in}{3.087719in}}%
\pgfpathlineto{\pgfqpoint{0.625000in}{3.084026in}}%
\pgfusepath{stroke}%
\end{pgfscope}%
\begin{pgfscope}%
\pgfpathrectangle{\pgfqpoint{0.625000in}{0.550000in}}{\pgfqpoint{3.875000in}{3.850000in}} %
\pgfusepath{clip}%
\pgfsetbuttcap%
\pgfsetroundjoin%
\pgfsetlinewidth{0.501875pt}%
\definecolor{currentstroke}{rgb}{0.000000,0.000000,0.000000}%
\pgfsetstrokecolor{currentstroke}%
\pgfsetdash{}{0pt}%
\pgfpathmoveto{\pgfqpoint{0.625000in}{3.136836in}}%
\pgfpathlineto{\pgfqpoint{0.630927in}{3.135965in}}%
\pgfpathlineto{\pgfqpoint{0.625229in}{3.126316in}}%
\pgfpathlineto{\pgfqpoint{0.625000in}{3.126244in}}%
\pgfusepath{stroke}%
\end{pgfscope}%
\begin{pgfscope}%
\pgfpathrectangle{\pgfqpoint{0.625000in}{0.550000in}}{\pgfqpoint{3.875000in}{3.850000in}} %
\pgfusepath{clip}%
\pgfsetbuttcap%
\pgfsetroundjoin%
\pgfsetlinewidth{0.501875pt}%
\definecolor{currentstroke}{rgb}{0.000000,0.000000,0.000000}%
\pgfsetstrokecolor{currentstroke}%
\pgfsetdash{}{0pt}%
\pgfpathmoveto{\pgfqpoint{0.625000in}{3.159053in}}%
\pgfpathlineto{\pgfqpoint{0.633344in}{3.164912in}}%
\pgfpathlineto{\pgfqpoint{0.625000in}{3.167191in}}%
\pgfusepath{stroke}%
\end{pgfscope}%
\begin{pgfscope}%
\pgfpathrectangle{\pgfqpoint{0.625000in}{0.550000in}}{\pgfqpoint{3.875000in}{3.850000in}} %
\pgfusepath{clip}%
\pgfsetbuttcap%
\pgfsetroundjoin%
\pgfsetlinewidth{0.501875pt}%
\definecolor{currentstroke}{rgb}{0.000000,0.000000,0.000000}%
\pgfsetstrokecolor{currentstroke}%
\pgfsetdash{}{0pt}%
\pgfpathmoveto{\pgfqpoint{0.625000in}{3.183065in}}%
\pgfpathlineto{\pgfqpoint{0.634712in}{3.180072in}}%
\pgfpathlineto{\pgfqpoint{0.644424in}{3.177237in}}%
\pgfpathlineto{\pgfqpoint{0.646267in}{3.174561in}}%
\pgfpathlineto{\pgfqpoint{0.644424in}{3.169708in}}%
\pgfpathlineto{\pgfqpoint{0.642107in}{3.164912in}}%
\pgfpathlineto{\pgfqpoint{0.638930in}{3.155263in}}%
\pgfpathlineto{\pgfqpoint{0.634712in}{3.146544in}}%
\pgfpathlineto{\pgfqpoint{0.625000in}{3.154392in}}%
\pgfusepath{stroke}%
\end{pgfscope}%
\begin{pgfscope}%
\pgfpathrectangle{\pgfqpoint{0.625000in}{0.550000in}}{\pgfqpoint{3.875000in}{3.850000in}} %
\pgfusepath{clip}%
\pgfsetbuttcap%
\pgfsetroundjoin%
\pgfsetlinewidth{0.501875pt}%
\definecolor{currentstroke}{rgb}{0.000000,0.000000,0.000000}%
\pgfsetstrokecolor{currentstroke}%
\pgfsetdash{}{0pt}%
\pgfpathmoveto{\pgfqpoint{0.625000in}{3.223498in}}%
\pgfpathlineto{\pgfqpoint{0.633074in}{3.232456in}}%
\pgfpathlineto{\pgfqpoint{0.634712in}{3.235864in}}%
\pgfpathlineto{\pgfqpoint{0.638645in}{3.242105in}}%
\pgfpathlineto{\pgfqpoint{0.644424in}{3.249166in}}%
\pgfpathlineto{\pgfqpoint{0.654135in}{3.246080in}}%
\pgfpathlineto{\pgfqpoint{0.662622in}{3.242105in}}%
\pgfpathlineto{\pgfqpoint{0.663847in}{3.240311in}}%
\pgfpathlineto{\pgfqpoint{0.671543in}{3.232456in}}%
\pgfpathlineto{\pgfqpoint{0.673559in}{3.225055in}}%
\pgfpathlineto{\pgfqpoint{0.674518in}{3.222807in}}%
\pgfpathlineto{\pgfqpoint{0.673857in}{3.213158in}}%
\pgfpathlineto{\pgfqpoint{0.673559in}{3.212428in}}%
\pgfpathlineto{\pgfqpoint{0.669627in}{3.203509in}}%
\pgfpathlineto{\pgfqpoint{0.663847in}{3.196902in}}%
\pgfpathlineto{\pgfqpoint{0.658310in}{3.193860in}}%
\pgfpathlineto{\pgfqpoint{0.654135in}{3.191044in}}%
\pgfpathlineto{\pgfqpoint{0.647645in}{3.193860in}}%
\pgfpathlineto{\pgfqpoint{0.644424in}{3.194282in}}%
\pgfpathlineto{\pgfqpoint{0.634712in}{3.198785in}}%
\pgfpathlineto{\pgfqpoint{0.629755in}{3.193860in}}%
\pgfpathlineto{\pgfqpoint{0.625000in}{3.188093in}}%
\pgfusepath{stroke}%
\end{pgfscope}%
\begin{pgfscope}%
\pgfpathrectangle{\pgfqpoint{0.625000in}{0.550000in}}{\pgfqpoint{3.875000in}{3.850000in}} %
\pgfusepath{clip}%
\pgfsetbuttcap%
\pgfsetroundjoin%
\pgfsetlinewidth{0.501875pt}%
\definecolor{currentstroke}{rgb}{0.000000,0.000000,0.000000}%
\pgfsetstrokecolor{currentstroke}%
\pgfsetdash{}{0pt}%
\pgfpathmoveto{\pgfqpoint{0.625000in}{3.262823in}}%
\pgfpathlineto{\pgfqpoint{0.626501in}{3.261404in}}%
\pgfpathlineto{\pgfqpoint{0.625000in}{3.259996in}}%
\pgfusepath{stroke}%
\end{pgfscope}%
\begin{pgfscope}%
\pgfpathrectangle{\pgfqpoint{0.625000in}{0.550000in}}{\pgfqpoint{3.875000in}{3.850000in}} %
\pgfusepath{clip}%
\pgfsetbuttcap%
\pgfsetroundjoin%
\pgfsetlinewidth{0.501875pt}%
\definecolor{currentstroke}{rgb}{0.000000,0.000000,0.000000}%
\pgfsetstrokecolor{currentstroke}%
\pgfsetdash{}{0pt}%
\pgfpathmoveto{\pgfqpoint{0.625000in}{3.291727in}}%
\pgfpathlineto{\pgfqpoint{0.626475in}{3.290351in}}%
\pgfpathlineto{\pgfqpoint{0.625000in}{3.288490in}}%
\pgfusepath{stroke}%
\end{pgfscope}%
\begin{pgfscope}%
\pgfpathrectangle{\pgfqpoint{0.625000in}{0.550000in}}{\pgfqpoint{3.875000in}{3.850000in}} %
\pgfusepath{clip}%
\pgfsetbuttcap%
\pgfsetroundjoin%
\pgfsetlinewidth{0.501875pt}%
\definecolor{currentstroke}{rgb}{0.000000,0.000000,0.000000}%
\pgfsetstrokecolor{currentstroke}%
\pgfsetdash{}{0pt}%
\pgfpathmoveto{\pgfqpoint{0.625000in}{3.340473in}}%
\pgfpathlineto{\pgfqpoint{0.628565in}{3.338596in}}%
\pgfpathlineto{\pgfqpoint{0.634712in}{3.331307in}}%
\pgfpathlineto{\pgfqpoint{0.638705in}{3.328947in}}%
\pgfpathlineto{\pgfqpoint{0.634712in}{3.323115in}}%
\pgfpathlineto{\pgfqpoint{0.632168in}{3.319298in}}%
\pgfpathlineto{\pgfqpoint{0.627951in}{3.309649in}}%
\pgfpathlineto{\pgfqpoint{0.625000in}{3.308273in}}%
\pgfusepath{stroke}%
\end{pgfscope}%
\begin{pgfscope}%
\pgfpathrectangle{\pgfqpoint{0.625000in}{0.550000in}}{\pgfqpoint{3.875000in}{3.850000in}} %
\pgfusepath{clip}%
\pgfsetbuttcap%
\pgfsetroundjoin%
\pgfsetlinewidth{0.501875pt}%
\definecolor{currentstroke}{rgb}{0.000000,0.000000,0.000000}%
\pgfsetstrokecolor{currentstroke}%
\pgfsetdash{}{0pt}%
\pgfpathmoveto{\pgfqpoint{0.625000in}{3.381010in}}%
\pgfpathlineto{\pgfqpoint{0.630809in}{3.377193in}}%
\pgfpathlineto{\pgfqpoint{0.628452in}{3.367544in}}%
\pgfpathlineto{\pgfqpoint{0.629076in}{3.357895in}}%
\pgfpathlineto{\pgfqpoint{0.625000in}{3.355752in}}%
\pgfusepath{stroke}%
\end{pgfscope}%
\begin{pgfscope}%
\pgfpathrectangle{\pgfqpoint{0.625000in}{0.550000in}}{\pgfqpoint{3.875000in}{3.850000in}} %
\pgfusepath{clip}%
\pgfsetbuttcap%
\pgfsetroundjoin%
\pgfsetlinewidth{0.501875pt}%
\definecolor{currentstroke}{rgb}{0.000000,0.000000,0.000000}%
\pgfsetstrokecolor{currentstroke}%
\pgfsetdash{}{0pt}%
\pgfpathmoveto{\pgfqpoint{0.625000in}{3.507776in}}%
\pgfpathlineto{\pgfqpoint{0.626675in}{3.512281in}}%
\pgfpathlineto{\pgfqpoint{0.625000in}{3.516785in}}%
\pgfusepath{stroke}%
\end{pgfscope}%
\begin{pgfscope}%
\pgfpathrectangle{\pgfqpoint{0.625000in}{0.550000in}}{\pgfqpoint{3.875000in}{3.850000in}} %
\pgfusepath{clip}%
\pgfsetbuttcap%
\pgfsetroundjoin%
\pgfsetlinewidth{0.501875pt}%
\definecolor{currentstroke}{rgb}{0.000000,0.000000,0.000000}%
\pgfsetstrokecolor{currentstroke}%
\pgfsetdash{}{0pt}%
\pgfpathmoveto{\pgfqpoint{0.625000in}{3.522757in}}%
\pgfpathlineto{\pgfqpoint{0.631769in}{3.531579in}}%
\pgfpathlineto{\pgfqpoint{0.625000in}{3.540401in}}%
\pgfusepath{stroke}%
\end{pgfscope}%
\begin{pgfscope}%
\pgfpathrectangle{\pgfqpoint{0.625000in}{0.550000in}}{\pgfqpoint{3.875000in}{3.850000in}} %
\pgfusepath{clip}%
\pgfsetbuttcap%
\pgfsetroundjoin%
\pgfsetlinewidth{0.501875pt}%
\definecolor{currentstroke}{rgb}{0.000000,0.000000,0.000000}%
\pgfsetstrokecolor{currentstroke}%
\pgfsetdash{}{0pt}%
\pgfpathmoveto{\pgfqpoint{0.625000in}{3.546818in}}%
\pgfpathlineto{\pgfqpoint{0.626295in}{3.550877in}}%
\pgfpathlineto{\pgfqpoint{0.625000in}{3.553316in}}%
\pgfusepath{stroke}%
\end{pgfscope}%
\begin{pgfscope}%
\pgfpathrectangle{\pgfqpoint{0.625000in}{0.550000in}}{\pgfqpoint{3.875000in}{3.850000in}} %
\pgfusepath{clip}%
\pgfsetbuttcap%
\pgfsetroundjoin%
\pgfsetlinewidth{0.501875pt}%
\definecolor{currentstroke}{rgb}{0.000000,0.000000,0.000000}%
\pgfsetstrokecolor{currentstroke}%
\pgfsetdash{}{0pt}%
\pgfpathmoveto{\pgfqpoint{0.625000in}{3.569070in}}%
\pgfpathlineto{\pgfqpoint{0.628267in}{3.570175in}}%
\pgfpathlineto{\pgfqpoint{0.632127in}{3.579825in}}%
\pgfpathlineto{\pgfqpoint{0.634712in}{3.584903in}}%
\pgfpathlineto{\pgfqpoint{0.639722in}{3.589474in}}%
\pgfpathlineto{\pgfqpoint{0.637319in}{3.599123in}}%
\pgfpathlineto{\pgfqpoint{0.634712in}{3.601505in}}%
\pgfpathlineto{\pgfqpoint{0.631084in}{3.599123in}}%
\pgfpathlineto{\pgfqpoint{0.625000in}{3.598124in}}%
\pgfusepath{stroke}%
\end{pgfscope}%
\begin{pgfscope}%
\pgfpathrectangle{\pgfqpoint{0.625000in}{0.550000in}}{\pgfqpoint{3.875000in}{3.850000in}} %
\pgfusepath{clip}%
\pgfsetbuttcap%
\pgfsetroundjoin%
\pgfsetlinewidth{0.501875pt}%
\definecolor{currentstroke}{rgb}{0.000000,0.000000,0.000000}%
\pgfsetstrokecolor{currentstroke}%
\pgfsetdash{}{0pt}%
\pgfpathmoveto{\pgfqpoint{0.625000in}{3.620863in}}%
\pgfpathlineto{\pgfqpoint{0.629816in}{3.628070in}}%
\pgfpathlineto{\pgfqpoint{0.625000in}{3.631651in}}%
\pgfusepath{stroke}%
\end{pgfscope}%
\begin{pgfscope}%
\pgfpathrectangle{\pgfqpoint{0.625000in}{0.550000in}}{\pgfqpoint{3.875000in}{3.850000in}} %
\pgfusepath{clip}%
\pgfsetbuttcap%
\pgfsetroundjoin%
\pgfsetlinewidth{0.501875pt}%
\definecolor{currentstroke}{rgb}{0.000000,0.000000,0.000000}%
\pgfsetstrokecolor{currentstroke}%
\pgfsetdash{}{0pt}%
\pgfpathmoveto{\pgfqpoint{0.625000in}{3.660358in}}%
\pgfpathlineto{\pgfqpoint{0.629302in}{3.666667in}}%
\pgfpathlineto{\pgfqpoint{0.625000in}{3.672975in}}%
\pgfusepath{stroke}%
\end{pgfscope}%
\begin{pgfscope}%
\pgfpathrectangle{\pgfqpoint{0.625000in}{0.550000in}}{\pgfqpoint{3.875000in}{3.850000in}} %
\pgfusepath{clip}%
\pgfsetbuttcap%
\pgfsetroundjoin%
\pgfsetlinewidth{0.501875pt}%
\definecolor{currentstroke}{rgb}{0.000000,0.000000,0.000000}%
\pgfsetstrokecolor{currentstroke}%
\pgfsetdash{}{0pt}%
\pgfpathmoveto{\pgfqpoint{0.625000in}{3.687448in}}%
\pgfpathlineto{\pgfqpoint{0.634712in}{3.690892in}}%
\pgfpathlineto{\pgfqpoint{0.644424in}{3.693357in}}%
\pgfpathlineto{\pgfqpoint{0.649509in}{3.695614in}}%
\pgfpathlineto{\pgfqpoint{0.654135in}{3.697962in}}%
\pgfpathlineto{\pgfqpoint{0.662123in}{3.705263in}}%
\pgfpathlineto{\pgfqpoint{0.663847in}{3.707559in}}%
\pgfpathlineto{\pgfqpoint{0.667936in}{3.714912in}}%
\pgfpathlineto{\pgfqpoint{0.670215in}{3.724561in}}%
\pgfpathlineto{\pgfqpoint{0.669761in}{3.734211in}}%
\pgfpathlineto{\pgfqpoint{0.666289in}{3.743860in}}%
\pgfpathlineto{\pgfqpoint{0.663847in}{3.746430in}}%
\pgfpathlineto{\pgfqpoint{0.659521in}{3.753509in}}%
\pgfpathlineto{\pgfqpoint{0.654135in}{3.756648in}}%
\pgfpathlineto{\pgfqpoint{0.644424in}{3.762422in}}%
\pgfpathlineto{\pgfqpoint{0.642605in}{3.763158in}}%
\pgfpathlineto{\pgfqpoint{0.644424in}{3.763926in}}%
\pgfpathlineto{\pgfqpoint{0.654135in}{3.768766in}}%
\pgfpathlineto{\pgfqpoint{0.663847in}{3.771103in}}%
\pgfpathlineto{\pgfqpoint{0.670772in}{3.772807in}}%
\pgfpathlineto{\pgfqpoint{0.673559in}{3.774115in}}%
\pgfpathlineto{\pgfqpoint{0.683271in}{3.778150in}}%
\pgfpathlineto{\pgfqpoint{0.692226in}{3.782456in}}%
\pgfpathlineto{\pgfqpoint{0.692982in}{3.783086in}}%
\pgfpathlineto{\pgfqpoint{0.702694in}{3.789975in}}%
\pgfpathlineto{\pgfqpoint{0.705843in}{3.792105in}}%
\pgfpathlineto{\pgfqpoint{0.712406in}{3.799244in}}%
\pgfpathlineto{\pgfqpoint{0.715044in}{3.801754in}}%
\pgfpathlineto{\pgfqpoint{0.721397in}{3.811404in}}%
\pgfpathlineto{\pgfqpoint{0.722118in}{3.813230in}}%
\pgfpathlineto{\pgfqpoint{0.725955in}{3.821053in}}%
\pgfpathlineto{\pgfqpoint{0.728539in}{3.830702in}}%
\pgfpathlineto{\pgfqpoint{0.729480in}{3.840351in}}%
\pgfpathlineto{\pgfqpoint{0.728903in}{3.850000in}}%
\pgfpathlineto{\pgfqpoint{0.726699in}{3.859649in}}%
\pgfpathlineto{\pgfqpoint{0.722557in}{3.869298in}}%
\pgfpathlineto{\pgfqpoint{0.722118in}{3.870035in}}%
\pgfpathlineto{\pgfqpoint{0.716387in}{3.878947in}}%
\pgfpathlineto{\pgfqpoint{0.712406in}{3.883415in}}%
\pgfpathlineto{\pgfqpoint{0.707109in}{3.888596in}}%
\pgfpathlineto{\pgfqpoint{0.702694in}{3.892047in}}%
\pgfpathlineto{\pgfqpoint{0.693095in}{3.898246in}}%
\pgfpathlineto{\pgfqpoint{0.692982in}{3.898307in}}%
\pgfpathlineto{\pgfqpoint{0.683271in}{3.903112in}}%
\pgfpathlineto{\pgfqpoint{0.673559in}{3.907291in}}%
\pgfpathlineto{\pgfqpoint{0.672319in}{3.907895in}}%
\pgfpathlineto{\pgfqpoint{0.663847in}{3.911071in}}%
\pgfpathlineto{\pgfqpoint{0.654135in}{3.914971in}}%
\pgfpathlineto{\pgfqpoint{0.649280in}{3.917544in}}%
\pgfpathlineto{\pgfqpoint{0.644424in}{3.919020in}}%
\pgfpathlineto{\pgfqpoint{0.634712in}{3.922368in}}%
\pgfpathlineto{\pgfqpoint{0.632090in}{3.917544in}}%
\pgfpathlineto{\pgfqpoint{0.628441in}{3.907895in}}%
\pgfpathlineto{\pgfqpoint{0.625000in}{3.905746in}}%
\pgfusepath{stroke}%
\end{pgfscope}%
\begin{pgfscope}%
\pgfpathrectangle{\pgfqpoint{0.625000in}{0.550000in}}{\pgfqpoint{3.875000in}{3.850000in}} %
\pgfusepath{clip}%
\pgfsetbuttcap%
\pgfsetroundjoin%
\pgfsetlinewidth{0.501875pt}%
\definecolor{currentstroke}{rgb}{0.000000,0.000000,0.000000}%
\pgfsetstrokecolor{currentstroke}%
\pgfsetdash{}{0pt}%
\pgfpathmoveto{\pgfqpoint{0.625000in}{3.707490in}}%
\pgfpathlineto{\pgfqpoint{0.628946in}{3.705263in}}%
\pgfpathlineto{\pgfqpoint{0.625000in}{3.703520in}}%
\pgfusepath{stroke}%
\end{pgfscope}%
\begin{pgfscope}%
\pgfpathrectangle{\pgfqpoint{0.625000in}{0.550000in}}{\pgfqpoint{3.875000in}{3.850000in}} %
\pgfusepath{clip}%
\pgfsetbuttcap%
\pgfsetroundjoin%
\pgfsetlinewidth{0.501875pt}%
\definecolor{currentstroke}{rgb}{0.000000,0.000000,0.000000}%
\pgfsetstrokecolor{currentstroke}%
\pgfsetdash{}{0pt}%
\pgfpathmoveto{\pgfqpoint{0.625000in}{3.774000in}}%
\pgfpathlineto{\pgfqpoint{0.626568in}{3.772807in}}%
\pgfpathlineto{\pgfqpoint{0.627769in}{3.763158in}}%
\pgfpathlineto{\pgfqpoint{0.630927in}{3.753509in}}%
\pgfpathlineto{\pgfqpoint{0.633020in}{3.743860in}}%
\pgfpathlineto{\pgfqpoint{0.625000in}{3.741599in}}%
\pgfusepath{stroke}%
\end{pgfscope}%
\begin{pgfscope}%
\pgfpathrectangle{\pgfqpoint{0.625000in}{0.550000in}}{\pgfqpoint{3.875000in}{3.850000in}} %
\pgfusepath{clip}%
\pgfsetbuttcap%
\pgfsetroundjoin%
\pgfsetlinewidth{0.501875pt}%
\definecolor{currentstroke}{rgb}{0.000000,0.000000,0.000000}%
\pgfsetstrokecolor{currentstroke}%
\pgfsetdash{}{0pt}%
\pgfpathmoveto{\pgfqpoint{0.625000in}{3.797511in}}%
\pgfpathlineto{\pgfqpoint{0.634712in}{3.794630in}}%
\pgfpathlineto{\pgfqpoint{0.637157in}{3.792105in}}%
\pgfpathlineto{\pgfqpoint{0.634712in}{3.787820in}}%
\pgfpathlineto{\pgfqpoint{0.625000in}{3.788113in}}%
\pgfusepath{stroke}%
\end{pgfscope}%
\begin{pgfscope}%
\pgfpathrectangle{\pgfqpoint{0.625000in}{0.550000in}}{\pgfqpoint{3.875000in}{3.850000in}} %
\pgfusepath{clip}%
\pgfsetbuttcap%
\pgfsetroundjoin%
\pgfsetlinewidth{0.501875pt}%
\definecolor{currentstroke}{rgb}{0.000000,0.000000,0.000000}%
\pgfsetstrokecolor{currentstroke}%
\pgfsetdash{}{0pt}%
\pgfpathmoveto{\pgfqpoint{0.625000in}{3.821815in}}%
\pgfpathlineto{\pgfqpoint{0.625586in}{3.821053in}}%
\pgfpathlineto{\pgfqpoint{0.625918in}{3.811404in}}%
\pgfpathlineto{\pgfqpoint{0.625000in}{3.809444in}}%
\pgfusepath{stroke}%
\end{pgfscope}%
\begin{pgfscope}%
\pgfpathrectangle{\pgfqpoint{0.625000in}{0.550000in}}{\pgfqpoint{3.875000in}{3.850000in}} %
\pgfusepath{clip}%
\pgfsetbuttcap%
\pgfsetroundjoin%
\pgfsetlinewidth{0.501875pt}%
\definecolor{currentstroke}{rgb}{0.000000,0.000000,0.000000}%
\pgfsetstrokecolor{currentstroke}%
\pgfsetdash{}{0pt}%
\pgfpathmoveto{\pgfqpoint{0.625000in}{3.842820in}}%
\pgfpathlineto{\pgfqpoint{0.628018in}{3.840351in}}%
\pgfpathlineto{\pgfqpoint{0.625000in}{3.837882in}}%
\pgfusepath{stroke}%
\end{pgfscope}%
\begin{pgfscope}%
\pgfpathrectangle{\pgfqpoint{0.625000in}{0.550000in}}{\pgfqpoint{3.875000in}{3.850000in}} %
\pgfusepath{clip}%
\pgfsetbuttcap%
\pgfsetroundjoin%
\pgfsetlinewidth{0.501875pt}%
\definecolor{currentstroke}{rgb}{0.000000,0.000000,0.000000}%
\pgfsetstrokecolor{currentstroke}%
\pgfsetdash{}{0pt}%
\pgfpathmoveto{\pgfqpoint{0.625000in}{3.861096in}}%
\pgfpathlineto{\pgfqpoint{0.627951in}{3.859649in}}%
\pgfpathlineto{\pgfqpoint{0.625000in}{3.858286in}}%
\pgfusepath{stroke}%
\end{pgfscope}%
\begin{pgfscope}%
\pgfpathrectangle{\pgfqpoint{0.625000in}{0.550000in}}{\pgfqpoint{3.875000in}{3.850000in}} %
\pgfusepath{clip}%
\pgfsetbuttcap%
\pgfsetroundjoin%
\pgfsetlinewidth{0.501875pt}%
\definecolor{currentstroke}{rgb}{0.000000,0.000000,0.000000}%
\pgfsetstrokecolor{currentstroke}%
\pgfsetdash{}{0pt}%
\pgfpathmoveto{\pgfqpoint{0.625000in}{3.881239in}}%
\pgfpathlineto{\pgfqpoint{0.634121in}{3.878947in}}%
\pgfpathlineto{\pgfqpoint{0.625000in}{3.876532in}}%
\pgfusepath{stroke}%
\end{pgfscope}%
\begin{pgfscope}%
\pgfpathrectangle{\pgfqpoint{0.625000in}{0.550000in}}{\pgfqpoint{3.875000in}{3.850000in}} %
\pgfusepath{clip}%
\pgfsetbuttcap%
\pgfsetroundjoin%
\pgfsetlinewidth{0.501875pt}%
\definecolor{currentstroke}{rgb}{0.000000,0.000000,0.000000}%
\pgfsetstrokecolor{currentstroke}%
\pgfsetdash{}{0pt}%
\pgfpathmoveto{\pgfqpoint{0.625000in}{3.998500in}}%
\pgfpathlineto{\pgfqpoint{0.634712in}{3.999588in}}%
\pgfpathlineto{\pgfqpoint{0.639056in}{4.004386in}}%
\pgfpathlineto{\pgfqpoint{0.641127in}{4.014035in}}%
\pgfpathlineto{\pgfqpoint{0.634712in}{4.021399in}}%
\pgfpathlineto{\pgfqpoint{0.630297in}{4.023684in}}%
\pgfpathlineto{\pgfqpoint{0.634712in}{4.025181in}}%
\pgfpathlineto{\pgfqpoint{0.644424in}{4.027012in}}%
\pgfpathlineto{\pgfqpoint{0.654135in}{4.030980in}}%
\pgfpathlineto{\pgfqpoint{0.659161in}{4.033333in}}%
\pgfpathlineto{\pgfqpoint{0.663847in}{4.040202in}}%
\pgfpathlineto{\pgfqpoint{0.666574in}{4.042982in}}%
\pgfpathlineto{\pgfqpoint{0.669006in}{4.052632in}}%
\pgfpathlineto{\pgfqpoint{0.666977in}{4.062281in}}%
\pgfpathlineto{\pgfqpoint{0.663847in}{4.066802in}}%
\pgfpathlineto{\pgfqpoint{0.659553in}{4.071930in}}%
\pgfpathlineto{\pgfqpoint{0.654135in}{4.075419in}}%
\pgfpathlineto{\pgfqpoint{0.644424in}{4.079850in}}%
\pgfpathlineto{\pgfqpoint{0.641835in}{4.081579in}}%
\pgfpathlineto{\pgfqpoint{0.634712in}{4.083650in}}%
\pgfpathlineto{\pgfqpoint{0.631486in}{4.081579in}}%
\pgfpathlineto{\pgfqpoint{0.625000in}{4.080025in}}%
\pgfusepath{stroke}%
\end{pgfscope}%
\begin{pgfscope}%
\pgfpathrectangle{\pgfqpoint{0.625000in}{0.550000in}}{\pgfqpoint{3.875000in}{3.850000in}} %
\pgfusepath{clip}%
\pgfsetbuttcap%
\pgfsetroundjoin%
\pgfsetlinewidth{0.501875pt}%
\definecolor{currentstroke}{rgb}{0.000000,0.000000,0.000000}%
\pgfsetstrokecolor{currentstroke}%
\pgfsetdash{}{0pt}%
\pgfpathmoveto{\pgfqpoint{0.625000in}{4.037302in}}%
\pgfpathlineto{\pgfqpoint{0.628234in}{4.033333in}}%
\pgfpathlineto{\pgfqpoint{0.625000in}{4.026640in}}%
\pgfusepath{stroke}%
\end{pgfscope}%
\begin{pgfscope}%
\pgfpathrectangle{\pgfqpoint{0.625000in}{0.550000in}}{\pgfqpoint{3.875000in}{3.850000in}} %
\pgfusepath{clip}%
\pgfsetbuttcap%
\pgfsetroundjoin%
\pgfsetlinewidth{0.501875pt}%
\definecolor{currentstroke}{rgb}{0.000000,0.000000,0.000000}%
\pgfsetstrokecolor{currentstroke}%
\pgfsetdash{}{0pt}%
\pgfpathmoveto{\pgfqpoint{0.625000in}{4.063835in}}%
\pgfpathlineto{\pgfqpoint{0.628126in}{4.062281in}}%
\pgfpathlineto{\pgfqpoint{0.626591in}{4.052632in}}%
\pgfpathlineto{\pgfqpoint{0.625000in}{4.051520in}}%
\pgfusepath{stroke}%
\end{pgfscope}%
\begin{pgfscope}%
\pgfpathrectangle{\pgfqpoint{0.625000in}{0.550000in}}{\pgfqpoint{3.875000in}{3.850000in}} %
\pgfusepath{clip}%
\pgfsetbuttcap%
\pgfsetroundjoin%
\pgfsetlinewidth{0.501875pt}%
\definecolor{currentstroke}{rgb}{0.000000,0.000000,0.000000}%
\pgfsetstrokecolor{currentstroke}%
\pgfsetdash{}{0pt}%
\pgfpathmoveto{\pgfqpoint{0.625000in}{4.141029in}}%
\pgfpathlineto{\pgfqpoint{0.626890in}{4.139474in}}%
\pgfpathlineto{\pgfqpoint{0.634712in}{4.134278in}}%
\pgfpathlineto{\pgfqpoint{0.644142in}{4.139474in}}%
\pgfpathlineto{\pgfqpoint{0.644424in}{4.140929in}}%
\pgfpathlineto{\pgfqpoint{0.649808in}{4.149123in}}%
\pgfpathlineto{\pgfqpoint{0.644887in}{4.158772in}}%
\pgfpathlineto{\pgfqpoint{0.644424in}{4.159087in}}%
\pgfpathlineto{\pgfqpoint{0.634712in}{4.163741in}}%
\pgfpathlineto{\pgfqpoint{0.625000in}{4.167106in}}%
\pgfusepath{stroke}%
\end{pgfscope}%
\begin{pgfscope}%
\pgfpathrectangle{\pgfqpoint{0.625000in}{0.550000in}}{\pgfqpoint{3.875000in}{3.850000in}} %
\pgfusepath{clip}%
\pgfsetbuttcap%
\pgfsetroundjoin%
\pgfsetlinewidth{0.501875pt}%
\definecolor{currentstroke}{rgb}{0.000000,0.000000,0.000000}%
\pgfsetstrokecolor{currentstroke}%
\pgfsetdash{}{0pt}%
\pgfpathmoveto{\pgfqpoint{0.625000in}{4.170327in}}%
\pgfpathlineto{\pgfqpoint{0.630765in}{4.178070in}}%
\pgfpathlineto{\pgfqpoint{0.629435in}{4.187719in}}%
\pgfpathlineto{\pgfqpoint{0.634712in}{4.197181in}}%
\pgfpathlineto{\pgfqpoint{0.635400in}{4.197368in}}%
\pgfpathlineto{\pgfqpoint{0.640957in}{4.207018in}}%
\pgfpathlineto{\pgfqpoint{0.634712in}{4.212161in}}%
\pgfpathlineto{\pgfqpoint{0.629013in}{4.207018in}}%
\pgfpathlineto{\pgfqpoint{0.625000in}{4.206019in}}%
\pgfusepath{stroke}%
\end{pgfscope}%
\begin{pgfscope}%
\pgfpathrectangle{\pgfqpoint{0.625000in}{0.550000in}}{\pgfqpoint{3.875000in}{3.850000in}} %
\pgfusepath{clip}%
\pgfsetbuttcap%
\pgfsetroundjoin%
\pgfsetlinewidth{0.501875pt}%
\definecolor{currentstroke}{rgb}{0.000000,0.000000,0.000000}%
\pgfsetstrokecolor{currentstroke}%
\pgfsetdash{}{0pt}%
\pgfpathmoveto{\pgfqpoint{0.625000in}{4.244875in}}%
\pgfpathlineto{\pgfqpoint{0.626735in}{4.245614in}}%
\pgfpathlineto{\pgfqpoint{0.625000in}{4.245771in}}%
\pgfusepath{stroke}%
\end{pgfscope}%
\begin{pgfscope}%
\pgfpathrectangle{\pgfqpoint{0.625000in}{0.550000in}}{\pgfqpoint{3.875000in}{3.850000in}} %
\pgfusepath{clip}%
\pgfsetbuttcap%
\pgfsetroundjoin%
\pgfsetlinewidth{0.501875pt}%
\definecolor{currentstroke}{rgb}{0.000000,0.000000,0.000000}%
\pgfsetstrokecolor{currentstroke}%
\pgfsetdash{}{0pt}%
\pgfpathmoveto{\pgfqpoint{0.625000in}{4.259957in}}%
\pgfpathlineto{\pgfqpoint{0.633647in}{4.264912in}}%
\pgfpathlineto{\pgfqpoint{0.625000in}{4.273024in}}%
\pgfusepath{stroke}%
\end{pgfscope}%
\begin{pgfscope}%
\pgfpathrectangle{\pgfqpoint{0.625000in}{0.550000in}}{\pgfqpoint{3.875000in}{3.850000in}} %
\pgfusepath{clip}%
\pgfsetbuttcap%
\pgfsetroundjoin%
\pgfsetlinewidth{0.501875pt}%
\definecolor{currentstroke}{rgb}{0.000000,0.000000,0.000000}%
\pgfsetstrokecolor{currentstroke}%
\pgfsetdash{}{0pt}%
\pgfpathmoveto{\pgfqpoint{0.625000in}{4.276330in}}%
\pgfpathlineto{\pgfqpoint{0.632950in}{4.284211in}}%
\pgfpathlineto{\pgfqpoint{0.625000in}{4.292091in}}%
\pgfusepath{stroke}%
\end{pgfscope}%
\begin{pgfscope}%
\pgfpathrectangle{\pgfqpoint{0.625000in}{0.550000in}}{\pgfqpoint{3.875000in}{3.850000in}} %
\pgfusepath{clip}%
\pgfsetbuttcap%
\pgfsetroundjoin%
\pgfsetlinewidth{0.501875pt}%
\definecolor{currentstroke}{rgb}{0.000000,0.000000,0.000000}%
\pgfsetstrokecolor{currentstroke}%
\pgfsetdash{}{0pt}%
\pgfpathmoveto{\pgfqpoint{0.625000in}{4.344704in}}%
\pgfpathlineto{\pgfqpoint{0.633073in}{4.351754in}}%
\pgfpathlineto{\pgfqpoint{0.625000in}{4.359997in}}%
\pgfusepath{stroke}%
\end{pgfscope}%
\begin{pgfscope}%
\pgfpathrectangle{\pgfqpoint{0.625000in}{0.550000in}}{\pgfqpoint{3.875000in}{3.850000in}} %
\pgfusepath{clip}%
\pgfsetbuttcap%
\pgfsetroundjoin%
\pgfsetlinewidth{0.501875pt}%
\definecolor{currentstroke}{rgb}{0.000000,0.000000,0.000000}%
\pgfsetstrokecolor{currentstroke}%
\pgfsetdash{}{0pt}%
\pgfpathmoveto{\pgfqpoint{0.654135in}{2.409511in}}%
\pgfpathlineto{\pgfqpoint{0.647788in}{2.412281in}}%
\pgfpathlineto{\pgfqpoint{0.654135in}{2.416870in}}%
\pgfpathlineto{\pgfqpoint{0.656617in}{2.412281in}}%
\pgfpathlineto{\pgfqpoint{0.654135in}{2.409511in}}%
\pgfusepath{stroke}%
\end{pgfscope}%
\begin{pgfscope}%
\pgfpathrectangle{\pgfqpoint{0.625000in}{0.550000in}}{\pgfqpoint{3.875000in}{3.850000in}} %
\pgfusepath{clip}%
\pgfsetbuttcap%
\pgfsetroundjoin%
\pgfsetlinewidth{0.501875pt}%
\definecolor{currentstroke}{rgb}{0.000000,0.000000,0.000000}%
\pgfsetstrokecolor{currentstroke}%
\pgfsetdash{}{0pt}%
\pgfpathmoveto{\pgfqpoint{0.663847in}{2.424036in}}%
\pgfpathlineto{\pgfqpoint{0.660456in}{2.431579in}}%
\pgfpathlineto{\pgfqpoint{0.654135in}{2.438830in}}%
\pgfpathlineto{\pgfqpoint{0.644424in}{2.436072in}}%
\pgfpathlineto{\pgfqpoint{0.639523in}{2.441228in}}%
\pgfpathlineto{\pgfqpoint{0.644424in}{2.444205in}}%
\pgfpathlineto{\pgfqpoint{0.651665in}{2.450877in}}%
\pgfpathlineto{\pgfqpoint{0.644424in}{2.458123in}}%
\pgfpathlineto{\pgfqpoint{0.634712in}{2.458568in}}%
\pgfpathlineto{\pgfqpoint{0.632562in}{2.460526in}}%
\pgfpathlineto{\pgfqpoint{0.632002in}{2.470175in}}%
\pgfpathlineto{\pgfqpoint{0.634712in}{2.475738in}}%
\pgfpathlineto{\pgfqpoint{0.644424in}{2.475298in}}%
\pgfpathlineto{\pgfqpoint{0.654135in}{2.474718in}}%
\pgfpathlineto{\pgfqpoint{0.657419in}{2.470175in}}%
\pgfpathlineto{\pgfqpoint{0.655295in}{2.460526in}}%
\pgfpathlineto{\pgfqpoint{0.663847in}{2.453340in}}%
\pgfpathlineto{\pgfqpoint{0.672983in}{2.450877in}}%
\pgfpathlineto{\pgfqpoint{0.666667in}{2.441228in}}%
\pgfpathlineto{\pgfqpoint{0.666977in}{2.431579in}}%
\pgfpathlineto{\pgfqpoint{0.663847in}{2.424036in}}%
\pgfusepath{stroke}%
\end{pgfscope}%
\begin{pgfscope}%
\pgfpathrectangle{\pgfqpoint{0.625000in}{0.550000in}}{\pgfqpoint{3.875000in}{3.850000in}} %
\pgfusepath{clip}%
\pgfsetbuttcap%
\pgfsetroundjoin%
\pgfsetlinewidth{0.501875pt}%
\definecolor{currentstroke}{rgb}{0.000000,0.000000,0.000000}%
\pgfsetstrokecolor{currentstroke}%
\pgfsetdash{}{0pt}%
\pgfpathmoveto{\pgfqpoint{0.634712in}{2.497142in}}%
\pgfpathlineto{\pgfqpoint{0.632215in}{2.499123in}}%
\pgfpathlineto{\pgfqpoint{0.634712in}{2.501405in}}%
\pgfpathlineto{\pgfqpoint{0.644001in}{2.499123in}}%
\pgfpathlineto{\pgfqpoint{0.634712in}{2.497142in}}%
\pgfusepath{stroke}%
\end{pgfscope}%
\begin{pgfscope}%
\pgfpathrectangle{\pgfqpoint{0.625000in}{0.550000in}}{\pgfqpoint{3.875000in}{3.850000in}} %
\pgfusepath{clip}%
\pgfsetbuttcap%
\pgfsetroundjoin%
\pgfsetlinewidth{0.501875pt}%
\definecolor{currentstroke}{rgb}{0.000000,0.000000,0.000000}%
\pgfsetstrokecolor{currentstroke}%
\pgfsetdash{}{0pt}%
\pgfpathmoveto{\pgfqpoint{0.654135in}{2.496923in}}%
\pgfpathlineto{\pgfqpoint{0.644786in}{2.499123in}}%
\pgfpathlineto{\pgfqpoint{0.651666in}{2.508772in}}%
\pgfpathlineto{\pgfqpoint{0.644424in}{2.515444in}}%
\pgfpathlineto{\pgfqpoint{0.639523in}{2.518421in}}%
\pgfpathlineto{\pgfqpoint{0.644424in}{2.523577in}}%
\pgfpathlineto{\pgfqpoint{0.654135in}{2.520820in}}%
\pgfpathlineto{\pgfqpoint{0.660456in}{2.528070in}}%
\pgfpathlineto{\pgfqpoint{0.663847in}{2.535613in}}%
\pgfpathlineto{\pgfqpoint{0.666977in}{2.528070in}}%
\pgfpathlineto{\pgfqpoint{0.666667in}{2.518421in}}%
\pgfpathlineto{\pgfqpoint{0.672983in}{2.508772in}}%
\pgfpathlineto{\pgfqpoint{0.663847in}{2.506301in}}%
\pgfpathlineto{\pgfqpoint{0.656326in}{2.499123in}}%
\pgfpathlineto{\pgfqpoint{0.654135in}{2.496923in}}%
\pgfusepath{stroke}%
\end{pgfscope}%
\begin{pgfscope}%
\pgfpathrectangle{\pgfqpoint{0.625000in}{0.550000in}}{\pgfqpoint{3.875000in}{3.850000in}} %
\pgfusepath{clip}%
\pgfsetbuttcap%
\pgfsetroundjoin%
\pgfsetlinewidth{0.501875pt}%
\definecolor{currentstroke}{rgb}{0.000000,0.000000,0.000000}%
\pgfsetstrokecolor{currentstroke}%
\pgfsetdash{}{0pt}%
\pgfpathmoveto{\pgfqpoint{0.654135in}{2.542779in}}%
\pgfpathlineto{\pgfqpoint{0.647788in}{2.547368in}}%
\pgfpathlineto{\pgfqpoint{0.654135in}{2.550138in}}%
\pgfpathlineto{\pgfqpoint{0.656617in}{2.547368in}}%
\pgfpathlineto{\pgfqpoint{0.654135in}{2.542779in}}%
\pgfusepath{stroke}%
\end{pgfscope}%
\begin{pgfscope}%
\pgfpathrectangle{\pgfqpoint{0.625000in}{0.550000in}}{\pgfqpoint{3.875000in}{3.850000in}} %
\pgfusepath{clip}%
\pgfsetbuttcap%
\pgfsetroundjoin%
\pgfsetlinewidth{0.501875pt}%
\definecolor{currentstroke}{rgb}{0.000000,0.000000,0.000000}%
\pgfsetstrokecolor{currentstroke}%
\pgfsetdash{}{0pt}%
\pgfpathmoveto{\pgfqpoint{0.625000in}{0.602007in}}%
\pgfpathlineto{\pgfqpoint{0.630767in}{0.607895in}}%
\pgfpathlineto{\pgfqpoint{0.625000in}{0.612931in}}%
\pgfusepath{stroke}%
\end{pgfscope}%
\begin{pgfscope}%
\pgfpathrectangle{\pgfqpoint{0.625000in}{0.550000in}}{\pgfqpoint{3.875000in}{3.850000in}} %
\pgfusepath{clip}%
\pgfsetbuttcap%
\pgfsetroundjoin%
\pgfsetlinewidth{0.501875pt}%
\definecolor{currentstroke}{rgb}{0.000000,0.000000,0.000000}%
\pgfsetstrokecolor{currentstroke}%
\pgfsetdash{}{0pt}%
\pgfpathmoveto{\pgfqpoint{0.625000in}{0.621025in}}%
\pgfpathlineto{\pgfqpoint{0.631894in}{0.627193in}}%
\pgfpathlineto{\pgfqpoint{0.625000in}{0.633264in}}%
\pgfusepath{stroke}%
\end{pgfscope}%
\begin{pgfscope}%
\pgfpathrectangle{\pgfqpoint{0.625000in}{0.550000in}}{\pgfqpoint{3.875000in}{3.850000in}} %
\pgfusepath{clip}%
\pgfsetbuttcap%
\pgfsetroundjoin%
\pgfsetlinewidth{0.501875pt}%
\definecolor{currentstroke}{rgb}{0.000000,0.000000,0.000000}%
\pgfsetstrokecolor{currentstroke}%
\pgfsetdash{}{0pt}%
\pgfpathmoveto{\pgfqpoint{0.625000in}{0.671094in}}%
\pgfpathlineto{\pgfqpoint{0.629383in}{0.675439in}}%
\pgfpathlineto{\pgfqpoint{0.625000in}{0.679783in}}%
\pgfusepath{stroke}%
\end{pgfscope}%
\begin{pgfscope}%
\pgfpathrectangle{\pgfqpoint{0.625000in}{0.550000in}}{\pgfqpoint{3.875000in}{3.850000in}} %
\pgfusepath{clip}%
\pgfsetbuttcap%
\pgfsetroundjoin%
\pgfsetlinewidth{0.501875pt}%
\definecolor{currentstroke}{rgb}{0.000000,0.000000,0.000000}%
\pgfsetstrokecolor{currentstroke}%
\pgfsetdash{}{0pt}%
\pgfpathmoveto{\pgfqpoint{0.625000in}{0.689699in}}%
\pgfpathlineto{\pgfqpoint{0.630370in}{0.694737in}}%
\pgfpathlineto{\pgfqpoint{0.625000in}{0.697814in}}%
\pgfusepath{stroke}%
\end{pgfscope}%
\begin{pgfscope}%
\pgfpathrectangle{\pgfqpoint{0.625000in}{0.550000in}}{\pgfqpoint{3.875000in}{3.850000in}} %
\pgfusepath{clip}%
\pgfsetbuttcap%
\pgfsetroundjoin%
\pgfsetlinewidth{0.501875pt}%
\definecolor{currentstroke}{rgb}{0.000000,0.000000,0.000000}%
\pgfsetstrokecolor{currentstroke}%
\pgfsetdash{}{0pt}%
\pgfpathmoveto{\pgfqpoint{0.625000in}{0.756102in}}%
\pgfpathlineto{\pgfqpoint{0.632028in}{0.762281in}}%
\pgfpathlineto{\pgfqpoint{0.627115in}{0.771930in}}%
\pgfpathlineto{\pgfqpoint{0.625000in}{0.779787in}}%
\pgfusepath{stroke}%
\end{pgfscope}%
\begin{pgfscope}%
\pgfpathrectangle{\pgfqpoint{0.625000in}{0.550000in}}{\pgfqpoint{3.875000in}{3.850000in}} %
\pgfusepath{clip}%
\pgfsetbuttcap%
\pgfsetroundjoin%
\pgfsetlinewidth{0.501875pt}%
\definecolor{currentstroke}{rgb}{0.000000,0.000000,0.000000}%
\pgfsetstrokecolor{currentstroke}%
\pgfsetdash{}{0pt}%
\pgfpathmoveto{\pgfqpoint{0.625000in}{0.794924in}}%
\pgfpathlineto{\pgfqpoint{0.633178in}{0.800877in}}%
\pgfpathlineto{\pgfqpoint{0.634712in}{0.804988in}}%
\pgfpathlineto{\pgfqpoint{0.639493in}{0.810526in}}%
\pgfpathlineto{\pgfqpoint{0.639586in}{0.820175in}}%
\pgfpathlineto{\pgfqpoint{0.634712in}{0.822861in}}%
\pgfpathlineto{\pgfqpoint{0.630669in}{0.820175in}}%
\pgfpathlineto{\pgfqpoint{0.625000in}{0.815509in}}%
\pgfusepath{stroke}%
\end{pgfscope}%
\begin{pgfscope}%
\pgfpathrectangle{\pgfqpoint{0.625000in}{0.550000in}}{\pgfqpoint{3.875000in}{3.850000in}} %
\pgfusepath{clip}%
\pgfsetbuttcap%
\pgfsetroundjoin%
\pgfsetlinewidth{0.501875pt}%
\definecolor{currentstroke}{rgb}{0.000000,0.000000,0.000000}%
\pgfsetstrokecolor{currentstroke}%
\pgfsetdash{}{0pt}%
\pgfpathmoveto{\pgfqpoint{0.625000in}{0.882732in}}%
\pgfpathlineto{\pgfqpoint{0.633470in}{0.887719in}}%
\pgfpathlineto{\pgfqpoint{0.625000in}{0.892707in}}%
\pgfusepath{stroke}%
\end{pgfscope}%
\begin{pgfscope}%
\pgfpathrectangle{\pgfqpoint{0.625000in}{0.550000in}}{\pgfqpoint{3.875000in}{3.850000in}} %
\pgfusepath{clip}%
\pgfsetbuttcap%
\pgfsetroundjoin%
\pgfsetlinewidth{0.501875pt}%
\definecolor{currentstroke}{rgb}{0.000000,0.000000,0.000000}%
\pgfsetstrokecolor{currentstroke}%
\pgfsetdash{}{0pt}%
\pgfpathmoveto{\pgfqpoint{0.625000in}{0.910568in}}%
\pgfpathlineto{\pgfqpoint{0.630083in}{0.907018in}}%
\pgfpathlineto{\pgfqpoint{0.634377in}{0.897368in}}%
\pgfpathlineto{\pgfqpoint{0.634712in}{0.895578in}}%
\pgfpathlineto{\pgfqpoint{0.638105in}{0.897368in}}%
\pgfpathlineto{\pgfqpoint{0.644424in}{0.898873in}}%
\pgfpathlineto{\pgfqpoint{0.651209in}{0.907018in}}%
\pgfpathlineto{\pgfqpoint{0.652723in}{0.916667in}}%
\pgfpathlineto{\pgfqpoint{0.649279in}{0.926316in}}%
\pgfpathlineto{\pgfqpoint{0.644424in}{0.928940in}}%
\pgfpathlineto{\pgfqpoint{0.634712in}{0.932804in}}%
\pgfpathlineto{\pgfqpoint{0.629556in}{0.926316in}}%
\pgfpathlineto{\pgfqpoint{0.625000in}{0.920724in}}%
\pgfusepath{stroke}%
\end{pgfscope}%
\begin{pgfscope}%
\pgfpathrectangle{\pgfqpoint{0.625000in}{0.550000in}}{\pgfqpoint{3.875000in}{3.850000in}} %
\pgfusepath{clip}%
\pgfsetbuttcap%
\pgfsetroundjoin%
\pgfsetlinewidth{0.501875pt}%
\definecolor{currentstroke}{rgb}{0.000000,0.000000,0.000000}%
\pgfsetstrokecolor{currentstroke}%
\pgfsetdash{}{0pt}%
\pgfpathmoveto{\pgfqpoint{0.625000in}{0.944724in}}%
\pgfpathlineto{\pgfqpoint{0.634712in}{0.940789in}}%
\pgfpathlineto{\pgfqpoint{0.638915in}{0.945614in}}%
\pgfpathlineto{\pgfqpoint{0.634712in}{0.954737in}}%
\pgfpathlineto{\pgfqpoint{0.634304in}{0.955263in}}%
\pgfpathlineto{\pgfqpoint{0.625000in}{0.959468in}}%
\pgfusepath{stroke}%
\end{pgfscope}%
\begin{pgfscope}%
\pgfpathrectangle{\pgfqpoint{0.625000in}{0.550000in}}{\pgfqpoint{3.875000in}{3.850000in}} %
\pgfusepath{clip}%
\pgfsetbuttcap%
\pgfsetroundjoin%
\pgfsetlinewidth{0.501875pt}%
\definecolor{currentstroke}{rgb}{0.000000,0.000000,0.000000}%
\pgfsetstrokecolor{currentstroke}%
\pgfsetdash{}{0pt}%
\pgfpathmoveto{\pgfqpoint{0.625000in}{1.058202in}}%
\pgfpathlineto{\pgfqpoint{0.634712in}{1.056541in}}%
\pgfpathlineto{\pgfqpoint{0.637338in}{1.061404in}}%
\pgfpathlineto{\pgfqpoint{0.642127in}{1.071053in}}%
\pgfpathlineto{\pgfqpoint{0.634712in}{1.074546in}}%
\pgfpathlineto{\pgfqpoint{0.625000in}{1.076308in}}%
\pgfusepath{stroke}%
\end{pgfscope}%
\begin{pgfscope}%
\pgfpathrectangle{\pgfqpoint{0.625000in}{0.550000in}}{\pgfqpoint{3.875000in}{3.850000in}} %
\pgfusepath{clip}%
\pgfsetbuttcap%
\pgfsetroundjoin%
\pgfsetlinewidth{0.501875pt}%
\definecolor{currentstroke}{rgb}{0.000000,0.000000,0.000000}%
\pgfsetstrokecolor{currentstroke}%
\pgfsetdash{}{0pt}%
\pgfpathmoveto{\pgfqpoint{0.625000in}{1.103731in}}%
\pgfpathlineto{\pgfqpoint{0.633075in}{1.100000in}}%
\pgfpathlineto{\pgfqpoint{0.634712in}{1.096663in}}%
\pgfpathlineto{\pgfqpoint{0.644424in}{1.092686in}}%
\pgfpathlineto{\pgfqpoint{0.654135in}{1.090505in}}%
\pgfpathlineto{\pgfqpoint{0.663847in}{1.092118in}}%
\pgfpathlineto{\pgfqpoint{0.673559in}{1.097326in}}%
\pgfpathlineto{\pgfqpoint{0.676822in}{1.100000in}}%
\pgfpathlineto{\pgfqpoint{0.683271in}{1.106329in}}%
\pgfpathlineto{\pgfqpoint{0.685864in}{1.109649in}}%
\pgfpathlineto{\pgfqpoint{0.691174in}{1.119298in}}%
\pgfpathlineto{\pgfqpoint{0.692982in}{1.125473in}}%
\pgfpathlineto{\pgfqpoint{0.693950in}{1.128947in}}%
\pgfpathlineto{\pgfqpoint{0.694716in}{1.138596in}}%
\pgfpathlineto{\pgfqpoint{0.693318in}{1.148246in}}%
\pgfpathlineto{\pgfqpoint{0.692982in}{1.149026in}}%
\pgfpathlineto{\pgfqpoint{0.690219in}{1.157895in}}%
\pgfpathlineto{\pgfqpoint{0.684125in}{1.167544in}}%
\pgfpathlineto{\pgfqpoint{0.683271in}{1.168284in}}%
\pgfpathlineto{\pgfqpoint{0.674827in}{1.177193in}}%
\pgfpathlineto{\pgfqpoint{0.673559in}{1.177884in}}%
\pgfpathlineto{\pgfqpoint{0.663847in}{1.183843in}}%
\pgfpathlineto{\pgfqpoint{0.659440in}{1.186842in}}%
\pgfpathlineto{\pgfqpoint{0.654135in}{1.188495in}}%
\pgfpathlineto{\pgfqpoint{0.644424in}{1.193979in}}%
\pgfpathlineto{\pgfqpoint{0.638472in}{1.196491in}}%
\pgfpathlineto{\pgfqpoint{0.644424in}{1.198901in}}%
\pgfpathlineto{\pgfqpoint{0.654135in}{1.205773in}}%
\pgfpathlineto{\pgfqpoint{0.654765in}{1.206140in}}%
\pgfpathlineto{\pgfqpoint{0.660855in}{1.215789in}}%
\pgfpathlineto{\pgfqpoint{0.662416in}{1.225439in}}%
\pgfpathlineto{\pgfqpoint{0.661084in}{1.235088in}}%
\pgfpathlineto{\pgfqpoint{0.655884in}{1.244737in}}%
\pgfpathlineto{\pgfqpoint{0.654135in}{1.246834in}}%
\pgfpathlineto{\pgfqpoint{0.644424in}{1.253770in}}%
\pgfpathlineto{\pgfqpoint{0.641207in}{1.254386in}}%
\pgfpathlineto{\pgfqpoint{0.634712in}{1.256289in}}%
\pgfpathlineto{\pgfqpoint{0.633038in}{1.264035in}}%
\pgfpathlineto{\pgfqpoint{0.625000in}{1.269868in}}%
\pgfusepath{stroke}%
\end{pgfscope}%
\begin{pgfscope}%
\pgfpathrectangle{\pgfqpoint{0.625000in}{0.550000in}}{\pgfqpoint{3.875000in}{3.850000in}} %
\pgfusepath{clip}%
\pgfsetbuttcap%
\pgfsetroundjoin%
\pgfsetlinewidth{0.501875pt}%
\definecolor{currentstroke}{rgb}{0.000000,0.000000,0.000000}%
\pgfsetstrokecolor{currentstroke}%
\pgfsetdash{}{0pt}%
\pgfpathmoveto{\pgfqpoint{0.625000in}{1.123819in}}%
\pgfpathlineto{\pgfqpoint{0.630526in}{1.119298in}}%
\pgfpathlineto{\pgfqpoint{0.625000in}{1.114778in}}%
\pgfusepath{stroke}%
\end{pgfscope}%
\begin{pgfscope}%
\pgfpathrectangle{\pgfqpoint{0.625000in}{0.550000in}}{\pgfqpoint{3.875000in}{3.850000in}} %
\pgfusepath{clip}%
\pgfsetbuttcap%
\pgfsetroundjoin%
\pgfsetlinewidth{0.501875pt}%
\definecolor{currentstroke}{rgb}{0.000000,0.000000,0.000000}%
\pgfsetstrokecolor{currentstroke}%
\pgfsetdash{}{0pt}%
\pgfpathmoveto{\pgfqpoint{0.625000in}{1.154123in}}%
\pgfpathlineto{\pgfqpoint{0.627755in}{1.148246in}}%
\pgfpathlineto{\pgfqpoint{0.627541in}{1.138596in}}%
\pgfpathlineto{\pgfqpoint{0.625000in}{1.135295in}}%
\pgfusepath{stroke}%
\end{pgfscope}%
\begin{pgfscope}%
\pgfpathrectangle{\pgfqpoint{0.625000in}{0.550000in}}{\pgfqpoint{3.875000in}{3.850000in}} %
\pgfusepath{clip}%
\pgfsetbuttcap%
\pgfsetroundjoin%
\pgfsetlinewidth{0.501875pt}%
\definecolor{currentstroke}{rgb}{0.000000,0.000000,0.000000}%
\pgfsetstrokecolor{currentstroke}%
\pgfsetdash{}{0pt}%
\pgfpathmoveto{\pgfqpoint{0.625000in}{1.173133in}}%
\pgfpathlineto{\pgfqpoint{0.634712in}{1.176579in}}%
\pgfpathlineto{\pgfqpoint{0.639868in}{1.167544in}}%
\pgfpathlineto{\pgfqpoint{0.634712in}{1.162220in}}%
\pgfpathlineto{\pgfqpoint{0.625000in}{1.159976in}}%
\pgfusepath{stroke}%
\end{pgfscope}%
\begin{pgfscope}%
\pgfpathrectangle{\pgfqpoint{0.625000in}{0.550000in}}{\pgfqpoint{3.875000in}{3.850000in}} %
\pgfusepath{clip}%
\pgfsetbuttcap%
\pgfsetroundjoin%
\pgfsetlinewidth{0.501875pt}%
\definecolor{currentstroke}{rgb}{0.000000,0.000000,0.000000}%
\pgfsetstrokecolor{currentstroke}%
\pgfsetdash{}{0pt}%
\pgfpathmoveto{\pgfqpoint{0.625000in}{1.220161in}}%
\pgfpathlineto{\pgfqpoint{0.634712in}{1.217655in}}%
\pgfpathlineto{\pgfqpoint{0.636303in}{1.215789in}}%
\pgfpathlineto{\pgfqpoint{0.636032in}{1.206140in}}%
\pgfpathlineto{\pgfqpoint{0.634712in}{1.202010in}}%
\pgfpathlineto{\pgfqpoint{0.631405in}{1.196491in}}%
\pgfpathlineto{\pgfqpoint{0.629703in}{1.186842in}}%
\pgfpathlineto{\pgfqpoint{0.625000in}{1.183262in}}%
\pgfusepath{stroke}%
\end{pgfscope}%
\begin{pgfscope}%
\pgfpathrectangle{\pgfqpoint{0.625000in}{0.550000in}}{\pgfqpoint{3.875000in}{3.850000in}} %
\pgfusepath{clip}%
\pgfsetbuttcap%
\pgfsetroundjoin%
\pgfsetlinewidth{0.501875pt}%
\definecolor{currentstroke}{rgb}{0.000000,0.000000,0.000000}%
\pgfsetstrokecolor{currentstroke}%
\pgfsetdash{}{0pt}%
\pgfpathmoveto{\pgfqpoint{0.625000in}{1.236626in}}%
\pgfpathlineto{\pgfqpoint{0.626363in}{1.235088in}}%
\pgfpathlineto{\pgfqpoint{0.625000in}{1.233422in}}%
\pgfusepath{stroke}%
\end{pgfscope}%
\begin{pgfscope}%
\pgfpathrectangle{\pgfqpoint{0.625000in}{0.550000in}}{\pgfqpoint{3.875000in}{3.850000in}} %
\pgfusepath{clip}%
\pgfsetbuttcap%
\pgfsetroundjoin%
\pgfsetlinewidth{0.501875pt}%
\definecolor{currentstroke}{rgb}{0.000000,0.000000,0.000000}%
\pgfsetstrokecolor{currentstroke}%
\pgfsetdash{}{0pt}%
\pgfpathmoveto{\pgfqpoint{0.625000in}{1.258388in}}%
\pgfpathlineto{\pgfqpoint{0.634058in}{1.254386in}}%
\pgfpathlineto{\pgfqpoint{0.625000in}{1.250606in}}%
\pgfusepath{stroke}%
\end{pgfscope}%
\begin{pgfscope}%
\pgfpathrectangle{\pgfqpoint{0.625000in}{0.550000in}}{\pgfqpoint{3.875000in}{3.850000in}} %
\pgfusepath{clip}%
\pgfsetbuttcap%
\pgfsetroundjoin%
\pgfsetlinewidth{0.501875pt}%
\definecolor{currentstroke}{rgb}{0.000000,0.000000,0.000000}%
\pgfsetstrokecolor{currentstroke}%
\pgfsetdash{}{0pt}%
\pgfpathmoveto{\pgfqpoint{0.625000in}{1.330425in}}%
\pgfpathlineto{\pgfqpoint{0.626552in}{1.331579in}}%
\pgfpathlineto{\pgfqpoint{0.625000in}{1.333901in}}%
\pgfusepath{stroke}%
\end{pgfscope}%
\begin{pgfscope}%
\pgfpathrectangle{\pgfqpoint{0.625000in}{0.550000in}}{\pgfqpoint{3.875000in}{3.850000in}} %
\pgfusepath{clip}%
\pgfsetbuttcap%
\pgfsetroundjoin%
\pgfsetlinewidth{0.501875pt}%
\definecolor{currentstroke}{rgb}{0.000000,0.000000,0.000000}%
\pgfsetstrokecolor{currentstroke}%
\pgfsetdash{}{0pt}%
\pgfpathmoveto{\pgfqpoint{0.625000in}{1.363522in}}%
\pgfpathlineto{\pgfqpoint{0.634712in}{1.368545in}}%
\pgfpathlineto{\pgfqpoint{0.635337in}{1.370175in}}%
\pgfpathlineto{\pgfqpoint{0.634712in}{1.370746in}}%
\pgfpathlineto{\pgfqpoint{0.630091in}{1.379825in}}%
\pgfpathlineto{\pgfqpoint{0.625000in}{1.387001in}}%
\pgfusepath{stroke}%
\end{pgfscope}%
\begin{pgfscope}%
\pgfpathrectangle{\pgfqpoint{0.625000in}{0.550000in}}{\pgfqpoint{3.875000in}{3.850000in}} %
\pgfusepath{clip}%
\pgfsetbuttcap%
\pgfsetroundjoin%
\pgfsetlinewidth{0.501875pt}%
\definecolor{currentstroke}{rgb}{0.000000,0.000000,0.000000}%
\pgfsetstrokecolor{currentstroke}%
\pgfsetdash{}{0pt}%
\pgfpathmoveto{\pgfqpoint{0.625000in}{1.394613in}}%
\pgfpathlineto{\pgfqpoint{0.627249in}{1.399123in}}%
\pgfpathlineto{\pgfqpoint{0.625000in}{1.404359in}}%
\pgfusepath{stroke}%
\end{pgfscope}%
\begin{pgfscope}%
\pgfpathrectangle{\pgfqpoint{0.625000in}{0.550000in}}{\pgfqpoint{3.875000in}{3.850000in}} %
\pgfusepath{clip}%
\pgfsetbuttcap%
\pgfsetroundjoin%
\pgfsetlinewidth{0.501875pt}%
\definecolor{currentstroke}{rgb}{0.000000,0.000000,0.000000}%
\pgfsetstrokecolor{currentstroke}%
\pgfsetdash{}{0pt}%
\pgfpathmoveto{\pgfqpoint{0.625000in}{1.420902in}}%
\pgfpathlineto{\pgfqpoint{0.630499in}{1.428070in}}%
\pgfpathlineto{\pgfqpoint{0.625000in}{1.435238in}}%
\pgfusepath{stroke}%
\end{pgfscope}%
\begin{pgfscope}%
\pgfpathrectangle{\pgfqpoint{0.625000in}{0.550000in}}{\pgfqpoint{3.875000in}{3.850000in}} %
\pgfusepath{clip}%
\pgfsetbuttcap%
\pgfsetroundjoin%
\pgfsetlinewidth{0.501875pt}%
\definecolor{currentstroke}{rgb}{0.000000,0.000000,0.000000}%
\pgfsetstrokecolor{currentstroke}%
\pgfsetdash{}{0pt}%
\pgfpathmoveto{\pgfqpoint{0.625000in}{1.522162in}}%
\pgfpathlineto{\pgfqpoint{0.629838in}{1.524561in}}%
\pgfpathlineto{\pgfqpoint{0.634373in}{1.534211in}}%
\pgfpathlineto{\pgfqpoint{0.634712in}{1.539035in}}%
\pgfpathlineto{\pgfqpoint{0.637300in}{1.543860in}}%
\pgfpathlineto{\pgfqpoint{0.641268in}{1.553509in}}%
\pgfpathlineto{\pgfqpoint{0.644424in}{1.561740in}}%
\pgfpathlineto{\pgfqpoint{0.645166in}{1.563158in}}%
\pgfpathlineto{\pgfqpoint{0.649281in}{1.572807in}}%
\pgfpathlineto{\pgfqpoint{0.653146in}{1.582456in}}%
\pgfpathlineto{\pgfqpoint{0.654135in}{1.585161in}}%
\pgfpathlineto{\pgfqpoint{0.657350in}{1.592105in}}%
\pgfpathlineto{\pgfqpoint{0.661915in}{1.601754in}}%
\pgfpathlineto{\pgfqpoint{0.663847in}{1.610577in}}%
\pgfpathlineto{\pgfqpoint{0.664106in}{1.611404in}}%
\pgfpathlineto{\pgfqpoint{0.664455in}{1.621053in}}%
\pgfpathlineto{\pgfqpoint{0.670415in}{1.630702in}}%
\pgfpathlineto{\pgfqpoint{0.673559in}{1.631907in}}%
\pgfpathlineto{\pgfqpoint{0.682416in}{1.640351in}}%
\pgfpathlineto{\pgfqpoint{0.683271in}{1.641197in}}%
\pgfpathlineto{\pgfqpoint{0.689607in}{1.650000in}}%
\pgfpathlineto{\pgfqpoint{0.692982in}{1.657945in}}%
\pgfpathlineto{\pgfqpoint{0.693693in}{1.659649in}}%
\pgfpathlineto{\pgfqpoint{0.695392in}{1.669298in}}%
\pgfpathlineto{\pgfqpoint{0.694032in}{1.678947in}}%
\pgfpathlineto{\pgfqpoint{0.692982in}{1.680812in}}%
\pgfpathlineto{\pgfqpoint{0.689413in}{1.688596in}}%
\pgfpathlineto{\pgfqpoint{0.683271in}{1.693759in}}%
\pgfpathlineto{\pgfqpoint{0.677919in}{1.698246in}}%
\pgfpathlineto{\pgfqpoint{0.673559in}{1.699941in}}%
\pgfpathlineto{\pgfqpoint{0.663847in}{1.703295in}}%
\pgfpathlineto{\pgfqpoint{0.654135in}{1.705463in}}%
\pgfpathlineto{\pgfqpoint{0.644424in}{1.706786in}}%
\pgfpathlineto{\pgfqpoint{0.634712in}{1.706781in}}%
\pgfpathlineto{\pgfqpoint{0.628989in}{1.698246in}}%
\pgfpathlineto{\pgfqpoint{0.625000in}{1.694475in}}%
\pgfusepath{stroke}%
\end{pgfscope}%
\begin{pgfscope}%
\pgfpathrectangle{\pgfqpoint{0.625000in}{0.550000in}}{\pgfqpoint{3.875000in}{3.850000in}} %
\pgfusepath{clip}%
\pgfsetbuttcap%
\pgfsetroundjoin%
\pgfsetlinewidth{0.501875pt}%
\definecolor{currentstroke}{rgb}{0.000000,0.000000,0.000000}%
\pgfsetstrokecolor{currentstroke}%
\pgfsetdash{}{0pt}%
\pgfpathmoveto{\pgfqpoint{0.625000in}{1.565763in}}%
\pgfpathlineto{\pgfqpoint{0.632681in}{1.563158in}}%
\pgfpathlineto{\pgfqpoint{0.630505in}{1.553509in}}%
\pgfpathlineto{\pgfqpoint{0.625000in}{1.552819in}}%
\pgfusepath{stroke}%
\end{pgfscope}%
\begin{pgfscope}%
\pgfpathrectangle{\pgfqpoint{0.625000in}{0.550000in}}{\pgfqpoint{3.875000in}{3.850000in}} %
\pgfusepath{clip}%
\pgfsetbuttcap%
\pgfsetroundjoin%
\pgfsetlinewidth{0.501875pt}%
\definecolor{currentstroke}{rgb}{0.000000,0.000000,0.000000}%
\pgfsetstrokecolor{currentstroke}%
\pgfsetdash{}{0pt}%
\pgfpathmoveto{\pgfqpoint{0.625000in}{1.606042in}}%
\pgfpathlineto{\pgfqpoint{0.633155in}{1.601754in}}%
\pgfpathlineto{\pgfqpoint{0.634712in}{1.593999in}}%
\pgfpathlineto{\pgfqpoint{0.638565in}{1.592105in}}%
\pgfpathlineto{\pgfqpoint{0.634712in}{1.590683in}}%
\pgfpathlineto{\pgfqpoint{0.633345in}{1.582456in}}%
\pgfpathlineto{\pgfqpoint{0.625000in}{1.576973in}}%
\pgfusepath{stroke}%
\end{pgfscope}%
\begin{pgfscope}%
\pgfpathrectangle{\pgfqpoint{0.625000in}{0.550000in}}{\pgfqpoint{3.875000in}{3.850000in}} %
\pgfusepath{clip}%
\pgfsetbuttcap%
\pgfsetroundjoin%
\pgfsetlinewidth{0.501875pt}%
\definecolor{currentstroke}{rgb}{0.000000,0.000000,0.000000}%
\pgfsetstrokecolor{currentstroke}%
\pgfsetdash{}{0pt}%
\pgfpathmoveto{\pgfqpoint{0.625000in}{1.654130in}}%
\pgfpathlineto{\pgfqpoint{0.633852in}{1.650000in}}%
\pgfpathlineto{\pgfqpoint{0.634712in}{1.648321in}}%
\pgfpathlineto{\pgfqpoint{0.644424in}{1.643802in}}%
\pgfpathlineto{\pgfqpoint{0.651556in}{1.640351in}}%
\pgfpathlineto{\pgfqpoint{0.654135in}{1.633998in}}%
\pgfpathlineto{\pgfqpoint{0.660060in}{1.630702in}}%
\pgfpathlineto{\pgfqpoint{0.654135in}{1.627089in}}%
\pgfpathlineto{\pgfqpoint{0.644424in}{1.623667in}}%
\pgfpathlineto{\pgfqpoint{0.634712in}{1.623338in}}%
\pgfpathlineto{\pgfqpoint{0.632785in}{1.621053in}}%
\pgfpathlineto{\pgfqpoint{0.625000in}{1.616955in}}%
\pgfusepath{stroke}%
\end{pgfscope}%
\begin{pgfscope}%
\pgfpathrectangle{\pgfqpoint{0.625000in}{0.550000in}}{\pgfqpoint{3.875000in}{3.850000in}} %
\pgfusepath{clip}%
\pgfsetbuttcap%
\pgfsetroundjoin%
\pgfsetlinewidth{0.501875pt}%
\definecolor{currentstroke}{rgb}{0.000000,0.000000,0.000000}%
\pgfsetstrokecolor{currentstroke}%
\pgfsetdash{}{0pt}%
\pgfpathmoveto{\pgfqpoint{0.625000in}{1.674882in}}%
\pgfpathlineto{\pgfqpoint{0.629425in}{1.669298in}}%
\pgfpathlineto{\pgfqpoint{0.625000in}{1.665168in}}%
\pgfusepath{stroke}%
\end{pgfscope}%
\begin{pgfscope}%
\pgfpathrectangle{\pgfqpoint{0.625000in}{0.550000in}}{\pgfqpoint{3.875000in}{3.850000in}} %
\pgfusepath{clip}%
\pgfsetbuttcap%
\pgfsetroundjoin%
\pgfsetlinewidth{0.501875pt}%
\definecolor{currentstroke}{rgb}{0.000000,0.000000,0.000000}%
\pgfsetstrokecolor{currentstroke}%
\pgfsetdash{}{0pt}%
\pgfpathmoveto{\pgfqpoint{0.625000in}{1.714143in}}%
\pgfpathlineto{\pgfqpoint{0.634712in}{1.712719in}}%
\pgfpathlineto{\pgfqpoint{0.635620in}{1.717544in}}%
\pgfpathlineto{\pgfqpoint{0.634712in}{1.718984in}}%
\pgfpathlineto{\pgfqpoint{0.630767in}{1.727193in}}%
\pgfpathlineto{\pgfqpoint{0.625000in}{1.733591in}}%
\pgfusepath{stroke}%
\end{pgfscope}%
\begin{pgfscope}%
\pgfpathrectangle{\pgfqpoint{0.625000in}{0.550000in}}{\pgfqpoint{3.875000in}{3.850000in}} %
\pgfusepath{clip}%
\pgfsetbuttcap%
\pgfsetroundjoin%
\pgfsetlinewidth{0.501875pt}%
\definecolor{currentstroke}{rgb}{0.000000,0.000000,0.000000}%
\pgfsetstrokecolor{currentstroke}%
\pgfsetdash{}{0pt}%
\pgfpathmoveto{\pgfqpoint{0.625000in}{1.806999in}}%
\pgfpathlineto{\pgfqpoint{0.632808in}{1.814035in}}%
\pgfpathlineto{\pgfqpoint{0.625000in}{1.821071in}}%
\pgfusepath{stroke}%
\end{pgfscope}%
\begin{pgfscope}%
\pgfpathrectangle{\pgfqpoint{0.625000in}{0.550000in}}{\pgfqpoint{3.875000in}{3.850000in}} %
\pgfusepath{clip}%
\pgfsetbuttcap%
\pgfsetroundjoin%
\pgfsetlinewidth{0.501875pt}%
\definecolor{currentstroke}{rgb}{0.000000,0.000000,0.000000}%
\pgfsetstrokecolor{currentstroke}%
\pgfsetdash{}{0pt}%
\pgfpathmoveto{\pgfqpoint{0.625000in}{1.836142in}}%
\pgfpathlineto{\pgfqpoint{0.633926in}{1.833333in}}%
\pgfpathlineto{\pgfqpoint{0.634712in}{1.832203in}}%
\pgfpathlineto{\pgfqpoint{0.741541in}{1.788620in}}%
\pgfpathlineto{\pgfqpoint{0.760965in}{1.783722in}}%
\pgfpathlineto{\pgfqpoint{0.780388in}{1.781925in}}%
\pgfpathlineto{\pgfqpoint{0.799812in}{1.783140in}}%
\pgfpathlineto{\pgfqpoint{0.819236in}{1.786742in}}%
\pgfpathlineto{\pgfqpoint{0.845430in}{1.794737in}}%
\pgfpathlineto{\pgfqpoint{0.858083in}{1.799626in}}%
\pgfpathlineto{\pgfqpoint{0.887218in}{1.813742in}}%
\pgfpathlineto{\pgfqpoint{0.906642in}{1.825261in}}%
\pgfpathlineto{\pgfqpoint{0.931700in}{1.842982in}}%
\pgfpathlineto{\pgfqpoint{0.945489in}{1.854247in}}%
\pgfpathlineto{\pgfqpoint{0.964912in}{1.872390in}}%
\pgfpathlineto{\pgfqpoint{0.984336in}{1.893753in}}%
\pgfpathlineto{\pgfqpoint{1.003759in}{1.919464in}}%
\pgfpathlineto{\pgfqpoint{1.016403in}{1.939474in}}%
\pgfpathlineto{\pgfqpoint{1.031407in}{1.968421in}}%
\pgfpathlineto{\pgfqpoint{1.042607in}{1.996253in}}%
\pgfpathlineto{\pgfqpoint{1.049051in}{2.016667in}}%
\pgfpathlineto{\pgfqpoint{1.055724in}{2.045614in}}%
\pgfpathlineto{\pgfqpoint{1.059693in}{2.074561in}}%
\pgfpathlineto{\pgfqpoint{1.061061in}{2.103509in}}%
\pgfpathlineto{\pgfqpoint{1.059905in}{2.132456in}}%
\pgfpathlineto{\pgfqpoint{1.056194in}{2.161404in}}%
\pgfpathlineto{\pgfqpoint{1.049845in}{2.190351in}}%
\pgfpathlineto{\pgfqpoint{1.040759in}{2.219298in}}%
\pgfpathlineto{\pgfqpoint{1.028770in}{2.248246in}}%
\pgfpathlineto{\pgfqpoint{1.013471in}{2.277202in}}%
\pgfpathlineto{\pgfqpoint{1.001382in}{2.296491in}}%
\pgfpathlineto{\pgfqpoint{0.979762in}{2.325439in}}%
\pgfpathlineto{\pgfqpoint{0.953198in}{2.354386in}}%
\pgfpathlineto{\pgfqpoint{0.926065in}{2.378604in}}%
\pgfpathlineto{\pgfqpoint{0.906642in}{2.393612in}}%
\pgfpathlineto{\pgfqpoint{0.877506in}{2.412871in}}%
\pgfpathlineto{\pgfqpoint{0.843235in}{2.431579in}}%
\pgfpathlineto{\pgfqpoint{0.809524in}{2.446175in}}%
\pgfpathlineto{\pgfqpoint{0.780388in}{2.456442in}}%
\pgfpathlineto{\pgfqpoint{0.751253in}{2.464652in}}%
\pgfpathlineto{\pgfqpoint{0.722118in}{2.471015in}}%
\pgfpathlineto{\pgfqpoint{0.692982in}{2.475506in}}%
\pgfpathlineto{\pgfqpoint{0.683271in}{2.473570in}}%
\pgfpathlineto{\pgfqpoint{0.679202in}{2.470175in}}%
\pgfpathlineto{\pgfqpoint{0.673559in}{2.463872in}}%
\pgfpathlineto{\pgfqpoint{0.667482in}{2.470175in}}%
\pgfpathlineto{\pgfqpoint{0.664603in}{2.479825in}}%
\pgfpathlineto{\pgfqpoint{0.668561in}{2.489474in}}%
\pgfpathlineto{\pgfqpoint{0.673559in}{2.495779in}}%
\pgfpathlineto{\pgfqpoint{0.683271in}{2.486079in}}%
\pgfpathlineto{\pgfqpoint{0.692982in}{2.484144in}}%
\pgfpathlineto{\pgfqpoint{0.726950in}{2.489474in}}%
\pgfpathlineto{\pgfqpoint{0.770677in}{2.500213in}}%
\pgfpathlineto{\pgfqpoint{0.799812in}{2.509767in}}%
\pgfpathlineto{\pgfqpoint{0.828947in}{2.521562in}}%
\pgfpathlineto{\pgfqpoint{0.861904in}{2.537719in}}%
\pgfpathlineto{\pgfqpoint{0.896930in}{2.559266in}}%
\pgfpathlineto{\pgfqpoint{0.920385in}{2.576316in}}%
\pgfpathlineto{\pgfqpoint{0.945489in}{2.597999in}}%
\pgfpathlineto{\pgfqpoint{0.964912in}{2.617446in}}%
\pgfpathlineto{\pgfqpoint{0.979762in}{2.634211in}}%
\pgfpathlineto{\pgfqpoint{1.001382in}{2.663158in}}%
\pgfpathlineto{\pgfqpoint{1.018999in}{2.692105in}}%
\pgfpathlineto{\pgfqpoint{1.033061in}{2.721053in}}%
\pgfpathlineto{\pgfqpoint{1.044094in}{2.750000in}}%
\pgfpathlineto{\pgfqpoint{1.052318in}{2.779407in}}%
\pgfpathlineto{\pgfqpoint{1.057722in}{2.807895in}}%
\pgfpathlineto{\pgfqpoint{1.060570in}{2.836842in}}%
\pgfpathlineto{\pgfqpoint{1.060888in}{2.865789in}}%
\pgfpathlineto{\pgfqpoint{1.058666in}{2.894737in}}%
\pgfpathlineto{\pgfqpoint{1.053797in}{2.923684in}}%
\pgfpathlineto{\pgfqpoint{1.046200in}{2.952632in}}%
\pgfpathlineto{\pgfqpoint{1.035638in}{2.981579in}}%
\pgfpathlineto{\pgfqpoint{1.023183in}{3.007922in}}%
\pgfpathlineto{\pgfqpoint{1.010557in}{3.029825in}}%
\pgfpathlineto{\pgfqpoint{0.994048in}{3.053682in}}%
\pgfpathlineto{\pgfqpoint{0.974624in}{3.077037in}}%
\pgfpathlineto{\pgfqpoint{0.954458in}{3.097368in}}%
\pgfpathlineto{\pgfqpoint{0.931700in}{3.116667in}}%
\pgfpathlineto{\pgfqpoint{0.916353in}{3.127930in}}%
\pgfpathlineto{\pgfqpoint{0.896930in}{3.140387in}}%
\pgfpathlineto{\pgfqpoint{0.867794in}{3.155714in}}%
\pgfpathlineto{\pgfqpoint{0.845430in}{3.164912in}}%
\pgfpathlineto{\pgfqpoint{0.828947in}{3.170367in}}%
\pgfpathlineto{\pgfqpoint{0.809524in}{3.174949in}}%
\pgfpathlineto{\pgfqpoint{0.790100in}{3.177455in}}%
\pgfpathlineto{\pgfqpoint{0.770677in}{3.177241in}}%
\pgfpathlineto{\pgfqpoint{0.751253in}{3.173802in}}%
\pgfpathlineto{\pgfqpoint{0.731830in}{3.167534in}}%
\pgfpathlineto{\pgfqpoint{0.673559in}{3.143206in}}%
\pgfpathlineto{\pgfqpoint{0.625000in}{3.123507in}}%
\pgfpathlineto{\pgfqpoint{0.625000in}{3.123507in}}%
\pgfusepath{stroke}%
\end{pgfscope}%
\begin{pgfscope}%
\pgfpathrectangle{\pgfqpoint{0.625000in}{0.550000in}}{\pgfqpoint{3.875000in}{3.850000in}} %
\pgfusepath{clip}%
\pgfsetbuttcap%
\pgfsetroundjoin%
\pgfsetlinewidth{0.501875pt}%
\definecolor{currentstroke}{rgb}{0.000000,0.000000,0.000000}%
\pgfsetstrokecolor{currentstroke}%
\pgfsetdash{}{0pt}%
\pgfpathmoveto{\pgfqpoint{0.625000in}{1.877324in}}%
\pgfpathlineto{\pgfqpoint{0.630811in}{1.871930in}}%
\pgfpathlineto{\pgfqpoint{0.629294in}{1.862281in}}%
\pgfpathlineto{\pgfqpoint{0.625000in}{1.858469in}}%
\pgfusepath{stroke}%
\end{pgfscope}%
\begin{pgfscope}%
\pgfpathrectangle{\pgfqpoint{0.625000in}{0.550000in}}{\pgfqpoint{3.875000in}{3.850000in}} %
\pgfusepath{clip}%
\pgfsetbuttcap%
\pgfsetroundjoin%
\pgfsetlinewidth{0.501875pt}%
\definecolor{currentstroke}{rgb}{0.000000,0.000000,0.000000}%
\pgfsetstrokecolor{currentstroke}%
\pgfsetdash{}{0pt}%
\pgfpathmoveto{\pgfqpoint{0.625000in}{1.988827in}}%
\pgfpathlineto{\pgfqpoint{0.626726in}{1.987719in}}%
\pgfpathlineto{\pgfqpoint{0.630621in}{1.978070in}}%
\pgfpathlineto{\pgfqpoint{0.632243in}{1.968421in}}%
\pgfpathlineto{\pgfqpoint{0.630697in}{1.958772in}}%
\pgfpathlineto{\pgfqpoint{0.634712in}{1.949476in}}%
\pgfpathlineto{\pgfqpoint{0.635278in}{1.949123in}}%
\pgfpathlineto{\pgfqpoint{0.641581in}{1.939474in}}%
\pgfpathlineto{\pgfqpoint{0.634712in}{1.933111in}}%
\pgfpathlineto{\pgfqpoint{0.630602in}{1.929825in}}%
\pgfpathlineto{\pgfqpoint{0.627544in}{1.920175in}}%
\pgfpathlineto{\pgfqpoint{0.625000in}{1.917402in}}%
\pgfusepath{stroke}%
\end{pgfscope}%
\begin{pgfscope}%
\pgfpathrectangle{\pgfqpoint{0.625000in}{0.550000in}}{\pgfqpoint{3.875000in}{3.850000in}} %
\pgfusepath{clip}%
\pgfsetbuttcap%
\pgfsetroundjoin%
\pgfsetlinewidth{0.501875pt}%
\definecolor{currentstroke}{rgb}{0.000000,0.000000,0.000000}%
\pgfsetstrokecolor{currentstroke}%
\pgfsetdash{}{0pt}%
\pgfpathmoveto{\pgfqpoint{0.625000in}{2.064266in}}%
\pgfpathlineto{\pgfqpoint{0.628823in}{2.055263in}}%
\pgfpathlineto{\pgfqpoint{0.627070in}{2.045614in}}%
\pgfpathlineto{\pgfqpoint{0.625000in}{2.042728in}}%
\pgfusepath{stroke}%
\end{pgfscope}%
\begin{pgfscope}%
\pgfpathrectangle{\pgfqpoint{0.625000in}{0.550000in}}{\pgfqpoint{3.875000in}{3.850000in}} %
\pgfusepath{clip}%
\pgfsetbuttcap%
\pgfsetroundjoin%
\pgfsetlinewidth{0.501875pt}%
\definecolor{currentstroke}{rgb}{0.000000,0.000000,0.000000}%
\pgfsetstrokecolor{currentstroke}%
\pgfsetdash{}{0pt}%
\pgfpathmoveto{\pgfqpoint{0.625000in}{2.087302in}}%
\pgfpathlineto{\pgfqpoint{0.628665in}{2.084211in}}%
\pgfpathlineto{\pgfqpoint{0.628152in}{2.074561in}}%
\pgfpathlineto{\pgfqpoint{0.625000in}{2.065559in}}%
\pgfusepath{stroke}%
\end{pgfscope}%
\begin{pgfscope}%
\pgfpathrectangle{\pgfqpoint{0.625000in}{0.550000in}}{\pgfqpoint{3.875000in}{3.850000in}} %
\pgfusepath{clip}%
\pgfsetbuttcap%
\pgfsetroundjoin%
\pgfsetlinewidth{0.501875pt}%
\definecolor{currentstroke}{rgb}{0.000000,0.000000,0.000000}%
\pgfsetstrokecolor{currentstroke}%
\pgfsetdash{}{0pt}%
\pgfpathmoveto{\pgfqpoint{0.625000in}{2.135423in}}%
\pgfpathlineto{\pgfqpoint{0.627636in}{2.132456in}}%
\pgfpathlineto{\pgfqpoint{0.627887in}{2.122807in}}%
\pgfpathlineto{\pgfqpoint{0.628174in}{2.113158in}}%
\pgfpathlineto{\pgfqpoint{0.625000in}{2.107048in}}%
\pgfusepath{stroke}%
\end{pgfscope}%
\begin{pgfscope}%
\pgfpathrectangle{\pgfqpoint{0.625000in}{0.550000in}}{\pgfqpoint{3.875000in}{3.850000in}} %
\pgfusepath{clip}%
\pgfsetbuttcap%
\pgfsetroundjoin%
\pgfsetlinewidth{0.501875pt}%
\definecolor{currentstroke}{rgb}{0.000000,0.000000,0.000000}%
\pgfsetstrokecolor{currentstroke}%
\pgfsetdash{}{0pt}%
\pgfpathmoveto{\pgfqpoint{0.625000in}{2.178619in}}%
\pgfpathlineto{\pgfqpoint{0.631446in}{2.171053in}}%
\pgfpathlineto{\pgfqpoint{0.625000in}{2.164137in}}%
\pgfusepath{stroke}%
\end{pgfscope}%
\begin{pgfscope}%
\pgfpathrectangle{\pgfqpoint{0.625000in}{0.550000in}}{\pgfqpoint{3.875000in}{3.850000in}} %
\pgfusepath{clip}%
\pgfsetbuttcap%
\pgfsetroundjoin%
\pgfsetlinewidth{0.501875pt}%
\definecolor{currentstroke}{rgb}{0.000000,0.000000,0.000000}%
\pgfsetstrokecolor{currentstroke}%
\pgfsetdash{}{0pt}%
\pgfpathmoveto{\pgfqpoint{0.625000in}{2.271423in}}%
\pgfpathlineto{\pgfqpoint{0.629641in}{2.277193in}}%
\pgfpathlineto{\pgfqpoint{0.625000in}{2.282962in}}%
\pgfusepath{stroke}%
\end{pgfscope}%
\begin{pgfscope}%
\pgfpathrectangle{\pgfqpoint{0.625000in}{0.550000in}}{\pgfqpoint{3.875000in}{3.850000in}} %
\pgfusepath{clip}%
\pgfsetbuttcap%
\pgfsetroundjoin%
\pgfsetlinewidth{0.501875pt}%
\definecolor{currentstroke}{rgb}{0.000000,0.000000,0.000000}%
\pgfsetstrokecolor{currentstroke}%
\pgfsetdash{}{0pt}%
\pgfpathmoveto{\pgfqpoint{0.625000in}{2.289275in}}%
\pgfpathlineto{\pgfqpoint{0.633843in}{2.296491in}}%
\pgfpathlineto{\pgfqpoint{0.634712in}{2.300048in}}%
\pgfpathlineto{\pgfqpoint{0.644424in}{2.301109in}}%
\pgfpathlineto{\pgfqpoint{0.654135in}{2.303207in}}%
\pgfpathlineto{\pgfqpoint{0.662685in}{2.306140in}}%
\pgfpathlineto{\pgfqpoint{0.663847in}{2.306524in}}%
\pgfpathlineto{\pgfqpoint{0.673559in}{2.310985in}}%
\pgfpathlineto{\pgfqpoint{0.681155in}{2.315789in}}%
\pgfpathlineto{\pgfqpoint{0.683271in}{2.317161in}}%
\pgfpathlineto{\pgfqpoint{0.692982in}{2.325345in}}%
\pgfpathlineto{\pgfqpoint{0.693074in}{2.325439in}}%
\pgfpathlineto{\pgfqpoint{0.701650in}{2.335088in}}%
\pgfpathlineto{\pgfqpoint{0.702694in}{2.336449in}}%
\pgfpathlineto{\pgfqpoint{0.708105in}{2.344737in}}%
\pgfpathlineto{\pgfqpoint{0.712406in}{2.353293in}}%
\pgfpathlineto{\pgfqpoint{0.712897in}{2.354386in}}%
\pgfpathlineto{\pgfqpoint{0.716378in}{2.364035in}}%
\pgfpathlineto{\pgfqpoint{0.718617in}{2.373684in}}%
\pgfpathlineto{\pgfqpoint{0.719729in}{2.383333in}}%
\pgfpathlineto{\pgfqpoint{0.719788in}{2.392982in}}%
\pgfpathlineto{\pgfqpoint{0.718815in}{2.402632in}}%
\pgfpathlineto{\pgfqpoint{0.716739in}{2.412281in}}%
\pgfpathlineto{\pgfqpoint{0.713321in}{2.421930in}}%
\pgfpathlineto{\pgfqpoint{0.712406in}{2.423583in}}%
\pgfpathlineto{\pgfqpoint{0.708899in}{2.431579in}}%
\pgfpathlineto{\pgfqpoint{0.702694in}{2.440857in}}%
\pgfpathlineto{\pgfqpoint{0.702501in}{2.441228in}}%
\pgfpathlineto{\pgfqpoint{0.694558in}{2.450877in}}%
\pgfpathlineto{\pgfqpoint{0.692982in}{2.453109in}}%
\pgfpathlineto{\pgfqpoint{0.683647in}{2.460526in}}%
\pgfpathlineto{\pgfqpoint{0.692982in}{2.463821in}}%
\pgfpathlineto{\pgfqpoint{0.699600in}{2.460526in}}%
\pgfpathlineto{\pgfqpoint{0.702694in}{2.458295in}}%
\pgfpathlineto{\pgfqpoint{0.712406in}{2.451956in}}%
\pgfpathlineto{\pgfqpoint{0.714206in}{2.450877in}}%
\pgfpathlineto{\pgfqpoint{0.722118in}{2.444150in}}%
\pgfpathlineto{\pgfqpoint{0.725905in}{2.441228in}}%
\pgfpathlineto{\pgfqpoint{0.731830in}{2.434872in}}%
\pgfpathlineto{\pgfqpoint{0.735320in}{2.431579in}}%
\pgfpathlineto{\pgfqpoint{0.741541in}{2.423591in}}%
\pgfpathlineto{\pgfqpoint{0.743029in}{2.421930in}}%
\pgfpathlineto{\pgfqpoint{0.749546in}{2.412281in}}%
\pgfpathlineto{\pgfqpoint{0.751253in}{2.408914in}}%
\pgfpathlineto{\pgfqpoint{0.754981in}{2.402632in}}%
\pgfpathlineto{\pgfqpoint{0.759315in}{2.392982in}}%
\pgfpathlineto{\pgfqpoint{0.760965in}{2.388081in}}%
\pgfpathlineto{\pgfqpoint{0.762831in}{2.383333in}}%
\pgfpathlineto{\pgfqpoint{0.765575in}{2.373684in}}%
\pgfpathlineto{\pgfqpoint{0.767480in}{2.364035in}}%
\pgfpathlineto{\pgfqpoint{0.768635in}{2.354386in}}%
\pgfpathlineto{\pgfqpoint{0.769081in}{2.344737in}}%
\pgfpathlineto{\pgfqpoint{0.768826in}{2.335088in}}%
\pgfpathlineto{\pgfqpoint{0.767854in}{2.325439in}}%
\pgfpathlineto{\pgfqpoint{0.766132in}{2.315789in}}%
\pgfpathlineto{\pgfqpoint{0.763614in}{2.306140in}}%
\pgfpathlineto{\pgfqpoint{0.760965in}{2.298420in}}%
\pgfpathlineto{\pgfqpoint{0.760264in}{2.296491in}}%
\pgfpathlineto{\pgfqpoint{0.756060in}{2.286842in}}%
\pgfpathlineto{\pgfqpoint{0.751253in}{2.277915in}}%
\pgfpathlineto{\pgfqpoint{0.750829in}{2.277193in}}%
\pgfpathlineto{\pgfqpoint{0.744469in}{2.267544in}}%
\pgfpathlineto{\pgfqpoint{0.741541in}{2.263629in}}%
\pgfpathlineto{\pgfqpoint{0.736733in}{2.257895in}}%
\pgfpathlineto{\pgfqpoint{0.731830in}{2.252561in}}%
\pgfpathlineto{\pgfqpoint{0.727282in}{2.248246in}}%
\pgfpathlineto{\pgfqpoint{0.722118in}{2.243631in}}%
\pgfpathlineto{\pgfqpoint{0.715506in}{2.238596in}}%
\pgfpathlineto{\pgfqpoint{0.712406in}{2.236318in}}%
\pgfpathlineto{\pgfqpoint{0.702694in}{2.230289in}}%
\pgfpathlineto{\pgfqpoint{0.700089in}{2.228947in}}%
\pgfpathlineto{\pgfqpoint{0.692982in}{2.225297in}}%
\pgfpathlineto{\pgfqpoint{0.683271in}{2.221305in}}%
\pgfpathlineto{\pgfqpoint{0.677010in}{2.219298in}}%
\pgfpathlineto{\pgfqpoint{0.673559in}{2.218167in}}%
\pgfpathlineto{\pgfqpoint{0.663847in}{2.215736in}}%
\pgfpathlineto{\pgfqpoint{0.654135in}{2.213948in}}%
\pgfpathlineto{\pgfqpoint{0.644424in}{2.212096in}}%
\pgfpathlineto{\pgfqpoint{0.640030in}{2.209649in}}%
\pgfpathlineto{\pgfqpoint{0.634712in}{2.208331in}}%
\pgfpathlineto{\pgfqpoint{0.625000in}{2.206529in}}%
\pgfusepath{stroke}%
\end{pgfscope}%
\begin{pgfscope}%
\pgfpathrectangle{\pgfqpoint{0.625000in}{0.550000in}}{\pgfqpoint{3.875000in}{3.850000in}} %
\pgfusepath{clip}%
\pgfsetbuttcap%
\pgfsetroundjoin%
\pgfsetlinewidth{0.501875pt}%
\definecolor{currentstroke}{rgb}{0.000000,0.000000,0.000000}%
\pgfsetstrokecolor{currentstroke}%
\pgfsetdash{}{0pt}%
\pgfpathmoveto{\pgfqpoint{0.625000in}{2.328490in}}%
\pgfpathlineto{\pgfqpoint{0.627784in}{2.325439in}}%
\pgfpathlineto{\pgfqpoint{0.629685in}{2.315789in}}%
\pgfpathlineto{\pgfqpoint{0.625000in}{2.311680in}}%
\pgfusepath{stroke}%
\end{pgfscope}%
\begin{pgfscope}%
\pgfpathrectangle{\pgfqpoint{0.625000in}{0.550000in}}{\pgfqpoint{3.875000in}{3.850000in}} %
\pgfusepath{clip}%
\pgfsetbuttcap%
\pgfsetroundjoin%
\pgfsetlinewidth{0.501875pt}%
\definecolor{currentstroke}{rgb}{0.000000,0.000000,0.000000}%
\pgfsetstrokecolor{currentstroke}%
\pgfsetdash{}{0pt}%
\pgfpathmoveto{\pgfqpoint{0.625000in}{2.368675in}}%
\pgfpathlineto{\pgfqpoint{0.630352in}{2.373684in}}%
\pgfpathlineto{\pgfqpoint{0.634712in}{2.377269in}}%
\pgfpathlineto{\pgfqpoint{0.644424in}{2.379770in}}%
\pgfpathlineto{\pgfqpoint{0.651586in}{2.383333in}}%
\pgfpathlineto{\pgfqpoint{0.654135in}{2.384580in}}%
\pgfpathlineto{\pgfqpoint{0.663847in}{2.392457in}}%
\pgfpathlineto{\pgfqpoint{0.664314in}{2.392982in}}%
\pgfpathlineto{\pgfqpoint{0.671496in}{2.402632in}}%
\pgfpathlineto{\pgfqpoint{0.673559in}{2.406733in}}%
\pgfpathlineto{\pgfqpoint{0.676008in}{2.412281in}}%
\pgfpathlineto{\pgfqpoint{0.678054in}{2.421930in}}%
\pgfpathlineto{\pgfqpoint{0.678484in}{2.431579in}}%
\pgfpathlineto{\pgfqpoint{0.679743in}{2.441228in}}%
\pgfpathlineto{\pgfqpoint{0.683271in}{2.444948in}}%
\pgfpathlineto{\pgfqpoint{0.685742in}{2.441228in}}%
\pgfpathlineto{\pgfqpoint{0.689103in}{2.431579in}}%
\pgfpathlineto{\pgfqpoint{0.690367in}{2.421930in}}%
\pgfpathlineto{\pgfqpoint{0.690871in}{2.412281in}}%
\pgfpathlineto{\pgfqpoint{0.689553in}{2.402632in}}%
\pgfpathlineto{\pgfqpoint{0.686584in}{2.392982in}}%
\pgfpathlineto{\pgfqpoint{0.683271in}{2.386056in}}%
\pgfpathlineto{\pgfqpoint{0.681710in}{2.383333in}}%
\pgfpathlineto{\pgfqpoint{0.674426in}{2.373684in}}%
\pgfpathlineto{\pgfqpoint{0.673559in}{2.372705in}}%
\pgfpathlineto{\pgfqpoint{0.663847in}{2.364440in}}%
\pgfpathlineto{\pgfqpoint{0.663159in}{2.364035in}}%
\pgfpathlineto{\pgfqpoint{0.654135in}{2.358798in}}%
\pgfpathlineto{\pgfqpoint{0.644424in}{2.355386in}}%
\pgfpathlineto{\pgfqpoint{0.638895in}{2.354386in}}%
\pgfpathlineto{\pgfqpoint{0.634712in}{2.353551in}}%
\pgfpathlineto{\pgfqpoint{0.625000in}{2.349192in}}%
\pgfusepath{stroke}%
\end{pgfscope}%
\begin{pgfscope}%
\pgfpathrectangle{\pgfqpoint{0.625000in}{0.550000in}}{\pgfqpoint{3.875000in}{3.850000in}} %
\pgfusepath{clip}%
\pgfsetbuttcap%
\pgfsetroundjoin%
\pgfsetlinewidth{0.501875pt}%
\definecolor{currentstroke}{rgb}{0.000000,0.000000,0.000000}%
\pgfsetstrokecolor{currentstroke}%
\pgfsetdash{}{0pt}%
\pgfpathmoveto{\pgfqpoint{0.625000in}{2.410592in}}%
\pgfpathlineto{\pgfqpoint{0.634712in}{2.407637in}}%
\pgfpathlineto{\pgfqpoint{0.643401in}{2.412281in}}%
\pgfpathlineto{\pgfqpoint{0.644424in}{2.413546in}}%
\pgfpathlineto{\pgfqpoint{0.654135in}{2.420424in}}%
\pgfpathlineto{\pgfqpoint{0.658539in}{2.412281in}}%
\pgfpathlineto{\pgfqpoint{0.654135in}{2.407366in}}%
\pgfpathlineto{\pgfqpoint{0.647171in}{2.402632in}}%
\pgfpathlineto{\pgfqpoint{0.644424in}{2.400859in}}%
\pgfpathlineto{\pgfqpoint{0.634712in}{2.397417in}}%
\pgfpathlineto{\pgfqpoint{0.625000in}{2.394077in}}%
\pgfusepath{stroke}%
\end{pgfscope}%
\begin{pgfscope}%
\pgfpathrectangle{\pgfqpoint{0.625000in}{0.550000in}}{\pgfqpoint{3.875000in}{3.850000in}} %
\pgfusepath{clip}%
\pgfsetbuttcap%
\pgfsetroundjoin%
\pgfsetlinewidth{0.501875pt}%
\definecolor{currentstroke}{rgb}{0.000000,0.000000,0.000000}%
\pgfsetstrokecolor{currentstroke}%
\pgfsetdash{}{0pt}%
\pgfpathmoveto{\pgfqpoint{0.625000in}{2.450939in}}%
\pgfpathlineto{\pgfqpoint{0.625053in}{2.450877in}}%
\pgfpathlineto{\pgfqpoint{0.634276in}{2.441228in}}%
\pgfpathlineto{\pgfqpoint{0.629884in}{2.431579in}}%
\pgfpathlineto{\pgfqpoint{0.634712in}{2.428300in}}%
\pgfpathlineto{\pgfqpoint{0.640072in}{2.421930in}}%
\pgfpathlineto{\pgfqpoint{0.634712in}{2.417202in}}%
\pgfpathlineto{\pgfqpoint{0.625000in}{2.415476in}}%
\pgfusepath{stroke}%
\end{pgfscope}%
\begin{pgfscope}%
\pgfpathrectangle{\pgfqpoint{0.625000in}{0.550000in}}{\pgfqpoint{3.875000in}{3.850000in}} %
\pgfusepath{clip}%
\pgfsetbuttcap%
\pgfsetroundjoin%
\pgfsetlinewidth{0.501875pt}%
\definecolor{currentstroke}{rgb}{0.000000,0.000000,0.000000}%
\pgfsetstrokecolor{currentstroke}%
\pgfsetdash{}{0pt}%
\pgfpathmoveto{\pgfqpoint{0.625000in}{2.544173in}}%
\pgfpathlineto{\pgfqpoint{0.634712in}{2.542447in}}%
\pgfpathlineto{\pgfqpoint{0.640072in}{2.537719in}}%
\pgfpathlineto{\pgfqpoint{0.634712in}{2.531349in}}%
\pgfpathlineto{\pgfqpoint{0.629884in}{2.528070in}}%
\pgfpathlineto{\pgfqpoint{0.634276in}{2.518421in}}%
\pgfpathlineto{\pgfqpoint{0.625053in}{2.508772in}}%
\pgfpathlineto{\pgfqpoint{0.625000in}{2.508710in}}%
\pgfusepath{stroke}%
\end{pgfscope}%
\begin{pgfscope}%
\pgfpathrectangle{\pgfqpoint{0.625000in}{0.550000in}}{\pgfqpoint{3.875000in}{3.850000in}} %
\pgfusepath{clip}%
\pgfsetbuttcap%
\pgfsetroundjoin%
\pgfsetlinewidth{0.501875pt}%
\definecolor{currentstroke}{rgb}{0.000000,0.000000,0.000000}%
\pgfsetstrokecolor{currentstroke}%
\pgfsetdash{}{0pt}%
\pgfpathmoveto{\pgfqpoint{0.625000in}{2.565572in}}%
\pgfpathlineto{\pgfqpoint{0.634712in}{2.562232in}}%
\pgfpathlineto{\pgfqpoint{0.644424in}{2.558790in}}%
\pgfpathlineto{\pgfqpoint{0.647171in}{2.557018in}}%
\pgfpathlineto{\pgfqpoint{0.654135in}{2.552283in}}%
\pgfpathlineto{\pgfqpoint{0.658539in}{2.547368in}}%
\pgfpathlineto{\pgfqpoint{0.654135in}{2.539225in}}%
\pgfpathlineto{\pgfqpoint{0.644424in}{2.546103in}}%
\pgfpathlineto{\pgfqpoint{0.643401in}{2.547368in}}%
\pgfpathlineto{\pgfqpoint{0.634712in}{2.552012in}}%
\pgfpathlineto{\pgfqpoint{0.625000in}{2.549058in}}%
\pgfusepath{stroke}%
\end{pgfscope}%
\begin{pgfscope}%
\pgfpathrectangle{\pgfqpoint{0.625000in}{0.550000in}}{\pgfqpoint{3.875000in}{3.850000in}} %
\pgfusepath{clip}%
\pgfsetbuttcap%
\pgfsetroundjoin%
\pgfsetlinewidth{0.501875pt}%
\definecolor{currentstroke}{rgb}{0.000000,0.000000,0.000000}%
\pgfsetstrokecolor{currentstroke}%
\pgfsetdash{}{0pt}%
\pgfpathmoveto{\pgfqpoint{0.625000in}{2.610458in}}%
\pgfpathlineto{\pgfqpoint{0.634712in}{2.606098in}}%
\pgfpathlineto{\pgfqpoint{0.638895in}{2.605263in}}%
\pgfpathlineto{\pgfqpoint{0.644424in}{2.604263in}}%
\pgfpathlineto{\pgfqpoint{0.654135in}{2.600851in}}%
\pgfpathlineto{\pgfqpoint{0.663159in}{2.595614in}}%
\pgfpathlineto{\pgfqpoint{0.663847in}{2.595209in}}%
\pgfpathlineto{\pgfqpoint{0.673559in}{2.586944in}}%
\pgfpathlineto{\pgfqpoint{0.674426in}{2.585965in}}%
\pgfpathlineto{\pgfqpoint{0.681710in}{2.576316in}}%
\pgfpathlineto{\pgfqpoint{0.683271in}{2.573593in}}%
\pgfpathlineto{\pgfqpoint{0.686584in}{2.566667in}}%
\pgfpathlineto{\pgfqpoint{0.689553in}{2.557018in}}%
\pgfpathlineto{\pgfqpoint{0.690871in}{2.547368in}}%
\pgfpathlineto{\pgfqpoint{0.690367in}{2.537719in}}%
\pgfpathlineto{\pgfqpoint{0.689103in}{2.528070in}}%
\pgfpathlineto{\pgfqpoint{0.685742in}{2.518421in}}%
\pgfpathlineto{\pgfqpoint{0.683271in}{2.514701in}}%
\pgfpathlineto{\pgfqpoint{0.679743in}{2.518421in}}%
\pgfpathlineto{\pgfqpoint{0.678484in}{2.528070in}}%
\pgfpathlineto{\pgfqpoint{0.678054in}{2.537719in}}%
\pgfpathlineto{\pgfqpoint{0.676008in}{2.547368in}}%
\pgfpathlineto{\pgfqpoint{0.673559in}{2.552916in}}%
\pgfpathlineto{\pgfqpoint{0.671496in}{2.557018in}}%
\pgfpathlineto{\pgfqpoint{0.664314in}{2.566667in}}%
\pgfpathlineto{\pgfqpoint{0.663847in}{2.567192in}}%
\pgfpathlineto{\pgfqpoint{0.654135in}{2.575069in}}%
\pgfpathlineto{\pgfqpoint{0.651586in}{2.576316in}}%
\pgfpathlineto{\pgfqpoint{0.644424in}{2.579880in}}%
\pgfpathlineto{\pgfqpoint{0.634712in}{2.582380in}}%
\pgfpathlineto{\pgfqpoint{0.630352in}{2.585965in}}%
\pgfpathlineto{\pgfqpoint{0.625000in}{2.590974in}}%
\pgfusepath{stroke}%
\end{pgfscope}%
\begin{pgfscope}%
\pgfpathrectangle{\pgfqpoint{0.625000in}{0.550000in}}{\pgfqpoint{3.875000in}{3.850000in}} %
\pgfusepath{clip}%
\pgfsetbuttcap%
\pgfsetroundjoin%
\pgfsetlinewidth{0.501875pt}%
\definecolor{currentstroke}{rgb}{0.000000,0.000000,0.000000}%
\pgfsetstrokecolor{currentstroke}%
\pgfsetdash{}{0pt}%
\pgfpathmoveto{\pgfqpoint{0.625000in}{2.647969in}}%
\pgfpathlineto{\pgfqpoint{0.629685in}{2.643860in}}%
\pgfpathlineto{\pgfqpoint{0.625000in}{2.639188in}}%
\pgfusepath{stroke}%
\end{pgfscope}%
\begin{pgfscope}%
\pgfpathrectangle{\pgfqpoint{0.625000in}{0.550000in}}{\pgfqpoint{3.875000in}{3.850000in}} %
\pgfusepath{clip}%
\pgfsetbuttcap%
\pgfsetroundjoin%
\pgfsetlinewidth{0.501875pt}%
\definecolor{currentstroke}{rgb}{0.000000,0.000000,0.000000}%
\pgfsetstrokecolor{currentstroke}%
\pgfsetdash{}{0pt}%
\pgfpathmoveto{\pgfqpoint{0.625000in}{2.676687in}}%
\pgfpathlineto{\pgfqpoint{0.629641in}{2.682456in}}%
\pgfpathlineto{\pgfqpoint{0.625000in}{2.688226in}}%
\pgfusepath{stroke}%
\end{pgfscope}%
\begin{pgfscope}%
\pgfpathrectangle{\pgfqpoint{0.625000in}{0.550000in}}{\pgfqpoint{3.875000in}{3.850000in}} %
\pgfusepath{clip}%
\pgfsetbuttcap%
\pgfsetroundjoin%
\pgfsetlinewidth{0.501875pt}%
\definecolor{currentstroke}{rgb}{0.000000,0.000000,0.000000}%
\pgfsetstrokecolor{currentstroke}%
\pgfsetdash{}{0pt}%
\pgfpathmoveto{\pgfqpoint{0.625000in}{2.753120in}}%
\pgfpathlineto{\pgfqpoint{0.634712in}{2.751318in}}%
\pgfpathlineto{\pgfqpoint{0.640030in}{2.750000in}}%
\pgfpathlineto{\pgfqpoint{0.644424in}{2.747553in}}%
\pgfpathlineto{\pgfqpoint{0.654135in}{2.745701in}}%
\pgfpathlineto{\pgfqpoint{0.663847in}{2.743913in}}%
\pgfpathlineto{\pgfqpoint{0.673559in}{2.741482in}}%
\pgfpathlineto{\pgfqpoint{0.677010in}{2.740351in}}%
\pgfpathlineto{\pgfqpoint{0.683271in}{2.738344in}}%
\pgfpathlineto{\pgfqpoint{0.692982in}{2.734352in}}%
\pgfpathlineto{\pgfqpoint{0.700089in}{2.730702in}}%
\pgfpathlineto{\pgfqpoint{0.702694in}{2.729361in}}%
\pgfpathlineto{\pgfqpoint{0.712406in}{2.723332in}}%
\pgfpathlineto{\pgfqpoint{0.715506in}{2.721053in}}%
\pgfpathlineto{\pgfqpoint{0.722118in}{2.716018in}}%
\pgfpathlineto{\pgfqpoint{0.727282in}{2.711404in}}%
\pgfpathlineto{\pgfqpoint{0.731830in}{2.707088in}}%
\pgfpathlineto{\pgfqpoint{0.736733in}{2.701754in}}%
\pgfpathlineto{\pgfqpoint{0.741541in}{2.696020in}}%
\pgfpathlineto{\pgfqpoint{0.744469in}{2.692105in}}%
\pgfpathlineto{\pgfqpoint{0.750829in}{2.682456in}}%
\pgfpathlineto{\pgfqpoint{0.751253in}{2.681734in}}%
\pgfpathlineto{\pgfqpoint{0.756060in}{2.672807in}}%
\pgfpathlineto{\pgfqpoint{0.760264in}{2.663158in}}%
\pgfpathlineto{\pgfqpoint{0.760965in}{2.661229in}}%
\pgfpathlineto{\pgfqpoint{0.763614in}{2.653509in}}%
\pgfpathlineto{\pgfqpoint{0.766132in}{2.643860in}}%
\pgfpathlineto{\pgfqpoint{0.767854in}{2.634211in}}%
\pgfpathlineto{\pgfqpoint{0.768826in}{2.624561in}}%
\pgfpathlineto{\pgfqpoint{0.769081in}{2.614912in}}%
\pgfpathlineto{\pgfqpoint{0.768635in}{2.605263in}}%
\pgfpathlineto{\pgfqpoint{0.767480in}{2.595614in}}%
\pgfpathlineto{\pgfqpoint{0.765575in}{2.585965in}}%
\pgfpathlineto{\pgfqpoint{0.762831in}{2.576316in}}%
\pgfpathlineto{\pgfqpoint{0.760965in}{2.571568in}}%
\pgfpathlineto{\pgfqpoint{0.759315in}{2.566667in}}%
\pgfpathlineto{\pgfqpoint{0.754981in}{2.557018in}}%
\pgfpathlineto{\pgfqpoint{0.751253in}{2.550735in}}%
\pgfpathlineto{\pgfqpoint{0.749546in}{2.547368in}}%
\pgfpathlineto{\pgfqpoint{0.743029in}{2.537719in}}%
\pgfpathlineto{\pgfqpoint{0.741541in}{2.536058in}}%
\pgfpathlineto{\pgfqpoint{0.735320in}{2.528070in}}%
\pgfpathlineto{\pgfqpoint{0.731830in}{2.524777in}}%
\pgfpathlineto{\pgfqpoint{0.725905in}{2.518421in}}%
\pgfpathlineto{\pgfqpoint{0.722118in}{2.515499in}}%
\pgfpathlineto{\pgfqpoint{0.714206in}{2.508772in}}%
\pgfpathlineto{\pgfqpoint{0.712406in}{2.507693in}}%
\pgfpathlineto{\pgfqpoint{0.702694in}{2.501354in}}%
\pgfpathlineto{\pgfqpoint{0.699600in}{2.499123in}}%
\pgfpathlineto{\pgfqpoint{0.692982in}{2.495828in}}%
\pgfpathlineto{\pgfqpoint{0.683647in}{2.499123in}}%
\pgfpathlineto{\pgfqpoint{0.692982in}{2.506540in}}%
\pgfpathlineto{\pgfqpoint{0.694558in}{2.508772in}}%
\pgfpathlineto{\pgfqpoint{0.702501in}{2.518421in}}%
\pgfpathlineto{\pgfqpoint{0.702694in}{2.518792in}}%
\pgfpathlineto{\pgfqpoint{0.708899in}{2.528070in}}%
\pgfpathlineto{\pgfqpoint{0.712406in}{2.536066in}}%
\pgfpathlineto{\pgfqpoint{0.713321in}{2.537719in}}%
\pgfpathlineto{\pgfqpoint{0.716739in}{2.547368in}}%
\pgfpathlineto{\pgfqpoint{0.718815in}{2.557018in}}%
\pgfpathlineto{\pgfqpoint{0.719788in}{2.566667in}}%
\pgfpathlineto{\pgfqpoint{0.719729in}{2.576316in}}%
\pgfpathlineto{\pgfqpoint{0.718617in}{2.585965in}}%
\pgfpathlineto{\pgfqpoint{0.716378in}{2.595614in}}%
\pgfpathlineto{\pgfqpoint{0.712897in}{2.605263in}}%
\pgfpathlineto{\pgfqpoint{0.712406in}{2.606356in}}%
\pgfpathlineto{\pgfqpoint{0.708105in}{2.614912in}}%
\pgfpathlineto{\pgfqpoint{0.702694in}{2.623200in}}%
\pgfpathlineto{\pgfqpoint{0.701650in}{2.624561in}}%
\pgfpathlineto{\pgfqpoint{0.693074in}{2.634211in}}%
\pgfpathlineto{\pgfqpoint{0.692982in}{2.634305in}}%
\pgfpathlineto{\pgfqpoint{0.683271in}{2.642488in}}%
\pgfpathlineto{\pgfqpoint{0.681155in}{2.643860in}}%
\pgfpathlineto{\pgfqpoint{0.673559in}{2.648664in}}%
\pgfpathlineto{\pgfqpoint{0.663847in}{2.653125in}}%
\pgfpathlineto{\pgfqpoint{0.662685in}{2.653509in}}%
\pgfpathlineto{\pgfqpoint{0.654135in}{2.656442in}}%
\pgfpathlineto{\pgfqpoint{0.644424in}{2.658541in}}%
\pgfpathlineto{\pgfqpoint{0.634712in}{2.659601in}}%
\pgfpathlineto{\pgfqpoint{0.633843in}{2.663158in}}%
\pgfpathlineto{\pgfqpoint{0.625000in}{2.670374in}}%
\pgfusepath{stroke}%
\end{pgfscope}%
\begin{pgfscope}%
\pgfpathrectangle{\pgfqpoint{0.625000in}{0.550000in}}{\pgfqpoint{3.875000in}{3.850000in}} %
\pgfusepath{clip}%
\pgfsetbuttcap%
\pgfsetroundjoin%
\pgfsetlinewidth{0.501875pt}%
\definecolor{currentstroke}{rgb}{0.000000,0.000000,0.000000}%
\pgfsetstrokecolor{currentstroke}%
\pgfsetdash{}{0pt}%
\pgfpathmoveto{\pgfqpoint{0.625000in}{2.852602in}}%
\pgfpathlineto{\pgfqpoint{0.628174in}{2.846491in}}%
\pgfpathlineto{\pgfqpoint{0.627887in}{2.836842in}}%
\pgfpathlineto{\pgfqpoint{0.627636in}{2.827193in}}%
\pgfpathlineto{\pgfqpoint{0.625000in}{2.824226in}}%
\pgfusepath{stroke}%
\end{pgfscope}%
\begin{pgfscope}%
\pgfpathrectangle{\pgfqpoint{0.625000in}{0.550000in}}{\pgfqpoint{3.875000in}{3.850000in}} %
\pgfusepath{clip}%
\pgfsetbuttcap%
\pgfsetroundjoin%
\pgfsetlinewidth{0.501875pt}%
\definecolor{currentstroke}{rgb}{0.000000,0.000000,0.000000}%
\pgfsetstrokecolor{currentstroke}%
\pgfsetdash{}{0pt}%
\pgfpathmoveto{\pgfqpoint{0.625000in}{2.894090in}}%
\pgfpathlineto{\pgfqpoint{0.628152in}{2.885088in}}%
\pgfpathlineto{\pgfqpoint{0.628665in}{2.875439in}}%
\pgfpathlineto{\pgfqpoint{0.625000in}{2.872347in}}%
\pgfusepath{stroke}%
\end{pgfscope}%
\begin{pgfscope}%
\pgfpathrectangle{\pgfqpoint{0.625000in}{0.550000in}}{\pgfqpoint{3.875000in}{3.850000in}} %
\pgfusepath{clip}%
\pgfsetbuttcap%
\pgfsetroundjoin%
\pgfsetlinewidth{0.501875pt}%
\definecolor{currentstroke}{rgb}{0.000000,0.000000,0.000000}%
\pgfsetstrokecolor{currentstroke}%
\pgfsetdash{}{0pt}%
\pgfpathmoveto{\pgfqpoint{0.625000in}{2.916921in}}%
\pgfpathlineto{\pgfqpoint{0.627070in}{2.914035in}}%
\pgfpathlineto{\pgfqpoint{0.628823in}{2.904386in}}%
\pgfpathlineto{\pgfqpoint{0.625000in}{2.895383in}}%
\pgfusepath{stroke}%
\end{pgfscope}%
\begin{pgfscope}%
\pgfpathrectangle{\pgfqpoint{0.625000in}{0.550000in}}{\pgfqpoint{3.875000in}{3.850000in}} %
\pgfusepath{clip}%
\pgfsetbuttcap%
\pgfsetroundjoin%
\pgfsetlinewidth{0.501875pt}%
\definecolor{currentstroke}{rgb}{0.000000,0.000000,0.000000}%
\pgfsetstrokecolor{currentstroke}%
\pgfsetdash{}{0pt}%
\pgfpathmoveto{\pgfqpoint{0.625000in}{2.948669in}}%
\pgfpathlineto{\pgfqpoint{0.629063in}{2.942982in}}%
\pgfpathlineto{\pgfqpoint{0.625000in}{2.938780in}}%
\pgfusepath{stroke}%
\end{pgfscope}%
\begin{pgfscope}%
\pgfpathrectangle{\pgfqpoint{0.625000in}{0.550000in}}{\pgfqpoint{3.875000in}{3.850000in}} %
\pgfusepath{clip}%
\pgfsetbuttcap%
\pgfsetroundjoin%
\pgfsetlinewidth{0.501875pt}%
\definecolor{currentstroke}{rgb}{0.000000,0.000000,0.000000}%
\pgfsetstrokecolor{currentstroke}%
\pgfsetdash{}{0pt}%
\pgfpathmoveto{\pgfqpoint{0.625000in}{3.042247in}}%
\pgfpathlineto{\pgfqpoint{0.627544in}{3.039474in}}%
\pgfpathlineto{\pgfqpoint{0.630602in}{3.029825in}}%
\pgfpathlineto{\pgfqpoint{0.634712in}{3.026538in}}%
\pgfpathlineto{\pgfqpoint{0.641581in}{3.020175in}}%
\pgfpathlineto{\pgfqpoint{0.635278in}{3.010526in}}%
\pgfpathlineto{\pgfqpoint{0.634712in}{3.010173in}}%
\pgfpathlineto{\pgfqpoint{0.630697in}{3.000877in}}%
\pgfpathlineto{\pgfqpoint{0.632243in}{2.991228in}}%
\pgfpathlineto{\pgfqpoint{0.630621in}{2.981579in}}%
\pgfpathlineto{\pgfqpoint{0.626726in}{2.971930in}}%
\pgfpathlineto{\pgfqpoint{0.625000in}{2.970822in}}%
\pgfusepath{stroke}%
\end{pgfscope}%
\begin{pgfscope}%
\pgfpathrectangle{\pgfqpoint{0.625000in}{0.550000in}}{\pgfqpoint{3.875000in}{3.850000in}} %
\pgfusepath{clip}%
\pgfsetbuttcap%
\pgfsetroundjoin%
\pgfsetlinewidth{0.501875pt}%
\definecolor{currentstroke}{rgb}{0.000000,0.000000,0.000000}%
\pgfsetstrokecolor{currentstroke}%
\pgfsetdash{}{0pt}%
\pgfpathmoveto{\pgfqpoint{0.625000in}{3.092855in}}%
\pgfpathlineto{\pgfqpoint{0.630811in}{3.087719in}}%
\pgfpathlineto{\pgfqpoint{0.625000in}{3.082325in}}%
\pgfusepath{stroke}%
\end{pgfscope}%
\begin{pgfscope}%
\pgfpathrectangle{\pgfqpoint{0.625000in}{0.550000in}}{\pgfqpoint{3.875000in}{3.850000in}} %
\pgfusepath{clip}%
\pgfsetbuttcap%
\pgfsetroundjoin%
\pgfsetlinewidth{0.501875pt}%
\definecolor{currentstroke}{rgb}{0.000000,0.000000,0.000000}%
\pgfsetstrokecolor{currentstroke}%
\pgfsetdash{}{0pt}%
\pgfpathmoveto{\pgfqpoint{0.625000in}{3.138578in}}%
\pgfpathlineto{\pgfqpoint{0.632808in}{3.145614in}}%
\pgfpathlineto{\pgfqpoint{0.625000in}{3.152650in}}%
\pgfusepath{stroke}%
\end{pgfscope}%
\begin{pgfscope}%
\pgfpathrectangle{\pgfqpoint{0.625000in}{0.550000in}}{\pgfqpoint{3.875000in}{3.850000in}} %
\pgfusepath{clip}%
\pgfsetbuttcap%
\pgfsetroundjoin%
\pgfsetlinewidth{0.501875pt}%
\definecolor{currentstroke}{rgb}{0.000000,0.000000,0.000000}%
\pgfsetstrokecolor{currentstroke}%
\pgfsetdash{}{0pt}%
\pgfpathmoveto{\pgfqpoint{0.625000in}{3.226058in}}%
\pgfpathlineto{\pgfqpoint{0.630767in}{3.232456in}}%
\pgfpathlineto{\pgfqpoint{0.634712in}{3.240665in}}%
\pgfpathlineto{\pgfqpoint{0.635620in}{3.242105in}}%
\pgfpathlineto{\pgfqpoint{0.634712in}{3.246930in}}%
\pgfpathlineto{\pgfqpoint{0.633684in}{3.251754in}}%
\pgfpathlineto{\pgfqpoint{0.634712in}{3.252868in}}%
\pgfpathlineto{\pgfqpoint{0.644424in}{3.252864in}}%
\pgfpathlineto{\pgfqpoint{0.654135in}{3.254186in}}%
\pgfpathlineto{\pgfqpoint{0.663847in}{3.256355in}}%
\pgfpathlineto{\pgfqpoint{0.673559in}{3.259708in}}%
\pgfpathlineto{\pgfqpoint{0.677919in}{3.261404in}}%
\pgfpathlineto{\pgfqpoint{0.683271in}{3.265890in}}%
\pgfpathlineto{\pgfqpoint{0.689413in}{3.271053in}}%
\pgfpathlineto{\pgfqpoint{0.692982in}{3.278837in}}%
\pgfpathlineto{\pgfqpoint{0.694032in}{3.280702in}}%
\pgfpathlineto{\pgfqpoint{0.695392in}{3.290351in}}%
\pgfpathlineto{\pgfqpoint{0.693693in}{3.300000in}}%
\pgfpathlineto{\pgfqpoint{0.692982in}{3.301704in}}%
\pgfpathlineto{\pgfqpoint{0.689607in}{3.309649in}}%
\pgfpathlineto{\pgfqpoint{0.683271in}{3.318453in}}%
\pgfpathlineto{\pgfqpoint{0.682416in}{3.319298in}}%
\pgfpathlineto{\pgfqpoint{0.673559in}{3.327742in}}%
\pgfpathlineto{\pgfqpoint{0.670415in}{3.328947in}}%
\pgfpathlineto{\pgfqpoint{0.664455in}{3.338596in}}%
\pgfpathlineto{\pgfqpoint{0.664106in}{3.348246in}}%
\pgfpathlineto{\pgfqpoint{0.663847in}{3.349072in}}%
\pgfpathlineto{\pgfqpoint{0.661915in}{3.357895in}}%
\pgfpathlineto{\pgfqpoint{0.657350in}{3.367544in}}%
\pgfpathlineto{\pgfqpoint{0.654135in}{3.374489in}}%
\pgfpathlineto{\pgfqpoint{0.653146in}{3.377193in}}%
\pgfpathlineto{\pgfqpoint{0.649281in}{3.386842in}}%
\pgfpathlineto{\pgfqpoint{0.645166in}{3.396491in}}%
\pgfpathlineto{\pgfqpoint{0.644424in}{3.397909in}}%
\pgfpathlineto{\pgfqpoint{0.641268in}{3.406140in}}%
\pgfpathlineto{\pgfqpoint{0.637300in}{3.415789in}}%
\pgfpathlineto{\pgfqpoint{0.634712in}{3.420614in}}%
\pgfpathlineto{\pgfqpoint{0.634373in}{3.425439in}}%
\pgfpathlineto{\pgfqpoint{0.629838in}{3.435088in}}%
\pgfpathlineto{\pgfqpoint{0.625000in}{3.437487in}}%
\pgfusepath{stroke}%
\end{pgfscope}%
\begin{pgfscope}%
\pgfpathrectangle{\pgfqpoint{0.625000in}{0.550000in}}{\pgfqpoint{3.875000in}{3.850000in}} %
\pgfusepath{clip}%
\pgfsetbuttcap%
\pgfsetroundjoin%
\pgfsetlinewidth{0.501875pt}%
\definecolor{currentstroke}{rgb}{0.000000,0.000000,0.000000}%
\pgfsetstrokecolor{currentstroke}%
\pgfsetdash{}{0pt}%
\pgfpathmoveto{\pgfqpoint{0.625000in}{3.265174in}}%
\pgfpathlineto{\pgfqpoint{0.628989in}{3.261404in}}%
\pgfpathlineto{\pgfqpoint{0.625000in}{3.257664in}}%
\pgfusepath{stroke}%
\end{pgfscope}%
\begin{pgfscope}%
\pgfpathrectangle{\pgfqpoint{0.625000in}{0.550000in}}{\pgfqpoint{3.875000in}{3.850000in}} %
\pgfusepath{clip}%
\pgfsetbuttcap%
\pgfsetroundjoin%
\pgfsetlinewidth{0.501875pt}%
\definecolor{currentstroke}{rgb}{0.000000,0.000000,0.000000}%
\pgfsetstrokecolor{currentstroke}%
\pgfsetdash{}{0pt}%
\pgfpathmoveto{\pgfqpoint{0.625000in}{3.294481in}}%
\pgfpathlineto{\pgfqpoint{0.629425in}{3.290351in}}%
\pgfpathlineto{\pgfqpoint{0.625000in}{3.284767in}}%
\pgfusepath{stroke}%
\end{pgfscope}%
\begin{pgfscope}%
\pgfpathrectangle{\pgfqpoint{0.625000in}{0.550000in}}{\pgfqpoint{3.875000in}{3.850000in}} %
\pgfusepath{clip}%
\pgfsetbuttcap%
\pgfsetroundjoin%
\pgfsetlinewidth{0.501875pt}%
\definecolor{currentstroke}{rgb}{0.000000,0.000000,0.000000}%
\pgfsetstrokecolor{currentstroke}%
\pgfsetdash{}{0pt}%
\pgfpathmoveto{\pgfqpoint{0.625000in}{3.342694in}}%
\pgfpathlineto{\pgfqpoint{0.632785in}{3.338596in}}%
\pgfpathlineto{\pgfqpoint{0.634712in}{3.336311in}}%
\pgfpathlineto{\pgfqpoint{0.644424in}{3.335982in}}%
\pgfpathlineto{\pgfqpoint{0.654135in}{3.332560in}}%
\pgfpathlineto{\pgfqpoint{0.660060in}{3.328947in}}%
\pgfpathlineto{\pgfqpoint{0.654135in}{3.325652in}}%
\pgfpathlineto{\pgfqpoint{0.651556in}{3.319298in}}%
\pgfpathlineto{\pgfqpoint{0.644424in}{3.315847in}}%
\pgfpathlineto{\pgfqpoint{0.634712in}{3.311328in}}%
\pgfpathlineto{\pgfqpoint{0.633852in}{3.309649in}}%
\pgfpathlineto{\pgfqpoint{0.625000in}{3.305519in}}%
\pgfusepath{stroke}%
\end{pgfscope}%
\begin{pgfscope}%
\pgfpathrectangle{\pgfqpoint{0.625000in}{0.550000in}}{\pgfqpoint{3.875000in}{3.850000in}} %
\pgfusepath{clip}%
\pgfsetbuttcap%
\pgfsetroundjoin%
\pgfsetlinewidth{0.501875pt}%
\definecolor{currentstroke}{rgb}{0.000000,0.000000,0.000000}%
\pgfsetstrokecolor{currentstroke}%
\pgfsetdash{}{0pt}%
\pgfpathmoveto{\pgfqpoint{0.625000in}{3.382676in}}%
\pgfpathlineto{\pgfqpoint{0.633345in}{3.377193in}}%
\pgfpathlineto{\pgfqpoint{0.634712in}{3.368966in}}%
\pgfpathlineto{\pgfqpoint{0.638565in}{3.367544in}}%
\pgfpathlineto{\pgfqpoint{0.634712in}{3.365650in}}%
\pgfpathlineto{\pgfqpoint{0.633155in}{3.357895in}}%
\pgfpathlineto{\pgfqpoint{0.625000in}{3.353607in}}%
\pgfusepath{stroke}%
\end{pgfscope}%
\begin{pgfscope}%
\pgfpathrectangle{\pgfqpoint{0.625000in}{0.550000in}}{\pgfqpoint{3.875000in}{3.850000in}} %
\pgfusepath{clip}%
\pgfsetbuttcap%
\pgfsetroundjoin%
\pgfsetlinewidth{0.501875pt}%
\definecolor{currentstroke}{rgb}{0.000000,0.000000,0.000000}%
\pgfsetstrokecolor{currentstroke}%
\pgfsetdash{}{0pt}%
\pgfpathmoveto{\pgfqpoint{0.625000in}{3.398586in}}%
\pgfpathlineto{\pgfqpoint{0.632681in}{3.396491in}}%
\pgfpathlineto{\pgfqpoint{0.625000in}{3.393886in}}%
\pgfusepath{stroke}%
\end{pgfscope}%
\begin{pgfscope}%
\pgfpathrectangle{\pgfqpoint{0.625000in}{0.550000in}}{\pgfqpoint{3.875000in}{3.850000in}} %
\pgfusepath{clip}%
\pgfsetbuttcap%
\pgfsetroundjoin%
\pgfsetlinewidth{0.501875pt}%
\definecolor{currentstroke}{rgb}{0.000000,0.000000,0.000000}%
\pgfsetstrokecolor{currentstroke}%
\pgfsetdash{}{0pt}%
\pgfpathmoveto{\pgfqpoint{0.625000in}{3.524411in}}%
\pgfpathlineto{\pgfqpoint{0.630499in}{3.531579in}}%
\pgfpathlineto{\pgfqpoint{0.625000in}{3.538747in}}%
\pgfusepath{stroke}%
\end{pgfscope}%
\begin{pgfscope}%
\pgfpathrectangle{\pgfqpoint{0.625000in}{0.550000in}}{\pgfqpoint{3.875000in}{3.850000in}} %
\pgfusepath{clip}%
\pgfsetbuttcap%
\pgfsetroundjoin%
\pgfsetlinewidth{0.501875pt}%
\definecolor{currentstroke}{rgb}{0.000000,0.000000,0.000000}%
\pgfsetstrokecolor{currentstroke}%
\pgfsetdash{}{0pt}%
\pgfpathmoveto{\pgfqpoint{0.625000in}{3.572648in}}%
\pgfpathlineto{\pgfqpoint{0.630091in}{3.579825in}}%
\pgfpathlineto{\pgfqpoint{0.634712in}{3.588903in}}%
\pgfpathlineto{\pgfqpoint{0.635337in}{3.589474in}}%
\pgfpathlineto{\pgfqpoint{0.634712in}{3.591104in}}%
\pgfpathlineto{\pgfqpoint{0.625000in}{3.596127in}}%
\pgfusepath{stroke}%
\end{pgfscope}%
\begin{pgfscope}%
\pgfpathrectangle{\pgfqpoint{0.625000in}{0.550000in}}{\pgfqpoint{3.875000in}{3.850000in}} %
\pgfusepath{clip}%
\pgfsetbuttcap%
\pgfsetroundjoin%
\pgfsetlinewidth{0.501875pt}%
\definecolor{currentstroke}{rgb}{0.000000,0.000000,0.000000}%
\pgfsetstrokecolor{currentstroke}%
\pgfsetdash{}{0pt}%
\pgfpathmoveto{\pgfqpoint{0.625000in}{3.625748in}}%
\pgfpathlineto{\pgfqpoint{0.626552in}{3.628070in}}%
\pgfpathlineto{\pgfqpoint{0.625000in}{3.629224in}}%
\pgfusepath{stroke}%
\end{pgfscope}%
\begin{pgfscope}%
\pgfpathrectangle{\pgfqpoint{0.625000in}{0.550000in}}{\pgfqpoint{3.875000in}{3.850000in}} %
\pgfusepath{clip}%
\pgfsetbuttcap%
\pgfsetroundjoin%
\pgfsetlinewidth{0.501875pt}%
\definecolor{currentstroke}{rgb}{0.000000,0.000000,0.000000}%
\pgfsetstrokecolor{currentstroke}%
\pgfsetdash{}{0pt}%
\pgfpathmoveto{\pgfqpoint{0.625000in}{3.689781in}}%
\pgfpathlineto{\pgfqpoint{0.633038in}{3.695614in}}%
\pgfpathlineto{\pgfqpoint{0.634712in}{3.703360in}}%
\pgfpathlineto{\pgfqpoint{0.641207in}{3.705263in}}%
\pgfpathlineto{\pgfqpoint{0.644424in}{3.705879in}}%
\pgfpathlineto{\pgfqpoint{0.654135in}{3.712815in}}%
\pgfpathlineto{\pgfqpoint{0.655884in}{3.714912in}}%
\pgfpathlineto{\pgfqpoint{0.661084in}{3.724561in}}%
\pgfpathlineto{\pgfqpoint{0.662416in}{3.734211in}}%
\pgfpathlineto{\pgfqpoint{0.660855in}{3.743860in}}%
\pgfpathlineto{\pgfqpoint{0.654765in}{3.753509in}}%
\pgfpathlineto{\pgfqpoint{0.654135in}{3.753876in}}%
\pgfpathlineto{\pgfqpoint{0.644424in}{3.760748in}}%
\pgfpathlineto{\pgfqpoint{0.638472in}{3.763158in}}%
\pgfpathlineto{\pgfqpoint{0.644424in}{3.765670in}}%
\pgfpathlineto{\pgfqpoint{0.654135in}{3.771154in}}%
\pgfpathlineto{\pgfqpoint{0.659440in}{3.772807in}}%
\pgfpathlineto{\pgfqpoint{0.663847in}{3.775806in}}%
\pgfpathlineto{\pgfqpoint{0.673559in}{3.781765in}}%
\pgfpathlineto{\pgfqpoint{0.674827in}{3.782456in}}%
\pgfpathlineto{\pgfqpoint{0.683271in}{3.791365in}}%
\pgfpathlineto{\pgfqpoint{0.684125in}{3.792105in}}%
\pgfpathlineto{\pgfqpoint{0.690219in}{3.801754in}}%
\pgfpathlineto{\pgfqpoint{0.692982in}{3.810623in}}%
\pgfpathlineto{\pgfqpoint{0.693318in}{3.811404in}}%
\pgfpathlineto{\pgfqpoint{0.694716in}{3.821053in}}%
\pgfpathlineto{\pgfqpoint{0.693950in}{3.830702in}}%
\pgfpathlineto{\pgfqpoint{0.692982in}{3.834176in}}%
\pgfpathlineto{\pgfqpoint{0.691174in}{3.840351in}}%
\pgfpathlineto{\pgfqpoint{0.685864in}{3.850000in}}%
\pgfpathlineto{\pgfqpoint{0.683271in}{3.853320in}}%
\pgfpathlineto{\pgfqpoint{0.676822in}{3.859649in}}%
\pgfpathlineto{\pgfqpoint{0.673559in}{3.862323in}}%
\pgfpathlineto{\pgfqpoint{0.663847in}{3.867531in}}%
\pgfpathlineto{\pgfqpoint{0.654135in}{3.869144in}}%
\pgfpathlineto{\pgfqpoint{0.644424in}{3.866964in}}%
\pgfpathlineto{\pgfqpoint{0.634712in}{3.862986in}}%
\pgfpathlineto{\pgfqpoint{0.632667in}{3.869298in}}%
\pgfpathlineto{\pgfqpoint{0.625000in}{3.874316in}}%
\pgfusepath{stroke}%
\end{pgfscope}%
\begin{pgfscope}%
\pgfpathrectangle{\pgfqpoint{0.625000in}{0.550000in}}{\pgfqpoint{3.875000in}{3.850000in}} %
\pgfusepath{clip}%
\pgfsetbuttcap%
\pgfsetroundjoin%
\pgfsetlinewidth{0.501875pt}%
\definecolor{currentstroke}{rgb}{0.000000,0.000000,0.000000}%
\pgfsetstrokecolor{currentstroke}%
\pgfsetdash{}{0pt}%
\pgfpathmoveto{\pgfqpoint{0.625000in}{3.710375in}}%
\pgfpathlineto{\pgfqpoint{0.634058in}{3.705263in}}%
\pgfpathlineto{\pgfqpoint{0.625000in}{3.701261in}}%
\pgfusepath{stroke}%
\end{pgfscope}%
\begin{pgfscope}%
\pgfpathrectangle{\pgfqpoint{0.625000in}{0.550000in}}{\pgfqpoint{3.875000in}{3.850000in}} %
\pgfusepath{clip}%
\pgfsetbuttcap%
\pgfsetroundjoin%
\pgfsetlinewidth{0.501875pt}%
\definecolor{currentstroke}{rgb}{0.000000,0.000000,0.000000}%
\pgfsetstrokecolor{currentstroke}%
\pgfsetdash{}{0pt}%
\pgfpathmoveto{\pgfqpoint{0.625000in}{3.726227in}}%
\pgfpathlineto{\pgfqpoint{0.626363in}{3.724561in}}%
\pgfpathlineto{\pgfqpoint{0.625000in}{3.722158in}}%
\pgfusepath{stroke}%
\end{pgfscope}%
\begin{pgfscope}%
\pgfpathrectangle{\pgfqpoint{0.625000in}{0.550000in}}{\pgfqpoint{3.875000in}{3.850000in}} %
\pgfusepath{clip}%
\pgfsetbuttcap%
\pgfsetroundjoin%
\pgfsetlinewidth{0.501875pt}%
\definecolor{currentstroke}{rgb}{0.000000,0.000000,0.000000}%
\pgfsetstrokecolor{currentstroke}%
\pgfsetdash{}{0pt}%
\pgfpathmoveto{\pgfqpoint{0.625000in}{3.776387in}}%
\pgfpathlineto{\pgfqpoint{0.629703in}{3.772807in}}%
\pgfpathlineto{\pgfqpoint{0.631405in}{3.763158in}}%
\pgfpathlineto{\pgfqpoint{0.634712in}{3.757639in}}%
\pgfpathlineto{\pgfqpoint{0.636032in}{3.753509in}}%
\pgfpathlineto{\pgfqpoint{0.636303in}{3.743860in}}%
\pgfpathlineto{\pgfqpoint{0.634712in}{3.741994in}}%
\pgfpathlineto{\pgfqpoint{0.625000in}{3.739488in}}%
\pgfusepath{stroke}%
\end{pgfscope}%
\begin{pgfscope}%
\pgfpathrectangle{\pgfqpoint{0.625000in}{0.550000in}}{\pgfqpoint{3.875000in}{3.850000in}} %
\pgfusepath{clip}%
\pgfsetbuttcap%
\pgfsetroundjoin%
\pgfsetlinewidth{0.501875pt}%
\definecolor{currentstroke}{rgb}{0.000000,0.000000,0.000000}%
\pgfsetstrokecolor{currentstroke}%
\pgfsetdash{}{0pt}%
\pgfpathmoveto{\pgfqpoint{0.625000in}{3.799673in}}%
\pgfpathlineto{\pgfqpoint{0.634712in}{3.797429in}}%
\pgfpathlineto{\pgfqpoint{0.639868in}{3.792105in}}%
\pgfpathlineto{\pgfqpoint{0.634712in}{3.783070in}}%
\pgfpathlineto{\pgfqpoint{0.625000in}{3.786516in}}%
\pgfusepath{stroke}%
\end{pgfscope}%
\begin{pgfscope}%
\pgfpathrectangle{\pgfqpoint{0.625000in}{0.550000in}}{\pgfqpoint{3.875000in}{3.850000in}} %
\pgfusepath{clip}%
\pgfsetbuttcap%
\pgfsetroundjoin%
\pgfsetlinewidth{0.501875pt}%
\definecolor{currentstroke}{rgb}{0.000000,0.000000,0.000000}%
\pgfsetstrokecolor{currentstroke}%
\pgfsetdash{}{0pt}%
\pgfpathmoveto{\pgfqpoint{0.625000in}{3.824354in}}%
\pgfpathlineto{\pgfqpoint{0.627541in}{3.821053in}}%
\pgfpathlineto{\pgfqpoint{0.627755in}{3.811404in}}%
\pgfpathlineto{\pgfqpoint{0.625000in}{3.805526in}}%
\pgfusepath{stroke}%
\end{pgfscope}%
\begin{pgfscope}%
\pgfpathrectangle{\pgfqpoint{0.625000in}{0.550000in}}{\pgfqpoint{3.875000in}{3.850000in}} %
\pgfusepath{clip}%
\pgfsetbuttcap%
\pgfsetroundjoin%
\pgfsetlinewidth{0.501875pt}%
\definecolor{currentstroke}{rgb}{0.000000,0.000000,0.000000}%
\pgfsetstrokecolor{currentstroke}%
\pgfsetdash{}{0pt}%
\pgfpathmoveto{\pgfqpoint{0.625000in}{3.844872in}}%
\pgfpathlineto{\pgfqpoint{0.630526in}{3.840351in}}%
\pgfpathlineto{\pgfqpoint{0.625000in}{3.835830in}}%
\pgfusepath{stroke}%
\end{pgfscope}%
\begin{pgfscope}%
\pgfpathrectangle{\pgfqpoint{0.625000in}{0.550000in}}{\pgfqpoint{3.875000in}{3.850000in}} %
\pgfusepath{clip}%
\pgfsetbuttcap%
\pgfsetroundjoin%
\pgfsetlinewidth{0.501875pt}%
\definecolor{currentstroke}{rgb}{0.000000,0.000000,0.000000}%
\pgfsetstrokecolor{currentstroke}%
\pgfsetdash{}{0pt}%
\pgfpathmoveto{\pgfqpoint{0.625000in}{3.863608in}}%
\pgfpathlineto{\pgfqpoint{0.633075in}{3.859649in}}%
\pgfpathlineto{\pgfqpoint{0.625000in}{3.855919in}}%
\pgfusepath{stroke}%
\end{pgfscope}%
\begin{pgfscope}%
\pgfpathrectangle{\pgfqpoint{0.625000in}{0.550000in}}{\pgfqpoint{3.875000in}{3.850000in}} %
\pgfusepath{clip}%
\pgfsetbuttcap%
\pgfsetroundjoin%
\pgfsetlinewidth{0.501875pt}%
\definecolor{currentstroke}{rgb}{0.000000,0.000000,0.000000}%
\pgfsetstrokecolor{currentstroke}%
\pgfsetdash{}{0pt}%
\pgfpathmoveto{\pgfqpoint{0.625000in}{3.883341in}}%
\pgfpathlineto{\pgfqpoint{0.634712in}{3.885103in}}%
\pgfpathlineto{\pgfqpoint{0.642127in}{3.888596in}}%
\pgfpathlineto{\pgfqpoint{0.637338in}{3.898246in}}%
\pgfpathlineto{\pgfqpoint{0.634712in}{3.903108in}}%
\pgfpathlineto{\pgfqpoint{0.625000in}{3.901448in}}%
\pgfusepath{stroke}%
\end{pgfscope}%
\begin{pgfscope}%
\pgfpathrectangle{\pgfqpoint{0.625000in}{0.550000in}}{\pgfqpoint{3.875000in}{3.850000in}} %
\pgfusepath{clip}%
\pgfsetbuttcap%
\pgfsetroundjoin%
\pgfsetlinewidth{0.501875pt}%
\definecolor{currentstroke}{rgb}{0.000000,0.000000,0.000000}%
\pgfsetstrokecolor{currentstroke}%
\pgfsetdash{}{0pt}%
\pgfpathmoveto{\pgfqpoint{0.625000in}{4.000182in}}%
\pgfpathlineto{\pgfqpoint{0.634304in}{4.004386in}}%
\pgfpathlineto{\pgfqpoint{0.634712in}{4.004912in}}%
\pgfpathlineto{\pgfqpoint{0.638915in}{4.014035in}}%
\pgfpathlineto{\pgfqpoint{0.634712in}{4.018860in}}%
\pgfpathlineto{\pgfqpoint{0.625392in}{4.023684in}}%
\pgfpathlineto{\pgfqpoint{0.625000in}{4.023903in}}%
\pgfusepath{stroke}%
\end{pgfscope}%
\begin{pgfscope}%
\pgfpathrectangle{\pgfqpoint{0.625000in}{0.550000in}}{\pgfqpoint{3.875000in}{3.850000in}} %
\pgfusepath{clip}%
\pgfsetbuttcap%
\pgfsetroundjoin%
\pgfsetlinewidth{0.501875pt}%
\definecolor{currentstroke}{rgb}{0.000000,0.000000,0.000000}%
\pgfsetstrokecolor{currentstroke}%
\pgfsetdash{}{0pt}%
\pgfpathmoveto{\pgfqpoint{0.625000in}{4.038925in}}%
\pgfpathlineto{\pgfqpoint{0.629556in}{4.033333in}}%
\pgfpathlineto{\pgfqpoint{0.634712in}{4.026845in}}%
\pgfpathlineto{\pgfqpoint{0.644424in}{4.030709in}}%
\pgfpathlineto{\pgfqpoint{0.649279in}{4.033333in}}%
\pgfpathlineto{\pgfqpoint{0.652723in}{4.042982in}}%
\pgfpathlineto{\pgfqpoint{0.651209in}{4.052632in}}%
\pgfpathlineto{\pgfqpoint{0.644424in}{4.060776in}}%
\pgfpathlineto{\pgfqpoint{0.638105in}{4.062281in}}%
\pgfpathlineto{\pgfqpoint{0.634712in}{4.064071in}}%
\pgfpathlineto{\pgfqpoint{0.634377in}{4.062281in}}%
\pgfpathlineto{\pgfqpoint{0.630083in}{4.052632in}}%
\pgfpathlineto{\pgfqpoint{0.625000in}{4.049081in}}%
\pgfusepath{stroke}%
\end{pgfscope}%
\begin{pgfscope}%
\pgfpathrectangle{\pgfqpoint{0.625000in}{0.550000in}}{\pgfqpoint{3.875000in}{3.850000in}} %
\pgfusepath{clip}%
\pgfsetbuttcap%
\pgfsetroundjoin%
\pgfsetlinewidth{0.501875pt}%
\definecolor{currentstroke}{rgb}{0.000000,0.000000,0.000000}%
\pgfsetstrokecolor{currentstroke}%
\pgfsetdash{}{0pt}%
\pgfpathmoveto{\pgfqpoint{0.625000in}{4.066942in}}%
\pgfpathlineto{\pgfqpoint{0.633470in}{4.071930in}}%
\pgfpathlineto{\pgfqpoint{0.625000in}{4.076917in}}%
\pgfusepath{stroke}%
\end{pgfscope}%
\begin{pgfscope}%
\pgfpathrectangle{\pgfqpoint{0.625000in}{0.550000in}}{\pgfqpoint{3.875000in}{3.850000in}} %
\pgfusepath{clip}%
\pgfsetbuttcap%
\pgfsetroundjoin%
\pgfsetlinewidth{0.501875pt}%
\definecolor{currentstroke}{rgb}{0.000000,0.000000,0.000000}%
\pgfsetstrokecolor{currentstroke}%
\pgfsetdash{}{0pt}%
\pgfpathmoveto{\pgfqpoint{0.625000in}{4.144140in}}%
\pgfpathlineto{\pgfqpoint{0.630669in}{4.139474in}}%
\pgfpathlineto{\pgfqpoint{0.634712in}{4.136788in}}%
\pgfpathlineto{\pgfqpoint{0.639586in}{4.139474in}}%
\pgfpathlineto{\pgfqpoint{0.639493in}{4.149123in}}%
\pgfpathlineto{\pgfqpoint{0.634712in}{4.154661in}}%
\pgfpathlineto{\pgfqpoint{0.633178in}{4.158772in}}%
\pgfpathlineto{\pgfqpoint{0.625000in}{4.164725in}}%
\pgfusepath{stroke}%
\end{pgfscope}%
\begin{pgfscope}%
\pgfpathrectangle{\pgfqpoint{0.625000in}{0.550000in}}{\pgfqpoint{3.875000in}{3.850000in}} %
\pgfusepath{clip}%
\pgfsetbuttcap%
\pgfsetroundjoin%
\pgfsetlinewidth{0.501875pt}%
\definecolor{currentstroke}{rgb}{0.000000,0.000000,0.000000}%
\pgfsetstrokecolor{currentstroke}%
\pgfsetdash{}{0pt}%
\pgfpathmoveto{\pgfqpoint{0.625000in}{4.173776in}}%
\pgfpathlineto{\pgfqpoint{0.628197in}{4.178070in}}%
\pgfpathlineto{\pgfqpoint{0.627115in}{4.187719in}}%
\pgfpathlineto{\pgfqpoint{0.632028in}{4.197368in}}%
\pgfpathlineto{\pgfqpoint{0.625000in}{4.203547in}}%
\pgfusepath{stroke}%
\end{pgfscope}%
\begin{pgfscope}%
\pgfpathrectangle{\pgfqpoint{0.625000in}{0.550000in}}{\pgfqpoint{3.875000in}{3.850000in}} %
\pgfusepath{clip}%
\pgfsetbuttcap%
\pgfsetroundjoin%
\pgfsetlinewidth{0.501875pt}%
\definecolor{currentstroke}{rgb}{0.000000,0.000000,0.000000}%
\pgfsetstrokecolor{currentstroke}%
\pgfsetdash{}{0pt}%
\pgfpathmoveto{\pgfqpoint{0.625000in}{4.261835in}}%
\pgfpathlineto{\pgfqpoint{0.630370in}{4.264912in}}%
\pgfpathlineto{\pgfqpoint{0.625000in}{4.269950in}}%
\pgfusepath{stroke}%
\end{pgfscope}%
\begin{pgfscope}%
\pgfpathrectangle{\pgfqpoint{0.625000in}{0.550000in}}{\pgfqpoint{3.875000in}{3.850000in}} %
\pgfusepath{clip}%
\pgfsetbuttcap%
\pgfsetroundjoin%
\pgfsetlinewidth{0.501875pt}%
\definecolor{currentstroke}{rgb}{0.000000,0.000000,0.000000}%
\pgfsetstrokecolor{currentstroke}%
\pgfsetdash{}{0pt}%
\pgfpathmoveto{\pgfqpoint{0.625000in}{4.279866in}}%
\pgfpathlineto{\pgfqpoint{0.629383in}{4.284211in}}%
\pgfpathlineto{\pgfqpoint{0.625000in}{4.288555in}}%
\pgfusepath{stroke}%
\end{pgfscope}%
\begin{pgfscope}%
\pgfpathrectangle{\pgfqpoint{0.625000in}{0.550000in}}{\pgfqpoint{3.875000in}{3.850000in}} %
\pgfusepath{clip}%
\pgfsetbuttcap%
\pgfsetroundjoin%
\pgfsetlinewidth{0.501875pt}%
\definecolor{currentstroke}{rgb}{0.000000,0.000000,0.000000}%
\pgfsetstrokecolor{currentstroke}%
\pgfsetdash{}{0pt}%
\pgfpathmoveto{\pgfqpoint{0.625000in}{4.346718in}}%
\pgfpathlineto{\pgfqpoint{0.630767in}{4.351754in}}%
\pgfpathlineto{\pgfqpoint{0.625000in}{4.357642in}}%
\pgfusepath{stroke}%
\end{pgfscope}%
\begin{pgfscope}%
\pgfpathrectangle{\pgfqpoint{0.625000in}{0.550000in}}{\pgfqpoint{3.875000in}{3.850000in}} %
\pgfusepath{clip}%
\pgfsetbuttcap%
\pgfsetroundjoin%
\pgfsetlinewidth{0.501875pt}%
\definecolor{currentstroke}{rgb}{0.000000,0.000000,0.000000}%
\pgfsetstrokecolor{currentstroke}%
\pgfsetdash{}{0pt}%
\pgfpathmoveto{\pgfqpoint{0.663847in}{2.420703in}}%
\pgfpathlineto{\pgfqpoint{0.659116in}{2.421930in}}%
\pgfpathlineto{\pgfqpoint{0.658871in}{2.431579in}}%
\pgfpathlineto{\pgfqpoint{0.654135in}{2.437012in}}%
\pgfpathlineto{\pgfqpoint{0.644583in}{2.431579in}}%
\pgfpathlineto{\pgfqpoint{0.644424in}{2.431354in}}%
\pgfpathlineto{\pgfqpoint{0.644071in}{2.431579in}}%
\pgfpathlineto{\pgfqpoint{0.635028in}{2.441228in}}%
\pgfpathlineto{\pgfqpoint{0.644424in}{2.446935in}}%
\pgfpathlineto{\pgfqpoint{0.648702in}{2.450877in}}%
\pgfpathlineto{\pgfqpoint{0.644424in}{2.455158in}}%
\pgfpathlineto{\pgfqpoint{0.634712in}{2.456601in}}%
\pgfpathlineto{\pgfqpoint{0.630401in}{2.460526in}}%
\pgfpathlineto{\pgfqpoint{0.630720in}{2.470175in}}%
\pgfpathlineto{\pgfqpoint{0.634712in}{2.478371in}}%
\pgfpathlineto{\pgfqpoint{0.644424in}{2.478321in}}%
\pgfpathlineto{\pgfqpoint{0.654135in}{2.478170in}}%
\pgfpathlineto{\pgfqpoint{0.659915in}{2.470175in}}%
\pgfpathlineto{\pgfqpoint{0.657039in}{2.460526in}}%
\pgfpathlineto{\pgfqpoint{0.663847in}{2.454806in}}%
\pgfpathlineto{\pgfqpoint{0.673559in}{2.454640in}}%
\pgfpathlineto{\pgfqpoint{0.676176in}{2.450877in}}%
\pgfpathlineto{\pgfqpoint{0.673559in}{2.448454in}}%
\pgfpathlineto{\pgfqpoint{0.668548in}{2.441228in}}%
\pgfpathlineto{\pgfqpoint{0.668439in}{2.431579in}}%
\pgfpathlineto{\pgfqpoint{0.665096in}{2.421930in}}%
\pgfpathlineto{\pgfqpoint{0.663847in}{2.420703in}}%
\pgfusepath{stroke}%
\end{pgfscope}%
\begin{pgfscope}%
\pgfpathrectangle{\pgfqpoint{0.625000in}{0.550000in}}{\pgfqpoint{3.875000in}{3.850000in}} %
\pgfusepath{clip}%
\pgfsetbuttcap%
\pgfsetroundjoin%
\pgfsetlinewidth{0.501875pt}%
\definecolor{currentstroke}{rgb}{0.000000,0.000000,0.000000}%
\pgfsetstrokecolor{currentstroke}%
\pgfsetdash{}{0pt}%
\pgfpathmoveto{\pgfqpoint{0.634712in}{2.495507in}}%
\pgfpathlineto{\pgfqpoint{0.630154in}{2.499123in}}%
\pgfpathlineto{\pgfqpoint{0.634712in}{2.503289in}}%
\pgfpathlineto{\pgfqpoint{0.644424in}{2.502940in}}%
\pgfpathlineto{\pgfqpoint{0.648702in}{2.508772in}}%
\pgfpathlineto{\pgfqpoint{0.644424in}{2.512714in}}%
\pgfpathlineto{\pgfqpoint{0.635028in}{2.518421in}}%
\pgfpathlineto{\pgfqpoint{0.644071in}{2.528070in}}%
\pgfpathlineto{\pgfqpoint{0.644424in}{2.528296in}}%
\pgfpathlineto{\pgfqpoint{0.644583in}{2.528070in}}%
\pgfpathlineto{\pgfqpoint{0.654135in}{2.522637in}}%
\pgfpathlineto{\pgfqpoint{0.658871in}{2.528070in}}%
\pgfpathlineto{\pgfqpoint{0.659116in}{2.537719in}}%
\pgfpathlineto{\pgfqpoint{0.663847in}{2.538947in}}%
\pgfpathlineto{\pgfqpoint{0.665096in}{2.537719in}}%
\pgfpathlineto{\pgfqpoint{0.668439in}{2.528070in}}%
\pgfpathlineto{\pgfqpoint{0.668548in}{2.518421in}}%
\pgfpathlineto{\pgfqpoint{0.673559in}{2.511195in}}%
\pgfpathlineto{\pgfqpoint{0.676176in}{2.508772in}}%
\pgfpathlineto{\pgfqpoint{0.673559in}{2.505009in}}%
\pgfpathlineto{\pgfqpoint{0.663847in}{2.504831in}}%
\pgfpathlineto{\pgfqpoint{0.657866in}{2.499123in}}%
\pgfpathlineto{\pgfqpoint{0.654135in}{2.495376in}}%
\pgfpathlineto{\pgfqpoint{0.644424in}{2.497305in}}%
\pgfpathlineto{\pgfqpoint{0.634712in}{2.495507in}}%
\pgfusepath{stroke}%
\end{pgfscope}%
\begin{pgfscope}%
\pgfpathrectangle{\pgfqpoint{0.625000in}{0.550000in}}{\pgfqpoint{3.875000in}{3.850000in}} %
\pgfusepath{clip}%
\pgfsetbuttcap%
\pgfsetroundjoin%
\pgfsetlinewidth{0.501875pt}%
\definecolor{currentstroke}{rgb}{0.000000,0.000000,0.000000}%
\pgfsetstrokecolor{currentstroke}%
\pgfsetdash{}{0pt}%
\pgfpathmoveto{\pgfqpoint{0.625000in}{0.604362in}}%
\pgfpathlineto{\pgfqpoint{0.628460in}{0.607895in}}%
\pgfpathlineto{\pgfqpoint{0.625000in}{0.610916in}}%
\pgfusepath{stroke}%
\end{pgfscope}%
\begin{pgfscope}%
\pgfpathrectangle{\pgfqpoint{0.625000in}{0.550000in}}{\pgfqpoint{3.875000in}{3.850000in}} %
\pgfusepath{clip}%
\pgfsetbuttcap%
\pgfsetroundjoin%
\pgfsetlinewidth{0.501875pt}%
\definecolor{currentstroke}{rgb}{0.000000,0.000000,0.000000}%
\pgfsetstrokecolor{currentstroke}%
\pgfsetdash{}{0pt}%
\pgfpathmoveto{\pgfqpoint{0.625000in}{0.622545in}}%
\pgfpathlineto{\pgfqpoint{0.630194in}{0.627193in}}%
\pgfpathlineto{\pgfqpoint{0.625000in}{0.631767in}}%
\pgfusepath{stroke}%
\end{pgfscope}%
\begin{pgfscope}%
\pgfpathrectangle{\pgfqpoint{0.625000in}{0.550000in}}{\pgfqpoint{3.875000in}{3.850000in}} %
\pgfusepath{clip}%
\pgfsetbuttcap%
\pgfsetroundjoin%
\pgfsetlinewidth{0.501875pt}%
\definecolor{currentstroke}{rgb}{0.000000,0.000000,0.000000}%
\pgfsetstrokecolor{currentstroke}%
\pgfsetdash{}{0pt}%
\pgfpathmoveto{\pgfqpoint{0.625000in}{0.674630in}}%
\pgfpathlineto{\pgfqpoint{0.625815in}{0.675439in}}%
\pgfpathlineto{\pgfqpoint{0.625000in}{0.676247in}}%
\pgfusepath{stroke}%
\end{pgfscope}%
\begin{pgfscope}%
\pgfpathrectangle{\pgfqpoint{0.625000in}{0.550000in}}{\pgfqpoint{3.875000in}{3.850000in}} %
\pgfusepath{clip}%
\pgfsetbuttcap%
\pgfsetroundjoin%
\pgfsetlinewidth{0.501875pt}%
\definecolor{currentstroke}{rgb}{0.000000,0.000000,0.000000}%
\pgfsetstrokecolor{currentstroke}%
\pgfsetdash{}{0pt}%
\pgfpathmoveto{\pgfqpoint{0.625000in}{0.692773in}}%
\pgfpathlineto{\pgfqpoint{0.627093in}{0.694737in}}%
\pgfpathlineto{\pgfqpoint{0.625000in}{0.695936in}}%
\pgfusepath{stroke}%
\end{pgfscope}%
\begin{pgfscope}%
\pgfpathrectangle{\pgfqpoint{0.625000in}{0.550000in}}{\pgfqpoint{3.875000in}{3.850000in}} %
\pgfusepath{clip}%
\pgfsetbuttcap%
\pgfsetroundjoin%
\pgfsetlinewidth{0.501875pt}%
\definecolor{currentstroke}{rgb}{0.000000,0.000000,0.000000}%
\pgfsetstrokecolor{currentstroke}%
\pgfsetdash{}{0pt}%
\pgfpathmoveto{\pgfqpoint{0.625000in}{0.758573in}}%
\pgfpathlineto{\pgfqpoint{0.629217in}{0.762281in}}%
\pgfpathlineto{\pgfqpoint{0.625000in}{0.771391in}}%
\pgfusepath{stroke}%
\end{pgfscope}%
\begin{pgfscope}%
\pgfpathrectangle{\pgfqpoint{0.625000in}{0.550000in}}{\pgfqpoint{3.875000in}{3.850000in}} %
\pgfusepath{clip}%
\pgfsetbuttcap%
\pgfsetroundjoin%
\pgfsetlinewidth{0.501875pt}%
\definecolor{currentstroke}{rgb}{0.000000,0.000000,0.000000}%
\pgfsetstrokecolor{currentstroke}%
\pgfsetdash{}{0pt}%
\pgfpathmoveto{\pgfqpoint{0.625000in}{0.797306in}}%
\pgfpathlineto{\pgfqpoint{0.629907in}{0.800877in}}%
\pgfpathlineto{\pgfqpoint{0.631025in}{0.810526in}}%
\pgfpathlineto{\pgfqpoint{0.625000in}{0.812398in}}%
\pgfusepath{stroke}%
\end{pgfscope}%
\begin{pgfscope}%
\pgfpathrectangle{\pgfqpoint{0.625000in}{0.550000in}}{\pgfqpoint{3.875000in}{3.850000in}} %
\pgfusepath{clip}%
\pgfsetbuttcap%
\pgfsetroundjoin%
\pgfsetlinewidth{0.501875pt}%
\definecolor{currentstroke}{rgb}{0.000000,0.000000,0.000000}%
\pgfsetstrokecolor{currentstroke}%
\pgfsetdash{}{0pt}%
\pgfpathmoveto{\pgfqpoint{0.625000in}{0.885840in}}%
\pgfpathlineto{\pgfqpoint{0.628192in}{0.887719in}}%
\pgfpathlineto{\pgfqpoint{0.625000in}{0.889599in}}%
\pgfusepath{stroke}%
\end{pgfscope}%
\begin{pgfscope}%
\pgfpathrectangle{\pgfqpoint{0.625000in}{0.550000in}}{\pgfqpoint{3.875000in}{3.850000in}} %
\pgfusepath{clip}%
\pgfsetbuttcap%
\pgfsetroundjoin%
\pgfsetlinewidth{0.501875pt}%
\definecolor{currentstroke}{rgb}{0.000000,0.000000,0.000000}%
\pgfsetstrokecolor{currentstroke}%
\pgfsetdash{}{0pt}%
\pgfpathmoveto{\pgfqpoint{0.625000in}{0.913008in}}%
\pgfpathlineto{\pgfqpoint{0.633575in}{0.907018in}}%
\pgfpathlineto{\pgfqpoint{0.634712in}{0.904548in}}%
\pgfpathlineto{\pgfqpoint{0.638737in}{0.907018in}}%
\pgfpathlineto{\pgfqpoint{0.644424in}{0.914943in}}%
\pgfpathlineto{\pgfqpoint{0.645178in}{0.916667in}}%
\pgfpathlineto{\pgfqpoint{0.644424in}{0.919139in}}%
\pgfpathlineto{\pgfqpoint{0.643540in}{0.926316in}}%
\pgfpathlineto{\pgfqpoint{0.634712in}{0.931140in}}%
\pgfpathlineto{\pgfqpoint{0.630878in}{0.926316in}}%
\pgfpathlineto{\pgfqpoint{0.625000in}{0.919101in}}%
\pgfusepath{stroke}%
\end{pgfscope}%
\begin{pgfscope}%
\pgfpathrectangle{\pgfqpoint{0.625000in}{0.550000in}}{\pgfqpoint{3.875000in}{3.850000in}} %
\pgfusepath{clip}%
\pgfsetbuttcap%
\pgfsetroundjoin%
\pgfsetlinewidth{0.501875pt}%
\definecolor{currentstroke}{rgb}{0.000000,0.000000,0.000000}%
\pgfsetstrokecolor{currentstroke}%
\pgfsetdash{}{0pt}%
\pgfpathmoveto{\pgfqpoint{0.625000in}{0.948886in}}%
\pgfpathlineto{\pgfqpoint{0.629476in}{0.945614in}}%
\pgfpathlineto{\pgfqpoint{0.634712in}{0.943329in}}%
\pgfpathlineto{\pgfqpoint{0.636703in}{0.945614in}}%
\pgfpathlineto{\pgfqpoint{0.634712in}{0.949935in}}%
\pgfpathlineto{\pgfqpoint{0.630582in}{0.955263in}}%
\pgfpathlineto{\pgfqpoint{0.625000in}{0.957786in}}%
\pgfusepath{stroke}%
\end{pgfscope}%
\begin{pgfscope}%
\pgfpathrectangle{\pgfqpoint{0.625000in}{0.550000in}}{\pgfqpoint{3.875000in}{3.850000in}} %
\pgfusepath{clip}%
\pgfsetbuttcap%
\pgfsetroundjoin%
\pgfsetlinewidth{0.501875pt}%
\definecolor{currentstroke}{rgb}{0.000000,0.000000,0.000000}%
\pgfsetstrokecolor{currentstroke}%
\pgfsetdash{}{0pt}%
\pgfpathmoveto{\pgfqpoint{0.625000in}{1.062806in}}%
\pgfpathlineto{\pgfqpoint{0.632385in}{1.071053in}}%
\pgfpathlineto{\pgfqpoint{0.625000in}{1.074206in}}%
\pgfusepath{stroke}%
\end{pgfscope}%
\begin{pgfscope}%
\pgfpathrectangle{\pgfqpoint{0.625000in}{0.550000in}}{\pgfqpoint{3.875000in}{3.850000in}} %
\pgfusepath{clip}%
\pgfsetbuttcap%
\pgfsetroundjoin%
\pgfsetlinewidth{0.501875pt}%
\definecolor{currentstroke}{rgb}{0.000000,0.000000,0.000000}%
\pgfsetstrokecolor{currentstroke}%
\pgfsetdash{}{0pt}%
\pgfpathmoveto{\pgfqpoint{0.625000in}{1.106098in}}%
\pgfpathlineto{\pgfqpoint{0.634712in}{1.106082in}}%
\pgfpathlineto{\pgfqpoint{0.644424in}{1.109126in}}%
\pgfpathlineto{\pgfqpoint{0.646075in}{1.109649in}}%
\pgfpathlineto{\pgfqpoint{0.654135in}{1.112833in}}%
\pgfpathlineto{\pgfqpoint{0.663847in}{1.119253in}}%
\pgfpathlineto{\pgfqpoint{0.663896in}{1.119298in}}%
\pgfpathlineto{\pgfqpoint{0.672003in}{1.128947in}}%
\pgfpathlineto{\pgfqpoint{0.673559in}{1.132135in}}%
\pgfpathlineto{\pgfqpoint{0.676225in}{1.138596in}}%
\pgfpathlineto{\pgfqpoint{0.677700in}{1.148246in}}%
\pgfpathlineto{\pgfqpoint{0.676657in}{1.157895in}}%
\pgfpathlineto{\pgfqpoint{0.673559in}{1.164919in}}%
\pgfpathlineto{\pgfqpoint{0.672822in}{1.167544in}}%
\pgfpathlineto{\pgfqpoint{0.666104in}{1.177193in}}%
\pgfpathlineto{\pgfqpoint{0.663847in}{1.178629in}}%
\pgfpathlineto{\pgfqpoint{0.654135in}{1.185776in}}%
\pgfpathlineto{\pgfqpoint{0.653256in}{1.186842in}}%
\pgfpathlineto{\pgfqpoint{0.644424in}{1.192234in}}%
\pgfpathlineto{\pgfqpoint{0.634712in}{1.195224in}}%
\pgfpathlineto{\pgfqpoint{0.632838in}{1.186842in}}%
\pgfpathlineto{\pgfqpoint{0.634712in}{1.182915in}}%
\pgfpathlineto{\pgfqpoint{0.643228in}{1.177193in}}%
\pgfpathlineto{\pgfqpoint{0.642578in}{1.167544in}}%
\pgfpathlineto{\pgfqpoint{0.634712in}{1.159421in}}%
\pgfpathlineto{\pgfqpoint{0.625622in}{1.157895in}}%
\pgfpathlineto{\pgfqpoint{0.629592in}{1.148246in}}%
\pgfpathlineto{\pgfqpoint{0.629495in}{1.138596in}}%
\pgfpathlineto{\pgfqpoint{0.625000in}{1.132756in}}%
\pgfusepath{stroke}%
\end{pgfscope}%
\begin{pgfscope}%
\pgfpathrectangle{\pgfqpoint{0.625000in}{0.550000in}}{\pgfqpoint{3.875000in}{3.850000in}} %
\pgfusepath{clip}%
\pgfsetbuttcap%
\pgfsetroundjoin%
\pgfsetlinewidth{0.501875pt}%
\definecolor{currentstroke}{rgb}{0.000000,0.000000,0.000000}%
\pgfsetstrokecolor{currentstroke}%
\pgfsetdash{}{0pt}%
\pgfpathmoveto{\pgfqpoint{0.625000in}{1.125870in}}%
\pgfpathlineto{\pgfqpoint{0.633034in}{1.119298in}}%
\pgfpathlineto{\pgfqpoint{0.625000in}{1.112726in}}%
\pgfusepath{stroke}%
\end{pgfscope}%
\begin{pgfscope}%
\pgfpathrectangle{\pgfqpoint{0.625000in}{0.550000in}}{\pgfqpoint{3.875000in}{3.850000in}} %
\pgfusepath{clip}%
\pgfsetbuttcap%
\pgfsetroundjoin%
\pgfsetlinewidth{0.501875pt}%
\definecolor{currentstroke}{rgb}{0.000000,0.000000,0.000000}%
\pgfsetstrokecolor{currentstroke}%
\pgfsetdash{}{0pt}%
\pgfpathmoveto{\pgfqpoint{0.625000in}{1.174730in}}%
\pgfpathlineto{\pgfqpoint{0.631208in}{1.177193in}}%
\pgfpathlineto{\pgfqpoint{0.625000in}{1.180875in}}%
\pgfusepath{stroke}%
\end{pgfscope}%
\begin{pgfscope}%
\pgfpathrectangle{\pgfqpoint{0.625000in}{0.550000in}}{\pgfqpoint{3.875000in}{3.850000in}} %
\pgfusepath{clip}%
\pgfsetbuttcap%
\pgfsetroundjoin%
\pgfsetlinewidth{0.501875pt}%
\definecolor{currentstroke}{rgb}{0.000000,0.000000,0.000000}%
\pgfsetstrokecolor{currentstroke}%
\pgfsetdash{}{0pt}%
\pgfpathmoveto{\pgfqpoint{0.625000in}{1.222272in}}%
\pgfpathlineto{\pgfqpoint{0.634712in}{1.220065in}}%
\pgfpathlineto{\pgfqpoint{0.638358in}{1.215789in}}%
\pgfpathlineto{\pgfqpoint{0.637972in}{1.206140in}}%
\pgfpathlineto{\pgfqpoint{0.644424in}{1.200575in}}%
\pgfpathlineto{\pgfqpoint{0.652126in}{1.206140in}}%
\pgfpathlineto{\pgfqpoint{0.654135in}{1.212242in}}%
\pgfpathlineto{\pgfqpoint{0.656389in}{1.215789in}}%
\pgfpathlineto{\pgfqpoint{0.656768in}{1.225439in}}%
\pgfpathlineto{\pgfqpoint{0.654135in}{1.232996in}}%
\pgfpathlineto{\pgfqpoint{0.653173in}{1.235088in}}%
\pgfpathlineto{\pgfqpoint{0.644424in}{1.244189in}}%
\pgfpathlineto{\pgfqpoint{0.643155in}{1.244737in}}%
\pgfpathlineto{\pgfqpoint{0.634712in}{1.248117in}}%
\pgfpathlineto{\pgfqpoint{0.625000in}{1.248472in}}%
\pgfusepath{stroke}%
\end{pgfscope}%
\begin{pgfscope}%
\pgfpathrectangle{\pgfqpoint{0.625000in}{0.550000in}}{\pgfqpoint{3.875000in}{3.850000in}} %
\pgfusepath{clip}%
\pgfsetbuttcap%
\pgfsetroundjoin%
\pgfsetlinewidth{0.501875pt}%
\definecolor{currentstroke}{rgb}{0.000000,0.000000,0.000000}%
\pgfsetstrokecolor{currentstroke}%
\pgfsetdash{}{0pt}%
\pgfpathmoveto{\pgfqpoint{0.625000in}{1.239575in}}%
\pgfpathlineto{\pgfqpoint{0.628976in}{1.235088in}}%
\pgfpathlineto{\pgfqpoint{0.625000in}{1.230229in}}%
\pgfusepath{stroke}%
\end{pgfscope}%
\begin{pgfscope}%
\pgfpathrectangle{\pgfqpoint{0.625000in}{0.550000in}}{\pgfqpoint{3.875000in}{3.850000in}} %
\pgfusepath{clip}%
\pgfsetbuttcap%
\pgfsetroundjoin%
\pgfsetlinewidth{0.501875pt}%
\definecolor{currentstroke}{rgb}{0.000000,0.000000,0.000000}%
\pgfsetstrokecolor{currentstroke}%
\pgfsetdash{}{0pt}%
\pgfpathmoveto{\pgfqpoint{0.625000in}{1.260647in}}%
\pgfpathlineto{\pgfqpoint{0.629823in}{1.264035in}}%
\pgfpathlineto{\pgfqpoint{0.625000in}{1.267535in}}%
\pgfusepath{stroke}%
\end{pgfscope}%
\begin{pgfscope}%
\pgfpathrectangle{\pgfqpoint{0.625000in}{0.550000in}}{\pgfqpoint{3.875000in}{3.850000in}} %
\pgfusepath{clip}%
\pgfsetbuttcap%
\pgfsetroundjoin%
\pgfsetlinewidth{0.501875pt}%
\definecolor{currentstroke}{rgb}{0.000000,0.000000,0.000000}%
\pgfsetstrokecolor{currentstroke}%
\pgfsetdash{}{0pt}%
\pgfpathmoveto{\pgfqpoint{0.625000in}{1.365520in}}%
\pgfpathlineto{\pgfqpoint{0.632100in}{1.370175in}}%
\pgfpathlineto{\pgfqpoint{0.628054in}{1.379825in}}%
\pgfpathlineto{\pgfqpoint{0.625000in}{1.384131in}}%
\pgfusepath{stroke}%
\end{pgfscope}%
\begin{pgfscope}%
\pgfpathrectangle{\pgfqpoint{0.625000in}{0.550000in}}{\pgfqpoint{3.875000in}{3.850000in}} %
\pgfusepath{clip}%
\pgfsetbuttcap%
\pgfsetroundjoin%
\pgfsetlinewidth{0.501875pt}%
\definecolor{currentstroke}{rgb}{0.000000,0.000000,0.000000}%
\pgfsetstrokecolor{currentstroke}%
\pgfsetdash{}{0pt}%
\pgfpathmoveto{\pgfqpoint{0.625000in}{1.422557in}}%
\pgfpathlineto{\pgfqpoint{0.629230in}{1.428070in}}%
\pgfpathlineto{\pgfqpoint{0.625000in}{1.433584in}}%
\pgfusepath{stroke}%
\end{pgfscope}%
\begin{pgfscope}%
\pgfpathrectangle{\pgfqpoint{0.625000in}{0.550000in}}{\pgfqpoint{3.875000in}{3.850000in}} %
\pgfusepath{clip}%
\pgfsetbuttcap%
\pgfsetroundjoin%
\pgfsetlinewidth{0.501875pt}%
\definecolor{currentstroke}{rgb}{0.000000,0.000000,0.000000}%
\pgfsetstrokecolor{currentstroke}%
\pgfsetdash{}{0pt}%
\pgfpathmoveto{\pgfqpoint{0.625000in}{1.526987in}}%
\pgfpathlineto{\pgfqpoint{0.630624in}{1.534211in}}%
\pgfpathlineto{\pgfqpoint{0.631863in}{1.543860in}}%
\pgfpathlineto{\pgfqpoint{0.625000in}{1.550012in}}%
\pgfusepath{stroke}%
\end{pgfscope}%
\begin{pgfscope}%
\pgfpathrectangle{\pgfqpoint{0.625000in}{0.550000in}}{\pgfqpoint{3.875000in}{3.850000in}} %
\pgfusepath{clip}%
\pgfsetbuttcap%
\pgfsetroundjoin%
\pgfsetlinewidth{0.501875pt}%
\definecolor{currentstroke}{rgb}{0.000000,0.000000,0.000000}%
\pgfsetstrokecolor{currentstroke}%
\pgfsetdash{}{0pt}%
\pgfpathmoveto{\pgfqpoint{0.625000in}{1.568581in}}%
\pgfpathlineto{\pgfqpoint{0.631390in}{1.572807in}}%
\pgfpathlineto{\pgfqpoint{0.625000in}{1.575306in}}%
\pgfusepath{stroke}%
\end{pgfscope}%
\begin{pgfscope}%
\pgfpathrectangle{\pgfqpoint{0.625000in}{0.550000in}}{\pgfqpoint{3.875000in}{3.850000in}} %
\pgfusepath{clip}%
\pgfsetbuttcap%
\pgfsetroundjoin%
\pgfsetlinewidth{0.501875pt}%
\definecolor{currentstroke}{rgb}{0.000000,0.000000,0.000000}%
\pgfsetstrokecolor{currentstroke}%
\pgfsetdash{}{0pt}%
\pgfpathmoveto{\pgfqpoint{0.625000in}{1.608187in}}%
\pgfpathlineto{\pgfqpoint{0.634712in}{1.607370in}}%
\pgfpathlineto{\pgfqpoint{0.639115in}{1.611404in}}%
\pgfpathlineto{\pgfqpoint{0.634712in}{1.615744in}}%
\pgfpathlineto{\pgfqpoint{0.625000in}{1.614735in}}%
\pgfusepath{stroke}%
\end{pgfscope}%
\begin{pgfscope}%
\pgfpathrectangle{\pgfqpoint{0.625000in}{0.550000in}}{\pgfqpoint{3.875000in}{3.850000in}} %
\pgfusepath{clip}%
\pgfsetbuttcap%
\pgfsetroundjoin%
\pgfsetlinewidth{0.501875pt}%
\definecolor{currentstroke}{rgb}{0.000000,0.000000,0.000000}%
\pgfsetstrokecolor{currentstroke}%
\pgfsetdash{}{0pt}%
\pgfpathmoveto{\pgfqpoint{0.625000in}{1.656883in}}%
\pgfpathlineto{\pgfqpoint{0.634712in}{1.657210in}}%
\pgfpathlineto{\pgfqpoint{0.642598in}{1.659649in}}%
\pgfpathlineto{\pgfqpoint{0.644424in}{1.660776in}}%
\pgfpathlineto{\pgfqpoint{0.654135in}{1.667305in}}%
\pgfpathlineto{\pgfqpoint{0.656066in}{1.669298in}}%
\pgfpathlineto{\pgfqpoint{0.660422in}{1.678947in}}%
\pgfpathlineto{\pgfqpoint{0.659838in}{1.688596in}}%
\pgfpathlineto{\pgfqpoint{0.654135in}{1.696220in}}%
\pgfpathlineto{\pgfqpoint{0.653385in}{1.698246in}}%
\pgfpathlineto{\pgfqpoint{0.644424in}{1.703088in}}%
\pgfpathlineto{\pgfqpoint{0.634712in}{1.703070in}}%
\pgfpathlineto{\pgfqpoint{0.631477in}{1.698246in}}%
\pgfpathlineto{\pgfqpoint{0.625000in}{1.692124in}}%
\pgfusepath{stroke}%
\end{pgfscope}%
\begin{pgfscope}%
\pgfpathrectangle{\pgfqpoint{0.625000in}{0.550000in}}{\pgfqpoint{3.875000in}{3.850000in}} %
\pgfusepath{clip}%
\pgfsetbuttcap%
\pgfsetroundjoin%
\pgfsetlinewidth{0.501875pt}%
\definecolor{currentstroke}{rgb}{0.000000,0.000000,0.000000}%
\pgfsetstrokecolor{currentstroke}%
\pgfsetdash{}{0pt}%
\pgfpathmoveto{\pgfqpoint{0.625000in}{1.678605in}}%
\pgfpathlineto{\pgfqpoint{0.632375in}{1.669298in}}%
\pgfpathlineto{\pgfqpoint{0.625000in}{1.662415in}}%
\pgfusepath{stroke}%
\end{pgfscope}%
\begin{pgfscope}%
\pgfpathrectangle{\pgfqpoint{0.625000in}{0.550000in}}{\pgfqpoint{3.875000in}{3.850000in}} %
\pgfusepath{clip}%
\pgfsetbuttcap%
\pgfsetroundjoin%
\pgfsetlinewidth{0.501875pt}%
\definecolor{currentstroke}{rgb}{0.000000,0.000000,0.000000}%
\pgfsetstrokecolor{currentstroke}%
\pgfsetdash{}{0pt}%
\pgfpathmoveto{\pgfqpoint{0.625000in}{1.716432in}}%
\pgfpathlineto{\pgfqpoint{0.628978in}{1.717544in}}%
\pgfpathlineto{\pgfqpoint{0.628460in}{1.727193in}}%
\pgfpathlineto{\pgfqpoint{0.625000in}{1.731032in}}%
\pgfusepath{stroke}%
\end{pgfscope}%
\begin{pgfscope}%
\pgfpathrectangle{\pgfqpoint{0.625000in}{0.550000in}}{\pgfqpoint{3.875000in}{3.850000in}} %
\pgfusepath{clip}%
\pgfsetbuttcap%
\pgfsetroundjoin%
\pgfsetlinewidth{0.501875pt}%
\definecolor{currentstroke}{rgb}{0.000000,0.000000,0.000000}%
\pgfsetstrokecolor{currentstroke}%
\pgfsetdash{}{0pt}%
\pgfpathmoveto{\pgfqpoint{0.625000in}{1.808742in}}%
\pgfpathlineto{\pgfqpoint{0.630875in}{1.814035in}}%
\pgfpathlineto{\pgfqpoint{0.625000in}{1.819329in}}%
\pgfusepath{stroke}%
\end{pgfscope}%
\begin{pgfscope}%
\pgfpathrectangle{\pgfqpoint{0.625000in}{0.550000in}}{\pgfqpoint{3.875000in}{3.850000in}} %
\pgfusepath{clip}%
\pgfsetbuttcap%
\pgfsetroundjoin%
\pgfsetlinewidth{0.501875pt}%
\definecolor{currentstroke}{rgb}{0.000000,0.000000,0.000000}%
\pgfsetstrokecolor{currentstroke}%
\pgfsetdash{}{0pt}%
\pgfpathmoveto{\pgfqpoint{0.625000in}{1.838878in}}%
\pgfpathlineto{\pgfqpoint{0.634712in}{1.840911in}}%
\pgfpathlineto{\pgfqpoint{0.641836in}{1.842982in}}%
\pgfpathlineto{\pgfqpoint{0.644424in}{1.844818in}}%
\pgfpathlineto{\pgfqpoint{0.712406in}{1.872000in}}%
\pgfpathlineto{\pgfqpoint{0.741541in}{1.881738in}}%
\pgfpathlineto{\pgfqpoint{0.790100in}{1.897990in}}%
\pgfpathlineto{\pgfqpoint{0.819236in}{1.910739in}}%
\pgfpathlineto{\pgfqpoint{0.838659in}{1.921194in}}%
\pgfpathlineto{\pgfqpoint{0.858083in}{1.933580in}}%
\pgfpathlineto{\pgfqpoint{0.878479in}{1.949123in}}%
\pgfpathlineto{\pgfqpoint{0.899460in}{1.968421in}}%
\pgfpathlineto{\pgfqpoint{0.916818in}{1.987719in}}%
\pgfpathlineto{\pgfqpoint{0.931319in}{2.007018in}}%
\pgfpathlineto{\pgfqpoint{0.945489in}{2.029951in}}%
\pgfpathlineto{\pgfqpoint{0.955201in}{2.049195in}}%
\pgfpathlineto{\pgfqpoint{0.965358in}{2.074561in}}%
\pgfpathlineto{\pgfqpoint{0.971247in}{2.093860in}}%
\pgfpathlineto{\pgfqpoint{0.977303in}{2.122807in}}%
\pgfpathlineto{\pgfqpoint{0.980253in}{2.151754in}}%
\pgfpathlineto{\pgfqpoint{0.980195in}{2.180702in}}%
\pgfpathlineto{\pgfqpoint{0.977141in}{2.209649in}}%
\pgfpathlineto{\pgfqpoint{0.971055in}{2.238596in}}%
\pgfpathlineto{\pgfqpoint{0.961741in}{2.267544in}}%
\pgfpathlineto{\pgfqpoint{0.953588in}{2.286842in}}%
\pgfpathlineto{\pgfqpoint{0.943783in}{2.306140in}}%
\pgfpathlineto{\pgfqpoint{0.932159in}{2.325439in}}%
\pgfpathlineto{\pgfqpoint{0.916353in}{2.347200in}}%
\pgfpathlineto{\pgfqpoint{0.902307in}{2.364035in}}%
\pgfpathlineto{\pgfqpoint{0.877506in}{2.388408in}}%
\pgfpathlineto{\pgfqpoint{0.860661in}{2.402632in}}%
\pgfpathlineto{\pgfqpoint{0.828947in}{2.424432in}}%
\pgfpathlineto{\pgfqpoint{0.798468in}{2.441228in}}%
\pgfpathlineto{\pgfqpoint{0.770677in}{2.453220in}}%
\pgfpathlineto{\pgfqpoint{0.741541in}{2.463165in}}%
\pgfpathlineto{\pgfqpoint{0.712406in}{2.470659in}}%
\pgfpathlineto{\pgfqpoint{0.692982in}{2.474272in}}%
\pgfpathlineto{\pgfqpoint{0.681344in}{2.470175in}}%
\pgfpathlineto{\pgfqpoint{0.683271in}{2.465378in}}%
\pgfpathlineto{\pgfqpoint{0.692982in}{2.465293in}}%
\pgfpathlineto{\pgfqpoint{0.712406in}{2.454501in}}%
\pgfpathlineto{\pgfqpoint{0.722118in}{2.447763in}}%
\pgfpathlineto{\pgfqpoint{0.731830in}{2.439896in}}%
\pgfpathlineto{\pgfqpoint{0.741541in}{2.430428in}}%
\pgfpathlineto{\pgfqpoint{0.751253in}{2.418801in}}%
\pgfpathlineto{\pgfqpoint{0.756297in}{2.412281in}}%
\pgfpathlineto{\pgfqpoint{0.767228in}{2.392982in}}%
\pgfpathlineto{\pgfqpoint{0.774642in}{2.373684in}}%
\pgfpathlineto{\pgfqpoint{0.778983in}{2.354386in}}%
\pgfpathlineto{\pgfqpoint{0.780580in}{2.335088in}}%
\pgfpathlineto{\pgfqpoint{0.779549in}{2.315789in}}%
\pgfpathlineto{\pgfqpoint{0.775763in}{2.296491in}}%
\pgfpathlineto{\pgfqpoint{0.768898in}{2.277193in}}%
\pgfpathlineto{\pgfqpoint{0.758500in}{2.257895in}}%
\pgfpathlineto{\pgfqpoint{0.751253in}{2.247692in}}%
\pgfpathlineto{\pgfqpoint{0.741541in}{2.236506in}}%
\pgfpathlineto{\pgfqpoint{0.731830in}{2.227379in}}%
\pgfpathlineto{\pgfqpoint{0.712406in}{2.213422in}}%
\pgfpathlineto{\pgfqpoint{0.692982in}{2.203767in}}%
\pgfpathlineto{\pgfqpoint{0.673559in}{2.197501in}}%
\pgfpathlineto{\pgfqpoint{0.654135in}{2.194576in}}%
\pgfpathlineto{\pgfqpoint{0.644424in}{2.195629in}}%
\pgfpathlineto{\pgfqpoint{0.625000in}{2.204449in}}%
\pgfpathlineto{\pgfqpoint{0.625000in}{2.204449in}}%
\pgfusepath{stroke}%
\end{pgfscope}%
\begin{pgfscope}%
\pgfpathrectangle{\pgfqpoint{0.625000in}{0.550000in}}{\pgfqpoint{3.875000in}{3.850000in}} %
\pgfusepath{clip}%
\pgfsetbuttcap%
\pgfsetroundjoin%
\pgfsetlinewidth{0.501875pt}%
\definecolor{currentstroke}{rgb}{0.000000,0.000000,0.000000}%
\pgfsetstrokecolor{currentstroke}%
\pgfsetdash{}{0pt}%
\pgfpathmoveto{\pgfqpoint{0.625000in}{1.879026in}}%
\pgfpathlineto{\pgfqpoint{0.632644in}{1.871930in}}%
\pgfpathlineto{\pgfqpoint{0.631946in}{1.862281in}}%
\pgfpathlineto{\pgfqpoint{0.625000in}{1.856115in}}%
\pgfusepath{stroke}%
\end{pgfscope}%
\begin{pgfscope}%
\pgfpathrectangle{\pgfqpoint{0.625000in}{0.550000in}}{\pgfqpoint{3.875000in}{3.850000in}} %
\pgfusepath{clip}%
\pgfsetbuttcap%
\pgfsetroundjoin%
\pgfsetlinewidth{0.501875pt}%
\definecolor{currentstroke}{rgb}{0.000000,0.000000,0.000000}%
\pgfsetstrokecolor{currentstroke}%
\pgfsetdash{}{0pt}%
\pgfpathmoveto{\pgfqpoint{0.625000in}{1.894925in}}%
\pgfpathlineto{\pgfqpoint{0.630589in}{1.891228in}}%
\pgfpathlineto{\pgfqpoint{0.625000in}{1.887532in}}%
\pgfusepath{stroke}%
\end{pgfscope}%
\begin{pgfscope}%
\pgfpathrectangle{\pgfqpoint{0.625000in}{0.550000in}}{\pgfqpoint{3.875000in}{3.850000in}} %
\pgfusepath{clip}%
\pgfsetbuttcap%
\pgfsetroundjoin%
\pgfsetlinewidth{0.501875pt}%
\definecolor{currentstroke}{rgb}{0.000000,0.000000,0.000000}%
\pgfsetstrokecolor{currentstroke}%
\pgfsetdash{}{0pt}%
\pgfpathmoveto{\pgfqpoint{0.625000in}{1.992243in}}%
\pgfpathlineto{\pgfqpoint{0.632050in}{1.987719in}}%
\pgfpathlineto{\pgfqpoint{0.634368in}{1.978070in}}%
\pgfpathlineto{\pgfqpoint{0.634712in}{1.973209in}}%
\pgfpathlineto{\pgfqpoint{0.637262in}{1.968421in}}%
\pgfpathlineto{\pgfqpoint{0.634712in}{1.962503in}}%
\pgfpathlineto{\pgfqpoint{0.634495in}{1.958772in}}%
\pgfpathlineto{\pgfqpoint{0.634712in}{1.958271in}}%
\pgfpathlineto{\pgfqpoint{0.644424in}{1.958220in}}%
\pgfpathlineto{\pgfqpoint{0.651344in}{1.949123in}}%
\pgfpathlineto{\pgfqpoint{0.651147in}{1.939474in}}%
\pgfpathlineto{\pgfqpoint{0.644424in}{1.934590in}}%
\pgfpathlineto{\pgfqpoint{0.638777in}{1.929825in}}%
\pgfpathlineto{\pgfqpoint{0.634712in}{1.928955in}}%
\pgfpathlineto{\pgfqpoint{0.629240in}{1.920175in}}%
\pgfpathlineto{\pgfqpoint{0.625000in}{1.915554in}}%
\pgfusepath{stroke}%
\end{pgfscope}%
\begin{pgfscope}%
\pgfpathrectangle{\pgfqpoint{0.625000in}{0.550000in}}{\pgfqpoint{3.875000in}{3.850000in}} %
\pgfusepath{clip}%
\pgfsetbuttcap%
\pgfsetroundjoin%
\pgfsetlinewidth{0.501875pt}%
\definecolor{currentstroke}{rgb}{0.000000,0.000000,0.000000}%
\pgfsetstrokecolor{currentstroke}%
\pgfsetdash{}{0pt}%
\pgfpathmoveto{\pgfqpoint{0.625000in}{2.088759in}}%
\pgfpathlineto{\pgfqpoint{0.630393in}{2.084211in}}%
\pgfpathlineto{\pgfqpoint{0.630253in}{2.074561in}}%
\pgfpathlineto{\pgfqpoint{0.628564in}{2.064912in}}%
\pgfpathlineto{\pgfqpoint{0.631371in}{2.055263in}}%
\pgfpathlineto{\pgfqpoint{0.629011in}{2.045614in}}%
\pgfpathlineto{\pgfqpoint{0.625000in}{2.040023in}}%
\pgfusepath{stroke}%
\end{pgfscope}%
\begin{pgfscope}%
\pgfpathrectangle{\pgfqpoint{0.625000in}{0.550000in}}{\pgfqpoint{3.875000in}{3.850000in}} %
\pgfusepath{clip}%
\pgfsetbuttcap%
\pgfsetroundjoin%
\pgfsetlinewidth{0.501875pt}%
\definecolor{currentstroke}{rgb}{0.000000,0.000000,0.000000}%
\pgfsetstrokecolor{currentstroke}%
\pgfsetdash{}{0pt}%
\pgfpathmoveto{\pgfqpoint{0.625000in}{2.137402in}}%
\pgfpathlineto{\pgfqpoint{0.629393in}{2.132456in}}%
\pgfpathlineto{\pgfqpoint{0.630480in}{2.122807in}}%
\pgfpathlineto{\pgfqpoint{0.630289in}{2.113158in}}%
\pgfpathlineto{\pgfqpoint{0.625484in}{2.103509in}}%
\pgfpathlineto{\pgfqpoint{0.625000in}{2.103160in}}%
\pgfusepath{stroke}%
\end{pgfscope}%
\begin{pgfscope}%
\pgfpathrectangle{\pgfqpoint{0.625000in}{0.550000in}}{\pgfqpoint{3.875000in}{3.850000in}} %
\pgfusepath{clip}%
\pgfsetbuttcap%
\pgfsetroundjoin%
\pgfsetlinewidth{0.501875pt}%
\definecolor{currentstroke}{rgb}{0.000000,0.000000,0.000000}%
\pgfsetstrokecolor{currentstroke}%
\pgfsetdash{}{0pt}%
\pgfpathmoveto{\pgfqpoint{0.625000in}{2.182257in}}%
\pgfpathlineto{\pgfqpoint{0.630789in}{2.180702in}}%
\pgfpathlineto{\pgfqpoint{0.633939in}{2.171053in}}%
\pgfpathlineto{\pgfqpoint{0.625000in}{2.161462in}}%
\pgfusepath{stroke}%
\end{pgfscope}%
\begin{pgfscope}%
\pgfpathrectangle{\pgfqpoint{0.625000in}{0.550000in}}{\pgfqpoint{3.875000in}{3.850000in}} %
\pgfusepath{clip}%
\pgfsetbuttcap%
\pgfsetroundjoin%
\pgfsetlinewidth{0.501875pt}%
\definecolor{currentstroke}{rgb}{0.000000,0.000000,0.000000}%
\pgfsetstrokecolor{currentstroke}%
\pgfsetdash{}{0pt}%
\pgfpathmoveto{\pgfqpoint{0.625000in}{2.274010in}}%
\pgfpathlineto{\pgfqpoint{0.627561in}{2.277193in}}%
\pgfpathlineto{\pgfqpoint{0.625000in}{2.280376in}}%
\pgfusepath{stroke}%
\end{pgfscope}%
\begin{pgfscope}%
\pgfpathrectangle{\pgfqpoint{0.625000in}{0.550000in}}{\pgfqpoint{3.875000in}{3.850000in}} %
\pgfusepath{clip}%
\pgfsetbuttcap%
\pgfsetroundjoin%
\pgfsetlinewidth{0.501875pt}%
\definecolor{currentstroke}{rgb}{0.000000,0.000000,0.000000}%
\pgfsetstrokecolor{currentstroke}%
\pgfsetdash{}{0pt}%
\pgfpathmoveto{\pgfqpoint{0.625000in}{2.290897in}}%
\pgfpathlineto{\pgfqpoint{0.631855in}{2.296491in}}%
\pgfpathlineto{\pgfqpoint{0.633319in}{2.306140in}}%
\pgfpathlineto{\pgfqpoint{0.625000in}{2.309464in}}%
\pgfusepath{stroke}%
\end{pgfscope}%
\begin{pgfscope}%
\pgfpathrectangle{\pgfqpoint{0.625000in}{0.550000in}}{\pgfqpoint{3.875000in}{3.850000in}} %
\pgfusepath{clip}%
\pgfsetbuttcap%
\pgfsetroundjoin%
\pgfsetlinewidth{0.501875pt}%
\definecolor{currentstroke}{rgb}{0.000000,0.000000,0.000000}%
\pgfsetstrokecolor{currentstroke}%
\pgfsetdash{}{0pt}%
\pgfpathmoveto{\pgfqpoint{0.625000in}{2.330667in}}%
\pgfpathlineto{\pgfqpoint{0.629770in}{2.325439in}}%
\pgfpathlineto{\pgfqpoint{0.632211in}{2.315789in}}%
\pgfpathlineto{\pgfqpoint{0.634712in}{2.308093in}}%
\pgfpathlineto{\pgfqpoint{0.644424in}{2.309253in}}%
\pgfpathlineto{\pgfqpoint{0.654135in}{2.311471in}}%
\pgfpathlineto{\pgfqpoint{0.663847in}{2.315030in}}%
\pgfpathlineto{\pgfqpoint{0.665371in}{2.315789in}}%
\pgfpathlineto{\pgfqpoint{0.673559in}{2.319846in}}%
\pgfpathlineto{\pgfqpoint{0.681723in}{2.325439in}}%
\pgfpathlineto{\pgfqpoint{0.683271in}{2.326540in}}%
\pgfpathlineto{\pgfqpoint{0.692542in}{2.335088in}}%
\pgfpathlineto{\pgfqpoint{0.692982in}{2.335537in}}%
\pgfpathlineto{\pgfqpoint{0.700355in}{2.344737in}}%
\pgfpathlineto{\pgfqpoint{0.702694in}{2.348271in}}%
\pgfpathlineto{\pgfqpoint{0.706179in}{2.354386in}}%
\pgfpathlineto{\pgfqpoint{0.710390in}{2.364035in}}%
\pgfpathlineto{\pgfqpoint{0.712406in}{2.370703in}}%
\pgfpathlineto{\pgfqpoint{0.713241in}{2.373684in}}%
\pgfpathlineto{\pgfqpoint{0.714950in}{2.383333in}}%
\pgfpathlineto{\pgfqpoint{0.715502in}{2.392982in}}%
\pgfpathlineto{\pgfqpoint{0.714916in}{2.402632in}}%
\pgfpathlineto{\pgfqpoint{0.713119in}{2.412281in}}%
\pgfpathlineto{\pgfqpoint{0.712406in}{2.414317in}}%
\pgfpathlineto{\pgfqpoint{0.710314in}{2.421930in}}%
\pgfpathlineto{\pgfqpoint{0.706154in}{2.431579in}}%
\pgfpathlineto{\pgfqpoint{0.702694in}{2.436753in}}%
\pgfpathlineto{\pgfqpoint{0.700364in}{2.441228in}}%
\pgfpathlineto{\pgfqpoint{0.692982in}{2.450045in}}%
\pgfpathlineto{\pgfqpoint{0.691174in}{2.450877in}}%
\pgfpathlineto{\pgfqpoint{0.683271in}{2.456765in}}%
\pgfpathlineto{\pgfqpoint{0.679103in}{2.450877in}}%
\pgfpathlineto{\pgfqpoint{0.683271in}{2.447394in}}%
\pgfpathlineto{\pgfqpoint{0.687368in}{2.441228in}}%
\pgfpathlineto{\pgfqpoint{0.690818in}{2.431579in}}%
\pgfpathlineto{\pgfqpoint{0.692608in}{2.421930in}}%
\pgfpathlineto{\pgfqpoint{0.692982in}{2.415172in}}%
\pgfpathlineto{\pgfqpoint{0.693184in}{2.412281in}}%
\pgfpathlineto{\pgfqpoint{0.692982in}{2.410239in}}%
\pgfpathlineto{\pgfqpoint{0.692261in}{2.402632in}}%
\pgfpathlineto{\pgfqpoint{0.689838in}{2.392982in}}%
\pgfpathlineto{\pgfqpoint{0.685623in}{2.383333in}}%
\pgfpathlineto{\pgfqpoint{0.683271in}{2.379310in}}%
\pgfpathlineto{\pgfqpoint{0.679236in}{2.373684in}}%
\pgfpathlineto{\pgfqpoint{0.673559in}{2.367269in}}%
\pgfpathlineto{\pgfqpoint{0.669719in}{2.364035in}}%
\pgfpathlineto{\pgfqpoint{0.663847in}{2.359467in}}%
\pgfpathlineto{\pgfqpoint{0.654157in}{2.354386in}}%
\pgfpathlineto{\pgfqpoint{0.654135in}{2.354375in}}%
\pgfpathlineto{\pgfqpoint{0.644424in}{2.350882in}}%
\pgfpathlineto{\pgfqpoint{0.634712in}{2.349096in}}%
\pgfpathlineto{\pgfqpoint{0.625000in}{2.347410in}}%
\pgfusepath{stroke}%
\end{pgfscope}%
\begin{pgfscope}%
\pgfpathrectangle{\pgfqpoint{0.625000in}{0.550000in}}{\pgfqpoint{3.875000in}{3.850000in}} %
\pgfusepath{clip}%
\pgfsetbuttcap%
\pgfsetroundjoin%
\pgfsetlinewidth{0.501875pt}%
\definecolor{currentstroke}{rgb}{0.000000,0.000000,0.000000}%
\pgfsetstrokecolor{currentstroke}%
\pgfsetdash{}{0pt}%
\pgfpathmoveto{\pgfqpoint{0.625000in}{2.371769in}}%
\pgfpathlineto{\pgfqpoint{0.627047in}{2.373684in}}%
\pgfpathlineto{\pgfqpoint{0.634712in}{2.379988in}}%
\pgfpathlineto{\pgfqpoint{0.644424in}{2.382708in}}%
\pgfpathlineto{\pgfqpoint{0.645681in}{2.383333in}}%
\pgfpathlineto{\pgfqpoint{0.654135in}{2.387468in}}%
\pgfpathlineto{\pgfqpoint{0.660846in}{2.392982in}}%
\pgfpathlineto{\pgfqpoint{0.663847in}{2.395781in}}%
\pgfpathlineto{\pgfqpoint{0.669021in}{2.402632in}}%
\pgfpathlineto{\pgfqpoint{0.673559in}{2.411653in}}%
\pgfpathlineto{\pgfqpoint{0.673836in}{2.412281in}}%
\pgfpathlineto{\pgfqpoint{0.676407in}{2.421930in}}%
\pgfpathlineto{\pgfqpoint{0.677076in}{2.431579in}}%
\pgfpathlineto{\pgfqpoint{0.677423in}{2.441228in}}%
\pgfpathlineto{\pgfqpoint{0.673559in}{2.445743in}}%
\pgfpathlineto{\pgfqpoint{0.670428in}{2.441228in}}%
\pgfpathlineto{\pgfqpoint{0.669902in}{2.431579in}}%
\pgfpathlineto{\pgfqpoint{0.668197in}{2.421930in}}%
\pgfpathlineto{\pgfqpoint{0.663847in}{2.417653in}}%
\pgfpathlineto{\pgfqpoint{0.660461in}{2.412281in}}%
\pgfpathlineto{\pgfqpoint{0.654135in}{2.405222in}}%
\pgfpathlineto{\pgfqpoint{0.650326in}{2.402632in}}%
\pgfpathlineto{\pgfqpoint{0.644424in}{2.398823in}}%
\pgfpathlineto{\pgfqpoint{0.634712in}{2.395612in}}%
\pgfpathlineto{\pgfqpoint{0.627669in}{2.392982in}}%
\pgfpathlineto{\pgfqpoint{0.625000in}{2.390387in}}%
\pgfusepath{stroke}%
\end{pgfscope}%
\begin{pgfscope}%
\pgfpathrectangle{\pgfqpoint{0.625000in}{0.550000in}}{\pgfqpoint{3.875000in}{3.850000in}} %
\pgfusepath{clip}%
\pgfsetbuttcap%
\pgfsetroundjoin%
\pgfsetlinewidth{0.501875pt}%
\definecolor{currentstroke}{rgb}{0.000000,0.000000,0.000000}%
\pgfsetstrokecolor{currentstroke}%
\pgfsetdash{}{0pt}%
\pgfpathmoveto{\pgfqpoint{0.625000in}{2.454774in}}%
\pgfpathlineto{\pgfqpoint{0.628241in}{2.460526in}}%
\pgfpathlineto{\pgfqpoint{0.629438in}{2.470175in}}%
\pgfpathlineto{\pgfqpoint{0.625000in}{2.478606in}}%
\pgfusepath{stroke}%
\end{pgfscope}%
\begin{pgfscope}%
\pgfpathrectangle{\pgfqpoint{0.625000in}{0.550000in}}{\pgfqpoint{3.875000in}{3.850000in}} %
\pgfusepath{clip}%
\pgfsetbuttcap%
\pgfsetroundjoin%
\pgfsetlinewidth{0.501875pt}%
\definecolor{currentstroke}{rgb}{0.000000,0.000000,0.000000}%
\pgfsetstrokecolor{currentstroke}%
\pgfsetdash{}{0pt}%
\pgfpathmoveto{\pgfqpoint{0.625000in}{2.481043in}}%
\pgfpathlineto{\pgfqpoint{0.634712in}{2.481202in}}%
\pgfpathlineto{\pgfqpoint{0.644424in}{2.481182in}}%
\pgfpathlineto{\pgfqpoint{0.654135in}{2.481331in}}%
\pgfpathlineto{\pgfqpoint{0.663847in}{2.482928in}}%
\pgfpathlineto{\pgfqpoint{0.666665in}{2.489474in}}%
\pgfpathlineto{\pgfqpoint{0.673559in}{2.498171in}}%
\pgfpathlineto{\pgfqpoint{0.676516in}{2.499123in}}%
\pgfpathlineto{\pgfqpoint{0.673559in}{2.500799in}}%
\pgfpathlineto{\pgfqpoint{0.663847in}{2.503360in}}%
\pgfpathlineto{\pgfqpoint{0.659407in}{2.499123in}}%
\pgfpathlineto{\pgfqpoint{0.654135in}{2.493829in}}%
\pgfpathlineto{\pgfqpoint{0.644424in}{2.495380in}}%
\pgfpathlineto{\pgfqpoint{0.634712in}{2.493872in}}%
\pgfpathlineto{\pgfqpoint{0.628092in}{2.499123in}}%
\pgfpathlineto{\pgfqpoint{0.625000in}{2.504875in}}%
\pgfusepath{stroke}%
\end{pgfscope}%
\begin{pgfscope}%
\pgfpathrectangle{\pgfqpoint{0.625000in}{0.550000in}}{\pgfqpoint{3.875000in}{3.850000in}} %
\pgfusepath{clip}%
\pgfsetbuttcap%
\pgfsetroundjoin%
\pgfsetlinewidth{0.501875pt}%
\definecolor{currentstroke}{rgb}{0.000000,0.000000,0.000000}%
\pgfsetstrokecolor{currentstroke}%
\pgfsetdash{}{0pt}%
\pgfpathmoveto{\pgfqpoint{0.625000in}{2.569263in}}%
\pgfpathlineto{\pgfqpoint{0.627669in}{2.566667in}}%
\pgfpathlineto{\pgfqpoint{0.634712in}{2.564037in}}%
\pgfpathlineto{\pgfqpoint{0.644424in}{2.560826in}}%
\pgfpathlineto{\pgfqpoint{0.650326in}{2.557018in}}%
\pgfpathlineto{\pgfqpoint{0.654135in}{2.554428in}}%
\pgfpathlineto{\pgfqpoint{0.660461in}{2.547368in}}%
\pgfpathlineto{\pgfqpoint{0.663847in}{2.541996in}}%
\pgfpathlineto{\pgfqpoint{0.668197in}{2.537719in}}%
\pgfpathlineto{\pgfqpoint{0.669902in}{2.528070in}}%
\pgfpathlineto{\pgfqpoint{0.670428in}{2.518421in}}%
\pgfpathlineto{\pgfqpoint{0.673559in}{2.513906in}}%
\pgfpathlineto{\pgfqpoint{0.677423in}{2.518421in}}%
\pgfpathlineto{\pgfqpoint{0.677076in}{2.528070in}}%
\pgfpathlineto{\pgfqpoint{0.676407in}{2.537719in}}%
\pgfpathlineto{\pgfqpoint{0.673836in}{2.547368in}}%
\pgfpathlineto{\pgfqpoint{0.673559in}{2.547996in}}%
\pgfpathlineto{\pgfqpoint{0.669021in}{2.557018in}}%
\pgfpathlineto{\pgfqpoint{0.663847in}{2.563868in}}%
\pgfpathlineto{\pgfqpoint{0.660846in}{2.566667in}}%
\pgfpathlineto{\pgfqpoint{0.654135in}{2.572181in}}%
\pgfpathlineto{\pgfqpoint{0.645681in}{2.576316in}}%
\pgfpathlineto{\pgfqpoint{0.644424in}{2.576941in}}%
\pgfpathlineto{\pgfqpoint{0.634712in}{2.579661in}}%
\pgfpathlineto{\pgfqpoint{0.627047in}{2.585965in}}%
\pgfpathlineto{\pgfqpoint{0.625000in}{2.587881in}}%
\pgfusepath{stroke}%
\end{pgfscope}%
\begin{pgfscope}%
\pgfpathrectangle{\pgfqpoint{0.625000in}{0.550000in}}{\pgfqpoint{3.875000in}{3.850000in}} %
\pgfusepath{clip}%
\pgfsetbuttcap%
\pgfsetroundjoin%
\pgfsetlinewidth{0.501875pt}%
\definecolor{currentstroke}{rgb}{0.000000,0.000000,0.000000}%
\pgfsetstrokecolor{currentstroke}%
\pgfsetdash{}{0pt}%
\pgfpathmoveto{\pgfqpoint{0.625000in}{2.612239in}}%
\pgfpathlineto{\pgfqpoint{0.634712in}{2.610553in}}%
\pgfpathlineto{\pgfqpoint{0.644424in}{2.608767in}}%
\pgfpathlineto{\pgfqpoint{0.654135in}{2.605274in}}%
\pgfpathlineto{\pgfqpoint{0.654157in}{2.605263in}}%
\pgfpathlineto{\pgfqpoint{0.663847in}{2.600182in}}%
\pgfpathlineto{\pgfqpoint{0.669719in}{2.595614in}}%
\pgfpathlineto{\pgfqpoint{0.673559in}{2.592380in}}%
\pgfpathlineto{\pgfqpoint{0.679236in}{2.585965in}}%
\pgfpathlineto{\pgfqpoint{0.683271in}{2.580339in}}%
\pgfpathlineto{\pgfqpoint{0.685623in}{2.576316in}}%
\pgfpathlineto{\pgfqpoint{0.689838in}{2.566667in}}%
\pgfpathlineto{\pgfqpoint{0.692261in}{2.557018in}}%
\pgfpathlineto{\pgfqpoint{0.692982in}{2.549410in}}%
\pgfpathlineto{\pgfqpoint{0.693184in}{2.547368in}}%
\pgfpathlineto{\pgfqpoint{0.692982in}{2.544477in}}%
\pgfpathlineto{\pgfqpoint{0.692608in}{2.537719in}}%
\pgfpathlineto{\pgfqpoint{0.690818in}{2.528070in}}%
\pgfpathlineto{\pgfqpoint{0.687368in}{2.518421in}}%
\pgfpathlineto{\pgfqpoint{0.683271in}{2.512255in}}%
\pgfpathlineto{\pgfqpoint{0.679103in}{2.508772in}}%
\pgfpathlineto{\pgfqpoint{0.683271in}{2.502885in}}%
\pgfpathlineto{\pgfqpoint{0.691174in}{2.508772in}}%
\pgfpathlineto{\pgfqpoint{0.692982in}{2.509604in}}%
\pgfpathlineto{\pgfqpoint{0.700364in}{2.518421in}}%
\pgfpathlineto{\pgfqpoint{0.702694in}{2.522896in}}%
\pgfpathlineto{\pgfqpoint{0.706154in}{2.528070in}}%
\pgfpathlineto{\pgfqpoint{0.710314in}{2.537719in}}%
\pgfpathlineto{\pgfqpoint{0.712406in}{2.545332in}}%
\pgfpathlineto{\pgfqpoint{0.713119in}{2.547368in}}%
\pgfpathlineto{\pgfqpoint{0.714916in}{2.557018in}}%
\pgfpathlineto{\pgfqpoint{0.715502in}{2.566667in}}%
\pgfpathlineto{\pgfqpoint{0.714950in}{2.576316in}}%
\pgfpathlineto{\pgfqpoint{0.713241in}{2.585965in}}%
\pgfpathlineto{\pgfqpoint{0.712406in}{2.588946in}}%
\pgfpathlineto{\pgfqpoint{0.710390in}{2.595614in}}%
\pgfpathlineto{\pgfqpoint{0.706179in}{2.605263in}}%
\pgfpathlineto{\pgfqpoint{0.702694in}{2.611378in}}%
\pgfpathlineto{\pgfqpoint{0.700355in}{2.614912in}}%
\pgfpathlineto{\pgfqpoint{0.692982in}{2.624112in}}%
\pgfpathlineto{\pgfqpoint{0.692542in}{2.624561in}}%
\pgfpathlineto{\pgfqpoint{0.683271in}{2.633109in}}%
\pgfpathlineto{\pgfqpoint{0.681723in}{2.634211in}}%
\pgfpathlineto{\pgfqpoint{0.673559in}{2.639803in}}%
\pgfpathlineto{\pgfqpoint{0.665371in}{2.643860in}}%
\pgfpathlineto{\pgfqpoint{0.663847in}{2.644619in}}%
\pgfpathlineto{\pgfqpoint{0.654135in}{2.648178in}}%
\pgfpathlineto{\pgfqpoint{0.644424in}{2.650396in}}%
\pgfpathlineto{\pgfqpoint{0.634712in}{2.651556in}}%
\pgfpathlineto{\pgfqpoint{0.632211in}{2.643860in}}%
\pgfpathlineto{\pgfqpoint{0.625000in}{2.636669in}}%
\pgfusepath{stroke}%
\end{pgfscope}%
\begin{pgfscope}%
\pgfpathrectangle{\pgfqpoint{0.625000in}{0.550000in}}{\pgfqpoint{3.875000in}{3.850000in}} %
\pgfusepath{clip}%
\pgfsetbuttcap%
\pgfsetroundjoin%
\pgfsetlinewidth{0.501875pt}%
\definecolor{currentstroke}{rgb}{0.000000,0.000000,0.000000}%
\pgfsetstrokecolor{currentstroke}%
\pgfsetdash{}{0pt}%
\pgfpathmoveto{\pgfqpoint{0.625000in}{2.650185in}}%
\pgfpathlineto{\pgfqpoint{0.633319in}{2.653509in}}%
\pgfpathlineto{\pgfqpoint{0.631855in}{2.663158in}}%
\pgfpathlineto{\pgfqpoint{0.625000in}{2.668752in}}%
\pgfusepath{stroke}%
\end{pgfscope}%
\begin{pgfscope}%
\pgfpathrectangle{\pgfqpoint{0.625000in}{0.550000in}}{\pgfqpoint{3.875000in}{3.850000in}} %
\pgfusepath{clip}%
\pgfsetbuttcap%
\pgfsetroundjoin%
\pgfsetlinewidth{0.501875pt}%
\definecolor{currentstroke}{rgb}{0.000000,0.000000,0.000000}%
\pgfsetstrokecolor{currentstroke}%
\pgfsetdash{}{0pt}%
\pgfpathmoveto{\pgfqpoint{0.625000in}{2.679273in}}%
\pgfpathlineto{\pgfqpoint{0.627561in}{2.682456in}}%
\pgfpathlineto{\pgfqpoint{0.625000in}{2.685639in}}%
\pgfusepath{stroke}%
\end{pgfscope}%
\begin{pgfscope}%
\pgfpathrectangle{\pgfqpoint{0.625000in}{0.550000in}}{\pgfqpoint{3.875000in}{3.850000in}} %
\pgfusepath{clip}%
\pgfsetbuttcap%
\pgfsetroundjoin%
\pgfsetlinewidth{0.501875pt}%
\definecolor{currentstroke}{rgb}{0.000000,0.000000,0.000000}%
\pgfsetstrokecolor{currentstroke}%
\pgfsetdash{}{0pt}%
\pgfpathmoveto{\pgfqpoint{0.625000in}{2.755200in}}%
\pgfpathlineto{\pgfqpoint{0.644424in}{2.764020in}}%
\pgfpathlineto{\pgfqpoint{0.654135in}{2.765073in}}%
\pgfpathlineto{\pgfqpoint{0.673559in}{2.762148in}}%
\pgfpathlineto{\pgfqpoint{0.692982in}{2.755882in}}%
\pgfpathlineto{\pgfqpoint{0.712406in}{2.746227in}}%
\pgfpathlineto{\pgfqpoint{0.731830in}{2.732270in}}%
\pgfpathlineto{\pgfqpoint{0.743483in}{2.721053in}}%
\pgfpathlineto{\pgfqpoint{0.751678in}{2.711404in}}%
\pgfpathlineto{\pgfqpoint{0.764198in}{2.692105in}}%
\pgfpathlineto{\pgfqpoint{0.772728in}{2.672807in}}%
\pgfpathlineto{\pgfqpoint{0.778020in}{2.653509in}}%
\pgfpathlineto{\pgfqpoint{0.780388in}{2.634211in}}%
\pgfpathlineto{\pgfqpoint{0.780100in}{2.614912in}}%
\pgfpathlineto{\pgfqpoint{0.777186in}{2.595614in}}%
\pgfpathlineto{\pgfqpoint{0.770677in}{2.575135in}}%
\pgfpathlineto{\pgfqpoint{0.762154in}{2.557018in}}%
\pgfpathlineto{\pgfqpoint{0.749152in}{2.537719in}}%
\pgfpathlineto{\pgfqpoint{0.730587in}{2.518421in}}%
\pgfpathlineto{\pgfqpoint{0.712406in}{2.505148in}}%
\pgfpathlineto{\pgfqpoint{0.692982in}{2.494357in}}%
\pgfpathlineto{\pgfqpoint{0.683271in}{2.494271in}}%
\pgfpathlineto{\pgfqpoint{0.681345in}{2.489474in}}%
\pgfpathlineto{\pgfqpoint{0.683271in}{2.487866in}}%
\pgfpathlineto{\pgfqpoint{0.692982in}{2.485378in}}%
\pgfpathlineto{\pgfqpoint{0.722118in}{2.491256in}}%
\pgfpathlineto{\pgfqpoint{0.751253in}{2.499457in}}%
\pgfpathlineto{\pgfqpoint{0.780388in}{2.510340in}}%
\pgfpathlineto{\pgfqpoint{0.809524in}{2.524165in}}%
\pgfpathlineto{\pgfqpoint{0.833240in}{2.537719in}}%
\pgfpathlineto{\pgfqpoint{0.860661in}{2.557018in}}%
\pgfpathlineto{\pgfqpoint{0.887218in}{2.580249in}}%
\pgfpathlineto{\pgfqpoint{0.902307in}{2.595614in}}%
\pgfpathlineto{\pgfqpoint{0.925508in}{2.624561in}}%
\pgfpathlineto{\pgfqpoint{0.945489in}{2.656843in}}%
\pgfpathlineto{\pgfqpoint{0.955201in}{2.676584in}}%
\pgfpathlineto{\pgfqpoint{0.961741in}{2.692105in}}%
\pgfpathlineto{\pgfqpoint{0.971055in}{2.721053in}}%
\pgfpathlineto{\pgfqpoint{0.977141in}{2.750000in}}%
\pgfpathlineto{\pgfqpoint{0.980195in}{2.778947in}}%
\pgfpathlineto{\pgfqpoint{0.980253in}{2.807895in}}%
\pgfpathlineto{\pgfqpoint{0.977303in}{2.836842in}}%
\pgfpathlineto{\pgfqpoint{0.971247in}{2.865789in}}%
\pgfpathlineto{\pgfqpoint{0.964912in}{2.886336in}}%
\pgfpathlineto{\pgfqpoint{0.955201in}{2.910454in}}%
\pgfpathlineto{\pgfqpoint{0.945489in}{2.929698in}}%
\pgfpathlineto{\pgfqpoint{0.935777in}{2.945924in}}%
\pgfpathlineto{\pgfqpoint{0.924388in}{2.962281in}}%
\pgfpathlineto{\pgfqpoint{0.906642in}{2.983678in}}%
\pgfpathlineto{\pgfqpoint{0.887218in}{3.002966in}}%
\pgfpathlineto{\pgfqpoint{0.866231in}{3.020175in}}%
\pgfpathlineto{\pgfqpoint{0.848371in}{3.032528in}}%
\pgfpathlineto{\pgfqpoint{0.828947in}{3.043922in}}%
\pgfpathlineto{\pgfqpoint{0.799812in}{3.057752in}}%
\pgfpathlineto{\pgfqpoint{0.770677in}{3.068627in}}%
\pgfpathlineto{\pgfqpoint{0.692982in}{3.095065in}}%
\pgfpathlineto{\pgfqpoint{0.663847in}{3.107105in}}%
\pgfpathlineto{\pgfqpoint{0.634712in}{3.118738in}}%
\pgfpathlineto{\pgfqpoint{0.625000in}{3.120771in}}%
\pgfpathlineto{\pgfqpoint{0.625000in}{3.120771in}}%
\pgfusepath{stroke}%
\end{pgfscope}%
\begin{pgfscope}%
\pgfpathrectangle{\pgfqpoint{0.625000in}{0.550000in}}{\pgfqpoint{3.875000in}{3.850000in}} %
\pgfusepath{clip}%
\pgfsetbuttcap%
\pgfsetroundjoin%
\pgfsetlinewidth{0.501875pt}%
\definecolor{currentstroke}{rgb}{0.000000,0.000000,0.000000}%
\pgfsetstrokecolor{currentstroke}%
\pgfsetdash{}{0pt}%
\pgfpathmoveto{\pgfqpoint{0.625000in}{2.779808in}}%
\pgfpathlineto{\pgfqpoint{0.630789in}{2.778947in}}%
\pgfpathlineto{\pgfqpoint{0.625000in}{2.777392in}}%
\pgfusepath{stroke}%
\end{pgfscope}%
\begin{pgfscope}%
\pgfpathrectangle{\pgfqpoint{0.625000in}{0.550000in}}{\pgfqpoint{3.875000in}{3.850000in}} %
\pgfusepath{clip}%
\pgfsetbuttcap%
\pgfsetroundjoin%
\pgfsetlinewidth{0.501875pt}%
\definecolor{currentstroke}{rgb}{0.000000,0.000000,0.000000}%
\pgfsetstrokecolor{currentstroke}%
\pgfsetdash{}{0pt}%
\pgfpathmoveto{\pgfqpoint{0.625000in}{2.856489in}}%
\pgfpathlineto{\pgfqpoint{0.625484in}{2.856140in}}%
\pgfpathlineto{\pgfqpoint{0.630289in}{2.846491in}}%
\pgfpathlineto{\pgfqpoint{0.630480in}{2.836842in}}%
\pgfpathlineto{\pgfqpoint{0.629393in}{2.827193in}}%
\pgfpathlineto{\pgfqpoint{0.625000in}{2.822248in}}%
\pgfusepath{stroke}%
\end{pgfscope}%
\begin{pgfscope}%
\pgfpathrectangle{\pgfqpoint{0.625000in}{0.550000in}}{\pgfqpoint{3.875000in}{3.850000in}} %
\pgfusepath{clip}%
\pgfsetbuttcap%
\pgfsetroundjoin%
\pgfsetlinewidth{0.501875pt}%
\definecolor{currentstroke}{rgb}{0.000000,0.000000,0.000000}%
\pgfsetstrokecolor{currentstroke}%
\pgfsetdash{}{0pt}%
\pgfpathmoveto{\pgfqpoint{0.625000in}{2.919627in}}%
\pgfpathlineto{\pgfqpoint{0.629011in}{2.914035in}}%
\pgfpathlineto{\pgfqpoint{0.631371in}{2.904386in}}%
\pgfpathlineto{\pgfqpoint{0.628564in}{2.894737in}}%
\pgfpathlineto{\pgfqpoint{0.630253in}{2.885088in}}%
\pgfpathlineto{\pgfqpoint{0.630393in}{2.875439in}}%
\pgfpathlineto{\pgfqpoint{0.625000in}{2.870890in}}%
\pgfusepath{stroke}%
\end{pgfscope}%
\begin{pgfscope}%
\pgfpathrectangle{\pgfqpoint{0.625000in}{0.550000in}}{\pgfqpoint{3.875000in}{3.850000in}} %
\pgfusepath{clip}%
\pgfsetbuttcap%
\pgfsetroundjoin%
\pgfsetlinewidth{0.501875pt}%
\definecolor{currentstroke}{rgb}{0.000000,0.000000,0.000000}%
\pgfsetstrokecolor{currentstroke}%
\pgfsetdash{}{0pt}%
\pgfpathmoveto{\pgfqpoint{0.625000in}{2.950690in}}%
\pgfpathlineto{\pgfqpoint{0.630508in}{2.942982in}}%
\pgfpathlineto{\pgfqpoint{0.625000in}{2.937287in}}%
\pgfusepath{stroke}%
\end{pgfscope}%
\begin{pgfscope}%
\pgfpathrectangle{\pgfqpoint{0.625000in}{0.550000in}}{\pgfqpoint{3.875000in}{3.850000in}} %
\pgfusepath{clip}%
\pgfsetbuttcap%
\pgfsetroundjoin%
\pgfsetlinewidth{0.501875pt}%
\definecolor{currentstroke}{rgb}{0.000000,0.000000,0.000000}%
\pgfsetstrokecolor{currentstroke}%
\pgfsetdash{}{0pt}%
\pgfpathmoveto{\pgfqpoint{0.625000in}{3.044095in}}%
\pgfpathlineto{\pgfqpoint{0.629240in}{3.039474in}}%
\pgfpathlineto{\pgfqpoint{0.634712in}{3.030694in}}%
\pgfpathlineto{\pgfqpoint{0.638777in}{3.029825in}}%
\pgfpathlineto{\pgfqpoint{0.644424in}{3.025059in}}%
\pgfpathlineto{\pgfqpoint{0.651147in}{3.020175in}}%
\pgfpathlineto{\pgfqpoint{0.651344in}{3.010526in}}%
\pgfpathlineto{\pgfqpoint{0.644424in}{3.001429in}}%
\pgfpathlineto{\pgfqpoint{0.634712in}{3.001378in}}%
\pgfpathlineto{\pgfqpoint{0.634495in}{3.000877in}}%
\pgfpathlineto{\pgfqpoint{0.634712in}{2.997146in}}%
\pgfpathlineto{\pgfqpoint{0.637262in}{2.991228in}}%
\pgfpathlineto{\pgfqpoint{0.634712in}{2.986441in}}%
\pgfpathlineto{\pgfqpoint{0.634368in}{2.981579in}}%
\pgfpathlineto{\pgfqpoint{0.632050in}{2.971930in}}%
\pgfpathlineto{\pgfqpoint{0.625000in}{2.967406in}}%
\pgfusepath{stroke}%
\end{pgfscope}%
\begin{pgfscope}%
\pgfpathrectangle{\pgfqpoint{0.625000in}{0.550000in}}{\pgfqpoint{3.875000in}{3.850000in}} %
\pgfusepath{clip}%
\pgfsetbuttcap%
\pgfsetroundjoin%
\pgfsetlinewidth{0.501875pt}%
\definecolor{currentstroke}{rgb}{0.000000,0.000000,0.000000}%
\pgfsetstrokecolor{currentstroke}%
\pgfsetdash{}{0pt}%
\pgfpathmoveto{\pgfqpoint{0.625000in}{3.072118in}}%
\pgfpathlineto{\pgfqpoint{0.630589in}{3.068421in}}%
\pgfpathlineto{\pgfqpoint{0.625000in}{3.064725in}}%
\pgfusepath{stroke}%
\end{pgfscope}%
\begin{pgfscope}%
\pgfpathrectangle{\pgfqpoint{0.625000in}{0.550000in}}{\pgfqpoint{3.875000in}{3.850000in}} %
\pgfusepath{clip}%
\pgfsetbuttcap%
\pgfsetroundjoin%
\pgfsetlinewidth{0.501875pt}%
\definecolor{currentstroke}{rgb}{0.000000,0.000000,0.000000}%
\pgfsetstrokecolor{currentstroke}%
\pgfsetdash{}{0pt}%
\pgfpathmoveto{\pgfqpoint{0.625000in}{3.094476in}}%
\pgfpathlineto{\pgfqpoint{0.632644in}{3.087719in}}%
\pgfpathlineto{\pgfqpoint{0.625000in}{3.080623in}}%
\pgfusepath{stroke}%
\end{pgfscope}%
\begin{pgfscope}%
\pgfpathrectangle{\pgfqpoint{0.625000in}{0.550000in}}{\pgfqpoint{3.875000in}{3.850000in}} %
\pgfusepath{clip}%
\pgfsetbuttcap%
\pgfsetroundjoin%
\pgfsetlinewidth{0.501875pt}%
\definecolor{currentstroke}{rgb}{0.000000,0.000000,0.000000}%
\pgfsetstrokecolor{currentstroke}%
\pgfsetdash{}{0pt}%
\pgfpathmoveto{\pgfqpoint{0.625000in}{3.140320in}}%
\pgfpathlineto{\pgfqpoint{0.630875in}{3.145614in}}%
\pgfpathlineto{\pgfqpoint{0.625000in}{3.150908in}}%
\pgfusepath{stroke}%
\end{pgfscope}%
\begin{pgfscope}%
\pgfpathrectangle{\pgfqpoint{0.625000in}{0.550000in}}{\pgfqpoint{3.875000in}{3.850000in}} %
\pgfusepath{clip}%
\pgfsetbuttcap%
\pgfsetroundjoin%
\pgfsetlinewidth{0.501875pt}%
\definecolor{currentstroke}{rgb}{0.000000,0.000000,0.000000}%
\pgfsetstrokecolor{currentstroke}%
\pgfsetdash{}{0pt}%
\pgfpathmoveto{\pgfqpoint{0.625000in}{3.228617in}}%
\pgfpathlineto{\pgfqpoint{0.628460in}{3.232456in}}%
\pgfpathlineto{\pgfqpoint{0.628978in}{3.242105in}}%
\pgfpathlineto{\pgfqpoint{0.630256in}{3.251754in}}%
\pgfpathlineto{\pgfqpoint{0.625000in}{3.255331in}}%
\pgfusepath{stroke}%
\end{pgfscope}%
\begin{pgfscope}%
\pgfpathrectangle{\pgfqpoint{0.625000in}{0.550000in}}{\pgfqpoint{3.875000in}{3.850000in}} %
\pgfusepath{clip}%
\pgfsetbuttcap%
\pgfsetroundjoin%
\pgfsetlinewidth{0.501875pt}%
\definecolor{currentstroke}{rgb}{0.000000,0.000000,0.000000}%
\pgfsetstrokecolor{currentstroke}%
\pgfsetdash{}{0pt}%
\pgfpathmoveto{\pgfqpoint{0.625000in}{3.267525in}}%
\pgfpathlineto{\pgfqpoint{0.631477in}{3.261404in}}%
\pgfpathlineto{\pgfqpoint{0.634712in}{3.256579in}}%
\pgfpathlineto{\pgfqpoint{0.644424in}{3.256561in}}%
\pgfpathlineto{\pgfqpoint{0.653385in}{3.261404in}}%
\pgfpathlineto{\pgfqpoint{0.654135in}{3.263429in}}%
\pgfpathlineto{\pgfqpoint{0.659838in}{3.271053in}}%
\pgfpathlineto{\pgfqpoint{0.660422in}{3.280702in}}%
\pgfpathlineto{\pgfqpoint{0.656066in}{3.290351in}}%
\pgfpathlineto{\pgfqpoint{0.654135in}{3.292344in}}%
\pgfpathlineto{\pgfqpoint{0.644424in}{3.298873in}}%
\pgfpathlineto{\pgfqpoint{0.642598in}{3.300000in}}%
\pgfpathlineto{\pgfqpoint{0.634712in}{3.302439in}}%
\pgfpathlineto{\pgfqpoint{0.625000in}{3.302766in}}%
\pgfusepath{stroke}%
\end{pgfscope}%
\begin{pgfscope}%
\pgfpathrectangle{\pgfqpoint{0.625000in}{0.550000in}}{\pgfqpoint{3.875000in}{3.850000in}} %
\pgfusepath{clip}%
\pgfsetbuttcap%
\pgfsetroundjoin%
\pgfsetlinewidth{0.501875pt}%
\definecolor{currentstroke}{rgb}{0.000000,0.000000,0.000000}%
\pgfsetstrokecolor{currentstroke}%
\pgfsetdash{}{0pt}%
\pgfpathmoveto{\pgfqpoint{0.625000in}{3.297234in}}%
\pgfpathlineto{\pgfqpoint{0.632375in}{3.290351in}}%
\pgfpathlineto{\pgfqpoint{0.625000in}{3.281044in}}%
\pgfusepath{stroke}%
\end{pgfscope}%
\begin{pgfscope}%
\pgfpathrectangle{\pgfqpoint{0.625000in}{0.550000in}}{\pgfqpoint{3.875000in}{3.850000in}} %
\pgfusepath{clip}%
\pgfsetbuttcap%
\pgfsetroundjoin%
\pgfsetlinewidth{0.501875pt}%
\definecolor{currentstroke}{rgb}{0.000000,0.000000,0.000000}%
\pgfsetstrokecolor{currentstroke}%
\pgfsetdash{}{0pt}%
\pgfpathmoveto{\pgfqpoint{0.625000in}{3.344914in}}%
\pgfpathlineto{\pgfqpoint{0.634712in}{3.343905in}}%
\pgfpathlineto{\pgfqpoint{0.639115in}{3.348246in}}%
\pgfpathlineto{\pgfqpoint{0.634712in}{3.352279in}}%
\pgfpathlineto{\pgfqpoint{0.625000in}{3.351463in}}%
\pgfusepath{stroke}%
\end{pgfscope}%
\begin{pgfscope}%
\pgfpathrectangle{\pgfqpoint{0.625000in}{0.550000in}}{\pgfqpoint{3.875000in}{3.850000in}} %
\pgfusepath{clip}%
\pgfsetbuttcap%
\pgfsetroundjoin%
\pgfsetlinewidth{0.501875pt}%
\definecolor{currentstroke}{rgb}{0.000000,0.000000,0.000000}%
\pgfsetstrokecolor{currentstroke}%
\pgfsetdash{}{0pt}%
\pgfpathmoveto{\pgfqpoint{0.625000in}{3.384343in}}%
\pgfpathlineto{\pgfqpoint{0.631390in}{3.386842in}}%
\pgfpathlineto{\pgfqpoint{0.625000in}{3.391068in}}%
\pgfusepath{stroke}%
\end{pgfscope}%
\begin{pgfscope}%
\pgfpathrectangle{\pgfqpoint{0.625000in}{0.550000in}}{\pgfqpoint{3.875000in}{3.850000in}} %
\pgfusepath{clip}%
\pgfsetbuttcap%
\pgfsetroundjoin%
\pgfsetlinewidth{0.501875pt}%
\definecolor{currentstroke}{rgb}{0.000000,0.000000,0.000000}%
\pgfsetstrokecolor{currentstroke}%
\pgfsetdash{}{0pt}%
\pgfpathmoveto{\pgfqpoint{0.625000in}{3.400851in}}%
\pgfpathlineto{\pgfqpoint{0.632205in}{3.406140in}}%
\pgfpathlineto{\pgfqpoint{0.631863in}{3.415789in}}%
\pgfpathlineto{\pgfqpoint{0.630624in}{3.425439in}}%
\pgfpathlineto{\pgfqpoint{0.625000in}{3.432662in}}%
\pgfusepath{stroke}%
\end{pgfscope}%
\begin{pgfscope}%
\pgfpathrectangle{\pgfqpoint{0.625000in}{0.550000in}}{\pgfqpoint{3.875000in}{3.850000in}} %
\pgfusepath{clip}%
\pgfsetbuttcap%
\pgfsetroundjoin%
\pgfsetlinewidth{0.501875pt}%
\definecolor{currentstroke}{rgb}{0.000000,0.000000,0.000000}%
\pgfsetstrokecolor{currentstroke}%
\pgfsetdash{}{0pt}%
\pgfpathmoveto{\pgfqpoint{0.625000in}{3.526066in}}%
\pgfpathlineto{\pgfqpoint{0.629230in}{3.531579in}}%
\pgfpathlineto{\pgfqpoint{0.625000in}{3.537092in}}%
\pgfusepath{stroke}%
\end{pgfscope}%
\begin{pgfscope}%
\pgfpathrectangle{\pgfqpoint{0.625000in}{0.550000in}}{\pgfqpoint{3.875000in}{3.850000in}} %
\pgfusepath{clip}%
\pgfsetbuttcap%
\pgfsetroundjoin%
\pgfsetlinewidth{0.501875pt}%
\definecolor{currentstroke}{rgb}{0.000000,0.000000,0.000000}%
\pgfsetstrokecolor{currentstroke}%
\pgfsetdash{}{0pt}%
\pgfpathmoveto{\pgfqpoint{0.625000in}{3.575518in}}%
\pgfpathlineto{\pgfqpoint{0.628054in}{3.579825in}}%
\pgfpathlineto{\pgfqpoint{0.632100in}{3.589474in}}%
\pgfpathlineto{\pgfqpoint{0.625000in}{3.594129in}}%
\pgfusepath{stroke}%
\end{pgfscope}%
\begin{pgfscope}%
\pgfpathrectangle{\pgfqpoint{0.625000in}{0.550000in}}{\pgfqpoint{3.875000in}{3.850000in}} %
\pgfusepath{clip}%
\pgfsetbuttcap%
\pgfsetroundjoin%
\pgfsetlinewidth{0.501875pt}%
\definecolor{currentstroke}{rgb}{0.000000,0.000000,0.000000}%
\pgfsetstrokecolor{currentstroke}%
\pgfsetdash{}{0pt}%
\pgfpathmoveto{\pgfqpoint{0.625000in}{3.692115in}}%
\pgfpathlineto{\pgfqpoint{0.629823in}{3.695614in}}%
\pgfpathlineto{\pgfqpoint{0.625000in}{3.699002in}}%
\pgfusepath{stroke}%
\end{pgfscope}%
\begin{pgfscope}%
\pgfpathrectangle{\pgfqpoint{0.625000in}{0.550000in}}{\pgfqpoint{3.875000in}{3.850000in}} %
\pgfusepath{clip}%
\pgfsetbuttcap%
\pgfsetroundjoin%
\pgfsetlinewidth{0.501875pt}%
\definecolor{currentstroke}{rgb}{0.000000,0.000000,0.000000}%
\pgfsetstrokecolor{currentstroke}%
\pgfsetdash{}{0pt}%
\pgfpathmoveto{\pgfqpoint{0.625000in}{3.713260in}}%
\pgfpathlineto{\pgfqpoint{0.634712in}{3.711532in}}%
\pgfpathlineto{\pgfqpoint{0.643155in}{3.714912in}}%
\pgfpathlineto{\pgfqpoint{0.644424in}{3.715461in}}%
\pgfpathlineto{\pgfqpoint{0.653173in}{3.724561in}}%
\pgfpathlineto{\pgfqpoint{0.654135in}{3.726653in}}%
\pgfpathlineto{\pgfqpoint{0.656768in}{3.734211in}}%
\pgfpathlineto{\pgfqpoint{0.656389in}{3.743860in}}%
\pgfpathlineto{\pgfqpoint{0.654135in}{3.747408in}}%
\pgfpathlineto{\pgfqpoint{0.652126in}{3.753509in}}%
\pgfpathlineto{\pgfqpoint{0.644424in}{3.759074in}}%
\pgfpathlineto{\pgfqpoint{0.637972in}{3.753509in}}%
\pgfpathlineto{\pgfqpoint{0.638358in}{3.743860in}}%
\pgfpathlineto{\pgfqpoint{0.634712in}{3.739584in}}%
\pgfpathlineto{\pgfqpoint{0.625000in}{3.737377in}}%
\pgfusepath{stroke}%
\end{pgfscope}%
\begin{pgfscope}%
\pgfpathrectangle{\pgfqpoint{0.625000in}{0.550000in}}{\pgfqpoint{3.875000in}{3.850000in}} %
\pgfusepath{clip}%
\pgfsetbuttcap%
\pgfsetroundjoin%
\pgfsetlinewidth{0.501875pt}%
\definecolor{currentstroke}{rgb}{0.000000,0.000000,0.000000}%
\pgfsetstrokecolor{currentstroke}%
\pgfsetdash{}{0pt}%
\pgfpathmoveto{\pgfqpoint{0.625000in}{3.729421in}}%
\pgfpathlineto{\pgfqpoint{0.628976in}{3.724561in}}%
\pgfpathlineto{\pgfqpoint{0.625000in}{3.717551in}}%
\pgfusepath{stroke}%
\end{pgfscope}%
\begin{pgfscope}%
\pgfpathrectangle{\pgfqpoint{0.625000in}{0.550000in}}{\pgfqpoint{3.875000in}{3.850000in}} %
\pgfusepath{clip}%
\pgfsetbuttcap%
\pgfsetroundjoin%
\pgfsetlinewidth{0.501875pt}%
\definecolor{currentstroke}{rgb}{0.000000,0.000000,0.000000}%
\pgfsetstrokecolor{currentstroke}%
\pgfsetdash{}{0pt}%
\pgfpathmoveto{\pgfqpoint{0.625000in}{3.778774in}}%
\pgfpathlineto{\pgfqpoint{0.631208in}{3.782456in}}%
\pgfpathlineto{\pgfqpoint{0.625000in}{3.784919in}}%
\pgfusepath{stroke}%
\end{pgfscope}%
\begin{pgfscope}%
\pgfpathrectangle{\pgfqpoint{0.625000in}{0.550000in}}{\pgfqpoint{3.875000in}{3.850000in}} %
\pgfusepath{clip}%
\pgfsetbuttcap%
\pgfsetroundjoin%
\pgfsetlinewidth{0.501875pt}%
\definecolor{currentstroke}{rgb}{0.000000,0.000000,0.000000}%
\pgfsetstrokecolor{currentstroke}%
\pgfsetdash{}{0pt}%
\pgfpathmoveto{\pgfqpoint{0.625000in}{3.826893in}}%
\pgfpathlineto{\pgfqpoint{0.629495in}{3.821053in}}%
\pgfpathlineto{\pgfqpoint{0.629592in}{3.811404in}}%
\pgfpathlineto{\pgfqpoint{0.625622in}{3.801754in}}%
\pgfpathlineto{\pgfqpoint{0.634712in}{3.800228in}}%
\pgfpathlineto{\pgfqpoint{0.642578in}{3.792105in}}%
\pgfpathlineto{\pgfqpoint{0.643228in}{3.782456in}}%
\pgfpathlineto{\pgfqpoint{0.634712in}{3.776734in}}%
\pgfpathlineto{\pgfqpoint{0.632838in}{3.772807in}}%
\pgfpathlineto{\pgfqpoint{0.634712in}{3.764425in}}%
\pgfpathlineto{\pgfqpoint{0.644424in}{3.767415in}}%
\pgfpathlineto{\pgfqpoint{0.653256in}{3.772807in}}%
\pgfpathlineto{\pgfqpoint{0.654135in}{3.773874in}}%
\pgfpathlineto{\pgfqpoint{0.663847in}{3.781020in}}%
\pgfpathlineto{\pgfqpoint{0.666104in}{3.782456in}}%
\pgfpathlineto{\pgfqpoint{0.672822in}{3.792105in}}%
\pgfpathlineto{\pgfqpoint{0.673559in}{3.794730in}}%
\pgfpathlineto{\pgfqpoint{0.676657in}{3.801754in}}%
\pgfpathlineto{\pgfqpoint{0.677700in}{3.811404in}}%
\pgfpathlineto{\pgfqpoint{0.676225in}{3.821053in}}%
\pgfpathlineto{\pgfqpoint{0.673559in}{3.827514in}}%
\pgfpathlineto{\pgfqpoint{0.672003in}{3.830702in}}%
\pgfpathlineto{\pgfqpoint{0.663896in}{3.840351in}}%
\pgfpathlineto{\pgfqpoint{0.663847in}{3.840396in}}%
\pgfpathlineto{\pgfqpoint{0.654135in}{3.846816in}}%
\pgfpathlineto{\pgfqpoint{0.646075in}{3.850000in}}%
\pgfpathlineto{\pgfqpoint{0.644424in}{3.850523in}}%
\pgfpathlineto{\pgfqpoint{0.634712in}{3.853567in}}%
\pgfpathlineto{\pgfqpoint{0.625000in}{3.853551in}}%
\pgfusepath{stroke}%
\end{pgfscope}%
\begin{pgfscope}%
\pgfpathrectangle{\pgfqpoint{0.625000in}{0.550000in}}{\pgfqpoint{3.875000in}{3.850000in}} %
\pgfusepath{clip}%
\pgfsetbuttcap%
\pgfsetroundjoin%
\pgfsetlinewidth{0.501875pt}%
\definecolor{currentstroke}{rgb}{0.000000,0.000000,0.000000}%
\pgfsetstrokecolor{currentstroke}%
\pgfsetdash{}{0pt}%
\pgfpathmoveto{\pgfqpoint{0.625000in}{3.846923in}}%
\pgfpathlineto{\pgfqpoint{0.633034in}{3.840351in}}%
\pgfpathlineto{\pgfqpoint{0.625000in}{3.833779in}}%
\pgfusepath{stroke}%
\end{pgfscope}%
\begin{pgfscope}%
\pgfpathrectangle{\pgfqpoint{0.625000in}{0.550000in}}{\pgfqpoint{3.875000in}{3.850000in}} %
\pgfusepath{clip}%
\pgfsetbuttcap%
\pgfsetroundjoin%
\pgfsetlinewidth{0.501875pt}%
\definecolor{currentstroke}{rgb}{0.000000,0.000000,0.000000}%
\pgfsetstrokecolor{currentstroke}%
\pgfsetdash{}{0pt}%
\pgfpathmoveto{\pgfqpoint{0.625000in}{3.866120in}}%
\pgfpathlineto{\pgfqpoint{0.629282in}{3.869298in}}%
\pgfpathlineto{\pgfqpoint{0.625000in}{3.872101in}}%
\pgfusepath{stroke}%
\end{pgfscope}%
\begin{pgfscope}%
\pgfpathrectangle{\pgfqpoint{0.625000in}{0.550000in}}{\pgfqpoint{3.875000in}{3.850000in}} %
\pgfusepath{clip}%
\pgfsetbuttcap%
\pgfsetroundjoin%
\pgfsetlinewidth{0.501875pt}%
\definecolor{currentstroke}{rgb}{0.000000,0.000000,0.000000}%
\pgfsetstrokecolor{currentstroke}%
\pgfsetdash{}{0pt}%
\pgfpathmoveto{\pgfqpoint{0.625000in}{3.885443in}}%
\pgfpathlineto{\pgfqpoint{0.632385in}{3.888596in}}%
\pgfpathlineto{\pgfqpoint{0.625000in}{3.896843in}}%
\pgfusepath{stroke}%
\end{pgfscope}%
\begin{pgfscope}%
\pgfpathrectangle{\pgfqpoint{0.625000in}{0.550000in}}{\pgfqpoint{3.875000in}{3.850000in}} %
\pgfusepath{clip}%
\pgfsetbuttcap%
\pgfsetroundjoin%
\pgfsetlinewidth{0.501875pt}%
\definecolor{currentstroke}{rgb}{0.000000,0.000000,0.000000}%
\pgfsetstrokecolor{currentstroke}%
\pgfsetdash{}{0pt}%
\pgfpathmoveto{\pgfqpoint{0.625000in}{4.001863in}}%
\pgfpathlineto{\pgfqpoint{0.630582in}{4.004386in}}%
\pgfpathlineto{\pgfqpoint{0.634712in}{4.009714in}}%
\pgfpathlineto{\pgfqpoint{0.636703in}{4.014035in}}%
\pgfpathlineto{\pgfqpoint{0.634712in}{4.016320in}}%
\pgfpathlineto{\pgfqpoint{0.629476in}{4.014035in}}%
\pgfpathlineto{\pgfqpoint{0.625000in}{4.010763in}}%
\pgfusepath{stroke}%
\end{pgfscope}%
\begin{pgfscope}%
\pgfpathrectangle{\pgfqpoint{0.625000in}{0.550000in}}{\pgfqpoint{3.875000in}{3.850000in}} %
\pgfusepath{clip}%
\pgfsetbuttcap%
\pgfsetroundjoin%
\pgfsetlinewidth{0.501875pt}%
\definecolor{currentstroke}{rgb}{0.000000,0.000000,0.000000}%
\pgfsetstrokecolor{currentstroke}%
\pgfsetdash{}{0pt}%
\pgfpathmoveto{\pgfqpoint{0.625000in}{4.040548in}}%
\pgfpathlineto{\pgfqpoint{0.630878in}{4.033333in}}%
\pgfpathlineto{\pgfqpoint{0.634712in}{4.028509in}}%
\pgfpathlineto{\pgfqpoint{0.643540in}{4.033333in}}%
\pgfpathlineto{\pgfqpoint{0.644424in}{4.040511in}}%
\pgfpathlineto{\pgfqpoint{0.645178in}{4.042982in}}%
\pgfpathlineto{\pgfqpoint{0.644424in}{4.044706in}}%
\pgfpathlineto{\pgfqpoint{0.638737in}{4.052632in}}%
\pgfpathlineto{\pgfqpoint{0.634712in}{4.055101in}}%
\pgfpathlineto{\pgfqpoint{0.633575in}{4.052632in}}%
\pgfpathlineto{\pgfqpoint{0.625000in}{4.046642in}}%
\pgfusepath{stroke}%
\end{pgfscope}%
\begin{pgfscope}%
\pgfpathrectangle{\pgfqpoint{0.625000in}{0.550000in}}{\pgfqpoint{3.875000in}{3.850000in}} %
\pgfusepath{clip}%
\pgfsetbuttcap%
\pgfsetroundjoin%
\pgfsetlinewidth{0.501875pt}%
\definecolor{currentstroke}{rgb}{0.000000,0.000000,0.000000}%
\pgfsetstrokecolor{currentstroke}%
\pgfsetdash{}{0pt}%
\pgfpathmoveto{\pgfqpoint{0.625000in}{4.070050in}}%
\pgfpathlineto{\pgfqpoint{0.628192in}{4.071930in}}%
\pgfpathlineto{\pgfqpoint{0.625000in}{4.073810in}}%
\pgfusepath{stroke}%
\end{pgfscope}%
\begin{pgfscope}%
\pgfpathrectangle{\pgfqpoint{0.625000in}{0.550000in}}{\pgfqpoint{3.875000in}{3.850000in}} %
\pgfusepath{clip}%
\pgfsetbuttcap%
\pgfsetroundjoin%
\pgfsetlinewidth{0.501875pt}%
\definecolor{currentstroke}{rgb}{0.000000,0.000000,0.000000}%
\pgfsetstrokecolor{currentstroke}%
\pgfsetdash{}{0pt}%
\pgfpathmoveto{\pgfqpoint{0.625000in}{4.147251in}}%
\pgfpathlineto{\pgfqpoint{0.631025in}{4.149123in}}%
\pgfpathlineto{\pgfqpoint{0.629907in}{4.158772in}}%
\pgfpathlineto{\pgfqpoint{0.625000in}{4.162344in}}%
\pgfusepath{stroke}%
\end{pgfscope}%
\begin{pgfscope}%
\pgfpathrectangle{\pgfqpoint{0.625000in}{0.550000in}}{\pgfqpoint{3.875000in}{3.850000in}} %
\pgfusepath{clip}%
\pgfsetbuttcap%
\pgfsetroundjoin%
\pgfsetlinewidth{0.501875pt}%
\definecolor{currentstroke}{rgb}{0.000000,0.000000,0.000000}%
\pgfsetstrokecolor{currentstroke}%
\pgfsetdash{}{0pt}%
\pgfpathmoveto{\pgfqpoint{0.625000in}{4.177226in}}%
\pgfpathlineto{\pgfqpoint{0.625629in}{4.178070in}}%
\pgfpathlineto{\pgfqpoint{0.625000in}{4.185154in}}%
\pgfusepath{stroke}%
\end{pgfscope}%
\begin{pgfscope}%
\pgfpathrectangle{\pgfqpoint{0.625000in}{0.550000in}}{\pgfqpoint{3.875000in}{3.850000in}} %
\pgfusepath{clip}%
\pgfsetbuttcap%
\pgfsetroundjoin%
\pgfsetlinewidth{0.501875pt}%
\definecolor{currentstroke}{rgb}{0.000000,0.000000,0.000000}%
\pgfsetstrokecolor{currentstroke}%
\pgfsetdash{}{0pt}%
\pgfpathmoveto{\pgfqpoint{0.625000in}{4.188258in}}%
\pgfpathlineto{\pgfqpoint{0.629217in}{4.197368in}}%
\pgfpathlineto{\pgfqpoint{0.625000in}{4.201076in}}%
\pgfusepath{stroke}%
\end{pgfscope}%
\begin{pgfscope}%
\pgfpathrectangle{\pgfqpoint{0.625000in}{0.550000in}}{\pgfqpoint{3.875000in}{3.850000in}} %
\pgfusepath{clip}%
\pgfsetbuttcap%
\pgfsetroundjoin%
\pgfsetlinewidth{0.501875pt}%
\definecolor{currentstroke}{rgb}{0.000000,0.000000,0.000000}%
\pgfsetstrokecolor{currentstroke}%
\pgfsetdash{}{0pt}%
\pgfpathmoveto{\pgfqpoint{0.625000in}{4.263713in}}%
\pgfpathlineto{\pgfqpoint{0.627093in}{4.264912in}}%
\pgfpathlineto{\pgfqpoint{0.625000in}{4.266876in}}%
\pgfusepath{stroke}%
\end{pgfscope}%
\begin{pgfscope}%
\pgfpathrectangle{\pgfqpoint{0.625000in}{0.550000in}}{\pgfqpoint{3.875000in}{3.850000in}} %
\pgfusepath{clip}%
\pgfsetbuttcap%
\pgfsetroundjoin%
\pgfsetlinewidth{0.501875pt}%
\definecolor{currentstroke}{rgb}{0.000000,0.000000,0.000000}%
\pgfsetstrokecolor{currentstroke}%
\pgfsetdash{}{0pt}%
\pgfpathmoveto{\pgfqpoint{0.625000in}{4.283402in}}%
\pgfpathlineto{\pgfqpoint{0.625815in}{4.284211in}}%
\pgfpathlineto{\pgfqpoint{0.625000in}{4.285019in}}%
\pgfusepath{stroke}%
\end{pgfscope}%
\begin{pgfscope}%
\pgfpathrectangle{\pgfqpoint{0.625000in}{0.550000in}}{\pgfqpoint{3.875000in}{3.850000in}} %
\pgfusepath{clip}%
\pgfsetbuttcap%
\pgfsetroundjoin%
\pgfsetlinewidth{0.501875pt}%
\definecolor{currentstroke}{rgb}{0.000000,0.000000,0.000000}%
\pgfsetstrokecolor{currentstroke}%
\pgfsetdash{}{0pt}%
\pgfpathmoveto{\pgfqpoint{0.625000in}{4.348733in}}%
\pgfpathlineto{\pgfqpoint{0.628460in}{4.351754in}}%
\pgfpathlineto{\pgfqpoint{0.625000in}{4.355287in}}%
\pgfusepath{stroke}%
\end{pgfscope}%
\begin{pgfscope}%
\pgfpathrectangle{\pgfqpoint{0.625000in}{0.550000in}}{\pgfqpoint{3.875000in}{3.850000in}} %
\pgfusepath{clip}%
\pgfsetbuttcap%
\pgfsetroundjoin%
\pgfsetlinewidth{0.501875pt}%
\definecolor{currentstroke}{rgb}{0.000000,0.000000,0.000000}%
\pgfsetstrokecolor{currentstroke}%
\pgfsetdash{}{0pt}%
\pgfpathmoveto{\pgfqpoint{0.634712in}{0.818638in}}%
\pgfpathlineto{\pgfqpoint{0.635030in}{0.820175in}}%
\pgfpathlineto{\pgfqpoint{0.634712in}{0.820351in}}%
\pgfpathlineto{\pgfqpoint{0.634448in}{0.820175in}}%
\pgfpathlineto{\pgfqpoint{0.634712in}{0.818638in}}%
\pgfusepath{stroke}%
\end{pgfscope}%
\begin{pgfscope}%
\pgfpathrectangle{\pgfqpoint{0.625000in}{0.550000in}}{\pgfqpoint{3.875000in}{3.850000in}} %
\pgfusepath{clip}%
\pgfsetbuttcap%
\pgfsetroundjoin%
\pgfsetlinewidth{0.501875pt}%
\definecolor{currentstroke}{rgb}{0.000000,0.000000,0.000000}%
\pgfsetstrokecolor{currentstroke}%
\pgfsetdash{}{0pt}%
\pgfpathmoveto{\pgfqpoint{0.634712in}{2.409369in}}%
\pgfpathlineto{\pgfqpoint{0.640160in}{2.412281in}}%
\pgfpathlineto{\pgfqpoint{0.634712in}{2.415366in}}%
\pgfpathlineto{\pgfqpoint{0.626289in}{2.412281in}}%
\pgfpathlineto{\pgfqpoint{0.634712in}{2.409369in}}%
\pgfusepath{stroke}%
\end{pgfscope}%
\begin{pgfscope}%
\pgfpathrectangle{\pgfqpoint{0.625000in}{0.550000in}}{\pgfqpoint{3.875000in}{3.850000in}} %
\pgfusepath{clip}%
\pgfsetbuttcap%
\pgfsetroundjoin%
\pgfsetlinewidth{0.501875pt}%
\definecolor{currentstroke}{rgb}{0.000000,0.000000,0.000000}%
\pgfsetstrokecolor{currentstroke}%
\pgfsetdash{}{0pt}%
\pgfpathmoveto{\pgfqpoint{0.644424in}{2.417557in}}%
\pgfpathlineto{\pgfqpoint{0.650777in}{2.421930in}}%
\pgfpathlineto{\pgfqpoint{0.654135in}{2.424098in}}%
\pgfpathlineto{\pgfqpoint{0.657287in}{2.431579in}}%
\pgfpathlineto{\pgfqpoint{0.654135in}{2.435194in}}%
\pgfpathlineto{\pgfqpoint{0.647780in}{2.431579in}}%
\pgfpathlineto{\pgfqpoint{0.644424in}{2.426845in}}%
\pgfpathlineto{\pgfqpoint{0.642154in}{2.421930in}}%
\pgfpathlineto{\pgfqpoint{0.644424in}{2.417557in}}%
\pgfusepath{stroke}%
\end{pgfscope}%
\begin{pgfscope}%
\pgfpathrectangle{\pgfqpoint{0.625000in}{0.550000in}}{\pgfqpoint{3.875000in}{3.850000in}} %
\pgfusepath{clip}%
\pgfsetbuttcap%
\pgfsetroundjoin%
\pgfsetlinewidth{0.501875pt}%
\definecolor{currentstroke}{rgb}{0.000000,0.000000,0.000000}%
\pgfsetstrokecolor{currentstroke}%
\pgfsetdash{}{0pt}%
\pgfpathmoveto{\pgfqpoint{0.634712in}{2.430774in}}%
\pgfpathlineto{\pgfqpoint{0.637010in}{2.431579in}}%
\pgfpathlineto{\pgfqpoint{0.634712in}{2.434082in}}%
\pgfpathlineto{\pgfqpoint{0.633526in}{2.431579in}}%
\pgfpathlineto{\pgfqpoint{0.634712in}{2.430774in}}%
\pgfusepath{stroke}%
\end{pgfscope}%
\begin{pgfscope}%
\pgfpathrectangle{\pgfqpoint{0.625000in}{0.550000in}}{\pgfqpoint{3.875000in}{3.850000in}} %
\pgfusepath{clip}%
\pgfsetbuttcap%
\pgfsetroundjoin%
\pgfsetlinewidth{0.501875pt}%
\definecolor{currentstroke}{rgb}{0.000000,0.000000,0.000000}%
\pgfsetstrokecolor{currentstroke}%
\pgfsetdash{}{0pt}%
\pgfpathmoveto{\pgfqpoint{0.634712in}{2.444387in}}%
\pgfpathlineto{\pgfqpoint{0.644424in}{2.449666in}}%
\pgfpathlineto{\pgfqpoint{0.645738in}{2.450877in}}%
\pgfpathlineto{\pgfqpoint{0.644424in}{2.452193in}}%
\pgfpathlineto{\pgfqpoint{0.634712in}{2.454634in}}%
\pgfpathlineto{\pgfqpoint{0.628373in}{2.450877in}}%
\pgfpathlineto{\pgfqpoint{0.634712in}{2.444387in}}%
\pgfusepath{stroke}%
\end{pgfscope}%
\begin{pgfscope}%
\pgfpathrectangle{\pgfqpoint{0.625000in}{0.550000in}}{\pgfqpoint{3.875000in}{3.850000in}} %
\pgfusepath{clip}%
\pgfsetbuttcap%
\pgfsetroundjoin%
\pgfsetlinewidth{0.501875pt}%
\definecolor{currentstroke}{rgb}{0.000000,0.000000,0.000000}%
\pgfsetstrokecolor{currentstroke}%
\pgfsetdash{}{0pt}%
\pgfpathmoveto{\pgfqpoint{0.663847in}{2.456272in}}%
\pgfpathlineto{\pgfqpoint{0.673559in}{2.458850in}}%
\pgfpathlineto{\pgfqpoint{0.676516in}{2.460526in}}%
\pgfpathlineto{\pgfqpoint{0.673559in}{2.461479in}}%
\pgfpathlineto{\pgfqpoint{0.665175in}{2.470175in}}%
\pgfpathlineto{\pgfqpoint{0.663847in}{2.474513in}}%
\pgfpathlineto{\pgfqpoint{0.662411in}{2.470175in}}%
\pgfpathlineto{\pgfqpoint{0.658784in}{2.460526in}}%
\pgfpathlineto{\pgfqpoint{0.663847in}{2.456272in}}%
\pgfusepath{stroke}%
\end{pgfscope}%
\begin{pgfscope}%
\pgfpathrectangle{\pgfqpoint{0.625000in}{0.550000in}}{\pgfqpoint{3.875000in}{3.850000in}} %
\pgfusepath{clip}%
\pgfsetbuttcap%
\pgfsetroundjoin%
\pgfsetlinewidth{0.501875pt}%
\definecolor{currentstroke}{rgb}{0.000000,0.000000,0.000000}%
\pgfsetstrokecolor{currentstroke}%
\pgfsetdash{}{0pt}%
\pgfpathmoveto{\pgfqpoint{0.634712in}{2.505174in}}%
\pgfpathlineto{\pgfqpoint{0.644424in}{2.506980in}}%
\pgfpathlineto{\pgfqpoint{0.645738in}{2.508772in}}%
\pgfpathlineto{\pgfqpoint{0.644424in}{2.509983in}}%
\pgfpathlineto{\pgfqpoint{0.634712in}{2.515262in}}%
\pgfpathlineto{\pgfqpoint{0.628373in}{2.508772in}}%
\pgfpathlineto{\pgfqpoint{0.634712in}{2.505174in}}%
\pgfusepath{stroke}%
\end{pgfscope}%
\begin{pgfscope}%
\pgfpathrectangle{\pgfqpoint{0.625000in}{0.550000in}}{\pgfqpoint{3.875000in}{3.850000in}} %
\pgfusepath{clip}%
\pgfsetbuttcap%
\pgfsetroundjoin%
\pgfsetlinewidth{0.501875pt}%
\definecolor{currentstroke}{rgb}{0.000000,0.000000,0.000000}%
\pgfsetstrokecolor{currentstroke}%
\pgfsetdash{}{0pt}%
\pgfpathmoveto{\pgfqpoint{0.634712in}{2.525567in}}%
\pgfpathlineto{\pgfqpoint{0.637010in}{2.528070in}}%
\pgfpathlineto{\pgfqpoint{0.634712in}{2.528876in}}%
\pgfpathlineto{\pgfqpoint{0.633526in}{2.528070in}}%
\pgfpathlineto{\pgfqpoint{0.634712in}{2.525567in}}%
\pgfusepath{stroke}%
\end{pgfscope}%
\begin{pgfscope}%
\pgfpathrectangle{\pgfqpoint{0.625000in}{0.550000in}}{\pgfqpoint{3.875000in}{3.850000in}} %
\pgfusepath{clip}%
\pgfsetbuttcap%
\pgfsetroundjoin%
\pgfsetlinewidth{0.501875pt}%
\definecolor{currentstroke}{rgb}{0.000000,0.000000,0.000000}%
\pgfsetstrokecolor{currentstroke}%
\pgfsetdash{}{0pt}%
\pgfpathmoveto{\pgfqpoint{0.654135in}{2.524455in}}%
\pgfpathlineto{\pgfqpoint{0.657287in}{2.528070in}}%
\pgfpathlineto{\pgfqpoint{0.654135in}{2.535551in}}%
\pgfpathlineto{\pgfqpoint{0.650777in}{2.537719in}}%
\pgfpathlineto{\pgfqpoint{0.644424in}{2.542092in}}%
\pgfpathlineto{\pgfqpoint{0.642154in}{2.537719in}}%
\pgfpathlineto{\pgfqpoint{0.644424in}{2.532804in}}%
\pgfpathlineto{\pgfqpoint{0.647780in}{2.528070in}}%
\pgfpathlineto{\pgfqpoint{0.654135in}{2.524455in}}%
\pgfusepath{stroke}%
\end{pgfscope}%
\begin{pgfscope}%
\pgfpathrectangle{\pgfqpoint{0.625000in}{0.550000in}}{\pgfqpoint{3.875000in}{3.850000in}} %
\pgfusepath{clip}%
\pgfsetbuttcap%
\pgfsetroundjoin%
\pgfsetlinewidth{0.501875pt}%
\definecolor{currentstroke}{rgb}{0.000000,0.000000,0.000000}%
\pgfsetstrokecolor{currentstroke}%
\pgfsetdash{}{0pt}%
\pgfpathmoveto{\pgfqpoint{0.634712in}{2.544283in}}%
\pgfpathlineto{\pgfqpoint{0.640160in}{2.547368in}}%
\pgfpathlineto{\pgfqpoint{0.634712in}{2.550280in}}%
\pgfpathlineto{\pgfqpoint{0.626289in}{2.547368in}}%
\pgfpathlineto{\pgfqpoint{0.634712in}{2.544283in}}%
\pgfusepath{stroke}%
\end{pgfscope}%
\begin{pgfscope}%
\pgfpathrectangle{\pgfqpoint{0.625000in}{0.550000in}}{\pgfqpoint{3.875000in}{3.850000in}} %
\pgfusepath{clip}%
\pgfsetbuttcap%
\pgfsetroundjoin%
\pgfsetlinewidth{0.501875pt}%
\definecolor{currentstroke}{rgb}{0.000000,0.000000,0.000000}%
\pgfsetstrokecolor{currentstroke}%
\pgfsetdash{}{0pt}%
\pgfpathmoveto{\pgfqpoint{0.634712in}{4.139299in}}%
\pgfpathlineto{\pgfqpoint{0.635030in}{4.139474in}}%
\pgfpathlineto{\pgfqpoint{0.634712in}{4.141011in}}%
\pgfpathlineto{\pgfqpoint{0.634448in}{4.139474in}}%
\pgfpathlineto{\pgfqpoint{0.634712in}{4.139299in}}%
\pgfusepath{stroke}%
\end{pgfscope}%
\begin{pgfscope}%
\pgfpathrectangle{\pgfqpoint{0.625000in}{0.550000in}}{\pgfqpoint{3.875000in}{3.850000in}} %
\pgfusepath{clip}%
\pgfsetbuttcap%
\pgfsetroundjoin%
\pgfsetlinewidth{0.501875pt}%
\definecolor{currentstroke}{rgb}{0.000000,0.000000,0.000000}%
\pgfsetstrokecolor{currentstroke}%
\pgfsetdash{}{0pt}%
\pgfpathmoveto{\pgfqpoint{0.625000in}{0.606717in}}%
\pgfpathlineto{\pgfqpoint{0.626153in}{0.607895in}}%
\pgfpathlineto{\pgfqpoint{0.625000in}{0.608902in}}%
\pgfusepath{stroke}%
\end{pgfscope}%
\begin{pgfscope}%
\pgfpathrectangle{\pgfqpoint{0.625000in}{0.550000in}}{\pgfqpoint{3.875000in}{3.850000in}} %
\pgfusepath{clip}%
\pgfsetbuttcap%
\pgfsetroundjoin%
\pgfsetlinewidth{0.501875pt}%
\definecolor{currentstroke}{rgb}{0.000000,0.000000,0.000000}%
\pgfsetstrokecolor{currentstroke}%
\pgfsetdash{}{0pt}%
\pgfpathmoveto{\pgfqpoint{0.625000in}{0.624066in}}%
\pgfpathlineto{\pgfqpoint{0.628495in}{0.627193in}}%
\pgfpathlineto{\pgfqpoint{0.625000in}{0.630271in}}%
\pgfusepath{stroke}%
\end{pgfscope}%
\begin{pgfscope}%
\pgfpathrectangle{\pgfqpoint{0.625000in}{0.550000in}}{\pgfqpoint{3.875000in}{3.850000in}} %
\pgfusepath{clip}%
\pgfsetbuttcap%
\pgfsetroundjoin%
\pgfsetlinewidth{0.501875pt}%
\definecolor{currentstroke}{rgb}{0.000000,0.000000,0.000000}%
\pgfsetstrokecolor{currentstroke}%
\pgfsetdash{}{0pt}%
\pgfpathmoveto{\pgfqpoint{0.625000in}{0.761045in}}%
\pgfpathlineto{\pgfqpoint{0.626406in}{0.762281in}}%
\pgfpathlineto{\pgfqpoint{0.625000in}{0.765318in}}%
\pgfusepath{stroke}%
\end{pgfscope}%
\begin{pgfscope}%
\pgfpathrectangle{\pgfqpoint{0.625000in}{0.550000in}}{\pgfqpoint{3.875000in}{3.850000in}} %
\pgfusepath{clip}%
\pgfsetbuttcap%
\pgfsetroundjoin%
\pgfsetlinewidth{0.501875pt}%
\definecolor{currentstroke}{rgb}{0.000000,0.000000,0.000000}%
\pgfsetstrokecolor{currentstroke}%
\pgfsetdash{}{0pt}%
\pgfpathmoveto{\pgfqpoint{0.625000in}{0.799687in}}%
\pgfpathlineto{\pgfqpoint{0.626636in}{0.800877in}}%
\pgfpathlineto{\pgfqpoint{0.625000in}{0.806248in}}%
\pgfusepath{stroke}%
\end{pgfscope}%
\begin{pgfscope}%
\pgfpathrectangle{\pgfqpoint{0.625000in}{0.550000in}}{\pgfqpoint{3.875000in}{3.850000in}} %
\pgfusepath{clip}%
\pgfsetbuttcap%
\pgfsetroundjoin%
\pgfsetlinewidth{0.501875pt}%
\definecolor{currentstroke}{rgb}{0.000000,0.000000,0.000000}%
\pgfsetstrokecolor{currentstroke}%
\pgfsetdash{}{0pt}%
\pgfpathmoveto{\pgfqpoint{0.625000in}{0.915447in}}%
\pgfpathlineto{\pgfqpoint{0.634712in}{0.910705in}}%
\pgfpathlineto{\pgfqpoint{0.640032in}{0.916667in}}%
\pgfpathlineto{\pgfqpoint{0.640496in}{0.926316in}}%
\pgfpathlineto{\pgfqpoint{0.634712in}{0.929477in}}%
\pgfpathlineto{\pgfqpoint{0.632200in}{0.926316in}}%
\pgfpathlineto{\pgfqpoint{0.625000in}{0.917478in}}%
\pgfusepath{stroke}%
\end{pgfscope}%
\begin{pgfscope}%
\pgfpathrectangle{\pgfqpoint{0.625000in}{0.550000in}}{\pgfqpoint{3.875000in}{3.850000in}} %
\pgfusepath{clip}%
\pgfsetbuttcap%
\pgfsetroundjoin%
\pgfsetlinewidth{0.501875pt}%
\definecolor{currentstroke}{rgb}{0.000000,0.000000,0.000000}%
\pgfsetstrokecolor{currentstroke}%
\pgfsetdash{}{0pt}%
\pgfpathmoveto{\pgfqpoint{0.625000in}{0.953137in}}%
\pgfpathlineto{\pgfqpoint{0.626861in}{0.955263in}}%
\pgfpathlineto{\pgfqpoint{0.625000in}{0.956104in}}%
\pgfusepath{stroke}%
\end{pgfscope}%
\begin{pgfscope}%
\pgfpathrectangle{\pgfqpoint{0.625000in}{0.550000in}}{\pgfqpoint{3.875000in}{3.850000in}} %
\pgfusepath{clip}%
\pgfsetbuttcap%
\pgfsetroundjoin%
\pgfsetlinewidth{0.501875pt}%
\definecolor{currentstroke}{rgb}{0.000000,0.000000,0.000000}%
\pgfsetstrokecolor{currentstroke}%
\pgfsetdash{}{0pt}%
\pgfpathmoveto{\pgfqpoint{0.625000in}{1.068304in}}%
\pgfpathlineto{\pgfqpoint{0.627462in}{1.071053in}}%
\pgfpathlineto{\pgfqpoint{0.625000in}{1.072104in}}%
\pgfusepath{stroke}%
\end{pgfscope}%
\begin{pgfscope}%
\pgfpathrectangle{\pgfqpoint{0.625000in}{0.550000in}}{\pgfqpoint{3.875000in}{3.850000in}} %
\pgfusepath{clip}%
\pgfsetbuttcap%
\pgfsetroundjoin%
\pgfsetlinewidth{0.501875pt}%
\definecolor{currentstroke}{rgb}{0.000000,0.000000,0.000000}%
\pgfsetstrokecolor{currentstroke}%
\pgfsetdash{}{0pt}%
\pgfpathmoveto{\pgfqpoint{0.625000in}{1.108465in}}%
\pgfpathlineto{\pgfqpoint{0.629411in}{1.109649in}}%
\pgfpathlineto{\pgfqpoint{0.625000in}{1.110675in}}%
\pgfusepath{stroke}%
\end{pgfscope}%
\begin{pgfscope}%
\pgfpathrectangle{\pgfqpoint{0.625000in}{0.550000in}}{\pgfqpoint{3.875000in}{3.850000in}} %
\pgfusepath{clip}%
\pgfsetbuttcap%
\pgfsetroundjoin%
\pgfsetlinewidth{0.501875pt}%
\definecolor{currentstroke}{rgb}{0.000000,0.000000,0.000000}%
\pgfsetstrokecolor{currentstroke}%
\pgfsetdash{}{0pt}%
\pgfpathmoveto{\pgfqpoint{0.625000in}{1.127922in}}%
\pgfpathlineto{\pgfqpoint{0.634712in}{1.123176in}}%
\pgfpathlineto{\pgfqpoint{0.644424in}{1.124675in}}%
\pgfpathlineto{\pgfqpoint{0.652081in}{1.128947in}}%
\pgfpathlineto{\pgfqpoint{0.654135in}{1.130168in}}%
\pgfpathlineto{\pgfqpoint{0.661858in}{1.138596in}}%
\pgfpathlineto{\pgfqpoint{0.663847in}{1.142436in}}%
\pgfpathlineto{\pgfqpoint{0.666244in}{1.148246in}}%
\pgfpathlineto{\pgfqpoint{0.667401in}{1.157895in}}%
\pgfpathlineto{\pgfqpoint{0.665422in}{1.167544in}}%
\pgfpathlineto{\pgfqpoint{0.663847in}{1.169735in}}%
\pgfpathlineto{\pgfqpoint{0.660651in}{1.177193in}}%
\pgfpathlineto{\pgfqpoint{0.654135in}{1.182311in}}%
\pgfpathlineto{\pgfqpoint{0.650398in}{1.186842in}}%
\pgfpathlineto{\pgfqpoint{0.644424in}{1.190489in}}%
\pgfpathlineto{\pgfqpoint{0.636280in}{1.186842in}}%
\pgfpathlineto{\pgfqpoint{0.644424in}{1.180046in}}%
\pgfpathlineto{\pgfqpoint{0.648044in}{1.177193in}}%
\pgfpathlineto{\pgfqpoint{0.645624in}{1.167544in}}%
\pgfpathlineto{\pgfqpoint{0.644424in}{1.165501in}}%
\pgfpathlineto{\pgfqpoint{0.637398in}{1.157895in}}%
\pgfpathlineto{\pgfqpoint{0.634712in}{1.155938in}}%
\pgfpathlineto{\pgfqpoint{0.631429in}{1.148246in}}%
\pgfpathlineto{\pgfqpoint{0.631450in}{1.138596in}}%
\pgfpathlineto{\pgfqpoint{0.625000in}{1.130217in}}%
\pgfusepath{stroke}%
\end{pgfscope}%
\begin{pgfscope}%
\pgfpathrectangle{\pgfqpoint{0.625000in}{0.550000in}}{\pgfqpoint{3.875000in}{3.850000in}} %
\pgfusepath{clip}%
\pgfsetbuttcap%
\pgfsetroundjoin%
\pgfsetlinewidth{0.501875pt}%
\definecolor{currentstroke}{rgb}{0.000000,0.000000,0.000000}%
\pgfsetstrokecolor{currentstroke}%
\pgfsetdash{}{0pt}%
\pgfpathmoveto{\pgfqpoint{0.625000in}{1.176327in}}%
\pgfpathlineto{\pgfqpoint{0.627183in}{1.177193in}}%
\pgfpathlineto{\pgfqpoint{0.625000in}{1.178488in}}%
\pgfusepath{stroke}%
\end{pgfscope}%
\begin{pgfscope}%
\pgfpathrectangle{\pgfqpoint{0.625000in}{0.550000in}}{\pgfqpoint{3.875000in}{3.850000in}} %
\pgfusepath{clip}%
\pgfsetbuttcap%
\pgfsetroundjoin%
\pgfsetlinewidth{0.501875pt}%
\definecolor{currentstroke}{rgb}{0.000000,0.000000,0.000000}%
\pgfsetstrokecolor{currentstroke}%
\pgfsetdash{}{0pt}%
\pgfpathmoveto{\pgfqpoint{0.625000in}{1.224383in}}%
\pgfpathlineto{\pgfqpoint{0.634712in}{1.222475in}}%
\pgfpathlineto{\pgfqpoint{0.640414in}{1.215789in}}%
\pgfpathlineto{\pgfqpoint{0.639912in}{1.206140in}}%
\pgfpathlineto{\pgfqpoint{0.644424in}{1.202248in}}%
\pgfpathlineto{\pgfqpoint{0.649810in}{1.206140in}}%
\pgfpathlineto{\pgfqpoint{0.652168in}{1.215789in}}%
\pgfpathlineto{\pgfqpoint{0.651771in}{1.225439in}}%
\pgfpathlineto{\pgfqpoint{0.645711in}{1.235088in}}%
\pgfpathlineto{\pgfqpoint{0.644424in}{1.236427in}}%
\pgfpathlineto{\pgfqpoint{0.634712in}{1.241774in}}%
\pgfpathlineto{\pgfqpoint{0.631589in}{1.235088in}}%
\pgfpathlineto{\pgfqpoint{0.625000in}{1.227035in}}%
\pgfusepath{stroke}%
\end{pgfscope}%
\begin{pgfscope}%
\pgfpathrectangle{\pgfqpoint{0.625000in}{0.550000in}}{\pgfqpoint{3.875000in}{3.850000in}} %
\pgfusepath{clip}%
\pgfsetbuttcap%
\pgfsetroundjoin%
\pgfsetlinewidth{0.501875pt}%
\definecolor{currentstroke}{rgb}{0.000000,0.000000,0.000000}%
\pgfsetstrokecolor{currentstroke}%
\pgfsetdash{}{0pt}%
\pgfpathmoveto{\pgfqpoint{0.625000in}{1.242523in}}%
\pgfpathlineto{\pgfqpoint{0.630694in}{1.244737in}}%
\pgfpathlineto{\pgfqpoint{0.625000in}{1.246339in}}%
\pgfusepath{stroke}%
\end{pgfscope}%
\begin{pgfscope}%
\pgfpathrectangle{\pgfqpoint{0.625000in}{0.550000in}}{\pgfqpoint{3.875000in}{3.850000in}} %
\pgfusepath{clip}%
\pgfsetbuttcap%
\pgfsetroundjoin%
\pgfsetlinewidth{0.501875pt}%
\definecolor{currentstroke}{rgb}{0.000000,0.000000,0.000000}%
\pgfsetstrokecolor{currentstroke}%
\pgfsetdash{}{0pt}%
\pgfpathmoveto{\pgfqpoint{0.625000in}{1.262906in}}%
\pgfpathlineto{\pgfqpoint{0.626608in}{1.264035in}}%
\pgfpathlineto{\pgfqpoint{0.625000in}{1.265202in}}%
\pgfusepath{stroke}%
\end{pgfscope}%
\begin{pgfscope}%
\pgfpathrectangle{\pgfqpoint{0.625000in}{0.550000in}}{\pgfqpoint{3.875000in}{3.850000in}} %
\pgfusepath{clip}%
\pgfsetbuttcap%
\pgfsetroundjoin%
\pgfsetlinewidth{0.501875pt}%
\definecolor{currentstroke}{rgb}{0.000000,0.000000,0.000000}%
\pgfsetstrokecolor{currentstroke}%
\pgfsetdash{}{0pt}%
\pgfpathmoveto{\pgfqpoint{0.625000in}{1.367517in}}%
\pgfpathlineto{\pgfqpoint{0.629054in}{1.370175in}}%
\pgfpathlineto{\pgfqpoint{0.626018in}{1.379825in}}%
\pgfpathlineto{\pgfqpoint{0.625000in}{1.381260in}}%
\pgfusepath{stroke}%
\end{pgfscope}%
\begin{pgfscope}%
\pgfpathrectangle{\pgfqpoint{0.625000in}{0.550000in}}{\pgfqpoint{3.875000in}{3.850000in}} %
\pgfusepath{clip}%
\pgfsetbuttcap%
\pgfsetroundjoin%
\pgfsetlinewidth{0.501875pt}%
\definecolor{currentstroke}{rgb}{0.000000,0.000000,0.000000}%
\pgfsetstrokecolor{currentstroke}%
\pgfsetdash{}{0pt}%
\pgfpathmoveto{\pgfqpoint{0.625000in}{1.424211in}}%
\pgfpathlineto{\pgfqpoint{0.627961in}{1.428070in}}%
\pgfpathlineto{\pgfqpoint{0.625000in}{1.431929in}}%
\pgfusepath{stroke}%
\end{pgfscope}%
\begin{pgfscope}%
\pgfpathrectangle{\pgfqpoint{0.625000in}{0.550000in}}{\pgfqpoint{3.875000in}{3.850000in}} %
\pgfusepath{clip}%
\pgfsetbuttcap%
\pgfsetroundjoin%
\pgfsetlinewidth{0.501875pt}%
\definecolor{currentstroke}{rgb}{0.000000,0.000000,0.000000}%
\pgfsetstrokecolor{currentstroke}%
\pgfsetdash{}{0pt}%
\pgfpathmoveto{\pgfqpoint{0.625000in}{1.531803in}}%
\pgfpathlineto{\pgfqpoint{0.626875in}{1.534211in}}%
\pgfpathlineto{\pgfqpoint{0.628732in}{1.543860in}}%
\pgfpathlineto{\pgfqpoint{0.625000in}{1.547205in}}%
\pgfusepath{stroke}%
\end{pgfscope}%
\begin{pgfscope}%
\pgfpathrectangle{\pgfqpoint{0.625000in}{0.550000in}}{\pgfqpoint{3.875000in}{3.850000in}} %
\pgfusepath{clip}%
\pgfsetbuttcap%
\pgfsetroundjoin%
\pgfsetlinewidth{0.501875pt}%
\definecolor{currentstroke}{rgb}{0.000000,0.000000,0.000000}%
\pgfsetstrokecolor{currentstroke}%
\pgfsetdash{}{0pt}%
\pgfpathmoveto{\pgfqpoint{0.625000in}{1.571398in}}%
\pgfpathlineto{\pgfqpoint{0.627130in}{1.572807in}}%
\pgfpathlineto{\pgfqpoint{0.625000in}{1.573640in}}%
\pgfusepath{stroke}%
\end{pgfscope}%
\begin{pgfscope}%
\pgfpathrectangle{\pgfqpoint{0.625000in}{0.550000in}}{\pgfqpoint{3.875000in}{3.850000in}} %
\pgfusepath{clip}%
\pgfsetbuttcap%
\pgfsetroundjoin%
\pgfsetlinewidth{0.501875pt}%
\definecolor{currentstroke}{rgb}{0.000000,0.000000,0.000000}%
\pgfsetstrokecolor{currentstroke}%
\pgfsetdash{}{0pt}%
\pgfpathmoveto{\pgfqpoint{0.625000in}{1.610331in}}%
\pgfpathlineto{\pgfqpoint{0.629599in}{1.611404in}}%
\pgfpathlineto{\pgfqpoint{0.625000in}{1.612514in}}%
\pgfusepath{stroke}%
\end{pgfscope}%
\begin{pgfscope}%
\pgfpathrectangle{\pgfqpoint{0.625000in}{0.550000in}}{\pgfqpoint{3.875000in}{3.850000in}} %
\pgfusepath{clip}%
\pgfsetbuttcap%
\pgfsetroundjoin%
\pgfsetlinewidth{0.501875pt}%
\definecolor{currentstroke}{rgb}{0.000000,0.000000,0.000000}%
\pgfsetstrokecolor{currentstroke}%
\pgfsetdash{}{0pt}%
\pgfpathmoveto{\pgfqpoint{0.625000in}{1.659636in}}%
\pgfpathlineto{\pgfqpoint{0.625063in}{1.659649in}}%
\pgfpathlineto{\pgfqpoint{0.625000in}{1.659662in}}%
\pgfusepath{stroke}%
\end{pgfscope}%
\begin{pgfscope}%
\pgfpathrectangle{\pgfqpoint{0.625000in}{0.550000in}}{\pgfqpoint{3.875000in}{3.850000in}} %
\pgfusepath{clip}%
\pgfsetbuttcap%
\pgfsetroundjoin%
\pgfsetlinewidth{0.501875pt}%
\definecolor{currentstroke}{rgb}{0.000000,0.000000,0.000000}%
\pgfsetstrokecolor{currentstroke}%
\pgfsetdash{}{0pt}%
\pgfpathmoveto{\pgfqpoint{0.625000in}{1.685170in}}%
\pgfpathlineto{\pgfqpoint{0.629852in}{1.678947in}}%
\pgfpathlineto{\pgfqpoint{0.634712in}{1.671093in}}%
\pgfpathlineto{\pgfqpoint{0.644424in}{1.676800in}}%
\pgfpathlineto{\pgfqpoint{0.646195in}{1.678947in}}%
\pgfpathlineto{\pgfqpoint{0.649705in}{1.688596in}}%
\pgfpathlineto{\pgfqpoint{0.646542in}{1.698246in}}%
\pgfpathlineto{\pgfqpoint{0.644424in}{1.699391in}}%
\pgfpathlineto{\pgfqpoint{0.634712in}{1.699359in}}%
\pgfpathlineto{\pgfqpoint{0.633965in}{1.698246in}}%
\pgfpathlineto{\pgfqpoint{0.625000in}{1.689772in}}%
\pgfusepath{stroke}%
\end{pgfscope}%
\begin{pgfscope}%
\pgfpathrectangle{\pgfqpoint{0.625000in}{0.550000in}}{\pgfqpoint{3.875000in}{3.850000in}} %
\pgfusepath{clip}%
\pgfsetbuttcap%
\pgfsetroundjoin%
\pgfsetlinewidth{0.501875pt}%
\definecolor{currentstroke}{rgb}{0.000000,0.000000,0.000000}%
\pgfsetstrokecolor{currentstroke}%
\pgfsetdash{}{0pt}%
\pgfpathmoveto{\pgfqpoint{0.625000in}{1.722437in}}%
\pgfpathlineto{\pgfqpoint{0.626153in}{1.727193in}}%
\pgfpathlineto{\pgfqpoint{0.625000in}{1.728473in}}%
\pgfusepath{stroke}%
\end{pgfscope}%
\begin{pgfscope}%
\pgfpathrectangle{\pgfqpoint{0.625000in}{0.550000in}}{\pgfqpoint{3.875000in}{3.850000in}} %
\pgfusepath{clip}%
\pgfsetbuttcap%
\pgfsetroundjoin%
\pgfsetlinewidth{0.501875pt}%
\definecolor{currentstroke}{rgb}{0.000000,0.000000,0.000000}%
\pgfsetstrokecolor{currentstroke}%
\pgfsetdash{}{0pt}%
\pgfpathmoveto{\pgfqpoint{0.625000in}{1.810484in}}%
\pgfpathlineto{\pgfqpoint{0.628941in}{1.814035in}}%
\pgfpathlineto{\pgfqpoint{0.625000in}{1.817586in}}%
\pgfusepath{stroke}%
\end{pgfscope}%
\begin{pgfscope}%
\pgfpathrectangle{\pgfqpoint{0.625000in}{0.550000in}}{\pgfqpoint{3.875000in}{3.850000in}} %
\pgfusepath{clip}%
\pgfsetbuttcap%
\pgfsetroundjoin%
\pgfsetlinewidth{0.501875pt}%
\definecolor{currentstroke}{rgb}{0.000000,0.000000,0.000000}%
\pgfsetstrokecolor{currentstroke}%
\pgfsetdash{}{0pt}%
\pgfpathmoveto{\pgfqpoint{0.625000in}{1.841614in}}%
\pgfpathlineto{\pgfqpoint{0.628881in}{1.842982in}}%
\pgfpathlineto{\pgfqpoint{0.631978in}{1.852632in}}%
\pgfpathlineto{\pgfqpoint{0.625000in}{1.853762in}}%
\pgfusepath{stroke}%
\end{pgfscope}%
\begin{pgfscope}%
\pgfpathrectangle{\pgfqpoint{0.625000in}{0.550000in}}{\pgfqpoint{3.875000in}{3.850000in}} %
\pgfusepath{clip}%
\pgfsetbuttcap%
\pgfsetroundjoin%
\pgfsetlinewidth{0.501875pt}%
\definecolor{currentstroke}{rgb}{0.000000,0.000000,0.000000}%
\pgfsetstrokecolor{currentstroke}%
\pgfsetdash{}{0pt}%
\pgfpathmoveto{\pgfqpoint{0.625000in}{1.880728in}}%
\pgfpathlineto{\pgfqpoint{0.634477in}{1.871930in}}%
\pgfpathlineto{\pgfqpoint{0.634597in}{1.862281in}}%
\pgfpathlineto{\pgfqpoint{0.634712in}{1.860476in}}%
\pgfpathlineto{\pgfqpoint{0.635871in}{1.862281in}}%
\pgfpathlineto{\pgfqpoint{0.639450in}{1.871930in}}%
\pgfpathlineto{\pgfqpoint{0.643645in}{1.881579in}}%
\pgfpathlineto{\pgfqpoint{0.644424in}{1.887650in}}%
\pgfpathlineto{\pgfqpoint{0.646104in}{1.891228in}}%
\pgfpathlineto{\pgfqpoint{0.654135in}{1.897122in}}%
\pgfpathlineto{\pgfqpoint{0.656723in}{1.900877in}}%
\pgfpathlineto{\pgfqpoint{0.659412in}{1.910526in}}%
\pgfpathlineto{\pgfqpoint{0.654440in}{1.920175in}}%
\pgfpathlineto{\pgfqpoint{0.654135in}{1.920329in}}%
\pgfpathlineto{\pgfqpoint{0.644424in}{1.925203in}}%
\pgfpathlineto{\pgfqpoint{0.634712in}{1.926234in}}%
\pgfpathlineto{\pgfqpoint{0.630937in}{1.920175in}}%
\pgfpathlineto{\pgfqpoint{0.625000in}{1.913705in}}%
\pgfusepath{stroke}%
\end{pgfscope}%
\begin{pgfscope}%
\pgfpathrectangle{\pgfqpoint{0.625000in}{0.550000in}}{\pgfqpoint{3.875000in}{3.850000in}} %
\pgfusepath{clip}%
\pgfsetbuttcap%
\pgfsetroundjoin%
\pgfsetlinewidth{0.501875pt}%
\definecolor{currentstroke}{rgb}{0.000000,0.000000,0.000000}%
\pgfsetstrokecolor{currentstroke}%
\pgfsetdash{}{0pt}%
\pgfpathmoveto{\pgfqpoint{0.625000in}{1.898893in}}%
\pgfpathlineto{\pgfqpoint{0.634712in}{1.894976in}}%
\pgfpathlineto{\pgfqpoint{0.643833in}{1.891228in}}%
\pgfpathlineto{\pgfqpoint{0.634712in}{1.886958in}}%
\pgfpathlineto{\pgfqpoint{0.625000in}{1.883563in}}%
\pgfusepath{stroke}%
\end{pgfscope}%
\begin{pgfscope}%
\pgfpathrectangle{\pgfqpoint{0.625000in}{0.550000in}}{\pgfqpoint{3.875000in}{3.850000in}} %
\pgfusepath{clip}%
\pgfsetbuttcap%
\pgfsetroundjoin%
\pgfsetlinewidth{0.501875pt}%
\definecolor{currentstroke}{rgb}{0.000000,0.000000,0.000000}%
\pgfsetstrokecolor{currentstroke}%
\pgfsetdash{}{0pt}%
\pgfpathmoveto{\pgfqpoint{0.625000in}{1.995660in}}%
\pgfpathlineto{\pgfqpoint{0.654135in}{1.985150in}}%
\pgfpathlineto{\pgfqpoint{0.692982in}{1.970276in}}%
\pgfpathlineto{\pgfqpoint{0.712406in}{1.968020in}}%
\pgfpathlineto{\pgfqpoint{0.731830in}{1.969858in}}%
\pgfpathlineto{\pgfqpoint{0.751253in}{1.974521in}}%
\pgfpathlineto{\pgfqpoint{0.770677in}{1.981385in}}%
\pgfpathlineto{\pgfqpoint{0.790100in}{1.990287in}}%
\pgfpathlineto{\pgfqpoint{0.809524in}{2.001344in}}%
\pgfpathlineto{\pgfqpoint{0.831171in}{2.016667in}}%
\pgfpathlineto{\pgfqpoint{0.848371in}{2.031490in}}%
\pgfpathlineto{\pgfqpoint{0.862132in}{2.045614in}}%
\pgfpathlineto{\pgfqpoint{0.877860in}{2.064912in}}%
\pgfpathlineto{\pgfqpoint{0.890677in}{2.084211in}}%
\pgfpathlineto{\pgfqpoint{0.901080in}{2.103509in}}%
\pgfpathlineto{\pgfqpoint{0.909385in}{2.122807in}}%
\pgfpathlineto{\pgfqpoint{0.916353in}{2.144153in}}%
\pgfpathlineto{\pgfqpoint{0.920527in}{2.161404in}}%
\pgfpathlineto{\pgfqpoint{0.923600in}{2.180702in}}%
\pgfpathlineto{\pgfqpoint{0.925103in}{2.200000in}}%
\pgfpathlineto{\pgfqpoint{0.925097in}{2.219298in}}%
\pgfpathlineto{\pgfqpoint{0.923596in}{2.238596in}}%
\pgfpathlineto{\pgfqpoint{0.920552in}{2.257895in}}%
\pgfpathlineto{\pgfqpoint{0.915860in}{2.277193in}}%
\pgfpathlineto{\pgfqpoint{0.909606in}{2.296491in}}%
\pgfpathlineto{\pgfqpoint{0.896715in}{2.325439in}}%
\pgfpathlineto{\pgfqpoint{0.885684in}{2.344737in}}%
\pgfpathlineto{\pgfqpoint{0.872324in}{2.364035in}}%
\pgfpathlineto{\pgfqpoint{0.856118in}{2.383333in}}%
\pgfpathlineto{\pgfqpoint{0.836532in}{2.402632in}}%
\pgfpathlineto{\pgfqpoint{0.809524in}{2.423805in}}%
\pgfpathlineto{\pgfqpoint{0.790100in}{2.436206in}}%
\pgfpathlineto{\pgfqpoint{0.770677in}{2.446674in}}%
\pgfpathlineto{\pgfqpoint{0.751253in}{2.455422in}}%
\pgfpathlineto{\pgfqpoint{0.731830in}{2.462679in}}%
\pgfpathlineto{\pgfqpoint{0.722118in}{2.465718in}}%
\pgfpathlineto{\pgfqpoint{0.712406in}{2.466719in}}%
\pgfpathlineto{\pgfqpoint{0.706519in}{2.470175in}}%
\pgfpathlineto{\pgfqpoint{0.692982in}{2.473038in}}%
\pgfpathlineto{\pgfqpoint{0.683674in}{2.470175in}}%
\pgfpathlineto{\pgfqpoint{0.692982in}{2.466764in}}%
\pgfpathlineto{\pgfqpoint{0.702694in}{2.466874in}}%
\pgfpathlineto{\pgfqpoint{0.706684in}{2.460526in}}%
\pgfpathlineto{\pgfqpoint{0.731830in}{2.444254in}}%
\pgfpathlineto{\pgfqpoint{0.741541in}{2.436187in}}%
\pgfpathlineto{\pgfqpoint{0.751253in}{2.426749in}}%
\pgfpathlineto{\pgfqpoint{0.760965in}{2.415515in}}%
\pgfpathlineto{\pgfqpoint{0.763760in}{2.412281in}}%
\pgfpathlineto{\pgfqpoint{0.776028in}{2.392982in}}%
\pgfpathlineto{\pgfqpoint{0.784732in}{2.373684in}}%
\pgfpathlineto{\pgfqpoint{0.790383in}{2.354386in}}%
\pgfpathlineto{\pgfqpoint{0.793564in}{2.335088in}}%
\pgfpathlineto{\pgfqpoint{0.794146in}{2.315789in}}%
\pgfpathlineto{\pgfqpoint{0.792206in}{2.296491in}}%
\pgfpathlineto{\pgfqpoint{0.787675in}{2.277193in}}%
\pgfpathlineto{\pgfqpoint{0.780249in}{2.257895in}}%
\pgfpathlineto{\pgfqpoint{0.769462in}{2.238596in}}%
\pgfpathlineto{\pgfqpoint{0.754293in}{2.219298in}}%
\pgfpathlineto{\pgfqpoint{0.741541in}{2.207005in}}%
\pgfpathlineto{\pgfqpoint{0.722118in}{2.192911in}}%
\pgfpathlineto{\pgfqpoint{0.702694in}{2.182863in}}%
\pgfpathlineto{\pgfqpoint{0.683271in}{2.175956in}}%
\pgfpathlineto{\pgfqpoint{0.644424in}{2.166631in}}%
\pgfpathlineto{\pgfqpoint{0.633891in}{2.161404in}}%
\pgfpathlineto{\pgfqpoint{0.625000in}{2.155019in}}%
\pgfpathlineto{\pgfqpoint{0.625000in}{2.155019in}}%
\pgfusepath{stroke}%
\end{pgfscope}%
\begin{pgfscope}%
\pgfpathrectangle{\pgfqpoint{0.625000in}{0.550000in}}{\pgfqpoint{3.875000in}{3.850000in}} %
\pgfusepath{clip}%
\pgfsetbuttcap%
\pgfsetroundjoin%
\pgfsetlinewidth{0.501875pt}%
\definecolor{currentstroke}{rgb}{0.000000,0.000000,0.000000}%
\pgfsetstrokecolor{currentstroke}%
\pgfsetdash{}{0pt}%
\pgfpathmoveto{\pgfqpoint{0.625000in}{2.007204in}}%
\pgfpathlineto{\pgfqpoint{0.625248in}{2.007018in}}%
\pgfpathlineto{\pgfqpoint{0.625000in}{2.006314in}}%
\pgfusepath{stroke}%
\end{pgfscope}%
\begin{pgfscope}%
\pgfpathrectangle{\pgfqpoint{0.625000in}{0.550000in}}{\pgfqpoint{3.875000in}{3.850000in}} %
\pgfusepath{clip}%
\pgfsetbuttcap%
\pgfsetroundjoin%
\pgfsetlinewidth{0.501875pt}%
\definecolor{currentstroke}{rgb}{0.000000,0.000000,0.000000}%
\pgfsetstrokecolor{currentstroke}%
\pgfsetdash{}{0pt}%
\pgfpathmoveto{\pgfqpoint{0.625000in}{2.090216in}}%
\pgfpathlineto{\pgfqpoint{0.632120in}{2.084211in}}%
\pgfpathlineto{\pgfqpoint{0.632354in}{2.074561in}}%
\pgfpathlineto{\pgfqpoint{0.632559in}{2.064912in}}%
\pgfpathlineto{\pgfqpoint{0.633920in}{2.055263in}}%
\pgfpathlineto{\pgfqpoint{0.630951in}{2.045614in}}%
\pgfpathlineto{\pgfqpoint{0.625000in}{2.037317in}}%
\pgfusepath{stroke}%
\end{pgfscope}%
\begin{pgfscope}%
\pgfpathrectangle{\pgfqpoint{0.625000in}{0.550000in}}{\pgfqpoint{3.875000in}{3.850000in}} %
\pgfusepath{clip}%
\pgfsetbuttcap%
\pgfsetroundjoin%
\pgfsetlinewidth{0.501875pt}%
\definecolor{currentstroke}{rgb}{0.000000,0.000000,0.000000}%
\pgfsetstrokecolor{currentstroke}%
\pgfsetdash{}{0pt}%
\pgfpathmoveto{\pgfqpoint{0.625000in}{2.139380in}}%
\pgfpathlineto{\pgfqpoint{0.631150in}{2.132456in}}%
\pgfpathlineto{\pgfqpoint{0.633073in}{2.122807in}}%
\pgfpathlineto{\pgfqpoint{0.632405in}{2.113158in}}%
\pgfpathlineto{\pgfqpoint{0.629175in}{2.103509in}}%
\pgfpathlineto{\pgfqpoint{0.625000in}{2.100503in}}%
\pgfusepath{stroke}%
\end{pgfscope}%
\begin{pgfscope}%
\pgfpathrectangle{\pgfqpoint{0.625000in}{0.550000in}}{\pgfqpoint{3.875000in}{3.850000in}} %
\pgfusepath{clip}%
\pgfsetbuttcap%
\pgfsetroundjoin%
\pgfsetlinewidth{0.501875pt}%
\definecolor{currentstroke}{rgb}{0.000000,0.000000,0.000000}%
\pgfsetstrokecolor{currentstroke}%
\pgfsetdash{}{0pt}%
\pgfpathmoveto{\pgfqpoint{0.625000in}{2.187653in}}%
\pgfpathlineto{\pgfqpoint{0.631364in}{2.190351in}}%
\pgfpathlineto{\pgfqpoint{0.630209in}{2.200000in}}%
\pgfpathlineto{\pgfqpoint{0.625000in}{2.202369in}}%
\pgfusepath{stroke}%
\end{pgfscope}%
\begin{pgfscope}%
\pgfpathrectangle{\pgfqpoint{0.625000in}{0.550000in}}{\pgfqpoint{3.875000in}{3.850000in}} %
\pgfusepath{clip}%
\pgfsetbuttcap%
\pgfsetroundjoin%
\pgfsetlinewidth{0.501875pt}%
\definecolor{currentstroke}{rgb}{0.000000,0.000000,0.000000}%
\pgfsetstrokecolor{currentstroke}%
\pgfsetdash{}{0pt}%
\pgfpathmoveto{\pgfqpoint{0.625000in}{2.276596in}}%
\pgfpathlineto{\pgfqpoint{0.625480in}{2.277193in}}%
\pgfpathlineto{\pgfqpoint{0.625000in}{2.277790in}}%
\pgfusepath{stroke}%
\end{pgfscope}%
\begin{pgfscope}%
\pgfpathrectangle{\pgfqpoint{0.625000in}{0.550000in}}{\pgfqpoint{3.875000in}{3.850000in}} %
\pgfusepath{clip}%
\pgfsetbuttcap%
\pgfsetroundjoin%
\pgfsetlinewidth{0.501875pt}%
\definecolor{currentstroke}{rgb}{0.000000,0.000000,0.000000}%
\pgfsetstrokecolor{currentstroke}%
\pgfsetdash{}{0pt}%
\pgfpathmoveto{\pgfqpoint{0.625000in}{2.292520in}}%
\pgfpathlineto{\pgfqpoint{0.629867in}{2.296491in}}%
\pgfpathlineto{\pgfqpoint{0.627773in}{2.306140in}}%
\pgfpathlineto{\pgfqpoint{0.625000in}{2.307248in}}%
\pgfusepath{stroke}%
\end{pgfscope}%
\begin{pgfscope}%
\pgfpathrectangle{\pgfqpoint{0.625000in}{0.550000in}}{\pgfqpoint{3.875000in}{3.850000in}} %
\pgfusepath{clip}%
\pgfsetbuttcap%
\pgfsetroundjoin%
\pgfsetlinewidth{0.501875pt}%
\definecolor{currentstroke}{rgb}{0.000000,0.000000,0.000000}%
\pgfsetstrokecolor{currentstroke}%
\pgfsetdash{}{0pt}%
\pgfpathmoveto{\pgfqpoint{0.625000in}{2.332844in}}%
\pgfpathlineto{\pgfqpoint{0.631756in}{2.325439in}}%
\pgfpathlineto{\pgfqpoint{0.634712in}{2.315854in}}%
\pgfpathlineto{\pgfqpoint{0.644424in}{2.316763in}}%
\pgfpathlineto{\pgfqpoint{0.654135in}{2.319079in}}%
\pgfpathlineto{\pgfqpoint{0.663847in}{2.322808in}}%
\pgfpathlineto{\pgfqpoint{0.668723in}{2.325439in}}%
\pgfpathlineto{\pgfqpoint{0.673559in}{2.328056in}}%
\pgfpathlineto{\pgfqpoint{0.682991in}{2.335088in}}%
\pgfpathlineto{\pgfqpoint{0.683271in}{2.335309in}}%
\pgfpathlineto{\pgfqpoint{0.692559in}{2.344737in}}%
\pgfpathlineto{\pgfqpoint{0.692982in}{2.345227in}}%
\pgfpathlineto{\pgfqpoint{0.699506in}{2.354386in}}%
\pgfpathlineto{\pgfqpoint{0.702694in}{2.360135in}}%
\pgfpathlineto{\pgfqpoint{0.704584in}{2.364035in}}%
\pgfpathlineto{\pgfqpoint{0.708153in}{2.373684in}}%
\pgfpathlineto{\pgfqpoint{0.710358in}{2.383333in}}%
\pgfpathlineto{\pgfqpoint{0.711340in}{2.392982in}}%
\pgfpathlineto{\pgfqpoint{0.711195in}{2.402632in}}%
\pgfpathlineto{\pgfqpoint{0.709943in}{2.412281in}}%
\pgfpathlineto{\pgfqpoint{0.707472in}{2.421930in}}%
\pgfpathlineto{\pgfqpoint{0.703410in}{2.431579in}}%
\pgfpathlineto{\pgfqpoint{0.702694in}{2.432649in}}%
\pgfpathlineto{\pgfqpoint{0.698227in}{2.441228in}}%
\pgfpathlineto{\pgfqpoint{0.692982in}{2.447493in}}%
\pgfpathlineto{\pgfqpoint{0.688993in}{2.441228in}}%
\pgfpathlineto{\pgfqpoint{0.692532in}{2.431579in}}%
\pgfpathlineto{\pgfqpoint{0.692982in}{2.429266in}}%
\pgfpathlineto{\pgfqpoint{0.696201in}{2.421930in}}%
\pgfpathlineto{\pgfqpoint{0.696012in}{2.412281in}}%
\pgfpathlineto{\pgfqpoint{0.695186in}{2.402632in}}%
\pgfpathlineto{\pgfqpoint{0.693127in}{2.392982in}}%
\pgfpathlineto{\pgfqpoint{0.692982in}{2.392623in}}%
\pgfpathlineto{\pgfqpoint{0.689544in}{2.383333in}}%
\pgfpathlineto{\pgfqpoint{0.684032in}{2.373684in}}%
\pgfpathlineto{\pgfqpoint{0.683271in}{2.372576in}}%
\pgfpathlineto{\pgfqpoint{0.675946in}{2.364035in}}%
\pgfpathlineto{\pgfqpoint{0.673559in}{2.361642in}}%
\pgfpathlineto{\pgfqpoint{0.663847in}{2.354445in}}%
\pgfpathlineto{\pgfqpoint{0.663734in}{2.354386in}}%
\pgfpathlineto{\pgfqpoint{0.654135in}{2.349340in}}%
\pgfpathlineto{\pgfqpoint{0.644424in}{2.346224in}}%
\pgfpathlineto{\pgfqpoint{0.635322in}{2.344737in}}%
\pgfpathlineto{\pgfqpoint{0.634712in}{2.344627in}}%
\pgfpathlineto{\pgfqpoint{0.634313in}{2.344737in}}%
\pgfpathlineto{\pgfqpoint{0.625000in}{2.345628in}}%
\pgfusepath{stroke}%
\end{pgfscope}%
\begin{pgfscope}%
\pgfpathrectangle{\pgfqpoint{0.625000in}{0.550000in}}{\pgfqpoint{3.875000in}{3.850000in}} %
\pgfusepath{clip}%
\pgfsetbuttcap%
\pgfsetroundjoin%
\pgfsetlinewidth{0.501875pt}%
\definecolor{currentstroke}{rgb}{0.000000,0.000000,0.000000}%
\pgfsetstrokecolor{currentstroke}%
\pgfsetdash{}{0pt}%
\pgfpathmoveto{\pgfqpoint{0.625000in}{2.377856in}}%
\pgfpathlineto{\pgfqpoint{0.634712in}{2.382706in}}%
\pgfpathlineto{\pgfqpoint{0.636914in}{2.383333in}}%
\pgfpathlineto{\pgfqpoint{0.644424in}{2.386193in}}%
\pgfpathlineto{\pgfqpoint{0.654135in}{2.390356in}}%
\pgfpathlineto{\pgfqpoint{0.657332in}{2.392982in}}%
\pgfpathlineto{\pgfqpoint{0.663847in}{2.399058in}}%
\pgfpathlineto{\pgfqpoint{0.666546in}{2.402632in}}%
\pgfpathlineto{\pgfqpoint{0.668375in}{2.412281in}}%
\pgfpathlineto{\pgfqpoint{0.663847in}{2.414604in}}%
\pgfpathlineto{\pgfqpoint{0.662383in}{2.412281in}}%
\pgfpathlineto{\pgfqpoint{0.654135in}{2.403077in}}%
\pgfpathlineto{\pgfqpoint{0.653480in}{2.402632in}}%
\pgfpathlineto{\pgfqpoint{0.644424in}{2.396787in}}%
\pgfpathlineto{\pgfqpoint{0.634712in}{2.393807in}}%
\pgfpathlineto{\pgfqpoint{0.632504in}{2.392982in}}%
\pgfpathlineto{\pgfqpoint{0.625000in}{2.385684in}}%
\pgfusepath{stroke}%
\end{pgfscope}%
\begin{pgfscope}%
\pgfpathrectangle{\pgfqpoint{0.625000in}{0.550000in}}{\pgfqpoint{3.875000in}{3.850000in}} %
\pgfusepath{clip}%
\pgfsetbuttcap%
\pgfsetroundjoin%
\pgfsetlinewidth{0.501875pt}%
\definecolor{currentstroke}{rgb}{0.000000,0.000000,0.000000}%
\pgfsetstrokecolor{currentstroke}%
\pgfsetdash{}{0pt}%
\pgfpathmoveto{\pgfqpoint{0.625000in}{2.458609in}}%
\pgfpathlineto{\pgfqpoint{0.626080in}{2.460526in}}%
\pgfpathlineto{\pgfqpoint{0.628155in}{2.470175in}}%
\pgfpathlineto{\pgfqpoint{0.625000in}{2.476170in}}%
\pgfusepath{stroke}%
\end{pgfscope}%
\begin{pgfscope}%
\pgfpathrectangle{\pgfqpoint{0.625000in}{0.550000in}}{\pgfqpoint{3.875000in}{3.850000in}} %
\pgfusepath{clip}%
\pgfsetbuttcap%
\pgfsetroundjoin%
\pgfsetlinewidth{0.501875pt}%
\definecolor{currentstroke}{rgb}{0.000000,0.000000,0.000000}%
\pgfsetstrokecolor{currentstroke}%
\pgfsetdash{}{0pt}%
\pgfpathmoveto{\pgfqpoint{0.625000in}{2.483479in}}%
\pgfpathlineto{\pgfqpoint{0.634712in}{2.484277in}}%
\pgfpathlineto{\pgfqpoint{0.644424in}{2.483883in}}%
\pgfpathlineto{\pgfqpoint{0.654135in}{2.484223in}}%
\pgfpathlineto{\pgfqpoint{0.663847in}{2.487331in}}%
\pgfpathlineto{\pgfqpoint{0.664770in}{2.489474in}}%
\pgfpathlineto{\pgfqpoint{0.671206in}{2.499123in}}%
\pgfpathlineto{\pgfqpoint{0.663847in}{2.501890in}}%
\pgfpathlineto{\pgfqpoint{0.660947in}{2.499123in}}%
\pgfpathlineto{\pgfqpoint{0.654135in}{2.492282in}}%
\pgfpathlineto{\pgfqpoint{0.644424in}{2.493456in}}%
\pgfpathlineto{\pgfqpoint{0.634712in}{2.492237in}}%
\pgfpathlineto{\pgfqpoint{0.626031in}{2.499123in}}%
\pgfpathlineto{\pgfqpoint{0.625000in}{2.501040in}}%
\pgfusepath{stroke}%
\end{pgfscope}%
\begin{pgfscope}%
\pgfpathrectangle{\pgfqpoint{0.625000in}{0.550000in}}{\pgfqpoint{3.875000in}{3.850000in}} %
\pgfusepath{clip}%
\pgfsetbuttcap%
\pgfsetroundjoin%
\pgfsetlinewidth{0.501875pt}%
\definecolor{currentstroke}{rgb}{0.000000,0.000000,0.000000}%
\pgfsetstrokecolor{currentstroke}%
\pgfsetdash{}{0pt}%
\pgfpathmoveto{\pgfqpoint{0.625000in}{2.573965in}}%
\pgfpathlineto{\pgfqpoint{0.632504in}{2.566667in}}%
\pgfpathlineto{\pgfqpoint{0.634712in}{2.565842in}}%
\pgfpathlineto{\pgfqpoint{0.644424in}{2.562862in}}%
\pgfpathlineto{\pgfqpoint{0.653480in}{2.557018in}}%
\pgfpathlineto{\pgfqpoint{0.654135in}{2.556572in}}%
\pgfpathlineto{\pgfqpoint{0.662383in}{2.547368in}}%
\pgfpathlineto{\pgfqpoint{0.663847in}{2.545045in}}%
\pgfpathlineto{\pgfqpoint{0.668375in}{2.547368in}}%
\pgfpathlineto{\pgfqpoint{0.666546in}{2.557018in}}%
\pgfpathlineto{\pgfqpoint{0.663847in}{2.560591in}}%
\pgfpathlineto{\pgfqpoint{0.657332in}{2.566667in}}%
\pgfpathlineto{\pgfqpoint{0.654135in}{2.569294in}}%
\pgfpathlineto{\pgfqpoint{0.644424in}{2.573456in}}%
\pgfpathlineto{\pgfqpoint{0.636914in}{2.576316in}}%
\pgfpathlineto{\pgfqpoint{0.634712in}{2.576943in}}%
\pgfpathlineto{\pgfqpoint{0.625000in}{2.581794in}}%
\pgfusepath{stroke}%
\end{pgfscope}%
\begin{pgfscope}%
\pgfpathrectangle{\pgfqpoint{0.625000in}{0.550000in}}{\pgfqpoint{3.875000in}{3.850000in}} %
\pgfusepath{clip}%
\pgfsetbuttcap%
\pgfsetroundjoin%
\pgfsetlinewidth{0.501875pt}%
\definecolor{currentstroke}{rgb}{0.000000,0.000000,0.000000}%
\pgfsetstrokecolor{currentstroke}%
\pgfsetdash{}{0pt}%
\pgfpathmoveto{\pgfqpoint{0.625000in}{2.614021in}}%
\pgfpathlineto{\pgfqpoint{0.634313in}{2.614912in}}%
\pgfpathlineto{\pgfqpoint{0.634712in}{2.615022in}}%
\pgfpathlineto{\pgfqpoint{0.635322in}{2.614912in}}%
\pgfpathlineto{\pgfqpoint{0.644424in}{2.613425in}}%
\pgfpathlineto{\pgfqpoint{0.654135in}{2.610310in}}%
\pgfpathlineto{\pgfqpoint{0.663734in}{2.605263in}}%
\pgfpathlineto{\pgfqpoint{0.663847in}{2.605204in}}%
\pgfpathlineto{\pgfqpoint{0.673559in}{2.598007in}}%
\pgfpathlineto{\pgfqpoint{0.675946in}{2.595614in}}%
\pgfpathlineto{\pgfqpoint{0.683271in}{2.587073in}}%
\pgfpathlineto{\pgfqpoint{0.684032in}{2.585965in}}%
\pgfpathlineto{\pgfqpoint{0.689544in}{2.576316in}}%
\pgfpathlineto{\pgfqpoint{0.692982in}{2.567027in}}%
\pgfpathlineto{\pgfqpoint{0.693127in}{2.566667in}}%
\pgfpathlineto{\pgfqpoint{0.695186in}{2.557018in}}%
\pgfpathlineto{\pgfqpoint{0.696012in}{2.547368in}}%
\pgfpathlineto{\pgfqpoint{0.696201in}{2.537719in}}%
\pgfpathlineto{\pgfqpoint{0.692982in}{2.530384in}}%
\pgfpathlineto{\pgfqpoint{0.692532in}{2.528070in}}%
\pgfpathlineto{\pgfqpoint{0.688993in}{2.518421in}}%
\pgfpathlineto{\pgfqpoint{0.692982in}{2.512156in}}%
\pgfpathlineto{\pgfqpoint{0.698227in}{2.518421in}}%
\pgfpathlineto{\pgfqpoint{0.702694in}{2.527000in}}%
\pgfpathlineto{\pgfqpoint{0.703410in}{2.528070in}}%
\pgfpathlineto{\pgfqpoint{0.707472in}{2.537719in}}%
\pgfpathlineto{\pgfqpoint{0.709943in}{2.547368in}}%
\pgfpathlineto{\pgfqpoint{0.711195in}{2.557018in}}%
\pgfpathlineto{\pgfqpoint{0.711340in}{2.566667in}}%
\pgfpathlineto{\pgfqpoint{0.710358in}{2.576316in}}%
\pgfpathlineto{\pgfqpoint{0.708153in}{2.585965in}}%
\pgfpathlineto{\pgfqpoint{0.704584in}{2.595614in}}%
\pgfpathlineto{\pgfqpoint{0.702694in}{2.599514in}}%
\pgfpathlineto{\pgfqpoint{0.699506in}{2.605263in}}%
\pgfpathlineto{\pgfqpoint{0.692982in}{2.614422in}}%
\pgfpathlineto{\pgfqpoint{0.692559in}{2.614912in}}%
\pgfpathlineto{\pgfqpoint{0.683271in}{2.624341in}}%
\pgfpathlineto{\pgfqpoint{0.682991in}{2.624561in}}%
\pgfpathlineto{\pgfqpoint{0.673559in}{2.631593in}}%
\pgfpathlineto{\pgfqpoint{0.668723in}{2.634211in}}%
\pgfpathlineto{\pgfqpoint{0.663847in}{2.636841in}}%
\pgfpathlineto{\pgfqpoint{0.654135in}{2.640570in}}%
\pgfpathlineto{\pgfqpoint{0.644424in}{2.642886in}}%
\pgfpathlineto{\pgfqpoint{0.634712in}{2.643795in}}%
\pgfpathlineto{\pgfqpoint{0.625155in}{2.634211in}}%
\pgfpathlineto{\pgfqpoint{0.625000in}{2.633990in}}%
\pgfusepath{stroke}%
\end{pgfscope}%
\begin{pgfscope}%
\pgfpathrectangle{\pgfqpoint{0.625000in}{0.550000in}}{\pgfqpoint{3.875000in}{3.850000in}} %
\pgfusepath{clip}%
\pgfsetbuttcap%
\pgfsetroundjoin%
\pgfsetlinewidth{0.501875pt}%
\definecolor{currentstroke}{rgb}{0.000000,0.000000,0.000000}%
\pgfsetstrokecolor{currentstroke}%
\pgfsetdash{}{0pt}%
\pgfpathmoveto{\pgfqpoint{0.625000in}{2.652401in}}%
\pgfpathlineto{\pgfqpoint{0.627773in}{2.653509in}}%
\pgfpathlineto{\pgfqpoint{0.629867in}{2.663158in}}%
\pgfpathlineto{\pgfqpoint{0.625000in}{2.667130in}}%
\pgfusepath{stroke}%
\end{pgfscope}%
\begin{pgfscope}%
\pgfpathrectangle{\pgfqpoint{0.625000in}{0.550000in}}{\pgfqpoint{3.875000in}{3.850000in}} %
\pgfusepath{clip}%
\pgfsetbuttcap%
\pgfsetroundjoin%
\pgfsetlinewidth{0.501875pt}%
\definecolor{currentstroke}{rgb}{0.000000,0.000000,0.000000}%
\pgfsetstrokecolor{currentstroke}%
\pgfsetdash{}{0pt}%
\pgfpathmoveto{\pgfqpoint{0.625000in}{2.681859in}}%
\pgfpathlineto{\pgfqpoint{0.625480in}{2.682456in}}%
\pgfpathlineto{\pgfqpoint{0.625000in}{2.683053in}}%
\pgfusepath{stroke}%
\end{pgfscope}%
\begin{pgfscope}%
\pgfpathrectangle{\pgfqpoint{0.625000in}{0.550000in}}{\pgfqpoint{3.875000in}{3.850000in}} %
\pgfusepath{clip}%
\pgfsetbuttcap%
\pgfsetroundjoin%
\pgfsetlinewidth{0.501875pt}%
\definecolor{currentstroke}{rgb}{0.000000,0.000000,0.000000}%
\pgfsetstrokecolor{currentstroke}%
\pgfsetdash{}{0pt}%
\pgfpathmoveto{\pgfqpoint{0.625000in}{2.757280in}}%
\pgfpathlineto{\pgfqpoint{0.630209in}{2.759649in}}%
\pgfpathlineto{\pgfqpoint{0.631364in}{2.769298in}}%
\pgfpathlineto{\pgfqpoint{0.625000in}{2.771996in}}%
\pgfusepath{stroke}%
\end{pgfscope}%
\begin{pgfscope}%
\pgfpathrectangle{\pgfqpoint{0.625000in}{0.550000in}}{\pgfqpoint{3.875000in}{3.850000in}} %
\pgfusepath{clip}%
\pgfsetbuttcap%
\pgfsetroundjoin%
\pgfsetlinewidth{0.501875pt}%
\definecolor{currentstroke}{rgb}{0.000000,0.000000,0.000000}%
\pgfsetstrokecolor{currentstroke}%
\pgfsetdash{}{0pt}%
\pgfpathmoveto{\pgfqpoint{0.625000in}{2.782796in}}%
\pgfpathlineto{\pgfqpoint{0.632165in}{2.788596in}}%
\pgfpathlineto{\pgfqpoint{0.634712in}{2.797129in}}%
\pgfpathlineto{\pgfqpoint{0.644424in}{2.793018in}}%
\pgfpathlineto{\pgfqpoint{0.663847in}{2.788161in}}%
\pgfpathlineto{\pgfqpoint{0.683271in}{2.783693in}}%
\pgfpathlineto{\pgfqpoint{0.702694in}{2.776786in}}%
\pgfpathlineto{\pgfqpoint{0.722118in}{2.766738in}}%
\pgfpathlineto{\pgfqpoint{0.741541in}{2.752644in}}%
\pgfpathlineto{\pgfqpoint{0.754293in}{2.740351in}}%
\pgfpathlineto{\pgfqpoint{0.762521in}{2.730702in}}%
\pgfpathlineto{\pgfqpoint{0.775335in}{2.711404in}}%
\pgfpathlineto{\pgfqpoint{0.784361in}{2.692105in}}%
\pgfpathlineto{\pgfqpoint{0.790249in}{2.672807in}}%
\pgfpathlineto{\pgfqpoint{0.793497in}{2.653509in}}%
\pgfpathlineto{\pgfqpoint{0.794169in}{2.634211in}}%
\pgfpathlineto{\pgfqpoint{0.792316in}{2.614912in}}%
\pgfpathlineto{\pgfqpoint{0.787919in}{2.595614in}}%
\pgfpathlineto{\pgfqpoint{0.780388in}{2.575833in}}%
\pgfpathlineto{\pgfqpoint{0.770262in}{2.557018in}}%
\pgfpathlineto{\pgfqpoint{0.756058in}{2.537719in}}%
\pgfpathlineto{\pgfqpoint{0.736169in}{2.518421in}}%
\pgfpathlineto{\pgfqpoint{0.706684in}{2.499123in}}%
\pgfpathlineto{\pgfqpoint{0.702694in}{2.492775in}}%
\pgfpathlineto{\pgfqpoint{0.692982in}{2.492886in}}%
\pgfpathlineto{\pgfqpoint{0.683674in}{2.489474in}}%
\pgfpathlineto{\pgfqpoint{0.692982in}{2.486612in}}%
\pgfpathlineto{\pgfqpoint{0.706519in}{2.489474in}}%
\pgfpathlineto{\pgfqpoint{0.712406in}{2.492930in}}%
\pgfpathlineto{\pgfqpoint{0.722118in}{2.493931in}}%
\pgfpathlineto{\pgfqpoint{0.741541in}{2.500382in}}%
\pgfpathlineto{\pgfqpoint{0.770677in}{2.512976in}}%
\pgfpathlineto{\pgfqpoint{0.790100in}{2.523443in}}%
\pgfpathlineto{\pgfqpoint{0.812422in}{2.537719in}}%
\pgfpathlineto{\pgfqpoint{0.838659in}{2.559113in}}%
\pgfpathlineto{\pgfqpoint{0.858083in}{2.578666in}}%
\pgfpathlineto{\pgfqpoint{0.872324in}{2.595614in}}%
\pgfpathlineto{\pgfqpoint{0.891532in}{2.624561in}}%
\pgfpathlineto{\pgfqpoint{0.901547in}{2.643860in}}%
\pgfpathlineto{\pgfqpoint{0.909606in}{2.663158in}}%
\pgfpathlineto{\pgfqpoint{0.916353in}{2.684436in}}%
\pgfpathlineto{\pgfqpoint{0.920552in}{2.701754in}}%
\pgfpathlineto{\pgfqpoint{0.923596in}{2.721053in}}%
\pgfpathlineto{\pgfqpoint{0.925097in}{2.740351in}}%
\pgfpathlineto{\pgfqpoint{0.925103in}{2.759649in}}%
\pgfpathlineto{\pgfqpoint{0.923600in}{2.778947in}}%
\pgfpathlineto{\pgfqpoint{0.920527in}{2.798246in}}%
\pgfpathlineto{\pgfqpoint{0.915800in}{2.817544in}}%
\pgfpathlineto{\pgfqpoint{0.906642in}{2.843663in}}%
\pgfpathlineto{\pgfqpoint{0.896930in}{2.864310in}}%
\pgfpathlineto{\pgfqpoint{0.887218in}{2.881032in}}%
\pgfpathlineto{\pgfqpoint{0.877506in}{2.895217in}}%
\pgfpathlineto{\pgfqpoint{0.862132in}{2.914035in}}%
\pgfpathlineto{\pgfqpoint{0.848371in}{2.928159in}}%
\pgfpathlineto{\pgfqpoint{0.828947in}{2.944745in}}%
\pgfpathlineto{\pgfqpoint{0.809524in}{2.958306in}}%
\pgfpathlineto{\pgfqpoint{0.784800in}{2.971930in}}%
\pgfpathlineto{\pgfqpoint{0.770677in}{2.978264in}}%
\pgfpathlineto{\pgfqpoint{0.751253in}{2.985128in}}%
\pgfpathlineto{\pgfqpoint{0.731830in}{2.989792in}}%
\pgfpathlineto{\pgfqpoint{0.712406in}{2.991629in}}%
\pgfpathlineto{\pgfqpoint{0.692982in}{2.989373in}}%
\pgfpathlineto{\pgfqpoint{0.673559in}{2.982462in}}%
\pgfpathlineto{\pgfqpoint{0.671951in}{2.981579in}}%
\pgfpathlineto{\pgfqpoint{0.625000in}{2.963989in}}%
\pgfpathlineto{\pgfqpoint{0.625000in}{2.963989in}}%
\pgfusepath{stroke}%
\end{pgfscope}%
\begin{pgfscope}%
\pgfpathrectangle{\pgfqpoint{0.625000in}{0.550000in}}{\pgfqpoint{3.875000in}{3.850000in}} %
\pgfusepath{clip}%
\pgfsetbuttcap%
\pgfsetroundjoin%
\pgfsetlinewidth{0.501875pt}%
\definecolor{currentstroke}{rgb}{0.000000,0.000000,0.000000}%
\pgfsetstrokecolor{currentstroke}%
\pgfsetdash{}{0pt}%
\pgfpathmoveto{\pgfqpoint{0.625000in}{2.804631in}}%
\pgfpathlineto{\pgfqpoint{0.633891in}{2.798246in}}%
\pgfpathlineto{\pgfqpoint{0.625000in}{2.795013in}}%
\pgfusepath{stroke}%
\end{pgfscope}%
\begin{pgfscope}%
\pgfpathrectangle{\pgfqpoint{0.625000in}{0.550000in}}{\pgfqpoint{3.875000in}{3.850000in}} %
\pgfusepath{clip}%
\pgfsetbuttcap%
\pgfsetroundjoin%
\pgfsetlinewidth{0.501875pt}%
\definecolor{currentstroke}{rgb}{0.000000,0.000000,0.000000}%
\pgfsetstrokecolor{currentstroke}%
\pgfsetdash{}{0pt}%
\pgfpathmoveto{\pgfqpoint{0.625000in}{2.859146in}}%
\pgfpathlineto{\pgfqpoint{0.629175in}{2.856140in}}%
\pgfpathlineto{\pgfqpoint{0.632405in}{2.846491in}}%
\pgfpathlineto{\pgfqpoint{0.633073in}{2.836842in}}%
\pgfpathlineto{\pgfqpoint{0.631150in}{2.827193in}}%
\pgfpathlineto{\pgfqpoint{0.625000in}{2.820269in}}%
\pgfusepath{stroke}%
\end{pgfscope}%
\begin{pgfscope}%
\pgfpathrectangle{\pgfqpoint{0.625000in}{0.550000in}}{\pgfqpoint{3.875000in}{3.850000in}} %
\pgfusepath{clip}%
\pgfsetbuttcap%
\pgfsetroundjoin%
\pgfsetlinewidth{0.501875pt}%
\definecolor{currentstroke}{rgb}{0.000000,0.000000,0.000000}%
\pgfsetstrokecolor{currentstroke}%
\pgfsetdash{}{0pt}%
\pgfpathmoveto{\pgfqpoint{0.625000in}{2.922332in}}%
\pgfpathlineto{\pgfqpoint{0.630951in}{2.914035in}}%
\pgfpathlineto{\pgfqpoint{0.633920in}{2.904386in}}%
\pgfpathlineto{\pgfqpoint{0.632559in}{2.894737in}}%
\pgfpathlineto{\pgfqpoint{0.632354in}{2.885088in}}%
\pgfpathlineto{\pgfqpoint{0.632120in}{2.875439in}}%
\pgfpathlineto{\pgfqpoint{0.625000in}{2.869433in}}%
\pgfusepath{stroke}%
\end{pgfscope}%
\begin{pgfscope}%
\pgfpathrectangle{\pgfqpoint{0.625000in}{0.550000in}}{\pgfqpoint{3.875000in}{3.850000in}} %
\pgfusepath{clip}%
\pgfsetbuttcap%
\pgfsetroundjoin%
\pgfsetlinewidth{0.501875pt}%
\definecolor{currentstroke}{rgb}{0.000000,0.000000,0.000000}%
\pgfsetstrokecolor{currentstroke}%
\pgfsetdash{}{0pt}%
\pgfpathmoveto{\pgfqpoint{0.625000in}{2.953335in}}%
\pgfpathlineto{\pgfqpoint{0.625248in}{2.952632in}}%
\pgfpathlineto{\pgfqpoint{0.631952in}{2.942982in}}%
\pgfpathlineto{\pgfqpoint{0.625000in}{2.935793in}}%
\pgfusepath{stroke}%
\end{pgfscope}%
\begin{pgfscope}%
\pgfpathrectangle{\pgfqpoint{0.625000in}{0.550000in}}{\pgfqpoint{3.875000in}{3.850000in}} %
\pgfusepath{clip}%
\pgfsetbuttcap%
\pgfsetroundjoin%
\pgfsetlinewidth{0.501875pt}%
\definecolor{currentstroke}{rgb}{0.000000,0.000000,0.000000}%
\pgfsetstrokecolor{currentstroke}%
\pgfsetdash{}{0pt}%
\pgfpathmoveto{\pgfqpoint{0.625000in}{3.045944in}}%
\pgfpathlineto{\pgfqpoint{0.630937in}{3.039474in}}%
\pgfpathlineto{\pgfqpoint{0.634712in}{3.033416in}}%
\pgfpathlineto{\pgfqpoint{0.644424in}{3.034446in}}%
\pgfpathlineto{\pgfqpoint{0.654135in}{3.039320in}}%
\pgfpathlineto{\pgfqpoint{0.654440in}{3.039474in}}%
\pgfpathlineto{\pgfqpoint{0.659412in}{3.049123in}}%
\pgfpathlineto{\pgfqpoint{0.656723in}{3.058772in}}%
\pgfpathlineto{\pgfqpoint{0.654135in}{3.062527in}}%
\pgfpathlineto{\pgfqpoint{0.646104in}{3.068421in}}%
\pgfpathlineto{\pgfqpoint{0.644424in}{3.071999in}}%
\pgfpathlineto{\pgfqpoint{0.643645in}{3.078070in}}%
\pgfpathlineto{\pgfqpoint{0.639450in}{3.087719in}}%
\pgfpathlineto{\pgfqpoint{0.635871in}{3.097368in}}%
\pgfpathlineto{\pgfqpoint{0.634712in}{3.099173in}}%
\pgfpathlineto{\pgfqpoint{0.631978in}{3.107018in}}%
\pgfpathlineto{\pgfqpoint{0.628881in}{3.116667in}}%
\pgfpathlineto{\pgfqpoint{0.625000in}{3.118035in}}%
\pgfusepath{stroke}%
\end{pgfscope}%
\begin{pgfscope}%
\pgfpathrectangle{\pgfqpoint{0.625000in}{0.550000in}}{\pgfqpoint{3.875000in}{3.850000in}} %
\pgfusepath{clip}%
\pgfsetbuttcap%
\pgfsetroundjoin%
\pgfsetlinewidth{0.501875pt}%
\definecolor{currentstroke}{rgb}{0.000000,0.000000,0.000000}%
\pgfsetstrokecolor{currentstroke}%
\pgfsetdash{}{0pt}%
\pgfpathmoveto{\pgfqpoint{0.625000in}{3.076086in}}%
\pgfpathlineto{\pgfqpoint{0.634712in}{3.072691in}}%
\pgfpathlineto{\pgfqpoint{0.643833in}{3.068421in}}%
\pgfpathlineto{\pgfqpoint{0.634712in}{3.064673in}}%
\pgfpathlineto{\pgfqpoint{0.625000in}{3.060756in}}%
\pgfusepath{stroke}%
\end{pgfscope}%
\begin{pgfscope}%
\pgfpathrectangle{\pgfqpoint{0.625000in}{0.550000in}}{\pgfqpoint{3.875000in}{3.850000in}} %
\pgfusepath{clip}%
\pgfsetbuttcap%
\pgfsetroundjoin%
\pgfsetlinewidth{0.501875pt}%
\definecolor{currentstroke}{rgb}{0.000000,0.000000,0.000000}%
\pgfsetstrokecolor{currentstroke}%
\pgfsetdash{}{0pt}%
\pgfpathmoveto{\pgfqpoint{0.625000in}{3.096096in}}%
\pgfpathlineto{\pgfqpoint{0.634477in}{3.087719in}}%
\pgfpathlineto{\pgfqpoint{0.625000in}{3.078921in}}%
\pgfusepath{stroke}%
\end{pgfscope}%
\begin{pgfscope}%
\pgfpathrectangle{\pgfqpoint{0.625000in}{0.550000in}}{\pgfqpoint{3.875000in}{3.850000in}} %
\pgfusepath{clip}%
\pgfsetbuttcap%
\pgfsetroundjoin%
\pgfsetlinewidth{0.501875pt}%
\definecolor{currentstroke}{rgb}{0.000000,0.000000,0.000000}%
\pgfsetstrokecolor{currentstroke}%
\pgfsetdash{}{0pt}%
\pgfpathmoveto{\pgfqpoint{0.625000in}{3.142063in}}%
\pgfpathlineto{\pgfqpoint{0.628941in}{3.145614in}}%
\pgfpathlineto{\pgfqpoint{0.625000in}{3.149165in}}%
\pgfusepath{stroke}%
\end{pgfscope}%
\begin{pgfscope}%
\pgfpathrectangle{\pgfqpoint{0.625000in}{0.550000in}}{\pgfqpoint{3.875000in}{3.850000in}} %
\pgfusepath{clip}%
\pgfsetbuttcap%
\pgfsetroundjoin%
\pgfsetlinewidth{0.501875pt}%
\definecolor{currentstroke}{rgb}{0.000000,0.000000,0.000000}%
\pgfsetstrokecolor{currentstroke}%
\pgfsetdash{}{0pt}%
\pgfpathmoveto{\pgfqpoint{0.625000in}{3.231176in}}%
\pgfpathlineto{\pgfqpoint{0.626153in}{3.232456in}}%
\pgfpathlineto{\pgfqpoint{0.625000in}{3.237212in}}%
\pgfusepath{stroke}%
\end{pgfscope}%
\begin{pgfscope}%
\pgfpathrectangle{\pgfqpoint{0.625000in}{0.550000in}}{\pgfqpoint{3.875000in}{3.850000in}} %
\pgfusepath{clip}%
\pgfsetbuttcap%
\pgfsetroundjoin%
\pgfsetlinewidth{0.501875pt}%
\definecolor{currentstroke}{rgb}{0.000000,0.000000,0.000000}%
\pgfsetstrokecolor{currentstroke}%
\pgfsetdash{}{0pt}%
\pgfpathmoveto{\pgfqpoint{0.625000in}{3.246841in}}%
\pgfpathlineto{\pgfqpoint{0.626829in}{3.251754in}}%
\pgfpathlineto{\pgfqpoint{0.625000in}{3.252999in}}%
\pgfusepath{stroke}%
\end{pgfscope}%
\begin{pgfscope}%
\pgfpathrectangle{\pgfqpoint{0.625000in}{0.550000in}}{\pgfqpoint{3.875000in}{3.850000in}} %
\pgfusepath{clip}%
\pgfsetbuttcap%
\pgfsetroundjoin%
\pgfsetlinewidth{0.501875pt}%
\definecolor{currentstroke}{rgb}{0.000000,0.000000,0.000000}%
\pgfsetstrokecolor{currentstroke}%
\pgfsetdash{}{0pt}%
\pgfpathmoveto{\pgfqpoint{0.625000in}{3.269877in}}%
\pgfpathlineto{\pgfqpoint{0.633965in}{3.261404in}}%
\pgfpathlineto{\pgfqpoint{0.634712in}{3.260290in}}%
\pgfpathlineto{\pgfqpoint{0.644424in}{3.260259in}}%
\pgfpathlineto{\pgfqpoint{0.646542in}{3.261404in}}%
\pgfpathlineto{\pgfqpoint{0.649705in}{3.271053in}}%
\pgfpathlineto{\pgfqpoint{0.646195in}{3.280702in}}%
\pgfpathlineto{\pgfqpoint{0.644424in}{3.282849in}}%
\pgfpathlineto{\pgfqpoint{0.634712in}{3.288557in}}%
\pgfpathlineto{\pgfqpoint{0.629852in}{3.280702in}}%
\pgfpathlineto{\pgfqpoint{0.625000in}{3.274479in}}%
\pgfusepath{stroke}%
\end{pgfscope}%
\begin{pgfscope}%
\pgfpathrectangle{\pgfqpoint{0.625000in}{0.550000in}}{\pgfqpoint{3.875000in}{3.850000in}} %
\pgfusepath{clip}%
\pgfsetbuttcap%
\pgfsetroundjoin%
\pgfsetlinewidth{0.501875pt}%
\definecolor{currentstroke}{rgb}{0.000000,0.000000,0.000000}%
\pgfsetstrokecolor{currentstroke}%
\pgfsetdash{}{0pt}%
\pgfpathmoveto{\pgfqpoint{0.625000in}{3.299987in}}%
\pgfpathlineto{\pgfqpoint{0.625063in}{3.300000in}}%
\pgfpathlineto{\pgfqpoint{0.625000in}{3.300013in}}%
\pgfusepath{stroke}%
\end{pgfscope}%
\begin{pgfscope}%
\pgfpathrectangle{\pgfqpoint{0.625000in}{0.550000in}}{\pgfqpoint{3.875000in}{3.850000in}} %
\pgfusepath{clip}%
\pgfsetbuttcap%
\pgfsetroundjoin%
\pgfsetlinewidth{0.501875pt}%
\definecolor{currentstroke}{rgb}{0.000000,0.000000,0.000000}%
\pgfsetstrokecolor{currentstroke}%
\pgfsetdash{}{0pt}%
\pgfpathmoveto{\pgfqpoint{0.625000in}{3.347135in}}%
\pgfpathlineto{\pgfqpoint{0.629599in}{3.348246in}}%
\pgfpathlineto{\pgfqpoint{0.625000in}{3.349318in}}%
\pgfusepath{stroke}%
\end{pgfscope}%
\begin{pgfscope}%
\pgfpathrectangle{\pgfqpoint{0.625000in}{0.550000in}}{\pgfqpoint{3.875000in}{3.850000in}} %
\pgfusepath{clip}%
\pgfsetbuttcap%
\pgfsetroundjoin%
\pgfsetlinewidth{0.501875pt}%
\definecolor{currentstroke}{rgb}{0.000000,0.000000,0.000000}%
\pgfsetstrokecolor{currentstroke}%
\pgfsetdash{}{0pt}%
\pgfpathmoveto{\pgfqpoint{0.625000in}{3.386009in}}%
\pgfpathlineto{\pgfqpoint{0.627130in}{3.386842in}}%
\pgfpathlineto{\pgfqpoint{0.625000in}{3.388251in}}%
\pgfusepath{stroke}%
\end{pgfscope}%
\begin{pgfscope}%
\pgfpathrectangle{\pgfqpoint{0.625000in}{0.550000in}}{\pgfqpoint{3.875000in}{3.850000in}} %
\pgfusepath{clip}%
\pgfsetbuttcap%
\pgfsetroundjoin%
\pgfsetlinewidth{0.501875pt}%
\definecolor{currentstroke}{rgb}{0.000000,0.000000,0.000000}%
\pgfsetstrokecolor{currentstroke}%
\pgfsetdash{}{0pt}%
\pgfpathmoveto{\pgfqpoint{0.625000in}{3.403117in}}%
\pgfpathlineto{\pgfqpoint{0.629119in}{3.406140in}}%
\pgfpathlineto{\pgfqpoint{0.628732in}{3.415789in}}%
\pgfpathlineto{\pgfqpoint{0.626875in}{3.425439in}}%
\pgfpathlineto{\pgfqpoint{0.625000in}{3.427847in}}%
\pgfusepath{stroke}%
\end{pgfscope}%
\begin{pgfscope}%
\pgfpathrectangle{\pgfqpoint{0.625000in}{0.550000in}}{\pgfqpoint{3.875000in}{3.850000in}} %
\pgfusepath{clip}%
\pgfsetbuttcap%
\pgfsetroundjoin%
\pgfsetlinewidth{0.501875pt}%
\definecolor{currentstroke}{rgb}{0.000000,0.000000,0.000000}%
\pgfsetstrokecolor{currentstroke}%
\pgfsetdash{}{0pt}%
\pgfpathmoveto{\pgfqpoint{0.625000in}{3.527720in}}%
\pgfpathlineto{\pgfqpoint{0.627961in}{3.531579in}}%
\pgfpathlineto{\pgfqpoint{0.625000in}{3.535438in}}%
\pgfusepath{stroke}%
\end{pgfscope}%
\begin{pgfscope}%
\pgfpathrectangle{\pgfqpoint{0.625000in}{0.550000in}}{\pgfqpoint{3.875000in}{3.850000in}} %
\pgfusepath{clip}%
\pgfsetbuttcap%
\pgfsetroundjoin%
\pgfsetlinewidth{0.501875pt}%
\definecolor{currentstroke}{rgb}{0.000000,0.000000,0.000000}%
\pgfsetstrokecolor{currentstroke}%
\pgfsetdash{}{0pt}%
\pgfpathmoveto{\pgfqpoint{0.625000in}{3.578389in}}%
\pgfpathlineto{\pgfqpoint{0.626018in}{3.579825in}}%
\pgfpathlineto{\pgfqpoint{0.629054in}{3.589474in}}%
\pgfpathlineto{\pgfqpoint{0.625000in}{3.592132in}}%
\pgfusepath{stroke}%
\end{pgfscope}%
\begin{pgfscope}%
\pgfpathrectangle{\pgfqpoint{0.625000in}{0.550000in}}{\pgfqpoint{3.875000in}{3.850000in}} %
\pgfusepath{clip}%
\pgfsetbuttcap%
\pgfsetroundjoin%
\pgfsetlinewidth{0.501875pt}%
\definecolor{currentstroke}{rgb}{0.000000,0.000000,0.000000}%
\pgfsetstrokecolor{currentstroke}%
\pgfsetdash{}{0pt}%
\pgfpathmoveto{\pgfqpoint{0.625000in}{3.694448in}}%
\pgfpathlineto{\pgfqpoint{0.626608in}{3.695614in}}%
\pgfpathlineto{\pgfqpoint{0.625000in}{3.696743in}}%
\pgfusepath{stroke}%
\end{pgfscope}%
\begin{pgfscope}%
\pgfpathrectangle{\pgfqpoint{0.625000in}{0.550000in}}{\pgfqpoint{3.875000in}{3.850000in}} %
\pgfusepath{clip}%
\pgfsetbuttcap%
\pgfsetroundjoin%
\pgfsetlinewidth{0.501875pt}%
\definecolor{currentstroke}{rgb}{0.000000,0.000000,0.000000}%
\pgfsetstrokecolor{currentstroke}%
\pgfsetdash{}{0pt}%
\pgfpathmoveto{\pgfqpoint{0.625000in}{3.732614in}}%
\pgfpathlineto{\pgfqpoint{0.631589in}{3.724561in}}%
\pgfpathlineto{\pgfqpoint{0.634712in}{3.717875in}}%
\pgfpathlineto{\pgfqpoint{0.644424in}{3.723222in}}%
\pgfpathlineto{\pgfqpoint{0.645711in}{3.724561in}}%
\pgfpathlineto{\pgfqpoint{0.651771in}{3.734211in}}%
\pgfpathlineto{\pgfqpoint{0.652168in}{3.743860in}}%
\pgfpathlineto{\pgfqpoint{0.649810in}{3.753509in}}%
\pgfpathlineto{\pgfqpoint{0.644424in}{3.757401in}}%
\pgfpathlineto{\pgfqpoint{0.639912in}{3.753509in}}%
\pgfpathlineto{\pgfqpoint{0.640414in}{3.743860in}}%
\pgfpathlineto{\pgfqpoint{0.634712in}{3.737174in}}%
\pgfpathlineto{\pgfqpoint{0.625000in}{3.735266in}}%
\pgfusepath{stroke}%
\end{pgfscope}%
\begin{pgfscope}%
\pgfpathrectangle{\pgfqpoint{0.625000in}{0.550000in}}{\pgfqpoint{3.875000in}{3.850000in}} %
\pgfusepath{clip}%
\pgfsetbuttcap%
\pgfsetroundjoin%
\pgfsetlinewidth{0.501875pt}%
\definecolor{currentstroke}{rgb}{0.000000,0.000000,0.000000}%
\pgfsetstrokecolor{currentstroke}%
\pgfsetdash{}{0pt}%
\pgfpathmoveto{\pgfqpoint{0.625000in}{3.781161in}}%
\pgfpathlineto{\pgfqpoint{0.627183in}{3.782456in}}%
\pgfpathlineto{\pgfqpoint{0.625000in}{3.783323in}}%
\pgfusepath{stroke}%
\end{pgfscope}%
\begin{pgfscope}%
\pgfpathrectangle{\pgfqpoint{0.625000in}{0.550000in}}{\pgfqpoint{3.875000in}{3.850000in}} %
\pgfusepath{clip}%
\pgfsetbuttcap%
\pgfsetroundjoin%
\pgfsetlinewidth{0.501875pt}%
\definecolor{currentstroke}{rgb}{0.000000,0.000000,0.000000}%
\pgfsetstrokecolor{currentstroke}%
\pgfsetdash{}{0pt}%
\pgfpathmoveto{\pgfqpoint{0.625000in}{3.829432in}}%
\pgfpathlineto{\pgfqpoint{0.631450in}{3.821053in}}%
\pgfpathlineto{\pgfqpoint{0.631429in}{3.811404in}}%
\pgfpathlineto{\pgfqpoint{0.634712in}{3.803711in}}%
\pgfpathlineto{\pgfqpoint{0.637398in}{3.801754in}}%
\pgfpathlineto{\pgfqpoint{0.644424in}{3.794148in}}%
\pgfpathlineto{\pgfqpoint{0.645624in}{3.792105in}}%
\pgfpathlineto{\pgfqpoint{0.648044in}{3.782456in}}%
\pgfpathlineto{\pgfqpoint{0.644424in}{3.779603in}}%
\pgfpathlineto{\pgfqpoint{0.636280in}{3.772807in}}%
\pgfpathlineto{\pgfqpoint{0.644424in}{3.769160in}}%
\pgfpathlineto{\pgfqpoint{0.650398in}{3.772807in}}%
\pgfpathlineto{\pgfqpoint{0.654135in}{3.777339in}}%
\pgfpathlineto{\pgfqpoint{0.660651in}{3.782456in}}%
\pgfpathlineto{\pgfqpoint{0.663847in}{3.789914in}}%
\pgfpathlineto{\pgfqpoint{0.665422in}{3.792105in}}%
\pgfpathlineto{\pgfqpoint{0.667401in}{3.801754in}}%
\pgfpathlineto{\pgfqpoint{0.666244in}{3.811404in}}%
\pgfpathlineto{\pgfqpoint{0.663847in}{3.817213in}}%
\pgfpathlineto{\pgfqpoint{0.661858in}{3.821053in}}%
\pgfpathlineto{\pgfqpoint{0.654135in}{3.829481in}}%
\pgfpathlineto{\pgfqpoint{0.652081in}{3.830702in}}%
\pgfpathlineto{\pgfqpoint{0.644424in}{3.834974in}}%
\pgfpathlineto{\pgfqpoint{0.634712in}{3.836473in}}%
\pgfpathlineto{\pgfqpoint{0.625000in}{3.831727in}}%
\pgfusepath{stroke}%
\end{pgfscope}%
\begin{pgfscope}%
\pgfpathrectangle{\pgfqpoint{0.625000in}{0.550000in}}{\pgfqpoint{3.875000in}{3.850000in}} %
\pgfusepath{clip}%
\pgfsetbuttcap%
\pgfsetroundjoin%
\pgfsetlinewidth{0.501875pt}%
\definecolor{currentstroke}{rgb}{0.000000,0.000000,0.000000}%
\pgfsetstrokecolor{currentstroke}%
\pgfsetdash{}{0pt}%
\pgfpathmoveto{\pgfqpoint{0.625000in}{3.848974in}}%
\pgfpathlineto{\pgfqpoint{0.629411in}{3.850000in}}%
\pgfpathlineto{\pgfqpoint{0.625000in}{3.851184in}}%
\pgfusepath{stroke}%
\end{pgfscope}%
\begin{pgfscope}%
\pgfpathrectangle{\pgfqpoint{0.625000in}{0.550000in}}{\pgfqpoint{3.875000in}{3.850000in}} %
\pgfusepath{clip}%
\pgfsetbuttcap%
\pgfsetroundjoin%
\pgfsetlinewidth{0.501875pt}%
\definecolor{currentstroke}{rgb}{0.000000,0.000000,0.000000}%
\pgfsetstrokecolor{currentstroke}%
\pgfsetdash{}{0pt}%
\pgfpathmoveto{\pgfqpoint{0.625000in}{3.868633in}}%
\pgfpathlineto{\pgfqpoint{0.625897in}{3.869298in}}%
\pgfpathlineto{\pgfqpoint{0.625000in}{3.869885in}}%
\pgfusepath{stroke}%
\end{pgfscope}%
\begin{pgfscope}%
\pgfpathrectangle{\pgfqpoint{0.625000in}{0.550000in}}{\pgfqpoint{3.875000in}{3.850000in}} %
\pgfusepath{clip}%
\pgfsetbuttcap%
\pgfsetroundjoin%
\pgfsetlinewidth{0.501875pt}%
\definecolor{currentstroke}{rgb}{0.000000,0.000000,0.000000}%
\pgfsetstrokecolor{currentstroke}%
\pgfsetdash{}{0pt}%
\pgfpathmoveto{\pgfqpoint{0.625000in}{3.887545in}}%
\pgfpathlineto{\pgfqpoint{0.627462in}{3.888596in}}%
\pgfpathlineto{\pgfqpoint{0.625000in}{3.891345in}}%
\pgfusepath{stroke}%
\end{pgfscope}%
\begin{pgfscope}%
\pgfpathrectangle{\pgfqpoint{0.625000in}{0.550000in}}{\pgfqpoint{3.875000in}{3.850000in}} %
\pgfusepath{clip}%
\pgfsetbuttcap%
\pgfsetroundjoin%
\pgfsetlinewidth{0.501875pt}%
\definecolor{currentstroke}{rgb}{0.000000,0.000000,0.000000}%
\pgfsetstrokecolor{currentstroke}%
\pgfsetdash{}{0pt}%
\pgfpathmoveto{\pgfqpoint{0.625000in}{4.003545in}}%
\pgfpathlineto{\pgfqpoint{0.626861in}{4.004386in}}%
\pgfpathlineto{\pgfqpoint{0.625000in}{4.006512in}}%
\pgfusepath{stroke}%
\end{pgfscope}%
\begin{pgfscope}%
\pgfpathrectangle{\pgfqpoint{0.625000in}{0.550000in}}{\pgfqpoint{3.875000in}{3.850000in}} %
\pgfusepath{clip}%
\pgfsetbuttcap%
\pgfsetroundjoin%
\pgfsetlinewidth{0.501875pt}%
\definecolor{currentstroke}{rgb}{0.000000,0.000000,0.000000}%
\pgfsetstrokecolor{currentstroke}%
\pgfsetdash{}{0pt}%
\pgfpathmoveto{\pgfqpoint{0.625000in}{4.042171in}}%
\pgfpathlineto{\pgfqpoint{0.632200in}{4.033333in}}%
\pgfpathlineto{\pgfqpoint{0.634712in}{4.030172in}}%
\pgfpathlineto{\pgfqpoint{0.640496in}{4.033333in}}%
\pgfpathlineto{\pgfqpoint{0.640032in}{4.042982in}}%
\pgfpathlineto{\pgfqpoint{0.634712in}{4.048944in}}%
\pgfpathlineto{\pgfqpoint{0.625000in}{4.044202in}}%
\pgfusepath{stroke}%
\end{pgfscope}%
\begin{pgfscope}%
\pgfpathrectangle{\pgfqpoint{0.625000in}{0.550000in}}{\pgfqpoint{3.875000in}{3.850000in}} %
\pgfusepath{clip}%
\pgfsetbuttcap%
\pgfsetroundjoin%
\pgfsetlinewidth{0.501875pt}%
\definecolor{currentstroke}{rgb}{0.000000,0.000000,0.000000}%
\pgfsetstrokecolor{currentstroke}%
\pgfsetdash{}{0pt}%
\pgfpathmoveto{\pgfqpoint{0.625000in}{4.153401in}}%
\pgfpathlineto{\pgfqpoint{0.626636in}{4.158772in}}%
\pgfpathlineto{\pgfqpoint{0.625000in}{4.159962in}}%
\pgfusepath{stroke}%
\end{pgfscope}%
\begin{pgfscope}%
\pgfpathrectangle{\pgfqpoint{0.625000in}{0.550000in}}{\pgfqpoint{3.875000in}{3.850000in}} %
\pgfusepath{clip}%
\pgfsetbuttcap%
\pgfsetroundjoin%
\pgfsetlinewidth{0.501875pt}%
\definecolor{currentstroke}{rgb}{0.000000,0.000000,0.000000}%
\pgfsetstrokecolor{currentstroke}%
\pgfsetdash{}{0pt}%
\pgfpathmoveto{\pgfqpoint{0.625000in}{4.194332in}}%
\pgfpathlineto{\pgfqpoint{0.626406in}{4.197368in}}%
\pgfpathlineto{\pgfqpoint{0.625000in}{4.198604in}}%
\pgfusepath{stroke}%
\end{pgfscope}%
\begin{pgfscope}%
\pgfpathrectangle{\pgfqpoint{0.625000in}{0.550000in}}{\pgfqpoint{3.875000in}{3.850000in}} %
\pgfusepath{clip}%
\pgfsetbuttcap%
\pgfsetroundjoin%
\pgfsetlinewidth{0.501875pt}%
\definecolor{currentstroke}{rgb}{0.000000,0.000000,0.000000}%
\pgfsetstrokecolor{currentstroke}%
\pgfsetdash{}{0pt}%
\pgfpathmoveto{\pgfqpoint{0.625000in}{4.350747in}}%
\pgfpathlineto{\pgfqpoint{0.626153in}{4.351754in}}%
\pgfpathlineto{\pgfqpoint{0.625000in}{4.352932in}}%
\pgfusepath{stroke}%
\end{pgfscope}%
\begin{pgfscope}%
\pgfpathrectangle{\pgfqpoint{0.625000in}{0.550000in}}{\pgfqpoint{3.875000in}{3.850000in}} %
\pgfusepath{clip}%
\pgfsetbuttcap%
\pgfsetroundjoin%
\pgfsetlinewidth{0.501875pt}%
\definecolor{currentstroke}{rgb}{0.000000,0.000000,0.000000}%
\pgfsetstrokecolor{currentstroke}%
\pgfsetdash{}{0pt}%
\pgfpathmoveto{\pgfqpoint{0.634712in}{2.411102in}}%
\pgfpathlineto{\pgfqpoint{0.636918in}{2.412281in}}%
\pgfpathlineto{\pgfqpoint{0.634712in}{2.413530in}}%
\pgfpathlineto{\pgfqpoint{0.631301in}{2.412281in}}%
\pgfpathlineto{\pgfqpoint{0.634712in}{2.411102in}}%
\pgfusepath{stroke}%
\end{pgfscope}%
\begin{pgfscope}%
\pgfpathrectangle{\pgfqpoint{0.625000in}{0.550000in}}{\pgfqpoint{3.875000in}{3.850000in}} %
\pgfusepath{clip}%
\pgfsetbuttcap%
\pgfsetroundjoin%
\pgfsetlinewidth{0.501875pt}%
\definecolor{currentstroke}{rgb}{0.000000,0.000000,0.000000}%
\pgfsetstrokecolor{currentstroke}%
\pgfsetdash{}{0pt}%
\pgfpathmoveto{\pgfqpoint{0.644424in}{2.421568in}}%
\pgfpathlineto{\pgfqpoint{0.644950in}{2.421930in}}%
\pgfpathlineto{\pgfqpoint{0.644424in}{2.422337in}}%
\pgfpathlineto{\pgfqpoint{0.644236in}{2.421930in}}%
\pgfpathlineto{\pgfqpoint{0.644424in}{2.421568in}}%
\pgfusepath{stroke}%
\end{pgfscope}%
\begin{pgfscope}%
\pgfpathrectangle{\pgfqpoint{0.625000in}{0.550000in}}{\pgfqpoint{3.875000in}{3.850000in}} %
\pgfusepath{clip}%
\pgfsetbuttcap%
\pgfsetroundjoin%
\pgfsetlinewidth{0.501875pt}%
\definecolor{currentstroke}{rgb}{0.000000,0.000000,0.000000}%
\pgfsetstrokecolor{currentstroke}%
\pgfsetdash{}{0pt}%
\pgfpathmoveto{\pgfqpoint{0.673559in}{2.417538in}}%
\pgfpathlineto{\pgfqpoint{0.674760in}{2.421930in}}%
\pgfpathlineto{\pgfqpoint{0.675669in}{2.431579in}}%
\pgfpathlineto{\pgfqpoint{0.675102in}{2.441228in}}%
\pgfpathlineto{\pgfqpoint{0.673559in}{2.443032in}}%
\pgfpathlineto{\pgfqpoint{0.672308in}{2.441228in}}%
\pgfpathlineto{\pgfqpoint{0.671365in}{2.431579in}}%
\pgfpathlineto{\pgfqpoint{0.671298in}{2.421930in}}%
\pgfpathlineto{\pgfqpoint{0.673559in}{2.417538in}}%
\pgfusepath{stroke}%
\end{pgfscope}%
\begin{pgfscope}%
\pgfpathrectangle{\pgfqpoint{0.625000in}{0.550000in}}{\pgfqpoint{3.875000in}{3.850000in}} %
\pgfusepath{clip}%
\pgfsetbuttcap%
\pgfsetroundjoin%
\pgfsetlinewidth{0.501875pt}%
\definecolor{currentstroke}{rgb}{0.000000,0.000000,0.000000}%
\pgfsetstrokecolor{currentstroke}%
\pgfsetdash{}{0pt}%
\pgfpathmoveto{\pgfqpoint{0.654135in}{2.427860in}}%
\pgfpathlineto{\pgfqpoint{0.655702in}{2.431579in}}%
\pgfpathlineto{\pgfqpoint{0.654135in}{2.433376in}}%
\pgfpathlineto{\pgfqpoint{0.650976in}{2.431579in}}%
\pgfpathlineto{\pgfqpoint{0.654135in}{2.427860in}}%
\pgfusepath{stroke}%
\end{pgfscope}%
\begin{pgfscope}%
\pgfpathrectangle{\pgfqpoint{0.625000in}{0.550000in}}{\pgfqpoint{3.875000in}{3.850000in}} %
\pgfusepath{clip}%
\pgfsetbuttcap%
\pgfsetroundjoin%
\pgfsetlinewidth{0.501875pt}%
\definecolor{currentstroke}{rgb}{0.000000,0.000000,0.000000}%
\pgfsetstrokecolor{currentstroke}%
\pgfsetdash{}{0pt}%
\pgfpathmoveto{\pgfqpoint{0.634712in}{2.447786in}}%
\pgfpathlineto{\pgfqpoint{0.640738in}{2.450877in}}%
\pgfpathlineto{\pgfqpoint{0.634712in}{2.452667in}}%
\pgfpathlineto{\pgfqpoint{0.631692in}{2.450877in}}%
\pgfpathlineto{\pgfqpoint{0.634712in}{2.447786in}}%
\pgfusepath{stroke}%
\end{pgfscope}%
\begin{pgfscope}%
\pgfpathrectangle{\pgfqpoint{0.625000in}{0.550000in}}{\pgfqpoint{3.875000in}{3.850000in}} %
\pgfusepath{clip}%
\pgfsetbuttcap%
\pgfsetroundjoin%
\pgfsetlinewidth{0.501875pt}%
\definecolor{currentstroke}{rgb}{0.000000,0.000000,0.000000}%
\pgfsetstrokecolor{currentstroke}%
\pgfsetdash{}{0pt}%
\pgfpathmoveto{\pgfqpoint{0.683271in}{2.449841in}}%
\pgfpathlineto{\pgfqpoint{0.685622in}{2.450877in}}%
\pgfpathlineto{\pgfqpoint{0.683271in}{2.452629in}}%
\pgfpathlineto{\pgfqpoint{0.682030in}{2.450877in}}%
\pgfpathlineto{\pgfqpoint{0.683271in}{2.449841in}}%
\pgfusepath{stroke}%
\end{pgfscope}%
\begin{pgfscope}%
\pgfpathrectangle{\pgfqpoint{0.625000in}{0.550000in}}{\pgfqpoint{3.875000in}{3.850000in}} %
\pgfusepath{clip}%
\pgfsetbuttcap%
\pgfsetroundjoin%
\pgfsetlinewidth{0.501875pt}%
\definecolor{currentstroke}{rgb}{0.000000,0.000000,0.000000}%
\pgfsetstrokecolor{currentstroke}%
\pgfsetdash{}{0pt}%
\pgfpathmoveto{\pgfqpoint{0.663847in}{2.457738in}}%
\pgfpathlineto{\pgfqpoint{0.671225in}{2.460526in}}%
\pgfpathlineto{\pgfqpoint{0.663847in}{2.468415in}}%
\pgfpathlineto{\pgfqpoint{0.660528in}{2.460526in}}%
\pgfpathlineto{\pgfqpoint{0.663847in}{2.457738in}}%
\pgfusepath{stroke}%
\end{pgfscope}%
\begin{pgfscope}%
\pgfpathrectangle{\pgfqpoint{0.625000in}{0.550000in}}{\pgfqpoint{3.875000in}{3.850000in}} %
\pgfusepath{clip}%
\pgfsetbuttcap%
\pgfsetroundjoin%
\pgfsetlinewidth{0.501875pt}%
\definecolor{currentstroke}{rgb}{0.000000,0.000000,0.000000}%
\pgfsetstrokecolor{currentstroke}%
\pgfsetdash{}{0pt}%
\pgfpathmoveto{\pgfqpoint{0.634712in}{2.507058in}}%
\pgfpathlineto{\pgfqpoint{0.640738in}{2.508772in}}%
\pgfpathlineto{\pgfqpoint{0.634712in}{2.511863in}}%
\pgfpathlineto{\pgfqpoint{0.631692in}{2.508772in}}%
\pgfpathlineto{\pgfqpoint{0.634712in}{2.507058in}}%
\pgfusepath{stroke}%
\end{pgfscope}%
\begin{pgfscope}%
\pgfpathrectangle{\pgfqpoint{0.625000in}{0.550000in}}{\pgfqpoint{3.875000in}{3.850000in}} %
\pgfusepath{clip}%
\pgfsetbuttcap%
\pgfsetroundjoin%
\pgfsetlinewidth{0.501875pt}%
\definecolor{currentstroke}{rgb}{0.000000,0.000000,0.000000}%
\pgfsetstrokecolor{currentstroke}%
\pgfsetdash{}{0pt}%
\pgfpathmoveto{\pgfqpoint{0.683271in}{2.507020in}}%
\pgfpathlineto{\pgfqpoint{0.685622in}{2.508772in}}%
\pgfpathlineto{\pgfqpoint{0.683271in}{2.509808in}}%
\pgfpathlineto{\pgfqpoint{0.682030in}{2.508772in}}%
\pgfpathlineto{\pgfqpoint{0.683271in}{2.507020in}}%
\pgfusepath{stroke}%
\end{pgfscope}%
\begin{pgfscope}%
\pgfpathrectangle{\pgfqpoint{0.625000in}{0.550000in}}{\pgfqpoint{3.875000in}{3.850000in}} %
\pgfusepath{clip}%
\pgfsetbuttcap%
\pgfsetroundjoin%
\pgfsetlinewidth{0.501875pt}%
\definecolor{currentstroke}{rgb}{0.000000,0.000000,0.000000}%
\pgfsetstrokecolor{currentstroke}%
\pgfsetdash{}{0pt}%
\pgfpathmoveto{\pgfqpoint{0.673559in}{2.516617in}}%
\pgfpathlineto{\pgfqpoint{0.675102in}{2.518421in}}%
\pgfpathlineto{\pgfqpoint{0.675669in}{2.528070in}}%
\pgfpathlineto{\pgfqpoint{0.674760in}{2.537719in}}%
\pgfpathlineto{\pgfqpoint{0.673559in}{2.542111in}}%
\pgfpathlineto{\pgfqpoint{0.671298in}{2.537719in}}%
\pgfpathlineto{\pgfqpoint{0.671365in}{2.528070in}}%
\pgfpathlineto{\pgfqpoint{0.672308in}{2.518421in}}%
\pgfpathlineto{\pgfqpoint{0.673559in}{2.516617in}}%
\pgfusepath{stroke}%
\end{pgfscope}%
\begin{pgfscope}%
\pgfpathrectangle{\pgfqpoint{0.625000in}{0.550000in}}{\pgfqpoint{3.875000in}{3.850000in}} %
\pgfusepath{clip}%
\pgfsetbuttcap%
\pgfsetroundjoin%
\pgfsetlinewidth{0.501875pt}%
\definecolor{currentstroke}{rgb}{0.000000,0.000000,0.000000}%
\pgfsetstrokecolor{currentstroke}%
\pgfsetdash{}{0pt}%
\pgfpathmoveto{\pgfqpoint{0.654135in}{2.526273in}}%
\pgfpathlineto{\pgfqpoint{0.655702in}{2.528070in}}%
\pgfpathlineto{\pgfqpoint{0.654135in}{2.531789in}}%
\pgfpathlineto{\pgfqpoint{0.650976in}{2.528070in}}%
\pgfpathlineto{\pgfqpoint{0.654135in}{2.526273in}}%
\pgfusepath{stroke}%
\end{pgfscope}%
\begin{pgfscope}%
\pgfpathrectangle{\pgfqpoint{0.625000in}{0.550000in}}{\pgfqpoint{3.875000in}{3.850000in}} %
\pgfusepath{clip}%
\pgfsetbuttcap%
\pgfsetroundjoin%
\pgfsetlinewidth{0.501875pt}%
\definecolor{currentstroke}{rgb}{0.000000,0.000000,0.000000}%
\pgfsetstrokecolor{currentstroke}%
\pgfsetdash{}{0pt}%
\pgfpathmoveto{\pgfqpoint{0.644424in}{2.537312in}}%
\pgfpathlineto{\pgfqpoint{0.644950in}{2.537719in}}%
\pgfpathlineto{\pgfqpoint{0.644424in}{2.538082in}}%
\pgfpathlineto{\pgfqpoint{0.644236in}{2.537719in}}%
\pgfpathlineto{\pgfqpoint{0.644424in}{2.537312in}}%
\pgfusepath{stroke}%
\end{pgfscope}%
\begin{pgfscope}%
\pgfpathrectangle{\pgfqpoint{0.625000in}{0.550000in}}{\pgfqpoint{3.875000in}{3.850000in}} %
\pgfusepath{clip}%
\pgfsetbuttcap%
\pgfsetroundjoin%
\pgfsetlinewidth{0.501875pt}%
\definecolor{currentstroke}{rgb}{0.000000,0.000000,0.000000}%
\pgfsetstrokecolor{currentstroke}%
\pgfsetdash{}{0pt}%
\pgfpathmoveto{\pgfqpoint{0.634712in}{2.546119in}}%
\pgfpathlineto{\pgfqpoint{0.636918in}{2.547368in}}%
\pgfpathlineto{\pgfqpoint{0.634712in}{2.548548in}}%
\pgfpathlineto{\pgfqpoint{0.631301in}{2.547368in}}%
\pgfpathlineto{\pgfqpoint{0.634712in}{2.546119in}}%
\pgfusepath{stroke}%
\end{pgfscope}%
\begin{pgfscope}%
\pgfpathrectangle{\pgfqpoint{0.625000in}{0.550000in}}{\pgfqpoint{3.875000in}{3.850000in}} %
\pgfusepath{clip}%
\pgfsetbuttcap%
\pgfsetroundjoin%
\pgfsetlinewidth{0.501875pt}%
\definecolor{currentstroke}{rgb}{0.000000,0.000000,0.000000}%
\pgfsetstrokecolor{currentstroke}%
\pgfsetdash{}{0pt}%
\pgfpathmoveto{\pgfqpoint{0.625000in}{0.625586in}}%
\pgfpathlineto{\pgfqpoint{0.626796in}{0.627193in}}%
\pgfpathlineto{\pgfqpoint{0.625000in}{0.628775in}}%
\pgfusepath{stroke}%
\end{pgfscope}%
\begin{pgfscope}%
\pgfpathrectangle{\pgfqpoint{0.625000in}{0.550000in}}{\pgfqpoint{3.875000in}{3.850000in}} %
\pgfusepath{clip}%
\pgfsetbuttcap%
\pgfsetroundjoin%
\pgfsetlinewidth{0.501875pt}%
\definecolor{currentstroke}{rgb}{0.000000,0.000000,0.000000}%
\pgfsetstrokecolor{currentstroke}%
\pgfsetdash{}{0pt}%
\pgfpathmoveto{\pgfqpoint{0.625000in}{1.369514in}}%
\pgfpathlineto{\pgfqpoint{0.626008in}{1.370175in}}%
\pgfpathlineto{\pgfqpoint{0.625000in}{1.374019in}}%
\pgfusepath{stroke}%
\end{pgfscope}%
\begin{pgfscope}%
\pgfpathrectangle{\pgfqpoint{0.625000in}{0.550000in}}{\pgfqpoint{3.875000in}{3.850000in}} %
\pgfusepath{clip}%
\pgfsetbuttcap%
\pgfsetroundjoin%
\pgfsetlinewidth{0.501875pt}%
\definecolor{currentstroke}{rgb}{0.000000,0.000000,0.000000}%
\pgfsetstrokecolor{currentstroke}%
\pgfsetdash{}{0pt}%
\pgfpathmoveto{\pgfqpoint{0.625000in}{1.425865in}}%
\pgfpathlineto{\pgfqpoint{0.626692in}{1.428070in}}%
\pgfpathlineto{\pgfqpoint{0.625000in}{1.430275in}}%
\pgfusepath{stroke}%
\end{pgfscope}%
\begin{pgfscope}%
\pgfpathrectangle{\pgfqpoint{0.625000in}{0.550000in}}{\pgfqpoint{3.875000in}{3.850000in}} %
\pgfusepath{clip}%
\pgfsetbuttcap%
\pgfsetroundjoin%
\pgfsetlinewidth{0.501875pt}%
\definecolor{currentstroke}{rgb}{0.000000,0.000000,0.000000}%
\pgfsetstrokecolor{currentstroke}%
\pgfsetdash{}{0pt}%
\pgfpathmoveto{\pgfqpoint{0.625000in}{1.541185in}}%
\pgfpathlineto{\pgfqpoint{0.625601in}{1.543860in}}%
\pgfpathlineto{\pgfqpoint{0.625000in}{1.544398in}}%
\pgfusepath{stroke}%
\end{pgfscope}%
\begin{pgfscope}%
\pgfpathrectangle{\pgfqpoint{0.625000in}{0.550000in}}{\pgfqpoint{3.875000in}{3.850000in}} %
\pgfusepath{clip}%
\pgfsetbuttcap%
\pgfsetroundjoin%
\pgfsetlinewidth{0.501875pt}%
\definecolor{currentstroke}{rgb}{0.000000,0.000000,0.000000}%
\pgfsetstrokecolor{currentstroke}%
\pgfsetdash{}{0pt}%
\pgfpathmoveto{\pgfqpoint{0.625000in}{1.812226in}}%
\pgfpathlineto{\pgfqpoint{0.627008in}{1.814035in}}%
\pgfpathlineto{\pgfqpoint{0.625000in}{1.815844in}}%
\pgfusepath{stroke}%
\end{pgfscope}%
\begin{pgfscope}%
\pgfpathrectangle{\pgfqpoint{0.625000in}{0.550000in}}{\pgfqpoint{3.875000in}{3.850000in}} %
\pgfusepath{clip}%
\pgfsetbuttcap%
\pgfsetroundjoin%
\pgfsetlinewidth{0.501875pt}%
\definecolor{currentstroke}{rgb}{0.000000,0.000000,0.000000}%
\pgfsetstrokecolor{currentstroke}%
\pgfsetdash{}{0pt}%
\pgfpathmoveto{\pgfqpoint{0.625000in}{1.904834in}}%
\pgfpathlineto{\pgfqpoint{0.634712in}{1.904977in}}%
\pgfpathlineto{\pgfqpoint{0.641640in}{1.910526in}}%
\pgfpathlineto{\pgfqpoint{0.642058in}{1.920175in}}%
\pgfpathlineto{\pgfqpoint{0.634712in}{1.923512in}}%
\pgfpathlineto{\pgfqpoint{0.632633in}{1.920175in}}%
\pgfpathlineto{\pgfqpoint{0.625000in}{1.911856in}}%
\pgfusepath{stroke}%
\end{pgfscope}%
\begin{pgfscope}%
\pgfpathrectangle{\pgfqpoint{0.625000in}{0.550000in}}{\pgfqpoint{3.875000in}{3.850000in}} %
\pgfusepath{clip}%
\pgfsetbuttcap%
\pgfsetroundjoin%
\pgfsetlinewidth{0.501875pt}%
\definecolor{currentstroke}{rgb}{0.000000,0.000000,0.000000}%
\pgfsetstrokecolor{currentstroke}%
\pgfsetdash{}{0pt}%
\pgfpathmoveto{\pgfqpoint{0.625000in}{2.011946in}}%
\pgfpathlineto{\pgfqpoint{0.631557in}{2.007018in}}%
\pgfpathlineto{\pgfqpoint{0.634712in}{2.002193in}}%
\pgfpathlineto{\pgfqpoint{0.654135in}{2.009593in}}%
\pgfpathlineto{\pgfqpoint{0.683271in}{2.020516in}}%
\pgfpathlineto{\pgfqpoint{0.741541in}{2.037847in}}%
\pgfpathlineto{\pgfqpoint{0.760965in}{2.046294in}}%
\pgfpathlineto{\pgfqpoint{0.780388in}{2.056966in}}%
\pgfpathlineto{\pgfqpoint{0.799812in}{2.070365in}}%
\pgfpathlineto{\pgfqpoint{0.819236in}{2.087239in}}%
\pgfpathlineto{\pgfqpoint{0.834286in}{2.103509in}}%
\pgfpathlineto{\pgfqpoint{0.848715in}{2.122807in}}%
\pgfpathlineto{\pgfqpoint{0.860157in}{2.142105in}}%
\pgfpathlineto{\pgfqpoint{0.869105in}{2.161404in}}%
\pgfpathlineto{\pgfqpoint{0.877506in}{2.186584in}}%
\pgfpathlineto{\pgfqpoint{0.880711in}{2.200000in}}%
\pgfpathlineto{\pgfqpoint{0.883697in}{2.219298in}}%
\pgfpathlineto{\pgfqpoint{0.884920in}{2.238596in}}%
\pgfpathlineto{\pgfqpoint{0.884441in}{2.257895in}}%
\pgfpathlineto{\pgfqpoint{0.882254in}{2.277193in}}%
\pgfpathlineto{\pgfqpoint{0.877506in}{2.299027in}}%
\pgfpathlineto{\pgfqpoint{0.872504in}{2.315789in}}%
\pgfpathlineto{\pgfqpoint{0.864723in}{2.335088in}}%
\pgfpathlineto{\pgfqpoint{0.854776in}{2.354386in}}%
\pgfpathlineto{\pgfqpoint{0.838659in}{2.378361in}}%
\pgfpathlineto{\pgfqpoint{0.826902in}{2.392982in}}%
\pgfpathlineto{\pgfqpoint{0.807741in}{2.412281in}}%
\pgfpathlineto{\pgfqpoint{0.783487in}{2.431579in}}%
\pgfpathlineto{\pgfqpoint{0.750828in}{2.450877in}}%
\pgfpathlineto{\pgfqpoint{0.741541in}{2.455137in}}%
\pgfpathlineto{\pgfqpoint{0.731830in}{2.457139in}}%
\pgfpathlineto{\pgfqpoint{0.729314in}{2.460526in}}%
\pgfpathlineto{\pgfqpoint{0.722118in}{2.463043in}}%
\pgfpathlineto{\pgfqpoint{0.712406in}{2.462188in}}%
\pgfpathlineto{\pgfqpoint{0.710871in}{2.460526in}}%
\pgfpathlineto{\pgfqpoint{0.712406in}{2.459593in}}%
\pgfpathlineto{\pgfqpoint{0.722118in}{2.456159in}}%
\pgfpathlineto{\pgfqpoint{0.731830in}{2.448372in}}%
\pgfpathlineto{\pgfqpoint{0.742205in}{2.441228in}}%
\pgfpathlineto{\pgfqpoint{0.763825in}{2.421930in}}%
\pgfpathlineto{\pgfqpoint{0.780388in}{2.401227in}}%
\pgfpathlineto{\pgfqpoint{0.791357in}{2.383333in}}%
\pgfpathlineto{\pgfqpoint{0.799916in}{2.364035in}}%
\pgfpathlineto{\pgfqpoint{0.805898in}{2.344737in}}%
\pgfpathlineto{\pgfqpoint{0.809524in}{2.321935in}}%
\pgfpathlineto{\pgfqpoint{0.810349in}{2.306140in}}%
\pgfpathlineto{\pgfqpoint{0.809135in}{2.286842in}}%
\pgfpathlineto{\pgfqpoint{0.805604in}{2.267544in}}%
\pgfpathlineto{\pgfqpoint{0.799469in}{2.248246in}}%
\pgfpathlineto{\pgfqpoint{0.790100in}{2.228211in}}%
\pgfpathlineto{\pgfqpoint{0.778100in}{2.209649in}}%
\pgfpathlineto{\pgfqpoint{0.761154in}{2.190351in}}%
\pgfpathlineto{\pgfqpoint{0.750297in}{2.180702in}}%
\pgfpathlineto{\pgfqpoint{0.731830in}{2.167609in}}%
\pgfpathlineto{\pgfqpoint{0.712406in}{2.157476in}}%
\pgfpathlineto{\pgfqpoint{0.692982in}{2.150370in}}%
\pgfpathlineto{\pgfqpoint{0.673559in}{2.146022in}}%
\pgfpathlineto{\pgfqpoint{0.663847in}{2.145291in}}%
\pgfpathlineto{\pgfqpoint{0.654135in}{2.146283in}}%
\pgfpathlineto{\pgfqpoint{0.634712in}{2.153826in}}%
\pgfpathlineto{\pgfqpoint{0.631486in}{2.151754in}}%
\pgfpathlineto{\pgfqpoint{0.625000in}{2.146258in}}%
\pgfpathlineto{\pgfqpoint{0.625000in}{2.146258in}}%
\pgfusepath{stroke}%
\end{pgfscope}%
\begin{pgfscope}%
\pgfpathrectangle{\pgfqpoint{0.625000in}{0.550000in}}{\pgfqpoint{3.875000in}{3.850000in}} %
\pgfusepath{clip}%
\pgfsetbuttcap%
\pgfsetroundjoin%
\pgfsetlinewidth{0.501875pt}%
\definecolor{currentstroke}{rgb}{0.000000,0.000000,0.000000}%
\pgfsetstrokecolor{currentstroke}%
\pgfsetdash{}{0pt}%
\pgfpathmoveto{\pgfqpoint{0.625000in}{2.091674in}}%
\pgfpathlineto{\pgfqpoint{0.633848in}{2.084211in}}%
\pgfpathlineto{\pgfqpoint{0.634455in}{2.074561in}}%
\pgfpathlineto{\pgfqpoint{0.634712in}{2.072543in}}%
\pgfpathlineto{\pgfqpoint{0.641167in}{2.064912in}}%
\pgfpathlineto{\pgfqpoint{0.641042in}{2.055263in}}%
\pgfpathlineto{\pgfqpoint{0.634712in}{2.051177in}}%
\pgfpathlineto{\pgfqpoint{0.632891in}{2.045614in}}%
\pgfpathlineto{\pgfqpoint{0.627503in}{2.035965in}}%
\pgfpathlineto{\pgfqpoint{0.625000in}{2.031759in}}%
\pgfusepath{stroke}%
\end{pgfscope}%
\begin{pgfscope}%
\pgfpathrectangle{\pgfqpoint{0.625000in}{0.550000in}}{\pgfqpoint{3.875000in}{3.850000in}} %
\pgfusepath{clip}%
\pgfsetbuttcap%
\pgfsetroundjoin%
\pgfsetlinewidth{0.501875pt}%
\definecolor{currentstroke}{rgb}{0.000000,0.000000,0.000000}%
\pgfsetstrokecolor{currentstroke}%
\pgfsetdash{}{0pt}%
\pgfpathmoveto{\pgfqpoint{0.625000in}{2.141358in}}%
\pgfpathlineto{\pgfqpoint{0.632908in}{2.132456in}}%
\pgfpathlineto{\pgfqpoint{0.634712in}{2.125354in}}%
\pgfpathlineto{\pgfqpoint{0.639718in}{2.122807in}}%
\pgfpathlineto{\pgfqpoint{0.634712in}{2.115059in}}%
\pgfpathlineto{\pgfqpoint{0.634521in}{2.113158in}}%
\pgfpathlineto{\pgfqpoint{0.632866in}{2.103509in}}%
\pgfpathlineto{\pgfqpoint{0.625000in}{2.097846in}}%
\pgfusepath{stroke}%
\end{pgfscope}%
\begin{pgfscope}%
\pgfpathrectangle{\pgfqpoint{0.625000in}{0.550000in}}{\pgfqpoint{3.875000in}{3.850000in}} %
\pgfusepath{clip}%
\pgfsetbuttcap%
\pgfsetroundjoin%
\pgfsetlinewidth{0.501875pt}%
\definecolor{currentstroke}{rgb}{0.000000,0.000000,0.000000}%
\pgfsetstrokecolor{currentstroke}%
\pgfsetdash{}{0pt}%
\pgfpathmoveto{\pgfqpoint{0.625000in}{2.197899in}}%
\pgfpathlineto{\pgfqpoint{0.625636in}{2.200000in}}%
\pgfpathlineto{\pgfqpoint{0.625000in}{2.200289in}}%
\pgfusepath{stroke}%
\end{pgfscope}%
\begin{pgfscope}%
\pgfpathrectangle{\pgfqpoint{0.625000in}{0.550000in}}{\pgfqpoint{3.875000in}{3.850000in}} %
\pgfusepath{clip}%
\pgfsetbuttcap%
\pgfsetroundjoin%
\pgfsetlinewidth{0.501875pt}%
\definecolor{currentstroke}{rgb}{0.000000,0.000000,0.000000}%
\pgfsetstrokecolor{currentstroke}%
\pgfsetdash{}{0pt}%
\pgfpathmoveto{\pgfqpoint{0.625000in}{2.294142in}}%
\pgfpathlineto{\pgfqpoint{0.627879in}{2.296491in}}%
\pgfpathlineto{\pgfqpoint{0.625000in}{2.303664in}}%
\pgfusepath{stroke}%
\end{pgfscope}%
\begin{pgfscope}%
\pgfpathrectangle{\pgfqpoint{0.625000in}{0.550000in}}{\pgfqpoint{3.875000in}{3.850000in}} %
\pgfusepath{clip}%
\pgfsetbuttcap%
\pgfsetroundjoin%
\pgfsetlinewidth{0.501875pt}%
\definecolor{currentstroke}{rgb}{0.000000,0.000000,0.000000}%
\pgfsetstrokecolor{currentstroke}%
\pgfsetdash{}{0pt}%
\pgfpathmoveto{\pgfqpoint{0.625000in}{2.335021in}}%
\pgfpathlineto{\pgfqpoint{0.633743in}{2.325439in}}%
\pgfpathlineto{\pgfqpoint{0.634712in}{2.322295in}}%
\pgfpathlineto{\pgfqpoint{0.644424in}{2.323573in}}%
\pgfpathlineto{\pgfqpoint{0.651298in}{2.325439in}}%
\pgfpathlineto{\pgfqpoint{0.654135in}{2.326163in}}%
\pgfpathlineto{\pgfqpoint{0.663847in}{2.330022in}}%
\pgfpathlineto{\pgfqpoint{0.672487in}{2.335088in}}%
\pgfpathlineto{\pgfqpoint{0.673559in}{2.335726in}}%
\pgfpathlineto{\pgfqpoint{0.683271in}{2.343483in}}%
\pgfpathlineto{\pgfqpoint{0.684506in}{2.344737in}}%
\pgfpathlineto{\pgfqpoint{0.692861in}{2.354386in}}%
\pgfpathlineto{\pgfqpoint{0.692982in}{2.354549in}}%
\pgfpathlineto{\pgfqpoint{0.698880in}{2.364035in}}%
\pgfpathlineto{\pgfqpoint{0.702694in}{2.372611in}}%
\pgfpathlineto{\pgfqpoint{0.703119in}{2.373684in}}%
\pgfpathlineto{\pgfqpoint{0.705977in}{2.383333in}}%
\pgfpathlineto{\pgfqpoint{0.707502in}{2.392982in}}%
\pgfpathlineto{\pgfqpoint{0.707795in}{2.402632in}}%
\pgfpathlineto{\pgfqpoint{0.706876in}{2.412281in}}%
\pgfpathlineto{\pgfqpoint{0.704629in}{2.421930in}}%
\pgfpathlineto{\pgfqpoint{0.702694in}{2.426555in}}%
\pgfpathlineto{\pgfqpoint{0.700064in}{2.421930in}}%
\pgfpathlineto{\pgfqpoint{0.698839in}{2.412281in}}%
\pgfpathlineto{\pgfqpoint{0.698189in}{2.402632in}}%
\pgfpathlineto{\pgfqpoint{0.697374in}{2.392982in}}%
\pgfpathlineto{\pgfqpoint{0.694354in}{2.383333in}}%
\pgfpathlineto{\pgfqpoint{0.692982in}{2.382167in}}%
\pgfpathlineto{\pgfqpoint{0.688753in}{2.373684in}}%
\pgfpathlineto{\pgfqpoint{0.683271in}{2.365705in}}%
\pgfpathlineto{\pgfqpoint{0.681839in}{2.364035in}}%
\pgfpathlineto{\pgfqpoint{0.673559in}{2.355735in}}%
\pgfpathlineto{\pgfqpoint{0.671736in}{2.354386in}}%
\pgfpathlineto{\pgfqpoint{0.663847in}{2.348838in}}%
\pgfpathlineto{\pgfqpoint{0.655157in}{2.344737in}}%
\pgfpathlineto{\pgfqpoint{0.654135in}{2.344176in}}%
\pgfpathlineto{\pgfqpoint{0.644424in}{2.341110in}}%
\pgfpathlineto{\pgfqpoint{0.634712in}{2.339490in}}%
\pgfpathlineto{\pgfqpoint{0.625000in}{2.335645in}}%
\pgfusepath{stroke}%
\end{pgfscope}%
\begin{pgfscope}%
\pgfpathrectangle{\pgfqpoint{0.625000in}{0.550000in}}{\pgfqpoint{3.875000in}{3.850000in}} %
\pgfusepath{clip}%
\pgfsetbuttcap%
\pgfsetroundjoin%
\pgfsetlinewidth{0.501875pt}%
\definecolor{currentstroke}{rgb}{0.000000,0.000000,0.000000}%
\pgfsetstrokecolor{currentstroke}%
\pgfsetdash{}{0pt}%
\pgfpathmoveto{\pgfqpoint{0.625000in}{2.462988in}}%
\pgfpathlineto{\pgfqpoint{0.626873in}{2.470175in}}%
\pgfpathlineto{\pgfqpoint{0.625000in}{2.473733in}}%
\pgfusepath{stroke}%
\end{pgfscope}%
\begin{pgfscope}%
\pgfpathrectangle{\pgfqpoint{0.625000in}{0.550000in}}{\pgfqpoint{3.875000in}{3.850000in}} %
\pgfusepath{clip}%
\pgfsetbuttcap%
\pgfsetroundjoin%
\pgfsetlinewidth{0.501875pt}%
\definecolor{currentstroke}{rgb}{0.000000,0.000000,0.000000}%
\pgfsetstrokecolor{currentstroke}%
\pgfsetdash{}{0pt}%
\pgfpathmoveto{\pgfqpoint{0.625000in}{2.485916in}}%
\pgfpathlineto{\pgfqpoint{0.634712in}{2.487352in}}%
\pgfpathlineto{\pgfqpoint{0.644424in}{2.486584in}}%
\pgfpathlineto{\pgfqpoint{0.654135in}{2.487116in}}%
\pgfpathlineto{\pgfqpoint{0.660095in}{2.489474in}}%
\pgfpathlineto{\pgfqpoint{0.654135in}{2.490735in}}%
\pgfpathlineto{\pgfqpoint{0.644424in}{2.491532in}}%
\pgfpathlineto{\pgfqpoint{0.634712in}{2.490602in}}%
\pgfpathlineto{\pgfqpoint{0.625000in}{2.496661in}}%
\pgfusepath{stroke}%
\end{pgfscope}%
\begin{pgfscope}%
\pgfpathrectangle{\pgfqpoint{0.625000in}{0.550000in}}{\pgfqpoint{3.875000in}{3.850000in}} %
\pgfusepath{clip}%
\pgfsetbuttcap%
\pgfsetroundjoin%
\pgfsetlinewidth{0.501875pt}%
\definecolor{currentstroke}{rgb}{0.000000,0.000000,0.000000}%
\pgfsetstrokecolor{currentstroke}%
\pgfsetdash{}{0pt}%
\pgfpathmoveto{\pgfqpoint{0.625000in}{2.624004in}}%
\pgfpathlineto{\pgfqpoint{0.634712in}{2.620159in}}%
\pgfpathlineto{\pgfqpoint{0.644424in}{2.618539in}}%
\pgfpathlineto{\pgfqpoint{0.654135in}{2.615473in}}%
\pgfpathlineto{\pgfqpoint{0.655157in}{2.614912in}}%
\pgfpathlineto{\pgfqpoint{0.663847in}{2.610811in}}%
\pgfpathlineto{\pgfqpoint{0.671736in}{2.605263in}}%
\pgfpathlineto{\pgfqpoint{0.673559in}{2.603914in}}%
\pgfpathlineto{\pgfqpoint{0.681839in}{2.595614in}}%
\pgfpathlineto{\pgfqpoint{0.683271in}{2.593944in}}%
\pgfpathlineto{\pgfqpoint{0.688753in}{2.585965in}}%
\pgfpathlineto{\pgfqpoint{0.692982in}{2.577482in}}%
\pgfpathlineto{\pgfqpoint{0.694354in}{2.576316in}}%
\pgfpathlineto{\pgfqpoint{0.697374in}{2.566667in}}%
\pgfpathlineto{\pgfqpoint{0.698189in}{2.557018in}}%
\pgfpathlineto{\pgfqpoint{0.698839in}{2.547368in}}%
\pgfpathlineto{\pgfqpoint{0.700064in}{2.537719in}}%
\pgfpathlineto{\pgfqpoint{0.702694in}{2.533094in}}%
\pgfpathlineto{\pgfqpoint{0.704629in}{2.537719in}}%
\pgfpathlineto{\pgfqpoint{0.706876in}{2.547368in}}%
\pgfpathlineto{\pgfqpoint{0.707795in}{2.557018in}}%
\pgfpathlineto{\pgfqpoint{0.707502in}{2.566667in}}%
\pgfpathlineto{\pgfqpoint{0.705977in}{2.576316in}}%
\pgfpathlineto{\pgfqpoint{0.703119in}{2.585965in}}%
\pgfpathlineto{\pgfqpoint{0.702694in}{2.587038in}}%
\pgfpathlineto{\pgfqpoint{0.698880in}{2.595614in}}%
\pgfpathlineto{\pgfqpoint{0.692982in}{2.605100in}}%
\pgfpathlineto{\pgfqpoint{0.692861in}{2.605263in}}%
\pgfpathlineto{\pgfqpoint{0.684506in}{2.614912in}}%
\pgfpathlineto{\pgfqpoint{0.683271in}{2.616167in}}%
\pgfpathlineto{\pgfqpoint{0.673559in}{2.623924in}}%
\pgfpathlineto{\pgfqpoint{0.672487in}{2.624561in}}%
\pgfpathlineto{\pgfqpoint{0.663847in}{2.629627in}}%
\pgfpathlineto{\pgfqpoint{0.654135in}{2.633486in}}%
\pgfpathlineto{\pgfqpoint{0.651298in}{2.634211in}}%
\pgfpathlineto{\pgfqpoint{0.644424in}{2.636076in}}%
\pgfpathlineto{\pgfqpoint{0.634712in}{2.637354in}}%
\pgfpathlineto{\pgfqpoint{0.631577in}{2.634211in}}%
\pgfpathlineto{\pgfqpoint{0.625000in}{2.624842in}}%
\pgfusepath{stroke}%
\end{pgfscope}%
\begin{pgfscope}%
\pgfpathrectangle{\pgfqpoint{0.625000in}{0.550000in}}{\pgfqpoint{3.875000in}{3.850000in}} %
\pgfusepath{clip}%
\pgfsetbuttcap%
\pgfsetroundjoin%
\pgfsetlinewidth{0.501875pt}%
\definecolor{currentstroke}{rgb}{0.000000,0.000000,0.000000}%
\pgfsetstrokecolor{currentstroke}%
\pgfsetdash{}{0pt}%
\pgfpathmoveto{\pgfqpoint{0.625000in}{2.655985in}}%
\pgfpathlineto{\pgfqpoint{0.627879in}{2.663158in}}%
\pgfpathlineto{\pgfqpoint{0.625000in}{2.665507in}}%
\pgfusepath{stroke}%
\end{pgfscope}%
\begin{pgfscope}%
\pgfpathrectangle{\pgfqpoint{0.625000in}{0.550000in}}{\pgfqpoint{3.875000in}{3.850000in}} %
\pgfusepath{clip}%
\pgfsetbuttcap%
\pgfsetroundjoin%
\pgfsetlinewidth{0.501875pt}%
\definecolor{currentstroke}{rgb}{0.000000,0.000000,0.000000}%
\pgfsetstrokecolor{currentstroke}%
\pgfsetdash{}{0pt}%
\pgfpathmoveto{\pgfqpoint{0.625000in}{2.759360in}}%
\pgfpathlineto{\pgfqpoint{0.625636in}{2.759649in}}%
\pgfpathlineto{\pgfqpoint{0.625000in}{2.761750in}}%
\pgfusepath{stroke}%
\end{pgfscope}%
\begin{pgfscope}%
\pgfpathrectangle{\pgfqpoint{0.625000in}{0.550000in}}{\pgfqpoint{3.875000in}{3.850000in}} %
\pgfusepath{clip}%
\pgfsetbuttcap%
\pgfsetroundjoin%
\pgfsetlinewidth{0.501875pt}%
\definecolor{currentstroke}{rgb}{0.000000,0.000000,0.000000}%
\pgfsetstrokecolor{currentstroke}%
\pgfsetdash{}{0pt}%
\pgfpathmoveto{\pgfqpoint{0.625000in}{2.785783in}}%
\pgfpathlineto{\pgfqpoint{0.628475in}{2.788596in}}%
\pgfpathlineto{\pgfqpoint{0.625000in}{2.791709in}}%
\pgfusepath{stroke}%
\end{pgfscope}%
\begin{pgfscope}%
\pgfpathrectangle{\pgfqpoint{0.625000in}{0.550000in}}{\pgfqpoint{3.875000in}{3.850000in}} %
\pgfusepath{clip}%
\pgfsetbuttcap%
\pgfsetroundjoin%
\pgfsetlinewidth{0.501875pt}%
\definecolor{currentstroke}{rgb}{0.000000,0.000000,0.000000}%
\pgfsetstrokecolor{currentstroke}%
\pgfsetdash{}{0pt}%
\pgfpathmoveto{\pgfqpoint{0.625000in}{2.813391in}}%
\pgfpathlineto{\pgfqpoint{0.631486in}{2.807895in}}%
\pgfpathlineto{\pgfqpoint{0.634712in}{2.805823in}}%
\pgfpathlineto{\pgfqpoint{0.641836in}{2.807895in}}%
\pgfpathlineto{\pgfqpoint{0.644424in}{2.809851in}}%
\pgfpathlineto{\pgfqpoint{0.654135in}{2.813366in}}%
\pgfpathlineto{\pgfqpoint{0.663847in}{2.814358in}}%
\pgfpathlineto{\pgfqpoint{0.683271in}{2.811848in}}%
\pgfpathlineto{\pgfqpoint{0.702694in}{2.806069in}}%
\pgfpathlineto{\pgfqpoint{0.722118in}{2.797478in}}%
\pgfpathlineto{\pgfqpoint{0.741541in}{2.785652in}}%
\pgfpathlineto{\pgfqpoint{0.761154in}{2.769298in}}%
\pgfpathlineto{\pgfqpoint{0.778100in}{2.750000in}}%
\pgfpathlineto{\pgfqpoint{0.784792in}{2.740351in}}%
\pgfpathlineto{\pgfqpoint{0.795393in}{2.721053in}}%
\pgfpathlineto{\pgfqpoint{0.802878in}{2.701754in}}%
\pgfpathlineto{\pgfqpoint{0.807678in}{2.682456in}}%
\pgfpathlineto{\pgfqpoint{0.810028in}{2.663158in}}%
\pgfpathlineto{\pgfqpoint{0.810082in}{2.643860in}}%
\pgfpathlineto{\pgfqpoint{0.807867in}{2.624561in}}%
\pgfpathlineto{\pgfqpoint{0.803279in}{2.605263in}}%
\pgfpathlineto{\pgfqpoint{0.796092in}{2.585965in}}%
\pgfpathlineto{\pgfqpoint{0.785982in}{2.566667in}}%
\pgfpathlineto{\pgfqpoint{0.770677in}{2.545716in}}%
\pgfpathlineto{\pgfqpoint{0.760965in}{2.535107in}}%
\pgfpathlineto{\pgfqpoint{0.751253in}{2.525974in}}%
\pgfpathlineto{\pgfqpoint{0.741541in}{2.517644in}}%
\pgfpathlineto{\pgfqpoint{0.728447in}{2.508772in}}%
\pgfpathlineto{\pgfqpoint{0.722118in}{2.503490in}}%
\pgfpathlineto{\pgfqpoint{0.710871in}{2.499123in}}%
\pgfpathlineto{\pgfqpoint{0.712406in}{2.497461in}}%
\pgfpathlineto{\pgfqpoint{0.722118in}{2.496606in}}%
\pgfpathlineto{\pgfqpoint{0.729314in}{2.499123in}}%
\pgfpathlineto{\pgfqpoint{0.731830in}{2.502510in}}%
\pgfpathlineto{\pgfqpoint{0.741541in}{2.504513in}}%
\pgfpathlineto{\pgfqpoint{0.760965in}{2.514263in}}%
\pgfpathlineto{\pgfqpoint{0.780388in}{2.526144in}}%
\pgfpathlineto{\pgfqpoint{0.790100in}{2.533109in}}%
\pgfpathlineto{\pgfqpoint{0.799812in}{2.540735in}}%
\pgfpathlineto{\pgfqpoint{0.817836in}{2.557018in}}%
\pgfpathlineto{\pgfqpoint{0.842341in}{2.585965in}}%
\pgfpathlineto{\pgfqpoint{0.858083in}{2.611547in}}%
\pgfpathlineto{\pgfqpoint{0.864723in}{2.624561in}}%
\pgfpathlineto{\pgfqpoint{0.875609in}{2.653509in}}%
\pgfpathlineto{\pgfqpoint{0.882254in}{2.682456in}}%
\pgfpathlineto{\pgfqpoint{0.884441in}{2.701754in}}%
\pgfpathlineto{\pgfqpoint{0.884920in}{2.721053in}}%
\pgfpathlineto{\pgfqpoint{0.883697in}{2.740351in}}%
\pgfpathlineto{\pgfqpoint{0.880711in}{2.759649in}}%
\pgfpathlineto{\pgfqpoint{0.875885in}{2.778947in}}%
\pgfpathlineto{\pgfqpoint{0.867794in}{2.801344in}}%
\pgfpathlineto{\pgfqpoint{0.858083in}{2.821369in}}%
\pgfpathlineto{\pgfqpoint{0.848371in}{2.837358in}}%
\pgfpathlineto{\pgfqpoint{0.834286in}{2.856140in}}%
\pgfpathlineto{\pgfqpoint{0.819236in}{2.872410in}}%
\pgfpathlineto{\pgfqpoint{0.799812in}{2.889284in}}%
\pgfpathlineto{\pgfqpoint{0.777493in}{2.904386in}}%
\pgfpathlineto{\pgfqpoint{0.759501in}{2.914035in}}%
\pgfpathlineto{\pgfqpoint{0.736317in}{2.923684in}}%
\pgfpathlineto{\pgfqpoint{0.712406in}{2.931193in}}%
\pgfpathlineto{\pgfqpoint{0.672587in}{2.942982in}}%
\pgfpathlineto{\pgfqpoint{0.654135in}{2.950057in}}%
\pgfpathlineto{\pgfqpoint{0.644424in}{2.954108in}}%
\pgfpathlineto{\pgfqpoint{0.634712in}{2.957456in}}%
\pgfpathlineto{\pgfqpoint{0.631557in}{2.952632in}}%
\pgfpathlineto{\pgfqpoint{0.633396in}{2.942982in}}%
\pgfpathlineto{\pgfqpoint{0.625000in}{2.934300in}}%
\pgfpathlineto{\pgfqpoint{0.625000in}{2.934300in}}%
\pgfusepath{stroke}%
\end{pgfscope}%
\begin{pgfscope}%
\pgfpathrectangle{\pgfqpoint{0.625000in}{0.550000in}}{\pgfqpoint{3.875000in}{3.850000in}} %
\pgfusepath{clip}%
\pgfsetbuttcap%
\pgfsetroundjoin%
\pgfsetlinewidth{0.501875pt}%
\definecolor{currentstroke}{rgb}{0.000000,0.000000,0.000000}%
\pgfsetstrokecolor{currentstroke}%
\pgfsetdash{}{0pt}%
\pgfpathmoveto{\pgfqpoint{0.625000in}{2.861804in}}%
\pgfpathlineto{\pgfqpoint{0.632866in}{2.856140in}}%
\pgfpathlineto{\pgfqpoint{0.634521in}{2.846491in}}%
\pgfpathlineto{\pgfqpoint{0.634712in}{2.844590in}}%
\pgfpathlineto{\pgfqpoint{0.639718in}{2.836842in}}%
\pgfpathlineto{\pgfqpoint{0.634712in}{2.834295in}}%
\pgfpathlineto{\pgfqpoint{0.632908in}{2.827193in}}%
\pgfpathlineto{\pgfqpoint{0.625000in}{2.818291in}}%
\pgfusepath{stroke}%
\end{pgfscope}%
\begin{pgfscope}%
\pgfpathrectangle{\pgfqpoint{0.625000in}{0.550000in}}{\pgfqpoint{3.875000in}{3.850000in}} %
\pgfusepath{clip}%
\pgfsetbuttcap%
\pgfsetroundjoin%
\pgfsetlinewidth{0.501875pt}%
\definecolor{currentstroke}{rgb}{0.000000,0.000000,0.000000}%
\pgfsetstrokecolor{currentstroke}%
\pgfsetdash{}{0pt}%
\pgfpathmoveto{\pgfqpoint{0.625000in}{2.927890in}}%
\pgfpathlineto{\pgfqpoint{0.627503in}{2.923684in}}%
\pgfpathlineto{\pgfqpoint{0.632891in}{2.914035in}}%
\pgfpathlineto{\pgfqpoint{0.634712in}{2.908472in}}%
\pgfpathlineto{\pgfqpoint{0.641042in}{2.904386in}}%
\pgfpathlineto{\pgfqpoint{0.641167in}{2.894737in}}%
\pgfpathlineto{\pgfqpoint{0.634712in}{2.887106in}}%
\pgfpathlineto{\pgfqpoint{0.634455in}{2.885088in}}%
\pgfpathlineto{\pgfqpoint{0.633848in}{2.875439in}}%
\pgfpathlineto{\pgfqpoint{0.625000in}{2.867975in}}%
\pgfusepath{stroke}%
\end{pgfscope}%
\begin{pgfscope}%
\pgfpathrectangle{\pgfqpoint{0.625000in}{0.550000in}}{\pgfqpoint{3.875000in}{3.850000in}} %
\pgfusepath{clip}%
\pgfsetbuttcap%
\pgfsetroundjoin%
\pgfsetlinewidth{0.501875pt}%
\definecolor{currentstroke}{rgb}{0.000000,0.000000,0.000000}%
\pgfsetstrokecolor{currentstroke}%
\pgfsetdash{}{0pt}%
\pgfpathmoveto{\pgfqpoint{0.625000in}{3.047793in}}%
\pgfpathlineto{\pgfqpoint{0.632633in}{3.039474in}}%
\pgfpathlineto{\pgfqpoint{0.634712in}{3.036137in}}%
\pgfpathlineto{\pgfqpoint{0.642058in}{3.039474in}}%
\pgfpathlineto{\pgfqpoint{0.641640in}{3.049123in}}%
\pgfpathlineto{\pgfqpoint{0.634712in}{3.054672in}}%
\pgfpathlineto{\pgfqpoint{0.625000in}{3.054815in}}%
\pgfusepath{stroke}%
\end{pgfscope}%
\begin{pgfscope}%
\pgfpathrectangle{\pgfqpoint{0.625000in}{0.550000in}}{\pgfqpoint{3.875000in}{3.850000in}} %
\pgfusepath{clip}%
\pgfsetbuttcap%
\pgfsetroundjoin%
\pgfsetlinewidth{0.501875pt}%
\definecolor{currentstroke}{rgb}{0.000000,0.000000,0.000000}%
\pgfsetstrokecolor{currentstroke}%
\pgfsetdash{}{0pt}%
\pgfpathmoveto{\pgfqpoint{0.625000in}{3.143805in}}%
\pgfpathlineto{\pgfqpoint{0.627008in}{3.145614in}}%
\pgfpathlineto{\pgfqpoint{0.625000in}{3.147423in}}%
\pgfusepath{stroke}%
\end{pgfscope}%
\begin{pgfscope}%
\pgfpathrectangle{\pgfqpoint{0.625000in}{0.550000in}}{\pgfqpoint{3.875000in}{3.850000in}} %
\pgfusepath{clip}%
\pgfsetbuttcap%
\pgfsetroundjoin%
\pgfsetlinewidth{0.501875pt}%
\definecolor{currentstroke}{rgb}{0.000000,0.000000,0.000000}%
\pgfsetstrokecolor{currentstroke}%
\pgfsetdash{}{0pt}%
\pgfpathmoveto{\pgfqpoint{0.625000in}{3.405382in}}%
\pgfpathlineto{\pgfqpoint{0.626033in}{3.406140in}}%
\pgfpathlineto{\pgfqpoint{0.625601in}{3.415789in}}%
\pgfpathlineto{\pgfqpoint{0.625000in}{3.418464in}}%
\pgfusepath{stroke}%
\end{pgfscope}%
\begin{pgfscope}%
\pgfpathrectangle{\pgfqpoint{0.625000in}{0.550000in}}{\pgfqpoint{3.875000in}{3.850000in}} %
\pgfusepath{clip}%
\pgfsetbuttcap%
\pgfsetroundjoin%
\pgfsetlinewidth{0.501875pt}%
\definecolor{currentstroke}{rgb}{0.000000,0.000000,0.000000}%
\pgfsetstrokecolor{currentstroke}%
\pgfsetdash{}{0pt}%
\pgfpathmoveto{\pgfqpoint{0.625000in}{3.529374in}}%
\pgfpathlineto{\pgfqpoint{0.626692in}{3.531579in}}%
\pgfpathlineto{\pgfqpoint{0.625000in}{3.533784in}}%
\pgfusepath{stroke}%
\end{pgfscope}%
\begin{pgfscope}%
\pgfpathrectangle{\pgfqpoint{0.625000in}{0.550000in}}{\pgfqpoint{3.875000in}{3.850000in}} %
\pgfusepath{clip}%
\pgfsetbuttcap%
\pgfsetroundjoin%
\pgfsetlinewidth{0.501875pt}%
\definecolor{currentstroke}{rgb}{0.000000,0.000000,0.000000}%
\pgfsetstrokecolor{currentstroke}%
\pgfsetdash{}{0pt}%
\pgfpathmoveto{\pgfqpoint{0.625000in}{3.585630in}}%
\pgfpathlineto{\pgfqpoint{0.626008in}{3.589474in}}%
\pgfpathlineto{\pgfqpoint{0.625000in}{3.590135in}}%
\pgfusepath{stroke}%
\end{pgfscope}%
\begin{pgfscope}%
\pgfpathrectangle{\pgfqpoint{0.625000in}{0.550000in}}{\pgfqpoint{3.875000in}{3.850000in}} %
\pgfusepath{clip}%
\pgfsetbuttcap%
\pgfsetroundjoin%
\pgfsetlinewidth{0.501875pt}%
\definecolor{currentstroke}{rgb}{0.000000,0.000000,0.000000}%
\pgfsetstrokecolor{currentstroke}%
\pgfsetdash{}{0pt}%
\pgfpathmoveto{\pgfqpoint{0.634712in}{0.916173in}}%
\pgfpathlineto{\pgfqpoint{0.635153in}{0.916667in}}%
\pgfpathlineto{\pgfqpoint{0.637451in}{0.926316in}}%
\pgfpathlineto{\pgfqpoint{0.634712in}{0.927813in}}%
\pgfpathlineto{\pgfqpoint{0.633522in}{0.926316in}}%
\pgfpathlineto{\pgfqpoint{0.633226in}{0.916667in}}%
\pgfpathlineto{\pgfqpoint{0.634712in}{0.916173in}}%
\pgfusepath{stroke}%
\end{pgfscope}%
\begin{pgfscope}%
\pgfpathrectangle{\pgfqpoint{0.625000in}{0.550000in}}{\pgfqpoint{3.875000in}{3.850000in}} %
\pgfusepath{clip}%
\pgfsetbuttcap%
\pgfsetroundjoin%
\pgfsetlinewidth{0.501875pt}%
\definecolor{currentstroke}{rgb}{0.000000,0.000000,0.000000}%
\pgfsetstrokecolor{currentstroke}%
\pgfsetdash{}{0pt}%
\pgfpathmoveto{\pgfqpoint{0.634712in}{1.133110in}}%
\pgfpathlineto{\pgfqpoint{0.644424in}{1.136094in}}%
\pgfpathlineto{\pgfqpoint{0.647957in}{1.138596in}}%
\pgfpathlineto{\pgfqpoint{0.654135in}{1.144232in}}%
\pgfpathlineto{\pgfqpoint{0.656845in}{1.148246in}}%
\pgfpathlineto{\pgfqpoint{0.660219in}{1.157895in}}%
\pgfpathlineto{\pgfqpoint{0.660121in}{1.167544in}}%
\pgfpathlineto{\pgfqpoint{0.656239in}{1.177193in}}%
\pgfpathlineto{\pgfqpoint{0.654135in}{1.178846in}}%
\pgfpathlineto{\pgfqpoint{0.652168in}{1.177193in}}%
\pgfpathlineto{\pgfqpoint{0.649383in}{1.167544in}}%
\pgfpathlineto{\pgfqpoint{0.644424in}{1.159104in}}%
\pgfpathlineto{\pgfqpoint{0.643306in}{1.157895in}}%
\pgfpathlineto{\pgfqpoint{0.634712in}{1.151633in}}%
\pgfpathlineto{\pgfqpoint{0.633266in}{1.148246in}}%
\pgfpathlineto{\pgfqpoint{0.633404in}{1.138596in}}%
\pgfpathlineto{\pgfqpoint{0.634712in}{1.133110in}}%
\pgfusepath{stroke}%
\end{pgfscope}%
\begin{pgfscope}%
\pgfpathrectangle{\pgfqpoint{0.625000in}{0.550000in}}{\pgfqpoint{3.875000in}{3.850000in}} %
\pgfusepath{clip}%
\pgfsetbuttcap%
\pgfsetroundjoin%
\pgfsetlinewidth{0.501875pt}%
\definecolor{currentstroke}{rgb}{0.000000,0.000000,0.000000}%
\pgfsetstrokecolor{currentstroke}%
\pgfsetdash{}{0pt}%
\pgfpathmoveto{\pgfqpoint{0.644424in}{1.183297in}}%
\pgfpathlineto{\pgfqpoint{0.647540in}{1.186842in}}%
\pgfpathlineto{\pgfqpoint{0.644424in}{1.188744in}}%
\pgfpathlineto{\pgfqpoint{0.640176in}{1.186842in}}%
\pgfpathlineto{\pgfqpoint{0.644424in}{1.183297in}}%
\pgfusepath{stroke}%
\end{pgfscope}%
\begin{pgfscope}%
\pgfpathrectangle{\pgfqpoint{0.625000in}{0.550000in}}{\pgfqpoint{3.875000in}{3.850000in}} %
\pgfusepath{clip}%
\pgfsetbuttcap%
\pgfsetroundjoin%
\pgfsetlinewidth{0.501875pt}%
\definecolor{currentstroke}{rgb}{0.000000,0.000000,0.000000}%
\pgfsetstrokecolor{currentstroke}%
\pgfsetdash{}{0pt}%
\pgfpathmoveto{\pgfqpoint{0.644424in}{1.203922in}}%
\pgfpathlineto{\pgfqpoint{0.647494in}{1.206140in}}%
\pgfpathlineto{\pgfqpoint{0.648198in}{1.215789in}}%
\pgfpathlineto{\pgfqpoint{0.647341in}{1.225439in}}%
\pgfpathlineto{\pgfqpoint{0.644424in}{1.229715in}}%
\pgfpathlineto{\pgfqpoint{0.636567in}{1.235088in}}%
\pgfpathlineto{\pgfqpoint{0.634712in}{1.236180in}}%
\pgfpathlineto{\pgfqpoint{0.634201in}{1.235088in}}%
\pgfpathlineto{\pgfqpoint{0.631654in}{1.225439in}}%
\pgfpathlineto{\pgfqpoint{0.634712in}{1.224885in}}%
\pgfpathlineto{\pgfqpoint{0.642469in}{1.215789in}}%
\pgfpathlineto{\pgfqpoint{0.641852in}{1.206140in}}%
\pgfpathlineto{\pgfqpoint{0.644424in}{1.203922in}}%
\pgfusepath{stroke}%
\end{pgfscope}%
\begin{pgfscope}%
\pgfpathrectangle{\pgfqpoint{0.625000in}{0.550000in}}{\pgfqpoint{3.875000in}{3.850000in}} %
\pgfusepath{clip}%
\pgfsetbuttcap%
\pgfsetroundjoin%
\pgfsetlinewidth{0.501875pt}%
\definecolor{currentstroke}{rgb}{0.000000,0.000000,0.000000}%
\pgfsetstrokecolor{currentstroke}%
\pgfsetdash{}{0pt}%
\pgfpathmoveto{\pgfqpoint{0.634712in}{1.679570in}}%
\pgfpathlineto{\pgfqpoint{0.642614in}{1.688596in}}%
\pgfpathlineto{\pgfqpoint{0.634712in}{1.694885in}}%
\pgfpathlineto{\pgfqpoint{0.627683in}{1.688596in}}%
\pgfpathlineto{\pgfqpoint{0.634712in}{1.679570in}}%
\pgfusepath{stroke}%
\end{pgfscope}%
\begin{pgfscope}%
\pgfpathrectangle{\pgfqpoint{0.625000in}{0.550000in}}{\pgfqpoint{3.875000in}{3.850000in}} %
\pgfusepath{clip}%
\pgfsetbuttcap%
\pgfsetroundjoin%
\pgfsetlinewidth{0.501875pt}%
\definecolor{currentstroke}{rgb}{0.000000,0.000000,0.000000}%
\pgfsetstrokecolor{currentstroke}%
\pgfsetdash{}{0pt}%
\pgfpathmoveto{\pgfqpoint{0.644424in}{2.389826in}}%
\pgfpathlineto{\pgfqpoint{0.653220in}{2.392982in}}%
\pgfpathlineto{\pgfqpoint{0.644424in}{2.394751in}}%
\pgfpathlineto{\pgfqpoint{0.638449in}{2.392982in}}%
\pgfpathlineto{\pgfqpoint{0.644424in}{2.389826in}}%
\pgfusepath{stroke}%
\end{pgfscope}%
\begin{pgfscope}%
\pgfpathrectangle{\pgfqpoint{0.625000in}{0.550000in}}{\pgfqpoint{3.875000in}{3.850000in}} %
\pgfusepath{clip}%
\pgfsetbuttcap%
\pgfsetroundjoin%
\pgfsetlinewidth{0.501875pt}%
\definecolor{currentstroke}{rgb}{0.000000,0.000000,0.000000}%
\pgfsetstrokecolor{currentstroke}%
\pgfsetdash{}{0pt}%
\pgfpathmoveto{\pgfqpoint{0.663847in}{2.402336in}}%
\pgfpathlineto{\pgfqpoint{0.664071in}{2.402632in}}%
\pgfpathlineto{\pgfqpoint{0.663847in}{2.405287in}}%
\pgfpathlineto{\pgfqpoint{0.662853in}{2.402632in}}%
\pgfpathlineto{\pgfqpoint{0.663847in}{2.402336in}}%
\pgfusepath{stroke}%
\end{pgfscope}%
\begin{pgfscope}%
\pgfpathrectangle{\pgfqpoint{0.625000in}{0.550000in}}{\pgfqpoint{3.875000in}{3.850000in}} %
\pgfusepath{clip}%
\pgfsetbuttcap%
\pgfsetroundjoin%
\pgfsetlinewidth{0.501875pt}%
\definecolor{currentstroke}{rgb}{0.000000,0.000000,0.000000}%
\pgfsetstrokecolor{currentstroke}%
\pgfsetdash{}{0pt}%
\pgfpathmoveto{\pgfqpoint{0.673559in}{2.425323in}}%
\pgfpathlineto{\pgfqpoint{0.674262in}{2.431579in}}%
\pgfpathlineto{\pgfqpoint{0.673559in}{2.437359in}}%
\pgfpathlineto{\pgfqpoint{0.672828in}{2.431579in}}%
\pgfpathlineto{\pgfqpoint{0.673559in}{2.425323in}}%
\pgfusepath{stroke}%
\end{pgfscope}%
\begin{pgfscope}%
\pgfpathrectangle{\pgfqpoint{0.625000in}{0.550000in}}{\pgfqpoint{3.875000in}{3.850000in}} %
\pgfusepath{clip}%
\pgfsetbuttcap%
\pgfsetroundjoin%
\pgfsetlinewidth{0.501875pt}%
\definecolor{currentstroke}{rgb}{0.000000,0.000000,0.000000}%
\pgfsetstrokecolor{currentstroke}%
\pgfsetdash{}{0pt}%
\pgfpathmoveto{\pgfqpoint{0.692982in}{2.434825in}}%
\pgfpathlineto{\pgfqpoint{0.696090in}{2.441228in}}%
\pgfpathlineto{\pgfqpoint{0.692982in}{2.444940in}}%
\pgfpathlineto{\pgfqpoint{0.690618in}{2.441228in}}%
\pgfpathlineto{\pgfqpoint{0.692982in}{2.434825in}}%
\pgfusepath{stroke}%
\end{pgfscope}%
\begin{pgfscope}%
\pgfpathrectangle{\pgfqpoint{0.625000in}{0.550000in}}{\pgfqpoint{3.875000in}{3.850000in}} %
\pgfusepath{clip}%
\pgfsetbuttcap%
\pgfsetroundjoin%
\pgfsetlinewidth{0.501875pt}%
\definecolor{currentstroke}{rgb}{0.000000,0.000000,0.000000}%
\pgfsetstrokecolor{currentstroke}%
\pgfsetdash{}{0pt}%
\pgfpathmoveto{\pgfqpoint{0.663847in}{2.459203in}}%
\pgfpathlineto{\pgfqpoint{0.667347in}{2.460526in}}%
\pgfpathlineto{\pgfqpoint{0.663847in}{2.464269in}}%
\pgfpathlineto{\pgfqpoint{0.662273in}{2.460526in}}%
\pgfpathlineto{\pgfqpoint{0.663847in}{2.459203in}}%
\pgfusepath{stroke}%
\end{pgfscope}%
\begin{pgfscope}%
\pgfpathrectangle{\pgfqpoint{0.625000in}{0.550000in}}{\pgfqpoint{3.875000in}{3.850000in}} %
\pgfusepath{clip}%
\pgfsetbuttcap%
\pgfsetroundjoin%
\pgfsetlinewidth{0.501875pt}%
\definecolor{currentstroke}{rgb}{0.000000,0.000000,0.000000}%
\pgfsetstrokecolor{currentstroke}%
\pgfsetdash{}{0pt}%
\pgfpathmoveto{\pgfqpoint{0.692982in}{2.468235in}}%
\pgfpathlineto{\pgfqpoint{0.700008in}{2.470175in}}%
\pgfpathlineto{\pgfqpoint{0.692982in}{2.471804in}}%
\pgfpathlineto{\pgfqpoint{0.687688in}{2.470175in}}%
\pgfpathlineto{\pgfqpoint{0.692982in}{2.468235in}}%
\pgfusepath{stroke}%
\end{pgfscope}%
\begin{pgfscope}%
\pgfpathrectangle{\pgfqpoint{0.625000in}{0.550000in}}{\pgfqpoint{3.875000in}{3.850000in}} %
\pgfusepath{clip}%
\pgfsetbuttcap%
\pgfsetroundjoin%
\pgfsetlinewidth{0.501875pt}%
\definecolor{currentstroke}{rgb}{0.000000,0.000000,0.000000}%
\pgfsetstrokecolor{currentstroke}%
\pgfsetdash{}{0pt}%
\pgfpathmoveto{\pgfqpoint{0.692982in}{2.487846in}}%
\pgfpathlineto{\pgfqpoint{0.700008in}{2.489474in}}%
\pgfpathlineto{\pgfqpoint{0.692982in}{2.491414in}}%
\pgfpathlineto{\pgfqpoint{0.687688in}{2.489474in}}%
\pgfpathlineto{\pgfqpoint{0.692982in}{2.487846in}}%
\pgfusepath{stroke}%
\end{pgfscope}%
\begin{pgfscope}%
\pgfpathrectangle{\pgfqpoint{0.625000in}{0.550000in}}{\pgfqpoint{3.875000in}{3.850000in}} %
\pgfusepath{clip}%
\pgfsetbuttcap%
\pgfsetroundjoin%
\pgfsetlinewidth{0.501875pt}%
\definecolor{currentstroke}{rgb}{0.000000,0.000000,0.000000}%
\pgfsetstrokecolor{currentstroke}%
\pgfsetdash{}{0pt}%
\pgfpathmoveto{\pgfqpoint{0.663847in}{2.493023in}}%
\pgfpathlineto{\pgfqpoint{0.667296in}{2.499123in}}%
\pgfpathlineto{\pgfqpoint{0.663847in}{2.500420in}}%
\pgfpathlineto{\pgfqpoint{0.662488in}{2.499123in}}%
\pgfpathlineto{\pgfqpoint{0.663847in}{2.493023in}}%
\pgfusepath{stroke}%
\end{pgfscope}%
\begin{pgfscope}%
\pgfpathrectangle{\pgfqpoint{0.625000in}{0.550000in}}{\pgfqpoint{3.875000in}{3.850000in}} %
\pgfusepath{clip}%
\pgfsetbuttcap%
\pgfsetroundjoin%
\pgfsetlinewidth{0.501875pt}%
\definecolor{currentstroke}{rgb}{0.000000,0.000000,0.000000}%
\pgfsetstrokecolor{currentstroke}%
\pgfsetdash{}{0pt}%
\pgfpathmoveto{\pgfqpoint{0.692982in}{2.514709in}}%
\pgfpathlineto{\pgfqpoint{0.696090in}{2.518421in}}%
\pgfpathlineto{\pgfqpoint{0.692982in}{2.524824in}}%
\pgfpathlineto{\pgfqpoint{0.690618in}{2.518421in}}%
\pgfpathlineto{\pgfqpoint{0.692982in}{2.514709in}}%
\pgfusepath{stroke}%
\end{pgfscope}%
\begin{pgfscope}%
\pgfpathrectangle{\pgfqpoint{0.625000in}{0.550000in}}{\pgfqpoint{3.875000in}{3.850000in}} %
\pgfusepath{clip}%
\pgfsetbuttcap%
\pgfsetroundjoin%
\pgfsetlinewidth{0.501875pt}%
\definecolor{currentstroke}{rgb}{0.000000,0.000000,0.000000}%
\pgfsetstrokecolor{currentstroke}%
\pgfsetdash{}{0pt}%
\pgfpathmoveto{\pgfqpoint{0.673559in}{2.522290in}}%
\pgfpathlineto{\pgfqpoint{0.674262in}{2.528070in}}%
\pgfpathlineto{\pgfqpoint{0.673559in}{2.534326in}}%
\pgfpathlineto{\pgfqpoint{0.672828in}{2.528070in}}%
\pgfpathlineto{\pgfqpoint{0.673559in}{2.522290in}}%
\pgfusepath{stroke}%
\end{pgfscope}%
\begin{pgfscope}%
\pgfpathrectangle{\pgfqpoint{0.625000in}{0.550000in}}{\pgfqpoint{3.875000in}{3.850000in}} %
\pgfusepath{clip}%
\pgfsetbuttcap%
\pgfsetroundjoin%
\pgfsetlinewidth{0.501875pt}%
\definecolor{currentstroke}{rgb}{0.000000,0.000000,0.000000}%
\pgfsetstrokecolor{currentstroke}%
\pgfsetdash{}{0pt}%
\pgfpathmoveto{\pgfqpoint{0.663847in}{2.554362in}}%
\pgfpathlineto{\pgfqpoint{0.664071in}{2.557018in}}%
\pgfpathlineto{\pgfqpoint{0.663847in}{2.557314in}}%
\pgfpathlineto{\pgfqpoint{0.662853in}{2.557018in}}%
\pgfpathlineto{\pgfqpoint{0.663847in}{2.554362in}}%
\pgfusepath{stroke}%
\end{pgfscope}%
\begin{pgfscope}%
\pgfpathrectangle{\pgfqpoint{0.625000in}{0.550000in}}{\pgfqpoint{3.875000in}{3.850000in}} %
\pgfusepath{clip}%
\pgfsetbuttcap%
\pgfsetroundjoin%
\pgfsetlinewidth{0.501875pt}%
\definecolor{currentstroke}{rgb}{0.000000,0.000000,0.000000}%
\pgfsetstrokecolor{currentstroke}%
\pgfsetdash{}{0pt}%
\pgfpathmoveto{\pgfqpoint{0.644424in}{2.564898in}}%
\pgfpathlineto{\pgfqpoint{0.653220in}{2.566667in}}%
\pgfpathlineto{\pgfqpoint{0.644424in}{2.569823in}}%
\pgfpathlineto{\pgfqpoint{0.638449in}{2.566667in}}%
\pgfpathlineto{\pgfqpoint{0.644424in}{2.564898in}}%
\pgfusepath{stroke}%
\end{pgfscope}%
\begin{pgfscope}%
\pgfpathrectangle{\pgfqpoint{0.625000in}{0.550000in}}{\pgfqpoint{3.875000in}{3.850000in}} %
\pgfusepath{clip}%
\pgfsetbuttcap%
\pgfsetroundjoin%
\pgfsetlinewidth{0.501875pt}%
\definecolor{currentstroke}{rgb}{0.000000,0.000000,0.000000}%
\pgfsetstrokecolor{currentstroke}%
\pgfsetdash{}{0pt}%
\pgfpathmoveto{\pgfqpoint{0.634712in}{3.264764in}}%
\pgfpathlineto{\pgfqpoint{0.642614in}{3.271053in}}%
\pgfpathlineto{\pgfqpoint{0.634712in}{3.280079in}}%
\pgfpathlineto{\pgfqpoint{0.627683in}{3.271053in}}%
\pgfpathlineto{\pgfqpoint{0.634712in}{3.264764in}}%
\pgfusepath{stroke}%
\end{pgfscope}%
\begin{pgfscope}%
\pgfpathrectangle{\pgfqpoint{0.625000in}{0.550000in}}{\pgfqpoint{3.875000in}{3.850000in}} %
\pgfusepath{clip}%
\pgfsetbuttcap%
\pgfsetroundjoin%
\pgfsetlinewidth{0.501875pt}%
\definecolor{currentstroke}{rgb}{0.000000,0.000000,0.000000}%
\pgfsetstrokecolor{currentstroke}%
\pgfsetdash{}{0pt}%
\pgfpathmoveto{\pgfqpoint{0.634712in}{3.723469in}}%
\pgfpathlineto{\pgfqpoint{0.636567in}{3.724561in}}%
\pgfpathlineto{\pgfqpoint{0.644424in}{3.729934in}}%
\pgfpathlineto{\pgfqpoint{0.647341in}{3.734211in}}%
\pgfpathlineto{\pgfqpoint{0.648198in}{3.743860in}}%
\pgfpathlineto{\pgfqpoint{0.647494in}{3.753509in}}%
\pgfpathlineto{\pgfqpoint{0.644424in}{3.755727in}}%
\pgfpathlineto{\pgfqpoint{0.641852in}{3.753509in}}%
\pgfpathlineto{\pgfqpoint{0.642469in}{3.743860in}}%
\pgfpathlineto{\pgfqpoint{0.634712in}{3.734764in}}%
\pgfpathlineto{\pgfqpoint{0.631654in}{3.734211in}}%
\pgfpathlineto{\pgfqpoint{0.634201in}{3.724561in}}%
\pgfpathlineto{\pgfqpoint{0.634712in}{3.723469in}}%
\pgfusepath{stroke}%
\end{pgfscope}%
\begin{pgfscope}%
\pgfpathrectangle{\pgfqpoint{0.625000in}{0.550000in}}{\pgfqpoint{3.875000in}{3.850000in}} %
\pgfusepath{clip}%
\pgfsetbuttcap%
\pgfsetroundjoin%
\pgfsetlinewidth{0.501875pt}%
\definecolor{currentstroke}{rgb}{0.000000,0.000000,0.000000}%
\pgfsetstrokecolor{currentstroke}%
\pgfsetdash{}{0pt}%
\pgfpathmoveto{\pgfqpoint{0.644424in}{3.770905in}}%
\pgfpathlineto{\pgfqpoint{0.647540in}{3.772807in}}%
\pgfpathlineto{\pgfqpoint{0.644424in}{3.776352in}}%
\pgfpathlineto{\pgfqpoint{0.640176in}{3.772807in}}%
\pgfpathlineto{\pgfqpoint{0.644424in}{3.770905in}}%
\pgfusepath{stroke}%
\end{pgfscope}%
\begin{pgfscope}%
\pgfpathrectangle{\pgfqpoint{0.625000in}{0.550000in}}{\pgfqpoint{3.875000in}{3.850000in}} %
\pgfusepath{clip}%
\pgfsetbuttcap%
\pgfsetroundjoin%
\pgfsetlinewidth{0.501875pt}%
\definecolor{currentstroke}{rgb}{0.000000,0.000000,0.000000}%
\pgfsetstrokecolor{currentstroke}%
\pgfsetdash{}{0pt}%
\pgfpathmoveto{\pgfqpoint{0.654135in}{3.780803in}}%
\pgfpathlineto{\pgfqpoint{0.656239in}{3.782456in}}%
\pgfpathlineto{\pgfqpoint{0.660121in}{3.792105in}}%
\pgfpathlineto{\pgfqpoint{0.660219in}{3.801754in}}%
\pgfpathlineto{\pgfqpoint{0.656845in}{3.811404in}}%
\pgfpathlineto{\pgfqpoint{0.654135in}{3.815417in}}%
\pgfpathlineto{\pgfqpoint{0.647957in}{3.821053in}}%
\pgfpathlineto{\pgfqpoint{0.644424in}{3.823555in}}%
\pgfpathlineto{\pgfqpoint{0.634712in}{3.826539in}}%
\pgfpathlineto{\pgfqpoint{0.633404in}{3.821053in}}%
\pgfpathlineto{\pgfqpoint{0.633266in}{3.811404in}}%
\pgfpathlineto{\pgfqpoint{0.634712in}{3.808016in}}%
\pgfpathlineto{\pgfqpoint{0.643306in}{3.801754in}}%
\pgfpathlineto{\pgfqpoint{0.644424in}{3.800545in}}%
\pgfpathlineto{\pgfqpoint{0.649383in}{3.792105in}}%
\pgfpathlineto{\pgfqpoint{0.652168in}{3.782456in}}%
\pgfpathlineto{\pgfqpoint{0.654135in}{3.780803in}}%
\pgfusepath{stroke}%
\end{pgfscope}%
\begin{pgfscope}%
\pgfpathrectangle{\pgfqpoint{0.625000in}{0.550000in}}{\pgfqpoint{3.875000in}{3.850000in}} %
\pgfusepath{clip}%
\pgfsetbuttcap%
\pgfsetroundjoin%
\pgfsetlinewidth{0.501875pt}%
\definecolor{currentstroke}{rgb}{0.000000,0.000000,0.000000}%
\pgfsetstrokecolor{currentstroke}%
\pgfsetdash{}{0pt}%
\pgfpathmoveto{\pgfqpoint{0.634712in}{4.031836in}}%
\pgfpathlineto{\pgfqpoint{0.637451in}{4.033333in}}%
\pgfpathlineto{\pgfqpoint{0.635153in}{4.042982in}}%
\pgfpathlineto{\pgfqpoint{0.634712in}{4.043476in}}%
\pgfpathlineto{\pgfqpoint{0.633226in}{4.042982in}}%
\pgfpathlineto{\pgfqpoint{0.633522in}{4.033333in}}%
\pgfpathlineto{\pgfqpoint{0.634712in}{4.031836in}}%
\pgfusepath{stroke}%
\end{pgfscope}%
\begin{pgfscope}%
\pgfpathrectangle{\pgfqpoint{0.625000in}{0.550000in}}{\pgfqpoint{3.875000in}{3.850000in}} %
\pgfusepath{clip}%
\pgfsetbuttcap%
\pgfsetroundjoin%
\pgfsetlinewidth{0.501875pt}%
\definecolor{currentstroke}{rgb}{0.000000,0.000000,0.000000}%
\pgfsetstrokecolor{currentstroke}%
\pgfsetdash{}{0pt}%
\pgfpathmoveto{\pgfqpoint{0.625000in}{0.627106in}}%
\pgfpathlineto{\pgfqpoint{0.625097in}{0.627193in}}%
\pgfpathlineto{\pgfqpoint{0.625000in}{0.627278in}}%
\pgfusepath{stroke}%
\end{pgfscope}%
\begin{pgfscope}%
\pgfpathrectangle{\pgfqpoint{0.625000in}{0.550000in}}{\pgfqpoint{3.875000in}{3.850000in}} %
\pgfusepath{clip}%
\pgfsetbuttcap%
\pgfsetroundjoin%
\pgfsetlinewidth{0.501875pt}%
\definecolor{currentstroke}{rgb}{0.000000,0.000000,0.000000}%
\pgfsetstrokecolor{currentstroke}%
\pgfsetdash{}{0pt}%
\pgfpathmoveto{\pgfqpoint{0.625000in}{1.427520in}}%
\pgfpathlineto{\pgfqpoint{0.625422in}{1.428070in}}%
\pgfpathlineto{\pgfqpoint{0.625000in}{1.428621in}}%
\pgfusepath{stroke}%
\end{pgfscope}%
\begin{pgfscope}%
\pgfpathrectangle{\pgfqpoint{0.625000in}{0.550000in}}{\pgfqpoint{3.875000in}{3.850000in}} %
\pgfusepath{clip}%
\pgfsetbuttcap%
\pgfsetroundjoin%
\pgfsetlinewidth{0.501875pt}%
\definecolor{currentstroke}{rgb}{0.000000,0.000000,0.000000}%
\pgfsetstrokecolor{currentstroke}%
\pgfsetdash{}{0pt}%
\pgfpathmoveto{\pgfqpoint{0.625000in}{1.813968in}}%
\pgfpathlineto{\pgfqpoint{0.625074in}{1.814035in}}%
\pgfpathlineto{\pgfqpoint{0.625000in}{1.814102in}}%
\pgfusepath{stroke}%
\end{pgfscope}%
\begin{pgfscope}%
\pgfpathrectangle{\pgfqpoint{0.625000in}{0.550000in}}{\pgfqpoint{3.875000in}{3.850000in}} %
\pgfusepath{clip}%
\pgfsetbuttcap%
\pgfsetroundjoin%
\pgfsetlinewidth{0.501875pt}%
\definecolor{currentstroke}{rgb}{0.000000,0.000000,0.000000}%
\pgfsetstrokecolor{currentstroke}%
\pgfsetdash{}{0pt}%
\pgfpathmoveto{\pgfqpoint{0.625000in}{2.093131in}}%
\pgfpathlineto{\pgfqpoint{0.634712in}{2.089035in}}%
\pgfpathlineto{\pgfqpoint{0.644424in}{2.085685in}}%
\pgfpathlineto{\pgfqpoint{0.649276in}{2.084211in}}%
\pgfpathlineto{\pgfqpoint{0.654135in}{2.081520in}}%
\pgfpathlineto{\pgfqpoint{0.663847in}{2.077831in}}%
\pgfpathlineto{\pgfqpoint{0.673559in}{2.075799in}}%
\pgfpathlineto{\pgfqpoint{0.683271in}{2.075405in}}%
\pgfpathlineto{\pgfqpoint{0.692982in}{2.076255in}}%
\pgfpathlineto{\pgfqpoint{0.702694in}{2.078016in}}%
\pgfpathlineto{\pgfqpoint{0.712406in}{2.080507in}}%
\pgfpathlineto{\pgfqpoint{0.722118in}{2.083656in}}%
\pgfpathlineto{\pgfqpoint{0.723573in}{2.084211in}}%
\pgfpathlineto{\pgfqpoint{0.731830in}{2.087321in}}%
\pgfpathlineto{\pgfqpoint{0.741541in}{2.091639in}}%
\pgfpathlineto{\pgfqpoint{0.745908in}{2.093860in}}%
\pgfpathlineto{\pgfqpoint{0.751253in}{2.096587in}}%
\pgfpathlineto{\pgfqpoint{0.760965in}{2.102249in}}%
\pgfpathlineto{\pgfqpoint{0.762905in}{2.103509in}}%
\pgfpathlineto{\pgfqpoint{0.770677in}{2.108645in}}%
\pgfpathlineto{\pgfqpoint{0.776753in}{2.113158in}}%
\pgfpathlineto{\pgfqpoint{0.780388in}{2.115933in}}%
\pgfpathlineto{\pgfqpoint{0.788496in}{2.122807in}}%
\pgfpathlineto{\pgfqpoint{0.790100in}{2.124220in}}%
\pgfpathlineto{\pgfqpoint{0.798609in}{2.132456in}}%
\pgfpathlineto{\pgfqpoint{0.799812in}{2.133681in}}%
\pgfpathlineto{\pgfqpoint{0.807420in}{2.142105in}}%
\pgfpathlineto{\pgfqpoint{0.809524in}{2.144591in}}%
\pgfpathlineto{\pgfqpoint{0.815152in}{2.151754in}}%
\pgfpathlineto{\pgfqpoint{0.819236in}{2.157394in}}%
\pgfpathlineto{\pgfqpoint{0.821956in}{2.161404in}}%
\pgfpathlineto{\pgfqpoint{0.827932in}{2.171053in}}%
\pgfpathlineto{\pgfqpoint{0.828947in}{2.172842in}}%
\pgfpathlineto{\pgfqpoint{0.833180in}{2.180702in}}%
\pgfpathlineto{\pgfqpoint{0.837733in}{2.190351in}}%
\pgfpathlineto{\pgfqpoint{0.838659in}{2.192544in}}%
\pgfpathlineto{\pgfqpoint{0.841685in}{2.200000in}}%
\pgfpathlineto{\pgfqpoint{0.845040in}{2.209649in}}%
\pgfpathlineto{\pgfqpoint{0.847817in}{2.219298in}}%
\pgfpathlineto{\pgfqpoint{0.848371in}{2.221603in}}%
\pgfpathlineto{\pgfqpoint{0.850099in}{2.228947in}}%
\pgfpathlineto{\pgfqpoint{0.851872in}{2.238596in}}%
\pgfpathlineto{\pgfqpoint{0.853138in}{2.248246in}}%
\pgfpathlineto{\pgfqpoint{0.853914in}{2.257895in}}%
\pgfpathlineto{\pgfqpoint{0.854211in}{2.267544in}}%
\pgfpathlineto{\pgfqpoint{0.854033in}{2.277193in}}%
\pgfpathlineto{\pgfqpoint{0.853380in}{2.286842in}}%
\pgfpathlineto{\pgfqpoint{0.852243in}{2.296491in}}%
\pgfpathlineto{\pgfqpoint{0.850603in}{2.306140in}}%
\pgfpathlineto{\pgfqpoint{0.848429in}{2.315789in}}%
\pgfpathlineto{\pgfqpoint{0.848371in}{2.315987in}}%
\pgfpathlineto{\pgfqpoint{0.845865in}{2.325439in}}%
\pgfpathlineto{\pgfqpoint{0.842729in}{2.335088in}}%
\pgfpathlineto{\pgfqpoint{0.838933in}{2.344737in}}%
\pgfpathlineto{\pgfqpoint{0.838659in}{2.345312in}}%
\pgfpathlineto{\pgfqpoint{0.834734in}{2.354386in}}%
\pgfpathlineto{\pgfqpoint{0.829746in}{2.364035in}}%
\pgfpathlineto{\pgfqpoint{0.828947in}{2.365328in}}%
\pgfpathlineto{\pgfqpoint{0.824255in}{2.373684in}}%
\pgfpathlineto{\pgfqpoint{0.819236in}{2.381129in}}%
\pgfpathlineto{\pgfqpoint{0.817582in}{2.383333in}}%
\pgfpathlineto{\pgfqpoint{0.810770in}{2.392982in}}%
\pgfpathlineto{\pgfqpoint{0.809524in}{2.394389in}}%
\pgfpathlineto{\pgfqpoint{0.802824in}{2.402632in}}%
\pgfpathlineto{\pgfqpoint{0.799812in}{2.405679in}}%
\pgfpathlineto{\pgfqpoint{0.793802in}{2.412281in}}%
\pgfpathlineto{\pgfqpoint{0.790100in}{2.415597in}}%
\pgfpathlineto{\pgfqpoint{0.783800in}{2.412281in}}%
\pgfpathlineto{\pgfqpoint{0.790100in}{2.402801in}}%
\pgfpathlineto{\pgfqpoint{0.790327in}{2.402632in}}%
\pgfpathlineto{\pgfqpoint{0.797400in}{2.392982in}}%
\pgfpathlineto{\pgfqpoint{0.799812in}{2.389014in}}%
\pgfpathlineto{\pgfqpoint{0.803656in}{2.383333in}}%
\pgfpathlineto{\pgfqpoint{0.808997in}{2.373684in}}%
\pgfpathlineto{\pgfqpoint{0.809524in}{2.372518in}}%
\pgfpathlineto{\pgfqpoint{0.813808in}{2.364035in}}%
\pgfpathlineto{\pgfqpoint{0.817799in}{2.354386in}}%
\pgfpathlineto{\pgfqpoint{0.819236in}{2.350089in}}%
\pgfpathlineto{\pgfqpoint{0.821232in}{2.344737in}}%
\pgfpathlineto{\pgfqpoint{0.824091in}{2.335088in}}%
\pgfpathlineto{\pgfqpoint{0.826326in}{2.325439in}}%
\pgfpathlineto{\pgfqpoint{0.827994in}{2.315789in}}%
\pgfpathlineto{\pgfqpoint{0.828947in}{2.307647in}}%
\pgfpathlineto{\pgfqpoint{0.829143in}{2.306140in}}%
\pgfpathlineto{\pgfqpoint{0.829803in}{2.296491in}}%
\pgfpathlineto{\pgfqpoint{0.829923in}{2.286842in}}%
\pgfpathlineto{\pgfqpoint{0.829507in}{2.277193in}}%
\pgfpathlineto{\pgfqpoint{0.828947in}{2.271435in}}%
\pgfpathlineto{\pgfqpoint{0.828566in}{2.267544in}}%
\pgfpathlineto{\pgfqpoint{0.827111in}{2.257895in}}%
\pgfpathlineto{\pgfqpoint{0.825092in}{2.248246in}}%
\pgfpathlineto{\pgfqpoint{0.822481in}{2.238596in}}%
\pgfpathlineto{\pgfqpoint{0.819244in}{2.228947in}}%
\pgfpathlineto{\pgfqpoint{0.819236in}{2.228925in}}%
\pgfpathlineto{\pgfqpoint{0.815421in}{2.219298in}}%
\pgfpathlineto{\pgfqpoint{0.810882in}{2.209649in}}%
\pgfpathlineto{\pgfqpoint{0.809524in}{2.207056in}}%
\pgfpathlineto{\pgfqpoint{0.805610in}{2.200000in}}%
\pgfpathlineto{\pgfqpoint{0.799812in}{2.190798in}}%
\pgfpathlineto{\pgfqpoint{0.799510in}{2.190351in}}%
\pgfpathlineto{\pgfqpoint{0.792484in}{2.180702in}}%
\pgfpathlineto{\pgfqpoint{0.790100in}{2.177695in}}%
\pgfpathlineto{\pgfqpoint{0.784378in}{2.171053in}}%
\pgfpathlineto{\pgfqpoint{0.780388in}{2.166725in}}%
\pgfpathlineto{\pgfqpoint{0.774993in}{2.161404in}}%
\pgfpathlineto{\pgfqpoint{0.770677in}{2.157357in}}%
\pgfpathlineto{\pgfqpoint{0.764021in}{2.151754in}}%
\pgfpathlineto{\pgfqpoint{0.760965in}{2.149274in}}%
\pgfpathlineto{\pgfqpoint{0.751253in}{2.142277in}}%
\pgfpathlineto{\pgfqpoint{0.750987in}{2.142105in}}%
\pgfpathlineto{\pgfqpoint{0.741541in}{2.136129in}}%
\pgfpathlineto{\pgfqpoint{0.734808in}{2.132456in}}%
\pgfpathlineto{\pgfqpoint{0.731830in}{2.130844in}}%
\pgfpathlineto{\pgfqpoint{0.722118in}{2.126248in}}%
\pgfpathlineto{\pgfqpoint{0.713504in}{2.122807in}}%
\pgfpathlineto{\pgfqpoint{0.712406in}{2.122366in}}%
\pgfpathlineto{\pgfqpoint{0.702694in}{2.119005in}}%
\pgfpathlineto{\pgfqpoint{0.692982in}{2.116210in}}%
\pgfpathlineto{\pgfqpoint{0.683271in}{2.113827in}}%
\pgfpathlineto{\pgfqpoint{0.680419in}{2.113158in}}%
\pgfpathlineto{\pgfqpoint{0.673559in}{2.111610in}}%
\pgfpathlineto{\pgfqpoint{0.663847in}{2.109230in}}%
\pgfpathlineto{\pgfqpoint{0.654135in}{2.106037in}}%
\pgfpathlineto{\pgfqpoint{0.649344in}{2.103509in}}%
\pgfpathlineto{\pgfqpoint{0.644424in}{2.102035in}}%
\pgfpathlineto{\pgfqpoint{0.634712in}{2.098684in}}%
\pgfpathlineto{\pgfqpoint{0.625000in}{2.095188in}}%
\pgfusepath{stroke}%
\end{pgfscope}%
\begin{pgfscope}%
\pgfpathrectangle{\pgfqpoint{0.625000in}{0.550000in}}{\pgfqpoint{3.875000in}{3.850000in}} %
\pgfusepath{clip}%
\pgfsetbuttcap%
\pgfsetroundjoin%
\pgfsetlinewidth{0.501875pt}%
\definecolor{currentstroke}{rgb}{0.000000,0.000000,0.000000}%
\pgfsetstrokecolor{currentstroke}%
\pgfsetdash{}{0pt}%
\pgfpathmoveto{\pgfqpoint{0.625000in}{2.295764in}}%
\pgfpathlineto{\pgfqpoint{0.625891in}{2.296491in}}%
\pgfpathlineto{\pgfqpoint{0.625000in}{2.298712in}}%
\pgfusepath{stroke}%
\end{pgfscope}%
\begin{pgfscope}%
\pgfpathrectangle{\pgfqpoint{0.625000in}{0.550000in}}{\pgfqpoint{3.875000in}{3.850000in}} %
\pgfusepath{clip}%
\pgfsetbuttcap%
\pgfsetroundjoin%
\pgfsetlinewidth{0.501875pt}%
\definecolor{currentstroke}{rgb}{0.000000,0.000000,0.000000}%
\pgfsetstrokecolor{currentstroke}%
\pgfsetdash{}{0pt}%
\pgfpathmoveto{\pgfqpoint{0.625000in}{2.467911in}}%
\pgfpathlineto{\pgfqpoint{0.625590in}{2.470175in}}%
\pgfpathlineto{\pgfqpoint{0.625000in}{2.471296in}}%
\pgfusepath{stroke}%
\end{pgfscope}%
\begin{pgfscope}%
\pgfpathrectangle{\pgfqpoint{0.625000in}{0.550000in}}{\pgfqpoint{3.875000in}{3.850000in}} %
\pgfusepath{clip}%
\pgfsetbuttcap%
\pgfsetroundjoin%
\pgfsetlinewidth{0.501875pt}%
\definecolor{currentstroke}{rgb}{0.000000,0.000000,0.000000}%
\pgfsetstrokecolor{currentstroke}%
\pgfsetdash{}{0pt}%
\pgfpathmoveto{\pgfqpoint{0.625000in}{2.488353in}}%
\pgfpathlineto{\pgfqpoint{0.630802in}{2.489474in}}%
\pgfpathlineto{\pgfqpoint{0.625000in}{2.491739in}}%
\pgfusepath{stroke}%
\end{pgfscope}%
\begin{pgfscope}%
\pgfpathrectangle{\pgfqpoint{0.625000in}{0.550000in}}{\pgfqpoint{3.875000in}{3.850000in}} %
\pgfusepath{clip}%
\pgfsetbuttcap%
\pgfsetroundjoin%
\pgfsetlinewidth{0.501875pt}%
\definecolor{currentstroke}{rgb}{0.000000,0.000000,0.000000}%
\pgfsetstrokecolor{currentstroke}%
\pgfsetdash{}{0pt}%
\pgfpathmoveto{\pgfqpoint{0.625000in}{2.660937in}}%
\pgfpathlineto{\pgfqpoint{0.625891in}{2.663158in}}%
\pgfpathlineto{\pgfqpoint{0.625000in}{2.663885in}}%
\pgfusepath{stroke}%
\end{pgfscope}%
\begin{pgfscope}%
\pgfpathrectangle{\pgfqpoint{0.625000in}{0.550000in}}{\pgfqpoint{3.875000in}{3.850000in}} %
\pgfusepath{clip}%
\pgfsetbuttcap%
\pgfsetroundjoin%
\pgfsetlinewidth{0.501875pt}%
\definecolor{currentstroke}{rgb}{0.000000,0.000000,0.000000}%
\pgfsetstrokecolor{currentstroke}%
\pgfsetdash{}{0pt}%
\pgfpathmoveto{\pgfqpoint{0.625000in}{2.864461in}}%
\pgfpathlineto{\pgfqpoint{0.634712in}{2.860965in}}%
\pgfpathlineto{\pgfqpoint{0.644424in}{2.857614in}}%
\pgfpathlineto{\pgfqpoint{0.649344in}{2.856140in}}%
\pgfpathlineto{\pgfqpoint{0.654135in}{2.853612in}}%
\pgfpathlineto{\pgfqpoint{0.663847in}{2.850419in}}%
\pgfpathlineto{\pgfqpoint{0.673559in}{2.848039in}}%
\pgfpathlineto{\pgfqpoint{0.680419in}{2.846491in}}%
\pgfpathlineto{\pgfqpoint{0.683271in}{2.845822in}}%
\pgfpathlineto{\pgfqpoint{0.692982in}{2.843439in}}%
\pgfpathlineto{\pgfqpoint{0.702694in}{2.840644in}}%
\pgfpathlineto{\pgfqpoint{0.712406in}{2.837283in}}%
\pgfpathlineto{\pgfqpoint{0.713504in}{2.836842in}}%
\pgfpathlineto{\pgfqpoint{0.722118in}{2.833401in}}%
\pgfpathlineto{\pgfqpoint{0.731830in}{2.828805in}}%
\pgfpathlineto{\pgfqpoint{0.734808in}{2.827193in}}%
\pgfpathlineto{\pgfqpoint{0.741541in}{2.823521in}}%
\pgfpathlineto{\pgfqpoint{0.750987in}{2.817544in}}%
\pgfpathlineto{\pgfqpoint{0.751253in}{2.817372in}}%
\pgfpathlineto{\pgfqpoint{0.760965in}{2.810375in}}%
\pgfpathlineto{\pgfqpoint{0.764021in}{2.807895in}}%
\pgfpathlineto{\pgfqpoint{0.770677in}{2.802292in}}%
\pgfpathlineto{\pgfqpoint{0.774993in}{2.798246in}}%
\pgfpathlineto{\pgfqpoint{0.780388in}{2.792924in}}%
\pgfpathlineto{\pgfqpoint{0.784378in}{2.788596in}}%
\pgfpathlineto{\pgfqpoint{0.790100in}{2.781955in}}%
\pgfpathlineto{\pgfqpoint{0.792484in}{2.778947in}}%
\pgfpathlineto{\pgfqpoint{0.799510in}{2.769298in}}%
\pgfpathlineto{\pgfqpoint{0.799812in}{2.768851in}}%
\pgfpathlineto{\pgfqpoint{0.805610in}{2.759649in}}%
\pgfpathlineto{\pgfqpoint{0.809524in}{2.752593in}}%
\pgfpathlineto{\pgfqpoint{0.810882in}{2.750000in}}%
\pgfpathlineto{\pgfqpoint{0.815421in}{2.740351in}}%
\pgfpathlineto{\pgfqpoint{0.819236in}{2.730724in}}%
\pgfpathlineto{\pgfqpoint{0.819244in}{2.730702in}}%
\pgfpathlineto{\pgfqpoint{0.822481in}{2.721053in}}%
\pgfpathlineto{\pgfqpoint{0.825092in}{2.711404in}}%
\pgfpathlineto{\pgfqpoint{0.827111in}{2.701754in}}%
\pgfpathlineto{\pgfqpoint{0.828566in}{2.692105in}}%
\pgfpathlineto{\pgfqpoint{0.828947in}{2.688214in}}%
\pgfpathlineto{\pgfqpoint{0.829507in}{2.682456in}}%
\pgfpathlineto{\pgfqpoint{0.829923in}{2.672807in}}%
\pgfpathlineto{\pgfqpoint{0.829803in}{2.663158in}}%
\pgfpathlineto{\pgfqpoint{0.829143in}{2.653509in}}%
\pgfpathlineto{\pgfqpoint{0.828947in}{2.652002in}}%
\pgfpathlineto{\pgfqpoint{0.827994in}{2.643860in}}%
\pgfpathlineto{\pgfqpoint{0.826326in}{2.634211in}}%
\pgfpathlineto{\pgfqpoint{0.824091in}{2.624561in}}%
\pgfpathlineto{\pgfqpoint{0.821232in}{2.614912in}}%
\pgfpathlineto{\pgfqpoint{0.819236in}{2.609560in}}%
\pgfpathlineto{\pgfqpoint{0.817799in}{2.605263in}}%
\pgfpathlineto{\pgfqpoint{0.813808in}{2.595614in}}%
\pgfpathlineto{\pgfqpoint{0.809524in}{2.587131in}}%
\pgfpathlineto{\pgfqpoint{0.808997in}{2.585965in}}%
\pgfpathlineto{\pgfqpoint{0.803656in}{2.576316in}}%
\pgfpathlineto{\pgfqpoint{0.799812in}{2.570636in}}%
\pgfpathlineto{\pgfqpoint{0.797400in}{2.566667in}}%
\pgfpathlineto{\pgfqpoint{0.790327in}{2.557018in}}%
\pgfpathlineto{\pgfqpoint{0.790100in}{2.556848in}}%
\pgfpathlineto{\pgfqpoint{0.783800in}{2.547368in}}%
\pgfpathlineto{\pgfqpoint{0.790100in}{2.544052in}}%
\pgfpathlineto{\pgfqpoint{0.793802in}{2.547368in}}%
\pgfpathlineto{\pgfqpoint{0.799812in}{2.553970in}}%
\pgfpathlineto{\pgfqpoint{0.802824in}{2.557018in}}%
\pgfpathlineto{\pgfqpoint{0.809524in}{2.565260in}}%
\pgfpathlineto{\pgfqpoint{0.810770in}{2.566667in}}%
\pgfpathlineto{\pgfqpoint{0.817582in}{2.576316in}}%
\pgfpathlineto{\pgfqpoint{0.819236in}{2.578520in}}%
\pgfpathlineto{\pgfqpoint{0.824255in}{2.585965in}}%
\pgfpathlineto{\pgfqpoint{0.828947in}{2.594321in}}%
\pgfpathlineto{\pgfqpoint{0.829746in}{2.595614in}}%
\pgfpathlineto{\pgfqpoint{0.834734in}{2.605263in}}%
\pgfpathlineto{\pgfqpoint{0.838659in}{2.614337in}}%
\pgfpathlineto{\pgfqpoint{0.838933in}{2.614912in}}%
\pgfpathlineto{\pgfqpoint{0.842729in}{2.624561in}}%
\pgfpathlineto{\pgfqpoint{0.845865in}{2.634211in}}%
\pgfpathlineto{\pgfqpoint{0.848371in}{2.643662in}}%
\pgfpathlineto{\pgfqpoint{0.848429in}{2.643860in}}%
\pgfpathlineto{\pgfqpoint{0.850603in}{2.653509in}}%
\pgfpathlineto{\pgfqpoint{0.852243in}{2.663158in}}%
\pgfpathlineto{\pgfqpoint{0.853380in}{2.672807in}}%
\pgfpathlineto{\pgfqpoint{0.854033in}{2.682456in}}%
\pgfpathlineto{\pgfqpoint{0.854211in}{2.692105in}}%
\pgfpathlineto{\pgfqpoint{0.853914in}{2.701754in}}%
\pgfpathlineto{\pgfqpoint{0.853138in}{2.711404in}}%
\pgfpathlineto{\pgfqpoint{0.851872in}{2.721053in}}%
\pgfpathlineto{\pgfqpoint{0.850099in}{2.730702in}}%
\pgfpathlineto{\pgfqpoint{0.848371in}{2.738046in}}%
\pgfpathlineto{\pgfqpoint{0.847817in}{2.740351in}}%
\pgfpathlineto{\pgfqpoint{0.845040in}{2.750000in}}%
\pgfpathlineto{\pgfqpoint{0.841685in}{2.759649in}}%
\pgfpathlineto{\pgfqpoint{0.838659in}{2.767105in}}%
\pgfpathlineto{\pgfqpoint{0.837733in}{2.769298in}}%
\pgfpathlineto{\pgfqpoint{0.833180in}{2.778947in}}%
\pgfpathlineto{\pgfqpoint{0.828947in}{2.786807in}}%
\pgfpathlineto{\pgfqpoint{0.827932in}{2.788596in}}%
\pgfpathlineto{\pgfqpoint{0.821956in}{2.798246in}}%
\pgfpathlineto{\pgfqpoint{0.819236in}{2.802255in}}%
\pgfpathlineto{\pgfqpoint{0.815152in}{2.807895in}}%
\pgfpathlineto{\pgfqpoint{0.809524in}{2.815058in}}%
\pgfpathlineto{\pgfqpoint{0.807420in}{2.817544in}}%
\pgfpathlineto{\pgfqpoint{0.799812in}{2.825968in}}%
\pgfpathlineto{\pgfqpoint{0.798609in}{2.827193in}}%
\pgfpathlineto{\pgfqpoint{0.790100in}{2.835429in}}%
\pgfpathlineto{\pgfqpoint{0.788496in}{2.836842in}}%
\pgfpathlineto{\pgfqpoint{0.780388in}{2.843717in}}%
\pgfpathlineto{\pgfqpoint{0.776753in}{2.846491in}}%
\pgfpathlineto{\pgfqpoint{0.770677in}{2.851004in}}%
\pgfpathlineto{\pgfqpoint{0.762905in}{2.856140in}}%
\pgfpathlineto{\pgfqpoint{0.760965in}{2.857400in}}%
\pgfpathlineto{\pgfqpoint{0.751253in}{2.863062in}}%
\pgfpathlineto{\pgfqpoint{0.745908in}{2.865789in}}%
\pgfpathlineto{\pgfqpoint{0.741541in}{2.868010in}}%
\pgfpathlineto{\pgfqpoint{0.731830in}{2.872328in}}%
\pgfpathlineto{\pgfqpoint{0.723573in}{2.875439in}}%
\pgfpathlineto{\pgfqpoint{0.722118in}{2.875994in}}%
\pgfpathlineto{\pgfqpoint{0.712406in}{2.879142in}}%
\pgfpathlineto{\pgfqpoint{0.702694in}{2.881633in}}%
\pgfpathlineto{\pgfqpoint{0.692982in}{2.883394in}}%
\pgfpathlineto{\pgfqpoint{0.683271in}{2.884244in}}%
\pgfpathlineto{\pgfqpoint{0.673559in}{2.883850in}}%
\pgfpathlineto{\pgfqpoint{0.663847in}{2.881818in}}%
\pgfpathlineto{\pgfqpoint{0.654135in}{2.878129in}}%
\pgfpathlineto{\pgfqpoint{0.649276in}{2.875439in}}%
\pgfpathlineto{\pgfqpoint{0.644424in}{2.873964in}}%
\pgfpathlineto{\pgfqpoint{0.634712in}{2.870614in}}%
\pgfpathlineto{\pgfqpoint{0.625000in}{2.866518in}}%
\pgfusepath{stroke}%
\end{pgfscope}%
\begin{pgfscope}%
\pgfpathrectangle{\pgfqpoint{0.625000in}{0.550000in}}{\pgfqpoint{3.875000in}{3.850000in}} %
\pgfusepath{clip}%
\pgfsetbuttcap%
\pgfsetroundjoin%
\pgfsetlinewidth{0.501875pt}%
\definecolor{currentstroke}{rgb}{0.000000,0.000000,0.000000}%
\pgfsetstrokecolor{currentstroke}%
\pgfsetdash{}{0pt}%
\pgfpathmoveto{\pgfqpoint{0.625000in}{3.145547in}}%
\pgfpathlineto{\pgfqpoint{0.625074in}{3.145614in}}%
\pgfpathlineto{\pgfqpoint{0.625000in}{3.145681in}}%
\pgfusepath{stroke}%
\end{pgfscope}%
\begin{pgfscope}%
\pgfpathrectangle{\pgfqpoint{0.625000in}{0.550000in}}{\pgfqpoint{3.875000in}{3.850000in}} %
\pgfusepath{clip}%
\pgfsetbuttcap%
\pgfsetroundjoin%
\pgfsetlinewidth{0.501875pt}%
\definecolor{currentstroke}{rgb}{0.000000,0.000000,0.000000}%
\pgfsetstrokecolor{currentstroke}%
\pgfsetdash{}{0pt}%
\pgfpathmoveto{\pgfqpoint{0.625000in}{3.531028in}}%
\pgfpathlineto{\pgfqpoint{0.625422in}{3.531579in}}%
\pgfpathlineto{\pgfqpoint{0.625000in}{3.532130in}}%
\pgfusepath{stroke}%
\end{pgfscope}%
\begin{pgfscope}%
\pgfpathrectangle{\pgfqpoint{0.625000in}{0.550000in}}{\pgfqpoint{3.875000in}{3.850000in}} %
\pgfusepath{clip}%
\pgfsetbuttcap%
\pgfsetroundjoin%
\pgfsetlinewidth{0.501875pt}%
\definecolor{currentstroke}{rgb}{0.000000,0.000000,0.000000}%
\pgfsetstrokecolor{currentstroke}%
\pgfsetdash{}{0pt}%
\pgfpathmoveto{\pgfqpoint{0.644424in}{1.144945in}}%
\pgfpathlineto{\pgfqpoint{0.647925in}{1.148246in}}%
\pgfpathlineto{\pgfqpoint{0.644424in}{1.151405in}}%
\pgfpathlineto{\pgfqpoint{0.638118in}{1.148246in}}%
\pgfpathlineto{\pgfqpoint{0.644424in}{1.144945in}}%
\pgfusepath{stroke}%
\end{pgfscope}%
\begin{pgfscope}%
\pgfpathrectangle{\pgfqpoint{0.625000in}{0.550000in}}{\pgfqpoint{3.875000in}{3.850000in}} %
\pgfusepath{clip}%
\pgfsetbuttcap%
\pgfsetroundjoin%
\pgfsetlinewidth{0.501875pt}%
\definecolor{currentstroke}{rgb}{0.000000,0.000000,0.000000}%
\pgfsetstrokecolor{currentstroke}%
\pgfsetdash{}{0pt}%
\pgfpathmoveto{\pgfqpoint{0.654135in}{1.158961in}}%
\pgfpathlineto{\pgfqpoint{0.655387in}{1.167544in}}%
\pgfpathlineto{\pgfqpoint{0.654135in}{1.170784in}}%
\pgfpathlineto{\pgfqpoint{0.653141in}{1.167544in}}%
\pgfpathlineto{\pgfqpoint{0.654135in}{1.158961in}}%
\pgfusepath{stroke}%
\end{pgfscope}%
\begin{pgfscope}%
\pgfpathrectangle{\pgfqpoint{0.625000in}{0.550000in}}{\pgfqpoint{3.875000in}{3.850000in}} %
\pgfusepath{clip}%
\pgfsetbuttcap%
\pgfsetroundjoin%
\pgfsetlinewidth{0.501875pt}%
\definecolor{currentstroke}{rgb}{0.000000,0.000000,0.000000}%
\pgfsetstrokecolor{currentstroke}%
\pgfsetdash{}{0pt}%
\pgfpathmoveto{\pgfqpoint{0.644424in}{1.186548in}}%
\pgfpathlineto{\pgfqpoint{0.644682in}{1.186842in}}%
\pgfpathlineto{\pgfqpoint{0.644424in}{1.187000in}}%
\pgfpathlineto{\pgfqpoint{0.644072in}{1.186842in}}%
\pgfpathlineto{\pgfqpoint{0.644424in}{1.186548in}}%
\pgfusepath{stroke}%
\end{pgfscope}%
\begin{pgfscope}%
\pgfpathrectangle{\pgfqpoint{0.625000in}{0.550000in}}{\pgfqpoint{3.875000in}{3.850000in}} %
\pgfusepath{clip}%
\pgfsetbuttcap%
\pgfsetroundjoin%
\pgfsetlinewidth{0.501875pt}%
\definecolor{currentstroke}{rgb}{0.000000,0.000000,0.000000}%
\pgfsetstrokecolor{currentstroke}%
\pgfsetdash{}{0pt}%
\pgfpathmoveto{\pgfqpoint{0.644424in}{1.205595in}}%
\pgfpathlineto{\pgfqpoint{0.645178in}{1.206140in}}%
\pgfpathlineto{\pgfqpoint{0.644424in}{1.214520in}}%
\pgfpathlineto{\pgfqpoint{0.643792in}{1.206140in}}%
\pgfpathlineto{\pgfqpoint{0.644424in}{1.205595in}}%
\pgfusepath{stroke}%
\end{pgfscope}%
\begin{pgfscope}%
\pgfpathrectangle{\pgfqpoint{0.625000in}{0.550000in}}{\pgfqpoint{3.875000in}{3.850000in}} %
\pgfusepath{clip}%
\pgfsetbuttcap%
\pgfsetroundjoin%
\pgfsetlinewidth{0.501875pt}%
\definecolor{currentstroke}{rgb}{0.000000,0.000000,0.000000}%
\pgfsetstrokecolor{currentstroke}%
\pgfsetdash{}{0pt}%
\pgfpathmoveto{\pgfqpoint{0.634712in}{1.686462in}}%
\pgfpathlineto{\pgfqpoint{0.636581in}{1.688596in}}%
\pgfpathlineto{\pgfqpoint{0.634712in}{1.690084in}}%
\pgfpathlineto{\pgfqpoint{0.633049in}{1.688596in}}%
\pgfpathlineto{\pgfqpoint{0.634712in}{1.686462in}}%
\pgfusepath{stroke}%
\end{pgfscope}%
\begin{pgfscope}%
\pgfpathrectangle{\pgfqpoint{0.625000in}{0.550000in}}{\pgfqpoint{3.875000in}{3.850000in}} %
\pgfusepath{clip}%
\pgfsetbuttcap%
\pgfsetroundjoin%
\pgfsetlinewidth{0.501875pt}%
\definecolor{currentstroke}{rgb}{0.000000,0.000000,0.000000}%
\pgfsetstrokecolor{currentstroke}%
\pgfsetdash{}{0pt}%
\pgfpathmoveto{\pgfqpoint{0.634712in}{1.916130in}}%
\pgfpathlineto{\pgfqpoint{0.636065in}{1.920175in}}%
\pgfpathlineto{\pgfqpoint{0.634712in}{1.920790in}}%
\pgfpathlineto{\pgfqpoint{0.634329in}{1.920175in}}%
\pgfpathlineto{\pgfqpoint{0.634712in}{1.916130in}}%
\pgfusepath{stroke}%
\end{pgfscope}%
\begin{pgfscope}%
\pgfpathrectangle{\pgfqpoint{0.625000in}{0.550000in}}{\pgfqpoint{3.875000in}{3.850000in}} %
\pgfusepath{clip}%
\pgfsetbuttcap%
\pgfsetroundjoin%
\pgfsetlinewidth{0.501875pt}%
\definecolor{currentstroke}{rgb}{0.000000,0.000000,0.000000}%
\pgfsetstrokecolor{currentstroke}%
\pgfsetdash{}{0pt}%
\pgfpathmoveto{\pgfqpoint{0.634712in}{2.026529in}}%
\pgfpathlineto{\pgfqpoint{0.644055in}{2.035965in}}%
\pgfpathlineto{\pgfqpoint{0.634712in}{2.044424in}}%
\pgfpathlineto{\pgfqpoint{0.632510in}{2.035965in}}%
\pgfpathlineto{\pgfqpoint{0.634712in}{2.026529in}}%
\pgfusepath{stroke}%
\end{pgfscope}%
\begin{pgfscope}%
\pgfpathrectangle{\pgfqpoint{0.625000in}{0.550000in}}{\pgfqpoint{3.875000in}{3.850000in}} %
\pgfusepath{clip}%
\pgfsetbuttcap%
\pgfsetroundjoin%
\pgfsetlinewidth{0.501875pt}%
\definecolor{currentstroke}{rgb}{0.000000,0.000000,0.000000}%
\pgfsetstrokecolor{currentstroke}%
\pgfsetdash{}{0pt}%
\pgfpathmoveto{\pgfqpoint{0.634712in}{2.132271in}}%
\pgfpathlineto{\pgfqpoint{0.635170in}{2.132456in}}%
\pgfpathlineto{\pgfqpoint{0.634712in}{2.133987in}}%
\pgfpathlineto{\pgfqpoint{0.634665in}{2.132456in}}%
\pgfpathlineto{\pgfqpoint{0.634712in}{2.132271in}}%
\pgfusepath{stroke}%
\end{pgfscope}%
\begin{pgfscope}%
\pgfpathrectangle{\pgfqpoint{0.625000in}{0.550000in}}{\pgfqpoint{3.875000in}{3.850000in}} %
\pgfusepath{clip}%
\pgfsetbuttcap%
\pgfsetroundjoin%
\pgfsetlinewidth{0.501875pt}%
\definecolor{currentstroke}{rgb}{0.000000,0.000000,0.000000}%
\pgfsetstrokecolor{currentstroke}%
\pgfsetdash{}{0pt}%
\pgfpathmoveto{\pgfqpoint{0.644424in}{2.333619in}}%
\pgfpathlineto{\pgfqpoint{0.654135in}{2.332573in}}%
\pgfpathlineto{\pgfqpoint{0.659872in}{2.335088in}}%
\pgfpathlineto{\pgfqpoint{0.654135in}{2.337649in}}%
\pgfpathlineto{\pgfqpoint{0.644424in}{2.335782in}}%
\pgfpathlineto{\pgfqpoint{0.639795in}{2.335088in}}%
\pgfpathlineto{\pgfqpoint{0.644424in}{2.333619in}}%
\pgfusepath{stroke}%
\end{pgfscope}%
\begin{pgfscope}%
\pgfpathrectangle{\pgfqpoint{0.625000in}{0.550000in}}{\pgfqpoint{3.875000in}{3.850000in}} %
\pgfusepath{clip}%
\pgfsetbuttcap%
\pgfsetroundjoin%
\pgfsetlinewidth{0.501875pt}%
\definecolor{currentstroke}{rgb}{0.000000,0.000000,0.000000}%
\pgfsetstrokecolor{currentstroke}%
\pgfsetdash{}{0pt}%
\pgfpathmoveto{\pgfqpoint{0.673559in}{2.342787in}}%
\pgfpathlineto{\pgfqpoint{0.675948in}{2.344737in}}%
\pgfpathlineto{\pgfqpoint{0.673559in}{2.347280in}}%
\pgfpathlineto{\pgfqpoint{0.668644in}{2.344737in}}%
\pgfpathlineto{\pgfqpoint{0.673559in}{2.342787in}}%
\pgfusepath{stroke}%
\end{pgfscope}%
\begin{pgfscope}%
\pgfpathrectangle{\pgfqpoint{0.625000in}{0.550000in}}{\pgfqpoint{3.875000in}{3.850000in}} %
\pgfusepath{clip}%
\pgfsetbuttcap%
\pgfsetroundjoin%
\pgfsetlinewidth{0.501875pt}%
\definecolor{currentstroke}{rgb}{0.000000,0.000000,0.000000}%
\pgfsetstrokecolor{currentstroke}%
\pgfsetdash{}{0pt}%
\pgfpathmoveto{\pgfqpoint{0.683271in}{2.351205in}}%
\pgfpathlineto{\pgfqpoint{0.686090in}{2.354386in}}%
\pgfpathlineto{\pgfqpoint{0.683271in}{2.357810in}}%
\pgfpathlineto{\pgfqpoint{0.679867in}{2.354386in}}%
\pgfpathlineto{\pgfqpoint{0.683271in}{2.351205in}}%
\pgfusepath{stroke}%
\end{pgfscope}%
\begin{pgfscope}%
\pgfpathrectangle{\pgfqpoint{0.625000in}{0.550000in}}{\pgfqpoint{3.875000in}{3.850000in}} %
\pgfusepath{clip}%
\pgfsetbuttcap%
\pgfsetroundjoin%
\pgfsetlinewidth{0.501875pt}%
\definecolor{currentstroke}{rgb}{0.000000,0.000000,0.000000}%
\pgfsetstrokecolor{currentstroke}%
\pgfsetdash{}{0pt}%
\pgfpathmoveto{\pgfqpoint{0.692982in}{2.363645in}}%
\pgfpathlineto{\pgfqpoint{0.693225in}{2.364035in}}%
\pgfpathlineto{\pgfqpoint{0.692982in}{2.366851in}}%
\pgfpathlineto{\pgfqpoint{0.692462in}{2.364035in}}%
\pgfpathlineto{\pgfqpoint{0.692982in}{2.363645in}}%
\pgfusepath{stroke}%
\end{pgfscope}%
\begin{pgfscope}%
\pgfpathrectangle{\pgfqpoint{0.625000in}{0.550000in}}{\pgfqpoint{3.875000in}{3.850000in}} %
\pgfusepath{clip}%
\pgfsetbuttcap%
\pgfsetroundjoin%
\pgfsetlinewidth{0.501875pt}%
\definecolor{currentstroke}{rgb}{0.000000,0.000000,0.000000}%
\pgfsetstrokecolor{currentstroke}%
\pgfsetdash{}{0pt}%
\pgfpathmoveto{\pgfqpoint{0.702694in}{2.388138in}}%
\pgfpathlineto{\pgfqpoint{0.703664in}{2.392982in}}%
\pgfpathlineto{\pgfqpoint{0.704394in}{2.402632in}}%
\pgfpathlineto{\pgfqpoint{0.703809in}{2.412281in}}%
\pgfpathlineto{\pgfqpoint{0.702694in}{2.417417in}}%
\pgfpathlineto{\pgfqpoint{0.701666in}{2.412281in}}%
\pgfpathlineto{\pgfqpoint{0.701193in}{2.402632in}}%
\pgfpathlineto{\pgfqpoint{0.701621in}{2.392982in}}%
\pgfpathlineto{\pgfqpoint{0.702694in}{2.388138in}}%
\pgfusepath{stroke}%
\end{pgfscope}%
\begin{pgfscope}%
\pgfpathrectangle{\pgfqpoint{0.625000in}{0.550000in}}{\pgfqpoint{3.875000in}{3.850000in}} %
\pgfusepath{clip}%
\pgfsetbuttcap%
\pgfsetroundjoin%
\pgfsetlinewidth{0.501875pt}%
\definecolor{currentstroke}{rgb}{0.000000,0.000000,0.000000}%
\pgfsetstrokecolor{currentstroke}%
\pgfsetdash{}{0pt}%
\pgfpathmoveto{\pgfqpoint{0.780388in}{2.415867in}}%
\pgfpathlineto{\pgfqpoint{0.783535in}{2.421930in}}%
\pgfpathlineto{\pgfqpoint{0.780388in}{2.424386in}}%
\pgfpathlineto{\pgfqpoint{0.775005in}{2.421930in}}%
\pgfpathlineto{\pgfqpoint{0.780388in}{2.415867in}}%
\pgfusepath{stroke}%
\end{pgfscope}%
\begin{pgfscope}%
\pgfpathrectangle{\pgfqpoint{0.625000in}{0.550000in}}{\pgfqpoint{3.875000in}{3.850000in}} %
\pgfusepath{clip}%
\pgfsetbuttcap%
\pgfsetroundjoin%
\pgfsetlinewidth{0.501875pt}%
\definecolor{currentstroke}{rgb}{0.000000,0.000000,0.000000}%
\pgfsetstrokecolor{currentstroke}%
\pgfsetdash{}{0pt}%
\pgfpathmoveto{\pgfqpoint{0.770677in}{2.428686in}}%
\pgfpathlineto{\pgfqpoint{0.771699in}{2.431579in}}%
\pgfpathlineto{\pgfqpoint{0.770677in}{2.432258in}}%
\pgfpathlineto{\pgfqpoint{0.766363in}{2.431579in}}%
\pgfpathlineto{\pgfqpoint{0.770677in}{2.428686in}}%
\pgfusepath{stroke}%
\end{pgfscope}%
\begin{pgfscope}%
\pgfpathrectangle{\pgfqpoint{0.625000in}{0.550000in}}{\pgfqpoint{3.875000in}{3.850000in}} %
\pgfusepath{clip}%
\pgfsetbuttcap%
\pgfsetroundjoin%
\pgfsetlinewidth{0.501875pt}%
\definecolor{currentstroke}{rgb}{0.000000,0.000000,0.000000}%
\pgfsetstrokecolor{currentstroke}%
\pgfsetdash{}{0pt}%
\pgfpathmoveto{\pgfqpoint{0.692982in}{2.439228in}}%
\pgfpathlineto{\pgfqpoint{0.693953in}{2.441228in}}%
\pgfpathlineto{\pgfqpoint{0.692982in}{2.442388in}}%
\pgfpathlineto{\pgfqpoint{0.692244in}{2.441228in}}%
\pgfpathlineto{\pgfqpoint{0.692982in}{2.439228in}}%
\pgfusepath{stroke}%
\end{pgfscope}%
\begin{pgfscope}%
\pgfpathrectangle{\pgfqpoint{0.625000in}{0.550000in}}{\pgfqpoint{3.875000in}{3.850000in}} %
\pgfusepath{clip}%
\pgfsetbuttcap%
\pgfsetroundjoin%
\pgfsetlinewidth{0.501875pt}%
\definecolor{currentstroke}{rgb}{0.000000,0.000000,0.000000}%
\pgfsetstrokecolor{currentstroke}%
\pgfsetdash{}{0pt}%
\pgfpathmoveto{\pgfqpoint{0.751253in}{2.439809in}}%
\pgfpathlineto{\pgfqpoint{0.755426in}{2.441228in}}%
\pgfpathlineto{\pgfqpoint{0.751253in}{2.442974in}}%
\pgfpathlineto{\pgfqpoint{0.749553in}{2.441228in}}%
\pgfpathlineto{\pgfqpoint{0.751253in}{2.439809in}}%
\pgfusepath{stroke}%
\end{pgfscope}%
\begin{pgfscope}%
\pgfpathrectangle{\pgfqpoint{0.625000in}{0.550000in}}{\pgfqpoint{3.875000in}{3.850000in}} %
\pgfusepath{clip}%
\pgfsetbuttcap%
\pgfsetroundjoin%
\pgfsetlinewidth{0.501875pt}%
\definecolor{currentstroke}{rgb}{0.000000,0.000000,0.000000}%
\pgfsetstrokecolor{currentstroke}%
\pgfsetdash{}{0pt}%
\pgfpathmoveto{\pgfqpoint{0.741541in}{2.450609in}}%
\pgfpathlineto{\pgfqpoint{0.741823in}{2.450877in}}%
\pgfpathlineto{\pgfqpoint{0.741541in}{2.451006in}}%
\pgfpathlineto{\pgfqpoint{0.740825in}{2.450877in}}%
\pgfpathlineto{\pgfqpoint{0.741541in}{2.450609in}}%
\pgfusepath{stroke}%
\end{pgfscope}%
\begin{pgfscope}%
\pgfpathrectangle{\pgfqpoint{0.625000in}{0.550000in}}{\pgfqpoint{3.875000in}{3.850000in}} %
\pgfusepath{clip}%
\pgfsetbuttcap%
\pgfsetroundjoin%
\pgfsetlinewidth{0.501875pt}%
\definecolor{currentstroke}{rgb}{0.000000,0.000000,0.000000}%
\pgfsetstrokecolor{currentstroke}%
\pgfsetdash{}{0pt}%
\pgfpathmoveto{\pgfqpoint{0.692982in}{2.469706in}}%
\pgfpathlineto{\pgfqpoint{0.694683in}{2.470175in}}%
\pgfpathlineto{\pgfqpoint{0.692982in}{2.470569in}}%
\pgfpathlineto{\pgfqpoint{0.691701in}{2.470175in}}%
\pgfpathlineto{\pgfqpoint{0.692982in}{2.469706in}}%
\pgfusepath{stroke}%
\end{pgfscope}%
\begin{pgfscope}%
\pgfpathrectangle{\pgfqpoint{0.625000in}{0.550000in}}{\pgfqpoint{3.875000in}{3.850000in}} %
\pgfusepath{clip}%
\pgfsetbuttcap%
\pgfsetroundjoin%
\pgfsetlinewidth{0.501875pt}%
\definecolor{currentstroke}{rgb}{0.000000,0.000000,0.000000}%
\pgfsetstrokecolor{currentstroke}%
\pgfsetdash{}{0pt}%
\pgfpathmoveto{\pgfqpoint{0.644424in}{2.489286in}}%
\pgfpathlineto{\pgfqpoint{0.647083in}{2.489474in}}%
\pgfpathlineto{\pgfqpoint{0.644424in}{2.489608in}}%
\pgfpathlineto{\pgfqpoint{0.642642in}{2.489474in}}%
\pgfpathlineto{\pgfqpoint{0.644424in}{2.489286in}}%
\pgfusepath{stroke}%
\end{pgfscope}%
\begin{pgfscope}%
\pgfpathrectangle{\pgfqpoint{0.625000in}{0.550000in}}{\pgfqpoint{3.875000in}{3.850000in}} %
\pgfusepath{clip}%
\pgfsetbuttcap%
\pgfsetroundjoin%
\pgfsetlinewidth{0.501875pt}%
\definecolor{currentstroke}{rgb}{0.000000,0.000000,0.000000}%
\pgfsetstrokecolor{currentstroke}%
\pgfsetdash{}{0pt}%
\pgfpathmoveto{\pgfqpoint{0.692982in}{2.489080in}}%
\pgfpathlineto{\pgfqpoint{0.694683in}{2.489474in}}%
\pgfpathlineto{\pgfqpoint{0.692982in}{2.489943in}}%
\pgfpathlineto{\pgfqpoint{0.691701in}{2.489474in}}%
\pgfpathlineto{\pgfqpoint{0.692982in}{2.489080in}}%
\pgfusepath{stroke}%
\end{pgfscope}%
\begin{pgfscope}%
\pgfpathrectangle{\pgfqpoint{0.625000in}{0.550000in}}{\pgfqpoint{3.875000in}{3.850000in}} %
\pgfusepath{clip}%
\pgfsetbuttcap%
\pgfsetroundjoin%
\pgfsetlinewidth{0.501875pt}%
\definecolor{currentstroke}{rgb}{0.000000,0.000000,0.000000}%
\pgfsetstrokecolor{currentstroke}%
\pgfsetdash{}{0pt}%
\pgfpathmoveto{\pgfqpoint{0.741541in}{2.508643in}}%
\pgfpathlineto{\pgfqpoint{0.741823in}{2.508772in}}%
\pgfpathlineto{\pgfqpoint{0.741541in}{2.509041in}}%
\pgfpathlineto{\pgfqpoint{0.740825in}{2.508772in}}%
\pgfpathlineto{\pgfqpoint{0.741541in}{2.508643in}}%
\pgfusepath{stroke}%
\end{pgfscope}%
\begin{pgfscope}%
\pgfpathrectangle{\pgfqpoint{0.625000in}{0.550000in}}{\pgfqpoint{3.875000in}{3.850000in}} %
\pgfusepath{clip}%
\pgfsetbuttcap%
\pgfsetroundjoin%
\pgfsetlinewidth{0.501875pt}%
\definecolor{currentstroke}{rgb}{0.000000,0.000000,0.000000}%
\pgfsetstrokecolor{currentstroke}%
\pgfsetdash{}{0pt}%
\pgfpathmoveto{\pgfqpoint{0.692982in}{2.517261in}}%
\pgfpathlineto{\pgfqpoint{0.693953in}{2.518421in}}%
\pgfpathlineto{\pgfqpoint{0.692982in}{2.520422in}}%
\pgfpathlineto{\pgfqpoint{0.692244in}{2.518421in}}%
\pgfpathlineto{\pgfqpoint{0.692982in}{2.517261in}}%
\pgfusepath{stroke}%
\end{pgfscope}%
\begin{pgfscope}%
\pgfpathrectangle{\pgfqpoint{0.625000in}{0.550000in}}{\pgfqpoint{3.875000in}{3.850000in}} %
\pgfusepath{clip}%
\pgfsetbuttcap%
\pgfsetroundjoin%
\pgfsetlinewidth{0.501875pt}%
\definecolor{currentstroke}{rgb}{0.000000,0.000000,0.000000}%
\pgfsetstrokecolor{currentstroke}%
\pgfsetdash{}{0pt}%
\pgfpathmoveto{\pgfqpoint{0.751253in}{2.516675in}}%
\pgfpathlineto{\pgfqpoint{0.755426in}{2.518421in}}%
\pgfpathlineto{\pgfqpoint{0.751253in}{2.519840in}}%
\pgfpathlineto{\pgfqpoint{0.749553in}{2.518421in}}%
\pgfpathlineto{\pgfqpoint{0.751253in}{2.516675in}}%
\pgfusepath{stroke}%
\end{pgfscope}%
\begin{pgfscope}%
\pgfpathrectangle{\pgfqpoint{0.625000in}{0.550000in}}{\pgfqpoint{3.875000in}{3.850000in}} %
\pgfusepath{clip}%
\pgfsetbuttcap%
\pgfsetroundjoin%
\pgfsetlinewidth{0.501875pt}%
\definecolor{currentstroke}{rgb}{0.000000,0.000000,0.000000}%
\pgfsetstrokecolor{currentstroke}%
\pgfsetdash{}{0pt}%
\pgfpathmoveto{\pgfqpoint{0.770677in}{2.527391in}}%
\pgfpathlineto{\pgfqpoint{0.771699in}{2.528070in}}%
\pgfpathlineto{\pgfqpoint{0.770677in}{2.530964in}}%
\pgfpathlineto{\pgfqpoint{0.766363in}{2.528070in}}%
\pgfpathlineto{\pgfqpoint{0.770677in}{2.527391in}}%
\pgfusepath{stroke}%
\end{pgfscope}%
\begin{pgfscope}%
\pgfpathrectangle{\pgfqpoint{0.625000in}{0.550000in}}{\pgfqpoint{3.875000in}{3.850000in}} %
\pgfusepath{clip}%
\pgfsetbuttcap%
\pgfsetroundjoin%
\pgfsetlinewidth{0.501875pt}%
\definecolor{currentstroke}{rgb}{0.000000,0.000000,0.000000}%
\pgfsetstrokecolor{currentstroke}%
\pgfsetdash{}{0pt}%
\pgfpathmoveto{\pgfqpoint{0.780388in}{2.535263in}}%
\pgfpathlineto{\pgfqpoint{0.783535in}{2.537719in}}%
\pgfpathlineto{\pgfqpoint{0.780388in}{2.543782in}}%
\pgfpathlineto{\pgfqpoint{0.775005in}{2.537719in}}%
\pgfpathlineto{\pgfqpoint{0.780388in}{2.535263in}}%
\pgfusepath{stroke}%
\end{pgfscope}%
\begin{pgfscope}%
\pgfpathrectangle{\pgfqpoint{0.625000in}{0.550000in}}{\pgfqpoint{3.875000in}{3.850000in}} %
\pgfusepath{clip}%
\pgfsetbuttcap%
\pgfsetroundjoin%
\pgfsetlinewidth{0.501875pt}%
\definecolor{currentstroke}{rgb}{0.000000,0.000000,0.000000}%
\pgfsetstrokecolor{currentstroke}%
\pgfsetdash{}{0pt}%
\pgfpathmoveto{\pgfqpoint{0.702694in}{2.542232in}}%
\pgfpathlineto{\pgfqpoint{0.703809in}{2.547368in}}%
\pgfpathlineto{\pgfqpoint{0.704394in}{2.557018in}}%
\pgfpathlineto{\pgfqpoint{0.703664in}{2.566667in}}%
\pgfpathlineto{\pgfqpoint{0.702694in}{2.571512in}}%
\pgfpathlineto{\pgfqpoint{0.701621in}{2.566667in}}%
\pgfpathlineto{\pgfqpoint{0.701193in}{2.557018in}}%
\pgfpathlineto{\pgfqpoint{0.701666in}{2.547368in}}%
\pgfpathlineto{\pgfqpoint{0.702694in}{2.542232in}}%
\pgfusepath{stroke}%
\end{pgfscope}%
\begin{pgfscope}%
\pgfpathrectangle{\pgfqpoint{0.625000in}{0.550000in}}{\pgfqpoint{3.875000in}{3.850000in}} %
\pgfusepath{clip}%
\pgfsetbuttcap%
\pgfsetroundjoin%
\pgfsetlinewidth{0.501875pt}%
\definecolor{currentstroke}{rgb}{0.000000,0.000000,0.000000}%
\pgfsetstrokecolor{currentstroke}%
\pgfsetdash{}{0pt}%
\pgfpathmoveto{\pgfqpoint{0.692982in}{2.592798in}}%
\pgfpathlineto{\pgfqpoint{0.693225in}{2.595614in}}%
\pgfpathlineto{\pgfqpoint{0.692982in}{2.596004in}}%
\pgfpathlineto{\pgfqpoint{0.692462in}{2.595614in}}%
\pgfpathlineto{\pgfqpoint{0.692982in}{2.592798in}}%
\pgfusepath{stroke}%
\end{pgfscope}%
\begin{pgfscope}%
\pgfpathrectangle{\pgfqpoint{0.625000in}{0.550000in}}{\pgfqpoint{3.875000in}{3.850000in}} %
\pgfusepath{clip}%
\pgfsetbuttcap%
\pgfsetroundjoin%
\pgfsetlinewidth{0.501875pt}%
\definecolor{currentstroke}{rgb}{0.000000,0.000000,0.000000}%
\pgfsetstrokecolor{currentstroke}%
\pgfsetdash{}{0pt}%
\pgfpathmoveto{\pgfqpoint{0.683271in}{2.601839in}}%
\pgfpathlineto{\pgfqpoint{0.686090in}{2.605263in}}%
\pgfpathlineto{\pgfqpoint{0.683271in}{2.608445in}}%
\pgfpathlineto{\pgfqpoint{0.679867in}{2.605263in}}%
\pgfpathlineto{\pgfqpoint{0.683271in}{2.601839in}}%
\pgfusepath{stroke}%
\end{pgfscope}%
\begin{pgfscope}%
\pgfpathrectangle{\pgfqpoint{0.625000in}{0.550000in}}{\pgfqpoint{3.875000in}{3.850000in}} %
\pgfusepath{clip}%
\pgfsetbuttcap%
\pgfsetroundjoin%
\pgfsetlinewidth{0.501875pt}%
\definecolor{currentstroke}{rgb}{0.000000,0.000000,0.000000}%
\pgfsetstrokecolor{currentstroke}%
\pgfsetdash{}{0pt}%
\pgfpathmoveto{\pgfqpoint{0.673559in}{2.612369in}}%
\pgfpathlineto{\pgfqpoint{0.675948in}{2.614912in}}%
\pgfpathlineto{\pgfqpoint{0.673559in}{2.616862in}}%
\pgfpathlineto{\pgfqpoint{0.668644in}{2.614912in}}%
\pgfpathlineto{\pgfqpoint{0.673559in}{2.612369in}}%
\pgfusepath{stroke}%
\end{pgfscope}%
\begin{pgfscope}%
\pgfpathrectangle{\pgfqpoint{0.625000in}{0.550000in}}{\pgfqpoint{3.875000in}{3.850000in}} %
\pgfusepath{clip}%
\pgfsetbuttcap%
\pgfsetroundjoin%
\pgfsetlinewidth{0.501875pt}%
\definecolor{currentstroke}{rgb}{0.000000,0.000000,0.000000}%
\pgfsetstrokecolor{currentstroke}%
\pgfsetdash{}{0pt}%
\pgfpathmoveto{\pgfqpoint{0.644424in}{2.623867in}}%
\pgfpathlineto{\pgfqpoint{0.654135in}{2.622000in}}%
\pgfpathlineto{\pgfqpoint{0.659872in}{2.624561in}}%
\pgfpathlineto{\pgfqpoint{0.654135in}{2.627076in}}%
\pgfpathlineto{\pgfqpoint{0.644424in}{2.626030in}}%
\pgfpathlineto{\pgfqpoint{0.639795in}{2.624561in}}%
\pgfpathlineto{\pgfqpoint{0.644424in}{2.623867in}}%
\pgfusepath{stroke}%
\end{pgfscope}%
\begin{pgfscope}%
\pgfpathrectangle{\pgfqpoint{0.625000in}{0.550000in}}{\pgfqpoint{3.875000in}{3.850000in}} %
\pgfusepath{clip}%
\pgfsetbuttcap%
\pgfsetroundjoin%
\pgfsetlinewidth{0.501875pt}%
\definecolor{currentstroke}{rgb}{0.000000,0.000000,0.000000}%
\pgfsetstrokecolor{currentstroke}%
\pgfsetdash{}{0pt}%
\pgfpathmoveto{\pgfqpoint{0.634712in}{2.825662in}}%
\pgfpathlineto{\pgfqpoint{0.635170in}{2.827193in}}%
\pgfpathlineto{\pgfqpoint{0.634712in}{2.827378in}}%
\pgfpathlineto{\pgfqpoint{0.634665in}{2.827193in}}%
\pgfpathlineto{\pgfqpoint{0.634712in}{2.825662in}}%
\pgfusepath{stroke}%
\end{pgfscope}%
\begin{pgfscope}%
\pgfpathrectangle{\pgfqpoint{0.625000in}{0.550000in}}{\pgfqpoint{3.875000in}{3.850000in}} %
\pgfusepath{clip}%
\pgfsetbuttcap%
\pgfsetroundjoin%
\pgfsetlinewidth{0.501875pt}%
\definecolor{currentstroke}{rgb}{0.000000,0.000000,0.000000}%
\pgfsetstrokecolor{currentstroke}%
\pgfsetdash{}{0pt}%
\pgfpathmoveto{\pgfqpoint{0.634712in}{2.915225in}}%
\pgfpathlineto{\pgfqpoint{0.644055in}{2.923684in}}%
\pgfpathlineto{\pgfqpoint{0.634712in}{2.933120in}}%
\pgfpathlineto{\pgfqpoint{0.632510in}{2.923684in}}%
\pgfpathlineto{\pgfqpoint{0.634712in}{2.915225in}}%
\pgfusepath{stroke}%
\end{pgfscope}%
\begin{pgfscope}%
\pgfpathrectangle{\pgfqpoint{0.625000in}{0.550000in}}{\pgfqpoint{3.875000in}{3.850000in}} %
\pgfusepath{clip}%
\pgfsetbuttcap%
\pgfsetroundjoin%
\pgfsetlinewidth{0.501875pt}%
\definecolor{currentstroke}{rgb}{0.000000,0.000000,0.000000}%
\pgfsetstrokecolor{currentstroke}%
\pgfsetdash{}{0pt}%
\pgfpathmoveto{\pgfqpoint{0.634712in}{3.038859in}}%
\pgfpathlineto{\pgfqpoint{0.636065in}{3.039474in}}%
\pgfpathlineto{\pgfqpoint{0.634712in}{3.043519in}}%
\pgfpathlineto{\pgfqpoint{0.634329in}{3.039474in}}%
\pgfpathlineto{\pgfqpoint{0.634712in}{3.038859in}}%
\pgfusepath{stroke}%
\end{pgfscope}%
\begin{pgfscope}%
\pgfpathrectangle{\pgfqpoint{0.625000in}{0.550000in}}{\pgfqpoint{3.875000in}{3.850000in}} %
\pgfusepath{clip}%
\pgfsetbuttcap%
\pgfsetroundjoin%
\pgfsetlinewidth{0.501875pt}%
\definecolor{currentstroke}{rgb}{0.000000,0.000000,0.000000}%
\pgfsetstrokecolor{currentstroke}%
\pgfsetdash{}{0pt}%
\pgfpathmoveto{\pgfqpoint{0.634712in}{3.269565in}}%
\pgfpathlineto{\pgfqpoint{0.636581in}{3.271053in}}%
\pgfpathlineto{\pgfqpoint{0.634712in}{3.273187in}}%
\pgfpathlineto{\pgfqpoint{0.633049in}{3.271053in}}%
\pgfpathlineto{\pgfqpoint{0.634712in}{3.269565in}}%
\pgfusepath{stroke}%
\end{pgfscope}%
\begin{pgfscope}%
\pgfpathrectangle{\pgfqpoint{0.625000in}{0.550000in}}{\pgfqpoint{3.875000in}{3.850000in}} %
\pgfusepath{clip}%
\pgfsetbuttcap%
\pgfsetroundjoin%
\pgfsetlinewidth{0.501875pt}%
\definecolor{currentstroke}{rgb}{0.000000,0.000000,0.000000}%
\pgfsetstrokecolor{currentstroke}%
\pgfsetdash{}{0pt}%
\pgfpathmoveto{\pgfqpoint{0.644424in}{3.745129in}}%
\pgfpathlineto{\pgfqpoint{0.645178in}{3.753509in}}%
\pgfpathlineto{\pgfqpoint{0.644424in}{3.754054in}}%
\pgfpathlineto{\pgfqpoint{0.643792in}{3.753509in}}%
\pgfpathlineto{\pgfqpoint{0.644424in}{3.745129in}}%
\pgfusepath{stroke}%
\end{pgfscope}%
\begin{pgfscope}%
\pgfpathrectangle{\pgfqpoint{0.625000in}{0.550000in}}{\pgfqpoint{3.875000in}{3.850000in}} %
\pgfusepath{clip}%
\pgfsetbuttcap%
\pgfsetroundjoin%
\pgfsetlinewidth{0.501875pt}%
\definecolor{currentstroke}{rgb}{0.000000,0.000000,0.000000}%
\pgfsetstrokecolor{currentstroke}%
\pgfsetdash{}{0pt}%
\pgfpathmoveto{\pgfqpoint{0.644424in}{3.772649in}}%
\pgfpathlineto{\pgfqpoint{0.644682in}{3.772807in}}%
\pgfpathlineto{\pgfqpoint{0.644424in}{3.773101in}}%
\pgfpathlineto{\pgfqpoint{0.644072in}{3.772807in}}%
\pgfpathlineto{\pgfqpoint{0.644424in}{3.772649in}}%
\pgfusepath{stroke}%
\end{pgfscope}%
\begin{pgfscope}%
\pgfpathrectangle{\pgfqpoint{0.625000in}{0.550000in}}{\pgfqpoint{3.875000in}{3.850000in}} %
\pgfusepath{clip}%
\pgfsetbuttcap%
\pgfsetroundjoin%
\pgfsetlinewidth{0.501875pt}%
\definecolor{currentstroke}{rgb}{0.000000,0.000000,0.000000}%
\pgfsetstrokecolor{currentstroke}%
\pgfsetdash{}{0pt}%
\pgfpathmoveto{\pgfqpoint{0.654135in}{3.788865in}}%
\pgfpathlineto{\pgfqpoint{0.655387in}{3.792105in}}%
\pgfpathlineto{\pgfqpoint{0.654135in}{3.800688in}}%
\pgfpathlineto{\pgfqpoint{0.653141in}{3.792105in}}%
\pgfpathlineto{\pgfqpoint{0.654135in}{3.788865in}}%
\pgfusepath{stroke}%
\end{pgfscope}%
\begin{pgfscope}%
\pgfpathrectangle{\pgfqpoint{0.625000in}{0.550000in}}{\pgfqpoint{3.875000in}{3.850000in}} %
\pgfusepath{clip}%
\pgfsetbuttcap%
\pgfsetroundjoin%
\pgfsetlinewidth{0.501875pt}%
\definecolor{currentstroke}{rgb}{0.000000,0.000000,0.000000}%
\pgfsetstrokecolor{currentstroke}%
\pgfsetdash{}{0pt}%
\pgfpathmoveto{\pgfqpoint{0.644424in}{3.808244in}}%
\pgfpathlineto{\pgfqpoint{0.647925in}{3.811404in}}%
\pgfpathlineto{\pgfqpoint{0.644424in}{3.814704in}}%
\pgfpathlineto{\pgfqpoint{0.638118in}{3.811404in}}%
\pgfpathlineto{\pgfqpoint{0.644424in}{3.808244in}}%
\pgfusepath{stroke}%
\end{pgfscope}%
\begin{pgfscope}%
\pgfsetrectcap%
\pgfsetmiterjoin%
\pgfsetlinewidth{0.803000pt}%
\definecolor{currentstroke}{rgb}{0.000000,0.000000,0.000000}%
\pgfsetstrokecolor{currentstroke}%
\pgfsetdash{}{0pt}%
\pgfpathmoveto{\pgfqpoint{0.625000in}{0.550000in}}%
\pgfpathlineto{\pgfqpoint{0.625000in}{4.400000in}}%
\pgfusepath{stroke}%
\end{pgfscope}%
\begin{pgfscope}%
\pgfsetrectcap%
\pgfsetmiterjoin%
\pgfsetlinewidth{0.803000pt}%
\definecolor{currentstroke}{rgb}{0.000000,0.000000,0.000000}%
\pgfsetstrokecolor{currentstroke}%
\pgfsetdash{}{0pt}%
\pgfpathmoveto{\pgfqpoint{4.500000in}{0.550000in}}%
\pgfpathlineto{\pgfqpoint{4.500000in}{4.400000in}}%
\pgfusepath{stroke}%
\end{pgfscope}%
\begin{pgfscope}%
\pgfsetrectcap%
\pgfsetmiterjoin%
\pgfsetlinewidth{0.803000pt}%
\definecolor{currentstroke}{rgb}{0.000000,0.000000,0.000000}%
\pgfsetstrokecolor{currentstroke}%
\pgfsetdash{}{0pt}%
\pgfpathmoveto{\pgfqpoint{0.625000in}{0.550000in}}%
\pgfpathlineto{\pgfqpoint{4.500000in}{0.550000in}}%
\pgfusepath{stroke}%
\end{pgfscope}%
\begin{pgfscope}%
\pgfsetrectcap%
\pgfsetmiterjoin%
\pgfsetlinewidth{0.803000pt}%
\definecolor{currentstroke}{rgb}{0.000000,0.000000,0.000000}%
\pgfsetstrokecolor{currentstroke}%
\pgfsetdash{}{0pt}%
\pgfpathmoveto{\pgfqpoint{0.625000in}{4.400000in}}%
\pgfpathlineto{\pgfqpoint{4.500000in}{4.400000in}}%
\pgfusepath{stroke}%
\end{pgfscope}%
\end{pgfpicture}%
\makeatother%
\endgroup%


\cmnt{didascalia immagine}

The issue is that $\T(\tau)$ coincides with the theta function computed on the real axis, outside of its domain of convergence. We argue $\T(t)$ can be constructed as the limit $\lim_{\epsilon\rightarrow 0} \vartheta(0;\tau + i\epsilon)$, where the limit is taken distributionally; that is to say the distribution is specified by the following action on a test function $\varphi(t)$:

\begin{equation}
    \intR dt \, \T(t) \varphi(t)  := \lim_{\epsilon\rightarrow 0} \intR dt \, \vartheta(0; \tau + i \epsilon) \varphi(t)
\end{equation}

We will prove there is convergence in this sense in section \ref{sec:convergence}.

$\vartheta(0;\tau)$ has remarkable transformation properties which carry over to $\T(\tau)$. These can be summarized in the following two identities that are of particular interest for the study of $\T(\tau)$:

\begin{align}
    \vartheta(0; \tau + 2) = \vartheta(0; \tau)\,, \label{t2map}\\
    \vartheta(0; -\tfrac{1}{\tau}) = \sqrt{-i\tau} \, \vartheta(0; \tau)\,. \label{smap}
\end{align}

The first is evident from the derived expression. That the function is periodic of period $2$ in $\tau$ means that the propagator at time $t^* = \frac{1}{\pi}$, or recovering units $t^* = 4\pi \frac{mR^2}{\hbar}$, is equal to the propagator at time $0$. In other words, any quantum state on the circle repeats every $t^*$. This fact is less remarkable if one recalls that all energy eigenvalues on the circle, forming a complete basis, evolve in time with the simple phase factor $e^{int/t^*}$, with $n\in \mathbb{N}$, so that all wavefunctions must also share the same periodicity in time. Still, it means a quantum particle placed exactly at one point on a circle (that is, in a position eigenstate) will return to the original position with probability $1$ every $t*$.

\newcommand{\modg}{PSL(2,\mathbb{Z})}

The second relation is however especially important in combination with the first. The two transformations in terms of $\tau$ belong to the modular group $\modg$, which is composed of all maps of the form

\begin{equation}
    \tau \rightarrow \frac{a \tau + b}{c\tau + d}\,, \quad a,b,c,d \in \mathbb{Z}\,,\; ad - bc = 1\,.
\end{equation}

These are all diffeomorphisms from the upper half-plane to itself. They can also be extended to act on the real axis, provided the latter is compactified with a point at infinity as $\hat{\mathbb{R}} = \mathbb{R} \cup \{\infty\}$. The modular group is also realizable as a matrix group: if a modular transformation is represented by the unitary integer matrix

\begin{equation}
    \mqty( a & b \\ c & d)
\end{equation}

then composition of modular maps is seen to coincide with matrix multiplication; this translates into a homomorphism from the modular group to $SL(2,\mathbb{Z})$. Since opposite matrices $A$ and $-A$ yield the same modular transformation, this can be turned into an isomorphism by quotienting the sign ambiguity: $SL(2,\mathbb{Z})/\mathbb{Z}_2 = PSL(2,\mathbb{Z})$ is therefore the modular group.

$\modg$ has an obvious but powerful property of mapping rationals to rationals (or better, extended rationals $\hat{\mathbb{Q}} = \mathbb{Q} \cup \{\infty\}$), and moreover this action is transitive, in the sense that given rationals $r$ and $s$ there exists a modular map sending $r\mapsto s$. The modular group is generated by the two transformations:

\begin{align}
   && T : \tau \rightarrow \tau + 1\,, \quad& T=\mqty(1 & 1 \\ 0 & 1) && \\
   && S : \tau \rightarrow -1/\tau\,, \quad& S=\mqty(0 & -1 \\ 1 & 0) &&
\end{align}

The Jacobi theta function $\vartheta(z,\tau)$ has well-defined transformation properties under $T$ and $S$, implying the same holds for general modular transformations. However, the transformation under $T$ involves a shift in the variable $z$, which is not especially useful for our purposes. The transformations that do keep $z$ constant are generated by maps \eqref{t2map} and \eqref{smap}:

\begin{align}
   && T^2 : \tau \rightarrow \tau + 2\,, \quad& T^2=\mqty(1 & 2 \\ 0 & 1) && \\
   && S : \tau \rightarrow -1/\tau\,, \quad& S=\mqty(0 & -1 \\ 1 & 0) &&
\end{align}

\newcommand{\emodg}{\Lambda}

These do generate a proper subgroup of $PSL(2,\mathbb{Z})$, as we will show now, which we call $\emodg$. Define the notion of \emph{parity} of a reduced fraction $p/q$ as the parity of $pq$. (The parity of $\infty \in \hat{\mathbb{Q}}$ is taken to be the same as $0$, so even). It's clear this parity is invariant under action of $\emodg$; moreover since at least two rationals of opposite parity exist (e.g. $0$ and $1$), transitivity implies $\emodg$ and $PSL(2,\mathbb{Z})$ cannot coincide.


\immagine{modgroup.pdf_tex}{20cm}{A fundamental domain $D$ for $\modg$ in the upper half plane (an ideal triangle with angles $0\deg$, $30 \deg$, $30 \deg$, and a few of its images under elements of the modular group. The quadrilateral $D \cup TD$ is a fundamental domain for $\emodg$.}{modular}

What is more useful is that there are \emph{only} two orbits, in the sense that any rational $r$ can be mapped through an element of $\emodg$ to either $0$ or $1$ depending on its parity. We consider the following explicit algorithm.

Taken a rational $r = \pm \frac{p}{q}$ in lowest terms, $r\neq 0, 1$,

\begin{itemize}
    \item if $r < -1$ or $r > 1$, apply relevant power of $T^2$ to place it in $[-1,1]$
    \item apply $S$
\end{itemize}

The above procedure has the property of always decreasing the magnitude of either $p$ or $q$. Since these are positive integers, iteration of the procedure has to reach $0$ or $1$ in a finite number of steps. A couple examples:

\newcommand{\sapp}{\,\;\ensuremath{\xrightarrow{\mathmakebox[2em]{S}}}\;\,}
%\newcommand{\tappz}{\ensuremath{\xrightarrow{T^2}}}
\newcommand{\tapp}[1]{\,\;\xrightarrow{\mathmakebox[2em]{(T^2)^{#1}}}\;\,}

\begin{equation}
    \frac{13}{4} \tapp{-2} - \frac{3}{4} \sapp \frac{4}{3} \tapp{-1} - \frac{2}{3} \sapp \frac{3}{2} \tapp{-1} -\frac{1}{2} \sapp 2 \tapp{-1} 0
\end{equation}

\begin{equation}
    \frac{7}{9} \sapp - \frac{9}{7} \tapp{1} \frac{5}{7} \sapp -\frac{7}{5} \tapp{1} \frac{3}{5} \sapp - \frac{5}{3} \tapp{1} \frac{1}{3} \sapp -3 \tapp{2} 1
\end{equation}

Returning to our physical object of interest, the amplitude, we see that properties \eqref{t2map} and \eqref{smap} relate the behaviour of $\T$ at a rational time\footnote{We are, of course, here referring to the normalized time $\tau$ for simplicity.} with the behaviour at time $0$ or $1$ depending on the parity. Therefore $\T$ has a sort of self-similar structure, and any qualitative features present at the points $0$ and $1$ will be repeated infinitely in two dense sets. For example, we anticipate (though this will be studied more precisely) that the amplitude has a $\tau^{-1/2}$ singularity for $\tau$ close to zero, since for very small time the propagator on a circle is expected to reduce to that of the free particle. If so, then there must also be a similar singularity at all even rationals; therefore $\T$ has a dense set of singularities.


\subsection{Even rationals}

It is easy to show $\T$ ``should'' have a $\sim \tau^{-1/2}$ singularity near $\tau = 0$ \cite{boxpdf}. Indeed, applying  \eqref{smap}:

\begin{equation}
    \vartheta(\tau) = \sqrt{\frac{i}{\tau}} \sumZ e^{-i\pi n^2 \frac{1}{\tau}}\,.
\end{equation}

Naively, all terms in the sum except $n=0$ are oscillatory and supposedly cancel for small times, and $\vartheta(\tau) \sim \sqrt{\frac{i}{\tau}}$. This corresponds to physical intuition: the propagator should reduce to the free particle propagator for very short times, in which the particle has not yet "probed" the global structure of the circle. If this is true, however, then similar singularities should be repeated at any even rational point. Thus $\T$ would have infinite singularities, and thus be unbounded, in any given interval. This makes it impossible for it to be ever asymptotic to $\tau^{-1/2}$ in the first place.

The asymptotic behaviour must not of course be interpreted as direct but in a distributional sense. In particular, a more sensible definition is that for a family of Schwartz functions $f_\sigma(\tau) = f(\tau/\sigma)$ the evaluation of $\vartheta$ on $f_\sigma$ is asymptotic to that of $f_\sigma$ on the tempered distribution $A t^{-1/2}$ for some constant $A$. In integration against a Schwartz function we expect the secundary singularities at each even rational to matter less and less as we approach the main singularity at $\tau = 0$. We now test explicitly $\vartheta$ against a family of shrinking Gaussians and verify this intuition.

Consider the normalized Gaussian $g_\sigma(\tau) = \frac{1}{\sigma \sqrt{2\pi}} \exp(-\frac{\tau^2}{2\sigma^2})$. The integral of $g_\sigma(\tau)$ against $A \tau^{-1/2}$ is

\begin{equation}
    \bra{\tau^{-1/2}}\ket{g_\sigma} =  A \intR d\tau \, g_\sigma(\tau) \tau^{-1/2} = \frac{2A}{\sigma\sqrt{2\pi}} 2^{1/4} \sqrt\sigma \int_0^\infty dx e^{-x} x^{-3/4} = \frac{A\Gamma(\frac{1}{4})}{2^{1/4}\sqrt{\pi\sigma}}\,.
\end{equation}

On the other hand, if $g_\sigma(\tau)$ is integrated against $\vartheta$, we obtain

\begin{equation}
    \bra{\vartheta}\ket{g_\sigma} = \sumZ \int d\tau \frac{1}{\sigma \sqrt{2\pi}} e^{-\frac{\tau^2}{2\sigma^2} + i\pi\tau n^2}  = \sumZ \exp(-\pi^2 \sigma^2 n^4 / 2)\,; 
\end{equation}

only the asymptotic behaviour for small $\sigma$ is relevant. Defined $N^4 := 2 \pi^{-2} \sigma^{-2}$, the summand is slowly-varying for large $N$. Thus we can approximate as an integral:

\begin{equation}
    = N \sumZ \frac{1}{N} \exp(-\frac{n^4}{N^4}) \sim N \intR d\xi \, e^{-\xi^4} = 2 N \, \Gamma(\tfrac{5}{4}) = \frac{2^{-3/4}}{\sqrt{\pi\sigma}} \Gamma(\tfrac{1}{4})\,.
\end{equation}

It is therefore seen%\footnote{The doubt might arise that $\vartheta(\tau)$ also has a component with support in $\{0\}$ that is invisible to the family of gaussian (in the sense that its integral against $g_\sigma$ is independent on $\sigma$). However, this component must be a linear combination of $\delta$ derivatives; the even derivatives do yield positive powers of $\sigma$ and are to be excluded,  }
\ that the action of $\vartheta$ on a Schwartz function shrinking around $\tau=0$ is asymptotic to that of the distribution $\sqrt{\frac{i}{\tau}}$.

Now that the existence of this singularity has been ascertained, let us verify a reduced copy of it exists at any even rational. Imagine the even rational $a$ gets mapped to the even rational $b$ by one step of the iterative procedure described in section \ref{sec:modular}. Assume it is known that in the vicinity of $b$ there is an inverse square-root singularity:

\begin{equation}
    \vartheta(b + \delta\tau') \sim K \delta \tau^{-1/2}
\end{equation}

Now, if $a$ and $b$ are related by a simple translation $(T^2)^n$, then it's obvious this singularity is copied with no change in the constant $K$ by the periodicity property \eqref{t2map}. If instead they are related by the map $S$, and $b = -1/a$, then

\begin{equation}
    \vartheta(a + \delta\tau) = \sqrt\frac{i}{a} \vartheta\left(-\frac{1}{a+\delta\tau}\right) \sim \sqrt\frac{i}{a} \vartheta(b + \frac{\delta \tau}{a^2}) = \sqrt{ia} K \delta\tau^{-1/2}\,,
\end{equation}

so that $a$ inherits the singularity of $b$ but suppressed by a factor of $\sqrt{a}$. By simple induction, a singularity appears at any even rational $\pm p/q$, and the constant of proportionality $K_{p/q}$ is equal to the square root of the product of all rational encountered in the procedure towards $\tau = 0$ at each application of $S$; it's clear this is simply $|K_{p/q}| = 1/\sqrt{q}$. So, in the vicinity of $p/q$ the distribution $\vartheta$ behaves as

\begin{equation}
    \vartheta(\tau) \sim \frac{1}{\sqrt{p-q\tau}}
\end{equation}

\cmnt{è possibile anche fare un calcolo della fase delle singolarità? Credo di sì}

\subsection{Odd rationals}

The point $\tau=1$ is equally peculiar. We argue $\vartheta$ vanishes there along with all of its distributional derivatives $\dv[k]{t} \vartheta(\tau)$. Again, the statement as it stands is non-sensical unless it is reformulated in distributional language because of the dense set of singularities. To express it we introduce again a family of shrinking Schwartz functions $f_\sigma(\tau) = f((\tau-1)/\sigma)$, this time centered around $1$. We claim that, for all $k \in \mathbb{N}$,

\begin{equation}
    \lim_{\sigma \rightarrow 0} \bra{ \dv[k]{t}\vartheta}\ket{ f_\sigma } = 0
\end{equation}

As before, it suffices to test against shrinking normalized Gaussians $g_\sigma(\tau) = \frac{1}{\sigma\sqrt{2\pi}} \exp( - \frac{\tau^2}{2\sigma^2} )$. We find

\begin{equation}
    \bra{\dv[k]{t}\vartheta}\ket{g_\sigma} = \sumZ (i\pi n^2)^k \frac{1}{\sigma\sqrt{2\pi}} \intR d\tau e^{- \frac{(\tau-1)^2}{2\sigma^2} + i\pi\tau n^2} = \sumZ (i\pi n^2)^k \exp( - \frac{\pi^2 \sigma^2 n^4}{2} + i\pi n^2)
\end{equation}

\begin{equation}\label{alternatingser}
    = (i\pi)^k \sumZ n^{2k} (-1)^n e^{-\pi^2 \sigma^2 n^4/2} =: (i\pi)^k S_k(\sigma)\,.\quad\quad(0^0 = 1)
\end{equation}

The series \eqref{alternatingser} is absolutely convergent for any positive $\sigma$ and natural $k$. Defined again $N^4 := 2/(\pi^2 \sigma^2)$, we intend to show  $\lim_{N \rightarrow \infty} S_k = 0$. First we recognize that $S_k$ it is a residue sum for the poles of the meromorphic function

\begin{equation}
    F_{k,N}(z) = \frac{\pi}{\sin{\pi z}} z^{2k} \exp(-\frac{z^4}{N^4})\,,
\end{equation}

%A heuristic reasoning that points to $\lim_{\sigma \rightarrow 0} S_k(\sigma) = 0$ is as follows. Rewrite the alternating series by arranging terms in pairs:

%\begin{equation}
%    S_k(\sigma) = \sum_m (2m)^k e^{-a^2 (2m)^4}  - (2m-1)^{k-1} e^{-a^2 (2m-1)^4}
%\end{equation}

which indeed has poles at all $n \neq 0$ with residue $(-1)^n n^{2k} e^{-n^4/N^4}$, and another pole at $0$ with residue $1$ for $k=0$ only, so that there is match with $S_k$ for any $k$. By the residue theorem, a partial sum for $S_k$ is $1/(2\pi i)$ the contour integral of $F$ around a loop circling the relevant poles; in the limit the entire sum is equal to the integral along a pair of lines:

\begin{equation}
    2\pi i S_k = -\int_{-\infty}^\infty F_{k,N}(\rho + it) d\rho + \int_\infty^\infty F_{k,N}(\rho-it)d\rho\,;
\end{equation}

And this limit is sensible since $F(z)$ is exponentially decaying for $\abs{z} \rightarrow \infty$ as long as $\arg{z} \neq (m+\frac{1}{2}) \frac{\pi}{2}$. Using $F(-z) = - F(z)$ we reduce the integral to the half-line:

\begin{equation}
    - \pi i S_k = \int_{it}^{it + \infty} \frac{\pi}{\sin{\pi z}} z^{2k} \exp(-\frac{z^4}{N^4})\,dz\,,
\end{equation}

Then, with the substitution $z = N \xi$:

\begin{equation}
    = N^{1+2k} \int_{it/N}^{it/N + \infty} \frac{\pi}{\sin{\pi N \xi}} \xi^{2k} \exp(-\xi^4) d\xi\,.
\end{equation}

Note $t$ was arbitrary, so we can choose to fix a constant $it/N =: K$. In addition, for large enough $N$, $\abs{\sin(\pi N \xi)} = \abs{\sin(\pi N (\Re \xi + iK))} \geq C e^{\pi N K}$ for some constant $C$. Thus the absolute value of the integral is bounded:

\begin{equation}
    \pi \abs{S_k} \leq N^{1+2k} \frac{\pi}{C} e^{-\pi N K} \int_0^{\infty} \abs{\xi}^{2k} \exp(-\Re{\xi^{4}}) d\Re\xi\,,
\end{equation}

and the last integral is a finite constant depending only on $K$. We can finally send $N\rightarrow \infty$ while keeping $K$ constant. Therefore $S_k \rightarrow 0$ for $N\rightarrow \infty$, for all $k$.

Having ascertained the existence of the "flat point" of $\vartheta$ in $\tau = 1$, it is now trivial to verify this structure is copied at all odd rationals through the action of the group $\emodg$.

\subsection{Distributional convergence}\label{sec:convergence}

We are now set to prove that there is convergence in the sense of tempered distributions of the series

\begin{equation}
    \vartheta(\tau) = \sumZ e^{i\pi \tau n^2}\,.
\end{equation}

By this, we mean that defined $\vartheta_N(\tau)$ as the partial sum with $|n| < N$, given any Schwartz function $f(\tau)$ the evaluation of $\vartheta_N$ over $f$ converges to a finite value for $N\rightarrow\infty$. The partial sum evaluations however are

\begin{equation}
    \intR d\tau \, \vartheta_N(\tau) f(\tau) = \sum_{|n|<N} \intR d\tau\, e^{i\pi\tau n^2} f(\tau) = \sum_{|n|<N} \hat f(-\pi n^2)\,,
\end{equation}

having recognized the Fourier transform of $f(\tau)$. We recall that the Fourier transform of a Schwartz function is itself Schwartz. This means $\norm\big{\hat f}_{1,0} = \sup_\tau \abs\big{\tau \hat f(\tau)}$ is finite, and thus $\abs\big{\hat f(\tau)} \leq C/\abs{\tau}$ for some $C>0$, so that the above series is absolutely convergent. Thus, $\T(\tau)$ is a well-defined tempered distribution.

Equivalently, we have just proven that the Fourier transform of the $\T(\tau)$ distribution, which is the ``quadratic $\delta$ comb'':

\begin{equation}
    \hat \T(\omega) = \sumZ \delta(\omega + \pi n^2)
\end{equation}

is a well-defined tempered distribution.



\cmnt{Segue dimostrazione alternativa con l'integrale, magari spostare in una sezione apposita con grafico di $B$}

Consider the following series for real $\tau$:

\begin{equation}
    B(\tau) = \frac{1}{i\pi}\sumZ \frac{e^{i\pi\tau n^2} }{n^2}
\end{equation}

The series is absoutely convergent and limited as a function of $\tau$, as $\sum_{n=0}^{\infty} \frac{1}{n^2} = \frac{\pi^2}{6}$. Therefore, $B(\tau)$ is a well-defined function in $\locint$.

\section{The theta function propagator}

Having now satisfyingly examined the qualitative structure of the propagator for $x=0$ as a function of time, we now return back to our original formulation to include the spatial dependence.

Recovering \eqref{amplitudeztau}, and recalling the definition \eqref{jacobidef} of the Jacobi theta function, we recognize

\begin{equation}
    A(z,\tau) = e^{-i\pi z^2 \tau} \frac{1}{\sqrt{-i\tau}}\, \vartheta\left(\frac{z}{\tau};-\frac{1}{\tau}\right)
\end{equation}

The full Jacobi theta function has more general modular transformation properties than the subcase $\vartheta(\tau) = \vartheta(0;\tau)$ we studied before. In particular, it holds that

\begin{equation}
    \vartheta(z+1;\tau) = \vartheta(z;\tau)
\end{equation}

\begin{equation}
    \vartheta(z;\tau + 1) = \vartheta(z+\frac{1}{2}; \tau)
\end{equation}

\begin{equation}\label{thtransS}
    \vartheta\left(\frac{z}{\tau}; -\frac{1}{\tau} \right) = \sqrt{-i\tau} \exp(\frac{\pi}{\tau} i z^2)
\end{equation}

Relation \eqref{thtransS} immediately implies that the fundamental solution $A$ is \emph{directly} a theta function:

\begin{equation}
    A(z,\tau) = \vartheta(z;\tau)
\end{equation}

The Jacobi theta for real values of the period is thus reinterpreted physically as the quantum propagator on the circle. There is an enlightening analogy with a better-known case; consider again the Schr\"odinger equation for a particle moving on some Riemannian manifold $M$:

\begin{equation}
    i \pdv{t} \Psi = - \frac{\nabla^2}{2} \Psi\,,
\end{equation}

where $\nabla^2$ is the Laplace-Beltrami operator on $M$, and perform the substitution $t \mapsto it$; this yields

\begin{equation}
    \pdv{t} \Psi = - \frac{\nabla^2}{2} \Psi\,,
\end{equation}

i.e. the heat / diffusion equation on $M$. Indeed, it is well-known $\vartheta(z;it)$ (much more regular than its real-period counterpart studied in this work) is the fundamental solution to the heat equation on the circle. This can be understood physically as diffusion being the Wick-rotated, or imaginary-time, counterpart of the Schr\"odinger equation. This, in fact, is to be seen as a minimal example of a general phenomenon in quantum theories in which the Wick-rotated theory (a statistical-mechanical system) has better converging observables; and observables in the quantum theory, only formally specified by non-converging oscillatory integrals, can be defined by analytic continuation of the imaginary-time observables.

We now show a remarkable structure of the theta function propagator that should clarify the nature of the singularities in the time dependence. Fixed a rational time $\tau = \sfrac{p}{q}$ in least terms, without loss of generality between $0$ and $1$, rewrite the summation index $n$ as $n = aq + b$, for $a \in \mathbb{Z}$, $b = 0,\ldots,q-1$. The amplitude becomes:

\newcommand{\sumA}{\sum_{a=-\infty}^\infty}
\newcommand{\sumB}{\sum_{b=0}^{q-1}}
\newcommand{\sumBnorm}{\sum_{b=0}^{|q|-1}}
\newcommand{\comb}{\operatorname{III}}

\begin{align}
    \T(z;\tau) = \sumA \sumB \exp(\pi i \frac{p}{q} (aq + b)^2  + 2\pi i (aq+b) z)\\
    = \left( \sumB e^{i\pi \frac{p b^2}{q} + 2\pi i b z} \right) \sumA (-1)^{pqa} e^{2\pi i a q} 
\end{align}

We recognize the sum over $a$ as the Fourier series for a Dirac comb, i.e. a periodic array of $\delta$ functions:

\begin{equation}
    \comb_T(z) := \sumZ \delta(z-nT) = \frac{1}{T} \sum_{k=-\infty}^\infty e^{2\pi i k z}\,,
\end{equation}

so that the sum over $a$ reduces to

\begin{align}
    \frac{1}{q} \comb_{\frac{1}{q}} (z+\Delta)\,,
\end{align}

where $\Delta = 0$ if $\sfrac{p}{q}$ is even, and $\Delta = \frac{1}{2}$ if it's odd. Thus we have the remarkable fact that at any fixed rational time $\tau = \sfrac{p}{q}$, the theta function amplitude is a sum of $q$ $\delta$-functions equally spaced around the circle; the Dirac deltas are then weighted by the phase factor

\begin{equation}
    \sumB \left(e^{i\pi/q}\right)^{pb^2 + 2bzq}
\end{equation}

Since $\comb_{\frac{1}{q}}$ only has support in $z = \frac{c+\Delta}{q}$ for $c = 0,\ldots,q-1$ (understood $\mod 1$), this factor is

\begin{equation}\label{betafactor}
    \beta_{p,q,c} = \sumBnorm \left(e^{i\pi/q}\right)^{pb^2 + 2bc + 2\Delta b}\,,
\end{equation}

Expression \eqref{betafactor} is a generalized quadratic Gauss sum, and its determination is extremely complex. It is however sufficient for our purposes to ascertain the norm $|\beta_{p,q,c}|$. We lift the restriction on $p$ and $q$ to be positive and for $p<q$ for the purpose of the proof. Immediately:

\begin{equation}
    \beta_{p+2kq,q,c} = \beta_{p,q,c}\,,
\end{equation}

so that $p$ can be taken to have $\abs{p} < \abs{q}$. In addition, we make use of the reciprocity formula for generalized quadratic Gauss sums\cite{berndt_gauss}:

\begin{equation}
    \abs{\beta_{p,q,c}} = \sqrt{\abs{\frac{q}{p}}} \; \abs{\beta_{-q,p,c}}\,.
\end{equation}

Having in addition noted that $\abs{\beta_{p,1,c}} = 1$ for all $p,c$, we can induct over all pairs $(p,q)$ and conclude\footnote{We remark the reasoning is strongly reminiscent of that employed in section \ref{sec:evens}, suggesting the calculations are connected. Indeed, a proof of the reciprocity formula can also be provided starting from the modular properties of the theta function\ref{berndt_gauss}.}

\begin{equation}
    |\beta_{p,q,c}| = \sqrt{q}
\end{equation}

%Indeed, defined $r=\exp(i\pi/q)$ and $l=2(c+\Delta)$, we have:
%
%\begin{equation}\label{thenorm}
%    \beta_{p,q,c} \beta^*_{p,q,c} = \sum_{b_1=0}^{q-1} \sum_{b_2=0}^{q-1} \left(e^{i\pi/q}\right)^{p(b_1^2 - b_2^2) + l (b_1 - b_2)} = q     
%\end{equation}
%
%\cmnt{Nessun progresso per dimostrare questo fatto. Numericamente sembra vero.}


Thus, the fundamental solution at fractional time becomes

\begin{equation}\label{finalrationalkernel}
    \vartheta\left( z ; \,\frac{p}{q} \right) = \frac{1}{\sqrt{q}} \sum_{c=0}^{q-1} \phi_{p,q,c} \, \delta \left(z-\frac{c}{q}\right)\,,
\end{equation}

for some phases $\phi_{p,q,c}$, a characterization which will become extremely useful when attempting a physical interpretation of time evolution of an initial state.

We just comment that the prefactor $1/\sqrt{q}$ is expected on the face of unitarity, or conservation of information. The time evolution operator of the Schr\"odinger equation is unitary, and preserves the $L^2$ norm of the wavefunction. Taken an initial smooth wavefunction $\psi(z)$ with compact support in an interval around $0$ smaller than $\frac{1}{2q}$, and normalized such that $\int_{\mathbb{S}^1} dx \abs{\psi(z)}^2 = 1$, time evolution by a fractional time $p/q$ is equivalent with convolution with the $\vartheta(z;\sfrac{p}{q})$ kernel, which results by \eqref{finalrationalkernel} in $q$ non-overlapping copies of the original wavefunction, with some phases and rescaled by $1/\sqrt{q}$. The final squared norm of the wavefunction is thus

\begin{equation}
    \frac{1}{q}\sum_{c=0}^{q-1} 1 = 1\,.
\end{equation}

\section{Physical interpretation}

\subsection{Fractional revivals and antirevivals}

\cmnt{Spiegazione della struttura a tempi razionali: a tempo razionale pari la funzione d'onda viene ``rivitalizzata'' in $q$ copie, mentre a tempo razionale dispari le copie rivitalizzate sono sfasate di $1/(2q)$ (antirevivals)}

%\immagine{3dplots/1-3.pgf}{20cm}{ }{13}
%% Creator: Matplotlib, PGF backend
%%
%% To include the figure in your LaTeX document, write
%%   \input{<filename>.pgf}
%%
%% Make sure the required packages are loaded in your preamble
%%   \usepackage{pgf}
%%
%% Figures using additional raster images can only be included by \input if
%% they are in the same directory as the main LaTeX file. For loading figures
%% from other directories you can use the `import` package
%%   \usepackage{import}
%% and then include the figures with
%%   \import{<path to file>}{<filename>.pgf}
%%
%% Matplotlib used the following preamble
%%   \usepackage{fontspec}
%%   \setmainfont{DejaVu Serif}
%%   \setsansfont{DejaVu Sans}
%%   \setmonofont{DejaVu Sans Mono}
%%
\begingroup%
\makeatletter%
\begin{pgfpicture}%
\pgfpathrectangle{\pgfpointorigin}{\pgfqpoint{1.610000in}{1.610000in}}%
\pgfusepath{use as bounding box, clip}%
\begin{pgfscope}%
\pgfsetbuttcap%
\pgfsetmiterjoin%
\definecolor{currentfill}{rgb}{1.000000,1.000000,1.000000}%
\pgfsetfillcolor{currentfill}%
\pgfsetlinewidth{0.000000pt}%
\definecolor{currentstroke}{rgb}{1.000000,1.000000,1.000000}%
\pgfsetstrokecolor{currentstroke}%
\pgfsetdash{}{0pt}%
\pgfpathmoveto{\pgfqpoint{0.000000in}{0.000000in}}%
\pgfpathlineto{\pgfqpoint{1.610000in}{0.000000in}}%
\pgfpathlineto{\pgfqpoint{1.610000in}{1.610000in}}%
\pgfpathlineto{\pgfqpoint{0.000000in}{1.610000in}}%
\pgfpathclose%
\pgfusepath{fill}%
\end{pgfscope}%
\begin{pgfscope}%
\pgfsetbuttcap%
\pgfsetmiterjoin%
\definecolor{currentfill}{rgb}{1.000000,1.000000,1.000000}%
\pgfsetfillcolor{currentfill}%
\pgfsetlinewidth{0.000000pt}%
\definecolor{currentstroke}{rgb}{0.000000,0.000000,0.000000}%
\pgfsetstrokecolor{currentstroke}%
\pgfsetstrokeopacity{0.000000}%
\pgfsetdash{}{0pt}%
\pgfpathmoveto{\pgfqpoint{0.035000in}{0.035000in}}%
\pgfpathlineto{\pgfqpoint{1.575000in}{0.035000in}}%
\pgfpathlineto{\pgfqpoint{1.575000in}{1.575000in}}%
\pgfpathlineto{\pgfqpoint{0.035000in}{1.575000in}}%
\pgfpathclose%
\pgfusepath{fill}%
\end{pgfscope}%
\begin{pgfscope}%
\pgfpathrectangle{\pgfqpoint{0.035000in}{0.035000in}}{\pgfqpoint{1.540000in}{1.540000in}} %
\pgfusepath{clip}%
\pgfsetrectcap%
\pgfsetroundjoin%
\pgfsetlinewidth{1.505625pt}%
\definecolor{currentstroke}{rgb}{0.000000,0.000000,0.000000}%
\pgfsetstrokecolor{currentstroke}%
\pgfsetdash{}{0pt}%
\pgfpathmoveto{\pgfqpoint{1.185061in}{0.362854in}}%
\pgfpathlineto{\pgfqpoint{1.199620in}{0.379249in}}%
\pgfpathlineto{\pgfqpoint{1.211556in}{0.396183in}}%
\pgfpathlineto{\pgfqpoint{1.220196in}{0.412186in}}%
\pgfpathlineto{\pgfqpoint{1.226521in}{0.428446in}}%
\pgfpathlineto{\pgfqpoint{1.230518in}{0.444865in}}%
\pgfpathlineto{\pgfqpoint{1.232184in}{0.461347in}}%
\pgfpathlineto{\pgfqpoint{1.231531in}{0.477797in}}%
\pgfpathlineto{\pgfqpoint{1.228587in}{0.494122in}}%
\pgfpathlineto{\pgfqpoint{1.223388in}{0.510231in}}%
\pgfpathlineto{\pgfqpoint{1.215985in}{0.526038in}}%
\pgfpathlineto{\pgfqpoint{1.206439in}{0.541457in}}%
\pgfpathlineto{\pgfqpoint{1.193763in}{0.557631in}}%
\pgfpathlineto{\pgfqpoint{1.178759in}{0.573158in}}%
\pgfpathlineto{\pgfqpoint{1.160131in}{0.589050in}}%
\pgfpathlineto{\pgfqpoint{1.139096in}{0.603980in}}%
\pgfpathlineto{\pgfqpoint{1.115826in}{0.617849in}}%
\pgfpathlineto{\pgfqpoint{1.088622in}{0.631429in}}%
\pgfpathlineto{\pgfqpoint{1.059302in}{0.643589in}}%
\pgfpathlineto{\pgfqpoint{1.025977in}{0.654897in}}%
\pgfpathlineto{\pgfqpoint{0.990843in}{0.664400in}}%
\pgfpathlineto{\pgfqpoint{0.954229in}{0.672023in}}%
\pgfpathlineto{\pgfqpoint{0.916466in}{0.677706in}}%
\pgfpathlineto{\pgfqpoint{0.875467in}{0.681570in}}%
\pgfpathlineto{\pgfqpoint{0.833970in}{0.683161in}}%
\pgfpathlineto{\pgfqpoint{0.792391in}{0.682466in}}%
\pgfpathlineto{\pgfqpoint{0.751148in}{0.679490in}}%
\pgfpathlineto{\pgfqpoint{0.710653in}{0.674260in}}%
\pgfpathlineto{\pgfqpoint{0.671319in}{0.666833in}}%
\pgfpathlineto{\pgfqpoint{0.603819in}{0.652707in}}%
\pgfpathlineto{\pgfqpoint{0.595550in}{0.653924in}}%
\pgfpathlineto{\pgfqpoint{0.589401in}{0.656622in}}%
\pgfpathlineto{\pgfqpoint{0.583283in}{0.661563in}}%
\pgfpathlineto{\pgfqpoint{0.577183in}{0.669584in}}%
\pgfpathlineto{\pgfqpoint{0.571079in}{0.681713in}}%
\pgfpathlineto{\pgfqpoint{0.564947in}{0.699172in}}%
\pgfpathlineto{\pgfqpoint{0.558756in}{0.723350in}}%
\pgfpathlineto{\pgfqpoint{0.552472in}{0.755749in}}%
\pgfpathlineto{\pgfqpoint{0.543885in}{0.814378in}}%
\pgfpathlineto{\pgfqpoint{0.534980in}{0.893645in}}%
\pgfpathlineto{\pgfqpoint{0.525685in}{0.995863in}}%
\pgfpathlineto{\pgfqpoint{0.513469in}{1.156397in}}%
\pgfpathlineto{\pgfqpoint{0.495417in}{1.430759in}}%
\pgfpathlineto{\pgfqpoint{0.485785in}{1.585000in}}%
\pgfpathmoveto{\pgfqpoint{0.433702in}{1.585000in}}%
\pgfpathlineto{\pgfqpoint{0.433850in}{1.058305in}}%
\pgfpathlineto{\pgfqpoint{0.432753in}{0.763829in}}%
\pgfpathlineto{\pgfqpoint{0.430083in}{0.614765in}}%
\pgfpathlineto{\pgfqpoint{0.426634in}{0.540739in}}%
\pgfpathlineto{\pgfqpoint{0.422428in}{0.498176in}}%
\pgfpathlineto{\pgfqpoint{0.419386in}{0.466585in}}%
\pgfpathlineto{\pgfqpoint{0.420347in}{0.449915in}}%
\pgfpathlineto{\pgfqpoint{0.423632in}{0.433451in}}%
\pgfpathlineto{\pgfqpoint{0.429247in}{0.417132in}}%
\pgfpathlineto{\pgfqpoint{0.437182in}{0.401040in}}%
\pgfpathlineto{\pgfqpoint{0.447412in}{0.385272in}}%
\pgfpathlineto{\pgfqpoint{0.459896in}{0.369926in}}%
\pgfpathlineto{\pgfqpoint{0.475901in}{0.353888in}}%
\pgfpathlineto{\pgfqpoint{0.494395in}{0.338578in}}%
\pgfpathlineto{\pgfqpoint{0.515268in}{0.324112in}}%
\pgfpathlineto{\pgfqpoint{0.540254in}{0.309604in}}%
\pgfpathlineto{\pgfqpoint{0.567656in}{0.296337in}}%
\pgfpathlineto{\pgfqpoint{0.597257in}{0.284433in}}%
\pgfpathlineto{\pgfqpoint{0.631141in}{0.273317in}}%
\pgfpathlineto{\pgfqpoint{0.666952in}{0.264014in}}%
\pgfpathlineto{\pgfqpoint{0.704344in}{0.256626in}}%
\pgfpathlineto{\pgfqpoint{0.742950in}{0.251234in}}%
\pgfpathlineto{\pgfqpoint{0.782387in}{0.247899in}}%
\pgfpathlineto{\pgfqpoint{0.822259in}{0.246658in}}%
\pgfpathlineto{\pgfqpoint{0.862167in}{0.247525in}}%
\pgfpathlineto{\pgfqpoint{0.901710in}{0.250489in}}%
\pgfpathlineto{\pgfqpoint{0.940492in}{0.255519in}}%
\pgfpathlineto{\pgfqpoint{0.978127in}{0.262558in}}%
\pgfpathlineto{\pgfqpoint{1.014246in}{0.271528in}}%
\pgfpathlineto{\pgfqpoint{1.048496in}{0.282330in}}%
\pgfpathlineto{\pgfqpoint{1.080552in}{0.294846in}}%
\pgfpathlineto{\pgfqpoint{1.108228in}{0.307957in}}%
\pgfpathlineto{\pgfqpoint{1.133512in}{0.322324in}}%
\pgfpathlineto{\pgfqpoint{1.156213in}{0.337815in}}%
\pgfpathlineto{\pgfqpoint{1.174832in}{0.353085in}}%
\pgfpathlineto{\pgfqpoint{1.183835in}{0.361618in}}%
\pgfpathlineto{\pgfqpoint{1.183835in}{0.361618in}}%
\pgfusepath{stroke}%
\end{pgfscope}%
\begin{pgfscope}%
\pgftext[x=0.825811in,y=0.076622in,,]{\rmfamily\fontsize{10.000000}{12.000000}\selectfont \(\displaystyle \tau = 0\)}%
\end{pgfscope}%
\end{pgfpicture}%
\makeatother%
\endgroup%

%% Creator: Matplotlib, PGF backend
%%
%% To include the figure in your LaTeX document, write
%%   \input{<filename>.pgf}
%%
%% Make sure the required packages are loaded in your preamble
%%   \usepackage{pgf}
%%
%% Figures using additional raster images can only be included by \input if
%% they are in the same directory as the main LaTeX file. For loading figures
%% from other directories you can use the `import` package
%%   \usepackage{import}
%% and then include the figures with
%%   \import{<path to file>}{<filename>.pgf}
%%
%% Matplotlib used the following preamble
%%   \usepackage{fontspec}
%%   \setmainfont{DejaVu Serif}
%%   \setsansfont{DejaVu Sans}
%%   \setmonofont{DejaVu Sans Mono}
%%
\begingroup%
\makeatletter%
\begin{pgfpicture}%
\pgfpathrectangle{\pgfpointorigin}{\pgfqpoint{1.610000in}{1.672747in}}%
\pgfusepath{use as bounding box, clip}%
\begin{pgfscope}%
\pgfsetbuttcap%
\pgfsetmiterjoin%
\definecolor{currentfill}{rgb}{1.000000,1.000000,1.000000}%
\pgfsetfillcolor{currentfill}%
\pgfsetlinewidth{0.000000pt}%
\definecolor{currentstroke}{rgb}{1.000000,1.000000,1.000000}%
\pgfsetstrokecolor{currentstroke}%
\pgfsetdash{}{0pt}%
\pgfpathmoveto{\pgfqpoint{0.000000in}{0.000000in}}%
\pgfpathlineto{\pgfqpoint{1.610000in}{0.000000in}}%
\pgfpathlineto{\pgfqpoint{1.610000in}{1.672747in}}%
\pgfpathlineto{\pgfqpoint{0.000000in}{1.672747in}}%
\pgfpathclose%
\pgfusepath{fill}%
\end{pgfscope}%
\begin{pgfscope}%
\pgfsetbuttcap%
\pgfsetmiterjoin%
\definecolor{currentfill}{rgb}{1.000000,1.000000,1.000000}%
\pgfsetfillcolor{currentfill}%
\pgfsetlinewidth{0.000000pt}%
\definecolor{currentstroke}{rgb}{0.000000,0.000000,0.000000}%
\pgfsetstrokecolor{currentstroke}%
\pgfsetstrokeopacity{0.000000}%
\pgfsetdash{}{0pt}%
\pgfpathmoveto{\pgfqpoint{0.035000in}{0.097747in}}%
\pgfpathlineto{\pgfqpoint{1.575000in}{0.097747in}}%
\pgfpathlineto{\pgfqpoint{1.575000in}{1.637747in}}%
\pgfpathlineto{\pgfqpoint{0.035000in}{1.637747in}}%
\pgfpathclose%
\pgfusepath{fill}%
\end{pgfscope}%
\begin{pgfscope}%
\pgfpathrectangle{\pgfqpoint{0.035000in}{0.097747in}}{\pgfqpoint{1.540000in}{1.540000in}} %
\pgfusepath{clip}%
\pgfsetrectcap%
\pgfsetroundjoin%
\pgfsetlinewidth{1.505625pt}%
\definecolor{currentstroke}{rgb}{0.000000,0.000000,0.000000}%
\pgfsetstrokecolor{currentstroke}%
\pgfsetdash{}{0pt}%
\pgfpathmoveto{\pgfqpoint{1.192294in}{0.706563in}}%
\pgfpathlineto{\pgfqpoint{1.193520in}{0.707442in}}%
\pgfpathlineto{\pgfqpoint{1.194714in}{0.707690in}}%
\pgfpathlineto{\pgfqpoint{1.197006in}{0.706327in}}%
\pgfpathlineto{\pgfqpoint{1.200208in}{0.699862in}}%
\pgfpathlineto{\pgfqpoint{1.204058in}{0.684021in}}%
\pgfpathlineto{\pgfqpoint{1.208267in}{0.655700in}}%
\pgfpathlineto{\pgfqpoint{1.215159in}{0.589579in}}%
\pgfpathlineto{\pgfqpoint{1.222241in}{0.526976in}}%
\pgfpathlineto{\pgfqpoint{1.225828in}{0.510636in}}%
\pgfpathlineto{\pgfqpoint{1.228085in}{0.507343in}}%
\pgfpathlineto{\pgfqpoint{1.229349in}{0.508144in}}%
\pgfpathlineto{\pgfqpoint{1.230841in}{0.512297in}}%
\pgfpathlineto{\pgfqpoint{1.232155in}{0.523447in}}%
\pgfpathlineto{\pgfqpoint{1.231676in}{0.539260in}}%
\pgfpathlineto{\pgfqpoint{1.228919in}{0.555518in}}%
\pgfpathlineto{\pgfqpoint{1.223907in}{0.571668in}}%
\pgfpathlineto{\pgfqpoint{1.215994in}{0.589094in}}%
\pgfpathlineto{\pgfqpoint{1.193036in}{0.634936in}}%
\pgfpathlineto{\pgfqpoint{1.185517in}{0.658383in}}%
\pgfpathlineto{\pgfqpoint{1.178745in}{0.686762in}}%
\pgfpathlineto{\pgfqpoint{1.170508in}{0.731555in}}%
\pgfpathlineto{\pgfqpoint{1.156711in}{0.823151in}}%
\pgfpathlineto{\pgfqpoint{1.145668in}{0.890822in}}%
\pgfpathlineto{\pgfqpoint{1.138141in}{0.924114in}}%
\pgfpathlineto{\pgfqpoint{1.131711in}{0.942097in}}%
\pgfpathlineto{\pgfqpoint{1.126638in}{0.949500in}}%
\pgfpathlineto{\pgfqpoint{1.123137in}{0.951355in}}%
\pgfpathlineto{\pgfqpoint{1.119543in}{0.950738in}}%
\pgfpathlineto{\pgfqpoint{1.115861in}{0.947718in}}%
\pgfpathlineto{\pgfqpoint{1.110182in}{0.938988in}}%
\pgfpathlineto{\pgfqpoint{1.102355in}{0.920638in}}%
\pgfpathlineto{\pgfqpoint{1.092242in}{0.890075in}}%
\pgfpathlineto{\pgfqpoint{1.064969in}{0.803349in}}%
\pgfpathlineto{\pgfqpoint{1.054300in}{0.777091in}}%
\pgfpathlineto{\pgfqpoint{1.045723in}{0.760716in}}%
\pgfpathlineto{\pgfqpoint{1.037105in}{0.748437in}}%
\pgfpathlineto{\pgfqpoint{1.028440in}{0.739845in}}%
\pgfpathlineto{\pgfqpoint{1.019718in}{0.734338in}}%
\pgfpathlineto{\pgfqpoint{1.010929in}{0.731245in}}%
\pgfpathlineto{\pgfqpoint{0.999838in}{0.729807in}}%
\pgfpathlineto{\pgfqpoint{0.984101in}{0.730531in}}%
\pgfpathlineto{\pgfqpoint{0.954231in}{0.734987in}}%
\pgfpathlineto{\pgfqpoint{0.906886in}{0.741575in}}%
\pgfpathlineto{\pgfqpoint{0.863296in}{0.745174in}}%
\pgfpathlineto{\pgfqpoint{0.819282in}{0.748718in}}%
\pgfpathlineto{\pgfqpoint{0.804595in}{0.751957in}}%
\pgfpathlineto{\pgfqpoint{0.792364in}{0.756895in}}%
\pgfpathlineto{\pgfqpoint{0.782586in}{0.763110in}}%
\pgfpathlineto{\pgfqpoint{0.772812in}{0.772047in}}%
\pgfpathlineto{\pgfqpoint{0.763039in}{0.784353in}}%
\pgfpathlineto{\pgfqpoint{0.753263in}{0.800546in}}%
\pgfpathlineto{\pgfqpoint{0.743480in}{0.820869in}}%
\pgfpathlineto{\pgfqpoint{0.731238in}{0.851744in}}%
\pgfpathlineto{\pgfqpoint{0.711648in}{0.909278in}}%
\pgfpathlineto{\pgfqpoint{0.694597in}{0.957702in}}%
\pgfpathlineto{\pgfqpoint{0.684957in}{0.979333in}}%
\pgfpathlineto{\pgfqpoint{0.677805in}{0.990865in}}%
\pgfpathlineto{\pgfqpoint{0.670739in}{0.997515in}}%
\pgfpathlineto{\pgfqpoint{0.666083in}{0.998982in}}%
\pgfpathlineto{\pgfqpoint{0.661475in}{0.997987in}}%
\pgfpathlineto{\pgfqpoint{0.656919in}{0.994530in}}%
\pgfpathlineto{\pgfqpoint{0.652416in}{0.988676in}}%
\pgfpathlineto{\pgfqpoint{0.645764in}{0.975700in}}%
\pgfpathlineto{\pgfqpoint{0.637087in}{0.951679in}}%
\pgfpathlineto{\pgfqpoint{0.626532in}{0.913981in}}%
\pgfpathlineto{\pgfqpoint{0.588584in}{0.768624in}}%
\pgfpathlineto{\pgfqpoint{0.578984in}{0.742613in}}%
\pgfpathlineto{\pgfqpoint{0.569524in}{0.722697in}}%
\pgfpathlineto{\pgfqpoint{0.560216in}{0.707849in}}%
\pgfpathlineto{\pgfqpoint{0.549280in}{0.694926in}}%
\pgfpathlineto{\pgfqpoint{0.536899in}{0.684047in}}%
\pgfpathlineto{\pgfqpoint{0.515174in}{0.668808in}}%
\pgfpathlineto{\pgfqpoint{0.488214in}{0.649233in}}%
\pgfpathlineto{\pgfqpoint{0.470003in}{0.633204in}}%
\pgfpathlineto{\pgfqpoint{0.453358in}{0.615497in}}%
\pgfpathlineto{\pgfqpoint{0.437398in}{0.598677in}}%
\pgfpathlineto{\pgfqpoint{0.433696in}{0.597664in}}%
\pgfpathlineto{\pgfqpoint{0.430933in}{0.599270in}}%
\pgfpathlineto{\pgfqpoint{0.428328in}{0.603758in}}%
\pgfpathlineto{\pgfqpoint{0.425266in}{0.614429in}}%
\pgfpathlineto{\pgfqpoint{0.421847in}{0.635875in}}%
\pgfpathlineto{\pgfqpoint{0.417735in}{0.678831in}}%
\pgfpathlineto{\pgfqpoint{0.413098in}{0.755364in}}%
\pgfpathlineto{\pgfqpoint{0.411899in}{0.793319in}}%
\pgfpathlineto{\pgfqpoint{0.412792in}{0.785126in}}%
\pgfpathlineto{\pgfqpoint{0.433731in}{0.537071in}}%
\pgfpathlineto{\pgfqpoint{0.440102in}{0.495928in}}%
\pgfpathlineto{\pgfqpoint{0.445896in}{0.470582in}}%
\pgfpathlineto{\pgfqpoint{0.452055in}{0.452343in}}%
\pgfpathlineto{\pgfqpoint{0.458639in}{0.438975in}}%
\pgfpathlineto{\pgfqpoint{0.466919in}{0.427068in}}%
\pgfpathlineto{\pgfqpoint{0.478575in}{0.414681in}}%
\pgfpathlineto{\pgfqpoint{0.494393in}{0.401388in}}%
\pgfpathlineto{\pgfqpoint{0.515267in}{0.386869in}}%
\pgfpathlineto{\pgfqpoint{0.538386in}{0.373412in}}%
\pgfpathlineto{\pgfqpoint{0.561581in}{0.362353in}}%
\pgfpathlineto{\pgfqpoint{0.577954in}{0.356565in}}%
\pgfpathlineto{\pgfqpoint{0.590665in}{0.354061in}}%
\pgfpathlineto{\pgfqpoint{0.601514in}{0.354129in}}%
\pgfpathlineto{\pgfqpoint{0.610344in}{0.356321in}}%
\pgfpathlineto{\pgfqpoint{0.619295in}{0.361112in}}%
\pgfpathlineto{\pgfqpoint{0.628357in}{0.369205in}}%
\pgfpathlineto{\pgfqpoint{0.637521in}{0.381262in}}%
\pgfpathlineto{\pgfqpoint{0.646781in}{0.397777in}}%
\pgfpathlineto{\pgfqpoint{0.656140in}{0.418913in}}%
\pgfpathlineto{\pgfqpoint{0.667988in}{0.451270in}}%
\pgfpathlineto{\pgfqpoint{0.707344in}{0.566683in}}%
\pgfpathlineto{\pgfqpoint{0.715018in}{0.582068in}}%
\pgfpathlineto{\pgfqpoint{0.722788in}{0.592955in}}%
\pgfpathlineto{\pgfqpoint{0.728019in}{0.597351in}}%
\pgfpathlineto{\pgfqpoint{0.733288in}{0.599284in}}%
\pgfpathlineto{\pgfqpoint{0.738591in}{0.598686in}}%
\pgfpathlineto{\pgfqpoint{0.743924in}{0.595554in}}%
\pgfpathlineto{\pgfqpoint{0.749283in}{0.589960in}}%
\pgfpathlineto{\pgfqpoint{0.757362in}{0.577269in}}%
\pgfpathlineto{\pgfqpoint{0.765474in}{0.560074in}}%
\pgfpathlineto{\pgfqpoint{0.776317in}{0.531800in}}%
\pgfpathlineto{\pgfqpoint{0.797994in}{0.466729in}}%
\pgfpathlineto{\pgfqpoint{0.816865in}{0.412908in}}%
\pgfpathlineto{\pgfqpoint{0.830274in}{0.381348in}}%
\pgfpathlineto{\pgfqpoint{0.840959in}{0.361246in}}%
\pgfpathlineto{\pgfqpoint{0.851611in}{0.345702in}}%
\pgfpathlineto{\pgfqpoint{0.862230in}{0.334292in}}%
\pgfpathlineto{\pgfqpoint{0.872817in}{0.326386in}}%
\pgfpathlineto{\pgfqpoint{0.883368in}{0.321282in}}%
\pgfpathlineto{\pgfqpoint{0.896499in}{0.317832in}}%
\pgfpathlineto{\pgfqpoint{0.912156in}{0.316582in}}%
\pgfpathlineto{\pgfqpoint{0.932820in}{0.317729in}}%
\pgfpathlineto{\pgfqpoint{0.960732in}{0.321860in}}%
\pgfpathlineto{\pgfqpoint{0.995193in}{0.329258in}}%
\pgfpathlineto{\pgfqpoint{1.030483in}{0.339164in}}%
\pgfpathlineto{\pgfqpoint{1.057282in}{0.348882in}}%
\pgfpathlineto{\pgfqpoint{1.074371in}{0.357153in}}%
\pgfpathlineto{\pgfqpoint{1.086769in}{0.365374in}}%
\pgfpathlineto{\pgfqpoint{1.096835in}{0.374554in}}%
\pgfpathlineto{\pgfqpoint{1.106676in}{0.386982in}}%
\pgfpathlineto{\pgfqpoint{1.114402in}{0.400128in}}%
\pgfpathlineto{\pgfqpoint{1.122012in}{0.416861in}}%
\pgfpathlineto{\pgfqpoint{1.131382in}{0.443682in}}%
\pgfpathlineto{\pgfqpoint{1.140606in}{0.477536in}}%
\pgfpathlineto{\pgfqpoint{1.151464in}{0.526565in}}%
\pgfpathlineto{\pgfqpoint{1.185696in}{0.692783in}}%
\pgfpathlineto{\pgfqpoint{1.191037in}{0.705052in}}%
\pgfpathlineto{\pgfqpoint{1.191037in}{0.705052in}}%
\pgfusepath{stroke}%
\end{pgfscope}%
\begin{pgfscope}%
\pgftext[x=0.825811in,y=0.139368in,,]{\rmfamily\fontsize{10.000000}{12.000000}\selectfont \(\displaystyle \tau = \frac{1}{5}\)}%
\end{pgfscope}%
\end{pgfpicture}%
\makeatother%
\endgroup%

%% Creator: Matplotlib, PGF backend
%%
%% To include the figure in your LaTeX document, write
%%   \input{<filename>.pgf}
%%
%% Make sure the required packages are loaded in your preamble
%%   \usepackage{pgf}
%%
%% Figures using additional raster images can only be included by \input if
%% they are in the same directory as the main LaTeX file. For loading figures
%% from other directories you can use the `import` package
%%   \usepackage{import}
%% and then include the figures with
%%   \import{<path to file>}{<filename>.pgf}
%%
%% Matplotlib used the following preamble
%%   \usepackage{fontspec}
%%   \setmainfont{DejaVu Serif}
%%   \setsansfont{DejaVu Sans}
%%   \setmonofont{DejaVu Sans Mono}
%%
\begingroup%
\makeatletter%
\begin{pgfpicture}%
\pgfpathrectangle{\pgfpointorigin}{\pgfqpoint{1.620000in}{1.672747in}}%
\pgfusepath{use as bounding box, clip}%
\begin{pgfscope}%
\pgfsetbuttcap%
\pgfsetmiterjoin%
\definecolor{currentfill}{rgb}{1.000000,1.000000,1.000000}%
\pgfsetfillcolor{currentfill}%
\pgfsetlinewidth{0.000000pt}%
\definecolor{currentstroke}{rgb}{1.000000,1.000000,1.000000}%
\pgfsetstrokecolor{currentstroke}%
\pgfsetdash{}{0pt}%
\pgfpathmoveto{\pgfqpoint{0.000000in}{0.000000in}}%
\pgfpathlineto{\pgfqpoint{1.620000in}{0.000000in}}%
\pgfpathlineto{\pgfqpoint{1.620000in}{1.672747in}}%
\pgfpathlineto{\pgfqpoint{0.000000in}{1.672747in}}%
\pgfpathclose%
\pgfusepath{fill}%
\end{pgfscope}%
\begin{pgfscope}%
\pgfsetbuttcap%
\pgfsetmiterjoin%
\definecolor{currentfill}{rgb}{1.000000,1.000000,1.000000}%
\pgfsetfillcolor{currentfill}%
\pgfsetlinewidth{0.000000pt}%
\definecolor{currentstroke}{rgb}{0.000000,0.000000,0.000000}%
\pgfsetstrokecolor{currentstroke}%
\pgfsetstrokeopacity{0.000000}%
\pgfsetdash{}{0pt}%
\pgfpathmoveto{\pgfqpoint{0.035000in}{0.097747in}}%
\pgfpathlineto{\pgfqpoint{1.585000in}{0.097747in}}%
\pgfpathlineto{\pgfqpoint{1.585000in}{1.637747in}}%
\pgfpathlineto{\pgfqpoint{0.035000in}{1.637747in}}%
\pgfpathclose%
\pgfusepath{fill}%
\end{pgfscope}%
\begin{pgfscope}%
\pgfpathrectangle{\pgfqpoint{0.035000in}{0.097747in}}{\pgfqpoint{1.550000in}{1.540000in}} %
\pgfusepath{clip}%
\pgfsetrectcap%
\pgfsetroundjoin%
\pgfsetlinewidth{1.505625pt}%
\definecolor{currentstroke}{rgb}{0.000000,0.000000,0.000000}%
\pgfsetstrokecolor{currentstroke}%
\pgfsetdash{}{0pt}%
\pgfpathmoveto{\pgfqpoint{1.205596in}{0.929897in}}%
\pgfpathlineto{\pgfqpoint{1.207920in}{0.925308in}}%
\pgfpathlineto{\pgfqpoint{1.211069in}{0.910321in}}%
\pgfpathlineto{\pgfqpoint{1.214678in}{0.877148in}}%
\pgfpathlineto{\pgfqpoint{1.218379in}{0.820261in}}%
\pgfpathlineto{\pgfqpoint{1.223981in}{0.689968in}}%
\pgfpathlineto{\pgfqpoint{1.230234in}{0.557560in}}%
\pgfpathlineto{\pgfqpoint{1.233855in}{0.520873in}}%
\pgfpathlineto{\pgfqpoint{1.236599in}{0.511143in}}%
\pgfpathlineto{\pgfqpoint{1.237720in}{0.511256in}}%
\pgfpathlineto{\pgfqpoint{1.239016in}{0.515031in}}%
\pgfpathlineto{\pgfqpoint{1.240011in}{0.526132in}}%
\pgfpathlineto{\pgfqpoint{1.239144in}{0.541982in}}%
\pgfpathlineto{\pgfqpoint{1.235987in}{0.558226in}}%
\pgfpathlineto{\pgfqpoint{1.230568in}{0.574308in}}%
\pgfpathlineto{\pgfqpoint{1.222935in}{0.590085in}}%
\pgfpathlineto{\pgfqpoint{1.213150in}{0.605469in}}%
\pgfpathlineto{\pgfqpoint{1.200208in}{0.621603in}}%
\pgfpathlineto{\pgfqpoint{1.183665in}{0.638358in}}%
\pgfpathlineto{\pgfqpoint{1.149580in}{0.671290in}}%
\pgfpathlineto{\pgfqpoint{1.141710in}{0.682636in}}%
\pgfpathlineto{\pgfqpoint{1.133646in}{0.698325in}}%
\pgfpathlineto{\pgfqpoint{1.125431in}{0.720462in}}%
\pgfpathlineto{\pgfqpoint{1.117107in}{0.751360in}}%
\pgfpathlineto{\pgfqpoint{1.108705in}{0.793048in}}%
\pgfpathlineto{\pgfqpoint{1.100228in}{0.846538in}}%
\pgfpathlineto{\pgfqpoint{1.088167in}{0.939178in}}%
\pgfpathlineto{\pgfqpoint{1.066284in}{1.110836in}}%
\pgfpathlineto{\pgfqpoint{1.058411in}{1.156178in}}%
\pgfpathlineto{\pgfqpoint{1.052262in}{1.181702in}}%
\pgfpathlineto{\pgfqpoint{1.045894in}{1.198353in}}%
\pgfpathlineto{\pgfqpoint{1.041526in}{1.204051in}}%
\pgfpathlineto{\pgfqpoint{1.037063in}{1.205266in}}%
\pgfpathlineto{\pgfqpoint{1.032511in}{1.202000in}}%
\pgfpathlineto{\pgfqpoint{1.027875in}{1.194381in}}%
\pgfpathlineto{\pgfqpoint{1.023164in}{1.182657in}}%
\pgfpathlineto{\pgfqpoint{1.015975in}{1.158173in}}%
\pgfpathlineto{\pgfqpoint{1.006216in}{1.115165in}}%
\pgfpathlineto{\pgfqpoint{0.991359in}{1.037580in}}%
\pgfpathlineto{\pgfqpoint{0.968994in}{0.920713in}}%
\pgfpathlineto{\pgfqpoint{0.956650in}{0.866977in}}%
\pgfpathlineto{\pgfqpoint{0.946830in}{0.832280in}}%
\pgfpathlineto{\pgfqpoint{0.937049in}{0.805086in}}%
\pgfpathlineto{\pgfqpoint{0.927295in}{0.784771in}}%
\pgfpathlineto{\pgfqpoint{0.917552in}{0.770307in}}%
\pgfpathlineto{\pgfqpoint{0.907808in}{0.760502in}}%
\pgfpathlineto{\pgfqpoint{0.898053in}{0.754194in}}%
\pgfpathlineto{\pgfqpoint{0.888282in}{0.750364in}}%
\pgfpathlineto{\pgfqpoint{0.876039in}{0.747827in}}%
\pgfpathlineto{\pgfqpoint{0.856392in}{0.746436in}}%
\pgfpathlineto{\pgfqpoint{0.765513in}{0.743142in}}%
\pgfpathlineto{\pgfqpoint{0.724541in}{0.738438in}}%
\pgfpathlineto{\pgfqpoint{0.684640in}{0.731517in}}%
\pgfpathlineto{\pgfqpoint{0.646216in}{0.722556in}}%
\pgfpathlineto{\pgfqpoint{0.607543in}{0.713519in}}%
\pgfpathlineto{\pgfqpoint{0.597192in}{0.713127in}}%
\pgfpathlineto{\pgfqpoint{0.589023in}{0.714743in}}%
\pgfpathlineto{\pgfqpoint{0.582951in}{0.717690in}}%
\pgfpathlineto{\pgfqpoint{0.576916in}{0.722689in}}%
\pgfpathlineto{\pgfqpoint{0.570908in}{0.730346in}}%
\pgfpathlineto{\pgfqpoint{0.564911in}{0.741345in}}%
\pgfpathlineto{\pgfqpoint{0.558910in}{0.756417in}}%
\pgfpathlineto{\pgfqpoint{0.550875in}{0.784099in}}%
\pgfpathlineto{\pgfqpoint{0.542764in}{0.821814in}}%
\pgfpathlineto{\pgfqpoint{0.534550in}{0.870501in}}%
\pgfpathlineto{\pgfqpoint{0.524125in}{0.946609in}}%
\pgfpathlineto{\pgfqpoint{0.509348in}{1.074977in}}%
\pgfpathlineto{\pgfqpoint{0.486913in}{1.273885in}}%
\pgfpathlineto{\pgfqpoint{0.479536in}{1.323474in}}%
\pgfpathlineto{\pgfqpoint{0.474444in}{1.346632in}}%
\pgfpathlineto{\pgfqpoint{0.471292in}{1.354393in}}%
\pgfpathlineto{\pgfqpoint{0.469792in}{1.355845in}}%
\pgfpathlineto{\pgfqpoint{0.468346in}{1.355650in}}%
\pgfpathlineto{\pgfqpoint{0.465613in}{1.350303in}}%
\pgfpathlineto{\pgfqpoint{0.463097in}{1.338441in}}%
\pgfpathlineto{\pgfqpoint{0.459720in}{1.309081in}}%
\pgfpathlineto{\pgfqpoint{0.455908in}{1.251084in}}%
\pgfpathlineto{\pgfqpoint{0.452077in}{1.156492in}}%
\pgfpathlineto{\pgfqpoint{0.446328in}{0.943170in}}%
\pgfpathlineto{\pgfqpoint{0.439806in}{0.725254in}}%
\pgfpathlineto{\pgfqpoint{0.435121in}{0.639325in}}%
\pgfpathlineto{\pgfqpoint{0.430610in}{0.592693in}}%
\pgfpathlineto{\pgfqpoint{0.424735in}{0.555926in}}%
\pgfpathlineto{\pgfqpoint{0.422008in}{0.533401in}}%
\pgfpathlineto{\pgfqpoint{0.422390in}{0.516780in}}%
\pgfpathlineto{\pgfqpoint{0.425109in}{0.500301in}}%
\pgfpathlineto{\pgfqpoint{0.430174in}{0.483940in}}%
\pgfpathlineto{\pgfqpoint{0.437579in}{0.467783in}}%
\pgfpathlineto{\pgfqpoint{0.447301in}{0.451928in}}%
\pgfpathlineto{\pgfqpoint{0.459303in}{0.436505in}}%
\pgfpathlineto{\pgfqpoint{0.473526in}{0.421897in}}%
\pgfpathlineto{\pgfqpoint{0.485587in}{0.412463in}}%
\pgfpathlineto{\pgfqpoint{0.494247in}{0.408112in}}%
\pgfpathlineto{\pgfqpoint{0.501806in}{0.406896in}}%
\pgfpathlineto{\pgfqpoint{0.508052in}{0.408490in}}%
\pgfpathlineto{\pgfqpoint{0.512838in}{0.411803in}}%
\pgfpathlineto{\pgfqpoint{0.517699in}{0.417461in}}%
\pgfpathlineto{\pgfqpoint{0.522625in}{0.425987in}}%
\pgfpathlineto{\pgfqpoint{0.529278in}{0.442740in}}%
\pgfpathlineto{\pgfqpoint{0.536010in}{0.466695in}}%
\pgfpathlineto{\pgfqpoint{0.542813in}{0.498650in}}%
\pgfpathlineto{\pgfqpoint{0.551427in}{0.549992in}}%
\pgfpathlineto{\pgfqpoint{0.563805in}{0.639179in}}%
\pgfpathlineto{\pgfqpoint{0.580735in}{0.760362in}}%
\pgfpathlineto{\pgfqpoint{0.588825in}{0.804693in}}%
\pgfpathlineto{\pgfqpoint{0.595175in}{0.829933in}}%
\pgfpathlineto{\pgfqpoint{0.601786in}{0.846449in}}%
\pgfpathlineto{\pgfqpoint{0.606339in}{0.852013in}}%
\pgfpathlineto{\pgfqpoint{0.611007in}{0.852968in}}%
\pgfpathlineto{\pgfqpoint{0.615785in}{0.849245in}}%
\pgfpathlineto{\pgfqpoint{0.620668in}{0.840910in}}%
\pgfpathlineto{\pgfqpoint{0.625648in}{0.828156in}}%
\pgfpathlineto{\pgfqpoint{0.633275in}{0.801465in}}%
\pgfpathlineto{\pgfqpoint{0.643682in}{0.754198in}}%
\pgfpathlineto{\pgfqpoint{0.659615in}{0.667627in}}%
\pgfpathlineto{\pgfqpoint{0.686415in}{0.520165in}}%
\pgfpathlineto{\pgfqpoint{0.699756in}{0.459041in}}%
\pgfpathlineto{\pgfqpoint{0.710381in}{0.419158in}}%
\pgfpathlineto{\pgfqpoint{0.720972in}{0.387435in}}%
\pgfpathlineto{\pgfqpoint{0.731544in}{0.363223in}}%
\pgfpathlineto{\pgfqpoint{0.742112in}{0.345441in}}%
\pgfpathlineto{\pgfqpoint{0.752690in}{0.332840in}}%
\pgfpathlineto{\pgfqpoint{0.763288in}{0.324200in}}%
\pgfpathlineto{\pgfqpoint{0.773912in}{0.318448in}}%
\pgfpathlineto{\pgfqpoint{0.787229in}{0.314026in}}%
\pgfpathlineto{\pgfqpoint{0.803259in}{0.311226in}}%
\pgfpathlineto{\pgfqpoint{0.827371in}{0.309687in}}%
\pgfpathlineto{\pgfqpoint{0.862195in}{0.310056in}}%
\pgfpathlineto{\pgfqpoint{0.902067in}{0.312721in}}%
\pgfpathlineto{\pgfqpoint{0.941226in}{0.317478in}}%
\pgfpathlineto{\pgfqpoint{0.979281in}{0.324253in}}%
\pgfpathlineto{\pgfqpoint{1.015859in}{0.332992in}}%
\pgfpathlineto{\pgfqpoint{1.046092in}{0.342304in}}%
\pgfpathlineto{\pgfqpoint{1.066115in}{0.350387in}}%
\pgfpathlineto{\pgfqpoint{1.081150in}{0.358857in}}%
\pgfpathlineto{\pgfqpoint{1.091608in}{0.367277in}}%
\pgfpathlineto{\pgfqpoint{1.099818in}{0.376385in}}%
\pgfpathlineto{\pgfqpoint{1.107907in}{0.388571in}}%
\pgfpathlineto{\pgfqpoint{1.115897in}{0.404938in}}%
\pgfpathlineto{\pgfqpoint{1.123814in}{0.426715in}}%
\pgfpathlineto{\pgfqpoint{1.131689in}{0.455115in}}%
\pgfpathlineto{\pgfqpoint{1.139546in}{0.491121in}}%
\pgfpathlineto{\pgfqpoint{1.149368in}{0.547488in}}%
\pgfpathlineto{\pgfqpoint{1.161133in}{0.630329in}}%
\pgfpathlineto{\pgfqpoint{1.198753in}{0.912097in}}%
\pgfpathlineto{\pgfqpoint{1.203029in}{0.927586in}}%
\pgfpathlineto{\pgfqpoint{1.205656in}{0.932225in}}%
\pgfpathlineto{\pgfqpoint{1.206901in}{0.932803in}}%
\pgfpathlineto{\pgfqpoint{1.206901in}{0.932803in}}%
\pgfusepath{stroke}%
\end{pgfscope}%
\begin{pgfscope}%
\pgftext[x=0.830946in,y=0.139368in,,]{\rmfamily\fontsize{10.000000}{12.000000}\selectfont \(\displaystyle \tau = \frac{1}{4}\)}%
\end{pgfscope}%
\end{pgfpicture}%
\makeatother%
\endgroup%

%% Creator: Matplotlib, PGF backend
%%
%% To include the figure in your LaTeX document, write
%%   \input{<filename>.pgf}
%%
%% Make sure the required packages are loaded in your preamble
%%   \usepackage{pgf}
%%
%% Figures using additional raster images can only be included by \input if
%% they are in the same directory as the main LaTeX file. For loading figures
%% from other directories you can use the `import` package
%%   \usepackage{import}
%% and then include the figures with
%%   \import{<path to file>}{<filename>.pgf}
%%
%% Matplotlib used the following preamble
%%   \usepackage{fontspec}
%%   \setmainfont{DejaVu Serif}
%%   \setsansfont{DejaVu Sans}
%%   \setmonofont{DejaVu Sans Mono}
%%
\begingroup%
\makeatletter%
\begin{pgfpicture}%
\pgfpathrectangle{\pgfpointorigin}{\pgfqpoint{1.620000in}{1.672747in}}%
\pgfusepath{use as bounding box, clip}%
\begin{pgfscope}%
\pgfsetbuttcap%
\pgfsetmiterjoin%
\definecolor{currentfill}{rgb}{1.000000,1.000000,1.000000}%
\pgfsetfillcolor{currentfill}%
\pgfsetlinewidth{0.000000pt}%
\definecolor{currentstroke}{rgb}{1.000000,1.000000,1.000000}%
\pgfsetstrokecolor{currentstroke}%
\pgfsetdash{}{0pt}%
\pgfpathmoveto{\pgfqpoint{0.000000in}{0.000000in}}%
\pgfpathlineto{\pgfqpoint{1.620000in}{0.000000in}}%
\pgfpathlineto{\pgfqpoint{1.620000in}{1.672747in}}%
\pgfpathlineto{\pgfqpoint{0.000000in}{1.672747in}}%
\pgfpathclose%
\pgfusepath{fill}%
\end{pgfscope}%
\begin{pgfscope}%
\pgfsetbuttcap%
\pgfsetmiterjoin%
\definecolor{currentfill}{rgb}{1.000000,1.000000,1.000000}%
\pgfsetfillcolor{currentfill}%
\pgfsetlinewidth{0.000000pt}%
\definecolor{currentstroke}{rgb}{0.000000,0.000000,0.000000}%
\pgfsetstrokecolor{currentstroke}%
\pgfsetstrokeopacity{0.000000}%
\pgfsetdash{}{0pt}%
\pgfpathmoveto{\pgfqpoint{0.035000in}{0.097747in}}%
\pgfpathlineto{\pgfqpoint{1.585000in}{0.097747in}}%
\pgfpathlineto{\pgfqpoint{1.585000in}{1.637747in}}%
\pgfpathlineto{\pgfqpoint{0.035000in}{1.637747in}}%
\pgfpathclose%
\pgfusepath{fill}%
\end{pgfscope}%
\begin{pgfscope}%
\pgfpathrectangle{\pgfqpoint{0.035000in}{0.097747in}}{\pgfqpoint{1.550000in}{1.540000in}} %
\pgfusepath{clip}%
\pgfsetrectcap%
\pgfsetroundjoin%
\pgfsetlinewidth{1.505625pt}%
\definecolor{currentstroke}{rgb}{0.000000,0.000000,0.000000}%
\pgfsetstrokecolor{currentstroke}%
\pgfsetdash{}{0pt}%
\pgfpathmoveto{\pgfqpoint{1.192530in}{0.425628in}}%
\pgfpathlineto{\pgfqpoint{1.206152in}{0.440853in}}%
\pgfpathlineto{\pgfqpoint{1.216687in}{0.455914in}}%
\pgfpathlineto{\pgfqpoint{1.223193in}{0.468841in}}%
\pgfpathlineto{\pgfqpoint{1.228141in}{0.483733in}}%
\pgfpathlineto{\pgfqpoint{1.231905in}{0.501848in}}%
\pgfpathlineto{\pgfqpoint{1.235872in}{0.533841in}}%
\pgfpathlineto{\pgfqpoint{1.239557in}{0.583485in}}%
\pgfpathlineto{\pgfqpoint{1.244097in}{0.680705in}}%
\pgfpathlineto{\pgfqpoint{1.249093in}{0.842236in}}%
\pgfpathlineto{\pgfqpoint{1.251787in}{0.993673in}}%
\pgfpathlineto{\pgfqpoint{1.251115in}{1.044093in}}%
\pgfpathlineto{\pgfqpoint{1.250115in}{1.049586in}}%
\pgfpathlineto{\pgfqpoint{1.249684in}{1.049158in}}%
\pgfpathlineto{\pgfqpoint{1.248105in}{1.041273in}}%
\pgfpathlineto{\pgfqpoint{1.244592in}{1.008211in}}%
\pgfpathlineto{\pgfqpoint{1.218248in}{0.725168in}}%
\pgfpathlineto{\pgfqpoint{1.211385in}{0.683816in}}%
\pgfpathlineto{\pgfqpoint{1.205634in}{0.661791in}}%
\pgfpathlineto{\pgfqpoint{1.200974in}{0.650975in}}%
\pgfpathlineto{\pgfqpoint{1.196226in}{0.644909in}}%
\pgfpathlineto{\pgfqpoint{1.191360in}{0.642419in}}%
\pgfpathlineto{\pgfqpoint{1.186351in}{0.642477in}}%
\pgfpathlineto{\pgfqpoint{1.179857in}{0.644887in}}%
\pgfpathlineto{\pgfqpoint{1.167455in}{0.652555in}}%
\pgfpathlineto{\pgfqpoint{1.131456in}{0.675771in}}%
\pgfpathlineto{\pgfqpoint{1.104842in}{0.689917in}}%
\pgfpathlineto{\pgfqpoint{1.066014in}{0.710020in}}%
\pgfpathlineto{\pgfqpoint{1.053742in}{0.719322in}}%
\pgfpathlineto{\pgfqpoint{1.043345in}{0.730483in}}%
\pgfpathlineto{\pgfqpoint{1.034935in}{0.742922in}}%
\pgfpathlineto{\pgfqpoint{1.026459in}{0.759682in}}%
\pgfpathlineto{\pgfqpoint{1.017928in}{0.781940in}}%
\pgfpathlineto{\pgfqpoint{1.009351in}{0.810766in}}%
\pgfpathlineto{\pgfqpoint{1.000728in}{0.846889in}}%
\pgfpathlineto{\pgfqpoint{0.989871in}{0.902403in}}%
\pgfpathlineto{\pgfqpoint{0.974453in}{0.995624in}}%
\pgfpathlineto{\pgfqpoint{0.954015in}{1.120146in}}%
\pgfpathlineto{\pgfqpoint{0.944629in}{1.166037in}}%
\pgfpathlineto{\pgfqpoint{0.937454in}{1.192762in}}%
\pgfpathlineto{\pgfqpoint{0.930165in}{1.211173in}}%
\pgfpathlineto{\pgfqpoint{0.925245in}{1.218274in}}%
\pgfpathlineto{\pgfqpoint{0.920282in}{1.221007in}}%
\pgfpathlineto{\pgfqpoint{0.915280in}{1.219312in}}%
\pgfpathlineto{\pgfqpoint{0.910244in}{1.213252in}}%
\pgfpathlineto{\pgfqpoint{0.905180in}{1.203010in}}%
\pgfpathlineto{\pgfqpoint{0.897548in}{1.180484in}}%
\pgfpathlineto{\pgfqpoint{0.889889in}{1.150634in}}%
\pgfpathlineto{\pgfqpoint{0.877118in}{1.089417in}}%
\pgfpathlineto{\pgfqpoint{0.839245in}{0.896219in}}%
\pgfpathlineto{\pgfqpoint{0.826810in}{0.848048in}}%
\pgfpathlineto{\pgfqpoint{0.816916in}{0.817868in}}%
\pgfpathlineto{\pgfqpoint{0.807060in}{0.794662in}}%
\pgfpathlineto{\pgfqpoint{0.797233in}{0.777495in}}%
\pgfpathlineto{\pgfqpoint{0.787434in}{0.765217in}}%
\pgfpathlineto{\pgfqpoint{0.777660in}{0.756661in}}%
\pgfpathlineto{\pgfqpoint{0.767915in}{0.750783in}}%
\pgfpathlineto{\pgfqpoint{0.755779in}{0.745908in}}%
\pgfpathlineto{\pgfqpoint{0.738892in}{0.741599in}}%
\pgfpathlineto{\pgfqpoint{0.630913in}{0.718550in}}%
\pgfpathlineto{\pgfqpoint{0.607543in}{0.713516in}}%
\pgfpathlineto{\pgfqpoint{0.597192in}{0.713121in}}%
\pgfpathlineto{\pgfqpoint{0.589023in}{0.714734in}}%
\pgfpathlineto{\pgfqpoint{0.582951in}{0.717679in}}%
\pgfpathlineto{\pgfqpoint{0.576916in}{0.722674in}}%
\pgfpathlineto{\pgfqpoint{0.570908in}{0.730327in}}%
\pgfpathlineto{\pgfqpoint{0.564911in}{0.741322in}}%
\pgfpathlineto{\pgfqpoint{0.558911in}{0.756389in}}%
\pgfpathlineto{\pgfqpoint{0.550875in}{0.784065in}}%
\pgfpathlineto{\pgfqpoint{0.542765in}{0.821775in}}%
\pgfpathlineto{\pgfqpoint{0.534551in}{0.870458in}}%
\pgfpathlineto{\pgfqpoint{0.524125in}{0.946564in}}%
\pgfpathlineto{\pgfqpoint{0.509349in}{1.074938in}}%
\pgfpathlineto{\pgfqpoint{0.486913in}{1.273870in}}%
\pgfpathlineto{\pgfqpoint{0.479536in}{1.323467in}}%
\pgfpathlineto{\pgfqpoint{0.474444in}{1.346630in}}%
\pgfpathlineto{\pgfqpoint{0.471292in}{1.354392in}}%
\pgfpathlineto{\pgfqpoint{0.469792in}{1.355845in}}%
\pgfpathlineto{\pgfqpoint{0.468346in}{1.355650in}}%
\pgfpathlineto{\pgfqpoint{0.465613in}{1.350303in}}%
\pgfpathlineto{\pgfqpoint{0.463097in}{1.338439in}}%
\pgfpathlineto{\pgfqpoint{0.459720in}{1.309075in}}%
\pgfpathlineto{\pgfqpoint{0.455909in}{1.251069in}}%
\pgfpathlineto{\pgfqpoint{0.452078in}{1.156465in}}%
\pgfpathlineto{\pgfqpoint{0.446329in}{0.943126in}}%
\pgfpathlineto{\pgfqpoint{0.439807in}{0.725216in}}%
\pgfpathlineto{\pgfqpoint{0.435122in}{0.639300in}}%
\pgfpathlineto{\pgfqpoint{0.430610in}{0.592680in}}%
\pgfpathlineto{\pgfqpoint{0.424735in}{0.555924in}}%
\pgfpathlineto{\pgfqpoint{0.422008in}{0.533402in}}%
\pgfpathlineto{\pgfqpoint{0.422390in}{0.516786in}}%
\pgfpathlineto{\pgfqpoint{0.425107in}{0.500362in}}%
\pgfpathlineto{\pgfqpoint{0.429649in}{0.485764in}}%
\pgfpathlineto{\pgfqpoint{0.435441in}{0.474219in}}%
\pgfpathlineto{\pgfqpoint{0.439630in}{0.469632in}}%
\pgfpathlineto{\pgfqpoint{0.442692in}{0.468553in}}%
\pgfpathlineto{\pgfqpoint{0.445943in}{0.469955in}}%
\pgfpathlineto{\pgfqpoint{0.449359in}{0.474841in}}%
\pgfpathlineto{\pgfqpoint{0.452908in}{0.484384in}}%
\pgfpathlineto{\pgfqpoint{0.457487in}{0.504744in}}%
\pgfpathlineto{\pgfqpoint{0.462181in}{0.536431in}}%
\pgfpathlineto{\pgfqpoint{0.467945in}{0.590827in}}%
\pgfpathlineto{\pgfqpoint{0.479220in}{0.726824in}}%
\pgfpathlineto{\pgfqpoint{0.487480in}{0.814705in}}%
\pgfpathlineto{\pgfqpoint{0.492839in}{0.854270in}}%
\pgfpathlineto{\pgfqpoint{0.497235in}{0.875382in}}%
\pgfpathlineto{\pgfqpoint{0.501984in}{0.887480in}}%
\pgfpathlineto{\pgfqpoint{0.505351in}{0.890039in}}%
\pgfpathlineto{\pgfqpoint{0.508879in}{0.888017in}}%
\pgfpathlineto{\pgfqpoint{0.512564in}{0.881413in}}%
\pgfpathlineto{\pgfqpoint{0.518370in}{0.863224in}}%
\pgfpathlineto{\pgfqpoint{0.524481in}{0.835978in}}%
\pgfpathlineto{\pgfqpoint{0.535220in}{0.774582in}}%
\pgfpathlineto{\pgfqpoint{0.578877in}{0.501434in}}%
\pgfpathlineto{\pgfqpoint{0.590487in}{0.449070in}}%
\pgfpathlineto{\pgfqpoint{0.599765in}{0.416180in}}%
\pgfpathlineto{\pgfqpoint{0.609051in}{0.390651in}}%
\pgfpathlineto{\pgfqpoint{0.618370in}{0.371425in}}%
\pgfpathlineto{\pgfqpoint{0.627745in}{0.357278in}}%
\pgfpathlineto{\pgfqpoint{0.637196in}{0.347012in}}%
\pgfpathlineto{\pgfqpoint{0.646737in}{0.339573in}}%
\pgfpathlineto{\pgfqpoint{0.658805in}{0.332980in}}%
\pgfpathlineto{\pgfqpoint{0.673504in}{0.327510in}}%
\pgfpathlineto{\pgfqpoint{0.695989in}{0.321884in}}%
\pgfpathlineto{\pgfqpoint{0.729286in}{0.316254in}}%
\pgfpathlineto{\pgfqpoint{0.768637in}{0.311991in}}%
\pgfpathlineto{\pgfqpoint{0.800587in}{0.310556in}}%
\pgfpathlineto{\pgfqpoint{0.822008in}{0.311556in}}%
\pgfpathlineto{\pgfqpoint{0.838098in}{0.314593in}}%
\pgfpathlineto{\pgfqpoint{0.848826in}{0.318572in}}%
\pgfpathlineto{\pgfqpoint{0.859552in}{0.324956in}}%
\pgfpathlineto{\pgfqpoint{0.867595in}{0.331893in}}%
\pgfpathlineto{\pgfqpoint{0.875639in}{0.341195in}}%
\pgfpathlineto{\pgfqpoint{0.883685in}{0.353381in}}%
\pgfpathlineto{\pgfqpoint{0.891738in}{0.368979in}}%
\pgfpathlineto{\pgfqpoint{0.899803in}{0.388483in}}%
\pgfpathlineto{\pgfqpoint{0.910583in}{0.421241in}}%
\pgfpathlineto{\pgfqpoint{0.921407in}{0.462135in}}%
\pgfpathlineto{\pgfqpoint{0.935008in}{0.524144in}}%
\pgfpathlineto{\pgfqpoint{0.954163in}{0.625299in}}%
\pgfpathlineto{\pgfqpoint{0.976020in}{0.740074in}}%
\pgfpathlineto{\pgfqpoint{0.986794in}{0.786293in}}%
\pgfpathlineto{\pgfqpoint{0.994744in}{0.812308in}}%
\pgfpathlineto{\pgfqpoint{0.999964in}{0.824672in}}%
\pgfpathlineto{\pgfqpoint{1.005108in}{0.832672in}}%
\pgfpathlineto{\pgfqpoint{1.010169in}{0.836114in}}%
\pgfpathlineto{\pgfqpoint{1.012666in}{0.836101in}}%
\pgfpathlineto{\pgfqpoint{1.017590in}{0.832627in}}%
\pgfpathlineto{\pgfqpoint{1.022417in}{0.824683in}}%
\pgfpathlineto{\pgfqpoint{1.027147in}{0.812526in}}%
\pgfpathlineto{\pgfqpoint{1.034058in}{0.787229in}}%
\pgfpathlineto{\pgfqpoint{1.042945in}{0.742943in}}%
\pgfpathlineto{\pgfqpoint{1.055666in}{0.663417in}}%
\pgfpathlineto{\pgfqpoint{1.077698in}{0.521903in}}%
\pgfpathlineto{\pgfqpoint{1.087396in}{0.473036in}}%
\pgfpathlineto{\pgfqpoint{1.096973in}{0.436784in}}%
\pgfpathlineto{\pgfqpoint{1.104551in}{0.416384in}}%
\pgfpathlineto{\pgfqpoint{1.112042in}{0.402580in}}%
\pgfpathlineto{\pgfqpoint{1.119429in}{0.394151in}}%
\pgfpathlineto{\pgfqpoint{1.126689in}{0.389868in}}%
\pgfpathlineto{\pgfqpoint{1.133801in}{0.388624in}}%
\pgfpathlineto{\pgfqpoint{1.142452in}{0.389981in}}%
\pgfpathlineto{\pgfqpoint{1.152441in}{0.394215in}}%
\pgfpathlineto{\pgfqpoint{1.165031in}{0.402109in}}%
\pgfpathlineto{\pgfqpoint{1.180881in}{0.414689in}}%
\pgfpathlineto{\pgfqpoint{1.193749in}{0.426866in}}%
\pgfpathlineto{\pgfqpoint{1.193749in}{0.426866in}}%
\pgfusepath{stroke}%
\end{pgfscope}%
\begin{pgfscope}%
\pgftext[x=0.830946in,y=0.139368in,,]{\rmfamily\fontsize{10.000000}{12.000000}\selectfont \(\displaystyle \tau = \frac{2}{5}\)}%
\end{pgfscope}%
\end{pgfpicture}%
\makeatother%
\endgroup%

%% Creator: Matplotlib, PGF backend
%%
%% To include the figure in your LaTeX document, write
%%   \input{<filename>.pgf}
%%
%% Make sure the required packages are loaded in your preamble
%%   \usepackage{pgf}
%%
%% Figures using additional raster images can only be included by \input if
%% they are in the same directory as the main LaTeX file. For loading figures
%% from other directories you can use the `import` package
%%   \usepackage{import}
%% and then include the figures with
%%   \import{<path to file>}{<filename>.pgf}
%%
%% Matplotlib used the following preamble
%%   \usepackage{fontspec}
%%   \setmainfont{DejaVu Serif}
%%   \setsansfont{DejaVu Sans}
%%   \setmonofont{DejaVu Sans Mono}
%%
\begingroup%
\makeatletter%
\begin{pgfpicture}%
\pgfpathrectangle{\pgfpointorigin}{\pgfqpoint{1.620000in}{1.672747in}}%
\pgfusepath{use as bounding box, clip}%
\begin{pgfscope}%
\pgfsetbuttcap%
\pgfsetmiterjoin%
\definecolor{currentfill}{rgb}{1.000000,1.000000,1.000000}%
\pgfsetfillcolor{currentfill}%
\pgfsetlinewidth{0.000000pt}%
\definecolor{currentstroke}{rgb}{1.000000,1.000000,1.000000}%
\pgfsetstrokecolor{currentstroke}%
\pgfsetdash{}{0pt}%
\pgfpathmoveto{\pgfqpoint{0.000000in}{0.000000in}}%
\pgfpathlineto{\pgfqpoint{1.620000in}{0.000000in}}%
\pgfpathlineto{\pgfqpoint{1.620000in}{1.672747in}}%
\pgfpathlineto{\pgfqpoint{0.000000in}{1.672747in}}%
\pgfpathclose%
\pgfusepath{fill}%
\end{pgfscope}%
\begin{pgfscope}%
\pgfsetbuttcap%
\pgfsetmiterjoin%
\definecolor{currentfill}{rgb}{1.000000,1.000000,1.000000}%
\pgfsetfillcolor{currentfill}%
\pgfsetlinewidth{0.000000pt}%
\definecolor{currentstroke}{rgb}{0.000000,0.000000,0.000000}%
\pgfsetstrokecolor{currentstroke}%
\pgfsetstrokeopacity{0.000000}%
\pgfsetdash{}{0pt}%
\pgfpathmoveto{\pgfqpoint{0.035000in}{0.097747in}}%
\pgfpathlineto{\pgfqpoint{1.585000in}{0.097747in}}%
\pgfpathlineto{\pgfqpoint{1.585000in}{1.637747in}}%
\pgfpathlineto{\pgfqpoint{0.035000in}{1.637747in}}%
\pgfpathclose%
\pgfusepath{fill}%
\end{pgfscope}%
\begin{pgfscope}%
\pgfpathrectangle{\pgfqpoint{0.035000in}{0.097747in}}{\pgfqpoint{1.550000in}{1.540000in}} %
\pgfusepath{clip}%
\pgfsetrectcap%
\pgfsetroundjoin%
\pgfsetlinewidth{1.505625pt}%
\definecolor{currentstroke}{rgb}{0.000000,0.000000,0.000000}%
\pgfsetstrokecolor{currentstroke}%
\pgfsetdash{}{0pt}%
\pgfpathmoveto{\pgfqpoint{1.211405in}{1.154089in}}%
\pgfpathlineto{\pgfqpoint{1.213680in}{1.146198in}}%
\pgfpathlineto{\pgfqpoint{1.216654in}{1.122485in}}%
\pgfpathlineto{\pgfqpoint{1.219858in}{1.071615in}}%
\pgfpathlineto{\pgfqpoint{1.223336in}{0.966665in}}%
\pgfpathlineto{\pgfqpoint{1.227725in}{0.737691in}}%
\pgfpathlineto{\pgfqpoint{1.231566in}{0.583873in}}%
\pgfpathlineto{\pgfqpoint{1.234923in}{0.528289in}}%
\pgfpathlineto{\pgfqpoint{1.237567in}{0.514759in}}%
\pgfpathlineto{\pgfqpoint{1.238278in}{0.514616in}}%
\pgfpathlineto{\pgfqpoint{1.238492in}{0.514884in}}%
\pgfpathlineto{\pgfqpoint{1.239498in}{0.518823in}}%
\pgfpathlineto{\pgfqpoint{1.239980in}{0.531425in}}%
\pgfpathlineto{\pgfqpoint{1.238346in}{0.547412in}}%
\pgfpathlineto{\pgfqpoint{1.234430in}{0.563615in}}%
\pgfpathlineto{\pgfqpoint{1.228266in}{0.579606in}}%
\pgfpathlineto{\pgfqpoint{1.219909in}{0.595261in}}%
\pgfpathlineto{\pgfqpoint{1.208455in}{0.611743in}}%
\pgfpathlineto{\pgfqpoint{1.194602in}{0.627630in}}%
\pgfpathlineto{\pgfqpoint{1.178458in}{0.642827in}}%
\pgfpathlineto{\pgfqpoint{1.158650in}{0.658319in}}%
\pgfpathlineto{\pgfqpoint{1.136492in}{0.672806in}}%
\pgfpathlineto{\pgfqpoint{1.110347in}{0.687104in}}%
\pgfpathlineto{\pgfqpoint{1.081942in}{0.700035in}}%
\pgfpathlineto{\pgfqpoint{1.049431in}{0.712274in}}%
\pgfpathlineto{\pgfqpoint{0.999352in}{0.730347in}}%
\pgfpathlineto{\pgfqpoint{0.988049in}{0.737003in}}%
\pgfpathlineto{\pgfqpoint{0.978929in}{0.744604in}}%
\pgfpathlineto{\pgfqpoint{0.969749in}{0.755370in}}%
\pgfpathlineto{\pgfqpoint{0.960523in}{0.770664in}}%
\pgfpathlineto{\pgfqpoint{0.953578in}{0.786065in}}%
\pgfpathlineto{\pgfqpoint{0.944290in}{0.813221in}}%
\pgfpathlineto{\pgfqpoint{0.934971in}{0.849455in}}%
\pgfpathlineto{\pgfqpoint{0.925616in}{0.896075in}}%
\pgfpathlineto{\pgfqpoint{0.916213in}{0.953658in}}%
\pgfpathlineto{\pgfqpoint{0.904359in}{1.040065in}}%
\pgfpathlineto{\pgfqpoint{0.862624in}{1.364532in}}%
\pgfpathlineto{\pgfqpoint{0.854991in}{1.404056in}}%
\pgfpathlineto{\pgfqpoint{0.847296in}{1.431589in}}%
\pgfpathlineto{\pgfqpoint{0.842142in}{1.442400in}}%
\pgfpathlineto{\pgfqpoint{0.836977in}{1.446791in}}%
\pgfpathlineto{\pgfqpoint{0.834392in}{1.446532in}}%
\pgfpathlineto{\pgfqpoint{0.831807in}{1.444635in}}%
\pgfpathlineto{\pgfqpoint{0.826641in}{1.435996in}}%
\pgfpathlineto{\pgfqpoint{0.821485in}{1.421123in}}%
\pgfpathlineto{\pgfqpoint{0.813786in}{1.388091in}}%
\pgfpathlineto{\pgfqpoint{0.806144in}{1.344057in}}%
\pgfpathlineto{\pgfqpoint{0.796067in}{1.272724in}}%
\pgfpathlineto{\pgfqpoint{0.773906in}{1.092136in}}%
\pgfpathlineto{\pgfqpoint{0.757088in}{0.965270in}}%
\pgfpathlineto{\pgfqpoint{0.745223in}{0.892903in}}%
\pgfpathlineto{\pgfqpoint{0.735789in}{0.847312in}}%
\pgfpathlineto{\pgfqpoint{0.726391in}{0.812020in}}%
\pgfpathlineto{\pgfqpoint{0.717026in}{0.785675in}}%
\pgfpathlineto{\pgfqpoint{0.707693in}{0.766607in}}%
\pgfpathlineto{\pgfqpoint{0.698399in}{0.753114in}}%
\pgfpathlineto{\pgfqpoint{0.689153in}{0.743662in}}%
\pgfpathlineto{\pgfqpoint{0.677680in}{0.735627in}}%
\pgfpathlineto{\pgfqpoint{0.664063in}{0.729195in}}%
\pgfpathlineto{\pgfqpoint{0.641807in}{0.721627in}}%
\pgfpathlineto{\pgfqpoint{0.595261in}{0.706087in}}%
\pgfpathlineto{\pgfqpoint{0.563904in}{0.693026in}}%
\pgfpathlineto{\pgfqpoint{0.536725in}{0.679326in}}%
\pgfpathlineto{\pgfqpoint{0.511937in}{0.664315in}}%
\pgfpathlineto{\pgfqpoint{0.491158in}{0.649220in}}%
\pgfpathlineto{\pgfqpoint{0.472828in}{0.633176in}}%
\pgfpathlineto{\pgfqpoint{0.458139in}{0.617524in}}%
\pgfpathlineto{\pgfqpoint{0.445812in}{0.601242in}}%
\pgfpathlineto{\pgfqpoint{0.435949in}{0.584430in}}%
\pgfpathlineto{\pgfqpoint{0.429109in}{0.568533in}}%
\pgfpathlineto{\pgfqpoint{0.424501in}{0.552385in}}%
\pgfpathlineto{\pgfqpoint{0.422166in}{0.536330in}}%
\pgfpathlineto{\pgfqpoint{0.422179in}{0.521271in}}%
\pgfpathlineto{\pgfqpoint{0.423399in}{0.517231in}}%
\pgfpathlineto{\pgfqpoint{0.423607in}{0.517329in}}%
\pgfpathlineto{\pgfqpoint{0.424545in}{0.519775in}}%
\pgfpathlineto{\pgfqpoint{0.426199in}{0.532030in}}%
\pgfpathlineto{\pgfqpoint{0.428316in}{0.566101in}}%
\pgfpathlineto{\pgfqpoint{0.430699in}{0.642115in}}%
\pgfpathlineto{\pgfqpoint{0.434417in}{0.857421in}}%
\pgfpathlineto{\pgfqpoint{0.438176in}{1.030515in}}%
\pgfpathlineto{\pgfqpoint{0.441487in}{1.106022in}}%
\pgfpathlineto{\pgfqpoint{0.445054in}{1.144396in}}%
\pgfpathlineto{\pgfqpoint{0.448339in}{1.157005in}}%
\pgfpathlineto{\pgfqpoint{0.449555in}{1.157872in}}%
\pgfpathlineto{\pgfqpoint{0.450831in}{1.157035in}}%
\pgfpathlineto{\pgfqpoint{0.453563in}{1.150251in}}%
\pgfpathlineto{\pgfqpoint{0.458095in}{1.127628in}}%
\pgfpathlineto{\pgfqpoint{0.464871in}{1.076368in}}%
\pgfpathlineto{\pgfqpoint{0.476270in}{0.965335in}}%
\pgfpathlineto{\pgfqpoint{0.507495in}{0.645959in}}%
\pgfpathlineto{\pgfqpoint{0.519958in}{0.546744in}}%
\pgfpathlineto{\pgfqpoint{0.530182in}{0.483991in}}%
\pgfpathlineto{\pgfqpoint{0.538288in}{0.446031in}}%
\pgfpathlineto{\pgfqpoint{0.546372in}{0.417378in}}%
\pgfpathlineto{\pgfqpoint{0.554479in}{0.396264in}}%
\pgfpathlineto{\pgfqpoint{0.562648in}{0.380930in}}%
\pgfpathlineto{\pgfqpoint{0.570914in}{0.369817in}}%
\pgfpathlineto{\pgfqpoint{0.581419in}{0.359951in}}%
\pgfpathlineto{\pgfqpoint{0.592141in}{0.352894in}}%
\pgfpathlineto{\pgfqpoint{0.607546in}{0.345585in}}%
\pgfpathlineto{\pgfqpoint{0.630344in}{0.337564in}}%
\pgfpathlineto{\pgfqpoint{0.661273in}{0.329066in}}%
\pgfpathlineto{\pgfqpoint{0.698520in}{0.321151in}}%
\pgfpathlineto{\pgfqpoint{0.737088in}{0.315219in}}%
\pgfpathlineto{\pgfqpoint{0.776596in}{0.311331in}}%
\pgfpathlineto{\pgfqpoint{0.816649in}{0.309529in}}%
\pgfpathlineto{\pgfqpoint{0.856846in}{0.309834in}}%
\pgfpathlineto{\pgfqpoint{0.896783in}{0.312242in}}%
\pgfpathlineto{\pgfqpoint{0.936060in}{0.316726in}}%
\pgfpathlineto{\pgfqpoint{0.974285in}{0.323236in}}%
\pgfpathlineto{\pgfqpoint{1.011080in}{0.331719in}}%
\pgfpathlineto{\pgfqpoint{1.041539in}{0.340819in}}%
\pgfpathlineto{\pgfqpoint{1.061739in}{0.348801in}}%
\pgfpathlineto{\pgfqpoint{1.074783in}{0.355926in}}%
\pgfpathlineto{\pgfqpoint{1.085401in}{0.364112in}}%
\pgfpathlineto{\pgfqpoint{1.093745in}{0.373137in}}%
\pgfpathlineto{\pgfqpoint{1.101976in}{0.385501in}}%
\pgfpathlineto{\pgfqpoint{1.110120in}{0.402575in}}%
\pgfpathlineto{\pgfqpoint{1.118215in}{0.426002in}}%
\pgfpathlineto{\pgfqpoint{1.126301in}{0.457568in}}%
\pgfpathlineto{\pgfqpoint{1.134424in}{0.498986in}}%
\pgfpathlineto{\pgfqpoint{1.142621in}{0.551572in}}%
\pgfpathlineto{\pgfqpoint{1.153002in}{0.633689in}}%
\pgfpathlineto{\pgfqpoint{1.165616in}{0.753446in}}%
\pgfpathlineto{\pgfqpoint{1.201068in}{1.108992in}}%
\pgfpathlineto{\pgfqpoint{1.207365in}{1.146491in}}%
\pgfpathlineto{\pgfqpoint{1.210177in}{1.155529in}}%
\pgfpathlineto{\pgfqpoint{1.212749in}{1.157772in}}%
\pgfpathlineto{\pgfqpoint{1.212749in}{1.157772in}}%
\pgfusepath{stroke}%
\end{pgfscope}%
\begin{pgfscope}%
\pgftext[x=0.830946in,y=0.139368in,,]{\rmfamily\fontsize{10.000000}{12.000000}\selectfont \(\displaystyle \tau = \frac{1}{3}\)}%
\end{pgfscope}%
\end{pgfpicture}%
\makeatother%
\endgroup%

%% Creator: Matplotlib, PGF backend
%%
%% To include the figure in your LaTeX document, write
%%   \input{<filename>.pgf}
%%
%% Make sure the required packages are loaded in your preamble
%%   \usepackage{pgf}
%%
%% Figures using additional raster images can only be included by \input if
%% they are in the same directory as the main LaTeX file. For loading figures
%% from other directories you can use the `import` package
%%   \usepackage{import}
%% and then include the figures with
%%   \import{<path to file>}{<filename>.pgf}
%%
%% Matplotlib used the following preamble
%%   \usepackage{fontspec}
%%   \setmainfont{DejaVu Serif}
%%   \setsansfont{DejaVu Sans}
%%   \setmonofont{DejaVu Sans Mono}
%%
\begingroup%
\makeatletter%
\begin{pgfpicture}%
\pgfpathrectangle{\pgfpointorigin}{\pgfqpoint{1.610000in}{1.672747in}}%
\pgfusepath{use as bounding box, clip}%
\begin{pgfscope}%
\pgfsetbuttcap%
\pgfsetmiterjoin%
\definecolor{currentfill}{rgb}{1.000000,1.000000,1.000000}%
\pgfsetfillcolor{currentfill}%
\pgfsetlinewidth{0.000000pt}%
\definecolor{currentstroke}{rgb}{1.000000,1.000000,1.000000}%
\pgfsetstrokecolor{currentstroke}%
\pgfsetdash{}{0pt}%
\pgfpathmoveto{\pgfqpoint{0.000000in}{0.000000in}}%
\pgfpathlineto{\pgfqpoint{1.610000in}{0.000000in}}%
\pgfpathlineto{\pgfqpoint{1.610000in}{1.672747in}}%
\pgfpathlineto{\pgfqpoint{0.000000in}{1.672747in}}%
\pgfpathclose%
\pgfusepath{fill}%
\end{pgfscope}%
\begin{pgfscope}%
\pgfsetbuttcap%
\pgfsetmiterjoin%
\definecolor{currentfill}{rgb}{1.000000,1.000000,1.000000}%
\pgfsetfillcolor{currentfill}%
\pgfsetlinewidth{0.000000pt}%
\definecolor{currentstroke}{rgb}{0.000000,0.000000,0.000000}%
\pgfsetstrokecolor{currentstroke}%
\pgfsetstrokeopacity{0.000000}%
\pgfsetdash{}{0pt}%
\pgfpathmoveto{\pgfqpoint{0.035000in}{0.097747in}}%
\pgfpathlineto{\pgfqpoint{1.575000in}{0.097747in}}%
\pgfpathlineto{\pgfqpoint{1.575000in}{1.637747in}}%
\pgfpathlineto{\pgfqpoint{0.035000in}{1.637747in}}%
\pgfpathclose%
\pgfusepath{fill}%
\end{pgfscope}%
\begin{pgfscope}%
\pgfpathrectangle{\pgfqpoint{0.035000in}{0.097747in}}{\pgfqpoint{1.540000in}{1.540000in}} %
\pgfusepath{clip}%
\pgfsetrectcap%
\pgfsetroundjoin%
\pgfsetlinewidth{1.505625pt}%
\definecolor{currentstroke}{rgb}{0.000000,0.000000,0.000000}%
\pgfsetstrokecolor{currentstroke}%
\pgfsetdash{}{0pt}%
\pgfpathmoveto{\pgfqpoint{1.203707in}{1.149881in}}%
\pgfpathlineto{\pgfqpoint{1.204954in}{1.150160in}}%
\pgfpathlineto{\pgfqpoint{1.206142in}{1.148757in}}%
\pgfpathlineto{\pgfqpoint{1.208340in}{1.140996in}}%
\pgfpathlineto{\pgfqpoint{1.211200in}{1.117611in}}%
\pgfpathlineto{\pgfqpoint{1.214260in}{1.067391in}}%
\pgfpathlineto{\pgfqpoint{1.217537in}{0.963722in}}%
\pgfpathlineto{\pgfqpoint{1.221547in}{0.737451in}}%
\pgfpathlineto{\pgfqpoint{1.225310in}{0.577352in}}%
\pgfpathlineto{\pgfqpoint{1.228321in}{0.527853in}}%
\pgfpathlineto{\pgfqpoint{1.230589in}{0.517166in}}%
\pgfpathlineto{\pgfqpoint{1.231332in}{0.517965in}}%
\pgfpathlineto{\pgfqpoint{1.232186in}{0.524545in}}%
\pgfpathlineto{\pgfqpoint{1.231680in}{0.539397in}}%
\pgfpathlineto{\pgfqpoint{1.228919in}{0.555534in}}%
\pgfpathlineto{\pgfqpoint{1.223906in}{0.571647in}}%
\pgfpathlineto{\pgfqpoint{1.216685in}{0.587481in}}%
\pgfpathlineto{\pgfqpoint{1.207315in}{0.602936in}}%
\pgfpathlineto{\pgfqpoint{1.194822in}{0.619155in}}%
\pgfpathlineto{\pgfqpoint{1.179993in}{0.634735in}}%
\pgfpathlineto{\pgfqpoint{1.161543in}{0.650693in}}%
\pgfpathlineto{\pgfqpoint{1.140674in}{0.665695in}}%
\pgfpathlineto{\pgfqpoint{1.117558in}{0.679642in}}%
\pgfpathlineto{\pgfqpoint{1.090504in}{0.693314in}}%
\pgfpathlineto{\pgfqpoint{1.061318in}{0.705571in}}%
\pgfpathlineto{\pgfqpoint{1.028116in}{0.716989in}}%
\pgfpathlineto{\pgfqpoint{0.993086in}{0.726607in}}%
\pgfpathlineto{\pgfqpoint{0.956554in}{0.734349in}}%
\pgfpathlineto{\pgfqpoint{0.918854in}{0.740155in}}%
\pgfpathlineto{\pgfqpoint{0.877897in}{0.744152in}}%
\pgfpathlineto{\pgfqpoint{0.836418in}{0.745878in}}%
\pgfpathlineto{\pgfqpoint{0.794832in}{0.745317in}}%
\pgfpathlineto{\pgfqpoint{0.753557in}{0.742475in}}%
\pgfpathlineto{\pgfqpoint{0.713007in}{0.737376in}}%
\pgfpathlineto{\pgfqpoint{0.673593in}{0.730070in}}%
\pgfpathlineto{\pgfqpoint{0.633545in}{0.720201in}}%
\pgfpathlineto{\pgfqpoint{0.601789in}{0.712688in}}%
\pgfpathlineto{\pgfqpoint{0.591547in}{0.712166in}}%
\pgfpathlineto{\pgfqpoint{0.583465in}{0.713634in}}%
\pgfpathlineto{\pgfqpoint{0.577460in}{0.716435in}}%
\pgfpathlineto{\pgfqpoint{0.571493in}{0.721250in}}%
\pgfpathlineto{\pgfqpoint{0.565553in}{0.728682in}}%
\pgfpathlineto{\pgfqpoint{0.559627in}{0.739409in}}%
\pgfpathlineto{\pgfqpoint{0.553698in}{0.754158in}}%
\pgfpathlineto{\pgfqpoint{0.545760in}{0.781330in}}%
\pgfpathlineto{\pgfqpoint{0.537749in}{0.818450in}}%
\pgfpathlineto{\pgfqpoint{0.529638in}{0.866469in}}%
\pgfpathlineto{\pgfqpoint{0.519345in}{0.941678in}}%
\pgfpathlineto{\pgfqpoint{0.504758in}{1.068781in}}%
\pgfpathlineto{\pgfqpoint{0.482618in}{1.266103in}}%
\pgfpathlineto{\pgfqpoint{0.475345in}{1.315333in}}%
\pgfpathlineto{\pgfqpoint{0.470329in}{1.338317in}}%
\pgfpathlineto{\pgfqpoint{0.467224in}{1.346009in}}%
\pgfpathlineto{\pgfqpoint{0.465748in}{1.347442in}}%
\pgfpathlineto{\pgfqpoint{0.464325in}{1.347237in}}%
\pgfpathlineto{\pgfqpoint{0.461639in}{1.341898in}}%
\pgfpathlineto{\pgfqpoint{0.459166in}{1.330083in}}%
\pgfpathlineto{\pgfqpoint{0.455852in}{1.300865in}}%
\pgfpathlineto{\pgfqpoint{0.452119in}{1.243182in}}%
\pgfpathlineto{\pgfqpoint{0.447728in}{1.128490in}}%
\pgfpathlineto{\pgfqpoint{0.440841in}{0.859731in}}%
\pgfpathlineto{\pgfqpoint{0.435449in}{0.698522in}}%
\pgfpathlineto{\pgfqpoint{0.430902in}{0.623935in}}%
\pgfpathlineto{\pgfqpoint{0.426215in}{0.580935in}}%
\pgfpathlineto{\pgfqpoint{0.419381in}{0.527839in}}%
\pgfpathlineto{\pgfqpoint{0.420532in}{0.511278in}}%
\pgfpathlineto{\pgfqpoint{0.424010in}{0.494831in}}%
\pgfpathlineto{\pgfqpoint{0.429819in}{0.478527in}}%
\pgfpathlineto{\pgfqpoint{0.437947in}{0.462459in}}%
\pgfpathlineto{\pgfqpoint{0.448367in}{0.446723in}}%
\pgfpathlineto{\pgfqpoint{0.461037in}{0.431416in}}%
\pgfpathlineto{\pgfqpoint{0.477236in}{0.415430in}}%
\pgfpathlineto{\pgfqpoint{0.495918in}{0.400180in}}%
\pgfpathlineto{\pgfqpoint{0.516968in}{0.385784in}}%
\pgfpathlineto{\pgfqpoint{0.542134in}{0.371361in}}%
\pgfpathlineto{\pgfqpoint{0.569700in}{0.358187in}}%
\pgfpathlineto{\pgfqpoint{0.599450in}{0.346385in}}%
\pgfpathlineto{\pgfqpoint{0.633472in}{0.335386in}}%
\pgfpathlineto{\pgfqpoint{0.669400in}{0.326208in}}%
\pgfpathlineto{\pgfqpoint{0.706884in}{0.318950in}}%
\pgfpathlineto{\pgfqpoint{0.745558in}{0.313694in}}%
\pgfpathlineto{\pgfqpoint{0.785036in}{0.310497in}}%
\pgfpathlineto{\pgfqpoint{0.824923in}{0.309397in}}%
\pgfpathlineto{\pgfqpoint{0.864819in}{0.310404in}}%
\pgfpathlineto{\pgfqpoint{0.904323in}{0.313508in}}%
\pgfpathlineto{\pgfqpoint{0.943041in}{0.318673in}}%
\pgfpathlineto{\pgfqpoint{0.980587in}{0.325846in}}%
\pgfpathlineto{\pgfqpoint{1.016592in}{0.334996in}}%
\pgfpathlineto{\pgfqpoint{1.044058in}{0.344101in}}%
\pgfpathlineto{\pgfqpoint{1.061636in}{0.352074in}}%
\pgfpathlineto{\pgfqpoint{1.074430in}{0.360450in}}%
\pgfpathlineto{\pgfqpoint{1.082783in}{0.368184in}}%
\pgfpathlineto{\pgfqpoint{1.091013in}{0.378657in}}%
\pgfpathlineto{\pgfqpoint{1.099141in}{0.393079in}}%
\pgfpathlineto{\pgfqpoint{1.107200in}{0.412951in}}%
\pgfpathlineto{\pgfqpoint{1.115228in}{0.439989in}}%
\pgfpathlineto{\pgfqpoint{1.123268in}{0.475960in}}%
\pgfpathlineto{\pgfqpoint{1.131361in}{0.522407in}}%
\pgfpathlineto{\pgfqpoint{1.141598in}{0.596569in}}%
\pgfpathlineto{\pgfqpoint{1.154060in}{0.708187in}}%
\pgfpathlineto{\pgfqpoint{1.172715in}{0.902860in}}%
\pgfpathlineto{\pgfqpoint{1.188162in}{1.059022in}}%
\pgfpathlineto{\pgfqpoint{1.196594in}{1.123453in}}%
\pgfpathlineto{\pgfqpoint{1.201033in}{1.144264in}}%
\pgfpathlineto{\pgfqpoint{1.202399in}{1.147914in}}%
\pgfpathlineto{\pgfqpoint{1.202399in}{1.147914in}}%
\pgfusepath{stroke}%
\end{pgfscope}%
\begin{pgfscope}%
\pgftext[x=0.825811in,y=0.139368in,,]{\rmfamily\fontsize{10.000000}{12.000000}\selectfont \(\displaystyle \tau = \frac{1}{2}\)}%
\end{pgfscope}%
\end{pgfpicture}%
\makeatother%
\endgroup%

%% Creator: Matplotlib, PGF backend
%%
%% To include the figure in your LaTeX document, write
%%   \input{<filename>.pgf}
%%
%% Make sure the required packages are loaded in your preamble
%%   \usepackage{pgf}
%%
%% Figures using additional raster images can only be included by \input if
%% they are in the same directory as the main LaTeX file. For loading figures
%% from other directories you can use the `import` package
%%   \usepackage{import}
%% and then include the figures with
%%   \import{<path to file>}{<filename>.pgf}
%%
%% Matplotlib used the following preamble
%%   \usepackage{fontspec}
%%   \setmainfont{DejaVu Serif}
%%   \setsansfont{DejaVu Sans}
%%   \setmonofont{DejaVu Sans Mono}
%%
\begingroup%
\makeatletter%
\begin{pgfpicture}%
\pgfpathrectangle{\pgfpointorigin}{\pgfqpoint{1.620000in}{1.672747in}}%
\pgfusepath{use as bounding box, clip}%
\begin{pgfscope}%
\pgfsetbuttcap%
\pgfsetmiterjoin%
\definecolor{currentfill}{rgb}{1.000000,1.000000,1.000000}%
\pgfsetfillcolor{currentfill}%
\pgfsetlinewidth{0.000000pt}%
\definecolor{currentstroke}{rgb}{1.000000,1.000000,1.000000}%
\pgfsetstrokecolor{currentstroke}%
\pgfsetdash{}{0pt}%
\pgfpathmoveto{\pgfqpoint{0.000000in}{0.000000in}}%
\pgfpathlineto{\pgfqpoint{1.620000in}{0.000000in}}%
\pgfpathlineto{\pgfqpoint{1.620000in}{1.672747in}}%
\pgfpathlineto{\pgfqpoint{0.000000in}{1.672747in}}%
\pgfpathclose%
\pgfusepath{fill}%
\end{pgfscope}%
\begin{pgfscope}%
\pgfsetbuttcap%
\pgfsetmiterjoin%
\definecolor{currentfill}{rgb}{1.000000,1.000000,1.000000}%
\pgfsetfillcolor{currentfill}%
\pgfsetlinewidth{0.000000pt}%
\definecolor{currentstroke}{rgb}{0.000000,0.000000,0.000000}%
\pgfsetstrokecolor{currentstroke}%
\pgfsetstrokeopacity{0.000000}%
\pgfsetdash{}{0pt}%
\pgfpathmoveto{\pgfqpoint{0.035000in}{0.097747in}}%
\pgfpathlineto{\pgfqpoint{1.585000in}{0.097747in}}%
\pgfpathlineto{\pgfqpoint{1.585000in}{1.637747in}}%
\pgfpathlineto{\pgfqpoint{0.035000in}{1.637747in}}%
\pgfpathclose%
\pgfusepath{fill}%
\end{pgfscope}%
\begin{pgfscope}%
\pgfpathrectangle{\pgfqpoint{0.035000in}{0.097747in}}{\pgfqpoint{1.550000in}{1.540000in}} %
\pgfusepath{clip}%
\pgfsetrectcap%
\pgfsetroundjoin%
\pgfsetlinewidth{1.505625pt}%
\definecolor{currentstroke}{rgb}{0.000000,0.000000,0.000000}%
\pgfsetstrokecolor{currentstroke}%
\pgfsetdash{}{0pt}%
\pgfpathmoveto{\pgfqpoint{1.211404in}{1.154082in}}%
\pgfpathlineto{\pgfqpoint{1.213680in}{1.146191in}}%
\pgfpathlineto{\pgfqpoint{1.216654in}{1.122478in}}%
\pgfpathlineto{\pgfqpoint{1.219858in}{1.071608in}}%
\pgfpathlineto{\pgfqpoint{1.223336in}{0.966659in}}%
\pgfpathlineto{\pgfqpoint{1.227724in}{0.737686in}}%
\pgfpathlineto{\pgfqpoint{1.231566in}{0.583869in}}%
\pgfpathlineto{\pgfqpoint{1.234923in}{0.528285in}}%
\pgfpathlineto{\pgfqpoint{1.237567in}{0.514756in}}%
\pgfpathlineto{\pgfqpoint{1.238278in}{0.514613in}}%
\pgfpathlineto{\pgfqpoint{1.238492in}{0.514881in}}%
\pgfpathlineto{\pgfqpoint{1.239498in}{0.518821in}}%
\pgfpathlineto{\pgfqpoint{1.239980in}{0.531426in}}%
\pgfpathlineto{\pgfqpoint{1.238346in}{0.547416in}}%
\pgfpathlineto{\pgfqpoint{1.234431in}{0.563648in}}%
\pgfpathlineto{\pgfqpoint{1.227664in}{0.581314in}}%
\pgfpathlineto{\pgfqpoint{1.210711in}{0.621707in}}%
\pgfpathlineto{\pgfqpoint{1.205147in}{0.643353in}}%
\pgfpathlineto{\pgfqpoint{1.199372in}{0.676684in}}%
\pgfpathlineto{\pgfqpoint{1.193521in}{0.727368in}}%
\pgfpathlineto{\pgfqpoint{1.187704in}{0.800592in}}%
\pgfpathlineto{\pgfqpoint{1.180984in}{0.917171in}}%
\pgfpathlineto{\pgfqpoint{1.161602in}{1.281521in}}%
\pgfpathlineto{\pgfqpoint{1.156061in}{1.338950in}}%
\pgfpathlineto{\pgfqpoint{1.151446in}{1.367889in}}%
\pgfpathlineto{\pgfqpoint{1.148138in}{1.379470in}}%
\pgfpathlineto{\pgfqpoint{1.144642in}{1.384532in}}%
\pgfpathlineto{\pgfqpoint{1.142825in}{1.384583in}}%
\pgfpathlineto{\pgfqpoint{1.140962in}{1.382985in}}%
\pgfpathlineto{\pgfqpoint{1.137107in}{1.374929in}}%
\pgfpathlineto{\pgfqpoint{1.133086in}{1.360652in}}%
\pgfpathlineto{\pgfqpoint{1.126778in}{1.328621in}}%
\pgfpathlineto{\pgfqpoint{1.117946in}{1.269588in}}%
\pgfpathlineto{\pgfqpoint{1.101771in}{1.139774in}}%
\pgfpathlineto{\pgfqpoint{1.080642in}{0.971378in}}%
\pgfpathlineto{\pgfqpoint{1.069063in}{0.894699in}}%
\pgfpathlineto{\pgfqpoint{1.059926in}{0.845558in}}%
\pgfpathlineto{\pgfqpoint{1.050890in}{0.807369in}}%
\pgfpathlineto{\pgfqpoint{1.041927in}{0.779175in}}%
\pgfpathlineto{\pgfqpoint{1.033003in}{0.759459in}}%
\pgfpathlineto{\pgfqpoint{1.026316in}{0.749195in}}%
\pgfpathlineto{\pgfqpoint{1.019619in}{0.742006in}}%
\pgfpathlineto{\pgfqpoint{1.010658in}{0.736077in}}%
\pgfpathlineto{\pgfqpoint{1.001643in}{0.733127in}}%
\pgfpathlineto{\pgfqpoint{0.990282in}{0.732089in}}%
\pgfpathlineto{\pgfqpoint{0.974182in}{0.733216in}}%
\pgfpathlineto{\pgfqpoint{0.834232in}{0.751222in}}%
\pgfpathlineto{\pgfqpoint{0.824372in}{0.755254in}}%
\pgfpathlineto{\pgfqpoint{0.814506in}{0.761694in}}%
\pgfpathlineto{\pgfqpoint{0.807103in}{0.768798in}}%
\pgfpathlineto{\pgfqpoint{0.799693in}{0.778509in}}%
\pgfpathlineto{\pgfqpoint{0.792275in}{0.791506in}}%
\pgfpathlineto{\pgfqpoint{0.784841in}{0.808515in}}%
\pgfpathlineto{\pgfqpoint{0.777386in}{0.830261in}}%
\pgfpathlineto{\pgfqpoint{0.767398in}{0.867746in}}%
\pgfpathlineto{\pgfqpoint{0.757339in}{0.915927in}}%
\pgfpathlineto{\pgfqpoint{0.747190in}{0.975058in}}%
\pgfpathlineto{\pgfqpoint{0.734363in}{1.062686in}}%
\pgfpathlineto{\pgfqpoint{0.695353in}{1.344603in}}%
\pgfpathlineto{\pgfqpoint{0.687648in}{1.385135in}}%
\pgfpathlineto{\pgfqpoint{0.680062in}{1.414154in}}%
\pgfpathlineto{\pgfqpoint{0.675091in}{1.426137in}}%
\pgfpathlineto{\pgfqpoint{0.670201in}{1.431773in}}%
\pgfpathlineto{\pgfqpoint{0.667790in}{1.432144in}}%
\pgfpathlineto{\pgfqpoint{0.665402in}{1.430871in}}%
\pgfpathlineto{\pgfqpoint{0.660699in}{1.423428in}}%
\pgfpathlineto{\pgfqpoint{0.656099in}{1.409631in}}%
\pgfpathlineto{\pgfqpoint{0.649394in}{1.377863in}}%
\pgfpathlineto{\pgfqpoint{0.642920in}{1.334539in}}%
\pgfpathlineto{\pgfqpoint{0.634621in}{1.263137in}}%
\pgfpathlineto{\pgfqpoint{0.620833in}{1.119608in}}%
\pgfpathlineto{\pgfqpoint{0.603909in}{0.945168in}}%
\pgfpathlineto{\pgfqpoint{0.592805in}{0.854427in}}%
\pgfpathlineto{\pgfqpoint{0.583570in}{0.797565in}}%
\pgfpathlineto{\pgfqpoint{0.574349in}{0.756215in}}%
\pgfpathlineto{\pgfqpoint{0.567001in}{0.732333in}}%
\pgfpathlineto{\pgfqpoint{0.559703in}{0.714768in}}%
\pgfpathlineto{\pgfqpoint{0.550693in}{0.699242in}}%
\pgfpathlineto{\pgfqpoint{0.541850in}{0.688425in}}%
\pgfpathlineto{\pgfqpoint{0.529825in}{0.677641in}}%
\pgfpathlineto{\pgfqpoint{0.504241in}{0.659202in}}%
\pgfpathlineto{\pgfqpoint{0.482991in}{0.642503in}}%
\pgfpathlineto{\pgfqpoint{0.464623in}{0.625143in}}%
\pgfpathlineto{\pgfqpoint{0.446525in}{0.607983in}}%
\pgfpathlineto{\pgfqpoint{0.442208in}{0.607131in}}%
\pgfpathlineto{\pgfqpoint{0.439737in}{0.608592in}}%
\pgfpathlineto{\pgfqpoint{0.436553in}{0.613909in}}%
\pgfpathlineto{\pgfqpoint{0.433458in}{0.624464in}}%
\pgfpathlineto{\pgfqpoint{0.429640in}{0.647645in}}%
\pgfpathlineto{\pgfqpoint{0.425004in}{0.694560in}}%
\pgfpathlineto{\pgfqpoint{0.419375in}{0.780684in}}%
\pgfpathlineto{\pgfqpoint{0.411915in}{0.936985in}}%
\pgfpathlineto{\pgfqpoint{0.403173in}{1.170576in}}%
\pgfpathlineto{\pgfqpoint{0.401516in}{1.237962in}}%
\pgfpathlineto{\pgfqpoint{0.402435in}{1.226263in}}%
\pgfpathlineto{\pgfqpoint{0.409882in}{1.072398in}}%
\pgfpathlineto{\pgfqpoint{0.424044in}{0.791545in}}%
\pgfpathlineto{\pgfqpoint{0.433260in}{0.646149in}}%
\pgfpathlineto{\pgfqpoint{0.441085in}{0.555443in}}%
\pgfpathlineto{\pgfqpoint{0.447675in}{0.503129in}}%
\pgfpathlineto{\pgfqpoint{0.454320in}{0.468800in}}%
\pgfpathlineto{\pgfqpoint{0.460020in}{0.449727in}}%
\pgfpathlineto{\pgfqpoint{0.467174in}{0.434021in}}%
\pgfpathlineto{\pgfqpoint{0.474755in}{0.422836in}}%
\pgfpathlineto{\pgfqpoint{0.485616in}{0.411318in}}%
\pgfpathlineto{\pgfqpoint{0.500455in}{0.399128in}}%
\pgfpathlineto{\pgfqpoint{0.521823in}{0.384717in}}%
\pgfpathlineto{\pgfqpoint{0.545423in}{0.371572in}}%
\pgfpathlineto{\pgfqpoint{0.564985in}{0.363156in}}%
\pgfpathlineto{\pgfqpoint{0.577256in}{0.359940in}}%
\pgfpathlineto{\pgfqpoint{0.587754in}{0.359507in}}%
\pgfpathlineto{\pgfqpoint{0.596308in}{0.361649in}}%
\pgfpathlineto{\pgfqpoint{0.602803in}{0.365392in}}%
\pgfpathlineto{\pgfqpoint{0.609355in}{0.371589in}}%
\pgfpathlineto{\pgfqpoint{0.615956in}{0.380907in}}%
\pgfpathlineto{\pgfqpoint{0.622596in}{0.394083in}}%
\pgfpathlineto{\pgfqpoint{0.629268in}{0.411890in}}%
\pgfpathlineto{\pgfqpoint{0.635966in}{0.435083in}}%
\pgfpathlineto{\pgfqpoint{0.644932in}{0.475505in}}%
\pgfpathlineto{\pgfqpoint{0.653941in}{0.527667in}}%
\pgfpathlineto{\pgfqpoint{0.665296in}{0.609242in}}%
\pgfpathlineto{\pgfqpoint{0.681505in}{0.747132in}}%
\pgfpathlineto{\pgfqpoint{0.700819in}{0.910179in}}%
\pgfpathlineto{\pgfqpoint{0.710915in}{0.978264in}}%
\pgfpathlineto{\pgfqpoint{0.718702in}{1.017640in}}%
\pgfpathlineto{\pgfqpoint{0.723995in}{1.036898in}}%
\pgfpathlineto{\pgfqpoint{0.729367in}{1.049880in}}%
\pgfpathlineto{\pgfqpoint{0.734811in}{1.056192in}}%
\pgfpathlineto{\pgfqpoint{0.737558in}{1.056775in}}%
\pgfpathlineto{\pgfqpoint{0.740320in}{1.055629in}}%
\pgfpathlineto{\pgfqpoint{0.743097in}{1.052761in}}%
\pgfpathlineto{\pgfqpoint{0.748687in}{1.041948in}}%
\pgfpathlineto{\pgfqpoint{0.754319in}{1.024632in}}%
\pgfpathlineto{\pgfqpoint{0.762821in}{0.987563in}}%
\pgfpathlineto{\pgfqpoint{0.771355in}{0.939240in}}%
\pgfpathlineto{\pgfqpoint{0.782728in}{0.862189in}}%
\pgfpathlineto{\pgfqpoint{0.830038in}{0.522817in}}%
\pgfpathlineto{\pgfqpoint{0.840887in}{0.465174in}}%
\pgfpathlineto{\pgfqpoint{0.851656in}{0.419403in}}%
\pgfpathlineto{\pgfqpoint{0.862364in}{0.384716in}}%
\pgfpathlineto{\pgfqpoint{0.870363in}{0.365097in}}%
\pgfpathlineto{\pgfqpoint{0.878338in}{0.350131in}}%
\pgfpathlineto{\pgfqpoint{0.886292in}{0.339045in}}%
\pgfpathlineto{\pgfqpoint{0.894224in}{0.331101in}}%
\pgfpathlineto{\pgfqpoint{0.902133in}{0.325634in}}%
\pgfpathlineto{\pgfqpoint{0.912639in}{0.321226in}}%
\pgfpathlineto{\pgfqpoint{0.923093in}{0.319137in}}%
\pgfpathlineto{\pgfqpoint{0.938658in}{0.318701in}}%
\pgfpathlineto{\pgfqpoint{0.959149in}{0.320794in}}%
\pgfpathlineto{\pgfqpoint{0.989195in}{0.326434in}}%
\pgfpathlineto{\pgfqpoint{1.022966in}{0.335000in}}%
\pgfpathlineto{\pgfqpoint{1.050611in}{0.344107in}}%
\pgfpathlineto{\pgfqpoint{1.068303in}{0.352094in}}%
\pgfpathlineto{\pgfqpoint{1.081180in}{0.360503in}}%
\pgfpathlineto{\pgfqpoint{1.089589in}{0.368278in}}%
\pgfpathlineto{\pgfqpoint{1.097873in}{0.378820in}}%
\pgfpathlineto{\pgfqpoint{1.106057in}{0.393348in}}%
\pgfpathlineto{\pgfqpoint{1.114171in}{0.413380in}}%
\pgfpathlineto{\pgfqpoint{1.122256in}{0.440651in}}%
\pgfpathlineto{\pgfqpoint{1.130355in}{0.476945in}}%
\pgfpathlineto{\pgfqpoint{1.138511in}{0.523822in}}%
\pgfpathlineto{\pgfqpoint{1.148831in}{0.598690in}}%
\pgfpathlineto{\pgfqpoint{1.161401in}{0.711398in}}%
\pgfpathlineto{\pgfqpoint{1.180228in}{0.908023in}}%
\pgfpathlineto{\pgfqpoint{1.195819in}{1.065774in}}%
\pgfpathlineto{\pgfqpoint{1.204325in}{1.130845in}}%
\pgfpathlineto{\pgfqpoint{1.208800in}{1.151845in}}%
\pgfpathlineto{\pgfqpoint{1.211493in}{1.157497in}}%
\pgfpathlineto{\pgfqpoint{1.212749in}{1.157765in}}%
\pgfpathlineto{\pgfqpoint{1.212749in}{1.157765in}}%
\pgfusepath{stroke}%
\end{pgfscope}%
\begin{pgfscope}%
\pgftext[x=0.830946in,y=0.139368in,,]{\rmfamily\fontsize{10.000000}{12.000000}\selectfont \(\displaystyle \tau = \frac{3}{5}\)}%
\end{pgfscope}%
\end{pgfpicture}%
\makeatother%
\endgroup%

%% Creator: Matplotlib, PGF backend
%%
%% To include the figure in your LaTeX document, write
%%   \input{<filename>.pgf}
%%
%% Make sure the required packages are loaded in your preamble
%%   \usepackage{pgf}
%%
%% Figures using additional raster images can only be included by \input if
%% they are in the same directory as the main LaTeX file. For loading figures
%% from other directories you can use the `import` package
%%   \usepackage{import}
%% and then include the figures with
%%   \import{<path to file>}{<filename>.pgf}
%%
%% Matplotlib used the following preamble
%%   \usepackage{fontspec}
%%   \setmainfont{DejaVu Serif}
%%   \setsansfont{DejaVu Sans}
%%   \setmonofont{DejaVu Sans Mono}
%%
\begingroup%
\makeatletter%
\begin{pgfpicture}%
\pgfpathrectangle{\pgfpointorigin}{\pgfqpoint{2.000000in}{2.000000in}}%
\pgfusepath{use as bounding box, clip}%
\begin{pgfscope}%
\pgfsetbuttcap%
\pgfsetmiterjoin%
\definecolor{currentfill}{rgb}{1.000000,1.000000,1.000000}%
\pgfsetfillcolor{currentfill}%
\pgfsetlinewidth{0.000000pt}%
\definecolor{currentstroke}{rgb}{1.000000,1.000000,1.000000}%
\pgfsetstrokecolor{currentstroke}%
\pgfsetdash{}{0pt}%
\pgfpathmoveto{\pgfqpoint{0.000000in}{0.000000in}}%
\pgfpathlineto{\pgfqpoint{2.000000in}{0.000000in}}%
\pgfpathlineto{\pgfqpoint{2.000000in}{2.000000in}}%
\pgfpathlineto{\pgfqpoint{0.000000in}{2.000000in}}%
\pgfpathclose%
\pgfusepath{fill}%
\end{pgfscope}%
\begin{pgfscope}%
\pgfsetbuttcap%
\pgfsetmiterjoin%
\definecolor{currentfill}{rgb}{1.000000,1.000000,1.000000}%
\pgfsetfillcolor{currentfill}%
\pgfsetlinewidth{0.000000pt}%
\definecolor{currentstroke}{rgb}{0.000000,0.000000,0.000000}%
\pgfsetstrokecolor{currentstroke}%
\pgfsetstrokeopacity{0.000000}%
\pgfsetdash{}{0pt}%
\pgfpathmoveto{\pgfqpoint{0.250000in}{0.220000in}}%
\pgfpathlineto{\pgfqpoint{1.800000in}{0.220000in}}%
\pgfpathlineto{\pgfqpoint{1.800000in}{1.760000in}}%
\pgfpathlineto{\pgfqpoint{0.250000in}{1.760000in}}%
\pgfpathclose%
\pgfusepath{fill}%
\end{pgfscope}%
\begin{pgfscope}%
\pgfpathrectangle{\pgfqpoint{0.250000in}{0.220000in}}{\pgfqpoint{1.550000in}{1.540000in}} %
\pgfusepath{clip}%
\pgfsetrectcap%
\pgfsetroundjoin%
\pgfsetlinewidth{1.505625pt}%
\definecolor{currentstroke}{rgb}{0.000000,0.000000,0.000000}%
\pgfsetstrokecolor{currentstroke}%
\pgfsetdash{}{0pt}%
\pgfpathmoveto{\pgfqpoint{1.407529in}{0.547854in}}%
\pgfpathlineto{\pgfqpoint{1.422182in}{0.564249in}}%
\pgfpathlineto{\pgfqpoint{1.434196in}{0.581183in}}%
\pgfpathlineto{\pgfqpoint{1.442892in}{0.597186in}}%
\pgfpathlineto{\pgfqpoint{1.449259in}{0.613446in}}%
\pgfpathlineto{\pgfqpoint{1.453281in}{0.629865in}}%
\pgfpathlineto{\pgfqpoint{1.454957in}{0.646347in}}%
\pgfpathlineto{\pgfqpoint{1.454301in}{0.662797in}}%
\pgfpathlineto{\pgfqpoint{1.451337in}{0.679129in}}%
\pgfpathlineto{\pgfqpoint{1.446112in}{0.695503in}}%
\pgfpathlineto{\pgfqpoint{1.437412in}{0.721272in}}%
\pgfpathlineto{\pgfqpoint{1.433851in}{0.739911in}}%
\pgfpathlineto{\pgfqpoint{1.430279in}{0.771934in}}%
\pgfpathlineto{\pgfqpoint{1.426891in}{0.824939in}}%
\pgfpathlineto{\pgfqpoint{1.423222in}{0.923144in}}%
\pgfpathlineto{\pgfqpoint{1.414703in}{1.193147in}}%
\pgfpathlineto{\pgfqpoint{1.411763in}{1.232852in}}%
\pgfpathlineto{\pgfqpoint{1.409450in}{1.248255in}}%
\pgfpathlineto{\pgfqpoint{1.406835in}{1.253770in}}%
\pgfpathlineto{\pgfqpoint{1.405415in}{1.252731in}}%
\pgfpathlineto{\pgfqpoint{1.402355in}{1.243232in}}%
\pgfpathlineto{\pgfqpoint{1.397270in}{1.212087in}}%
\pgfpathlineto{\pgfqpoint{1.389782in}{1.146762in}}%
\pgfpathlineto{\pgfqpoint{1.365912in}{0.924944in}}%
\pgfpathlineto{\pgfqpoint{1.358263in}{0.875896in}}%
\pgfpathlineto{\pgfqpoint{1.350839in}{0.843486in}}%
\pgfpathlineto{\pgfqpoint{1.345373in}{0.828376in}}%
\pgfpathlineto{\pgfqpoint{1.339947in}{0.819295in}}%
\pgfpathlineto{\pgfqpoint{1.334525in}{0.814594in}}%
\pgfpathlineto{\pgfqpoint{1.329075in}{0.812866in}}%
\pgfpathlineto{\pgfqpoint{1.321728in}{0.813360in}}%
\pgfpathlineto{\pgfqpoint{1.310473in}{0.816929in}}%
\pgfpathlineto{\pgfqpoint{1.251710in}{0.838581in}}%
\pgfpathlineto{\pgfqpoint{1.216559in}{0.848314in}}%
\pgfpathlineto{\pgfqpoint{1.179874in}{0.856175in}}%
\pgfpathlineto{\pgfqpoint{1.141993in}{0.862103in}}%
\pgfpathlineto{\pgfqpoint{1.100814in}{0.866233in}}%
\pgfpathlineto{\pgfqpoint{1.059084in}{0.868093in}}%
\pgfpathlineto{\pgfqpoint{1.017224in}{0.867667in}}%
\pgfpathlineto{\pgfqpoint{0.975650in}{0.864958in}}%
\pgfpathlineto{\pgfqpoint{0.934783in}{0.859991in}}%
\pgfpathlineto{\pgfqpoint{0.897335in}{0.853291in}}%
\pgfpathlineto{\pgfqpoint{0.861217in}{0.844681in}}%
\pgfpathlineto{\pgfqpoint{0.826763in}{0.834229in}}%
\pgfpathlineto{\pgfqpoint{0.773202in}{0.817141in}}%
\pgfpathlineto{\pgfqpoint{0.767593in}{0.818538in}}%
\pgfpathlineto{\pgfqpoint{0.762011in}{0.822883in}}%
\pgfpathlineto{\pgfqpoint{0.756421in}{0.831788in}}%
\pgfpathlineto{\pgfqpoint{0.750778in}{0.847317in}}%
\pgfpathlineto{\pgfqpoint{0.745027in}{0.871850in}}%
\pgfpathlineto{\pgfqpoint{0.739109in}{0.907772in}}%
\pgfpathlineto{\pgfqpoint{0.730872in}{0.976496in}}%
\pgfpathlineto{\pgfqpoint{0.722204in}{1.069623in}}%
\pgfpathlineto{\pgfqpoint{0.706414in}{1.269925in}}%
\pgfpathlineto{\pgfqpoint{0.693606in}{1.420705in}}%
\pgfpathlineto{\pgfqpoint{0.688060in}{1.464546in}}%
\pgfpathlineto{\pgfqpoint{0.684817in}{1.477080in}}%
\pgfpathlineto{\pgfqpoint{0.683346in}{1.477904in}}%
\pgfpathlineto{\pgfqpoint{0.681977in}{1.475036in}}%
\pgfpathlineto{\pgfqpoint{0.679551in}{1.458399in}}%
\pgfpathlineto{\pgfqpoint{0.676648in}{1.408320in}}%
\pgfpathlineto{\pgfqpoint{0.673921in}{1.305434in}}%
\pgfpathlineto{\pgfqpoint{0.664218in}{0.804891in}}%
\pgfpathlineto{\pgfqpoint{0.660058in}{0.751140in}}%
\pgfpathlineto{\pgfqpoint{0.655708in}{0.724475in}}%
\pgfpathlineto{\pgfqpoint{0.650899in}{0.708478in}}%
\pgfpathlineto{\pgfqpoint{0.639799in}{0.675963in}}%
\pgfpathlineto{\pgfqpoint{0.637282in}{0.659605in}}%
\pgfpathlineto{\pgfqpoint{0.637077in}{0.643141in}}%
\pgfpathlineto{\pgfqpoint{0.639209in}{0.626664in}}%
\pgfpathlineto{\pgfqpoint{0.643688in}{0.610269in}}%
\pgfpathlineto{\pgfqpoint{0.650509in}{0.594051in}}%
\pgfpathlineto{\pgfqpoint{0.659655in}{0.578109in}}%
\pgfpathlineto{\pgfqpoint{0.671092in}{0.562541in}}%
\pgfpathlineto{\pgfqpoint{0.686012in}{0.546207in}}%
\pgfpathlineto{\pgfqpoint{0.703482in}{0.530546in}}%
\pgfpathlineto{\pgfqpoint{0.723401in}{0.515676in}}%
\pgfpathlineto{\pgfqpoint{0.747448in}{0.500677in}}%
\pgfpathlineto{\pgfqpoint{0.774011in}{0.486863in}}%
\pgfpathlineto{\pgfqpoint{0.802884in}{0.474361in}}%
\pgfpathlineto{\pgfqpoint{0.833835in}{0.463286in}}%
\pgfpathlineto{\pgfqpoint{0.869015in}{0.453127in}}%
\pgfpathlineto{\pgfqpoint{0.900936in}{0.446057in}}%
\pgfpathlineto{\pgfqpoint{0.918581in}{0.443929in}}%
\pgfpathlineto{\pgfqpoint{0.928779in}{0.444408in}}%
\pgfpathlineto{\pgfqpoint{0.936471in}{0.446485in}}%
\pgfpathlineto{\pgfqpoint{0.944194in}{0.451011in}}%
\pgfpathlineto{\pgfqpoint{0.951940in}{0.459234in}}%
\pgfpathlineto{\pgfqpoint{0.957113in}{0.467532in}}%
\pgfpathlineto{\pgfqpoint{0.964883in}{0.485505in}}%
\pgfpathlineto{\pgfqpoint{0.972664in}{0.511670in}}%
\pgfpathlineto{\pgfqpoint{0.980460in}{0.547473in}}%
\pgfpathlineto{\pgfqpoint{0.988282in}{0.593557in}}%
\pgfpathlineto{\pgfqpoint{0.998783in}{0.669568in}}%
\pgfpathlineto{\pgfqpoint{1.025660in}{0.876637in}}%
\pgfpathlineto{\pgfqpoint{1.033920in}{0.918581in}}%
\pgfpathlineto{\pgfqpoint{1.039462in}{0.935624in}}%
\pgfpathlineto{\pgfqpoint{1.042240in}{0.940420in}}%
\pgfpathlineto{\pgfqpoint{1.045019in}{0.942615in}}%
\pgfpathlineto{\pgfqpoint{1.047799in}{0.942175in}}%
\pgfpathlineto{\pgfqpoint{1.050578in}{0.939107in}}%
\pgfpathlineto{\pgfqpoint{1.056127in}{0.925325in}}%
\pgfpathlineto{\pgfqpoint{1.061652in}{0.902137in}}%
\pgfpathlineto{\pgfqpoint{1.069877in}{0.852932in}}%
\pgfpathlineto{\pgfqpoint{1.080696in}{0.770474in}}%
\pgfpathlineto{\pgfqpoint{1.101866in}{0.605157in}}%
\pgfpathlineto{\pgfqpoint{1.112299in}{0.542989in}}%
\pgfpathlineto{\pgfqpoint{1.120094in}{0.508308in}}%
\pgfpathlineto{\pgfqpoint{1.127873in}{0.483134in}}%
\pgfpathlineto{\pgfqpoint{1.135641in}{0.465965in}}%
\pgfpathlineto{\pgfqpoint{1.143397in}{0.455029in}}%
\pgfpathlineto{\pgfqpoint{1.151133in}{0.448625in}}%
\pgfpathlineto{\pgfqpoint{1.158844in}{0.445321in}}%
\pgfpathlineto{\pgfqpoint{1.169071in}{0.443896in}}%
\pgfpathlineto{\pgfqpoint{1.181748in}{0.444650in}}%
\pgfpathlineto{\pgfqpoint{1.204195in}{0.448694in}}%
\pgfpathlineto{\pgfqpoint{1.237965in}{0.457192in}}%
\pgfpathlineto{\pgfqpoint{1.272300in}{0.468112in}}%
\pgfpathlineto{\pgfqpoint{1.304406in}{0.480739in}}%
\pgfpathlineto{\pgfqpoint{1.332097in}{0.493943in}}%
\pgfpathlineto{\pgfqpoint{1.357366in}{0.508395in}}%
\pgfpathlineto{\pgfqpoint{1.380022in}{0.523961in}}%
\pgfpathlineto{\pgfqpoint{1.398573in}{0.539291in}}%
\pgfpathlineto{\pgfqpoint{1.408748in}{0.549093in}}%
\pgfpathlineto{\pgfqpoint{1.408748in}{0.549093in}}%
\pgfusepath{stroke}%
\end{pgfscope}%
\begin{pgfscope}%
\pgftext[x=1.045946in,y=0.659149in,,]{\rmfamily\fontsize{10.000000}{12.000000}\selectfont \(\displaystyle \tau = \frac{2}{3}\)}%
\end{pgfscope}%
\end{pgfpicture}%
\makeatother%
\endgroup%

%% Creator: Matplotlib, PGF backend
%%
%% To include the figure in your LaTeX document, write
%%   \input{<filename>.pgf}
%%
%% Make sure the required packages are loaded in your preamble
%%   \usepackage{pgf}
%%
%% Figures using additional raster images can only be included by \input if
%% they are in the same directory as the main LaTeX file. For loading figures
%% from other directories you can use the `import` package
%%   \usepackage{import}
%% and then include the figures with
%%   \import{<path to file>}{<filename>.pgf}
%%
%% Matplotlib used the following preamble
%%   \usepackage{fontspec}
%%   \setmainfont{DejaVu Serif}
%%   \setsansfont{DejaVu Sans}
%%   \setmonofont{DejaVu Sans Mono}
%%
\begingroup%
\makeatletter%
\begin{pgfpicture}%
\pgfpathrectangle{\pgfpointorigin}{\pgfqpoint{1.610000in}{1.672747in}}%
\pgfusepath{use as bounding box, clip}%
\begin{pgfscope}%
\pgfsetbuttcap%
\pgfsetmiterjoin%
\definecolor{currentfill}{rgb}{1.000000,1.000000,1.000000}%
\pgfsetfillcolor{currentfill}%
\pgfsetlinewidth{0.000000pt}%
\definecolor{currentstroke}{rgb}{1.000000,1.000000,1.000000}%
\pgfsetstrokecolor{currentstroke}%
\pgfsetdash{}{0pt}%
\pgfpathmoveto{\pgfqpoint{0.000000in}{0.000000in}}%
\pgfpathlineto{\pgfqpoint{1.610000in}{0.000000in}}%
\pgfpathlineto{\pgfqpoint{1.610000in}{1.672747in}}%
\pgfpathlineto{\pgfqpoint{0.000000in}{1.672747in}}%
\pgfpathclose%
\pgfusepath{fill}%
\end{pgfscope}%
\begin{pgfscope}%
\pgfsetbuttcap%
\pgfsetmiterjoin%
\definecolor{currentfill}{rgb}{1.000000,1.000000,1.000000}%
\pgfsetfillcolor{currentfill}%
\pgfsetlinewidth{0.000000pt}%
\definecolor{currentstroke}{rgb}{0.000000,0.000000,0.000000}%
\pgfsetstrokecolor{currentstroke}%
\pgfsetstrokeopacity{0.000000}%
\pgfsetdash{}{0pt}%
\pgfpathmoveto{\pgfqpoint{0.035000in}{0.097747in}}%
\pgfpathlineto{\pgfqpoint{1.575000in}{0.097747in}}%
\pgfpathlineto{\pgfqpoint{1.575000in}{1.637747in}}%
\pgfpathlineto{\pgfqpoint{0.035000in}{1.637747in}}%
\pgfpathclose%
\pgfusepath{fill}%
\end{pgfscope}%
\begin{pgfscope}%
\pgfpathrectangle{\pgfqpoint{0.035000in}{0.097747in}}{\pgfqpoint{1.540000in}{1.540000in}} %
\pgfusepath{clip}%
\pgfsetrectcap%
\pgfsetroundjoin%
\pgfsetlinewidth{1.505625pt}%
\definecolor{currentstroke}{rgb}{0.000000,0.000000,0.000000}%
\pgfsetstrokecolor{currentstroke}%
\pgfsetdash{}{0pt}%
\pgfpathmoveto{\pgfqpoint{1.194148in}{0.778581in}}%
\pgfpathlineto{\pgfqpoint{1.195378in}{0.779364in}}%
\pgfpathlineto{\pgfqpoint{1.196571in}{0.779351in}}%
\pgfpathlineto{\pgfqpoint{1.198848in}{0.776974in}}%
\pgfpathlineto{\pgfqpoint{1.201996in}{0.767824in}}%
\pgfpathlineto{\pgfqpoint{1.205721in}{0.746511in}}%
\pgfpathlineto{\pgfqpoint{1.209708in}{0.709149in}}%
\pgfpathlineto{\pgfqpoint{1.216070in}{0.622623in}}%
\pgfpathlineto{\pgfqpoint{1.223051in}{0.535694in}}%
\pgfpathlineto{\pgfqpoint{1.226761in}{0.513582in}}%
\pgfpathlineto{\pgfqpoint{1.229112in}{0.509862in}}%
\pgfpathlineto{\pgfqpoint{1.230194in}{0.511117in}}%
\pgfpathlineto{\pgfqpoint{1.231562in}{0.516800in}}%
\pgfpathlineto{\pgfqpoint{1.232234in}{0.529979in}}%
\pgfpathlineto{\pgfqpoint{1.230804in}{0.546045in}}%
\pgfpathlineto{\pgfqpoint{1.227102in}{0.562270in}}%
\pgfpathlineto{\pgfqpoint{1.221162in}{0.578285in}}%
\pgfpathlineto{\pgfqpoint{1.213038in}{0.593971in}}%
\pgfpathlineto{\pgfqpoint{1.201846in}{0.610495in}}%
\pgfpathlineto{\pgfqpoint{1.188262in}{0.626439in}}%
\pgfpathlineto{\pgfqpoint{1.171084in}{0.642959in}}%
\pgfpathlineto{\pgfqpoint{1.136038in}{0.674897in}}%
\pgfpathlineto{\pgfqpoint{1.126337in}{0.687875in}}%
\pgfpathlineto{\pgfqpoint{1.118034in}{0.702837in}}%
\pgfpathlineto{\pgfqpoint{1.109567in}{0.723254in}}%
\pgfpathlineto{\pgfqpoint{1.100961in}{0.750665in}}%
\pgfpathlineto{\pgfqpoint{1.092231in}{0.786104in}}%
\pgfpathlineto{\pgfqpoint{1.081576in}{0.839053in}}%
\pgfpathlineto{\pgfqpoint{1.049528in}{1.008012in}}%
\pgfpathlineto{\pgfqpoint{1.041360in}{1.035944in}}%
\pgfpathlineto{\pgfqpoint{1.035039in}{1.050174in}}%
\pgfpathlineto{\pgfqpoint{1.030731in}{1.056018in}}%
\pgfpathlineto{\pgfqpoint{1.026349in}{1.058790in}}%
\pgfpathlineto{\pgfqpoint{1.021897in}{1.058449in}}%
\pgfpathlineto{\pgfqpoint{1.017378in}{1.055039in}}%
\pgfpathlineto{\pgfqpoint{1.012798in}{1.048689in}}%
\pgfpathlineto{\pgfqpoint{1.005826in}{1.034118in}}%
\pgfpathlineto{\pgfqpoint{0.996374in}{1.006845in}}%
\pgfpathlineto{\pgfqpoint{0.981976in}{0.955188in}}%
\pgfpathlineto{\pgfqpoint{0.955335in}{0.857680in}}%
\pgfpathlineto{\pgfqpoint{0.943234in}{0.822167in}}%
\pgfpathlineto{\pgfqpoint{0.933568in}{0.799566in}}%
\pgfpathlineto{\pgfqpoint{0.923910in}{0.782091in}}%
\pgfpathlineto{\pgfqpoint{0.914252in}{0.769235in}}%
\pgfpathlineto{\pgfqpoint{0.904585in}{0.760240in}}%
\pgfpathlineto{\pgfqpoint{0.894904in}{0.754270in}}%
\pgfpathlineto{\pgfqpoint{0.882775in}{0.749862in}}%
\pgfpathlineto{\pgfqpoint{0.868176in}{0.747379in}}%
\pgfpathlineto{\pgfqpoint{0.846203in}{0.746316in}}%
\pgfpathlineto{\pgfqpoint{0.775344in}{0.744265in}}%
\pgfpathlineto{\pgfqpoint{0.734359in}{0.740354in}}%
\pgfpathlineto{\pgfqpoint{0.694293in}{0.734208in}}%
\pgfpathlineto{\pgfqpoint{0.657789in}{0.726438in}}%
\pgfpathlineto{\pgfqpoint{0.616421in}{0.715213in}}%
\pgfpathlineto{\pgfqpoint{0.591597in}{0.709123in}}%
\pgfpathlineto{\pgfqpoint{0.579577in}{0.708339in}}%
\pgfpathlineto{\pgfqpoint{0.571700in}{0.709673in}}%
\pgfpathlineto{\pgfqpoint{0.563919in}{0.713277in}}%
\pgfpathlineto{\pgfqpoint{0.556220in}{0.719955in}}%
\pgfpathlineto{\pgfqpoint{0.548588in}{0.730553in}}%
\pgfpathlineto{\pgfqpoint{0.541004in}{0.745837in}}%
\pgfpathlineto{\pgfqpoint{0.533458in}{0.766326in}}%
\pgfpathlineto{\pgfqpoint{0.524070in}{0.799320in}}%
\pgfpathlineto{\pgfqpoint{0.512910in}{0.847822in}}%
\pgfpathlineto{\pgfqpoint{0.488188in}{0.960690in}}%
\pgfpathlineto{\pgfqpoint{0.481805in}{0.979560in}}%
\pgfpathlineto{\pgfqpoint{0.477306in}{0.986629in}}%
\pgfpathlineto{\pgfqpoint{0.474458in}{0.987582in}}%
\pgfpathlineto{\pgfqpoint{0.471736in}{0.985411in}}%
\pgfpathlineto{\pgfqpoint{0.469143in}{0.980114in}}%
\pgfpathlineto{\pgfqpoint{0.465495in}{0.966511in}}%
\pgfpathlineto{\pgfqpoint{0.461063in}{0.938878in}}%
\pgfpathlineto{\pgfqpoint{0.456151in}{0.892582in}}%
\pgfpathlineto{\pgfqpoint{0.448639in}{0.794687in}}%
\pgfpathlineto{\pgfqpoint{0.438174in}{0.658615in}}%
\pgfpathlineto{\pgfqpoint{0.432059in}{0.606598in}}%
\pgfpathlineto{\pgfqpoint{0.425514in}{0.570579in}}%
\pgfpathlineto{\pgfqpoint{0.420034in}{0.540319in}}%
\pgfpathlineto{\pgfqpoint{0.419452in}{0.523658in}}%
\pgfpathlineto{\pgfqpoint{0.421183in}{0.507155in}}%
\pgfpathlineto{\pgfqpoint{0.425244in}{0.490738in}}%
\pgfpathlineto{\pgfqpoint{0.431635in}{0.474484in}}%
\pgfpathlineto{\pgfqpoint{0.440339in}{0.458490in}}%
\pgfpathlineto{\pgfqpoint{0.451325in}{0.442862in}}%
\pgfpathlineto{\pgfqpoint{0.464547in}{0.427780in}}%
\pgfpathlineto{\pgfqpoint{0.478567in}{0.415015in}}%
\pgfpathlineto{\pgfqpoint{0.489846in}{0.407517in}}%
\pgfpathlineto{\pgfqpoint{0.498857in}{0.404236in}}%
\pgfpathlineto{\pgfqpoint{0.505102in}{0.404030in}}%
\pgfpathlineto{\pgfqpoint{0.511520in}{0.406183in}}%
\pgfpathlineto{\pgfqpoint{0.518090in}{0.411508in}}%
\pgfpathlineto{\pgfqpoint{0.524796in}{0.420884in}}%
\pgfpathlineto{\pgfqpoint{0.531622in}{0.435145in}}%
\pgfpathlineto{\pgfqpoint{0.538559in}{0.454916in}}%
\pgfpathlineto{\pgfqpoint{0.547387in}{0.487653in}}%
\pgfpathlineto{\pgfqpoint{0.558256in}{0.537202in}}%
\pgfpathlineto{\pgfqpoint{0.581379in}{0.645179in}}%
\pgfpathlineto{\pgfqpoint{0.589672in}{0.672917in}}%
\pgfpathlineto{\pgfqpoint{0.596125in}{0.687700in}}%
\pgfpathlineto{\pgfqpoint{0.600542in}{0.694094in}}%
\pgfpathlineto{\pgfqpoint{0.605051in}{0.697455in}}%
\pgfpathlineto{\pgfqpoint{0.609650in}{0.697650in}}%
\pgfpathlineto{\pgfqpoint{0.614337in}{0.694631in}}%
\pgfpathlineto{\pgfqpoint{0.619107in}{0.688441in}}%
\pgfpathlineto{\pgfqpoint{0.626403in}{0.673525in}}%
\pgfpathlineto{\pgfqpoint{0.633848in}{0.652573in}}%
\pgfpathlineto{\pgfqpoint{0.643961in}{0.617205in}}%
\pgfpathlineto{\pgfqpoint{0.661989in}{0.543989in}}%
\pgfpathlineto{\pgfqpoint{0.682814in}{0.461285in}}%
\pgfpathlineto{\pgfqpoint{0.695852in}{0.417832in}}%
\pgfpathlineto{\pgfqpoint{0.706280in}{0.389367in}}%
\pgfpathlineto{\pgfqpoint{0.716710in}{0.366631in}}%
\pgfpathlineto{\pgfqpoint{0.727150in}{0.349189in}}%
\pgfpathlineto{\pgfqpoint{0.737608in}{0.336299in}}%
\pgfpathlineto{\pgfqpoint{0.748092in}{0.327089in}}%
\pgfpathlineto{\pgfqpoint{0.758607in}{0.320706in}}%
\pgfpathlineto{\pgfqpoint{0.771795in}{0.315573in}}%
\pgfpathlineto{\pgfqpoint{0.787682in}{0.312118in}}%
\pgfpathlineto{\pgfqpoint{0.808944in}{0.310065in}}%
\pgfpathlineto{\pgfqpoint{0.840903in}{0.309599in}}%
\pgfpathlineto{\pgfqpoint{0.880690in}{0.311397in}}%
\pgfpathlineto{\pgfqpoint{0.919927in}{0.315329in}}%
\pgfpathlineto{\pgfqpoint{0.958221in}{0.321305in}}%
\pgfpathlineto{\pgfqpoint{0.995193in}{0.329260in}}%
\pgfpathlineto{\pgfqpoint{1.028190in}{0.338459in}}%
\pgfpathlineto{\pgfqpoint{1.055104in}{0.348117in}}%
\pgfpathlineto{\pgfqpoint{1.072278in}{0.356463in}}%
\pgfpathlineto{\pgfqpoint{1.084748in}{0.364942in}}%
\pgfpathlineto{\pgfqpoint{1.094882in}{0.374627in}}%
\pgfpathlineto{\pgfqpoint{1.102834in}{0.384986in}}%
\pgfpathlineto{\pgfqpoint{1.110664in}{0.398524in}}%
\pgfpathlineto{\pgfqpoint{1.118391in}{0.416107in}}%
\pgfpathlineto{\pgfqpoint{1.126032in}{0.438550in}}%
\pgfpathlineto{\pgfqpoint{1.135486in}{0.474323in}}%
\pgfpathlineto{\pgfqpoint{1.144840in}{0.518828in}}%
\pgfpathlineto{\pgfqpoint{1.157707in}{0.592845in}}%
\pgfpathlineto{\pgfqpoint{1.183056in}{0.744405in}}%
\pgfpathlineto{\pgfqpoint{1.188873in}{0.767498in}}%
\pgfpathlineto{\pgfqpoint{1.192883in}{0.776997in}}%
\pgfpathlineto{\pgfqpoint{1.192883in}{0.776997in}}%
\pgfusepath{stroke}%
\end{pgfscope}%
\begin{pgfscope}%
\pgftext[x=0.825811in,y=0.139368in,,]{\rmfamily\fontsize{10.000000}{12.000000}\selectfont \(\displaystyle \tau = \frac{3}{4}\)}%
\end{pgfscope}%
\end{pgfpicture}%
\makeatother%
\endgroup%


\cmnt{(c'è un problema di scala nei grafici; rivedere il codice)}


\printbibliography

\end{document}
